% % % % % % % % % % % % % % 
%
%	U.S. Chartbook
%	Brian W. Dew (brianwdew@gmail.com)
%	Updated: December 15, 2019
%	GitHub repo contains to do list (issues)
%   https://github.com/bdecon/US-chartbook
%
% % % % % % % % % % % % % %
\PassOptionsToPackage{table}{xcolor}
\documentclass{report}

%
% % % % % % Packages % % % % % % % % % 
%
	
	\usepackage[letterpaper, margin=1.18in]{geometry}
	\usepackage{microtype}
	\usepackage[default]{lato}
	\usepackage{pgfplots, pgfplotstable}
	\usepackage[eulergreek]{sansmath}
	\usepackage{xcolor}
	\usepackage{array}
	\usepackage{fontawesome5}
	\usepackage{titlesec}
	\usepackage{fancyhdr}
	\usepackage[colorlinks, linkcolor=blue, filecolor=blue, 
		citecolor=blue, urlcolor=blue, linktoc=all, 
		pdfencoding=auto]{hyperref}
	\usetikzlibrary{pgfplots.dateplot, pgfplots.fillbetween, patterns}

%
% % % % % Document Settings % % % % % % % 
%

	% Paragraph spacing
	\usepackage{parskip}
	\setlength\parindent{0pt}
	\setlength{\parskip}{8pt}
	\makeatletter
		\newcommand{\@minipagerestore}{\setlength{\parskip}{8pt}}
	\makeatother
	
	% Section and Subsection Headings
	\titleformat{\section}
  		{\color{darkgray} \LARGE \seriffont \bfseries}
  		{\thesection}{1em}{}
	\titleformat{\subsection}
  		{\color{black!70} \seriffont \bfseries \large}
  		{\thesection}{1em}{}
	\titleformat{\subsubsection}
  		{\color{black!70} \seriffont \bfseries \normalsize}
  		{\thesection}{1em}{}		
%
% % % % % Graph Settings % % % % % % % 
%
	
	% Header and footer
	
	\pagestyle{fancy}
	\fancyhf{}
	\renewcommand{\headrulewidth}{0pt}
	\rfoot{\hyperlink{toc}{\faList}}
	\cfoot{\thepage}	
	
	
	% Color square
	\newcommand{\cbox}[1]{
		\begin{tikzpicture} \draw [#1, line width=6](0,0) -- (.2,0);  
		\end{tikzpicture}}
	\newcommand{\colorline}[2]{
		\begin{tikzpicture} \draw [#1, line width=1.8](0,0.2) -- +(0.6,0) node[right, black!80] {#2}; 
		\end{tikzpicture}}
		
	% Table link
	\newcommand{\tbllink}[1]{\href{https://raw.githubusercontent.com/bdecon/US-chartbook/master/chartbook/data/#1}{\faTable}}
	
	% Last two digits of year
	\makeatletter
	\newcommand*\short[1]{\expandafter\@gobbletwo\number\numexpr#1\relax}
	\makeatother	
	
	% Column width and alignment
	\newcolumntype{R}[1]{>{\raggedleft\let\newline\\\arraybackslash\hspace{0pt}}m{#1}}	
	
	% Style for date plots
	\pgfplotsset{compat=newest, 
		scaled y ticks=false,
		axis line style={black!20}, 
		xtick style={black!20}, ytick style={draw=none},
		every tick label/.style={black!50, font=\scriptsize,
			/pgf/number format/assume math mode=true},
		width=12.8cm, height=4.8cm, 
		xticklabel style={align=left}, 
		yticklabel style={text width=0.9em, align=right},       
		axis x line*=bottom, x axis line style={black!50},
	    axis y line=left, y axis line style={opacity=0},
	    ymajorgrids, grid style={very thin, black!10},	        
	    every node near coord/.style={/pgf/number format/fixed,
	    	font=\scriptsize, style={black!70}},
	    legend style={legend columns=-1, draw=none, fill=none,
	    	/tikz/every even column/.append style={column sep=0.3cm}}}
	    	
	
	% stacked diverging bar
	\newcommand{\sbar}[4]{
		\addplot[ybar stacked, bar width=2.45pt, draw opacity=0, fill=#1] 
			table [x=#2, y=#3, col sep=comma]{#4};}
			
	% thin stacked diverging bar
	\newcommand{\tsbar}[4]{
		\addplot[ybar stacked, bar width=2.2pt, draw opacity=0, fill=#1] 
			table [x=#2, y=#3, col sep=comma]{#4};}
			
	% custom width stacked diverging bar
	\newcommand{\ctsbar}[5]{
		\addplot[ybar stacked, bar width=#5, draw opacity=0, fill=#1] 
			table [x=#2, y=#3, col sep=comma]{#4};}
			
	% area plot segment
	\newcommand{\abar}[4]{
		\addplot[stack plots=y, area style, draw=none, fill=#1] 
			table [x=#2, y=#3, col sep=comma]{#4}\closedcycle;}
					
	% text node
	\newcommand{\stdnode}[3]{\node[below, align=left, shift=({#1,#2})]{#3};}	
	
	% text node located by data	 
	\newcommand{\absnode}[3]{\node[below right, align=left] at (axis cs: #1,#2) {#3};}        
		        
	% Date (X) Axis Tick Marks, one tick per year, every even year labeled
	\newcommand{\dateaxisticks}{
		date coordinates in=x, axis line style={draw=none},
		xmax={2022-02-28},
		max space between ticks=40,	    
		xtick={{1990-01-01}, {1992-01-01}, {1994-01-01}, 
			{1996-01-01}, {1998-01-01}, {2000-01-01}, 
			{2002-01-01}, {2004-01-01}, {2006-01-01},
			{2008-01-01}, {2010-01-01}, {2012-01-01}, {2014-01-01},
		    {2016-01-01}, {2018-01-01}, {2020-01-01}, {2022-01-01}, 
		    {2024-01-01}, {2026-01-01}},
		minor xtick={{1989-01-01}, {1991-01-01}, {1993-01-01},
			{1995-01-01}, {1997-01-01}, {1999-01-01}, 
			{2001-01-01}, {2003-01-01}, {2005-01-01}, {2007-01-01},
		    {2009-01-01}, {2011-01-01}, {2013-01-01}, {2015-01-01},
		    {2017-01-01}, {2019-01-01}, {2021-01-01}, {2023-01-01}, 
		    {2025-01-01}, {2027-01-01}},
		enlarge y limits={0.06}, enlarge x limits={0.01},
		}
		
	% Date (X) Axis Tick Marks, one tick per year, every even year labeled
	\newcommand{\shdateaxisticks}{
		date coordinates in=x, axis line style={draw=none},
		xmax={2022-02-28},
		max space between ticks=40,	    
		xtick={{1990-01-01}, {1995-01-01}, {2000-01-01}, 
			{2005-01-01}, {2010-01-01}, {2015-01-01}, {2020-01-01}},
		minor xtick={},
		enlarge y limits={0.06}, enlarge x limits={0.01},
		}
		
	% Date (X) Axis Tick Marks, one tick per year, every even year labeled
	\newcommand{\ltdateaxisticks}{
		date coordinates in=x, axis line style={draw=none},
		xmax={2022-02-28},
		max space between ticks=40,	    
		xtick={{2015-01-01}, {2016-01-01}, {2017-01-01}, {2018-01-01}, 
		    {2019-01-01}, {2020-01-01}, {2021-01-01}, {2022-01-01}},
		enlarge y limits={0.06}, enlarge x limits={0.01},
		}
		
	% Date (X) Axis Tick Marks, one tick per year, every even year labeled
	\newcommand{\lfdateaxisticks}{
		date coordinates in=x, axis line style={draw=none},
		xmin={2018-01-01}, xmax={2022-02-28},
		max space between ticks=40,	    
		xtick={{2018-01-01}, {2019-01-01}, {2020-01-01}, {2021-01-01}, {2022-01-01}},
		enlarge y limits={value=0.06, upper}, enlarge x limits={0.02}, ymin=0,
		height=6.2cm, width=6.7cm,
		}
		
	% Date (X) Axis Tick Marks, one tick per year, every even year labeled
	\newcommand{\tydateaxisticks}{
		date coordinates in=x, axis line style={draw=none},
		xmax={2022-02-28}, max space between ticks=40,	    
		xtick={{2011-01-01}, {2012-01-01}, {2013-01-01}, {2014-01-01}, {2015-01-01}, {2016-01-01}, 
			{2017-01-01}, {2018-01-01}, {2019-01-01}, {2020-01-01}, {2021-01-01}, {2022-01-01}},
		enlarge y limits={0.06}, enlarge x limits={0.01},
		}
		
	% Settings for y label text in horizontal bar charts
	\newcommand{\barylab}[2]{yticklabel style={text width=#1, align=right, 
		style={black!70}, text height=#2},}
	
	% Solid bars at significant  x or y values
	\newcommand{\bbar}[2]{extra #1 ticks = {{#2}}, extra #1 tick labels = ,
		extra #1 tick style = {grid=major, grid style={thick, black!25}},}
		
			% Solid bars at significant  x or y values
	\newcommand{\dbar}[2]{extra #1 ticks = {{#2}}, extra #1 tick labels = ,
		extra #1 tick style = {grid=major, grid style={dashed, thick, black!50}},}
		
	% Standard line
	\newcommand{\stdline}[4]{\addplot[very thick, no markers, color=#1] 
		table [x=#2, y=#3, col sep=comma] {#4};	}
		
	% Thin line
	\newcommand{\thinline}[4]{\addplot[no markers, color=#1] 
		table [x=#2, y=#3, col sep=comma] {#4};	}
		
	% Dashed line
	\newcommand{\dashline}[4]{\addplot[very thick, dashed, no markers, color=#1] 
		table [x=#2, y=#3, col sep=comma] {#4};	}
		
	% Thicker line
	\newcommand{\thickline}[4]{\addplot[ultra thick, no markers, color=#1] 
		table [x=#2, y=#3, col sep=comma] {#4};	}
		
	% Style for bar plots legend symbol		
		\pgfplotsset{/pgfplots/area legend/.style={/pgfplots/legend image code/.code={
            \fill[##1] (0cm, -0.1cm) rectangle (0.6cm, 0.1cm);}},}		
		
	% Additional bar plot settings
	\newcommand{\barplotnogrid}{xbar=0pt, axis line style={draw=none},
	    yticklabel style={align=left, anchor=east},
      		xmajorticks=false, ymajorgrids=false,   
	    ytick=data, tickwidth=0pt, area legend, reverse legend,
	    nodes near coords align={horizontal},}  
		
	% Recession bars		
	\newcommand{\rbars}{
		\fill[color=black!10] (axis cs:{1990-07-01},\pgfkeysvalueof{/pgfplots/ymin}) rectangle 
			(axis cs:{1991-03-01}, \pgfkeysvalueof{/pgfplots/ymax});
		\fill[color=black!10] (axis cs:{2007-12-01},\pgfkeysvalueof{/pgfplots/ymin}) rectangle 
			(axis cs:{2009-07-01}, \pgfkeysvalueof{/pgfplots/ymax});
		\fill[color=black!10] (axis cs:{2001-03-01},\pgfkeysvalueof{/pgfplots/ymin}) rectangle 
			(axis cs:{2001-11-01}, \pgfkeysvalueof{/pgfplots/ymax});
		\fill[color=black!10] (axis cs:{2020-02-01},\pgfkeysvalueof{/pgfplots/ymin}) rectangle 
			(axis cs:{2020-05-01}, \pgfkeysvalueof{/pgfplots/ymax});}
			
	\newcommand{\rebars}{
		\fill[color=black!10] (axis cs:{2007-12-01},\pgfkeysvalueof{/pgfplots/ymin}) rectangle 
			(axis cs:{2009-07-01}, \pgfkeysvalueof{/pgfplots/ymax});
		\fill[color=black!10] (axis cs:{2001-03-01},\pgfkeysvalueof{/pgfplots/ymin}) rectangle 
			(axis cs:{2001-11-01}, \pgfkeysvalueof{/pgfplots/ymax});
		\fill[color=black!10] (axis cs:{2020-02-01},\pgfkeysvalueof{/pgfplots/ymin}) rectangle 
			(axis cs:{2020-05-01}, \pgfkeysvalueof{/pgfplots/ymax});}
			
	\newcommand{\recbars}{
		\fill[color=black!10] (axis cs:{2007-12-01},\pgfkeysvalueof{/pgfplots/ymin}) rectangle 
			(axis cs:{2009-07-01}, \pgfkeysvalueof{/pgfplots/ymax});
		\fill[color=black!10] (axis cs:{2020-02-01},\pgfkeysvalueof{/pgfplots/ymin}) rectangle 
			(axis cs:{2020-05-01}, \pgfkeysvalueof{/pgfplots/ymax});}
			
	\newcommand{\rbar}{
		\fill[color=black!10] (axis cs:{2020-02-01},\pgfkeysvalueof{/pgfplots/ymin}) rectangle 
			(axis cs:{2020-05-01}, \pgfkeysvalueof{/pgfplots/ymax});}
	
	\newfontfamily\seriffont{RobotoSlab}	
	
	\pgfplotstableread[header=true, col sep=comma]{data/cpi_comp.csv}\cpi
	\pgfplotstableread[header=true, col sep=semicolon]{data/ip_comp.csv}\ip
	\pgfplotstableread[header=true, col sep=comma]{data/rs_comp.csv}\rs
	\pgfplotstableread[header=true, col sep=comma]{data/ahe_ind.csv}\ahe
	\pgfplotstableread[header=true, col sep=comma]{data/poor.csv}\poor
	\pgfplotstableread[header=true, col sep=comma]{data/poor2.csv}\pvrt
	\pgfplotstableread[header=true, col sep=comma]{data/spmtbl20.csv}\spm
	\pgfplotstableread[header=true, col sep=semicolon]{data/occs.csv}\occ
	\pgfplotstableread[header=true, col sep=comma]{data/empgroups.csv}\emp
	\pgfplotstableread[header=true, col sep=comma]{data/empgroups2.csv}\empt
	\pgfplotstableread[header=true, col sep=comma]{data/unempgroups.csv}\unemp
	\pgfplotstableread[header=true, col sep=comma]{data/unempgroups2.csv}\unempt
	\pgfplotstableread[header=true, col sep=comma]{data/unempgroups3.csv}\unemptt
	\pgfplotstableread[header=true, col sep=semicolon]{data/cps_educ.csv}\edsh
	\pgfplotstableread[header=true, col sep=comma]{data/cps_educ_tot.csv}\edtot
	\pgfplotstableread[header=true, col sep=comma]{data/cps_age.csv}\agesh
	\pgfplotstableread[header=true, col sep=semicolon]{data/union_ind.csv}\unmem
	\pgfplotstableread[header=true, col sep=semicolon]{data/quits_ind.csv}\quits
	\pgfplotstableread[header=true, col sep=semicolon]{data/state_pa_epop.csv}\paepop
	\pgfplotstableread[header=true, col sep=semicolon]{data/state_pa_epop2.csv}\paepopt
	\pgfplotstableread[header=true, col sep=semicolon]{data/state_pa_epop3.csv}\paepoptt
	\pgfplotstableread[header=true, col sep=semicolon]{data/openings_ind.csv}\opens
	\pgfplotstableread[header=true, col sep=comma]{data/nilf_comp.csv}\nilf
	\pgfplotstableread[header=true, col sep=comma]{data/pinc.csv}\pinc
	\pgfplotstableread[header=true, col sep=comma]{data/unemp_grp.csv}\ungrp
	\pgfplotstableread[header=true, col sep=comma]{data/unemp_grpsh.csv}\ungrpsh
	\pgfplotstableread[header=true, col sep=comma]{data/ce_age.csv}\ceage
	\pgfplotstableread[header=true, col sep=comma]{data/ce_inc.csv}\ceinc
	\pgfplotstableread[header=true, col sep=comma]{data/cpi_monthly.csv}\cpimo
	\pgfplotstableread[header=true, col sep=comma]{data/ccdebtbar.csv}\ccbar
	
	% Required for bar plots with individual bar colors for categories
	\pgfplotsset{discard if not/.style 2 args={
        x filter/.code={
            \edef\tempa{\thisrow{#1}}
            \edef\tempb{#2}
            \ifx\tempa\tempb
            \else
                \def\pgfmathresult{inf}
            \fi}}}	
	
% % % % % % % %
%
%  Begin Document
%
% % % % % % % %		
\begin{document}
\newpage
\begin{minipage}{0.76\textwidth}
\normalsize Manufacturing\\
\hspace*{-2mm} \begin{tikzpicture}
	\begin{axis}[\bbar{y}{0}, \ltdateaxisticks
		yticklabel style={text width=1.5em}, enlarge y limits={0.1},
		xticklabel={`\short{\year}}, width=4.6cm, height=3.4cm,
		clip=false]
	\rbar
	\thickline{blue!60!black}{date}{Man}{data/ces_ind_sh.csv}
	\end{axis}
\end{tikzpicture}\\
%\footnotesize{Source: Federal Reserve} \hfill \tbllink{msa_unemp_rate.csv}
\end{minipage}
\newpage
\begin{minipage}{0.76\textwidth}
\subsubsection*{Unemployment by Metro Area}
\vspace{-1mm}

\small The Bureau of Labor Statistics \href{https://www.bls.gov/lau/}{produce} local area estimates of unemployment, including the \textbf{unemployment rate for metro areas}. The following map shows changes since 2019 in metro area unemployment rates. An increase in the unemployment rate is shown by a blue circle and a decrease is shown by a light green circle; circle size is the magnitude of the change. 

\input{text/msa_unemp_ch_n.txt}
\end{minipage}
\vspace{1mm}

\begin{minipage}{0.82\textwidth}
\normalsize \textbf{Change in Unemployment Rate by Metro Area}\\
\footnotesize{\textit{from \input{text/unemp_map_date.txt}\unskip, percentage points}}
\vspace{-4mm}

\hspace*{-8mm} %% Creator: Matplotlib, PGF backend
%%
%% To include the figure in your LaTeX document, write
%%   \input{<filename>.pgf}
%%
%% Make sure the required packages are loaded in your preamble
%%   \usepackage{pgf}
%%
%% and, on pdftex
%%   \usepackage[utf8]{inputenc}\DeclareUnicodeCharacter{2212}{-}
%%
%% or, on luatex and xetex
%%   \usepackage{unicode-math}
%%
%% Figures using additional raster images can only be included by \input if
%% they are in the same directory as the main LaTeX file. For loading figures
%% from other directories you can use the `import` package
%%   \usepackage{import}
%%
%% and then include the figures with
%%   \import{<path to file>}{<filename>.pgf}
%%
%% Matplotlib used the following preamble
%%   \usepackage{fontspec}
%%   \setmainfont{DejaVuSerif.ttf}[Path=/home/brian/miniconda3/lib/python3.8/site-packages/matplotlib/mpl-data/fonts/ttf/]
%%   \setsansfont{DejaVuSans.ttf}[Path=/home/brian/miniconda3/lib/python3.8/site-packages/matplotlib/mpl-data/fonts/ttf/]
%%   \setmonofont{DejaVuSansMono.ttf}[Path=/home/brian/miniconda3/lib/python3.8/site-packages/matplotlib/mpl-data/fonts/ttf/]
%%
\begingroup%
\makeatletter%
\begin{pgfpicture}%
\pgfpathrectangle{\pgfpointorigin}{\pgfqpoint{5.507240in}{3.597500in}}%
\pgfusepath{use as bounding box, clip}%
\begin{pgfscope}%
\pgfsetbuttcap%
\pgfsetmiterjoin%
\definecolor{currentfill}{rgb}{1.000000,1.000000,1.000000}%
\pgfsetfillcolor{currentfill}%
\pgfsetlinewidth{0.000000pt}%
\definecolor{currentstroke}{rgb}{1.000000,1.000000,1.000000}%
\pgfsetstrokecolor{currentstroke}%
\pgfsetdash{}{0pt}%
\pgfpathmoveto{\pgfqpoint{0.000000in}{0.000000in}}%
\pgfpathlineto{\pgfqpoint{5.507240in}{0.000000in}}%
\pgfpathlineto{\pgfqpoint{5.507240in}{3.597500in}}%
\pgfpathlineto{\pgfqpoint{0.000000in}{3.597500in}}%
\pgfpathclose%
\pgfusepath{fill}%
\end{pgfscope}%
\begin{pgfscope}%
\pgfpathrectangle{\pgfqpoint{0.100000in}{0.100000in}}{\pgfqpoint{5.307240in}{3.397500in}}%
\pgfusepath{clip}%
\pgfsetbuttcap%
\pgfsetmiterjoin%
\definecolor{currentfill}{rgb}{1.000000,1.000000,1.000000}%
\pgfsetfillcolor{currentfill}%
\pgfsetlinewidth{0.501875pt}%
\definecolor{currentstroke}{rgb}{0.827451,0.827451,0.827451}%
\pgfsetstrokecolor{currentstroke}%
\pgfsetdash{}{0pt}%
\pgfpathmoveto{\pgfqpoint{1.871453in}{0.896655in}}%
\pgfpathlineto{\pgfqpoint{1.856200in}{0.896905in}}%
\pgfpathlineto{\pgfqpoint{1.845818in}{0.908395in}}%
\pgfpathlineto{\pgfqpoint{1.838759in}{0.938911in}}%
\pgfpathlineto{\pgfqpoint{1.855165in}{0.942734in}}%
\pgfpathlineto{\pgfqpoint{1.871832in}{0.938623in}}%
\pgfpathlineto{\pgfqpoint{1.888794in}{0.926601in}}%
\pgfpathlineto{\pgfqpoint{1.888786in}{0.909976in}}%
\pgfpathclose%
\pgfusepath{stroke,fill}%
\end{pgfscope}%
\begin{pgfscope}%
\pgfpathrectangle{\pgfqpoint{0.100000in}{0.100000in}}{\pgfqpoint{5.307240in}{3.397500in}}%
\pgfusepath{clip}%
\pgfsetbuttcap%
\pgfsetmiterjoin%
\definecolor{currentfill}{rgb}{1.000000,1.000000,1.000000}%
\pgfsetfillcolor{currentfill}%
\pgfsetlinewidth{0.501875pt}%
\definecolor{currentstroke}{rgb}{0.827451,0.827451,0.827451}%
\pgfsetstrokecolor{currentstroke}%
\pgfsetdash{}{0pt}%
\pgfpathmoveto{\pgfqpoint{1.964299in}{0.715486in}}%
\pgfpathlineto{\pgfqpoint{1.947234in}{0.721983in}}%
\pgfpathlineto{\pgfqpoint{1.947155in}{0.734730in}}%
\pgfpathlineto{\pgfqpoint{1.926541in}{0.745508in}}%
\pgfpathlineto{\pgfqpoint{1.929052in}{0.772649in}}%
\pgfpathlineto{\pgfqpoint{1.935022in}{0.785456in}}%
\pgfpathlineto{\pgfqpoint{1.947628in}{0.773669in}}%
\pgfpathlineto{\pgfqpoint{1.968129in}{0.775638in}}%
\pgfpathlineto{\pgfqpoint{1.962951in}{0.738125in}}%
\pgfpathclose%
\pgfusepath{stroke,fill}%
\end{pgfscope}%
\begin{pgfscope}%
\pgfpathrectangle{\pgfqpoint{0.100000in}{0.100000in}}{\pgfqpoint{5.307240in}{3.397500in}}%
\pgfusepath{clip}%
\pgfsetbuttcap%
\pgfsetmiterjoin%
\definecolor{currentfill}{rgb}{1.000000,1.000000,1.000000}%
\pgfsetfillcolor{currentfill}%
\pgfsetlinewidth{0.501875pt}%
\definecolor{currentstroke}{rgb}{0.827451,0.827451,0.827451}%
\pgfsetstrokecolor{currentstroke}%
\pgfsetdash{}{0pt}%
\pgfpathmoveto{\pgfqpoint{2.035926in}{0.631815in}}%
\pgfpathlineto{\pgfqpoint{2.014270in}{0.635502in}}%
\pgfpathlineto{\pgfqpoint{2.002296in}{0.654042in}}%
\pgfpathlineto{\pgfqpoint{2.000604in}{0.669285in}}%
\pgfpathclose%
\pgfusepath{stroke,fill}%
\end{pgfscope}%
\begin{pgfscope}%
\pgfpathrectangle{\pgfqpoint{0.100000in}{0.100000in}}{\pgfqpoint{5.307240in}{3.397500in}}%
\pgfusepath{clip}%
\pgfsetbuttcap%
\pgfsetmiterjoin%
\definecolor{currentfill}{rgb}{1.000000,1.000000,1.000000}%
\pgfsetfillcolor{currentfill}%
\pgfsetlinewidth{0.501875pt}%
\definecolor{currentstroke}{rgb}{0.827451,0.827451,0.827451}%
\pgfsetstrokecolor{currentstroke}%
\pgfsetdash{}{0pt}%
\pgfpathmoveto{\pgfqpoint{1.991549in}{0.635047in}}%
\pgfpathlineto{\pgfqpoint{2.002902in}{0.625831in}}%
\pgfpathlineto{\pgfqpoint{2.005367in}{0.609984in}}%
\pgfpathlineto{\pgfqpoint{1.982512in}{0.610859in}}%
\pgfpathclose%
\pgfusepath{stroke,fill}%
\end{pgfscope}%
\begin{pgfscope}%
\pgfpathrectangle{\pgfqpoint{0.100000in}{0.100000in}}{\pgfqpoint{5.307240in}{3.397500in}}%
\pgfusepath{clip}%
\pgfsetbuttcap%
\pgfsetmiterjoin%
\definecolor{currentfill}{rgb}{1.000000,1.000000,1.000000}%
\pgfsetfillcolor{currentfill}%
\pgfsetlinewidth{0.501875pt}%
\definecolor{currentstroke}{rgb}{0.827451,0.827451,0.827451}%
\pgfsetstrokecolor{currentstroke}%
\pgfsetdash{}{0pt}%
\pgfpathmoveto{\pgfqpoint{2.043690in}{0.546530in}}%
\pgfpathlineto{\pgfqpoint{2.020881in}{0.554729in}}%
\pgfpathlineto{\pgfqpoint{2.027418in}{0.574607in}}%
\pgfpathlineto{\pgfqpoint{2.017858in}{0.594861in}}%
\pgfpathlineto{\pgfqpoint{2.020860in}{0.608784in}}%
\pgfpathlineto{\pgfqpoint{2.039215in}{0.614092in}}%
\pgfpathlineto{\pgfqpoint{2.038861in}{0.591784in}}%
\pgfpathlineto{\pgfqpoint{2.060570in}{0.579411in}}%
\pgfpathlineto{\pgfqpoint{2.072987in}{0.549496in}}%
\pgfpathlineto{\pgfqpoint{2.059149in}{0.537976in}}%
\pgfpathclose%
\pgfusepath{stroke,fill}%
\end{pgfscope}%
\begin{pgfscope}%
\pgfpathrectangle{\pgfqpoint{0.100000in}{0.100000in}}{\pgfqpoint{5.307240in}{3.397500in}}%
\pgfusepath{clip}%
\pgfsetbuttcap%
\pgfsetmiterjoin%
\definecolor{currentfill}{rgb}{1.000000,1.000000,1.000000}%
\pgfsetfillcolor{currentfill}%
\pgfsetlinewidth{0.501875pt}%
\definecolor{currentstroke}{rgb}{0.827451,0.827451,0.827451}%
\pgfsetstrokecolor{currentstroke}%
\pgfsetdash{}{0pt}%
\pgfpathmoveto{\pgfqpoint{1.961315in}{0.344264in}}%
\pgfpathlineto{\pgfqpoint{1.950448in}{0.366711in}}%
\pgfpathlineto{\pgfqpoint{1.955259in}{0.381809in}}%
\pgfpathlineto{\pgfqpoint{1.976573in}{0.402709in}}%
\pgfpathlineto{\pgfqpoint{1.978591in}{0.418729in}}%
\pgfpathlineto{\pgfqpoint{1.992539in}{0.454198in}}%
\pgfpathlineto{\pgfqpoint{2.029225in}{0.460258in}}%
\pgfpathlineto{\pgfqpoint{2.036247in}{0.481020in}}%
\pgfpathlineto{\pgfqpoint{2.052421in}{0.473235in}}%
\pgfpathlineto{\pgfqpoint{2.085287in}{0.414965in}}%
\pgfpathlineto{\pgfqpoint{2.085702in}{0.395941in}}%
\pgfpathlineto{\pgfqpoint{2.076412in}{0.379476in}}%
\pgfpathlineto{\pgfqpoint{2.088199in}{0.341671in}}%
\pgfpathlineto{\pgfqpoint{2.079963in}{0.335576in}}%
\pgfpathlineto{\pgfqpoint{2.051908in}{0.337348in}}%
\pgfpathlineto{\pgfqpoint{2.018171in}{0.348623in}}%
\pgfpathlineto{\pgfqpoint{1.989817in}{0.353279in}}%
\pgfpathlineto{\pgfqpoint{1.982244in}{0.346664in}}%
\pgfpathclose%
\pgfusepath{stroke,fill}%
\end{pgfscope}%
\begin{pgfscope}%
\pgfpathrectangle{\pgfqpoint{0.100000in}{0.100000in}}{\pgfqpoint{5.307240in}{3.397500in}}%
\pgfusepath{clip}%
\pgfsetbuttcap%
\pgfsetmiterjoin%
\definecolor{currentfill}{rgb}{1.000000,1.000000,1.000000}%
\pgfsetfillcolor{currentfill}%
\pgfsetlinewidth{0.501875pt}%
\definecolor{currentstroke}{rgb}{0.827451,0.827451,0.827451}%
\pgfsetstrokecolor{currentstroke}%
\pgfsetdash{}{0pt}%
\pgfpathmoveto{\pgfqpoint{1.035091in}{0.658486in}}%
\pgfpathlineto{\pgfqpoint{1.032911in}{0.656567in}}%
\pgfpathlineto{\pgfqpoint{1.022907in}{0.659125in}}%
\pgfpathlineto{\pgfqpoint{1.023654in}{0.664161in}}%
\pgfpathlineto{\pgfqpoint{1.029317in}{0.666466in}}%
\pgfpathlineto{\pgfqpoint{1.037390in}{0.678072in}}%
\pgfpathlineto{\pgfqpoint{1.038732in}{0.685916in}}%
\pgfpathlineto{\pgfqpoint{1.043270in}{0.687972in}}%
\pgfpathlineto{\pgfqpoint{1.050655in}{0.687980in}}%
\pgfpathlineto{\pgfqpoint{1.058375in}{0.713437in}}%
\pgfpathlineto{\pgfqpoint{1.061399in}{0.715531in}}%
\pgfpathlineto{\pgfqpoint{1.058687in}{0.720508in}}%
\pgfpathlineto{\pgfqpoint{1.052338in}{0.715473in}}%
\pgfpathlineto{\pgfqpoint{1.033684in}{0.722260in}}%
\pgfpathlineto{\pgfqpoint{1.022550in}{0.733694in}}%
\pgfpathlineto{\pgfqpoint{1.027405in}{0.738910in}}%
\pgfpathlineto{\pgfqpoint{1.025197in}{0.746268in}}%
\pgfpathlineto{\pgfqpoint{1.026558in}{0.754960in}}%
\pgfpathlineto{\pgfqpoint{1.024163in}{0.765122in}}%
\pgfpathlineto{\pgfqpoint{1.023780in}{0.775295in}}%
\pgfpathlineto{\pgfqpoint{1.034496in}{0.773593in}}%
\pgfpathlineto{\pgfqpoint{1.036618in}{0.776591in}}%
\pgfpathlineto{\pgfqpoint{1.041948in}{0.776301in}}%
\pgfpathlineto{\pgfqpoint{1.048086in}{0.765451in}}%
\pgfpathlineto{\pgfqpoint{1.054005in}{0.757507in}}%
\pgfpathlineto{\pgfqpoint{1.049264in}{0.752336in}}%
\pgfpathlineto{\pgfqpoint{1.057462in}{0.750352in}}%
\pgfpathlineto{\pgfqpoint{1.059329in}{0.752888in}}%
\pgfpathlineto{\pgfqpoint{1.054945in}{0.763179in}}%
\pgfpathlineto{\pgfqpoint{1.045715in}{0.771707in}}%
\pgfpathlineto{\pgfqpoint{1.046844in}{0.775281in}}%
\pgfpathlineto{\pgfqpoint{1.039478in}{0.784312in}}%
\pgfpathlineto{\pgfqpoint{1.047000in}{0.786146in}}%
\pgfpathlineto{\pgfqpoint{1.040483in}{0.805483in}}%
\pgfpathlineto{\pgfqpoint{1.045782in}{0.808164in}}%
\pgfpathlineto{\pgfqpoint{1.047074in}{0.812800in}}%
\pgfpathlineto{\pgfqpoint{1.043627in}{0.818307in}}%
\pgfpathlineto{\pgfqpoint{1.050742in}{0.819686in}}%
\pgfpathlineto{\pgfqpoint{1.052019in}{0.823851in}}%
\pgfpathlineto{\pgfqpoint{1.060505in}{0.817599in}}%
\pgfpathlineto{\pgfqpoint{1.062390in}{0.824378in}}%
\pgfpathlineto{\pgfqpoint{1.066849in}{0.827361in}}%
\pgfpathlineto{\pgfqpoint{1.088548in}{0.831510in}}%
\pgfpathlineto{\pgfqpoint{1.095209in}{0.830043in}}%
\pgfpathlineto{\pgfqpoint{1.100790in}{0.838277in}}%
\pgfpathlineto{\pgfqpoint{1.114031in}{0.843678in}}%
\pgfpathlineto{\pgfqpoint{1.124229in}{0.841630in}}%
\pgfpathlineto{\pgfqpoint{1.128882in}{0.834459in}}%
\pgfpathlineto{\pgfqpoint{1.128638in}{0.823066in}}%
\pgfpathlineto{\pgfqpoint{1.133857in}{0.820483in}}%
\pgfpathlineto{\pgfqpoint{1.144503in}{0.820844in}}%
\pgfpathlineto{\pgfqpoint{1.158140in}{0.823808in}}%
\pgfpathlineto{\pgfqpoint{1.157972in}{0.820164in}}%
\pgfpathlineto{\pgfqpoint{1.175209in}{0.809776in}}%
\pgfpathlineto{\pgfqpoint{1.187264in}{0.812562in}}%
\pgfpathlineto{\pgfqpoint{1.190813in}{0.816168in}}%
\pgfpathlineto{\pgfqpoint{1.194613in}{0.825400in}}%
\pgfpathlineto{\pgfqpoint{1.199327in}{0.831313in}}%
\pgfpathlineto{\pgfqpoint{1.199747in}{0.840173in}}%
\pgfpathlineto{\pgfqpoint{1.195432in}{0.843658in}}%
\pgfpathlineto{\pgfqpoint{1.201822in}{0.846807in}}%
\pgfpathlineto{\pgfqpoint{1.212560in}{0.842076in}}%
\pgfpathlineto{\pgfqpoint{1.216057in}{0.844997in}}%
\pgfpathlineto{\pgfqpoint{1.216672in}{0.857532in}}%
\pgfpathlineto{\pgfqpoint{1.209732in}{0.854912in}}%
\pgfpathlineto{\pgfqpoint{1.204919in}{0.859256in}}%
\pgfpathlineto{\pgfqpoint{1.197105in}{0.861963in}}%
\pgfpathlineto{\pgfqpoint{1.184702in}{0.861512in}}%
\pgfpathlineto{\pgfqpoint{1.172819in}{0.865629in}}%
\pgfpathlineto{\pgfqpoint{1.172223in}{0.875889in}}%
\pgfpathlineto{\pgfqpoint{1.161595in}{0.886787in}}%
\pgfpathlineto{\pgfqpoint{1.148117in}{0.891362in}}%
\pgfpathlineto{\pgfqpoint{1.136353in}{0.912459in}}%
\pgfpathlineto{\pgfqpoint{1.137319in}{0.920266in}}%
\pgfpathlineto{\pgfqpoint{1.143652in}{0.923826in}}%
\pgfpathlineto{\pgfqpoint{1.142720in}{0.931216in}}%
\pgfpathlineto{\pgfqpoint{1.144190in}{0.938568in}}%
\pgfpathlineto{\pgfqpoint{1.149202in}{0.933536in}}%
\pgfpathlineto{\pgfqpoint{1.155820in}{0.935635in}}%
\pgfpathlineto{\pgfqpoint{1.155489in}{0.942021in}}%
\pgfpathlineto{\pgfqpoint{1.146314in}{0.954843in}}%
\pgfpathlineto{\pgfqpoint{1.143628in}{0.968068in}}%
\pgfpathlineto{\pgfqpoint{1.151308in}{0.969077in}}%
\pgfpathlineto{\pgfqpoint{1.162803in}{0.968265in}}%
\pgfpathlineto{\pgfqpoint{1.166423in}{0.963841in}}%
\pgfpathlineto{\pgfqpoint{1.176728in}{0.965424in}}%
\pgfpathlineto{\pgfqpoint{1.186307in}{0.963346in}}%
\pgfpathlineto{\pgfqpoint{1.192782in}{0.959693in}}%
\pgfpathlineto{\pgfqpoint{1.197262in}{0.961362in}}%
\pgfpathlineto{\pgfqpoint{1.209163in}{0.957712in}}%
\pgfpathlineto{\pgfqpoint{1.209272in}{0.954479in}}%
\pgfpathlineto{\pgfqpoint{1.218334in}{0.955570in}}%
\pgfpathlineto{\pgfqpoint{1.230485in}{0.947683in}}%
\pgfpathlineto{\pgfqpoint{1.230013in}{0.942448in}}%
\pgfpathlineto{\pgfqpoint{1.224231in}{0.940031in}}%
\pgfpathlineto{\pgfqpoint{1.218086in}{0.934282in}}%
\pgfpathlineto{\pgfqpoint{1.217344in}{0.926387in}}%
\pgfpathlineto{\pgfqpoint{1.230581in}{0.916319in}}%
\pgfpathlineto{\pgfqpoint{1.228464in}{0.911495in}}%
\pgfpathlineto{\pgfqpoint{1.237883in}{0.907790in}}%
\pgfpathlineto{\pgfqpoint{1.240196in}{0.901739in}}%
\pgfpathlineto{\pgfqpoint{1.252150in}{0.906547in}}%
\pgfpathlineto{\pgfqpoint{1.257414in}{0.900245in}}%
\pgfpathlineto{\pgfqpoint{1.259843in}{0.904059in}}%
\pgfpathlineto{\pgfqpoint{1.252690in}{0.916475in}}%
\pgfpathlineto{\pgfqpoint{1.255232in}{0.920194in}}%
\pgfpathlineto{\pgfqpoint{1.256358in}{0.930001in}}%
\pgfpathlineto{\pgfqpoint{1.253509in}{0.934316in}}%
\pgfpathlineto{\pgfqpoint{1.255647in}{0.940114in}}%
\pgfpathlineto{\pgfqpoint{1.262651in}{0.939569in}}%
\pgfpathlineto{\pgfqpoint{1.261201in}{0.929703in}}%
\pgfpathlineto{\pgfqpoint{1.256730in}{0.926412in}}%
\pgfpathlineto{\pgfqpoint{1.257047in}{0.913140in}}%
\pgfpathlineto{\pgfqpoint{1.265070in}{0.911553in}}%
\pgfpathlineto{\pgfqpoint{1.265168in}{0.901685in}}%
\pgfpathlineto{\pgfqpoint{1.273930in}{0.895283in}}%
\pgfpathlineto{\pgfqpoint{1.277958in}{0.899320in}}%
\pgfpathlineto{\pgfqpoint{1.273556in}{0.911607in}}%
\pgfpathlineto{\pgfqpoint{1.269003in}{0.914715in}}%
\pgfpathlineto{\pgfqpoint{1.264524in}{0.912670in}}%
\pgfpathlineto{\pgfqpoint{1.260583in}{0.915281in}}%
\pgfpathlineto{\pgfqpoint{1.261271in}{0.926634in}}%
\pgfpathlineto{\pgfqpoint{1.267847in}{0.931278in}}%
\pgfpathlineto{\pgfqpoint{1.274319in}{0.930850in}}%
\pgfpathlineto{\pgfqpoint{1.271638in}{0.937274in}}%
\pgfpathlineto{\pgfqpoint{1.261936in}{0.942195in}}%
\pgfpathlineto{\pgfqpoint{1.249088in}{0.961800in}}%
\pgfpathlineto{\pgfqpoint{1.255500in}{0.971203in}}%
\pgfpathlineto{\pgfqpoint{1.259069in}{0.980060in}}%
\pgfpathlineto{\pgfqpoint{1.259350in}{0.997845in}}%
\pgfpathlineto{\pgfqpoint{1.257976in}{1.012848in}}%
\pgfpathlineto{\pgfqpoint{1.254055in}{1.022700in}}%
\pgfpathlineto{\pgfqpoint{1.254837in}{1.032549in}}%
\pgfpathlineto{\pgfqpoint{1.264864in}{1.041092in}}%
\pgfpathlineto{\pgfqpoint{1.274835in}{1.051244in}}%
\pgfpathlineto{\pgfqpoint{1.299835in}{1.030503in}}%
\pgfpathlineto{\pgfqpoint{1.316179in}{1.028974in}}%
\pgfpathlineto{\pgfqpoint{1.329670in}{1.034139in}}%
\pgfpathlineto{\pgfqpoint{1.341513in}{1.040940in}}%
\pgfpathlineto{\pgfqpoint{1.344029in}{1.044180in}}%
\pgfpathlineto{\pgfqpoint{1.373263in}{1.051570in}}%
\pgfpathlineto{\pgfqpoint{1.375802in}{1.046157in}}%
\pgfpathlineto{\pgfqpoint{1.385736in}{1.041727in}}%
\pgfpathlineto{\pgfqpoint{1.404694in}{1.043086in}}%
\pgfpathlineto{\pgfqpoint{1.408157in}{1.041876in}}%
\pgfpathlineto{\pgfqpoint{1.416677in}{1.045108in}}%
\pgfpathlineto{\pgfqpoint{1.420103in}{1.039519in}}%
\pgfpathlineto{\pgfqpoint{1.428285in}{1.034411in}}%
\pgfpathlineto{\pgfqpoint{1.432378in}{1.034989in}}%
\pgfpathlineto{\pgfqpoint{1.440300in}{1.028852in}}%
\pgfpathlineto{\pgfqpoint{1.453687in}{1.029980in}}%
\pgfpathlineto{\pgfqpoint{1.467058in}{1.034949in}}%
\pgfpathlineto{\pgfqpoint{1.477978in}{1.019062in}}%
\pgfpathlineto{\pgfqpoint{1.477130in}{1.015496in}}%
\pgfpathlineto{\pgfqpoint{1.465082in}{1.011399in}}%
\pgfpathlineto{\pgfqpoint{1.468301in}{1.005203in}}%
\pgfpathlineto{\pgfqpoint{1.479393in}{1.011724in}}%
\pgfpathlineto{\pgfqpoint{1.483922in}{1.010145in}}%
\pgfpathlineto{\pgfqpoint{1.485849in}{1.002344in}}%
\pgfpathlineto{\pgfqpoint{1.479227in}{0.999046in}}%
\pgfpathlineto{\pgfqpoint{1.483956in}{0.989186in}}%
\pgfpathlineto{\pgfqpoint{1.490824in}{0.990852in}}%
\pgfpathlineto{\pgfqpoint{1.500790in}{0.985764in}}%
\pgfpathlineto{\pgfqpoint{1.510321in}{0.971640in}}%
\pgfpathlineto{\pgfqpoint{1.501934in}{0.967719in}}%
\pgfpathlineto{\pgfqpoint{1.509343in}{0.957312in}}%
\pgfpathlineto{\pgfqpoint{1.503101in}{0.954880in}}%
\pgfpathlineto{\pgfqpoint{1.510867in}{0.944996in}}%
\pgfpathlineto{\pgfqpoint{1.520337in}{0.946049in}}%
\pgfpathlineto{\pgfqpoint{1.525323in}{0.941602in}}%
\pgfpathlineto{\pgfqpoint{1.537937in}{0.935045in}}%
\pgfpathlineto{\pgfqpoint{1.554207in}{0.911769in}}%
\pgfpathlineto{\pgfqpoint{1.553840in}{0.906202in}}%
\pgfpathlineto{\pgfqpoint{1.578240in}{0.888372in}}%
\pgfpathlineto{\pgfqpoint{1.578766in}{0.882065in}}%
\pgfpathlineto{\pgfqpoint{1.585269in}{0.872628in}}%
\pgfpathlineto{\pgfqpoint{1.605246in}{0.867911in}}%
\pgfpathlineto{\pgfqpoint{1.612654in}{0.864557in}}%
\pgfpathlineto{\pgfqpoint{1.619625in}{0.854154in}}%
\pgfpathlineto{\pgfqpoint{1.620522in}{0.845042in}}%
\pgfpathlineto{\pgfqpoint{1.626513in}{0.837615in}}%
\pgfpathlineto{\pgfqpoint{1.628312in}{0.829239in}}%
\pgfpathlineto{\pgfqpoint{1.633926in}{0.826180in}}%
\pgfpathlineto{\pgfqpoint{1.606115in}{0.775986in}}%
\pgfpathlineto{\pgfqpoint{1.548389in}{0.671779in}}%
\pgfpathlineto{\pgfqpoint{1.463693in}{0.518857in}}%
\pgfpathlineto{\pgfqpoint{1.430534in}{0.458972in}}%
\pgfpathlineto{\pgfqpoint{1.437597in}{0.450930in}}%
\pgfpathlineto{\pgfqpoint{1.440708in}{0.453497in}}%
\pgfpathlineto{\pgfqpoint{1.447016in}{0.444133in}}%
\pgfpathlineto{\pgfqpoint{1.455774in}{0.446849in}}%
\pgfpathlineto{\pgfqpoint{1.467399in}{0.441240in}}%
\pgfpathlineto{\pgfqpoint{1.459869in}{0.432474in}}%
\pgfpathlineto{\pgfqpoint{1.465684in}{0.420644in}}%
\pgfpathlineto{\pgfqpoint{1.464482in}{0.414885in}}%
\pgfpathlineto{\pgfqpoint{1.473756in}{0.384513in}}%
\pgfpathlineto{\pgfqpoint{1.469832in}{0.370701in}}%
\pgfpathlineto{\pgfqpoint{1.487551in}{0.374121in}}%
\pgfpathlineto{\pgfqpoint{1.491901in}{0.371880in}}%
\pgfpathlineto{\pgfqpoint{1.496624in}{0.375524in}}%
\pgfpathlineto{\pgfqpoint{1.499983in}{0.382410in}}%
\pgfpathlineto{\pgfqpoint{1.504795in}{0.386358in}}%
\pgfpathlineto{\pgfqpoint{1.525329in}{0.385858in}}%
\pgfpathlineto{\pgfqpoint{1.529969in}{0.372604in}}%
\pgfpathlineto{\pgfqpoint{1.525987in}{0.360970in}}%
\pgfpathlineto{\pgfqpoint{1.530512in}{0.357131in}}%
\pgfpathlineto{\pgfqpoint{1.532891in}{0.350565in}}%
\pgfpathlineto{\pgfqpoint{1.532372in}{0.338216in}}%
\pgfpathlineto{\pgfqpoint{1.538409in}{0.329362in}}%
\pgfpathlineto{\pgfqpoint{1.541719in}{0.315487in}}%
\pgfpathlineto{\pgfqpoint{1.540020in}{0.307992in}}%
\pgfpathlineto{\pgfqpoint{1.542070in}{0.300396in}}%
\pgfpathlineto{\pgfqpoint{1.545025in}{0.256410in}}%
\pgfpathlineto{\pgfqpoint{1.540790in}{0.252945in}}%
\pgfpathlineto{\pgfqpoint{1.546377in}{0.248157in}}%
\pgfpathlineto{\pgfqpoint{1.541987in}{0.242335in}}%
\pgfpathlineto{\pgfqpoint{1.545947in}{0.237587in}}%
\pgfpathlineto{\pgfqpoint{1.543396in}{0.229483in}}%
\pgfpathlineto{\pgfqpoint{1.549100in}{0.227286in}}%
\pgfpathlineto{\pgfqpoint{1.556435in}{0.214961in}}%
\pgfpathlineto{\pgfqpoint{1.561887in}{0.211248in}}%
\pgfpathlineto{\pgfqpoint{1.563313in}{0.205818in}}%
\pgfpathlineto{\pgfqpoint{1.566472in}{0.203467in}}%
\pgfpathlineto{\pgfqpoint{1.565813in}{0.198959in}}%
\pgfpathlineto{\pgfqpoint{1.572620in}{0.195693in}}%
\pgfpathlineto{\pgfqpoint{1.570939in}{0.187020in}}%
\pgfpathlineto{\pgfqpoint{1.565080in}{0.182477in}}%
\pgfpathlineto{\pgfqpoint{1.562418in}{0.174035in}}%
\pgfpathlineto{\pgfqpoint{1.561711in}{0.162731in}}%
\pgfpathlineto{\pgfqpoint{1.558455in}{0.161649in}}%
\pgfpathlineto{\pgfqpoint{1.547816in}{0.151740in}}%
\pgfpathlineto{\pgfqpoint{1.538358in}{0.149125in}}%
\pgfpathlineto{\pgfqpoint{1.533845in}{0.153121in}}%
\pgfpathlineto{\pgfqpoint{1.538206in}{0.163727in}}%
\pgfpathlineto{\pgfqpoint{1.536874in}{0.168932in}}%
\pgfpathlineto{\pgfqpoint{1.543122in}{0.171570in}}%
\pgfpathlineto{\pgfqpoint{1.548965in}{0.186914in}}%
\pgfpathlineto{\pgfqpoint{1.546497in}{0.200020in}}%
\pgfpathlineto{\pgfqpoint{1.542548in}{0.203114in}}%
\pgfpathlineto{\pgfqpoint{1.529340in}{0.202112in}}%
\pgfpathlineto{\pgfqpoint{1.532002in}{0.198689in}}%
\pgfpathlineto{\pgfqpoint{1.522443in}{0.188861in}}%
\pgfpathlineto{\pgfqpoint{1.519362in}{0.194325in}}%
\pgfpathlineto{\pgfqpoint{1.520164in}{0.201464in}}%
\pgfpathlineto{\pgfqpoint{1.525652in}{0.201332in}}%
\pgfpathlineto{\pgfqpoint{1.530538in}{0.206577in}}%
\pgfpathlineto{\pgfqpoint{1.533429in}{0.214234in}}%
\pgfpathlineto{\pgfqpoint{1.544322in}{0.212073in}}%
\pgfpathlineto{\pgfqpoint{1.535601in}{0.216164in}}%
\pgfpathlineto{\pgfqpoint{1.536369in}{0.221643in}}%
\pgfpathlineto{\pgfqpoint{1.532740in}{0.224117in}}%
\pgfpathlineto{\pgfqpoint{1.531254in}{0.231582in}}%
\pgfpathlineto{\pgfqpoint{1.533830in}{0.235470in}}%
\pgfpathlineto{\pgfqpoint{1.527686in}{0.247546in}}%
\pgfpathlineto{\pgfqpoint{1.528204in}{0.255228in}}%
\pgfpathlineto{\pgfqpoint{1.521005in}{0.262070in}}%
\pgfpathlineto{\pgfqpoint{1.517111in}{0.267604in}}%
\pgfpathlineto{\pgfqpoint{1.522489in}{0.272880in}}%
\pgfpathlineto{\pgfqpoint{1.524482in}{0.280763in}}%
\pgfpathlineto{\pgfqpoint{1.522703in}{0.285939in}}%
\pgfpathlineto{\pgfqpoint{1.524784in}{0.292113in}}%
\pgfpathlineto{\pgfqpoint{1.522132in}{0.312241in}}%
\pgfpathlineto{\pgfqpoint{1.518167in}{0.321372in}}%
\pgfpathlineto{\pgfqpoint{1.513632in}{0.324804in}}%
\pgfpathlineto{\pgfqpoint{1.515562in}{0.336682in}}%
\pgfpathlineto{\pgfqpoint{1.514201in}{0.345960in}}%
\pgfpathlineto{\pgfqpoint{1.516667in}{0.351910in}}%
\pgfpathlineto{\pgfqpoint{1.511563in}{0.352744in}}%
\pgfpathlineto{\pgfqpoint{1.510196in}{0.338291in}}%
\pgfpathlineto{\pgfqpoint{1.504461in}{0.322166in}}%
\pgfpathlineto{\pgfqpoint{1.499856in}{0.325838in}}%
\pgfpathlineto{\pgfqpoint{1.499241in}{0.331138in}}%
\pgfpathlineto{\pgfqpoint{1.493607in}{0.338377in}}%
\pgfpathlineto{\pgfqpoint{1.496469in}{0.343116in}}%
\pgfpathlineto{\pgfqpoint{1.495580in}{0.353785in}}%
\pgfpathlineto{\pgfqpoint{1.488289in}{0.350033in}}%
\pgfpathlineto{\pgfqpoint{1.489719in}{0.344246in}}%
\pgfpathlineto{\pgfqpoint{1.488346in}{0.336899in}}%
\pgfpathlineto{\pgfqpoint{1.480161in}{0.336913in}}%
\pgfpathlineto{\pgfqpoint{1.476493in}{0.341467in}}%
\pgfpathlineto{\pgfqpoint{1.470956in}{0.342336in}}%
\pgfpathlineto{\pgfqpoint{1.467230in}{0.347238in}}%
\pgfpathlineto{\pgfqpoint{1.460820in}{0.360701in}}%
\pgfpathlineto{\pgfqpoint{1.458742in}{0.371155in}}%
\pgfpathlineto{\pgfqpoint{1.460071in}{0.375597in}}%
\pgfpathlineto{\pgfqpoint{1.456969in}{0.385382in}}%
\pgfpathlineto{\pgfqpoint{1.451827in}{0.391047in}}%
\pgfpathlineto{\pgfqpoint{1.442934in}{0.405522in}}%
\pgfpathlineto{\pgfqpoint{1.436485in}{0.418096in}}%
\pgfpathlineto{\pgfqpoint{1.443065in}{0.418285in}}%
\pgfpathlineto{\pgfqpoint{1.446938in}{0.420925in}}%
\pgfpathlineto{\pgfqpoint{1.447699in}{0.429097in}}%
\pgfpathlineto{\pgfqpoint{1.429078in}{0.430083in}}%
\pgfpathlineto{\pgfqpoint{1.413240in}{0.447775in}}%
\pgfpathlineto{\pgfqpoint{1.417405in}{0.452429in}}%
\pgfpathlineto{\pgfqpoint{1.409745in}{0.453224in}}%
\pgfpathlineto{\pgfqpoint{1.394534in}{0.469672in}}%
\pgfpathlineto{\pgfqpoint{1.370436in}{0.477920in}}%
\pgfpathlineto{\pgfqpoint{1.367191in}{0.487587in}}%
\pgfpathlineto{\pgfqpoint{1.362190in}{0.492622in}}%
\pgfpathlineto{\pgfqpoint{1.361964in}{0.497349in}}%
\pgfpathlineto{\pgfqpoint{1.358202in}{0.500860in}}%
\pgfpathlineto{\pgfqpoint{1.363138in}{0.506727in}}%
\pgfpathlineto{\pgfqpoint{1.351880in}{0.507293in}}%
\pgfpathlineto{\pgfqpoint{1.349018in}{0.514886in}}%
\pgfpathlineto{\pgfqpoint{1.343894in}{0.518119in}}%
\pgfpathlineto{\pgfqpoint{1.354156in}{0.522021in}}%
\pgfpathlineto{\pgfqpoint{1.348819in}{0.528981in}}%
\pgfpathlineto{\pgfqpoint{1.338998in}{0.531902in}}%
\pgfpathlineto{\pgfqpoint{1.340526in}{0.544758in}}%
\pgfpathlineto{\pgfqpoint{1.337710in}{0.549309in}}%
\pgfpathlineto{\pgfqpoint{1.332639in}{0.552514in}}%
\pgfpathlineto{\pgfqpoint{1.328303in}{0.549299in}}%
\pgfpathlineto{\pgfqpoint{1.325741in}{0.552322in}}%
\pgfpathlineto{\pgfqpoint{1.318269in}{0.553189in}}%
\pgfpathlineto{\pgfqpoint{1.318870in}{0.564855in}}%
\pgfpathlineto{\pgfqpoint{1.308038in}{0.556975in}}%
\pgfpathlineto{\pgfqpoint{1.308266in}{0.540737in}}%
\pgfpathlineto{\pgfqpoint{1.296638in}{0.537624in}}%
\pgfpathlineto{\pgfqpoint{1.289044in}{0.531139in}}%
\pgfpathlineto{\pgfqpoint{1.282940in}{0.529773in}}%
\pgfpathlineto{\pgfqpoint{1.276335in}{0.536309in}}%
\pgfpathlineto{\pgfqpoint{1.270454in}{0.535442in}}%
\pgfpathlineto{\pgfqpoint{1.271545in}{0.541412in}}%
\pgfpathlineto{\pgfqpoint{1.263183in}{0.538198in}}%
\pgfpathlineto{\pgfqpoint{1.255437in}{0.537836in}}%
\pgfpathlineto{\pgfqpoint{1.246370in}{0.534577in}}%
\pgfpathlineto{\pgfqpoint{1.227971in}{0.536412in}}%
\pgfpathlineto{\pgfqpoint{1.222063in}{0.535205in}}%
\pgfpathlineto{\pgfqpoint{1.217800in}{0.540573in}}%
\pgfpathlineto{\pgfqpoint{1.209933in}{0.541109in}}%
\pgfpathlineto{\pgfqpoint{1.208296in}{0.548065in}}%
\pgfpathlineto{\pgfqpoint{1.212986in}{0.551855in}}%
\pgfpathlineto{\pgfqpoint{1.218487in}{0.551498in}}%
\pgfpathlineto{\pgfqpoint{1.223078in}{0.546332in}}%
\pgfpathlineto{\pgfqpoint{1.226102in}{0.554268in}}%
\pgfpathlineto{\pgfqpoint{1.221883in}{0.563060in}}%
\pgfpathlineto{\pgfqpoint{1.231433in}{0.570800in}}%
\pgfpathlineto{\pgfqpoint{1.241003in}{0.573652in}}%
\pgfpathlineto{\pgfqpoint{1.247665in}{0.578305in}}%
\pgfpathlineto{\pgfqpoint{1.251981in}{0.583317in}}%
\pgfpathlineto{\pgfqpoint{1.254390in}{0.591533in}}%
\pgfpathlineto{\pgfqpoint{1.261998in}{0.589296in}}%
\pgfpathlineto{\pgfqpoint{1.278496in}{0.590445in}}%
\pgfpathlineto{\pgfqpoint{1.282037in}{0.581437in}}%
\pgfpathlineto{\pgfqpoint{1.287824in}{0.581044in}}%
\pgfpathlineto{\pgfqpoint{1.298355in}{0.567255in}}%
\pgfpathlineto{\pgfqpoint{1.298374in}{0.572281in}}%
\pgfpathlineto{\pgfqpoint{1.294695in}{0.573916in}}%
\pgfpathlineto{\pgfqpoint{1.287963in}{0.590377in}}%
\pgfpathlineto{\pgfqpoint{1.291239in}{0.592614in}}%
\pgfpathlineto{\pgfqpoint{1.282247in}{0.601202in}}%
\pgfpathlineto{\pgfqpoint{1.274210in}{0.602874in}}%
\pgfpathlineto{\pgfqpoint{1.266552in}{0.599905in}}%
\pgfpathlineto{\pgfqpoint{1.253436in}{0.602217in}}%
\pgfpathlineto{\pgfqpoint{1.247216in}{0.598057in}}%
\pgfpathlineto{\pgfqpoint{1.233052in}{0.594843in}}%
\pgfpathlineto{\pgfqpoint{1.231756in}{0.590217in}}%
\pgfpathlineto{\pgfqpoint{1.220875in}{0.588956in}}%
\pgfpathlineto{\pgfqpoint{1.218082in}{0.583667in}}%
\pgfpathlineto{\pgfqpoint{1.211858in}{0.579793in}}%
\pgfpathlineto{\pgfqpoint{1.204338in}{0.580459in}}%
\pgfpathlineto{\pgfqpoint{1.200584in}{0.576713in}}%
\pgfpathlineto{\pgfqpoint{1.195847in}{0.576889in}}%
\pgfpathlineto{\pgfqpoint{1.191320in}{0.580260in}}%
\pgfpathlineto{\pgfqpoint{1.183407in}{0.579659in}}%
\pgfpathlineto{\pgfqpoint{1.181543in}{0.576551in}}%
\pgfpathlineto{\pgfqpoint{1.173230in}{0.579741in}}%
\pgfpathlineto{\pgfqpoint{1.166172in}{0.571509in}}%
\pgfpathlineto{\pgfqpoint{1.172940in}{0.563543in}}%
\pgfpathlineto{\pgfqpoint{1.175543in}{0.556992in}}%
\pgfpathlineto{\pgfqpoint{1.173143in}{0.551444in}}%
\pgfpathlineto{\pgfqpoint{1.163527in}{0.547615in}}%
\pgfpathlineto{\pgfqpoint{1.157955in}{0.550731in}}%
\pgfpathlineto{\pgfqpoint{1.150944in}{0.549044in}}%
\pgfpathlineto{\pgfqpoint{1.150098in}{0.544181in}}%
\pgfpathlineto{\pgfqpoint{1.141562in}{0.545902in}}%
\pgfpathlineto{\pgfqpoint{1.143452in}{0.541042in}}%
\pgfpathlineto{\pgfqpoint{1.130862in}{0.540018in}}%
\pgfpathlineto{\pgfqpoint{1.122712in}{0.546186in}}%
\pgfpathlineto{\pgfqpoint{1.118358in}{0.542263in}}%
\pgfpathlineto{\pgfqpoint{1.106774in}{0.546439in}}%
\pgfpathlineto{\pgfqpoint{1.103890in}{0.542210in}}%
\pgfpathlineto{\pgfqpoint{1.098147in}{0.540328in}}%
\pgfpathlineto{\pgfqpoint{1.093539in}{0.545286in}}%
\pgfpathlineto{\pgfqpoint{1.077786in}{0.544054in}}%
\pgfpathlineto{\pgfqpoint{1.077803in}{0.536740in}}%
\pgfpathlineto{\pgfqpoint{1.071475in}{0.534805in}}%
\pgfpathlineto{\pgfqpoint{1.066551in}{0.538564in}}%
\pgfpathlineto{\pgfqpoint{1.060044in}{0.536475in}}%
\pgfpathlineto{\pgfqpoint{1.051621in}{0.535978in}}%
\pgfpathlineto{\pgfqpoint{1.039073in}{0.541266in}}%
\pgfpathlineto{\pgfqpoint{1.031889in}{0.535813in}}%
\pgfpathlineto{\pgfqpoint{1.021612in}{0.543100in}}%
\pgfpathlineto{\pgfqpoint{1.018252in}{0.540338in}}%
\pgfpathlineto{\pgfqpoint{1.015893in}{0.533201in}}%
\pgfpathlineto{\pgfqpoint{1.009287in}{0.529017in}}%
\pgfpathlineto{\pgfqpoint{0.998515in}{0.533807in}}%
\pgfpathlineto{\pgfqpoint{0.980669in}{0.545640in}}%
\pgfpathlineto{\pgfqpoint{0.975467in}{0.546692in}}%
\pgfpathlineto{\pgfqpoint{0.972533in}{0.544226in}}%
\pgfpathlineto{\pgfqpoint{0.964683in}{0.545902in}}%
\pgfpathlineto{\pgfqpoint{0.960854in}{0.549586in}}%
\pgfpathlineto{\pgfqpoint{0.953014in}{0.551571in}}%
\pgfpathlineto{\pgfqpoint{0.942307in}{0.551685in}}%
\pgfpathlineto{\pgfqpoint{0.938056in}{0.556078in}}%
\pgfpathlineto{\pgfqpoint{0.946203in}{0.560625in}}%
\pgfpathlineto{\pgfqpoint{0.944302in}{0.565112in}}%
\pgfpathlineto{\pgfqpoint{0.938704in}{0.564104in}}%
\pgfpathlineto{\pgfqpoint{0.935908in}{0.560576in}}%
\pgfpathlineto{\pgfqpoint{0.922803in}{0.555518in}}%
\pgfpathlineto{\pgfqpoint{0.916176in}{0.557239in}}%
\pgfpathlineto{\pgfqpoint{0.913387in}{0.567509in}}%
\pgfpathlineto{\pgfqpoint{0.905121in}{0.563093in}}%
\pgfpathlineto{\pgfqpoint{0.902718in}{0.576108in}}%
\pgfpathlineto{\pgfqpoint{0.898801in}{0.572754in}}%
\pgfpathlineto{\pgfqpoint{0.899456in}{0.567746in}}%
\pgfpathlineto{\pgfqpoint{0.890449in}{0.569069in}}%
\pgfpathlineto{\pgfqpoint{0.899819in}{0.577796in}}%
\pgfpathlineto{\pgfqpoint{0.909315in}{0.572610in}}%
\pgfpathlineto{\pgfqpoint{0.911351in}{0.575213in}}%
\pgfpathlineto{\pgfqpoint{0.921340in}{0.572093in}}%
\pgfpathlineto{\pgfqpoint{0.920716in}{0.575646in}}%
\pgfpathlineto{\pgfqpoint{0.947970in}{0.576822in}}%
\pgfpathlineto{\pgfqpoint{0.961266in}{0.571280in}}%
\pgfpathlineto{\pgfqpoint{0.966411in}{0.565812in}}%
\pgfpathlineto{\pgfqpoint{0.964884in}{0.561039in}}%
\pgfpathlineto{\pgfqpoint{0.970050in}{0.559233in}}%
\pgfpathlineto{\pgfqpoint{0.982887in}{0.566766in}}%
\pgfpathlineto{\pgfqpoint{0.999493in}{0.567249in}}%
\pgfpathlineto{\pgfqpoint{1.014575in}{0.562708in}}%
\pgfpathlineto{\pgfqpoint{1.023118in}{0.563470in}}%
\pgfpathlineto{\pgfqpoint{1.026299in}{0.556425in}}%
\pgfpathlineto{\pgfqpoint{1.031077in}{0.564192in}}%
\pgfpathlineto{\pgfqpoint{1.034015in}{0.565873in}}%
\pgfpathlineto{\pgfqpoint{1.047608in}{0.569250in}}%
\pgfpathlineto{\pgfqpoint{1.052661in}{0.567187in}}%
\pgfpathlineto{\pgfqpoint{1.058124in}{0.569636in}}%
\pgfpathlineto{\pgfqpoint{1.064642in}{0.568355in}}%
\pgfpathlineto{\pgfqpoint{1.066180in}{0.571209in}}%
\pgfpathlineto{\pgfqpoint{1.079778in}{0.584306in}}%
\pgfpathlineto{\pgfqpoint{1.085537in}{0.586248in}}%
\pgfpathlineto{\pgfqpoint{1.088940in}{0.593316in}}%
\pgfpathlineto{\pgfqpoint{1.107003in}{0.598791in}}%
\pgfpathlineto{\pgfqpoint{1.109264in}{0.602966in}}%
\pgfpathlineto{\pgfqpoint{1.083665in}{0.609411in}}%
\pgfpathlineto{\pgfqpoint{1.084672in}{0.616343in}}%
\pgfpathlineto{\pgfqpoint{1.082040in}{0.625497in}}%
\pgfpathlineto{\pgfqpoint{1.076079in}{0.624736in}}%
\pgfpathlineto{\pgfqpoint{1.072024in}{0.612856in}}%
\pgfpathlineto{\pgfqpoint{1.067317in}{0.611877in}}%
\pgfpathlineto{\pgfqpoint{1.064253in}{0.615723in}}%
\pgfpathlineto{\pgfqpoint{1.067827in}{0.632546in}}%
\pgfpathlineto{\pgfqpoint{1.064268in}{0.639596in}}%
\pgfpathlineto{\pgfqpoint{1.057374in}{0.647906in}}%
\pgfpathlineto{\pgfqpoint{1.060687in}{0.654248in}}%
\pgfpathlineto{\pgfqpoint{1.046139in}{0.654476in}}%
\pgfpathlineto{\pgfqpoint{1.046471in}{0.656936in}}%
\pgfpathlineto{\pgfqpoint{1.037790in}{0.660383in}}%
\pgfpathclose%
\pgfusepath{stroke,fill}%
\end{pgfscope}%
\begin{pgfscope}%
\pgfpathrectangle{\pgfqpoint{0.100000in}{0.100000in}}{\pgfqpoint{5.307240in}{3.397500in}}%
\pgfusepath{clip}%
\pgfsetbuttcap%
\pgfsetmiterjoin%
\definecolor{currentfill}{rgb}{1.000000,1.000000,1.000000}%
\pgfsetfillcolor{currentfill}%
\pgfsetlinewidth{0.501875pt}%
\definecolor{currentstroke}{rgb}{0.827451,0.827451,0.827451}%
\pgfsetstrokecolor{currentstroke}%
\pgfsetdash{}{0pt}%
\pgfpathmoveto{\pgfqpoint{1.009129in}{0.781017in}}%
\pgfpathlineto{\pgfqpoint{1.007330in}{0.777856in}}%
\pgfpathlineto{\pgfqpoint{1.012090in}{0.771124in}}%
\pgfpathlineto{\pgfqpoint{1.004566in}{0.764827in}}%
\pgfpathlineto{\pgfqpoint{1.004776in}{0.759410in}}%
\pgfpathlineto{\pgfqpoint{1.001374in}{0.758133in}}%
\pgfpathlineto{\pgfqpoint{0.989682in}{0.766210in}}%
\pgfpathlineto{\pgfqpoint{0.981347in}{0.783892in}}%
\pgfpathlineto{\pgfqpoint{0.980789in}{0.789001in}}%
\pgfpathlineto{\pgfqpoint{0.982932in}{0.795116in}}%
\pgfpathlineto{\pgfqpoint{0.991963in}{0.785761in}}%
\pgfpathlineto{\pgfqpoint{0.994726in}{0.787566in}}%
\pgfpathlineto{\pgfqpoint{1.002386in}{0.785849in}}%
\pgfpathclose%
\pgfusepath{stroke,fill}%
\end{pgfscope}%
\begin{pgfscope}%
\pgfpathrectangle{\pgfqpoint{0.100000in}{0.100000in}}{\pgfqpoint{5.307240in}{3.397500in}}%
\pgfusepath{clip}%
\pgfsetbuttcap%
\pgfsetmiterjoin%
\definecolor{currentfill}{rgb}{1.000000,1.000000,1.000000}%
\pgfsetfillcolor{currentfill}%
\pgfsetlinewidth{0.501875pt}%
\definecolor{currentstroke}{rgb}{0.827451,0.827451,0.827451}%
\pgfsetstrokecolor{currentstroke}%
\pgfsetdash{}{0pt}%
\pgfpathmoveto{\pgfqpoint{0.870460in}{0.576174in}}%
\pgfpathlineto{\pgfqpoint{0.862243in}{0.574840in}}%
\pgfpathlineto{\pgfqpoint{0.856296in}{0.577589in}}%
\pgfpathlineto{\pgfqpoint{0.853696in}{0.582064in}}%
\pgfpathlineto{\pgfqpoint{0.856643in}{0.588484in}}%
\pgfpathlineto{\pgfqpoint{0.863204in}{0.586786in}}%
\pgfpathlineto{\pgfqpoint{0.873990in}{0.590595in}}%
\pgfpathlineto{\pgfqpoint{0.877768in}{0.585334in}}%
\pgfpathlineto{\pgfqpoint{0.881430in}{0.586305in}}%
\pgfpathlineto{\pgfqpoint{0.890296in}{0.583031in}}%
\pgfpathlineto{\pgfqpoint{0.894128in}{0.578872in}}%
\pgfpathlineto{\pgfqpoint{0.889453in}{0.567994in}}%
\pgfpathlineto{\pgfqpoint{0.884802in}{0.565688in}}%
\pgfpathlineto{\pgfqpoint{0.880623in}{0.566931in}}%
\pgfpathclose%
\pgfusepath{stroke,fill}%
\end{pgfscope}%
\begin{pgfscope}%
\pgfpathrectangle{\pgfqpoint{0.100000in}{0.100000in}}{\pgfqpoint{5.307240in}{3.397500in}}%
\pgfusepath{clip}%
\pgfsetbuttcap%
\pgfsetmiterjoin%
\definecolor{currentfill}{rgb}{1.000000,1.000000,1.000000}%
\pgfsetfillcolor{currentfill}%
\pgfsetlinewidth{0.501875pt}%
\definecolor{currentstroke}{rgb}{0.827451,0.827451,0.827451}%
\pgfsetstrokecolor{currentstroke}%
\pgfsetdash{}{0pt}%
\pgfpathmoveto{\pgfqpoint{0.796984in}{0.586059in}}%
\pgfpathlineto{\pgfqpoint{0.785339in}{0.591613in}}%
\pgfpathlineto{\pgfqpoint{0.774564in}{0.593265in}}%
\pgfpathlineto{\pgfqpoint{0.771167in}{0.595630in}}%
\pgfpathlineto{\pgfqpoint{0.775035in}{0.600286in}}%
\pgfpathlineto{\pgfqpoint{0.781898in}{0.594470in}}%
\pgfpathlineto{\pgfqpoint{0.787655in}{0.595272in}}%
\pgfpathlineto{\pgfqpoint{0.797016in}{0.599300in}}%
\pgfpathlineto{\pgfqpoint{0.797773in}{0.604488in}}%
\pgfpathlineto{\pgfqpoint{0.809271in}{0.601920in}}%
\pgfpathlineto{\pgfqpoint{0.806578in}{0.594109in}}%
\pgfpathlineto{\pgfqpoint{0.813904in}{0.593538in}}%
\pgfpathlineto{\pgfqpoint{0.807758in}{0.584246in}}%
\pgfpathlineto{\pgfqpoint{0.801806in}{0.587257in}}%
\pgfpathclose%
\pgfusepath{stroke,fill}%
\end{pgfscope}%
\begin{pgfscope}%
\pgfpathrectangle{\pgfqpoint{0.100000in}{0.100000in}}{\pgfqpoint{5.307240in}{3.397500in}}%
\pgfusepath{clip}%
\pgfsetbuttcap%
\pgfsetmiterjoin%
\definecolor{currentfill}{rgb}{1.000000,1.000000,1.000000}%
\pgfsetfillcolor{currentfill}%
\pgfsetlinewidth{0.501875pt}%
\definecolor{currentstroke}{rgb}{0.827451,0.827451,0.827451}%
\pgfsetstrokecolor{currentstroke}%
\pgfsetdash{}{0pt}%
\pgfpathmoveto{\pgfqpoint{0.776526in}{0.605261in}}%
\pgfpathlineto{\pgfqpoint{0.772246in}{0.602933in}}%
\pgfpathlineto{\pgfqpoint{0.760585in}{0.606510in}}%
\pgfpathlineto{\pgfqpoint{0.751871in}{0.604439in}}%
\pgfpathlineto{\pgfqpoint{0.747444in}{0.610323in}}%
\pgfpathlineto{\pgfqpoint{0.749375in}{0.613018in}}%
\pgfpathlineto{\pgfqpoint{0.755979in}{0.613468in}}%
\pgfpathlineto{\pgfqpoint{0.761029in}{0.611777in}}%
\pgfpathlineto{\pgfqpoint{0.766534in}{0.614454in}}%
\pgfpathlineto{\pgfqpoint{0.774972in}{0.610874in}}%
\pgfpathclose%
\pgfusepath{stroke,fill}%
\end{pgfscope}%
\begin{pgfscope}%
\pgfpathrectangle{\pgfqpoint{0.100000in}{0.100000in}}{\pgfqpoint{5.307240in}{3.397500in}}%
\pgfusepath{clip}%
\pgfsetbuttcap%
\pgfsetmiterjoin%
\definecolor{currentfill}{rgb}{1.000000,1.000000,1.000000}%
\pgfsetfillcolor{currentfill}%
\pgfsetlinewidth{0.501875pt}%
\definecolor{currentstroke}{rgb}{0.827451,0.827451,0.827451}%
\pgfsetstrokecolor{currentstroke}%
\pgfsetdash{}{0pt}%
\pgfpathmoveto{\pgfqpoint{0.639065in}{0.691871in}}%
\pgfpathlineto{\pgfqpoint{0.638756in}{0.686100in}}%
\pgfpathlineto{\pgfqpoint{0.630242in}{0.682474in}}%
\pgfpathlineto{\pgfqpoint{0.622468in}{0.691551in}}%
\pgfpathlineto{\pgfqpoint{0.631186in}{0.694576in}}%
\pgfpathclose%
\pgfusepath{stroke,fill}%
\end{pgfscope}%
\begin{pgfscope}%
\pgfpathrectangle{\pgfqpoint{0.100000in}{0.100000in}}{\pgfqpoint{5.307240in}{3.397500in}}%
\pgfusepath{clip}%
\pgfsetbuttcap%
\pgfsetmiterjoin%
\definecolor{currentfill}{rgb}{1.000000,1.000000,1.000000}%
\pgfsetfillcolor{currentfill}%
\pgfsetlinewidth{0.501875pt}%
\definecolor{currentstroke}{rgb}{0.827451,0.827451,0.827451}%
\pgfsetstrokecolor{currentstroke}%
\pgfsetdash{}{0pt}%
\pgfpathmoveto{\pgfqpoint{0.568852in}{0.724309in}}%
\pgfpathlineto{\pgfqpoint{0.570660in}{0.726818in}}%
\pgfpathlineto{\pgfqpoint{0.581102in}{0.726386in}}%
\pgfpathlineto{\pgfqpoint{0.582934in}{0.730118in}}%
\pgfpathlineto{\pgfqpoint{0.587515in}{0.726861in}}%
\pgfpathlineto{\pgfqpoint{0.584870in}{0.720595in}}%
\pgfpathlineto{\pgfqpoint{0.580978in}{0.720064in}}%
\pgfpathlineto{\pgfqpoint{0.571677in}{0.722031in}}%
\pgfpathclose%
\pgfusepath{stroke,fill}%
\end{pgfscope}%
\begin{pgfscope}%
\pgfpathrectangle{\pgfqpoint{0.100000in}{0.100000in}}{\pgfqpoint{5.307240in}{3.397500in}}%
\pgfusepath{clip}%
\pgfsetbuttcap%
\pgfsetmiterjoin%
\definecolor{currentfill}{rgb}{1.000000,1.000000,1.000000}%
\pgfsetfillcolor{currentfill}%
\pgfsetlinewidth{0.501875pt}%
\definecolor{currentstroke}{rgb}{0.827451,0.827451,0.827451}%
\pgfsetstrokecolor{currentstroke}%
\pgfsetdash{}{0pt}%
\pgfpathmoveto{\pgfqpoint{0.554044in}{0.743731in}}%
\pgfpathlineto{\pgfqpoint{0.553410in}{0.749383in}}%
\pgfpathlineto{\pgfqpoint{0.558859in}{0.748964in}}%
\pgfpathlineto{\pgfqpoint{0.558479in}{0.756902in}}%
\pgfpathlineto{\pgfqpoint{0.563960in}{0.752617in}}%
\pgfpathlineto{\pgfqpoint{0.562064in}{0.746426in}}%
\pgfpathclose%
\pgfusepath{stroke,fill}%
\end{pgfscope}%
\begin{pgfscope}%
\pgfpathrectangle{\pgfqpoint{0.100000in}{0.100000in}}{\pgfqpoint{5.307240in}{3.397500in}}%
\pgfusepath{clip}%
\pgfsetbuttcap%
\pgfsetmiterjoin%
\definecolor{currentfill}{rgb}{1.000000,1.000000,1.000000}%
\pgfsetfillcolor{currentfill}%
\pgfsetlinewidth{0.501875pt}%
\definecolor{currentstroke}{rgb}{0.827451,0.827451,0.827451}%
\pgfsetstrokecolor{currentstroke}%
\pgfsetdash{}{0pt}%
\pgfpathmoveto{\pgfqpoint{1.034069in}{0.945167in}}%
\pgfpathlineto{\pgfqpoint{1.025379in}{0.947043in}}%
\pgfpathlineto{\pgfqpoint{1.023512in}{0.952850in}}%
\pgfpathlineto{\pgfqpoint{1.026499in}{0.958655in}}%
\pgfpathlineto{\pgfqpoint{1.033188in}{0.961687in}}%
\pgfpathlineto{\pgfqpoint{1.040846in}{0.946786in}}%
\pgfpathlineto{\pgfqpoint{1.047793in}{0.946125in}}%
\pgfpathlineto{\pgfqpoint{1.052975in}{0.941526in}}%
\pgfpathlineto{\pgfqpoint{1.052873in}{0.935200in}}%
\pgfpathlineto{\pgfqpoint{1.049084in}{0.932167in}}%
\pgfpathlineto{\pgfqpoint{1.055246in}{0.913457in}}%
\pgfpathlineto{\pgfqpoint{1.061988in}{0.906591in}}%
\pgfpathlineto{\pgfqpoint{1.055094in}{0.904413in}}%
\pgfpathlineto{\pgfqpoint{1.049515in}{0.912116in}}%
\pgfpathlineto{\pgfqpoint{1.037388in}{0.909731in}}%
\pgfpathlineto{\pgfqpoint{1.041160in}{0.917354in}}%
\pgfpathlineto{\pgfqpoint{1.039404in}{0.920391in}}%
\pgfpathlineto{\pgfqpoint{1.040433in}{0.928337in}}%
\pgfpathlineto{\pgfqpoint{1.038030in}{0.942702in}}%
\pgfpathclose%
\pgfusepath{stroke,fill}%
\end{pgfscope}%
\begin{pgfscope}%
\pgfpathrectangle{\pgfqpoint{0.100000in}{0.100000in}}{\pgfqpoint{5.307240in}{3.397500in}}%
\pgfusepath{clip}%
\pgfsetbuttcap%
\pgfsetmiterjoin%
\definecolor{currentfill}{rgb}{1.000000,1.000000,1.000000}%
\pgfsetfillcolor{currentfill}%
\pgfsetlinewidth{0.501875pt}%
\definecolor{currentstroke}{rgb}{0.827451,0.827451,0.827451}%
\pgfsetstrokecolor{currentstroke}%
\pgfsetdash{}{0pt}%
\pgfpathmoveto{\pgfqpoint{1.339371in}{0.517152in}}%
\pgfpathlineto{\pgfqpoint{1.326774in}{0.517295in}}%
\pgfpathlineto{\pgfqpoint{1.327745in}{0.523618in}}%
\pgfpathlineto{\pgfqpoint{1.332786in}{0.525879in}}%
\pgfpathclose%
\pgfusepath{stroke,fill}%
\end{pgfscope}%
\begin{pgfscope}%
\pgfpathrectangle{\pgfqpoint{0.100000in}{0.100000in}}{\pgfqpoint{5.307240in}{3.397500in}}%
\pgfusepath{clip}%
\pgfsetbuttcap%
\pgfsetmiterjoin%
\definecolor{currentfill}{rgb}{1.000000,1.000000,1.000000}%
\pgfsetfillcolor{currentfill}%
\pgfsetlinewidth{0.501875pt}%
\definecolor{currentstroke}{rgb}{0.827451,0.827451,0.827451}%
\pgfsetstrokecolor{currentstroke}%
\pgfsetdash{}{0pt}%
\pgfpathmoveto{\pgfqpoint{1.321890in}{0.522879in}}%
\pgfpathlineto{\pgfqpoint{1.312865in}{0.520696in}}%
\pgfpathlineto{\pgfqpoint{1.304290in}{0.516956in}}%
\pgfpathlineto{\pgfqpoint{1.301642in}{0.521362in}}%
\pgfpathlineto{\pgfqpoint{1.316390in}{0.524345in}}%
\pgfpathclose%
\pgfusepath{stroke,fill}%
\end{pgfscope}%
\begin{pgfscope}%
\pgfpathrectangle{\pgfqpoint{0.100000in}{0.100000in}}{\pgfqpoint{5.307240in}{3.397500in}}%
\pgfusepath{clip}%
\pgfsetbuttcap%
\pgfsetmiterjoin%
\definecolor{currentfill}{rgb}{1.000000,1.000000,1.000000}%
\pgfsetfillcolor{currentfill}%
\pgfsetlinewidth{0.501875pt}%
\definecolor{currentstroke}{rgb}{0.827451,0.827451,0.827451}%
\pgfsetstrokecolor{currentstroke}%
\pgfsetdash{}{0pt}%
\pgfpathmoveto{\pgfqpoint{1.183553in}{0.518403in}}%
\pgfpathlineto{\pgfqpoint{1.183898in}{0.514030in}}%
\pgfpathlineto{\pgfqpoint{1.178697in}{0.511311in}}%
\pgfpathlineto{\pgfqpoint{1.164185in}{0.517545in}}%
\pgfpathlineto{\pgfqpoint{1.161935in}{0.515038in}}%
\pgfpathlineto{\pgfqpoint{1.158424in}{0.526901in}}%
\pgfpathlineto{\pgfqpoint{1.165694in}{0.528551in}}%
\pgfpathlineto{\pgfqpoint{1.169055in}{0.523677in}}%
\pgfpathlineto{\pgfqpoint{1.176353in}{0.529779in}}%
\pgfpathclose%
\pgfusepath{stroke,fill}%
\end{pgfscope}%
\begin{pgfscope}%
\pgfpathrectangle{\pgfqpoint{0.100000in}{0.100000in}}{\pgfqpoint{5.307240in}{3.397500in}}%
\pgfusepath{clip}%
\pgfsetbuttcap%
\pgfsetmiterjoin%
\definecolor{currentfill}{rgb}{1.000000,1.000000,1.000000}%
\pgfsetfillcolor{currentfill}%
\pgfsetlinewidth{0.501875pt}%
\definecolor{currentstroke}{rgb}{0.827451,0.827451,0.827451}%
\pgfsetstrokecolor{currentstroke}%
\pgfsetdash{}{0pt}%
\pgfpathmoveto{\pgfqpoint{1.510906in}{0.275989in}}%
\pgfpathlineto{\pgfqpoint{1.498736in}{0.271458in}}%
\pgfpathlineto{\pgfqpoint{1.492290in}{0.271187in}}%
\pgfpathlineto{\pgfqpoint{1.497537in}{0.280798in}}%
\pgfpathlineto{\pgfqpoint{1.502019in}{0.284383in}}%
\pgfpathlineto{\pgfqpoint{1.501875in}{0.294336in}}%
\pgfpathlineto{\pgfqpoint{1.506747in}{0.312304in}}%
\pgfpathlineto{\pgfqpoint{1.511235in}{0.315717in}}%
\pgfpathlineto{\pgfqpoint{1.521594in}{0.310776in}}%
\pgfpathlineto{\pgfqpoint{1.520081in}{0.307945in}}%
\pgfpathlineto{\pgfqpoint{1.520866in}{0.294343in}}%
\pgfpathlineto{\pgfqpoint{1.517507in}{0.290672in}}%
\pgfpathlineto{\pgfqpoint{1.516065in}{0.281012in}}%
\pgfpathclose%
\pgfusepath{stroke,fill}%
\end{pgfscope}%
\begin{pgfscope}%
\pgfpathrectangle{\pgfqpoint{0.100000in}{0.100000in}}{\pgfqpoint{5.307240in}{3.397500in}}%
\pgfusepath{clip}%
\pgfsetbuttcap%
\pgfsetmiterjoin%
\definecolor{currentfill}{rgb}{1.000000,1.000000,1.000000}%
\pgfsetfillcolor{currentfill}%
\pgfsetlinewidth{0.501875pt}%
\definecolor{currentstroke}{rgb}{0.827451,0.827451,0.827451}%
\pgfsetstrokecolor{currentstroke}%
\pgfsetdash{}{0pt}%
\pgfpathmoveto{\pgfqpoint{1.483956in}{0.320530in}}%
\pgfpathlineto{\pgfqpoint{1.491085in}{0.308220in}}%
\pgfpathlineto{\pgfqpoint{1.489764in}{0.305284in}}%
\pgfpathlineto{\pgfqpoint{1.499299in}{0.302298in}}%
\pgfpathlineto{\pgfqpoint{1.496561in}{0.290983in}}%
\pgfpathlineto{\pgfqpoint{1.492409in}{0.291506in}}%
\pgfpathlineto{\pgfqpoint{1.481500in}{0.311231in}}%
\pgfpathlineto{\pgfqpoint{1.483051in}{0.302353in}}%
\pgfpathlineto{\pgfqpoint{1.481194in}{0.297241in}}%
\pgfpathlineto{\pgfqpoint{1.476574in}{0.294965in}}%
\pgfpathlineto{\pgfqpoint{1.473947in}{0.297417in}}%
\pgfpathlineto{\pgfqpoint{1.472481in}{0.308784in}}%
\pgfpathlineto{\pgfqpoint{1.473346in}{0.316155in}}%
\pgfpathlineto{\pgfqpoint{1.470269in}{0.319373in}}%
\pgfpathlineto{\pgfqpoint{1.478397in}{0.328685in}}%
\pgfpathlineto{\pgfqpoint{1.483731in}{0.331448in}}%
\pgfpathlineto{\pgfqpoint{1.485506in}{0.327817in}}%
\pgfpathlineto{\pgfqpoint{1.491071in}{0.330627in}}%
\pgfpathlineto{\pgfqpoint{1.495102in}{0.322997in}}%
\pgfpathlineto{\pgfqpoint{1.503821in}{0.313290in}}%
\pgfpathlineto{\pgfqpoint{1.502630in}{0.308894in}}%
\pgfpathlineto{\pgfqpoint{1.497790in}{0.304103in}}%
\pgfpathlineto{\pgfqpoint{1.493313in}{0.306361in}}%
\pgfpathclose%
\pgfusepath{stroke,fill}%
\end{pgfscope}%
\begin{pgfscope}%
\pgfpathrectangle{\pgfqpoint{0.100000in}{0.100000in}}{\pgfqpoint{5.307240in}{3.397500in}}%
\pgfusepath{clip}%
\pgfsetbuttcap%
\pgfsetmiterjoin%
\definecolor{currentfill}{rgb}{1.000000,1.000000,1.000000}%
\pgfsetfillcolor{currentfill}%
\pgfsetlinewidth{0.501875pt}%
\definecolor{currentstroke}{rgb}{0.827451,0.827451,0.827451}%
\pgfsetstrokecolor{currentstroke}%
\pgfsetdash{}{0pt}%
\pgfpathmoveto{\pgfqpoint{1.130181in}{0.495071in}}%
\pgfpathlineto{\pgfqpoint{1.121350in}{0.493921in}}%
\pgfpathlineto{\pgfqpoint{1.108913in}{0.489140in}}%
\pgfpathlineto{\pgfqpoint{1.106012in}{0.491738in}}%
\pgfpathlineto{\pgfqpoint{1.119004in}{0.494674in}}%
\pgfpathlineto{\pgfqpoint{1.115521in}{0.497235in}}%
\pgfpathlineto{\pgfqpoint{1.107466in}{0.499112in}}%
\pgfpathlineto{\pgfqpoint{1.105316in}{0.505114in}}%
\pgfpathlineto{\pgfqpoint{1.109369in}{0.510286in}}%
\pgfpathlineto{\pgfqpoint{1.112782in}{0.522155in}}%
\pgfpathlineto{\pgfqpoint{1.125250in}{0.524366in}}%
\pgfpathlineto{\pgfqpoint{1.131286in}{0.519239in}}%
\pgfpathlineto{\pgfqpoint{1.136956in}{0.524202in}}%
\pgfpathlineto{\pgfqpoint{1.144067in}{0.522984in}}%
\pgfpathlineto{\pgfqpoint{1.149178in}{0.513818in}}%
\pgfpathlineto{\pgfqpoint{1.153506in}{0.521965in}}%
\pgfpathlineto{\pgfqpoint{1.165789in}{0.502908in}}%
\pgfpathlineto{\pgfqpoint{1.161352in}{0.501384in}}%
\pgfpathlineto{\pgfqpoint{1.158628in}{0.497253in}}%
\pgfpathlineto{\pgfqpoint{1.163774in}{0.493483in}}%
\pgfpathlineto{\pgfqpoint{1.155795in}{0.489949in}}%
\pgfpathlineto{\pgfqpoint{1.150010in}{0.491895in}}%
\pgfpathlineto{\pgfqpoint{1.139083in}{0.498345in}}%
\pgfpathclose%
\pgfusepath{stroke,fill}%
\end{pgfscope}%
\begin{pgfscope}%
\pgfpathrectangle{\pgfqpoint{0.100000in}{0.100000in}}{\pgfqpoint{5.307240in}{3.397500in}}%
\pgfusepath{clip}%
\pgfsetbuttcap%
\pgfsetmiterjoin%
\definecolor{currentfill}{rgb}{1.000000,1.000000,1.000000}%
\pgfsetfillcolor{currentfill}%
\pgfsetlinewidth{0.501875pt}%
\definecolor{currentstroke}{rgb}{0.827451,0.827451,0.827451}%
\pgfsetstrokecolor{currentstroke}%
\pgfsetdash{}{0pt}%
\pgfpathmoveto{\pgfqpoint{1.476541in}{0.239629in}}%
\pgfpathlineto{\pgfqpoint{1.474790in}{0.257229in}}%
\pgfpathlineto{\pgfqpoint{1.479509in}{0.263691in}}%
\pgfpathlineto{\pgfqpoint{1.474463in}{0.264141in}}%
\pgfpathlineto{\pgfqpoint{1.476308in}{0.276709in}}%
\pgfpathlineto{\pgfqpoint{1.478483in}{0.284171in}}%
\pgfpathlineto{\pgfqpoint{1.476593in}{0.294195in}}%
\pgfpathlineto{\pgfqpoint{1.481215in}{0.296136in}}%
\pgfpathlineto{\pgfqpoint{1.482723in}{0.298728in}}%
\pgfpathlineto{\pgfqpoint{1.487277in}{0.297745in}}%
\pgfpathlineto{\pgfqpoint{1.491219in}{0.289713in}}%
\pgfpathlineto{\pgfqpoint{1.491951in}{0.282355in}}%
\pgfpathlineto{\pgfqpoint{1.487129in}{0.260317in}}%
\pgfpathclose%
\pgfusepath{stroke,fill}%
\end{pgfscope}%
\begin{pgfscope}%
\pgfpathrectangle{\pgfqpoint{0.100000in}{0.100000in}}{\pgfqpoint{5.307240in}{3.397500in}}%
\pgfusepath{clip}%
\pgfsetbuttcap%
\pgfsetmiterjoin%
\definecolor{currentfill}{rgb}{1.000000,1.000000,1.000000}%
\pgfsetfillcolor{currentfill}%
\pgfsetlinewidth{0.501875pt}%
\definecolor{currentstroke}{rgb}{0.827451,0.827451,0.827451}%
\pgfsetstrokecolor{currentstroke}%
\pgfsetdash{}{0pt}%
\pgfpathmoveto{\pgfqpoint{1.475494in}{0.293355in}}%
\pgfpathlineto{\pgfqpoint{1.475252in}{0.284774in}}%
\pgfpathlineto{\pgfqpoint{1.471294in}{0.280878in}}%
\pgfpathlineto{\pgfqpoint{1.466720in}{0.282318in}}%
\pgfpathlineto{\pgfqpoint{1.472502in}{0.294783in}}%
\pgfpathclose%
\pgfusepath{stroke,fill}%
\end{pgfscope}%
\begin{pgfscope}%
\pgfpathrectangle{\pgfqpoint{0.100000in}{0.100000in}}{\pgfqpoint{5.307240in}{3.397500in}}%
\pgfusepath{clip}%
\pgfsetbuttcap%
\pgfsetmiterjoin%
\definecolor{currentfill}{rgb}{1.000000,1.000000,1.000000}%
\pgfsetfillcolor{currentfill}%
\pgfsetlinewidth{0.501875pt}%
\definecolor{currentstroke}{rgb}{0.827451,0.827451,0.827451}%
\pgfsetstrokecolor{currentstroke}%
\pgfsetdash{}{0pt}%
\pgfpathmoveto{\pgfqpoint{1.524054in}{0.255256in}}%
\pgfpathlineto{\pgfqpoint{1.520679in}{0.240893in}}%
\pgfpathlineto{\pgfqpoint{1.512816in}{0.236354in}}%
\pgfpathlineto{\pgfqpoint{1.502500in}{0.240536in}}%
\pgfpathlineto{\pgfqpoint{1.505343in}{0.246117in}}%
\pgfpathlineto{\pgfqpoint{1.505285in}{0.249982in}}%
\pgfpathlineto{\pgfqpoint{1.509120in}{0.256549in}}%
\pgfpathlineto{\pgfqpoint{1.506424in}{0.255227in}}%
\pgfpathlineto{\pgfqpoint{1.508389in}{0.258433in}}%
\pgfpathlineto{\pgfqpoint{1.504656in}{0.265719in}}%
\pgfpathlineto{\pgfqpoint{1.508893in}{0.267062in}}%
\pgfpathlineto{\pgfqpoint{1.518701in}{0.258537in}}%
\pgfpathclose%
\pgfusepath{stroke,fill}%
\end{pgfscope}%
\begin{pgfscope}%
\pgfpathrectangle{\pgfqpoint{0.100000in}{0.100000in}}{\pgfqpoint{5.307240in}{3.397500in}}%
\pgfusepath{clip}%
\pgfsetbuttcap%
\pgfsetmiterjoin%
\definecolor{currentfill}{rgb}{1.000000,1.000000,1.000000}%
\pgfsetfillcolor{currentfill}%
\pgfsetlinewidth{0.501875pt}%
\definecolor{currentstroke}{rgb}{0.827451,0.827451,0.827451}%
\pgfsetstrokecolor{currentstroke}%
\pgfsetdash{}{0pt}%
\pgfpathmoveto{\pgfqpoint{1.489633in}{0.229861in}}%
\pgfpathlineto{\pgfqpoint{1.485185in}{0.228132in}}%
\pgfpathlineto{\pgfqpoint{1.487726in}{0.243287in}}%
\pgfpathlineto{\pgfqpoint{1.493063in}{0.246823in}}%
\pgfpathlineto{\pgfqpoint{1.491762in}{0.258178in}}%
\pgfpathlineto{\pgfqpoint{1.493931in}{0.263213in}}%
\pgfpathlineto{\pgfqpoint{1.498221in}{0.265101in}}%
\pgfpathlineto{\pgfqpoint{1.502726in}{0.260161in}}%
\pgfpathlineto{\pgfqpoint{1.506591in}{0.253459in}}%
\pgfpathlineto{\pgfqpoint{1.494165in}{0.240329in}}%
\pgfpathclose%
\pgfusepath{stroke,fill}%
\end{pgfscope}%
\begin{pgfscope}%
\pgfpathrectangle{\pgfqpoint{0.100000in}{0.100000in}}{\pgfqpoint{5.307240in}{3.397500in}}%
\pgfusepath{clip}%
\pgfsetbuttcap%
\pgfsetmiterjoin%
\definecolor{currentfill}{rgb}{1.000000,1.000000,1.000000}%
\pgfsetfillcolor{currentfill}%
\pgfsetlinewidth{0.501875pt}%
\definecolor{currentstroke}{rgb}{0.827451,0.827451,0.827451}%
\pgfsetstrokecolor{currentstroke}%
\pgfsetdash{}{0pt}%
\pgfpathmoveto{\pgfqpoint{1.526027in}{0.245656in}}%
\pgfpathlineto{\pgfqpoint{1.528206in}{0.235199in}}%
\pgfpathlineto{\pgfqpoint{1.518468in}{0.236395in}}%
\pgfpathclose%
\pgfusepath{stroke,fill}%
\end{pgfscope}%
\begin{pgfscope}%
\pgfpathrectangle{\pgfqpoint{0.100000in}{0.100000in}}{\pgfqpoint{5.307240in}{3.397500in}}%
\pgfusepath{clip}%
\pgfsetbuttcap%
\pgfsetmiterjoin%
\definecolor{currentfill}{rgb}{1.000000,1.000000,1.000000}%
\pgfsetfillcolor{currentfill}%
\pgfsetlinewidth{0.501875pt}%
\definecolor{currentstroke}{rgb}{0.827451,0.827451,0.827451}%
\pgfsetstrokecolor{currentstroke}%
\pgfsetdash{}{0pt}%
\pgfpathmoveto{\pgfqpoint{1.530674in}{0.229677in}}%
\pgfpathlineto{\pgfqpoint{1.532270in}{0.223225in}}%
\pgfpathlineto{\pgfqpoint{1.535324in}{0.221162in}}%
\pgfpathlineto{\pgfqpoint{1.534769in}{0.215089in}}%
\pgfpathlineto{\pgfqpoint{1.530501in}{0.213047in}}%
\pgfpathlineto{\pgfqpoint{1.527036in}{0.222112in}}%
\pgfpathclose%
\pgfusepath{stroke,fill}%
\end{pgfscope}%
\begin{pgfscope}%
\pgfpathrectangle{\pgfqpoint{0.100000in}{0.100000in}}{\pgfqpoint{5.307240in}{3.397500in}}%
\pgfusepath{clip}%
\pgfsetbuttcap%
\pgfsetmiterjoin%
\definecolor{currentfill}{rgb}{1.000000,1.000000,1.000000}%
\pgfsetfillcolor{currentfill}%
\pgfsetlinewidth{0.501875pt}%
\definecolor{currentstroke}{rgb}{0.827451,0.827451,0.827451}%
\pgfsetstrokecolor{currentstroke}%
\pgfsetdash{}{0pt}%
\pgfpathmoveto{\pgfqpoint{1.522547in}{0.232152in}}%
\pgfpathlineto{\pgfqpoint{1.519898in}{0.224658in}}%
\pgfpathlineto{\pgfqpoint{1.515785in}{0.224933in}}%
\pgfpathlineto{\pgfqpoint{1.513190in}{0.231113in}}%
\pgfpathlineto{\pgfqpoint{1.517417in}{0.234040in}}%
\pgfpathclose%
\pgfusepath{stroke,fill}%
\end{pgfscope}%
\begin{pgfscope}%
\pgfpathrectangle{\pgfqpoint{0.100000in}{0.100000in}}{\pgfqpoint{5.307240in}{3.397500in}}%
\pgfusepath{clip}%
\pgfsetbuttcap%
\pgfsetmiterjoin%
\definecolor{currentfill}{rgb}{1.000000,1.000000,1.000000}%
\pgfsetfillcolor{currentfill}%
\pgfsetlinewidth{0.501875pt}%
\definecolor{currentstroke}{rgb}{0.827451,0.827451,0.827451}%
\pgfsetstrokecolor{currentstroke}%
\pgfsetdash{}{0pt}%
\pgfpathmoveto{\pgfqpoint{1.511448in}{0.193516in}}%
\pgfpathlineto{\pgfqpoint{1.515119in}{0.189501in}}%
\pgfpathlineto{\pgfqpoint{1.515928in}{0.181042in}}%
\pgfpathlineto{\pgfqpoint{1.517859in}{0.178583in}}%
\pgfpathlineto{\pgfqpoint{1.514704in}{0.170928in}}%
\pgfpathlineto{\pgfqpoint{1.510553in}{0.166618in}}%
\pgfpathlineto{\pgfqpoint{1.508343in}{0.158621in}}%
\pgfpathlineto{\pgfqpoint{1.503805in}{0.171083in}}%
\pgfpathlineto{\pgfqpoint{1.503053in}{0.182397in}}%
\pgfpathlineto{\pgfqpoint{1.500513in}{0.188029in}}%
\pgfpathlineto{\pgfqpoint{1.491768in}{0.192596in}}%
\pgfpathlineto{\pgfqpoint{1.499972in}{0.202118in}}%
\pgfpathlineto{\pgfqpoint{1.494502in}{0.206188in}}%
\pgfpathlineto{\pgfqpoint{1.499320in}{0.210244in}}%
\pgfpathlineto{\pgfqpoint{1.505717in}{0.225814in}}%
\pgfpathlineto{\pgfqpoint{1.499087in}{0.231304in}}%
\pgfpathlineto{\pgfqpoint{1.501641in}{0.236574in}}%
\pgfpathlineto{\pgfqpoint{1.510210in}{0.231898in}}%
\pgfpathlineto{\pgfqpoint{1.511078in}{0.224520in}}%
\pgfpathlineto{\pgfqpoint{1.507836in}{0.220319in}}%
\pgfpathlineto{\pgfqpoint{1.514261in}{0.214635in}}%
\pgfpathlineto{\pgfqpoint{1.516284in}{0.205781in}}%
\pgfpathlineto{\pgfqpoint{1.516010in}{0.193362in}}%
\pgfpathclose%
\pgfusepath{stroke,fill}%
\end{pgfscope}%
\begin{pgfscope}%
\pgfpathrectangle{\pgfqpoint{0.100000in}{0.100000in}}{\pgfqpoint{5.307240in}{3.397500in}}%
\pgfusepath{clip}%
\pgfsetbuttcap%
\pgfsetmiterjoin%
\definecolor{currentfill}{rgb}{1.000000,1.000000,1.000000}%
\pgfsetfillcolor{currentfill}%
\pgfsetlinewidth{0.501875pt}%
\definecolor{currentstroke}{rgb}{0.827451,0.827451,0.827451}%
\pgfsetstrokecolor{currentstroke}%
\pgfsetdash{}{0pt}%
\pgfpathmoveto{\pgfqpoint{1.527138in}{0.224688in}}%
\pgfpathlineto{\pgfqpoint{1.525140in}{0.220381in}}%
\pgfpathlineto{\pgfqpoint{1.527017in}{0.213386in}}%
\pgfpathlineto{\pgfqpoint{1.522211in}{0.212798in}}%
\pgfpathlineto{\pgfqpoint{1.517173in}{0.216037in}}%
\pgfpathlineto{\pgfqpoint{1.518808in}{0.223170in}}%
\pgfpathclose%
\pgfusepath{stroke,fill}%
\end{pgfscope}%
\begin{pgfscope}%
\pgfpathrectangle{\pgfqpoint{0.100000in}{0.100000in}}{\pgfqpoint{5.307240in}{3.397500in}}%
\pgfusepath{clip}%
\pgfsetbuttcap%
\pgfsetmiterjoin%
\definecolor{currentfill}{rgb}{1.000000,1.000000,1.000000}%
\pgfsetfillcolor{currentfill}%
\pgfsetlinewidth{0.501875pt}%
\definecolor{currentstroke}{rgb}{0.827451,0.827451,0.827451}%
\pgfsetstrokecolor{currentstroke}%
\pgfsetdash{}{0pt}%
\pgfpathmoveto{\pgfqpoint{1.504741in}{0.225176in}}%
\pgfpathlineto{\pgfqpoint{1.494134in}{0.220419in}}%
\pgfpathlineto{\pgfqpoint{1.490528in}{0.222130in}}%
\pgfpathlineto{\pgfqpoint{1.497763in}{0.228407in}}%
\pgfpathclose%
\pgfusepath{stroke,fill}%
\end{pgfscope}%
\begin{pgfscope}%
\pgfpathrectangle{\pgfqpoint{0.100000in}{0.100000in}}{\pgfqpoint{5.307240in}{3.397500in}}%
\pgfusepath{clip}%
\pgfsetbuttcap%
\pgfsetmiterjoin%
\definecolor{currentfill}{rgb}{1.000000,1.000000,1.000000}%
\pgfsetfillcolor{currentfill}%
\pgfsetlinewidth{0.501875pt}%
\definecolor{currentstroke}{rgb}{0.827451,0.827451,0.827451}%
\pgfsetstrokecolor{currentstroke}%
\pgfsetdash{}{0pt}%
\pgfpathmoveto{\pgfqpoint{1.527255in}{0.178765in}}%
\pgfpathlineto{\pgfqpoint{1.524726in}{0.185928in}}%
\pgfpathlineto{\pgfqpoint{1.530075in}{0.187691in}}%
\pgfpathlineto{\pgfqpoint{1.531694in}{0.195416in}}%
\pgfpathlineto{\pgfqpoint{1.537491in}{0.195424in}}%
\pgfpathlineto{\pgfqpoint{1.537466in}{0.200802in}}%
\pgfpathlineto{\pgfqpoint{1.545521in}{0.199531in}}%
\pgfpathlineto{\pgfqpoint{1.546894in}{0.184949in}}%
\pgfpathlineto{\pgfqpoint{1.542222in}{0.175653in}}%
\pgfpathlineto{\pgfqpoint{1.535289in}{0.169846in}}%
\pgfpathclose%
\pgfusepath{stroke,fill}%
\end{pgfscope}%
\begin{pgfscope}%
\pgfpathrectangle{\pgfqpoint{0.100000in}{0.100000in}}{\pgfqpoint{5.307240in}{3.397500in}}%
\pgfusepath{clip}%
\pgfsetbuttcap%
\pgfsetmiterjoin%
\definecolor{currentfill}{rgb}{1.000000,1.000000,1.000000}%
\pgfsetfillcolor{currentfill}%
\pgfsetlinewidth{0.501875pt}%
\definecolor{currentstroke}{rgb}{0.827451,0.827451,0.827451}%
\pgfsetstrokecolor{currentstroke}%
\pgfsetdash{}{0pt}%
\pgfpathmoveto{\pgfqpoint{1.524123in}{0.184275in}}%
\pgfpathlineto{\pgfqpoint{1.526407in}{0.177492in}}%
\pgfpathlineto{\pgfqpoint{1.520775in}{0.176668in}}%
\pgfpathclose%
\pgfusepath{stroke,fill}%
\end{pgfscope}%
\begin{pgfscope}%
\pgfpathrectangle{\pgfqpoint{0.100000in}{0.100000in}}{\pgfqpoint{5.307240in}{3.397500in}}%
\pgfusepath{clip}%
\pgfsetbuttcap%
\pgfsetmiterjoin%
\definecolor{currentfill}{rgb}{1.000000,1.000000,1.000000}%
\pgfsetfillcolor{currentfill}%
\pgfsetlinewidth{0.501875pt}%
\definecolor{currentstroke}{rgb}{0.827451,0.827451,0.827451}%
\pgfsetstrokecolor{currentstroke}%
\pgfsetdash{}{0pt}%
\pgfpathmoveto{\pgfqpoint{1.528872in}{0.174299in}}%
\pgfpathlineto{\pgfqpoint{1.528086in}{0.165653in}}%
\pgfpathlineto{\pgfqpoint{1.522144in}{0.166369in}}%
\pgfpathlineto{\pgfqpoint{1.524066in}{0.170546in}}%
\pgfpathclose%
\pgfusepath{stroke,fill}%
\end{pgfscope}%
\begin{pgfscope}%
\pgfpathrectangle{\pgfqpoint{0.100000in}{0.100000in}}{\pgfqpoint{5.307240in}{3.397500in}}%
\pgfusepath{clip}%
\pgfsetbuttcap%
\pgfsetroundjoin%
\pgfsetlinewidth{0.501875pt}%
\definecolor{currentstroke}{rgb}{0.827451,0.827451,0.827451}%
\pgfsetstrokecolor{currentstroke}%
\pgfsetdash{}{0pt}%
\pgfusepath{stroke}%
\end{pgfscope}%
\begin{pgfscope}%
\pgfpathrectangle{\pgfqpoint{0.100000in}{0.100000in}}{\pgfqpoint{5.307240in}{3.397500in}}%
\pgfusepath{clip}%
\pgfsetbuttcap%
\pgfsetroundjoin%
\pgfsetlinewidth{0.501875pt}%
\definecolor{currentstroke}{rgb}{0.827451,0.827451,0.827451}%
\pgfsetstrokecolor{currentstroke}%
\pgfsetdash{}{0pt}%
\pgfusepath{stroke}%
\end{pgfscope}%
\begin{pgfscope}%
\pgfpathrectangle{\pgfqpoint{0.100000in}{0.100000in}}{\pgfqpoint{5.307240in}{3.397500in}}%
\pgfusepath{clip}%
\pgfsetbuttcap%
\pgfsetroundjoin%
\pgfsetlinewidth{0.501875pt}%
\definecolor{currentstroke}{rgb}{0.827451,0.827451,0.827451}%
\pgfsetstrokecolor{currentstroke}%
\pgfsetdash{}{0pt}%
\pgfusepath{stroke}%
\end{pgfscope}%
\begin{pgfscope}%
\pgfpathrectangle{\pgfqpoint{0.100000in}{0.100000in}}{\pgfqpoint{5.307240in}{3.397500in}}%
\pgfusepath{clip}%
\pgfsetbuttcap%
\pgfsetroundjoin%
\pgfsetlinewidth{0.501875pt}%
\definecolor{currentstroke}{rgb}{0.827451,0.827451,0.827451}%
\pgfsetstrokecolor{currentstroke}%
\pgfsetdash{}{0pt}%
\pgfusepath{stroke}%
\end{pgfscope}%
\begin{pgfscope}%
\pgfpathrectangle{\pgfqpoint{0.100000in}{0.100000in}}{\pgfqpoint{5.307240in}{3.397500in}}%
\pgfusepath{clip}%
\pgfsetbuttcap%
\pgfsetroundjoin%
\pgfsetlinewidth{0.501875pt}%
\definecolor{currentstroke}{rgb}{0.827451,0.827451,0.827451}%
\pgfsetstrokecolor{currentstroke}%
\pgfsetdash{}{0pt}%
\pgfusepath{stroke}%
\end{pgfscope}%
\begin{pgfscope}%
\pgfpathrectangle{\pgfqpoint{0.100000in}{0.100000in}}{\pgfqpoint{5.307240in}{3.397500in}}%
\pgfusepath{clip}%
\pgfsetbuttcap%
\pgfsetroundjoin%
\pgfsetlinewidth{0.501875pt}%
\definecolor{currentstroke}{rgb}{0.827451,0.827451,0.827451}%
\pgfsetstrokecolor{currentstroke}%
\pgfsetdash{}{0pt}%
\pgfusepath{stroke}%
\end{pgfscope}%
\begin{pgfscope}%
\pgfpathrectangle{\pgfqpoint{0.100000in}{0.100000in}}{\pgfqpoint{5.307240in}{3.397500in}}%
\pgfusepath{clip}%
\pgfsetbuttcap%
\pgfsetroundjoin%
\pgfsetlinewidth{0.501875pt}%
\definecolor{currentstroke}{rgb}{0.827451,0.827451,0.827451}%
\pgfsetstrokecolor{currentstroke}%
\pgfsetdash{}{0pt}%
\pgfpathmoveto{\pgfqpoint{1.361052in}{3.297686in}}%
\pgfpathlineto{\pgfqpoint{1.333529in}{3.188632in}}%
\pgfpathlineto{\pgfqpoint{1.313959in}{3.110493in}}%
\pgfpathlineto{\pgfqpoint{1.290691in}{3.016494in}}%
\pgfpathlineto{\pgfqpoint{1.286122in}{2.993325in}}%
\pgfpathlineto{\pgfqpoint{1.289106in}{2.972075in}}%
\pgfpathlineto{\pgfqpoint{1.285152in}{2.952501in}}%
\pgfpathlineto{\pgfqpoint{1.122727in}{2.994885in}}%
\pgfpathlineto{\pgfqpoint{1.108054in}{2.989963in}}%
\pgfpathlineto{\pgfqpoint{1.095498in}{2.994235in}}%
\pgfpathlineto{\pgfqpoint{1.037364in}{2.995631in}}%
\pgfpathlineto{\pgfqpoint{1.017701in}{2.990088in}}%
\pgfpathlineto{\pgfqpoint{0.998176in}{2.991961in}}%
\pgfpathlineto{\pgfqpoint{0.989900in}{3.000638in}}%
\pgfpathlineto{\pgfqpoint{0.937853in}{2.998746in}}%
\pgfpathlineto{\pgfqpoint{0.929633in}{3.012925in}}%
\pgfpathlineto{\pgfqpoint{0.914029in}{3.020396in}}%
\pgfpathlineto{\pgfqpoint{0.891326in}{3.024948in}}%
\pgfpathlineto{\pgfqpoint{0.852139in}{3.018282in}}%
\pgfpathlineto{\pgfqpoint{0.837673in}{3.024813in}}%
\pgfpathlineto{\pgfqpoint{0.815370in}{3.042180in}}%
\pgfpathlineto{\pgfqpoint{0.822073in}{3.076220in}}%
\pgfpathlineto{\pgfqpoint{0.819723in}{3.093604in}}%
\pgfpathlineto{\pgfqpoint{0.802046in}{3.112059in}}%
\pgfpathlineto{\pgfqpoint{0.790724in}{3.110914in}}%
\pgfpathlineto{\pgfqpoint{0.782601in}{3.129581in}}%
\pgfpathlineto{\pgfqpoint{0.763361in}{3.137126in}}%
\pgfpathlineto{\pgfqpoint{0.749375in}{3.136165in}}%
\pgfpathlineto{\pgfqpoint{0.744851in}{3.155337in}}%
\pgfpathlineto{\pgfqpoint{0.758783in}{3.153296in}}%
\pgfpathlineto{\pgfqpoint{0.757677in}{3.179943in}}%
\pgfpathlineto{\pgfqpoint{0.751604in}{3.195766in}}%
\pgfpathlineto{\pgfqpoint{0.754281in}{3.216017in}}%
\pgfpathlineto{\pgfqpoint{0.761571in}{3.231249in}}%
\pgfpathlineto{\pgfqpoint{0.761160in}{3.260153in}}%
\pgfpathlineto{\pgfqpoint{0.757268in}{3.270640in}}%
\pgfpathlineto{\pgfqpoint{0.763989in}{3.304285in}}%
\pgfpathlineto{\pgfqpoint{0.762049in}{3.325969in}}%
\pgfpathlineto{\pgfqpoint{0.755330in}{3.336366in}}%
\pgfpathlineto{\pgfqpoint{0.756355in}{3.370360in}}%
\pgfpathlineto{\pgfqpoint{0.766129in}{3.395360in}}%
\pgfpathlineto{\pgfqpoint{0.776758in}{3.389241in}}%
\pgfpathlineto{\pgfqpoint{0.811955in}{3.352883in}}%
\pgfpathlineto{\pgfqpoint{0.854606in}{3.332986in}}%
\pgfpathlineto{\pgfqpoint{0.876445in}{3.330631in}}%
\pgfpathlineto{\pgfqpoint{0.897019in}{3.322160in}}%
\pgfpathlineto{\pgfqpoint{0.902858in}{3.293655in}}%
\pgfpathlineto{\pgfqpoint{0.884828in}{3.288255in}}%
\pgfpathlineto{\pgfqpoint{0.872880in}{3.265790in}}%
\pgfpathlineto{\pgfqpoint{0.886908in}{3.267038in}}%
\pgfpathlineto{\pgfqpoint{0.892566in}{3.277122in}}%
\pgfpathlineto{\pgfqpoint{0.912350in}{3.289731in}}%
\pgfpathlineto{\pgfqpoint{0.911328in}{3.271005in}}%
\pgfpathlineto{\pgfqpoint{0.898118in}{3.267986in}}%
\pgfpathlineto{\pgfqpoint{0.900311in}{3.243917in}}%
\pgfpathlineto{\pgfqpoint{0.884226in}{3.219806in}}%
\pgfpathlineto{\pgfqpoint{0.873930in}{3.231271in}}%
\pgfpathlineto{\pgfqpoint{0.843335in}{3.225038in}}%
\pgfpathlineto{\pgfqpoint{0.841725in}{3.211011in}}%
\pgfpathlineto{\pgfqpoint{0.852272in}{3.202466in}}%
\pgfpathlineto{\pgfqpoint{0.868363in}{3.201626in}}%
\pgfpathlineto{\pgfqpoint{0.890612in}{3.219886in}}%
\pgfpathlineto{\pgfqpoint{0.908221in}{3.221403in}}%
\pgfpathlineto{\pgfqpoint{0.908896in}{3.241615in}}%
\pgfpathlineto{\pgfqpoint{0.917972in}{3.271302in}}%
\pgfpathlineto{\pgfqpoint{0.939479in}{3.293381in}}%
\pgfpathlineto{\pgfqpoint{0.932308in}{3.310482in}}%
\pgfpathlineto{\pgfqpoint{0.937207in}{3.328894in}}%
\pgfpathlineto{\pgfqpoint{0.927665in}{3.347243in}}%
\pgfpathlineto{\pgfqpoint{0.941263in}{3.370709in}}%
\pgfpathlineto{\pgfqpoint{0.943300in}{3.384826in}}%
\pgfpathlineto{\pgfqpoint{0.931494in}{3.394105in}}%
\pgfpathlineto{\pgfqpoint{0.933459in}{3.417838in}}%
\pgfpathlineto{\pgfqpoint{1.074918in}{3.374860in}}%
\pgfpathlineto{\pgfqpoint{1.225133in}{3.332736in}}%
\pgfpathlineto{\pgfqpoint{1.361052in}{3.297686in}}%
\pgfusepath{stroke}%
\end{pgfscope}%
\begin{pgfscope}%
\pgfpathrectangle{\pgfqpoint{0.100000in}{0.100000in}}{\pgfqpoint{5.307240in}{3.397500in}}%
\pgfusepath{clip}%
\pgfsetbuttcap%
\pgfsetroundjoin%
\pgfsetlinewidth{0.501875pt}%
\definecolor{currentstroke}{rgb}{0.827451,0.827451,0.827451}%
\pgfsetstrokecolor{currentstroke}%
\pgfsetdash{}{0pt}%
\pgfpathmoveto{\pgfqpoint{0.910189in}{3.335732in}}%
\pgfpathlineto{\pgfqpoint{0.917155in}{3.308879in}}%
\pgfpathlineto{\pgfqpoint{0.928078in}{3.299958in}}%
\pgfpathlineto{\pgfqpoint{0.919362in}{3.288603in}}%
\pgfpathlineto{\pgfqpoint{0.910867in}{3.305242in}}%
\pgfpathlineto{\pgfqpoint{0.910189in}{3.335732in}}%
\pgfusepath{stroke}%
\end{pgfscope}%
\begin{pgfscope}%
\pgfpathrectangle{\pgfqpoint{0.100000in}{0.100000in}}{\pgfqpoint{5.307240in}{3.397500in}}%
\pgfusepath{clip}%
\pgfsetbuttcap%
\pgfsetroundjoin%
\pgfsetlinewidth{0.501875pt}%
\definecolor{currentstroke}{rgb}{0.827451,0.827451,0.827451}%
\pgfsetstrokecolor{currentstroke}%
\pgfsetdash{}{0pt}%
\pgfpathmoveto{\pgfqpoint{1.433998in}{3.280061in}}%
\pgfpathlineto{\pgfqpoint{1.585209in}{3.246127in}}%
\pgfpathlineto{\pgfqpoint{1.727616in}{3.217342in}}%
\pgfpathlineto{\pgfqpoint{1.837182in}{3.197266in}}%
\pgfpathlineto{\pgfqpoint{1.932703in}{3.181219in}}%
\pgfpathlineto{\pgfqpoint{2.028436in}{3.166488in}}%
\pgfpathlineto{\pgfqpoint{2.109963in}{3.155003in}}%
\pgfpathlineto{\pgfqpoint{2.191617in}{3.144472in}}%
\pgfpathlineto{\pgfqpoint{2.273388in}{3.134895in}}%
\pgfpathlineto{\pgfqpoint{2.350447in}{3.126757in}}%
\pgfpathlineto{\pgfqpoint{2.339850in}{3.009317in}}%
\pgfpathlineto{\pgfqpoint{2.324075in}{2.850613in}}%
\pgfpathlineto{\pgfqpoint{2.315802in}{2.768988in}}%
\pgfpathlineto{\pgfqpoint{2.303904in}{2.658984in}}%
\pgfpathlineto{\pgfqpoint{2.219669in}{2.668182in}}%
\pgfpathlineto{\pgfqpoint{2.123230in}{2.679129in}}%
\pgfpathlineto{\pgfqpoint{1.989324in}{2.697455in}}%
\pgfpathlineto{\pgfqpoint{1.929517in}{2.705987in}}%
\pgfpathlineto{\pgfqpoint{1.782171in}{2.729116in}}%
\pgfpathlineto{\pgfqpoint{1.731448in}{2.738395in}}%
\pgfpathlineto{\pgfqpoint{1.720877in}{2.678287in}}%
\pgfpathlineto{\pgfqpoint{1.715097in}{2.682576in}}%
\pgfpathlineto{\pgfqpoint{1.704235in}{2.711492in}}%
\pgfpathlineto{\pgfqpoint{1.690273in}{2.709711in}}%
\pgfpathlineto{\pgfqpoint{1.680280in}{2.695093in}}%
\pgfpathlineto{\pgfqpoint{1.659541in}{2.693138in}}%
\pgfpathlineto{\pgfqpoint{1.655374in}{2.700132in}}%
\pgfpathlineto{\pgfqpoint{1.635948in}{2.699285in}}%
\pgfpathlineto{\pgfqpoint{1.626182in}{2.706058in}}%
\pgfpathlineto{\pgfqpoint{1.612527in}{2.695580in}}%
\pgfpathlineto{\pgfqpoint{1.579359in}{2.705009in}}%
\pgfpathlineto{\pgfqpoint{1.564884in}{2.699945in}}%
\pgfpathlineto{\pgfqpoint{1.557096in}{2.722413in}}%
\pgfpathlineto{\pgfqpoint{1.556732in}{2.743790in}}%
\pgfpathlineto{\pgfqpoint{1.533868in}{2.759165in}}%
\pgfpathlineto{\pgfqpoint{1.537597in}{2.781322in}}%
\pgfpathlineto{\pgfqpoint{1.521317in}{2.817922in}}%
\pgfpathlineto{\pgfqpoint{1.522413in}{2.849034in}}%
\pgfpathlineto{\pgfqpoint{1.508422in}{2.864261in}}%
\pgfpathlineto{\pgfqpoint{1.495293in}{2.850790in}}%
\pgfpathlineto{\pgfqpoint{1.477424in}{2.843503in}}%
\pgfpathlineto{\pgfqpoint{1.463822in}{2.858522in}}%
\pgfpathlineto{\pgfqpoint{1.466034in}{2.882503in}}%
\pgfpathlineto{\pgfqpoint{1.482346in}{2.891952in}}%
\pgfpathlineto{\pgfqpoint{1.478010in}{2.907146in}}%
\pgfpathlineto{\pgfqpoint{1.506367in}{2.980689in}}%
\pgfpathlineto{\pgfqpoint{1.484007in}{2.982532in}}%
\pgfpathlineto{\pgfqpoint{1.481709in}{2.995730in}}%
\pgfpathlineto{\pgfqpoint{1.465421in}{3.007111in}}%
\pgfpathlineto{\pgfqpoint{1.466462in}{3.019833in}}%
\pgfpathlineto{\pgfqpoint{1.457870in}{3.029717in}}%
\pgfpathlineto{\pgfqpoint{1.443884in}{3.065602in}}%
\pgfpathlineto{\pgfqpoint{1.431104in}{3.072920in}}%
\pgfpathlineto{\pgfqpoint{1.421952in}{3.103891in}}%
\pgfpathlineto{\pgfqpoint{1.424112in}{3.124477in}}%
\pgfpathlineto{\pgfqpoint{1.407038in}{3.162391in}}%
\pgfpathlineto{\pgfqpoint{1.433998in}{3.280061in}}%
\pgfusepath{stroke}%
\end{pgfscope}%
\begin{pgfscope}%
\pgfpathrectangle{\pgfqpoint{0.100000in}{0.100000in}}{\pgfqpoint{5.307240in}{3.397500in}}%
\pgfusepath{clip}%
\pgfsetbuttcap%
\pgfsetroundjoin%
\pgfsetlinewidth{0.501875pt}%
\definecolor{currentstroke}{rgb}{0.827451,0.827451,0.827451}%
\pgfsetstrokecolor{currentstroke}%
\pgfsetdash{}{0pt}%
\pgfpathmoveto{\pgfqpoint{5.084347in}{2.674872in}}%
\pgfpathlineto{\pgfqpoint{5.081321in}{2.687740in}}%
\pgfpathlineto{\pgfqpoint{5.064500in}{2.699014in}}%
\pgfpathlineto{\pgfqpoint{5.020052in}{2.844408in}}%
\pgfpathlineto{\pgfqpoint{4.996094in}{2.914571in}}%
\pgfpathlineto{\pgfqpoint{5.015354in}{2.934785in}}%
\pgfpathlineto{\pgfqpoint{5.034078in}{2.984255in}}%
\pgfpathlineto{\pgfqpoint{5.043415in}{3.000064in}}%
\pgfpathlineto{\pgfqpoint{5.036800in}{3.006703in}}%
\pgfpathlineto{\pgfqpoint{5.034585in}{3.050617in}}%
\pgfpathlineto{\pgfqpoint{5.042939in}{3.064078in}}%
\pgfpathlineto{\pgfqpoint{5.039277in}{3.095369in}}%
\pgfpathlineto{\pgfqpoint{5.072619in}{3.198028in}}%
\pgfpathlineto{\pgfqpoint{5.087641in}{3.198589in}}%
\pgfpathlineto{\pgfqpoint{5.093878in}{3.180251in}}%
\pgfpathlineto{\pgfqpoint{5.107228in}{3.174986in}}%
\pgfpathlineto{\pgfqpoint{5.132449in}{3.196555in}}%
\pgfpathlineto{\pgfqpoint{5.152224in}{3.209295in}}%
\pgfpathlineto{\pgfqpoint{5.195796in}{3.186861in}}%
\pgfpathlineto{\pgfqpoint{5.235161in}{3.062297in}}%
\pgfpathlineto{\pgfqpoint{5.242644in}{3.031651in}}%
\pgfpathlineto{\pgfqpoint{5.259886in}{3.028057in}}%
\pgfpathlineto{\pgfqpoint{5.283271in}{3.007243in}}%
\pgfpathlineto{\pgfqpoint{5.281947in}{2.995116in}}%
\pgfpathlineto{\pgfqpoint{5.297877in}{2.980742in}}%
\pgfpathlineto{\pgfqpoint{5.310817in}{2.988989in}}%
\pgfpathlineto{\pgfqpoint{5.337896in}{2.957408in}}%
\pgfpathlineto{\pgfqpoint{5.325794in}{2.932134in}}%
\pgfpathlineto{\pgfqpoint{5.309565in}{2.931634in}}%
\pgfpathlineto{\pgfqpoint{5.296545in}{2.909073in}}%
\pgfpathlineto{\pgfqpoint{5.280773in}{2.905873in}}%
\pgfpathlineto{\pgfqpoint{5.269194in}{2.893860in}}%
\pgfpathlineto{\pgfqpoint{5.234609in}{2.880977in}}%
\pgfpathlineto{\pgfqpoint{5.213677in}{2.860214in}}%
\pgfpathlineto{\pgfqpoint{5.203045in}{2.875117in}}%
\pgfpathlineto{\pgfqpoint{5.193382in}{2.864436in}}%
\pgfpathlineto{\pgfqpoint{5.196028in}{2.821257in}}%
\pgfpathlineto{\pgfqpoint{5.188378in}{2.804154in}}%
\pgfpathlineto{\pgfqpoint{5.171691in}{2.808825in}}%
\pgfpathlineto{\pgfqpoint{5.168998in}{2.791290in}}%
\pgfpathlineto{\pgfqpoint{5.161787in}{2.784080in}}%
\pgfpathlineto{\pgfqpoint{5.147356in}{2.785837in}}%
\pgfpathlineto{\pgfqpoint{5.148266in}{2.769443in}}%
\pgfpathlineto{\pgfqpoint{5.126393in}{2.774099in}}%
\pgfpathlineto{\pgfqpoint{5.114447in}{2.751385in}}%
\pgfpathlineto{\pgfqpoint{5.118958in}{2.739500in}}%
\pgfpathlineto{\pgfqpoint{5.111853in}{2.719732in}}%
\pgfpathlineto{\pgfqpoint{5.100667in}{2.705193in}}%
\pgfpathlineto{\pgfqpoint{5.097852in}{2.674841in}}%
\pgfpathlineto{\pgfqpoint{5.084347in}{2.674872in}}%
\pgfusepath{stroke}%
\end{pgfscope}%
\begin{pgfscope}%
\pgfpathrectangle{\pgfqpoint{0.100000in}{0.100000in}}{\pgfqpoint{5.307240in}{3.397500in}}%
\pgfusepath{clip}%
\pgfsetbuttcap%
\pgfsetroundjoin%
\pgfsetlinewidth{0.501875pt}%
\definecolor{currentstroke}{rgb}{0.827451,0.827451,0.827451}%
\pgfsetstrokecolor{currentstroke}%
\pgfsetdash{}{0pt}%
\pgfpathmoveto{\pgfqpoint{5.240821in}{2.872110in}}%
\pgfpathlineto{\pgfqpoint{5.250697in}{2.882453in}}%
\pgfpathlineto{\pgfqpoint{5.260098in}{2.872731in}}%
\pgfpathlineto{\pgfqpoint{5.243211in}{2.859850in}}%
\pgfpathlineto{\pgfqpoint{5.240821in}{2.872110in}}%
\pgfusepath{stroke}%
\end{pgfscope}%
\begin{pgfscope}%
\pgfpathrectangle{\pgfqpoint{0.100000in}{0.100000in}}{\pgfqpoint{5.307240in}{3.397500in}}%
\pgfusepath{clip}%
\pgfsetbuttcap%
\pgfsetroundjoin%
\pgfsetlinewidth{0.501875pt}%
\definecolor{currentstroke}{rgb}{0.827451,0.827451,0.827451}%
\pgfsetstrokecolor{currentstroke}%
\pgfsetdash{}{0pt}%
\pgfpathmoveto{\pgfqpoint{2.315802in}{2.768988in}}%
\pgfpathlineto{\pgfqpoint{2.324075in}{2.850613in}}%
\pgfpathlineto{\pgfqpoint{2.339850in}{3.009317in}}%
\pgfpathlineto{\pgfqpoint{2.350447in}{3.126757in}}%
\pgfpathlineto{\pgfqpoint{2.437239in}{3.118615in}}%
\pgfpathlineto{\pgfqpoint{2.548272in}{3.109775in}}%
\pgfpathlineto{\pgfqpoint{2.649762in}{3.103237in}}%
\pgfpathlineto{\pgfqpoint{2.741657in}{3.098588in}}%
\pgfpathlineto{\pgfqpoint{2.878797in}{3.093884in}}%
\pgfpathlineto{\pgfqpoint{2.888156in}{3.055232in}}%
\pgfpathlineto{\pgfqpoint{2.884806in}{3.036761in}}%
\pgfpathlineto{\pgfqpoint{2.883703in}{2.998794in}}%
\pgfpathlineto{\pgfqpoint{2.890071in}{2.970360in}}%
\pgfpathlineto{\pgfqpoint{2.904659in}{2.928412in}}%
\pgfpathlineto{\pgfqpoint{2.904625in}{2.875599in}}%
\pgfpathlineto{\pgfqpoint{2.907325in}{2.814246in}}%
\pgfpathlineto{\pgfqpoint{2.911031in}{2.797711in}}%
\pgfpathlineto{\pgfqpoint{2.921873in}{2.779558in}}%
\pgfpathlineto{\pgfqpoint{2.925449in}{2.751273in}}%
\pgfpathlineto{\pgfqpoint{2.923906in}{2.732389in}}%
\pgfpathlineto{\pgfqpoint{2.808937in}{2.734880in}}%
\pgfpathlineto{\pgfqpoint{2.725299in}{2.739119in}}%
\pgfpathlineto{\pgfqpoint{2.602691in}{2.745646in}}%
\pgfpathlineto{\pgfqpoint{2.481781in}{2.754234in}}%
\pgfpathlineto{\pgfqpoint{2.401249in}{2.760731in}}%
\pgfpathlineto{\pgfqpoint{2.315802in}{2.768988in}}%
\pgfusepath{stroke}%
\end{pgfscope}%
\begin{pgfscope}%
\pgfpathrectangle{\pgfqpoint{0.100000in}{0.100000in}}{\pgfqpoint{5.307240in}{3.397500in}}%
\pgfusepath{clip}%
\pgfsetbuttcap%
\pgfsetroundjoin%
\pgfsetlinewidth{0.501875pt}%
\definecolor{currentstroke}{rgb}{0.827451,0.827451,0.827451}%
\pgfsetstrokecolor{currentstroke}%
\pgfsetdash{}{0pt}%
\pgfpathmoveto{\pgfqpoint{2.281050in}{2.427386in}}%
\pgfpathlineto{\pgfqpoint{2.294323in}{2.564212in}}%
\pgfpathlineto{\pgfqpoint{2.303904in}{2.658984in}}%
\pgfpathlineto{\pgfqpoint{2.315802in}{2.768988in}}%
\pgfpathlineto{\pgfqpoint{2.401249in}{2.760731in}}%
\pgfpathlineto{\pgfqpoint{2.481781in}{2.754234in}}%
\pgfpathlineto{\pgfqpoint{2.602691in}{2.745646in}}%
\pgfpathlineto{\pgfqpoint{2.725299in}{2.739119in}}%
\pgfpathlineto{\pgfqpoint{2.808937in}{2.734880in}}%
\pgfpathlineto{\pgfqpoint{2.923906in}{2.732389in}}%
\pgfpathlineto{\pgfqpoint{2.916121in}{2.709674in}}%
\pgfpathlineto{\pgfqpoint{2.900584in}{2.691842in}}%
\pgfpathlineto{\pgfqpoint{2.912483in}{2.671312in}}%
\pgfpathlineto{\pgfqpoint{2.925602in}{2.666925in}}%
\pgfpathlineto{\pgfqpoint{2.931827in}{2.655132in}}%
\pgfpathlineto{\pgfqpoint{2.930398in}{2.569027in}}%
\pgfpathlineto{\pgfqpoint{2.928005in}{2.447849in}}%
\pgfpathlineto{\pgfqpoint{2.919157in}{2.416021in}}%
\pgfpathlineto{\pgfqpoint{2.927071in}{2.398416in}}%
\pgfpathlineto{\pgfqpoint{2.911181in}{2.360584in}}%
\pgfpathlineto{\pgfqpoint{2.927922in}{2.330086in}}%
\pgfpathlineto{\pgfqpoint{2.913699in}{2.332421in}}%
\pgfpathlineto{\pgfqpoint{2.903995in}{2.351395in}}%
\pgfpathlineto{\pgfqpoint{2.869363in}{2.364402in}}%
\pgfpathlineto{\pgfqpoint{2.847523in}{2.375842in}}%
\pgfpathlineto{\pgfqpoint{2.810873in}{2.376794in}}%
\pgfpathlineto{\pgfqpoint{2.798151in}{2.366371in}}%
\pgfpathlineto{\pgfqpoint{2.756669in}{2.386894in}}%
\pgfpathlineto{\pgfqpoint{2.753486in}{2.393385in}}%
\pgfpathlineto{\pgfqpoint{2.608712in}{2.400237in}}%
\pgfpathlineto{\pgfqpoint{2.520756in}{2.405279in}}%
\pgfpathlineto{\pgfqpoint{2.448103in}{2.410968in}}%
\pgfpathlineto{\pgfqpoint{2.328055in}{2.422311in}}%
\pgfpathlineto{\pgfqpoint{2.281050in}{2.427386in}}%
\pgfusepath{stroke}%
\end{pgfscope}%
\begin{pgfscope}%
\pgfpathrectangle{\pgfqpoint{0.100000in}{0.100000in}}{\pgfqpoint{5.307240in}{3.397500in}}%
\pgfusepath{clip}%
\pgfsetbuttcap%
\pgfsetroundjoin%
\pgfsetlinewidth{0.501875pt}%
\definecolor{currentstroke}{rgb}{0.827451,0.827451,0.827451}%
\pgfsetstrokecolor{currentstroke}%
\pgfsetdash{}{0pt}%
\pgfpathmoveto{\pgfqpoint{2.258281in}{2.195701in}}%
\pgfpathlineto{\pgfqpoint{2.181100in}{2.202750in}}%
\pgfpathlineto{\pgfqpoint{2.012863in}{2.223514in}}%
\pgfpathlineto{\pgfqpoint{1.921344in}{2.236649in}}%
\pgfpathlineto{\pgfqpoint{1.823186in}{2.250740in}}%
\pgfpathlineto{\pgfqpoint{1.740503in}{2.263971in}}%
\pgfpathlineto{\pgfqpoint{1.649747in}{2.279509in}}%
\pgfpathlineto{\pgfqpoint{1.670381in}{2.393976in}}%
\pgfpathlineto{\pgfqpoint{1.691283in}{2.511349in}}%
\pgfpathlineto{\pgfqpoint{1.720877in}{2.678287in}}%
\pgfpathlineto{\pgfqpoint{1.731448in}{2.738395in}}%
\pgfpathlineto{\pgfqpoint{1.782171in}{2.729116in}}%
\pgfpathlineto{\pgfqpoint{1.929517in}{2.705987in}}%
\pgfpathlineto{\pgfqpoint{1.989324in}{2.697455in}}%
\pgfpathlineto{\pgfqpoint{2.123230in}{2.679129in}}%
\pgfpathlineto{\pgfqpoint{2.219669in}{2.668182in}}%
\pgfpathlineto{\pgfqpoint{2.303904in}{2.658984in}}%
\pgfpathlineto{\pgfqpoint{2.294323in}{2.564212in}}%
\pgfpathlineto{\pgfqpoint{2.281050in}{2.427386in}}%
\pgfpathlineto{\pgfqpoint{2.269657in}{2.311103in}}%
\pgfpathlineto{\pgfqpoint{2.258281in}{2.195701in}}%
\pgfusepath{stroke}%
\end{pgfscope}%
\begin{pgfscope}%
\pgfpathrectangle{\pgfqpoint{0.100000in}{0.100000in}}{\pgfqpoint{5.307240in}{3.397500in}}%
\pgfusepath{clip}%
\pgfsetbuttcap%
\pgfsetroundjoin%
\pgfsetlinewidth{0.501875pt}%
\definecolor{currentstroke}{rgb}{0.827451,0.827451,0.827451}%
\pgfsetstrokecolor{currentstroke}%
\pgfsetdash{}{0pt}%
\pgfpathmoveto{\pgfqpoint{3.669420in}{2.353596in}}%
\pgfpathlineto{\pgfqpoint{3.571613in}{2.346648in}}%
\pgfpathlineto{\pgfqpoint{3.425827in}{2.340435in}}%
\pgfpathlineto{\pgfqpoint{3.420287in}{2.355161in}}%
\pgfpathlineto{\pgfqpoint{3.387959in}{2.366170in}}%
\pgfpathlineto{\pgfqpoint{3.380828in}{2.386968in}}%
\pgfpathlineto{\pgfqpoint{3.377842in}{2.412698in}}%
\pgfpathlineto{\pgfqpoint{3.385127in}{2.425880in}}%
\pgfpathlineto{\pgfqpoint{3.373621in}{2.438530in}}%
\pgfpathlineto{\pgfqpoint{3.370846in}{2.453619in}}%
\pgfpathlineto{\pgfqpoint{3.367137in}{2.486997in}}%
\pgfpathlineto{\pgfqpoint{3.356109in}{2.505128in}}%
\pgfpathlineto{\pgfqpoint{3.336526in}{2.515304in}}%
\pgfpathlineto{\pgfqpoint{3.315214in}{2.532097in}}%
\pgfpathlineto{\pgfqpoint{3.304198in}{2.551965in}}%
\pgfpathlineto{\pgfqpoint{3.284432in}{2.559966in}}%
\pgfpathlineto{\pgfqpoint{3.272812in}{2.572988in}}%
\pgfpathlineto{\pgfqpoint{3.258735in}{2.575197in}}%
\pgfpathlineto{\pgfqpoint{3.233603in}{2.594532in}}%
\pgfpathlineto{\pgfqpoint{3.237685in}{2.616782in}}%
\pgfpathlineto{\pgfqpoint{3.236898in}{2.659093in}}%
\pgfpathlineto{\pgfqpoint{3.244649in}{2.670740in}}%
\pgfpathlineto{\pgfqpoint{3.237685in}{2.688327in}}%
\pgfpathlineto{\pgfqpoint{3.225418in}{2.691728in}}%
\pgfpathlineto{\pgfqpoint{3.226432in}{2.707174in}}%
\pgfpathlineto{\pgfqpoint{3.241652in}{2.731574in}}%
\pgfpathlineto{\pgfqpoint{3.271793in}{2.750869in}}%
\pgfpathlineto{\pgfqpoint{3.269915in}{2.819489in}}%
\pgfpathlineto{\pgfqpoint{3.284996in}{2.829796in}}%
\pgfpathlineto{\pgfqpoint{3.299269in}{2.822930in}}%
\pgfpathlineto{\pgfqpoint{3.328343in}{2.832982in}}%
\pgfpathlineto{\pgfqpoint{3.383043in}{2.858245in}}%
\pgfpathlineto{\pgfqpoint{3.390168in}{2.850423in}}%
\pgfpathlineto{\pgfqpoint{3.379808in}{2.814992in}}%
\pgfpathlineto{\pgfqpoint{3.395222in}{2.822768in}}%
\pgfpathlineto{\pgfqpoint{3.421621in}{2.815004in}}%
\pgfpathlineto{\pgfqpoint{3.437843in}{2.808524in}}%
\pgfpathlineto{\pgfqpoint{3.446938in}{2.789533in}}%
\pgfpathlineto{\pgfqpoint{3.530161in}{2.771611in}}%
\pgfpathlineto{\pgfqpoint{3.555037in}{2.759316in}}%
\pgfpathlineto{\pgfqpoint{3.580330in}{2.759461in}}%
\pgfpathlineto{\pgfqpoint{3.606318in}{2.754388in}}%
\pgfpathlineto{\pgfqpoint{3.623185in}{2.737063in}}%
\pgfpathlineto{\pgfqpoint{3.639301in}{2.728503in}}%
\pgfpathlineto{\pgfqpoint{3.642262in}{2.703811in}}%
\pgfpathlineto{\pgfqpoint{3.637501in}{2.688307in}}%
\pgfpathlineto{\pgfqpoint{3.655463in}{2.687161in}}%
\pgfpathlineto{\pgfqpoint{3.649412in}{2.669179in}}%
\pgfpathlineto{\pgfqpoint{3.655168in}{2.662792in}}%
\pgfpathlineto{\pgfqpoint{3.660892in}{2.645818in}}%
\pgfpathlineto{\pgfqpoint{3.643387in}{2.636831in}}%
\pgfpathlineto{\pgfqpoint{3.633223in}{2.611793in}}%
\pgfpathlineto{\pgfqpoint{3.630034in}{2.594087in}}%
\pgfpathlineto{\pgfqpoint{3.639785in}{2.591056in}}%
\pgfpathlineto{\pgfqpoint{3.652270in}{2.604333in}}%
\pgfpathlineto{\pgfqpoint{3.662884in}{2.627342in}}%
\pgfpathlineto{\pgfqpoint{3.677245in}{2.635333in}}%
\pgfpathlineto{\pgfqpoint{3.688037in}{2.624938in}}%
\pgfpathlineto{\pgfqpoint{3.677414in}{2.593477in}}%
\pgfpathlineto{\pgfqpoint{3.674081in}{2.569054in}}%
\pgfpathlineto{\pgfqpoint{3.677220in}{2.551497in}}%
\pgfpathlineto{\pgfqpoint{3.667355in}{2.541565in}}%
\pgfpathlineto{\pgfqpoint{3.662428in}{2.517294in}}%
\pgfpathlineto{\pgfqpoint{3.666454in}{2.491786in}}%
\pgfpathlineto{\pgfqpoint{3.654812in}{2.454063in}}%
\pgfpathlineto{\pgfqpoint{3.655057in}{2.435213in}}%
\pgfpathlineto{\pgfqpoint{3.664256in}{2.394365in}}%
\pgfpathlineto{\pgfqpoint{3.670219in}{2.387357in}}%
\pgfpathlineto{\pgfqpoint{3.669420in}{2.353596in}}%
\pgfusepath{stroke}%
\end{pgfscope}%
\begin{pgfscope}%
\pgfpathrectangle{\pgfqpoint{0.100000in}{0.100000in}}{\pgfqpoint{5.307240in}{3.397500in}}%
\pgfusepath{clip}%
\pgfsetbuttcap%
\pgfsetroundjoin%
\pgfsetlinewidth{0.501875pt}%
\definecolor{currentstroke}{rgb}{0.827451,0.827451,0.827451}%
\pgfsetstrokecolor{currentstroke}%
\pgfsetdash{}{0pt}%
\pgfpathmoveto{\pgfqpoint{3.706091in}{2.684658in}}%
\pgfpathlineto{\pgfqpoint{3.708274in}{2.668365in}}%
\pgfpathlineto{\pgfqpoint{3.688273in}{2.625434in}}%
\pgfpathlineto{\pgfqpoint{3.679386in}{2.637869in}}%
\pgfpathlineto{\pgfqpoint{3.706091in}{2.684658in}}%
\pgfusepath{stroke}%
\end{pgfscope}%
\begin{pgfscope}%
\pgfpathrectangle{\pgfqpoint{0.100000in}{0.100000in}}{\pgfqpoint{5.307240in}{3.397500in}}%
\pgfusepath{clip}%
\pgfsetbuttcap%
\pgfsetroundjoin%
\pgfsetlinewidth{0.501875pt}%
\definecolor{currentstroke}{rgb}{0.827451,0.827451,0.827451}%
\pgfsetstrokecolor{currentstroke}%
\pgfsetdash{}{0pt}%
\pgfpathmoveto{\pgfqpoint{1.285152in}{2.952501in}}%
\pgfpathlineto{\pgfqpoint{1.289106in}{2.972075in}}%
\pgfpathlineto{\pgfqpoint{1.286122in}{2.993325in}}%
\pgfpathlineto{\pgfqpoint{1.290691in}{3.016494in}}%
\pgfpathlineto{\pgfqpoint{1.313959in}{3.110493in}}%
\pgfpathlineto{\pgfqpoint{1.333529in}{3.188632in}}%
\pgfpathlineto{\pgfqpoint{1.361052in}{3.297686in}}%
\pgfpathlineto{\pgfqpoint{1.433998in}{3.280061in}}%
\pgfpathlineto{\pgfqpoint{1.407038in}{3.162391in}}%
\pgfpathlineto{\pgfqpoint{1.424112in}{3.124477in}}%
\pgfpathlineto{\pgfqpoint{1.421952in}{3.103891in}}%
\pgfpathlineto{\pgfqpoint{1.431104in}{3.072920in}}%
\pgfpathlineto{\pgfqpoint{1.443884in}{3.065602in}}%
\pgfpathlineto{\pgfqpoint{1.457870in}{3.029717in}}%
\pgfpathlineto{\pgfqpoint{1.466462in}{3.019833in}}%
\pgfpathlineto{\pgfqpoint{1.465421in}{3.007111in}}%
\pgfpathlineto{\pgfqpoint{1.481709in}{2.995730in}}%
\pgfpathlineto{\pgfqpoint{1.484007in}{2.982532in}}%
\pgfpathlineto{\pgfqpoint{1.506367in}{2.980689in}}%
\pgfpathlineto{\pgfqpoint{1.478010in}{2.907146in}}%
\pgfpathlineto{\pgfqpoint{1.482346in}{2.891952in}}%
\pgfpathlineto{\pgfqpoint{1.466034in}{2.882503in}}%
\pgfpathlineto{\pgfqpoint{1.463822in}{2.858522in}}%
\pgfpathlineto{\pgfqpoint{1.477424in}{2.843503in}}%
\pgfpathlineto{\pgfqpoint{1.495293in}{2.850790in}}%
\pgfpathlineto{\pgfqpoint{1.508422in}{2.864261in}}%
\pgfpathlineto{\pgfqpoint{1.522413in}{2.849034in}}%
\pgfpathlineto{\pgfqpoint{1.521317in}{2.817922in}}%
\pgfpathlineto{\pgfqpoint{1.537597in}{2.781322in}}%
\pgfpathlineto{\pgfqpoint{1.533868in}{2.759165in}}%
\pgfpathlineto{\pgfqpoint{1.556732in}{2.743790in}}%
\pgfpathlineto{\pgfqpoint{1.557096in}{2.722413in}}%
\pgfpathlineto{\pgfqpoint{1.564884in}{2.699945in}}%
\pgfpathlineto{\pgfqpoint{1.579359in}{2.705009in}}%
\pgfpathlineto{\pgfqpoint{1.612527in}{2.695580in}}%
\pgfpathlineto{\pgfqpoint{1.626182in}{2.706058in}}%
\pgfpathlineto{\pgfqpoint{1.635948in}{2.699285in}}%
\pgfpathlineto{\pgfqpoint{1.655374in}{2.700132in}}%
\pgfpathlineto{\pgfqpoint{1.659541in}{2.693138in}}%
\pgfpathlineto{\pgfqpoint{1.680280in}{2.695093in}}%
\pgfpathlineto{\pgfqpoint{1.690273in}{2.709711in}}%
\pgfpathlineto{\pgfqpoint{1.704235in}{2.711492in}}%
\pgfpathlineto{\pgfqpoint{1.715097in}{2.682576in}}%
\pgfpathlineto{\pgfqpoint{1.720877in}{2.678287in}}%
\pgfpathlineto{\pgfqpoint{1.691283in}{2.511349in}}%
\pgfpathlineto{\pgfqpoint{1.670381in}{2.393976in}}%
\pgfpathlineto{\pgfqpoint{1.505531in}{2.425800in}}%
\pgfpathlineto{\pgfqpoint{1.416486in}{2.443512in}}%
\pgfpathlineto{\pgfqpoint{1.333196in}{2.461827in}}%
\pgfpathlineto{\pgfqpoint{1.255346in}{2.479453in}}%
\pgfpathlineto{\pgfqpoint{1.165249in}{2.501181in}}%
\pgfpathlineto{\pgfqpoint{1.211721in}{2.691827in}}%
\pgfpathlineto{\pgfqpoint{1.214227in}{2.705760in}}%
\pgfpathlineto{\pgfqpoint{1.234893in}{2.742266in}}%
\pgfpathlineto{\pgfqpoint{1.213494in}{2.764209in}}%
\pgfpathlineto{\pgfqpoint{1.217938in}{2.785756in}}%
\pgfpathlineto{\pgfqpoint{1.226083in}{2.791149in}}%
\pgfpathlineto{\pgfqpoint{1.240632in}{2.813352in}}%
\pgfpathlineto{\pgfqpoint{1.261838in}{2.828713in}}%
\pgfpathlineto{\pgfqpoint{1.262996in}{2.840079in}}%
\pgfpathlineto{\pgfqpoint{1.275796in}{2.851435in}}%
\pgfpathlineto{\pgfqpoint{1.286423in}{2.872691in}}%
\pgfpathlineto{\pgfqpoint{1.308222in}{2.895179in}}%
\pgfpathlineto{\pgfqpoint{1.306675in}{2.917390in}}%
\pgfpathlineto{\pgfqpoint{1.291770in}{2.929728in}}%
\pgfpathlineto{\pgfqpoint{1.285152in}{2.952501in}}%
\pgfusepath{stroke}%
\end{pgfscope}%
\begin{pgfscope}%
\pgfpathrectangle{\pgfqpoint{0.100000in}{0.100000in}}{\pgfqpoint{5.307240in}{3.397500in}}%
\pgfusepath{clip}%
\pgfsetbuttcap%
\pgfsetroundjoin%
\pgfsetlinewidth{0.501875pt}%
\definecolor{currentstroke}{rgb}{0.827451,0.827451,0.827451}%
\pgfsetstrokecolor{currentstroke}%
\pgfsetdash{}{0pt}%
\pgfpathmoveto{\pgfqpoint{4.893809in}{2.580457in}}%
\pgfpathlineto{\pgfqpoint{4.886960in}{2.602080in}}%
\pgfpathlineto{\pgfqpoint{4.874248in}{2.667671in}}%
\pgfpathlineto{\pgfqpoint{4.857815in}{2.687851in}}%
\pgfpathlineto{\pgfqpoint{4.843399in}{2.724140in}}%
\pgfpathlineto{\pgfqpoint{4.848132in}{2.751046in}}%
\pgfpathlineto{\pgfqpoint{4.844337in}{2.770876in}}%
\pgfpathlineto{\pgfqpoint{4.832020in}{2.790353in}}%
\pgfpathlineto{\pgfqpoint{4.831514in}{2.811808in}}%
\pgfpathlineto{\pgfqpoint{4.824362in}{2.834948in}}%
\pgfpathlineto{\pgfqpoint{4.888362in}{2.850699in}}%
\pgfpathlineto{\pgfqpoint{4.971493in}{2.873137in}}%
\pgfpathlineto{\pgfqpoint{4.974811in}{2.860271in}}%
\pgfpathlineto{\pgfqpoint{4.969518in}{2.839786in}}%
\pgfpathlineto{\pgfqpoint{4.981944in}{2.823396in}}%
\pgfpathlineto{\pgfqpoint{4.975344in}{2.802628in}}%
\pgfpathlineto{\pgfqpoint{4.949119in}{2.776520in}}%
\pgfpathlineto{\pgfqpoint{4.956328in}{2.756909in}}%
\pgfpathlineto{\pgfqpoint{4.950985in}{2.715457in}}%
\pgfpathlineto{\pgfqpoint{4.943506in}{2.692085in}}%
\pgfpathlineto{\pgfqpoint{4.953048in}{2.625623in}}%
\pgfpathlineto{\pgfqpoint{4.951762in}{2.602239in}}%
\pgfpathlineto{\pgfqpoint{4.960994in}{2.594734in}}%
\pgfpathlineto{\pgfqpoint{4.893809in}{2.580457in}}%
\pgfusepath{stroke}%
\end{pgfscope}%
\begin{pgfscope}%
\pgfpathrectangle{\pgfqpoint{0.100000in}{0.100000in}}{\pgfqpoint{5.307240in}{3.397500in}}%
\pgfusepath{clip}%
\pgfsetbuttcap%
\pgfsetroundjoin%
\pgfsetlinewidth{0.501875pt}%
\definecolor{currentstroke}{rgb}{0.827451,0.827451,0.827451}%
\pgfsetstrokecolor{currentstroke}%
\pgfsetdash{}{0pt}%
\pgfpathmoveto{\pgfqpoint{2.928005in}{2.447849in}}%
\pgfpathlineto{\pgfqpoint{2.930398in}{2.569027in}}%
\pgfpathlineto{\pgfqpoint{2.931827in}{2.655132in}}%
\pgfpathlineto{\pgfqpoint{2.925602in}{2.666925in}}%
\pgfpathlineto{\pgfqpoint{2.912483in}{2.671312in}}%
\pgfpathlineto{\pgfqpoint{2.900584in}{2.691842in}}%
\pgfpathlineto{\pgfqpoint{2.916121in}{2.709674in}}%
\pgfpathlineto{\pgfqpoint{2.923906in}{2.732389in}}%
\pgfpathlineto{\pgfqpoint{2.925449in}{2.751273in}}%
\pgfpathlineto{\pgfqpoint{2.921873in}{2.779558in}}%
\pgfpathlineto{\pgfqpoint{2.911031in}{2.797711in}}%
\pgfpathlineto{\pgfqpoint{2.907325in}{2.814246in}}%
\pgfpathlineto{\pgfqpoint{2.904625in}{2.875599in}}%
\pgfpathlineto{\pgfqpoint{2.904659in}{2.928412in}}%
\pgfpathlineto{\pgfqpoint{2.890071in}{2.970360in}}%
\pgfpathlineto{\pgfqpoint{2.883703in}{2.998794in}}%
\pgfpathlineto{\pgfqpoint{2.884806in}{3.036761in}}%
\pgfpathlineto{\pgfqpoint{2.888156in}{3.055232in}}%
\pgfpathlineto{\pgfqpoint{2.878797in}{3.093884in}}%
\pgfpathlineto{\pgfqpoint{2.942517in}{3.092608in}}%
\pgfpathlineto{\pgfqpoint{3.039304in}{3.091776in}}%
\pgfpathlineto{\pgfqpoint{3.039833in}{3.135707in}}%
\pgfpathlineto{\pgfqpoint{3.064469in}{3.130870in}}%
\pgfpathlineto{\pgfqpoint{3.076276in}{3.077304in}}%
\pgfpathlineto{\pgfqpoint{3.084981in}{3.058044in}}%
\pgfpathlineto{\pgfqpoint{3.106628in}{3.057476in}}%
\pgfpathlineto{\pgfqpoint{3.111473in}{3.050941in}}%
\pgfpathlineto{\pgfqpoint{3.141668in}{3.048046in}}%
\pgfpathlineto{\pgfqpoint{3.146743in}{3.034766in}}%
\pgfpathlineto{\pgfqpoint{3.167552in}{3.037751in}}%
\pgfpathlineto{\pgfqpoint{3.183714in}{3.050172in}}%
\pgfpathlineto{\pgfqpoint{3.211586in}{3.049707in}}%
\pgfpathlineto{\pgfqpoint{3.228833in}{3.039718in}}%
\pgfpathlineto{\pgfqpoint{3.230815in}{3.030350in}}%
\pgfpathlineto{\pgfqpoint{3.247225in}{3.028390in}}%
\pgfpathlineto{\pgfqpoint{3.257934in}{3.002837in}}%
\pgfpathlineto{\pgfqpoint{3.264844in}{3.018548in}}%
\pgfpathlineto{\pgfqpoint{3.283695in}{3.019514in}}%
\pgfpathlineto{\pgfqpoint{3.288463in}{3.007278in}}%
\pgfpathlineto{\pgfqpoint{3.309673in}{3.001693in}}%
\pgfpathlineto{\pgfqpoint{3.321370in}{2.984069in}}%
\pgfpathlineto{\pgfqpoint{3.347312in}{2.989541in}}%
\pgfpathlineto{\pgfqpoint{3.375857in}{3.011169in}}%
\pgfpathlineto{\pgfqpoint{3.386264in}{2.992103in}}%
\pgfpathlineto{\pgfqpoint{3.433103in}{2.997322in}}%
\pgfpathlineto{\pgfqpoint{3.453125in}{2.984236in}}%
\pgfpathlineto{\pgfqpoint{3.464778in}{2.988908in}}%
\pgfpathlineto{\pgfqpoint{3.474123in}{2.981541in}}%
\pgfpathlineto{\pgfqpoint{3.446401in}{2.964064in}}%
\pgfpathlineto{\pgfqpoint{3.406831in}{2.948490in}}%
\pgfpathlineto{\pgfqpoint{3.367639in}{2.917363in}}%
\pgfpathlineto{\pgfqpoint{3.333662in}{2.876429in}}%
\pgfpathlineto{\pgfqpoint{3.307941in}{2.852228in}}%
\pgfpathlineto{\pgfqpoint{3.285412in}{2.835594in}}%
\pgfpathlineto{\pgfqpoint{3.269915in}{2.819489in}}%
\pgfpathlineto{\pgfqpoint{3.271793in}{2.750869in}}%
\pgfpathlineto{\pgfqpoint{3.241652in}{2.731574in}}%
\pgfpathlineto{\pgfqpoint{3.226432in}{2.707174in}}%
\pgfpathlineto{\pgfqpoint{3.225418in}{2.691728in}}%
\pgfpathlineto{\pgfqpoint{3.237685in}{2.688327in}}%
\pgfpathlineto{\pgfqpoint{3.244649in}{2.670740in}}%
\pgfpathlineto{\pgfqpoint{3.236898in}{2.659093in}}%
\pgfpathlineto{\pgfqpoint{3.237685in}{2.616782in}}%
\pgfpathlineto{\pgfqpoint{3.233603in}{2.594532in}}%
\pgfpathlineto{\pgfqpoint{3.258735in}{2.575197in}}%
\pgfpathlineto{\pgfqpoint{3.272812in}{2.572988in}}%
\pgfpathlineto{\pgfqpoint{3.284432in}{2.559966in}}%
\pgfpathlineto{\pgfqpoint{3.304198in}{2.551965in}}%
\pgfpathlineto{\pgfqpoint{3.315214in}{2.532097in}}%
\pgfpathlineto{\pgfqpoint{3.336526in}{2.515304in}}%
\pgfpathlineto{\pgfqpoint{3.356109in}{2.505128in}}%
\pgfpathlineto{\pgfqpoint{3.367137in}{2.486997in}}%
\pgfpathlineto{\pgfqpoint{3.370846in}{2.453619in}}%
\pgfpathlineto{\pgfqpoint{3.266914in}{2.449844in}}%
\pgfpathlineto{\pgfqpoint{3.178322in}{2.447992in}}%
\pgfpathlineto{\pgfqpoint{3.097605in}{2.446806in}}%
\pgfpathlineto{\pgfqpoint{3.012216in}{2.446936in}}%
\pgfpathlineto{\pgfqpoint{2.928005in}{2.447849in}}%
\pgfusepath{stroke}%
\end{pgfscope}%
\begin{pgfscope}%
\pgfpathrectangle{\pgfqpoint{0.100000in}{0.100000in}}{\pgfqpoint{5.307240in}{3.397500in}}%
\pgfusepath{clip}%
\pgfsetbuttcap%
\pgfsetroundjoin%
\pgfsetlinewidth{0.501875pt}%
\definecolor{currentstroke}{rgb}{0.827451,0.827451,0.827451}%
\pgfsetstrokecolor{currentstroke}%
\pgfsetdash{}{0pt}%
\pgfpathmoveto{\pgfqpoint{0.568241in}{2.674388in}}%
\pgfpathlineto{\pgfqpoint{0.560010in}{2.689534in}}%
\pgfpathlineto{\pgfqpoint{0.565266in}{2.728410in}}%
\pgfpathlineto{\pgfqpoint{0.575410in}{2.748727in}}%
\pgfpathlineto{\pgfqpoint{0.570339in}{2.776211in}}%
\pgfpathlineto{\pgfqpoint{0.580921in}{2.787769in}}%
\pgfpathlineto{\pgfqpoint{0.600227in}{2.818922in}}%
\pgfpathlineto{\pgfqpoint{0.616596in}{2.837728in}}%
\pgfpathlineto{\pgfqpoint{0.640524in}{2.878685in}}%
\pgfpathlineto{\pgfqpoint{0.658876in}{2.923302in}}%
\pgfpathlineto{\pgfqpoint{0.678358in}{2.965148in}}%
\pgfpathlineto{\pgfqpoint{0.682296in}{2.982625in}}%
\pgfpathlineto{\pgfqpoint{0.709172in}{3.032622in}}%
\pgfpathlineto{\pgfqpoint{0.714347in}{3.054580in}}%
\pgfpathlineto{\pgfqpoint{0.725790in}{3.077639in}}%
\pgfpathlineto{\pgfqpoint{0.729880in}{3.105768in}}%
\pgfpathlineto{\pgfqpoint{0.751725in}{3.119484in}}%
\pgfpathlineto{\pgfqpoint{0.777628in}{3.126397in}}%
\pgfpathlineto{\pgfqpoint{0.790724in}{3.110914in}}%
\pgfpathlineto{\pgfqpoint{0.802046in}{3.112059in}}%
\pgfpathlineto{\pgfqpoint{0.819723in}{3.093604in}}%
\pgfpathlineto{\pgfqpoint{0.822073in}{3.076220in}}%
\pgfpathlineto{\pgfqpoint{0.815370in}{3.042180in}}%
\pgfpathlineto{\pgfqpoint{0.837673in}{3.024813in}}%
\pgfpathlineto{\pgfqpoint{0.852139in}{3.018282in}}%
\pgfpathlineto{\pgfqpoint{0.891326in}{3.024948in}}%
\pgfpathlineto{\pgfqpoint{0.914029in}{3.020396in}}%
\pgfpathlineto{\pgfqpoint{0.929633in}{3.012925in}}%
\pgfpathlineto{\pgfqpoint{0.937853in}{2.998746in}}%
\pgfpathlineto{\pgfqpoint{0.989900in}{3.000638in}}%
\pgfpathlineto{\pgfqpoint{0.998176in}{2.991961in}}%
\pgfpathlineto{\pgfqpoint{1.017701in}{2.990088in}}%
\pgfpathlineto{\pgfqpoint{1.037364in}{2.995631in}}%
\pgfpathlineto{\pgfqpoint{1.095498in}{2.994235in}}%
\pgfpathlineto{\pgfqpoint{1.108054in}{2.989963in}}%
\pgfpathlineto{\pgfqpoint{1.122727in}{2.994885in}}%
\pgfpathlineto{\pgfqpoint{1.285152in}{2.952501in}}%
\pgfpathlineto{\pgfqpoint{1.291770in}{2.929728in}}%
\pgfpathlineto{\pgfqpoint{1.306675in}{2.917390in}}%
\pgfpathlineto{\pgfqpoint{1.308222in}{2.895179in}}%
\pgfpathlineto{\pgfqpoint{1.286423in}{2.872691in}}%
\pgfpathlineto{\pgfqpoint{1.275796in}{2.851435in}}%
\pgfpathlineto{\pgfqpoint{1.262996in}{2.840079in}}%
\pgfpathlineto{\pgfqpoint{1.261838in}{2.828713in}}%
\pgfpathlineto{\pgfqpoint{1.240632in}{2.813352in}}%
\pgfpathlineto{\pgfqpoint{1.226083in}{2.791149in}}%
\pgfpathlineto{\pgfqpoint{1.217938in}{2.785756in}}%
\pgfpathlineto{\pgfqpoint{1.213494in}{2.764209in}}%
\pgfpathlineto{\pgfqpoint{1.234893in}{2.742266in}}%
\pgfpathlineto{\pgfqpoint{1.214227in}{2.705760in}}%
\pgfpathlineto{\pgfqpoint{1.211721in}{2.691827in}}%
\pgfpathlineto{\pgfqpoint{1.165249in}{2.501181in}}%
\pgfpathlineto{\pgfqpoint{1.067498in}{2.526228in}}%
\pgfpathlineto{\pgfqpoint{0.973198in}{2.550543in}}%
\pgfpathlineto{\pgfqpoint{0.916310in}{2.566374in}}%
\pgfpathlineto{\pgfqpoint{0.843216in}{2.587195in}}%
\pgfpathlineto{\pgfqpoint{0.726624in}{2.623893in}}%
\pgfpathlineto{\pgfqpoint{0.599882in}{2.663353in}}%
\pgfpathlineto{\pgfqpoint{0.568241in}{2.674388in}}%
\pgfusepath{stroke}%
\end{pgfscope}%
\begin{pgfscope}%
\pgfpathrectangle{\pgfqpoint{0.100000in}{0.100000in}}{\pgfqpoint{5.307240in}{3.397500in}}%
\pgfusepath{clip}%
\pgfsetbuttcap%
\pgfsetroundjoin%
\pgfsetlinewidth{0.501875pt}%
\definecolor{currentstroke}{rgb}{0.827451,0.827451,0.827451}%
\pgfsetstrokecolor{currentstroke}%
\pgfsetdash{}{0pt}%
\pgfpathmoveto{\pgfqpoint{4.960994in}{2.594734in}}%
\pgfpathlineto{\pgfqpoint{4.951762in}{2.602239in}}%
\pgfpathlineto{\pgfqpoint{4.953048in}{2.625623in}}%
\pgfpathlineto{\pgfqpoint{4.943506in}{2.692085in}}%
\pgfpathlineto{\pgfqpoint{4.950985in}{2.715457in}}%
\pgfpathlineto{\pgfqpoint{4.956328in}{2.756909in}}%
\pgfpathlineto{\pgfqpoint{4.949119in}{2.776520in}}%
\pgfpathlineto{\pgfqpoint{4.975344in}{2.802628in}}%
\pgfpathlineto{\pgfqpoint{4.981944in}{2.823396in}}%
\pgfpathlineto{\pgfqpoint{4.969518in}{2.839786in}}%
\pgfpathlineto{\pgfqpoint{4.974811in}{2.860271in}}%
\pgfpathlineto{\pgfqpoint{4.971493in}{2.873137in}}%
\pgfpathlineto{\pgfqpoint{4.974342in}{2.900695in}}%
\pgfpathlineto{\pgfqpoint{4.979669in}{2.909202in}}%
\pgfpathlineto{\pgfqpoint{4.996094in}{2.914571in}}%
\pgfpathlineto{\pgfqpoint{5.020052in}{2.844408in}}%
\pgfpathlineto{\pgfqpoint{5.064500in}{2.699014in}}%
\pgfpathlineto{\pgfqpoint{5.081321in}{2.687740in}}%
\pgfpathlineto{\pgfqpoint{5.084347in}{2.674872in}}%
\pgfpathlineto{\pgfqpoint{5.093225in}{2.669672in}}%
\pgfpathlineto{\pgfqpoint{5.092550in}{2.646333in}}%
\pgfpathlineto{\pgfqpoint{5.083143in}{2.645958in}}%
\pgfpathlineto{\pgfqpoint{5.064086in}{2.631424in}}%
\pgfpathlineto{\pgfqpoint{5.058590in}{2.616843in}}%
\pgfpathlineto{\pgfqpoint{5.007583in}{2.604224in}}%
\pgfpathlineto{\pgfqpoint{4.960994in}{2.594734in}}%
\pgfusepath{stroke}%
\end{pgfscope}%
\begin{pgfscope}%
\pgfpathrectangle{\pgfqpoint{0.100000in}{0.100000in}}{\pgfqpoint{5.307240in}{3.397500in}}%
\pgfusepath{clip}%
\pgfsetbuttcap%
\pgfsetroundjoin%
\pgfsetlinewidth{0.501875pt}%
\definecolor{currentstroke}{rgb}{0.827451,0.827451,0.827451}%
\pgfsetstrokecolor{currentstroke}%
\pgfsetdash{}{0pt}%
\pgfpathmoveto{\pgfqpoint{3.366045in}{2.088896in}}%
\pgfpathlineto{\pgfqpoint{3.339119in}{2.115563in}}%
\pgfpathlineto{\pgfqpoint{3.253044in}{2.110598in}}%
\pgfpathlineto{\pgfqpoint{3.118802in}{2.106174in}}%
\pgfpathlineto{\pgfqpoint{2.983752in}{2.108281in}}%
\pgfpathlineto{\pgfqpoint{2.974273in}{2.124810in}}%
\pgfpathlineto{\pgfqpoint{2.978143in}{2.141041in}}%
\pgfpathlineto{\pgfqpoint{2.976305in}{2.168846in}}%
\pgfpathlineto{\pgfqpoint{2.969168in}{2.195777in}}%
\pgfpathlineto{\pgfqpoint{2.969998in}{2.209999in}}%
\pgfpathlineto{\pgfqpoint{2.954827in}{2.226068in}}%
\pgfpathlineto{\pgfqpoint{2.958128in}{2.248427in}}%
\pgfpathlineto{\pgfqpoint{2.934855in}{2.292592in}}%
\pgfpathlineto{\pgfqpoint{2.927922in}{2.330086in}}%
\pgfpathlineto{\pgfqpoint{2.911181in}{2.360584in}}%
\pgfpathlineto{\pgfqpoint{2.927071in}{2.398416in}}%
\pgfpathlineto{\pgfqpoint{2.919157in}{2.416021in}}%
\pgfpathlineto{\pgfqpoint{2.928005in}{2.447849in}}%
\pgfpathlineto{\pgfqpoint{3.012216in}{2.446936in}}%
\pgfpathlineto{\pgfqpoint{3.097605in}{2.446806in}}%
\pgfpathlineto{\pgfqpoint{3.178322in}{2.447992in}}%
\pgfpathlineto{\pgfqpoint{3.266914in}{2.449844in}}%
\pgfpathlineto{\pgfqpoint{3.370846in}{2.453619in}}%
\pgfpathlineto{\pgfqpoint{3.373621in}{2.438530in}}%
\pgfpathlineto{\pgfqpoint{3.385127in}{2.425880in}}%
\pgfpathlineto{\pgfqpoint{3.377842in}{2.412698in}}%
\pgfpathlineto{\pgfqpoint{3.380828in}{2.386968in}}%
\pgfpathlineto{\pgfqpoint{3.387959in}{2.366170in}}%
\pgfpathlineto{\pgfqpoint{3.420287in}{2.355161in}}%
\pgfpathlineto{\pgfqpoint{3.425827in}{2.340435in}}%
\pgfpathlineto{\pgfqpoint{3.443568in}{2.323901in}}%
\pgfpathlineto{\pgfqpoint{3.450805in}{2.306797in}}%
\pgfpathlineto{\pgfqpoint{3.468780in}{2.295323in}}%
\pgfpathlineto{\pgfqpoint{3.471593in}{2.281510in}}%
\pgfpathlineto{\pgfqpoint{3.468093in}{2.260598in}}%
\pgfpathlineto{\pgfqpoint{3.458944in}{2.254331in}}%
\pgfpathlineto{\pgfqpoint{3.456184in}{2.234425in}}%
\pgfpathlineto{\pgfqpoint{3.429891in}{2.218615in}}%
\pgfpathlineto{\pgfqpoint{3.395619in}{2.209953in}}%
\pgfpathlineto{\pgfqpoint{3.392479in}{2.190039in}}%
\pgfpathlineto{\pgfqpoint{3.405700in}{2.175811in}}%
\pgfpathlineto{\pgfqpoint{3.406239in}{2.157921in}}%
\pgfpathlineto{\pgfqpoint{3.395572in}{2.143860in}}%
\pgfpathlineto{\pgfqpoint{3.389973in}{2.122969in}}%
\pgfpathlineto{\pgfqpoint{3.371432in}{2.116053in}}%
\pgfpathlineto{\pgfqpoint{3.372614in}{2.092773in}}%
\pgfpathlineto{\pgfqpoint{3.366045in}{2.088896in}}%
\pgfusepath{stroke}%
\end{pgfscope}%
\begin{pgfscope}%
\pgfpathrectangle{\pgfqpoint{0.100000in}{0.100000in}}{\pgfqpoint{5.307240in}{3.397500in}}%
\pgfusepath{clip}%
\pgfsetbuttcap%
\pgfsetroundjoin%
\pgfsetlinewidth{0.501875pt}%
\definecolor{currentstroke}{rgb}{0.827451,0.827451,0.827451}%
\pgfsetstrokecolor{currentstroke}%
\pgfsetdash{}{0pt}%
\pgfpathmoveto{\pgfqpoint{5.036742in}{2.527456in}}%
\pgfpathlineto{\pgfqpoint{5.011238in}{2.523432in}}%
\pgfpathlineto{\pgfqpoint{4.894088in}{2.496811in}}%
\pgfpathlineto{\pgfqpoint{4.892041in}{2.499914in}}%
\pgfpathlineto{\pgfqpoint{4.893809in}{2.580457in}}%
\pgfpathlineto{\pgfqpoint{4.960994in}{2.594734in}}%
\pgfpathlineto{\pgfqpoint{5.007583in}{2.604224in}}%
\pgfpathlineto{\pgfqpoint{5.058590in}{2.616843in}}%
\pgfpathlineto{\pgfqpoint{5.064086in}{2.631424in}}%
\pgfpathlineto{\pgfqpoint{5.083143in}{2.645958in}}%
\pgfpathlineto{\pgfqpoint{5.092550in}{2.646333in}}%
\pgfpathlineto{\pgfqpoint{5.104916in}{2.625130in}}%
\pgfpathlineto{\pgfqpoint{5.093695in}{2.594178in}}%
\pgfpathlineto{\pgfqpoint{5.092048in}{2.576029in}}%
\pgfpathlineto{\pgfqpoint{5.114760in}{2.577734in}}%
\pgfpathlineto{\pgfqpoint{5.125060in}{2.569028in}}%
\pgfpathlineto{\pgfqpoint{5.148158in}{2.533441in}}%
\pgfpathlineto{\pgfqpoint{5.159638in}{2.529112in}}%
\pgfpathlineto{\pgfqpoint{5.178868in}{2.530717in}}%
\pgfpathlineto{\pgfqpoint{5.192250in}{2.542811in}}%
\pgfpathlineto{\pgfqpoint{5.201146in}{2.532018in}}%
\pgfpathlineto{\pgfqpoint{5.145206in}{2.502499in}}%
\pgfpathlineto{\pgfqpoint{5.143458in}{2.523681in}}%
\pgfpathlineto{\pgfqpoint{5.118047in}{2.490759in}}%
\pgfpathlineto{\pgfqpoint{5.109113in}{2.485088in}}%
\pgfpathlineto{\pgfqpoint{5.096651in}{2.504084in}}%
\pgfpathlineto{\pgfqpoint{5.093240in}{2.506687in}}%
\pgfpathlineto{\pgfqpoint{5.081639in}{2.512832in}}%
\pgfpathlineto{\pgfqpoint{5.071522in}{2.537756in}}%
\pgfpathlineto{\pgfqpoint{5.036742in}{2.527456in}}%
\pgfusepath{stroke}%
\end{pgfscope}%
\begin{pgfscope}%
\pgfpathrectangle{\pgfqpoint{0.100000in}{0.100000in}}{\pgfqpoint{5.307240in}{3.397500in}}%
\pgfusepath{clip}%
\pgfsetbuttcap%
\pgfsetroundjoin%
\pgfsetlinewidth{0.501875pt}%
\definecolor{currentstroke}{rgb}{0.827451,0.827451,0.827451}%
\pgfsetstrokecolor{currentstroke}%
\pgfsetdash{}{0pt}%
\pgfpathmoveto{\pgfqpoint{2.424111in}{2.063731in}}%
\pgfpathlineto{\pgfqpoint{2.433450in}{2.179544in}}%
\pgfpathlineto{\pgfqpoint{2.380554in}{2.183838in}}%
\pgfpathlineto{\pgfqpoint{2.258281in}{2.195701in}}%
\pgfpathlineto{\pgfqpoint{2.269657in}{2.311103in}}%
\pgfpathlineto{\pgfqpoint{2.281050in}{2.427386in}}%
\pgfpathlineto{\pgfqpoint{2.328055in}{2.422311in}}%
\pgfpathlineto{\pgfqpoint{2.448103in}{2.410968in}}%
\pgfpathlineto{\pgfqpoint{2.520756in}{2.405279in}}%
\pgfpathlineto{\pgfqpoint{2.608712in}{2.400237in}}%
\pgfpathlineto{\pgfqpoint{2.753486in}{2.393385in}}%
\pgfpathlineto{\pgfqpoint{2.756669in}{2.386894in}}%
\pgfpathlineto{\pgfqpoint{2.798151in}{2.366371in}}%
\pgfpathlineto{\pgfqpoint{2.810873in}{2.376794in}}%
\pgfpathlineto{\pgfqpoint{2.847523in}{2.375842in}}%
\pgfpathlineto{\pgfqpoint{2.869363in}{2.364402in}}%
\pgfpathlineto{\pgfqpoint{2.903995in}{2.351395in}}%
\pgfpathlineto{\pgfqpoint{2.913699in}{2.332421in}}%
\pgfpathlineto{\pgfqpoint{2.927922in}{2.330086in}}%
\pgfpathlineto{\pgfqpoint{2.934855in}{2.292592in}}%
\pgfpathlineto{\pgfqpoint{2.958128in}{2.248427in}}%
\pgfpathlineto{\pgfqpoint{2.954827in}{2.226068in}}%
\pgfpathlineto{\pgfqpoint{2.969998in}{2.209999in}}%
\pgfpathlineto{\pgfqpoint{2.969168in}{2.195777in}}%
\pgfpathlineto{\pgfqpoint{2.976305in}{2.168846in}}%
\pgfpathlineto{\pgfqpoint{2.978143in}{2.141041in}}%
\pgfpathlineto{\pgfqpoint{2.974273in}{2.124810in}}%
\pgfpathlineto{\pgfqpoint{2.983752in}{2.108281in}}%
\pgfpathlineto{\pgfqpoint{2.996754in}{2.078223in}}%
\pgfpathlineto{\pgfqpoint{3.009198in}{2.065990in}}%
\pgfpathlineto{\pgfqpoint{3.024033in}{2.039481in}}%
\pgfpathlineto{\pgfqpoint{2.981989in}{2.039045in}}%
\pgfpathlineto{\pgfqpoint{2.891084in}{2.040451in}}%
\pgfpathlineto{\pgfqpoint{2.790646in}{2.043528in}}%
\pgfpathlineto{\pgfqpoint{2.689618in}{2.047405in}}%
\pgfpathlineto{\pgfqpoint{2.541075in}{2.055519in}}%
\pgfpathlineto{\pgfqpoint{2.424111in}{2.063731in}}%
\pgfusepath{stroke}%
\end{pgfscope}%
\begin{pgfscope}%
\pgfpathrectangle{\pgfqpoint{0.100000in}{0.100000in}}{\pgfqpoint{5.307240in}{3.397500in}}%
\pgfusepath{clip}%
\pgfsetbuttcap%
\pgfsetroundjoin%
\pgfsetlinewidth{0.501875pt}%
\definecolor{currentstroke}{rgb}{0.827451,0.827451,0.827451}%
\pgfsetstrokecolor{currentstroke}%
\pgfsetdash{}{0pt}%
\pgfpathmoveto{\pgfqpoint{4.358205in}{2.413066in}}%
\pgfpathlineto{\pgfqpoint{4.404717in}{2.457415in}}%
\pgfpathlineto{\pgfqpoint{4.410499in}{2.473180in}}%
\pgfpathlineto{\pgfqpoint{4.424128in}{2.486676in}}%
\pgfpathlineto{\pgfqpoint{4.413901in}{2.506326in}}%
\pgfpathlineto{\pgfqpoint{4.401079in}{2.517819in}}%
\pgfpathlineto{\pgfqpoint{4.397373in}{2.538186in}}%
\pgfpathlineto{\pgfqpoint{4.445098in}{2.559098in}}%
\pgfpathlineto{\pgfqpoint{4.484612in}{2.565712in}}%
\pgfpathlineto{\pgfqpoint{4.505844in}{2.566155in}}%
\pgfpathlineto{\pgfqpoint{4.522018in}{2.558164in}}%
\pgfpathlineto{\pgfqpoint{4.537785in}{2.565300in}}%
\pgfpathlineto{\pgfqpoint{4.576239in}{2.573296in}}%
\pgfpathlineto{\pgfqpoint{4.589526in}{2.583618in}}%
\pgfpathlineto{\pgfqpoint{4.609228in}{2.606468in}}%
\pgfpathlineto{\pgfqpoint{4.627151in}{2.616565in}}%
\pgfpathlineto{\pgfqpoint{4.628415in}{2.626249in}}%
\pgfpathlineto{\pgfqpoint{4.619008in}{2.648342in}}%
\pgfpathlineto{\pgfqpoint{4.625809in}{2.661344in}}%
\pgfpathlineto{\pgfqpoint{4.616652in}{2.675324in}}%
\pgfpathlineto{\pgfqpoint{4.602631in}{2.676300in}}%
\pgfpathlineto{\pgfqpoint{4.637724in}{2.718529in}}%
\pgfpathlineto{\pgfqpoint{4.641892in}{2.734623in}}%
\pgfpathlineto{\pgfqpoint{4.669512in}{2.775667in}}%
\pgfpathlineto{\pgfqpoint{4.695152in}{2.797878in}}%
\pgfpathlineto{\pgfqpoint{4.712758in}{2.807154in}}%
\pgfpathlineto{\pgfqpoint{4.770367in}{2.820207in}}%
\pgfpathlineto{\pgfqpoint{4.824362in}{2.834948in}}%
\pgfpathlineto{\pgfqpoint{4.831514in}{2.811808in}}%
\pgfpathlineto{\pgfqpoint{4.832020in}{2.790353in}}%
\pgfpathlineto{\pgfqpoint{4.844337in}{2.770876in}}%
\pgfpathlineto{\pgfqpoint{4.848132in}{2.751046in}}%
\pgfpathlineto{\pgfqpoint{4.843399in}{2.724140in}}%
\pgfpathlineto{\pgfqpoint{4.857815in}{2.687851in}}%
\pgfpathlineto{\pgfqpoint{4.874248in}{2.667671in}}%
\pgfpathlineto{\pgfqpoint{4.886960in}{2.602080in}}%
\pgfpathlineto{\pgfqpoint{4.893809in}{2.580457in}}%
\pgfpathlineto{\pgfqpoint{4.892041in}{2.499914in}}%
\pgfpathlineto{\pgfqpoint{4.894088in}{2.496811in}}%
\pgfpathlineto{\pgfqpoint{4.909052in}{2.410186in}}%
\pgfpathlineto{\pgfqpoint{4.917444in}{2.402292in}}%
\pgfpathlineto{\pgfqpoint{4.899394in}{2.384754in}}%
\pgfpathlineto{\pgfqpoint{4.908300in}{2.374688in}}%
\pgfpathlineto{\pgfqpoint{4.900471in}{2.359463in}}%
\pgfpathlineto{\pgfqpoint{4.900538in}{2.352981in}}%
\pgfpathlineto{\pgfqpoint{4.890751in}{2.347133in}}%
\pgfpathlineto{\pgfqpoint{4.885985in}{2.334204in}}%
\pgfpathlineto{\pgfqpoint{4.887497in}{2.369755in}}%
\pgfpathlineto{\pgfqpoint{4.857162in}{2.377582in}}%
\pgfpathlineto{\pgfqpoint{4.809723in}{2.393764in}}%
\pgfpathlineto{\pgfqpoint{4.803013in}{2.401735in}}%
\pgfpathlineto{\pgfqpoint{4.783190in}{2.403644in}}%
\pgfpathlineto{\pgfqpoint{4.771787in}{2.415508in}}%
\pgfpathlineto{\pgfqpoint{4.765634in}{2.439135in}}%
\pgfpathlineto{\pgfqpoint{4.749469in}{2.442074in}}%
\pgfpathlineto{\pgfqpoint{4.738658in}{2.454469in}}%
\pgfpathlineto{\pgfqpoint{4.601031in}{2.426722in}}%
\pgfpathlineto{\pgfqpoint{4.535255in}{2.413119in}}%
\pgfpathlineto{\pgfqpoint{4.435389in}{2.394891in}}%
\pgfpathlineto{\pgfqpoint{4.363472in}{2.382760in}}%
\pgfpathlineto{\pgfqpoint{4.358205in}{2.413066in}}%
\pgfusepath{stroke}%
\end{pgfscope}%
\begin{pgfscope}%
\pgfpathrectangle{\pgfqpoint{0.100000in}{0.100000in}}{\pgfqpoint{5.307240in}{3.397500in}}%
\pgfusepath{clip}%
\pgfsetbuttcap%
\pgfsetroundjoin%
\pgfsetlinewidth{0.501875pt}%
\definecolor{currentstroke}{rgb}{0.827451,0.827451,0.827451}%
\pgfsetstrokecolor{currentstroke}%
\pgfsetdash{}{0pt}%
\pgfpathmoveto{\pgfqpoint{4.910722in}{2.326983in}}%
\pgfpathlineto{\pgfqpoint{4.889438in}{2.320352in}}%
\pgfpathlineto{\pgfqpoint{4.885857in}{2.326448in}}%
\pgfpathlineto{\pgfqpoint{4.892689in}{2.346911in}}%
\pgfpathlineto{\pgfqpoint{4.905342in}{2.348926in}}%
\pgfpathlineto{\pgfqpoint{4.904209in}{2.355369in}}%
\pgfpathlineto{\pgfqpoint{4.915570in}{2.365038in}}%
\pgfpathlineto{\pgfqpoint{4.948415in}{2.372741in}}%
\pgfpathlineto{\pgfqpoint{4.963036in}{2.384403in}}%
\pgfpathlineto{\pgfqpoint{4.995883in}{2.394121in}}%
\pgfpathlineto{\pgfqpoint{5.010865in}{2.390564in}}%
\pgfpathlineto{\pgfqpoint{5.023467in}{2.406232in}}%
\pgfpathlineto{\pgfqpoint{5.042514in}{2.408302in}}%
\pgfpathlineto{\pgfqpoint{5.010019in}{2.377741in}}%
\pgfpathlineto{\pgfqpoint{4.910722in}{2.326983in}}%
\pgfusepath{stroke}%
\end{pgfscope}%
\begin{pgfscope}%
\pgfpathrectangle{\pgfqpoint{0.100000in}{0.100000in}}{\pgfqpoint{5.307240in}{3.397500in}}%
\pgfusepath{clip}%
\pgfsetbuttcap%
\pgfsetroundjoin%
\pgfsetlinewidth{0.501875pt}%
\definecolor{currentstroke}{rgb}{0.827451,0.827451,0.827451}%
\pgfsetstrokecolor{currentstroke}%
\pgfsetdash{}{0pt}%
\pgfpathmoveto{\pgfqpoint{4.432529in}{2.125186in}}%
\pgfpathlineto{\pgfqpoint{4.340486in}{2.109967in}}%
\pgfpathlineto{\pgfqpoint{4.323790in}{2.215155in}}%
\pgfpathlineto{\pgfqpoint{4.299000in}{2.370208in}}%
\pgfpathlineto{\pgfqpoint{4.358205in}{2.413066in}}%
\pgfpathlineto{\pgfqpoint{4.363472in}{2.382760in}}%
\pgfpathlineto{\pgfqpoint{4.435389in}{2.394891in}}%
\pgfpathlineto{\pgfqpoint{4.535255in}{2.413119in}}%
\pgfpathlineto{\pgfqpoint{4.601031in}{2.426722in}}%
\pgfpathlineto{\pgfqpoint{4.738658in}{2.454469in}}%
\pgfpathlineto{\pgfqpoint{4.749469in}{2.442074in}}%
\pgfpathlineto{\pgfqpoint{4.765634in}{2.439135in}}%
\pgfpathlineto{\pgfqpoint{4.771787in}{2.415508in}}%
\pgfpathlineto{\pgfqpoint{4.783190in}{2.403644in}}%
\pgfpathlineto{\pgfqpoint{4.803013in}{2.401735in}}%
\pgfpathlineto{\pgfqpoint{4.809723in}{2.393764in}}%
\pgfpathlineto{\pgfqpoint{4.802911in}{2.387614in}}%
\pgfpathlineto{\pgfqpoint{4.796774in}{2.365856in}}%
\pgfpathlineto{\pgfqpoint{4.782395in}{2.341408in}}%
\pgfpathlineto{\pgfqpoint{4.792047in}{2.330776in}}%
\pgfpathlineto{\pgfqpoint{4.784148in}{2.319385in}}%
\pgfpathlineto{\pgfqpoint{4.788604in}{2.294404in}}%
\pgfpathlineto{\pgfqpoint{4.802829in}{2.281290in}}%
\pgfpathlineto{\pgfqpoint{4.836582in}{2.259986in}}%
\pgfpathlineto{\pgfqpoint{4.809400in}{2.229926in}}%
\pgfpathlineto{\pgfqpoint{4.809019in}{2.218516in}}%
\pgfpathlineto{\pgfqpoint{4.786890in}{2.203802in}}%
\pgfpathlineto{\pgfqpoint{4.762421in}{2.201041in}}%
\pgfpathlineto{\pgfqpoint{4.756369in}{2.188252in}}%
\pgfpathlineto{\pgfqpoint{4.688308in}{2.173525in}}%
\pgfpathlineto{\pgfqpoint{4.608891in}{2.157530in}}%
\pgfpathlineto{\pgfqpoint{4.554301in}{2.147755in}}%
\pgfpathlineto{\pgfqpoint{4.432529in}{2.125186in}}%
\pgfusepath{stroke}%
\end{pgfscope}%
\begin{pgfscope}%
\pgfpathrectangle{\pgfqpoint{0.100000in}{0.100000in}}{\pgfqpoint{5.307240in}{3.397500in}}%
\pgfusepath{clip}%
\pgfsetbuttcap%
\pgfsetroundjoin%
\pgfsetlinewidth{0.501875pt}%
\definecolor{currentstroke}{rgb}{0.827451,0.827451,0.827451}%
\pgfsetstrokecolor{currentstroke}%
\pgfsetdash{}{0pt}%
\pgfpathmoveto{\pgfqpoint{4.894088in}{2.496811in}}%
\pgfpathlineto{\pgfqpoint{5.011238in}{2.523432in}}%
\pgfpathlineto{\pgfqpoint{5.036742in}{2.527456in}}%
\pgfpathlineto{\pgfqpoint{5.053622in}{2.461088in}}%
\pgfpathlineto{\pgfqpoint{5.050950in}{2.449203in}}%
\pgfpathlineto{\pgfqpoint{4.996725in}{2.428210in}}%
\pgfpathlineto{\pgfqpoint{4.964355in}{2.420873in}}%
\pgfpathlineto{\pgfqpoint{4.950600in}{2.404414in}}%
\pgfpathlineto{\pgfqpoint{4.908300in}{2.374688in}}%
\pgfpathlineto{\pgfqpoint{4.899394in}{2.384754in}}%
\pgfpathlineto{\pgfqpoint{4.917444in}{2.402292in}}%
\pgfpathlineto{\pgfqpoint{4.909052in}{2.410186in}}%
\pgfpathlineto{\pgfqpoint{4.894088in}{2.496811in}}%
\pgfusepath{stroke}%
\end{pgfscope}%
\begin{pgfscope}%
\pgfpathrectangle{\pgfqpoint{0.100000in}{0.100000in}}{\pgfqpoint{5.307240in}{3.397500in}}%
\pgfusepath{clip}%
\pgfsetbuttcap%
\pgfsetroundjoin%
\pgfsetlinewidth{0.501875pt}%
\definecolor{currentstroke}{rgb}{0.827451,0.827451,0.827451}%
\pgfsetstrokecolor{currentstroke}%
\pgfsetdash{}{0pt}%
\pgfpathmoveto{\pgfqpoint{5.036742in}{2.527456in}}%
\pgfpathlineto{\pgfqpoint{5.071522in}{2.537756in}}%
\pgfpathlineto{\pgfqpoint{5.081639in}{2.512832in}}%
\pgfpathlineto{\pgfqpoint{5.093240in}{2.506687in}}%
\pgfpathlineto{\pgfqpoint{5.078924in}{2.496189in}}%
\pgfpathlineto{\pgfqpoint{5.080730in}{2.465355in}}%
\pgfpathlineto{\pgfqpoint{5.050950in}{2.449203in}}%
\pgfpathlineto{\pgfqpoint{5.053622in}{2.461088in}}%
\pgfpathlineto{\pgfqpoint{5.036742in}{2.527456in}}%
\pgfusepath{stroke}%
\end{pgfscope}%
\begin{pgfscope}%
\pgfpathrectangle{\pgfqpoint{0.100000in}{0.100000in}}{\pgfqpoint{5.307240in}{3.397500in}}%
\pgfusepath{clip}%
\pgfsetbuttcap%
\pgfsetroundjoin%
\pgfsetlinewidth{0.501875pt}%
\definecolor{currentstroke}{rgb}{0.827451,0.827451,0.827451}%
\pgfsetstrokecolor{currentstroke}%
\pgfsetdash{}{0pt}%
\pgfpathmoveto{\pgfqpoint{4.782969in}{2.192935in}}%
\pgfpathlineto{\pgfqpoint{4.786890in}{2.203802in}}%
\pgfpathlineto{\pgfqpoint{4.809019in}{2.218516in}}%
\pgfpathlineto{\pgfqpoint{4.809400in}{2.229926in}}%
\pgfpathlineto{\pgfqpoint{4.836582in}{2.259986in}}%
\pgfpathlineto{\pgfqpoint{4.802829in}{2.281290in}}%
\pgfpathlineto{\pgfqpoint{4.788604in}{2.294404in}}%
\pgfpathlineto{\pgfqpoint{4.784148in}{2.319385in}}%
\pgfpathlineto{\pgfqpoint{4.792047in}{2.330776in}}%
\pgfpathlineto{\pgfqpoint{4.782395in}{2.341408in}}%
\pgfpathlineto{\pgfqpoint{4.796774in}{2.365856in}}%
\pgfpathlineto{\pgfqpoint{4.802911in}{2.387614in}}%
\pgfpathlineto{\pgfqpoint{4.809723in}{2.393764in}}%
\pgfpathlineto{\pgfqpoint{4.857162in}{2.377582in}}%
\pgfpathlineto{\pgfqpoint{4.887497in}{2.369755in}}%
\pgfpathlineto{\pgfqpoint{4.885985in}{2.334204in}}%
\pgfpathlineto{\pgfqpoint{4.867558in}{2.307244in}}%
\pgfpathlineto{\pgfqpoint{4.882742in}{2.303278in}}%
\pgfpathlineto{\pgfqpoint{4.898507in}{2.291709in}}%
\pgfpathlineto{\pgfqpoint{4.899437in}{2.260065in}}%
\pgfpathlineto{\pgfqpoint{4.894666in}{2.237664in}}%
\pgfpathlineto{\pgfqpoint{4.897841in}{2.219264in}}%
\pgfpathlineto{\pgfqpoint{4.870436in}{2.157174in}}%
\pgfpathlineto{\pgfqpoint{4.860587in}{2.128182in}}%
\pgfpathlineto{\pgfqpoint{4.846836in}{2.142274in}}%
\pgfpathlineto{\pgfqpoint{4.828600in}{2.139873in}}%
\pgfpathlineto{\pgfqpoint{4.782961in}{2.166241in}}%
\pgfpathlineto{\pgfqpoint{4.778291in}{2.180362in}}%
\pgfpathlineto{\pgfqpoint{4.782969in}{2.192935in}}%
\pgfusepath{stroke}%
\end{pgfscope}%
\begin{pgfscope}%
\pgfpathrectangle{\pgfqpoint{0.100000in}{0.100000in}}{\pgfqpoint{5.307240in}{3.397500in}}%
\pgfusepath{clip}%
\pgfsetbuttcap%
\pgfsetroundjoin%
\pgfsetlinewidth{0.501875pt}%
\definecolor{currentstroke}{rgb}{0.827451,0.827451,0.827451}%
\pgfsetstrokecolor{currentstroke}%
\pgfsetdash{}{0pt}%
\pgfpathmoveto{\pgfqpoint{3.690726in}{1.808908in}}%
\pgfpathlineto{\pgfqpoint{3.688592in}{1.837711in}}%
\pgfpathlineto{\pgfqpoint{3.696885in}{1.852002in}}%
\pgfpathlineto{\pgfqpoint{3.691419in}{1.859004in}}%
\pgfpathlineto{\pgfqpoint{3.711351in}{1.881931in}}%
\pgfpathlineto{\pgfqpoint{3.710209in}{1.887456in}}%
\pgfpathlineto{\pgfqpoint{3.729575in}{1.927454in}}%
\pgfpathlineto{\pgfqpoint{3.725661in}{1.946744in}}%
\pgfpathlineto{\pgfqpoint{3.711618in}{1.966939in}}%
\pgfpathlineto{\pgfqpoint{3.721366in}{1.991510in}}%
\pgfpathlineto{\pgfqpoint{3.708731in}{2.153200in}}%
\pgfpathlineto{\pgfqpoint{3.699632in}{2.266631in}}%
\pgfpathlineto{\pgfqpoint{3.712201in}{2.257232in}}%
\pgfpathlineto{\pgfqpoint{3.726224in}{2.257487in}}%
\pgfpathlineto{\pgfqpoint{3.759344in}{2.276672in}}%
\pgfpathlineto{\pgfqpoint{3.860932in}{2.286151in}}%
\pgfpathlineto{\pgfqpoint{3.936124in}{2.294075in}}%
\pgfpathlineto{\pgfqpoint{3.936789in}{2.286718in}}%
\pgfpathlineto{\pgfqpoint{3.953953in}{2.131277in}}%
\pgfpathlineto{\pgfqpoint{3.968727in}{1.986644in}}%
\pgfpathlineto{\pgfqpoint{3.962359in}{1.979854in}}%
\pgfpathlineto{\pgfqpoint{3.972141in}{1.963000in}}%
\pgfpathlineto{\pgfqpoint{3.972091in}{1.950854in}}%
\pgfpathlineto{\pgfqpoint{3.958132in}{1.947802in}}%
\pgfpathlineto{\pgfqpoint{3.942512in}{1.936093in}}%
\pgfpathlineto{\pgfqpoint{3.931936in}{1.940703in}}%
\pgfpathlineto{\pgfqpoint{3.916107in}{1.933207in}}%
\pgfpathlineto{\pgfqpoint{3.921005in}{1.918153in}}%
\pgfpathlineto{\pgfqpoint{3.904735in}{1.903029in}}%
\pgfpathlineto{\pgfqpoint{3.900242in}{1.885534in}}%
\pgfpathlineto{\pgfqpoint{3.889056in}{1.882664in}}%
\pgfpathlineto{\pgfqpoint{3.880707in}{1.869415in}}%
\pgfpathlineto{\pgfqpoint{3.881800in}{1.856072in}}%
\pgfpathlineto{\pgfqpoint{3.871978in}{1.846689in}}%
\pgfpathlineto{\pgfqpoint{3.857199in}{1.848137in}}%
\pgfpathlineto{\pgfqpoint{3.845965in}{1.862534in}}%
\pgfpathlineto{\pgfqpoint{3.826930in}{1.848670in}}%
\pgfpathlineto{\pgfqpoint{3.828266in}{1.836556in}}%
\pgfpathlineto{\pgfqpoint{3.807095in}{1.829477in}}%
\pgfpathlineto{\pgfqpoint{3.801776in}{1.838386in}}%
\pgfpathlineto{\pgfqpoint{3.781166in}{1.828285in}}%
\pgfpathlineto{\pgfqpoint{3.773636in}{1.813823in}}%
\pgfpathlineto{\pgfqpoint{3.748814in}{1.828652in}}%
\pgfpathlineto{\pgfqpoint{3.709475in}{1.818828in}}%
\pgfpathlineto{\pgfqpoint{3.690726in}{1.808908in}}%
\pgfusepath{stroke}%
\end{pgfscope}%
\begin{pgfscope}%
\pgfpathrectangle{\pgfqpoint{0.100000in}{0.100000in}}{\pgfqpoint{5.307240in}{3.397500in}}%
\pgfusepath{clip}%
\pgfsetbuttcap%
\pgfsetroundjoin%
\pgfsetlinewidth{0.501875pt}%
\definecolor{currentstroke}{rgb}{0.827451,0.827451,0.827451}%
\pgfsetstrokecolor{currentstroke}%
\pgfsetdash{}{0pt}%
\pgfpathmoveto{\pgfqpoint{0.916310in}{2.566374in}}%
\pgfpathlineto{\pgfqpoint{0.973198in}{2.550543in}}%
\pgfpathlineto{\pgfqpoint{1.067498in}{2.526228in}}%
\pgfpathlineto{\pgfqpoint{1.165249in}{2.501181in}}%
\pgfpathlineto{\pgfqpoint{1.255346in}{2.479453in}}%
\pgfpathlineto{\pgfqpoint{1.333196in}{2.461827in}}%
\pgfpathlineto{\pgfqpoint{1.416486in}{2.443512in}}%
\pgfpathlineto{\pgfqpoint{1.392421in}{2.329946in}}%
\pgfpathlineto{\pgfqpoint{1.370982in}{2.229092in}}%
\pgfpathlineto{\pgfqpoint{1.335813in}{2.066366in}}%
\pgfpathlineto{\pgfqpoint{1.309409in}{1.943544in}}%
\pgfpathlineto{\pgfqpoint{1.295144in}{1.875002in}}%
\pgfpathlineto{\pgfqpoint{1.276853in}{1.786042in}}%
\pgfpathlineto{\pgfqpoint{1.264198in}{1.767995in}}%
\pgfpathlineto{\pgfqpoint{1.254012in}{1.767392in}}%
\pgfpathlineto{\pgfqpoint{1.246747in}{1.783146in}}%
\pgfpathlineto{\pgfqpoint{1.230066in}{1.788867in}}%
\pgfpathlineto{\pgfqpoint{1.212091in}{1.786862in}}%
\pgfpathlineto{\pgfqpoint{1.206989in}{1.773976in}}%
\pgfpathlineto{\pgfqpoint{1.206457in}{1.729228in}}%
\pgfpathlineto{\pgfqpoint{1.201074in}{1.719012in}}%
\pgfpathlineto{\pgfqpoint{1.204151in}{1.683088in}}%
\pgfpathlineto{\pgfqpoint{1.192910in}{1.659155in}}%
\pgfpathlineto{\pgfqpoint{1.101757in}{1.799516in}}%
\pgfpathlineto{\pgfqpoint{1.011988in}{1.935908in}}%
\pgfpathlineto{\pgfqpoint{0.965259in}{2.007429in}}%
\pgfpathlineto{\pgfqpoint{0.926278in}{2.069161in}}%
\pgfpathlineto{\pgfqpoint{0.877159in}{2.145525in}}%
\pgfpathlineto{\pgfqpoint{0.821559in}{2.231183in}}%
\pgfpathlineto{\pgfqpoint{0.844418in}{2.312488in}}%
\pgfpathlineto{\pgfqpoint{0.890420in}{2.475547in}}%
\pgfpathlineto{\pgfqpoint{0.916310in}{2.566374in}}%
\pgfusepath{stroke}%
\end{pgfscope}%
\begin{pgfscope}%
\pgfpathrectangle{\pgfqpoint{0.100000in}{0.100000in}}{\pgfqpoint{5.307240in}{3.397500in}}%
\pgfusepath{clip}%
\pgfsetbuttcap%
\pgfsetroundjoin%
\pgfsetlinewidth{0.501875pt}%
\definecolor{currentstroke}{rgb}{0.827451,0.827451,0.827451}%
\pgfsetstrokecolor{currentstroke}%
\pgfsetdash{}{0pt}%
\pgfpathmoveto{\pgfqpoint{1.416486in}{2.443512in}}%
\pgfpathlineto{\pgfqpoint{1.505531in}{2.425800in}}%
\pgfpathlineto{\pgfqpoint{1.670381in}{2.393976in}}%
\pgfpathlineto{\pgfqpoint{1.649747in}{2.279509in}}%
\pgfpathlineto{\pgfqpoint{1.740503in}{2.263971in}}%
\pgfpathlineto{\pgfqpoint{1.823186in}{2.250740in}}%
\pgfpathlineto{\pgfqpoint{1.808819in}{2.160244in}}%
\pgfpathlineto{\pgfqpoint{1.793595in}{2.062657in}}%
\pgfpathlineto{\pgfqpoint{1.773212in}{1.934523in}}%
\pgfpathlineto{\pgfqpoint{1.772685in}{1.923782in}}%
\pgfpathlineto{\pgfqpoint{1.751527in}{1.790989in}}%
\pgfpathlineto{\pgfqpoint{1.664433in}{1.804488in}}%
\pgfpathlineto{\pgfqpoint{1.620052in}{1.813400in}}%
\pgfpathlineto{\pgfqpoint{1.459638in}{1.841599in}}%
\pgfpathlineto{\pgfqpoint{1.399229in}{1.853512in}}%
\pgfpathlineto{\pgfqpoint{1.295144in}{1.875002in}}%
\pgfpathlineto{\pgfqpoint{1.309409in}{1.943544in}}%
\pgfpathlineto{\pgfqpoint{1.335813in}{2.066366in}}%
\pgfpathlineto{\pgfqpoint{1.370982in}{2.229092in}}%
\pgfpathlineto{\pgfqpoint{1.392421in}{2.329946in}}%
\pgfpathlineto{\pgfqpoint{1.416486in}{2.443512in}}%
\pgfusepath{stroke}%
\end{pgfscope}%
\begin{pgfscope}%
\pgfpathrectangle{\pgfqpoint{0.100000in}{0.100000in}}{\pgfqpoint{5.307240in}{3.397500in}}%
\pgfusepath{clip}%
\pgfsetbuttcap%
\pgfsetroundjoin%
\pgfsetlinewidth{0.501875pt}%
\definecolor{currentstroke}{rgb}{0.827451,0.827451,0.827451}%
\pgfsetstrokecolor{currentstroke}%
\pgfsetdash{}{0pt}%
\pgfpathmoveto{\pgfqpoint{0.568241in}{2.674388in}}%
\pgfpathlineto{\pgfqpoint{0.599882in}{2.663353in}}%
\pgfpathlineto{\pgfqpoint{0.726624in}{2.623893in}}%
\pgfpathlineto{\pgfqpoint{0.843216in}{2.587195in}}%
\pgfpathlineto{\pgfqpoint{0.916310in}{2.566374in}}%
\pgfpathlineto{\pgfqpoint{0.890420in}{2.475547in}}%
\pgfpathlineto{\pgfqpoint{0.844418in}{2.312488in}}%
\pgfpathlineto{\pgfqpoint{0.821559in}{2.231183in}}%
\pgfpathlineto{\pgfqpoint{0.877159in}{2.145525in}}%
\pgfpathlineto{\pgfqpoint{0.926278in}{2.069161in}}%
\pgfpathlineto{\pgfqpoint{0.965259in}{2.007429in}}%
\pgfpathlineto{\pgfqpoint{1.011988in}{1.935908in}}%
\pgfpathlineto{\pgfqpoint{1.101757in}{1.799516in}}%
\pgfpathlineto{\pgfqpoint{1.192910in}{1.659155in}}%
\pgfpathlineto{\pgfqpoint{1.189251in}{1.645236in}}%
\pgfpathlineto{\pgfqpoint{1.200245in}{1.623068in}}%
\pgfpathlineto{\pgfqpoint{1.202419in}{1.592744in}}%
\pgfpathlineto{\pgfqpoint{1.218387in}{1.578033in}}%
\pgfpathlineto{\pgfqpoint{1.219019in}{1.566181in}}%
\pgfpathlineto{\pgfqpoint{1.190428in}{1.552728in}}%
\pgfpathlineto{\pgfqpoint{1.176806in}{1.539260in}}%
\pgfpathlineto{\pgfqpoint{1.172559in}{1.509519in}}%
\pgfpathlineto{\pgfqpoint{1.165664in}{1.493300in}}%
\pgfpathlineto{\pgfqpoint{1.151137in}{1.479628in}}%
\pgfpathlineto{\pgfqpoint{1.136688in}{1.444078in}}%
\pgfpathlineto{\pgfqpoint{1.157004in}{1.425559in}}%
\pgfpathlineto{\pgfqpoint{1.154381in}{1.410292in}}%
\pgfpathlineto{\pgfqpoint{1.137723in}{1.399670in}}%
\pgfpathlineto{\pgfqpoint{1.126237in}{1.401544in}}%
\pgfpathlineto{\pgfqpoint{0.990668in}{1.420358in}}%
\pgfpathlineto{\pgfqpoint{0.890537in}{1.434563in}}%
\pgfpathlineto{\pgfqpoint{0.894915in}{1.450729in}}%
\pgfpathlineto{\pgfqpoint{0.888397in}{1.477569in}}%
\pgfpathlineto{\pgfqpoint{0.887739in}{1.504642in}}%
\pgfpathlineto{\pgfqpoint{0.883480in}{1.520485in}}%
\pgfpathlineto{\pgfqpoint{0.870357in}{1.543136in}}%
\pgfpathlineto{\pgfqpoint{0.832611in}{1.595397in}}%
\pgfpathlineto{\pgfqpoint{0.820248in}{1.601801in}}%
\pgfpathlineto{\pgfqpoint{0.804320in}{1.601700in}}%
\pgfpathlineto{\pgfqpoint{0.807959in}{1.618186in}}%
\pgfpathlineto{\pgfqpoint{0.800416in}{1.638803in}}%
\pgfpathlineto{\pgfqpoint{0.763368in}{1.649050in}}%
\pgfpathlineto{\pgfqpoint{0.740785in}{1.668024in}}%
\pgfpathlineto{\pgfqpoint{0.738912in}{1.679633in}}%
\pgfpathlineto{\pgfqpoint{0.712908in}{1.708376in}}%
\pgfpathlineto{\pgfqpoint{0.688156in}{1.713876in}}%
\pgfpathlineto{\pgfqpoint{0.665222in}{1.728483in}}%
\pgfpathlineto{\pgfqpoint{0.635059in}{1.733537in}}%
\pgfpathlineto{\pgfqpoint{0.622172in}{1.753019in}}%
\pgfpathlineto{\pgfqpoint{0.634411in}{1.783856in}}%
\pgfpathlineto{\pgfqpoint{0.630668in}{1.790790in}}%
\pgfpathlineto{\pgfqpoint{0.640859in}{1.816497in}}%
\pgfpathlineto{\pgfqpoint{0.622729in}{1.830177in}}%
\pgfpathlineto{\pgfqpoint{0.628612in}{1.854979in}}%
\pgfpathlineto{\pgfqpoint{0.618927in}{1.861318in}}%
\pgfpathlineto{\pgfqpoint{0.610566in}{1.884760in}}%
\pgfpathlineto{\pgfqpoint{0.600476in}{1.891910in}}%
\pgfpathlineto{\pgfqpoint{0.599717in}{1.908842in}}%
\pgfpathlineto{\pgfqpoint{0.591812in}{1.920782in}}%
\pgfpathlineto{\pgfqpoint{0.579899in}{1.960994in}}%
\pgfpathlineto{\pgfqpoint{0.566866in}{1.980254in}}%
\pgfpathlineto{\pgfqpoint{0.569699in}{2.012950in}}%
\pgfpathlineto{\pgfqpoint{0.585026in}{2.016240in}}%
\pgfpathlineto{\pgfqpoint{0.595007in}{2.033974in}}%
\pgfpathlineto{\pgfqpoint{0.588976in}{2.053208in}}%
\pgfpathlineto{\pgfqpoint{0.572674in}{2.056419in}}%
\pgfpathlineto{\pgfqpoint{0.564541in}{2.065415in}}%
\pgfpathlineto{\pgfqpoint{0.551371in}{2.098506in}}%
\pgfpathlineto{\pgfqpoint{0.557526in}{2.110390in}}%
\pgfpathlineto{\pgfqpoint{0.553160in}{2.132518in}}%
\pgfpathlineto{\pgfqpoint{0.562828in}{2.161178in}}%
\pgfpathlineto{\pgfqpoint{0.574154in}{2.150624in}}%
\pgfpathlineto{\pgfqpoint{0.569014in}{2.138164in}}%
\pgfpathlineto{\pgfqpoint{0.588642in}{2.118411in}}%
\pgfpathlineto{\pgfqpoint{0.587389in}{2.147742in}}%
\pgfpathlineto{\pgfqpoint{0.578978in}{2.155609in}}%
\pgfpathlineto{\pgfqpoint{0.588591in}{2.181399in}}%
\pgfpathlineto{\pgfqpoint{0.624953in}{2.177291in}}%
\pgfpathlineto{\pgfqpoint{0.620059in}{2.186882in}}%
\pgfpathlineto{\pgfqpoint{0.596061in}{2.185938in}}%
\pgfpathlineto{\pgfqpoint{0.584645in}{2.200463in}}%
\pgfpathlineto{\pgfqpoint{0.570275in}{2.187565in}}%
\pgfpathlineto{\pgfqpoint{0.562650in}{2.166004in}}%
\pgfpathlineto{\pgfqpoint{0.542298in}{2.195016in}}%
\pgfpathlineto{\pgfqpoint{0.534460in}{2.200298in}}%
\pgfpathlineto{\pgfqpoint{0.537446in}{2.231873in}}%
\pgfpathlineto{\pgfqpoint{0.531243in}{2.250492in}}%
\pgfpathlineto{\pgfqpoint{0.519998in}{2.267986in}}%
\pgfpathlineto{\pgfqpoint{0.496970in}{2.321586in}}%
\pgfpathlineto{\pgfqpoint{0.504499in}{2.333439in}}%
\pgfpathlineto{\pgfqpoint{0.504412in}{2.370897in}}%
\pgfpathlineto{\pgfqpoint{0.516839in}{2.391790in}}%
\pgfpathlineto{\pgfqpoint{0.519689in}{2.424442in}}%
\pgfpathlineto{\pgfqpoint{0.507970in}{2.461852in}}%
\pgfpathlineto{\pgfqpoint{0.492423in}{2.485665in}}%
\pgfpathlineto{\pgfqpoint{0.495196in}{2.507153in}}%
\pgfpathlineto{\pgfqpoint{0.538856in}{2.559159in}}%
\pgfpathlineto{\pgfqpoint{0.541025in}{2.576899in}}%
\pgfpathlineto{\pgfqpoint{0.560698in}{2.610740in}}%
\pgfpathlineto{\pgfqpoint{0.563427in}{2.642816in}}%
\pgfpathlineto{\pgfqpoint{0.557102in}{2.651000in}}%
\pgfpathlineto{\pgfqpoint{0.568241in}{2.674388in}}%
\pgfusepath{stroke}%
\end{pgfscope}%
\begin{pgfscope}%
\pgfpathrectangle{\pgfqpoint{0.100000in}{0.100000in}}{\pgfqpoint{5.307240in}{3.397500in}}%
\pgfusepath{clip}%
\pgfsetbuttcap%
\pgfsetroundjoin%
\pgfsetlinewidth{0.501875pt}%
\definecolor{currentstroke}{rgb}{0.827451,0.827451,0.827451}%
\pgfsetstrokecolor{currentstroke}%
\pgfsetdash{}{0pt}%
\pgfpathmoveto{\pgfqpoint{3.968727in}{1.986644in}}%
\pgfpathlineto{\pgfqpoint{3.953953in}{2.131277in}}%
\pgfpathlineto{\pgfqpoint{3.936789in}{2.286718in}}%
\pgfpathlineto{\pgfqpoint{4.049197in}{2.303459in}}%
\pgfpathlineto{\pgfqpoint{4.079030in}{2.295694in}}%
\pgfpathlineto{\pgfqpoint{4.093338in}{2.287265in}}%
\pgfpathlineto{\pgfqpoint{4.111268in}{2.289603in}}%
\pgfpathlineto{\pgfqpoint{4.134908in}{2.275648in}}%
\pgfpathlineto{\pgfqpoint{4.178941in}{2.296417in}}%
\pgfpathlineto{\pgfqpoint{4.203241in}{2.297111in}}%
\pgfpathlineto{\pgfqpoint{4.231617in}{2.328769in}}%
\pgfpathlineto{\pgfqpoint{4.260489in}{2.348026in}}%
\pgfpathlineto{\pgfqpoint{4.299000in}{2.370208in}}%
\pgfpathlineto{\pgfqpoint{4.323790in}{2.215155in}}%
\pgfpathlineto{\pgfqpoint{4.312272in}{2.205197in}}%
\pgfpathlineto{\pgfqpoint{4.319708in}{2.196051in}}%
\pgfpathlineto{\pgfqpoint{4.322671in}{2.176000in}}%
\pgfpathlineto{\pgfqpoint{4.316901in}{2.157167in}}%
\pgfpathlineto{\pgfqpoint{4.316138in}{2.129640in}}%
\pgfpathlineto{\pgfqpoint{4.310719in}{2.093812in}}%
\pgfpathlineto{\pgfqpoint{4.283302in}{2.061857in}}%
\pgfpathlineto{\pgfqpoint{4.271760in}{2.055075in}}%
\pgfpathlineto{\pgfqpoint{4.262785in}{2.060914in}}%
\pgfpathlineto{\pgfqpoint{4.240573in}{2.030458in}}%
\pgfpathlineto{\pgfqpoint{4.242653in}{2.001132in}}%
\pgfpathlineto{\pgfqpoint{4.217916in}{2.008185in}}%
\pgfpathlineto{\pgfqpoint{4.206203in}{1.978896in}}%
\pgfpathlineto{\pgfqpoint{4.212124in}{1.958142in}}%
\pgfpathlineto{\pgfqpoint{4.202855in}{1.955077in}}%
\pgfpathlineto{\pgfqpoint{4.201558in}{1.938678in}}%
\pgfpathlineto{\pgfqpoint{4.178806in}{1.932063in}}%
\pgfpathlineto{\pgfqpoint{4.166984in}{1.945304in}}%
\pgfpathlineto{\pgfqpoint{4.151772in}{1.950438in}}%
\pgfpathlineto{\pgfqpoint{4.148197in}{1.963863in}}%
\pgfpathlineto{\pgfqpoint{4.134442in}{1.961501in}}%
\pgfpathlineto{\pgfqpoint{4.125431in}{1.949153in}}%
\pgfpathlineto{\pgfqpoint{4.112540in}{1.944811in}}%
\pgfpathlineto{\pgfqpoint{4.089769in}{1.953549in}}%
\pgfpathlineto{\pgfqpoint{4.077175in}{1.943152in}}%
\pgfpathlineto{\pgfqpoint{4.059276in}{1.955453in}}%
\pgfpathlineto{\pgfqpoint{4.024937in}{1.959229in}}%
\pgfpathlineto{\pgfqpoint{4.014579in}{1.981576in}}%
\pgfpathlineto{\pgfqpoint{4.001456in}{1.991477in}}%
\pgfpathlineto{\pgfqpoint{3.988734in}{1.985139in}}%
\pgfpathlineto{\pgfqpoint{3.968727in}{1.986644in}}%
\pgfusepath{stroke}%
\end{pgfscope}%
\begin{pgfscope}%
\pgfpathrectangle{\pgfqpoint{0.100000in}{0.100000in}}{\pgfqpoint{5.307240in}{3.397500in}}%
\pgfusepath{clip}%
\pgfsetbuttcap%
\pgfsetroundjoin%
\pgfsetlinewidth{0.501875pt}%
\definecolor{currentstroke}{rgb}{0.827451,0.827451,0.827451}%
\pgfsetstrokecolor{currentstroke}%
\pgfsetdash{}{0pt}%
\pgfpathmoveto{\pgfqpoint{3.596578in}{1.706929in}}%
\pgfpathlineto{\pgfqpoint{3.583408in}{1.717628in}}%
\pgfpathlineto{\pgfqpoint{3.572669in}{1.712538in}}%
\pgfpathlineto{\pgfqpoint{3.558940in}{1.738173in}}%
\pgfpathlineto{\pgfqpoint{3.565954in}{1.754285in}}%
\pgfpathlineto{\pgfqpoint{3.555846in}{1.772421in}}%
\pgfpathlineto{\pgfqpoint{3.555303in}{1.786699in}}%
\pgfpathlineto{\pgfqpoint{3.541644in}{1.791787in}}%
\pgfpathlineto{\pgfqpoint{3.535310in}{1.802545in}}%
\pgfpathlineto{\pgfqpoint{3.508604in}{1.815980in}}%
\pgfpathlineto{\pgfqpoint{3.485411in}{1.832534in}}%
\pgfpathlineto{\pgfqpoint{3.474633in}{1.845032in}}%
\pgfpathlineto{\pgfqpoint{3.474410in}{1.860274in}}%
\pgfpathlineto{\pgfqpoint{3.488831in}{1.889539in}}%
\pgfpathlineto{\pgfqpoint{3.493268in}{1.911923in}}%
\pgfpathlineto{\pgfqpoint{3.481519in}{1.924570in}}%
\pgfpathlineto{\pgfqpoint{3.465958in}{1.929336in}}%
\pgfpathlineto{\pgfqpoint{3.447081in}{1.918915in}}%
\pgfpathlineto{\pgfqpoint{3.439090in}{1.936806in}}%
\pgfpathlineto{\pgfqpoint{3.435058in}{1.961057in}}%
\pgfpathlineto{\pgfqpoint{3.407238in}{1.982675in}}%
\pgfpathlineto{\pgfqpoint{3.401678in}{1.992269in}}%
\pgfpathlineto{\pgfqpoint{3.376268in}{2.013986in}}%
\pgfpathlineto{\pgfqpoint{3.368304in}{2.029775in}}%
\pgfpathlineto{\pgfqpoint{3.361124in}{2.061086in}}%
\pgfpathlineto{\pgfqpoint{3.366045in}{2.088896in}}%
\pgfpathlineto{\pgfqpoint{3.372614in}{2.092773in}}%
\pgfpathlineto{\pgfqpoint{3.371432in}{2.116053in}}%
\pgfpathlineto{\pgfqpoint{3.389973in}{2.122969in}}%
\pgfpathlineto{\pgfqpoint{3.395572in}{2.143860in}}%
\pgfpathlineto{\pgfqpoint{3.406239in}{2.157921in}}%
\pgfpathlineto{\pgfqpoint{3.405700in}{2.175811in}}%
\pgfpathlineto{\pgfqpoint{3.392479in}{2.190039in}}%
\pgfpathlineto{\pgfqpoint{3.395619in}{2.209953in}}%
\pgfpathlineto{\pgfqpoint{3.429891in}{2.218615in}}%
\pgfpathlineto{\pgfqpoint{3.456184in}{2.234425in}}%
\pgfpathlineto{\pgfqpoint{3.458944in}{2.254331in}}%
\pgfpathlineto{\pgfqpoint{3.468093in}{2.260598in}}%
\pgfpathlineto{\pgfqpoint{3.471593in}{2.281510in}}%
\pgfpathlineto{\pgfqpoint{3.468780in}{2.295323in}}%
\pgfpathlineto{\pgfqpoint{3.450805in}{2.306797in}}%
\pgfpathlineto{\pgfqpoint{3.443568in}{2.323901in}}%
\pgfpathlineto{\pgfqpoint{3.425827in}{2.340435in}}%
\pgfpathlineto{\pgfqpoint{3.571613in}{2.346648in}}%
\pgfpathlineto{\pgfqpoint{3.669420in}{2.353596in}}%
\pgfpathlineto{\pgfqpoint{3.667630in}{2.333022in}}%
\pgfpathlineto{\pgfqpoint{3.684299in}{2.304644in}}%
\pgfpathlineto{\pgfqpoint{3.691298in}{2.280397in}}%
\pgfpathlineto{\pgfqpoint{3.699632in}{2.266631in}}%
\pgfpathlineto{\pgfqpoint{3.708731in}{2.153200in}}%
\pgfpathlineto{\pgfqpoint{3.721366in}{1.991510in}}%
\pgfpathlineto{\pgfqpoint{3.711618in}{1.966939in}}%
\pgfpathlineto{\pgfqpoint{3.725661in}{1.946744in}}%
\pgfpathlineto{\pgfqpoint{3.729575in}{1.927454in}}%
\pgfpathlineto{\pgfqpoint{3.710209in}{1.887456in}}%
\pgfpathlineto{\pgfqpoint{3.711351in}{1.881931in}}%
\pgfpathlineto{\pgfqpoint{3.691419in}{1.859004in}}%
\pgfpathlineto{\pgfqpoint{3.696885in}{1.852002in}}%
\pgfpathlineto{\pgfqpoint{3.688592in}{1.837711in}}%
\pgfpathlineto{\pgfqpoint{3.690726in}{1.808908in}}%
\pgfpathlineto{\pgfqpoint{3.680653in}{1.791232in}}%
\pgfpathlineto{\pgfqpoint{3.688845in}{1.770346in}}%
\pgfpathlineto{\pgfqpoint{3.654522in}{1.758997in}}%
\pgfpathlineto{\pgfqpoint{3.651358in}{1.746654in}}%
\pgfpathlineto{\pgfqpoint{3.660731in}{1.731011in}}%
\pgfpathlineto{\pgfqpoint{3.656419in}{1.720814in}}%
\pgfpathlineto{\pgfqpoint{3.619614in}{1.733410in}}%
\pgfpathlineto{\pgfqpoint{3.607470in}{1.734677in}}%
\pgfpathlineto{\pgfqpoint{3.592330in}{1.715529in}}%
\pgfpathlineto{\pgfqpoint{3.596578in}{1.706929in}}%
\pgfusepath{stroke}%
\end{pgfscope}%
\begin{pgfscope}%
\pgfpathrectangle{\pgfqpoint{0.100000in}{0.100000in}}{\pgfqpoint{5.307240in}{3.397500in}}%
\pgfusepath{clip}%
\pgfsetbuttcap%
\pgfsetroundjoin%
\pgfsetlinewidth{0.501875pt}%
\definecolor{currentstroke}{rgb}{0.827451,0.827451,0.827451}%
\pgfsetstrokecolor{currentstroke}%
\pgfsetdash{}{0pt}%
\pgfpathmoveto{\pgfqpoint{4.667922in}{2.058831in}}%
\pgfpathlineto{\pgfqpoint{4.657793in}{2.073871in}}%
\pgfpathlineto{\pgfqpoint{4.663508in}{2.082288in}}%
\pgfpathlineto{\pgfqpoint{4.677506in}{2.072836in}}%
\pgfpathlineto{\pgfqpoint{4.667922in}{2.058831in}}%
\pgfusepath{stroke}%
\end{pgfscope}%
\begin{pgfscope}%
\pgfpathrectangle{\pgfqpoint{0.100000in}{0.100000in}}{\pgfqpoint{5.307240in}{3.397500in}}%
\pgfusepath{clip}%
\pgfsetbuttcap%
\pgfsetroundjoin%
\pgfsetlinewidth{0.501875pt}%
\definecolor{currentstroke}{rgb}{0.827451,0.827451,0.827451}%
\pgfsetstrokecolor{currentstroke}%
\pgfsetdash{}{0pt}%
\pgfpathmoveto{\pgfqpoint{4.756369in}{2.188252in}}%
\pgfpathlineto{\pgfqpoint{4.762421in}{2.201041in}}%
\pgfpathlineto{\pgfqpoint{4.786890in}{2.203802in}}%
\pgfpathlineto{\pgfqpoint{4.782969in}{2.192935in}}%
\pgfpathlineto{\pgfqpoint{4.774897in}{2.179051in}}%
\pgfpathlineto{\pgfqpoint{4.780367in}{2.162510in}}%
\pgfpathlineto{\pgfqpoint{4.801955in}{2.142690in}}%
\pgfpathlineto{\pgfqpoint{4.806966in}{2.121819in}}%
\pgfpathlineto{\pgfqpoint{4.831839in}{2.095829in}}%
\pgfpathlineto{\pgfqpoint{4.841596in}{2.096942in}}%
\pgfpathlineto{\pgfqpoint{4.853751in}{2.057925in}}%
\pgfpathlineto{\pgfqpoint{4.851758in}{2.057537in}}%
\pgfpathlineto{\pgfqpoint{4.849542in}{2.057102in}}%
\pgfpathlineto{\pgfqpoint{4.795330in}{2.046724in}}%
\pgfpathlineto{\pgfqpoint{4.766326in}{2.149874in}}%
\pgfpathlineto{\pgfqpoint{4.756369in}{2.188252in}}%
\pgfusepath{stroke}%
\end{pgfscope}%
\begin{pgfscope}%
\pgfpathrectangle{\pgfqpoint{0.100000in}{0.100000in}}{\pgfqpoint{5.307240in}{3.397500in}}%
\pgfusepath{clip}%
\pgfsetbuttcap%
\pgfsetroundjoin%
\pgfsetlinewidth{0.501875pt}%
\definecolor{currentstroke}{rgb}{0.827451,0.827451,0.827451}%
\pgfsetstrokecolor{currentstroke}%
\pgfsetdash{}{0pt}%
\pgfpathmoveto{\pgfqpoint{4.250021in}{1.838613in}}%
\pgfpathlineto{\pgfqpoint{4.232890in}{1.839243in}}%
\pgfpathlineto{\pgfqpoint{4.217105in}{1.850116in}}%
\pgfpathlineto{\pgfqpoint{4.195781in}{1.883163in}}%
\pgfpathlineto{\pgfqpoint{4.177628in}{1.900665in}}%
\pgfpathlineto{\pgfqpoint{4.182372in}{1.914183in}}%
\pgfpathlineto{\pgfqpoint{4.178806in}{1.932063in}}%
\pgfpathlineto{\pgfqpoint{4.201558in}{1.938678in}}%
\pgfpathlineto{\pgfqpoint{4.202855in}{1.955077in}}%
\pgfpathlineto{\pgfqpoint{4.212124in}{1.958142in}}%
\pgfpathlineto{\pgfqpoint{4.206203in}{1.978896in}}%
\pgfpathlineto{\pgfqpoint{4.217916in}{2.008185in}}%
\pgfpathlineto{\pgfqpoint{4.242653in}{2.001132in}}%
\pgfpathlineto{\pgfqpoint{4.240573in}{2.030458in}}%
\pgfpathlineto{\pgfqpoint{4.262785in}{2.060914in}}%
\pgfpathlineto{\pgfqpoint{4.271760in}{2.055075in}}%
\pgfpathlineto{\pgfqpoint{4.283302in}{2.061857in}}%
\pgfpathlineto{\pgfqpoint{4.310719in}{2.093812in}}%
\pgfpathlineto{\pgfqpoint{4.316138in}{2.129640in}}%
\pgfpathlineto{\pgfqpoint{4.316901in}{2.157167in}}%
\pgfpathlineto{\pgfqpoint{4.322671in}{2.176000in}}%
\pgfpathlineto{\pgfqpoint{4.319708in}{2.196051in}}%
\pgfpathlineto{\pgfqpoint{4.312272in}{2.205197in}}%
\pgfpathlineto{\pgfqpoint{4.323790in}{2.215155in}}%
\pgfpathlineto{\pgfqpoint{4.340486in}{2.109967in}}%
\pgfpathlineto{\pgfqpoint{4.432529in}{2.125186in}}%
\pgfpathlineto{\pgfqpoint{4.442069in}{2.065148in}}%
\pgfpathlineto{\pgfqpoint{4.475392in}{2.104776in}}%
\pgfpathlineto{\pgfqpoint{4.483243in}{2.100825in}}%
\pgfpathlineto{\pgfqpoint{4.494793in}{2.123787in}}%
\pgfpathlineto{\pgfqpoint{4.510795in}{2.116587in}}%
\pgfpathlineto{\pgfqpoint{4.524747in}{2.117956in}}%
\pgfpathlineto{\pgfqpoint{4.533952in}{2.133918in}}%
\pgfpathlineto{\pgfqpoint{4.547313in}{2.142798in}}%
\pgfpathlineto{\pgfqpoint{4.565889in}{2.134978in}}%
\pgfpathlineto{\pgfqpoint{4.578111in}{2.137680in}}%
\pgfpathlineto{\pgfqpoint{4.595622in}{2.107375in}}%
\pgfpathlineto{\pgfqpoint{4.590558in}{2.084433in}}%
\pgfpathlineto{\pgfqpoint{4.544824in}{2.110264in}}%
\pgfpathlineto{\pgfqpoint{4.536257in}{2.089059in}}%
\pgfpathlineto{\pgfqpoint{4.539083in}{2.079302in}}%
\pgfpathlineto{\pgfqpoint{4.519619in}{2.046321in}}%
\pgfpathlineto{\pgfqpoint{4.510432in}{2.039761in}}%
\pgfpathlineto{\pgfqpoint{4.506284in}{2.025217in}}%
\pgfpathlineto{\pgfqpoint{4.490622in}{2.026738in}}%
\pgfpathlineto{\pgfqpoint{4.479375in}{1.987043in}}%
\pgfpathlineto{\pgfqpoint{4.473080in}{1.977933in}}%
\pgfpathlineto{\pgfqpoint{4.456902in}{1.980971in}}%
\pgfpathlineto{\pgfqpoint{4.440354in}{1.993491in}}%
\pgfpathlineto{\pgfqpoint{4.439782in}{1.974275in}}%
\pgfpathlineto{\pgfqpoint{4.423762in}{1.941979in}}%
\pgfpathlineto{\pgfqpoint{4.419778in}{1.918988in}}%
\pgfpathlineto{\pgfqpoint{4.407487in}{1.903676in}}%
\pgfpathlineto{\pgfqpoint{4.398169in}{1.879223in}}%
\pgfpathlineto{\pgfqpoint{4.397711in}{1.856587in}}%
\pgfpathlineto{\pgfqpoint{4.383320in}{1.852359in}}%
\pgfpathlineto{\pgfqpoint{4.366996in}{1.839544in}}%
\pgfpathlineto{\pgfqpoint{4.351276in}{1.837104in}}%
\pgfpathlineto{\pgfqpoint{4.347654in}{1.826258in}}%
\pgfpathlineto{\pgfqpoint{4.322327in}{1.815150in}}%
\pgfpathlineto{\pgfqpoint{4.308173in}{1.824614in}}%
\pgfpathlineto{\pgfqpoint{4.292353in}{1.806644in}}%
\pgfpathlineto{\pgfqpoint{4.265158in}{1.811941in}}%
\pgfpathlineto{\pgfqpoint{4.248479in}{1.830802in}}%
\pgfpathlineto{\pgfqpoint{4.250021in}{1.838613in}}%
\pgfusepath{stroke}%
\end{pgfscope}%
\begin{pgfscope}%
\pgfpathrectangle{\pgfqpoint{0.100000in}{0.100000in}}{\pgfqpoint{5.307240in}{3.397500in}}%
\pgfusepath{clip}%
\pgfsetbuttcap%
\pgfsetroundjoin%
\pgfsetlinewidth{0.501875pt}%
\definecolor{currentstroke}{rgb}{0.827451,0.827451,0.827451}%
\pgfsetstrokecolor{currentstroke}%
\pgfsetdash{}{0pt}%
\pgfpathmoveto{\pgfqpoint{4.849542in}{2.057102in}}%
\pgfpathlineto{\pgfqpoint{4.848847in}{2.035905in}}%
\pgfpathlineto{\pgfqpoint{4.840689in}{2.025485in}}%
\pgfpathlineto{\pgfqpoint{4.835483in}{2.002351in}}%
\pgfpathlineto{\pgfqpoint{4.811974in}{1.991688in}}%
\pgfpathlineto{\pgfqpoint{4.792250in}{1.988580in}}%
\pgfpathlineto{\pgfqpoint{4.797998in}{2.003794in}}%
\pgfpathlineto{\pgfqpoint{4.767628in}{2.016542in}}%
\pgfpathlineto{\pgfqpoint{4.742956in}{2.032504in}}%
\pgfpathlineto{\pgfqpoint{4.758231in}{2.056364in}}%
\pgfpathlineto{\pgfqpoint{4.745989in}{2.074910in}}%
\pgfpathlineto{\pgfqpoint{4.747430in}{2.090908in}}%
\pgfpathlineto{\pgfqpoint{4.738579in}{2.095299in}}%
\pgfpathlineto{\pgfqpoint{4.731396in}{2.121289in}}%
\pgfpathlineto{\pgfqpoint{4.748381in}{2.156521in}}%
\pgfpathlineto{\pgfqpoint{4.735620in}{2.162274in}}%
\pgfpathlineto{\pgfqpoint{4.732313in}{2.144893in}}%
\pgfpathlineto{\pgfqpoint{4.714097in}{2.140054in}}%
\pgfpathlineto{\pgfqpoint{4.716012in}{2.082866in}}%
\pgfpathlineto{\pgfqpoint{4.712682in}{2.064461in}}%
\pgfpathlineto{\pgfqpoint{4.721818in}{2.038229in}}%
\pgfpathlineto{\pgfqpoint{4.735875in}{2.025608in}}%
\pgfpathlineto{\pgfqpoint{4.729500in}{2.017683in}}%
\pgfpathlineto{\pgfqpoint{4.743848in}{2.006104in}}%
\pgfpathlineto{\pgfqpoint{4.749035in}{1.987308in}}%
\pgfpathlineto{\pgfqpoint{4.722736in}{2.002873in}}%
\pgfpathlineto{\pgfqpoint{4.706093in}{2.000851in}}%
\pgfpathlineto{\pgfqpoint{4.693169in}{2.016861in}}%
\pgfpathlineto{\pgfqpoint{4.680011in}{2.018421in}}%
\pgfpathlineto{\pgfqpoint{4.661310in}{2.010401in}}%
\pgfpathlineto{\pgfqpoint{4.654057in}{2.020395in}}%
\pgfpathlineto{\pgfqpoint{4.663589in}{2.041370in}}%
\pgfpathlineto{\pgfqpoint{4.667922in}{2.058831in}}%
\pgfpathlineto{\pgfqpoint{4.677506in}{2.072836in}}%
\pgfpathlineto{\pgfqpoint{4.663508in}{2.082288in}}%
\pgfpathlineto{\pgfqpoint{4.657793in}{2.073871in}}%
\pgfpathlineto{\pgfqpoint{4.643810in}{2.082415in}}%
\pgfpathlineto{\pgfqpoint{4.619049in}{2.088107in}}%
\pgfpathlineto{\pgfqpoint{4.621295in}{2.100623in}}%
\pgfpathlineto{\pgfqpoint{4.610071in}{2.107883in}}%
\pgfpathlineto{\pgfqpoint{4.595622in}{2.107375in}}%
\pgfpathlineto{\pgfqpoint{4.578111in}{2.137680in}}%
\pgfpathlineto{\pgfqpoint{4.565889in}{2.134978in}}%
\pgfpathlineto{\pgfqpoint{4.547313in}{2.142798in}}%
\pgfpathlineto{\pgfqpoint{4.533952in}{2.133918in}}%
\pgfpathlineto{\pgfqpoint{4.524747in}{2.117956in}}%
\pgfpathlineto{\pgfqpoint{4.510795in}{2.116587in}}%
\pgfpathlineto{\pgfqpoint{4.494793in}{2.123787in}}%
\pgfpathlineto{\pgfqpoint{4.483243in}{2.100825in}}%
\pgfpathlineto{\pgfqpoint{4.475392in}{2.104776in}}%
\pgfpathlineto{\pgfqpoint{4.442069in}{2.065148in}}%
\pgfpathlineto{\pgfqpoint{4.432529in}{2.125186in}}%
\pgfpathlineto{\pgfqpoint{4.554301in}{2.147755in}}%
\pgfpathlineto{\pgfqpoint{4.608891in}{2.157530in}}%
\pgfpathlineto{\pgfqpoint{4.688308in}{2.173525in}}%
\pgfpathlineto{\pgfqpoint{4.756369in}{2.188252in}}%
\pgfpathlineto{\pgfqpoint{4.766326in}{2.149874in}}%
\pgfpathlineto{\pgfqpoint{4.795330in}{2.046724in}}%
\pgfpathlineto{\pgfqpoint{4.849542in}{2.057102in}}%
\pgfusepath{stroke}%
\end{pgfscope}%
\begin{pgfscope}%
\pgfpathrectangle{\pgfqpoint{0.100000in}{0.100000in}}{\pgfqpoint{5.307240in}{3.397500in}}%
\pgfusepath{clip}%
\pgfsetbuttcap%
\pgfsetroundjoin%
\pgfsetlinewidth{0.501875pt}%
\definecolor{currentstroke}{rgb}{0.827451,0.827451,0.827451}%
\pgfsetstrokecolor{currentstroke}%
\pgfsetdash{}{0pt}%
\pgfpathmoveto{\pgfqpoint{2.398311in}{1.714720in}}%
\pgfpathlineto{\pgfqpoint{2.309468in}{1.723199in}}%
\pgfpathlineto{\pgfqpoint{2.217300in}{1.731367in}}%
\pgfpathlineto{\pgfqpoint{2.110797in}{1.742476in}}%
\pgfpathlineto{\pgfqpoint{1.952561in}{1.761296in}}%
\pgfpathlineto{\pgfqpoint{1.896426in}{1.769945in}}%
\pgfpathlineto{\pgfqpoint{1.751527in}{1.790989in}}%
\pgfpathlineto{\pgfqpoint{1.772685in}{1.923782in}}%
\pgfpathlineto{\pgfqpoint{1.773212in}{1.934523in}}%
\pgfpathlineto{\pgfqpoint{1.793595in}{2.062657in}}%
\pgfpathlineto{\pgfqpoint{1.808819in}{2.160244in}}%
\pgfpathlineto{\pgfqpoint{1.823186in}{2.250740in}}%
\pgfpathlineto{\pgfqpoint{1.921344in}{2.236649in}}%
\pgfpathlineto{\pgfqpoint{2.012863in}{2.223514in}}%
\pgfpathlineto{\pgfqpoint{2.181100in}{2.202750in}}%
\pgfpathlineto{\pgfqpoint{2.258281in}{2.195701in}}%
\pgfpathlineto{\pgfqpoint{2.380554in}{2.183838in}}%
\pgfpathlineto{\pgfqpoint{2.433450in}{2.179544in}}%
\pgfpathlineto{\pgfqpoint{2.424111in}{2.063731in}}%
\pgfpathlineto{\pgfqpoint{2.415674in}{1.952223in}}%
\pgfpathlineto{\pgfqpoint{2.408883in}{1.861429in}}%
\pgfpathlineto{\pgfqpoint{2.398311in}{1.714720in}}%
\pgfusepath{stroke}%
\end{pgfscope}%
\begin{pgfscope}%
\pgfpathrectangle{\pgfqpoint{0.100000in}{0.100000in}}{\pgfqpoint{5.307240in}{3.397500in}}%
\pgfusepath{clip}%
\pgfsetbuttcap%
\pgfsetroundjoin%
\pgfsetlinewidth{0.501875pt}%
\definecolor{currentstroke}{rgb}{0.827451,0.827451,0.827451}%
\pgfsetstrokecolor{currentstroke}%
\pgfsetdash{}{0pt}%
\pgfpathmoveto{\pgfqpoint{3.573632in}{1.648885in}}%
\pgfpathlineto{\pgfqpoint{3.576627in}{1.662310in}}%
\pgfpathlineto{\pgfqpoint{3.592151in}{1.659288in}}%
\pgfpathlineto{\pgfqpoint{3.596578in}{1.706929in}}%
\pgfpathlineto{\pgfqpoint{3.592330in}{1.715529in}}%
\pgfpathlineto{\pgfqpoint{3.607470in}{1.734677in}}%
\pgfpathlineto{\pgfqpoint{3.619614in}{1.733410in}}%
\pgfpathlineto{\pgfqpoint{3.656419in}{1.720814in}}%
\pgfpathlineto{\pgfqpoint{3.660731in}{1.731011in}}%
\pgfpathlineto{\pgfqpoint{3.651358in}{1.746654in}}%
\pgfpathlineto{\pgfqpoint{3.654522in}{1.758997in}}%
\pgfpathlineto{\pgfqpoint{3.688845in}{1.770346in}}%
\pgfpathlineto{\pgfqpoint{3.680653in}{1.791232in}}%
\pgfpathlineto{\pgfqpoint{3.690726in}{1.808908in}}%
\pgfpathlineto{\pgfqpoint{3.709475in}{1.818828in}}%
\pgfpathlineto{\pgfqpoint{3.748814in}{1.828652in}}%
\pgfpathlineto{\pgfqpoint{3.773636in}{1.813823in}}%
\pgfpathlineto{\pgfqpoint{3.781166in}{1.828285in}}%
\pgfpathlineto{\pgfqpoint{3.801776in}{1.838386in}}%
\pgfpathlineto{\pgfqpoint{3.807095in}{1.829477in}}%
\pgfpathlineto{\pgfqpoint{3.828266in}{1.836556in}}%
\pgfpathlineto{\pgfqpoint{3.826930in}{1.848670in}}%
\pgfpathlineto{\pgfqpoint{3.845965in}{1.862534in}}%
\pgfpathlineto{\pgfqpoint{3.857199in}{1.848137in}}%
\pgfpathlineto{\pgfqpoint{3.871978in}{1.846689in}}%
\pgfpathlineto{\pgfqpoint{3.881800in}{1.856072in}}%
\pgfpathlineto{\pgfqpoint{3.880707in}{1.869415in}}%
\pgfpathlineto{\pgfqpoint{3.889056in}{1.882664in}}%
\pgfpathlineto{\pgfqpoint{3.900242in}{1.885534in}}%
\pgfpathlineto{\pgfqpoint{3.904735in}{1.903029in}}%
\pgfpathlineto{\pgfqpoint{3.921005in}{1.918153in}}%
\pgfpathlineto{\pgfqpoint{3.916107in}{1.933207in}}%
\pgfpathlineto{\pgfqpoint{3.931936in}{1.940703in}}%
\pgfpathlineto{\pgfqpoint{3.942512in}{1.936093in}}%
\pgfpathlineto{\pgfqpoint{3.958132in}{1.947802in}}%
\pgfpathlineto{\pgfqpoint{3.972091in}{1.950854in}}%
\pgfpathlineto{\pgfqpoint{3.972141in}{1.963000in}}%
\pgfpathlineto{\pgfqpoint{3.962359in}{1.979854in}}%
\pgfpathlineto{\pgfqpoint{3.968727in}{1.986644in}}%
\pgfpathlineto{\pgfqpoint{3.988734in}{1.985139in}}%
\pgfpathlineto{\pgfqpoint{4.001456in}{1.991477in}}%
\pgfpathlineto{\pgfqpoint{4.014579in}{1.981576in}}%
\pgfpathlineto{\pgfqpoint{4.024937in}{1.959229in}}%
\pgfpathlineto{\pgfqpoint{4.059276in}{1.955453in}}%
\pgfpathlineto{\pgfqpoint{4.077175in}{1.943152in}}%
\pgfpathlineto{\pgfqpoint{4.089769in}{1.953549in}}%
\pgfpathlineto{\pgfqpoint{4.112540in}{1.944811in}}%
\pgfpathlineto{\pgfqpoint{4.125431in}{1.949153in}}%
\pgfpathlineto{\pgfqpoint{4.134442in}{1.961501in}}%
\pgfpathlineto{\pgfqpoint{4.148197in}{1.963863in}}%
\pgfpathlineto{\pgfqpoint{4.151772in}{1.950438in}}%
\pgfpathlineto{\pgfqpoint{4.166984in}{1.945304in}}%
\pgfpathlineto{\pgfqpoint{4.178806in}{1.932063in}}%
\pgfpathlineto{\pgfqpoint{4.182372in}{1.914183in}}%
\pgfpathlineto{\pgfqpoint{4.177628in}{1.900665in}}%
\pgfpathlineto{\pgfqpoint{4.195781in}{1.883163in}}%
\pgfpathlineto{\pgfqpoint{4.217105in}{1.850116in}}%
\pgfpathlineto{\pgfqpoint{4.232890in}{1.839243in}}%
\pgfpathlineto{\pgfqpoint{4.250021in}{1.838613in}}%
\pgfpathlineto{\pgfqpoint{4.218389in}{1.802319in}}%
\pgfpathlineto{\pgfqpoint{4.187254in}{1.780351in}}%
\pgfpathlineto{\pgfqpoint{4.175795in}{1.762905in}}%
\pgfpathlineto{\pgfqpoint{4.175989in}{1.753437in}}%
\pgfpathlineto{\pgfqpoint{4.159125in}{1.746181in}}%
\pgfpathlineto{\pgfqpoint{4.154293in}{1.732526in}}%
\pgfpathlineto{\pgfqpoint{4.119161in}{1.718774in}}%
\pgfpathlineto{\pgfqpoint{4.106708in}{1.709841in}}%
\pgfpathlineto{\pgfqpoint{4.105023in}{1.707935in}}%
\pgfpathlineto{\pgfqpoint{4.003968in}{1.698353in}}%
\pgfpathlineto{\pgfqpoint{3.942941in}{1.693221in}}%
\pgfpathlineto{\pgfqpoint{3.842800in}{1.687554in}}%
\pgfpathlineto{\pgfqpoint{3.718040in}{1.675212in}}%
\pgfpathlineto{\pgfqpoint{3.697428in}{1.678066in}}%
\pgfpathlineto{\pgfqpoint{3.701717in}{1.657025in}}%
\pgfpathlineto{\pgfqpoint{3.573632in}{1.648885in}}%
\pgfusepath{stroke}%
\end{pgfscope}%
\begin{pgfscope}%
\pgfpathrectangle{\pgfqpoint{0.100000in}{0.100000in}}{\pgfqpoint{5.307240in}{3.397500in}}%
\pgfusepath{clip}%
\pgfsetbuttcap%
\pgfsetroundjoin%
\pgfsetlinewidth{0.501875pt}%
\definecolor{currentstroke}{rgb}{0.827451,0.827451,0.827451}%
\pgfsetstrokecolor{currentstroke}%
\pgfsetdash{}{0pt}%
\pgfpathmoveto{\pgfqpoint{2.398311in}{1.714720in}}%
\pgfpathlineto{\pgfqpoint{2.408883in}{1.861429in}}%
\pgfpathlineto{\pgfqpoint{2.415674in}{1.952223in}}%
\pgfpathlineto{\pgfqpoint{2.424111in}{2.063731in}}%
\pgfpathlineto{\pgfqpoint{2.541075in}{2.055519in}}%
\pgfpathlineto{\pgfqpoint{2.689618in}{2.047405in}}%
\pgfpathlineto{\pgfqpoint{2.790646in}{2.043528in}}%
\pgfpathlineto{\pgfqpoint{2.891084in}{2.040451in}}%
\pgfpathlineto{\pgfqpoint{2.981989in}{2.039045in}}%
\pgfpathlineto{\pgfqpoint{3.024033in}{2.039481in}}%
\pgfpathlineto{\pgfqpoint{3.042540in}{2.024378in}}%
\pgfpathlineto{\pgfqpoint{3.057036in}{2.027423in}}%
\pgfpathlineto{\pgfqpoint{3.057489in}{2.014250in}}%
\pgfpathlineto{\pgfqpoint{3.046724in}{1.991464in}}%
\pgfpathlineto{\pgfqpoint{3.047891in}{1.977067in}}%
\pgfpathlineto{\pgfqpoint{3.058199in}{1.967574in}}%
\pgfpathlineto{\pgfqpoint{3.067667in}{1.947794in}}%
\pgfpathlineto{\pgfqpoint{3.086873in}{1.936438in}}%
\pgfpathlineto{\pgfqpoint{3.086231in}{1.861879in}}%
\pgfpathlineto{\pgfqpoint{3.086796in}{1.690427in}}%
\pgfpathlineto{\pgfqpoint{2.958063in}{1.691020in}}%
\pgfpathlineto{\pgfqpoint{2.822503in}{1.693392in}}%
\pgfpathlineto{\pgfqpoint{2.722710in}{1.696913in}}%
\pgfpathlineto{\pgfqpoint{2.639410in}{1.700127in}}%
\pgfpathlineto{\pgfqpoint{2.499075in}{1.708402in}}%
\pgfpathlineto{\pgfqpoint{2.398311in}{1.714720in}}%
\pgfusepath{stroke}%
\end{pgfscope}%
\begin{pgfscope}%
\pgfpathrectangle{\pgfqpoint{0.100000in}{0.100000in}}{\pgfqpoint{5.307240in}{3.397500in}}%
\pgfusepath{clip}%
\pgfsetbuttcap%
\pgfsetroundjoin%
\pgfsetlinewidth{0.501875pt}%
\definecolor{currentstroke}{rgb}{0.827451,0.827451,0.827451}%
\pgfsetstrokecolor{currentstroke}%
\pgfsetdash{}{0pt}%
\pgfpathmoveto{\pgfqpoint{4.292466in}{1.734146in}}%
\pgfpathlineto{\pgfqpoint{4.149886in}{1.714117in}}%
\pgfpathlineto{\pgfqpoint{4.106708in}{1.709841in}}%
\pgfpathlineto{\pgfqpoint{4.119161in}{1.718774in}}%
\pgfpathlineto{\pgfqpoint{4.154293in}{1.732526in}}%
\pgfpathlineto{\pgfqpoint{4.159125in}{1.746181in}}%
\pgfpathlineto{\pgfqpoint{4.175989in}{1.753437in}}%
\pgfpathlineto{\pgfqpoint{4.175795in}{1.762905in}}%
\pgfpathlineto{\pgfqpoint{4.187254in}{1.780351in}}%
\pgfpathlineto{\pgfqpoint{4.218389in}{1.802319in}}%
\pgfpathlineto{\pgfqpoint{4.250021in}{1.838613in}}%
\pgfpathlineto{\pgfqpoint{4.248479in}{1.830802in}}%
\pgfpathlineto{\pgfqpoint{4.265158in}{1.811941in}}%
\pgfpathlineto{\pgfqpoint{4.292353in}{1.806644in}}%
\pgfpathlineto{\pgfqpoint{4.308173in}{1.824614in}}%
\pgfpathlineto{\pgfqpoint{4.322327in}{1.815150in}}%
\pgfpathlineto{\pgfqpoint{4.347654in}{1.826258in}}%
\pgfpathlineto{\pgfqpoint{4.351276in}{1.837104in}}%
\pgfpathlineto{\pgfqpoint{4.366996in}{1.839544in}}%
\pgfpathlineto{\pgfqpoint{4.383320in}{1.852359in}}%
\pgfpathlineto{\pgfqpoint{4.397711in}{1.856587in}}%
\pgfpathlineto{\pgfqpoint{4.398169in}{1.879223in}}%
\pgfpathlineto{\pgfqpoint{4.407487in}{1.903676in}}%
\pgfpathlineto{\pgfqpoint{4.419778in}{1.918988in}}%
\pgfpathlineto{\pgfqpoint{4.423762in}{1.941979in}}%
\pgfpathlineto{\pgfqpoint{4.439782in}{1.974275in}}%
\pgfpathlineto{\pgfqpoint{4.440354in}{1.993491in}}%
\pgfpathlineto{\pgfqpoint{4.456902in}{1.980971in}}%
\pgfpathlineto{\pgfqpoint{4.473080in}{1.977933in}}%
\pgfpathlineto{\pgfqpoint{4.479375in}{1.987043in}}%
\pgfpathlineto{\pgfqpoint{4.490622in}{2.026738in}}%
\pgfpathlineto{\pgfqpoint{4.506284in}{2.025217in}}%
\pgfpathlineto{\pgfqpoint{4.510432in}{2.039761in}}%
\pgfpathlineto{\pgfqpoint{4.519619in}{2.046321in}}%
\pgfpathlineto{\pgfqpoint{4.539083in}{2.079302in}}%
\pgfpathlineto{\pgfqpoint{4.536257in}{2.089059in}}%
\pgfpathlineto{\pgfqpoint{4.544824in}{2.110264in}}%
\pgfpathlineto{\pgfqpoint{4.590558in}{2.084433in}}%
\pgfpathlineto{\pgfqpoint{4.595622in}{2.107375in}}%
\pgfpathlineto{\pgfqpoint{4.610071in}{2.107883in}}%
\pgfpathlineto{\pgfqpoint{4.621295in}{2.100623in}}%
\pgfpathlineto{\pgfqpoint{4.619049in}{2.088107in}}%
\pgfpathlineto{\pgfqpoint{4.643810in}{2.082415in}}%
\pgfpathlineto{\pgfqpoint{4.657793in}{2.073871in}}%
\pgfpathlineto{\pgfqpoint{4.667922in}{2.058831in}}%
\pgfpathlineto{\pgfqpoint{4.663589in}{2.041370in}}%
\pgfpathlineto{\pgfqpoint{4.654839in}{2.039939in}}%
\pgfpathlineto{\pgfqpoint{4.649764in}{2.013591in}}%
\pgfpathlineto{\pgfqpoint{4.660881in}{2.003290in}}%
\pgfpathlineto{\pgfqpoint{4.676529in}{2.011617in}}%
\pgfpathlineto{\pgfqpoint{4.691049in}{1.994033in}}%
\pgfpathlineto{\pgfqpoint{4.723491in}{1.990872in}}%
\pgfpathlineto{\pgfqpoint{4.729080in}{1.980757in}}%
\pgfpathlineto{\pgfqpoint{4.759103in}{1.970920in}}%
\pgfpathlineto{\pgfqpoint{4.755398in}{1.959311in}}%
\pgfpathlineto{\pgfqpoint{4.759304in}{1.926503in}}%
\pgfpathlineto{\pgfqpoint{4.771421in}{1.914115in}}%
\pgfpathlineto{\pgfqpoint{4.753540in}{1.908886in}}%
\pgfpathlineto{\pgfqpoint{4.755911in}{1.894880in}}%
\pgfpathlineto{\pgfqpoint{4.774986in}{1.883005in}}%
\pgfpathlineto{\pgfqpoint{4.776706in}{1.871291in}}%
\pgfpathlineto{\pgfqpoint{4.765936in}{1.862488in}}%
\pgfpathlineto{\pgfqpoint{4.744207in}{1.883342in}}%
\pgfpathlineto{\pgfqpoint{4.742058in}{1.868127in}}%
\pgfpathlineto{\pgfqpoint{4.760249in}{1.860891in}}%
\pgfpathlineto{\pgfqpoint{4.762069in}{1.853409in}}%
\pgfpathlineto{\pgfqpoint{4.787020in}{1.863298in}}%
\pgfpathlineto{\pgfqpoint{4.806139in}{1.865914in}}%
\pgfpathlineto{\pgfqpoint{4.825725in}{1.826398in}}%
\pgfpathlineto{\pgfqpoint{4.823540in}{1.825970in}}%
\pgfpathlineto{\pgfqpoint{4.814690in}{1.824141in}}%
\pgfpathlineto{\pgfqpoint{4.812082in}{1.823595in}}%
\pgfpathlineto{\pgfqpoint{4.810358in}{1.823258in}}%
\pgfpathlineto{\pgfqpoint{4.693746in}{1.799070in}}%
\pgfpathlineto{\pgfqpoint{4.589445in}{1.777722in}}%
\pgfpathlineto{\pgfqpoint{4.445236in}{1.752879in}}%
\pgfpathlineto{\pgfqpoint{4.322761in}{1.736703in}}%
\pgfpathlineto{\pgfqpoint{4.292466in}{1.734146in}}%
\pgfusepath{stroke}%
\end{pgfscope}%
\begin{pgfscope}%
\pgfpathrectangle{\pgfqpoint{0.100000in}{0.100000in}}{\pgfqpoint{5.307240in}{3.397500in}}%
\pgfusepath{clip}%
\pgfsetbuttcap%
\pgfsetroundjoin%
\pgfsetlinewidth{0.501875pt}%
\definecolor{currentstroke}{rgb}{0.827451,0.827451,0.827451}%
\pgfsetstrokecolor{currentstroke}%
\pgfsetdash{}{0pt}%
\pgfpathmoveto{\pgfqpoint{4.811974in}{1.991688in}}%
\pgfpathlineto{\pgfqpoint{4.835483in}{2.002351in}}%
\pgfpathlineto{\pgfqpoint{4.816709in}{1.947433in}}%
\pgfpathlineto{\pgfqpoint{4.804294in}{1.918351in}}%
\pgfpathlineto{\pgfqpoint{4.795351in}{1.934809in}}%
\pgfpathlineto{\pgfqpoint{4.811255in}{1.974211in}}%
\pgfpathlineto{\pgfqpoint{4.811974in}{1.991688in}}%
\pgfusepath{stroke}%
\end{pgfscope}%
\begin{pgfscope}%
\pgfpathrectangle{\pgfqpoint{0.100000in}{0.100000in}}{\pgfqpoint{5.307240in}{3.397500in}}%
\pgfusepath{clip}%
\pgfsetbuttcap%
\pgfsetroundjoin%
\pgfsetlinewidth{0.501875pt}%
\definecolor{currentstroke}{rgb}{0.827451,0.827451,0.827451}%
\pgfsetstrokecolor{currentstroke}%
\pgfsetdash{}{0pt}%
\pgfpathmoveto{\pgfqpoint{3.573632in}{1.648885in}}%
\pgfpathlineto{\pgfqpoint{3.567955in}{1.648063in}}%
\pgfpathlineto{\pgfqpoint{3.562601in}{1.647687in}}%
\pgfpathlineto{\pgfqpoint{3.564893in}{1.631250in}}%
\pgfpathlineto{\pgfqpoint{3.551081in}{1.614681in}}%
\pgfpathlineto{\pgfqpoint{3.548359in}{1.588755in}}%
\pgfpathlineto{\pgfqpoint{3.486612in}{1.584188in}}%
\pgfpathlineto{\pgfqpoint{3.491995in}{1.596362in}}%
\pgfpathlineto{\pgfqpoint{3.514257in}{1.618593in}}%
\pgfpathlineto{\pgfqpoint{3.514874in}{1.631480in}}%
\pgfpathlineto{\pgfqpoint{3.505017in}{1.643681in}}%
\pgfpathlineto{\pgfqpoint{3.413115in}{1.638822in}}%
\pgfpathlineto{\pgfqpoint{3.252439in}{1.633721in}}%
\pgfpathlineto{\pgfqpoint{3.158409in}{1.632007in}}%
\pgfpathlineto{\pgfqpoint{3.087335in}{1.631367in}}%
\pgfpathlineto{\pgfqpoint{3.086796in}{1.690427in}}%
\pgfpathlineto{\pgfqpoint{3.086231in}{1.861879in}}%
\pgfpathlineto{\pgfqpoint{3.086873in}{1.936438in}}%
\pgfpathlineto{\pgfqpoint{3.067667in}{1.947794in}}%
\pgfpathlineto{\pgfqpoint{3.058199in}{1.967574in}}%
\pgfpathlineto{\pgfqpoint{3.047891in}{1.977067in}}%
\pgfpathlineto{\pgfqpoint{3.046724in}{1.991464in}}%
\pgfpathlineto{\pgfqpoint{3.057489in}{2.014250in}}%
\pgfpathlineto{\pgfqpoint{3.057036in}{2.027423in}}%
\pgfpathlineto{\pgfqpoint{3.042540in}{2.024378in}}%
\pgfpathlineto{\pgfqpoint{3.024033in}{2.039481in}}%
\pgfpathlineto{\pgfqpoint{3.009198in}{2.065990in}}%
\pgfpathlineto{\pgfqpoint{2.996754in}{2.078223in}}%
\pgfpathlineto{\pgfqpoint{2.983752in}{2.108281in}}%
\pgfpathlineto{\pgfqpoint{3.118802in}{2.106174in}}%
\pgfpathlineto{\pgfqpoint{3.253044in}{2.110598in}}%
\pgfpathlineto{\pgfqpoint{3.339119in}{2.115563in}}%
\pgfpathlineto{\pgfqpoint{3.366045in}{2.088896in}}%
\pgfpathlineto{\pgfqpoint{3.361124in}{2.061086in}}%
\pgfpathlineto{\pgfqpoint{3.368304in}{2.029775in}}%
\pgfpathlineto{\pgfqpoint{3.376268in}{2.013986in}}%
\pgfpathlineto{\pgfqpoint{3.401678in}{1.992269in}}%
\pgfpathlineto{\pgfqpoint{3.407238in}{1.982675in}}%
\pgfpathlineto{\pgfqpoint{3.435058in}{1.961057in}}%
\pgfpathlineto{\pgfqpoint{3.439090in}{1.936806in}}%
\pgfpathlineto{\pgfqpoint{3.447081in}{1.918915in}}%
\pgfpathlineto{\pgfqpoint{3.465958in}{1.929336in}}%
\pgfpathlineto{\pgfqpoint{3.481519in}{1.924570in}}%
\pgfpathlineto{\pgfqpoint{3.493268in}{1.911923in}}%
\pgfpathlineto{\pgfqpoint{3.488831in}{1.889539in}}%
\pgfpathlineto{\pgfqpoint{3.474410in}{1.860274in}}%
\pgfpathlineto{\pgfqpoint{3.474633in}{1.845032in}}%
\pgfpathlineto{\pgfqpoint{3.485411in}{1.832534in}}%
\pgfpathlineto{\pgfqpoint{3.508604in}{1.815980in}}%
\pgfpathlineto{\pgfqpoint{3.535310in}{1.802545in}}%
\pgfpathlineto{\pgfqpoint{3.541644in}{1.791787in}}%
\pgfpathlineto{\pgfqpoint{3.555303in}{1.786699in}}%
\pgfpathlineto{\pgfqpoint{3.555846in}{1.772421in}}%
\pgfpathlineto{\pgfqpoint{3.565954in}{1.754285in}}%
\pgfpathlineto{\pgfqpoint{3.558940in}{1.738173in}}%
\pgfpathlineto{\pgfqpoint{3.572669in}{1.712538in}}%
\pgfpathlineto{\pgfqpoint{3.583408in}{1.717628in}}%
\pgfpathlineto{\pgfqpoint{3.596578in}{1.706929in}}%
\pgfpathlineto{\pgfqpoint{3.592151in}{1.659288in}}%
\pgfpathlineto{\pgfqpoint{3.576627in}{1.662310in}}%
\pgfpathlineto{\pgfqpoint{3.573632in}{1.648885in}}%
\pgfusepath{stroke}%
\end{pgfscope}%
\begin{pgfscope}%
\pgfpathrectangle{\pgfqpoint{0.100000in}{0.100000in}}{\pgfqpoint{5.307240in}{3.397500in}}%
\pgfusepath{clip}%
\pgfsetbuttcap%
\pgfsetroundjoin%
\pgfsetlinewidth{0.501875pt}%
\definecolor{currentstroke}{rgb}{0.827451,0.827451,0.827451}%
\pgfsetstrokecolor{currentstroke}%
\pgfsetdash{}{0pt}%
\pgfpathmoveto{\pgfqpoint{1.192910in}{1.659155in}}%
\pgfpathlineto{\pgfqpoint{1.204151in}{1.683088in}}%
\pgfpathlineto{\pgfqpoint{1.201074in}{1.719012in}}%
\pgfpathlineto{\pgfqpoint{1.206457in}{1.729228in}}%
\pgfpathlineto{\pgfqpoint{1.206989in}{1.773976in}}%
\pgfpathlineto{\pgfqpoint{1.212091in}{1.786862in}}%
\pgfpathlineto{\pgfqpoint{1.230066in}{1.788867in}}%
\pgfpathlineto{\pgfqpoint{1.246747in}{1.783146in}}%
\pgfpathlineto{\pgfqpoint{1.254012in}{1.767392in}}%
\pgfpathlineto{\pgfqpoint{1.264198in}{1.767995in}}%
\pgfpathlineto{\pgfqpoint{1.276853in}{1.786042in}}%
\pgfpathlineto{\pgfqpoint{1.295144in}{1.875002in}}%
\pgfpathlineto{\pgfqpoint{1.399229in}{1.853512in}}%
\pgfpathlineto{\pgfqpoint{1.459638in}{1.841599in}}%
\pgfpathlineto{\pgfqpoint{1.620052in}{1.813400in}}%
\pgfpathlineto{\pgfqpoint{1.664433in}{1.804488in}}%
\pgfpathlineto{\pgfqpoint{1.751527in}{1.790989in}}%
\pgfpathlineto{\pgfqpoint{1.733669in}{1.676002in}}%
\pgfpathlineto{\pgfqpoint{1.715096in}{1.556058in}}%
\pgfpathlineto{\pgfqpoint{1.693706in}{1.421119in}}%
\pgfpathlineto{\pgfqpoint{1.669651in}{1.266224in}}%
\pgfpathlineto{\pgfqpoint{1.650218in}{1.139006in}}%
\pgfpathlineto{\pgfqpoint{1.511026in}{1.161108in}}%
\pgfpathlineto{\pgfqpoint{1.449869in}{1.171582in}}%
\pgfpathlineto{\pgfqpoint{1.422549in}{1.187939in}}%
\pgfpathlineto{\pgfqpoint{1.244420in}{1.295413in}}%
\pgfpathlineto{\pgfqpoint{1.110742in}{1.376974in}}%
\pgfpathlineto{\pgfqpoint{1.115194in}{1.391430in}}%
\pgfpathlineto{\pgfqpoint{1.126237in}{1.401544in}}%
\pgfpathlineto{\pgfqpoint{1.137723in}{1.399670in}}%
\pgfpathlineto{\pgfqpoint{1.154381in}{1.410292in}}%
\pgfpathlineto{\pgfqpoint{1.157004in}{1.425559in}}%
\pgfpathlineto{\pgfqpoint{1.136688in}{1.444078in}}%
\pgfpathlineto{\pgfqpoint{1.151137in}{1.479628in}}%
\pgfpathlineto{\pgfqpoint{1.165664in}{1.493300in}}%
\pgfpathlineto{\pgfqpoint{1.172559in}{1.509519in}}%
\pgfpathlineto{\pgfqpoint{1.176806in}{1.539260in}}%
\pgfpathlineto{\pgfqpoint{1.190428in}{1.552728in}}%
\pgfpathlineto{\pgfqpoint{1.219019in}{1.566181in}}%
\pgfpathlineto{\pgfqpoint{1.218387in}{1.578033in}}%
\pgfpathlineto{\pgfqpoint{1.202419in}{1.592744in}}%
\pgfpathlineto{\pgfqpoint{1.200245in}{1.623068in}}%
\pgfpathlineto{\pgfqpoint{1.189251in}{1.645236in}}%
\pgfpathlineto{\pgfqpoint{1.192910in}{1.659155in}}%
\pgfusepath{stroke}%
\end{pgfscope}%
\begin{pgfscope}%
\pgfpathrectangle{\pgfqpoint{0.100000in}{0.100000in}}{\pgfqpoint{5.307240in}{3.397500in}}%
\pgfusepath{clip}%
\pgfsetbuttcap%
\pgfsetroundjoin%
\pgfsetlinewidth{0.501875pt}%
\definecolor{currentstroke}{rgb}{0.827451,0.827451,0.827451}%
\pgfsetstrokecolor{currentstroke}%
\pgfsetdash{}{0pt}%
\pgfpathmoveto{\pgfqpoint{3.102403in}{1.298435in}}%
\pgfpathlineto{\pgfqpoint{3.076600in}{1.306416in}}%
\pgfpathlineto{\pgfqpoint{3.043537in}{1.331736in}}%
\pgfpathlineto{\pgfqpoint{3.028852in}{1.337178in}}%
\pgfpathlineto{\pgfqpoint{3.019520in}{1.326242in}}%
\pgfpathlineto{\pgfqpoint{3.007735in}{1.325690in}}%
\pgfpathlineto{\pgfqpoint{2.992825in}{1.334974in}}%
\pgfpathlineto{\pgfqpoint{2.969429in}{1.323078in}}%
\pgfpathlineto{\pgfqpoint{2.961101in}{1.328936in}}%
\pgfpathlineto{\pgfqpoint{2.938079in}{1.315058in}}%
\pgfpathlineto{\pgfqpoint{2.913792in}{1.317619in}}%
\pgfpathlineto{\pgfqpoint{2.895059in}{1.326637in}}%
\pgfpathlineto{\pgfqpoint{2.881939in}{1.323242in}}%
\pgfpathlineto{\pgfqpoint{2.860939in}{1.335998in}}%
\pgfpathlineto{\pgfqpoint{2.849258in}{1.313502in}}%
\pgfpathlineto{\pgfqpoint{2.837312in}{1.332849in}}%
\pgfpathlineto{\pgfqpoint{2.813716in}{1.325342in}}%
\pgfpathlineto{\pgfqpoint{2.793077in}{1.343718in}}%
\pgfpathlineto{\pgfqpoint{2.775043in}{1.328907in}}%
\pgfpathlineto{\pgfqpoint{2.766011in}{1.342518in}}%
\pgfpathlineto{\pgfqpoint{2.752993in}{1.346938in}}%
\pgfpathlineto{\pgfqpoint{2.745064in}{1.360061in}}%
\pgfpathlineto{\pgfqpoint{2.728038in}{1.363795in}}%
\pgfpathlineto{\pgfqpoint{2.718213in}{1.353918in}}%
\pgfpathlineto{\pgfqpoint{2.701520in}{1.366719in}}%
\pgfpathlineto{\pgfqpoint{2.693738in}{1.363796in}}%
\pgfpathlineto{\pgfqpoint{2.666072in}{1.374168in}}%
\pgfpathlineto{\pgfqpoint{2.648744in}{1.375325in}}%
\pgfpathlineto{\pgfqpoint{2.640988in}{1.397353in}}%
\pgfpathlineto{\pgfqpoint{2.627097in}{1.394609in}}%
\pgfpathlineto{\pgfqpoint{2.611186in}{1.400045in}}%
\pgfpathlineto{\pgfqpoint{2.600709in}{1.396904in}}%
\pgfpathlineto{\pgfqpoint{2.578240in}{1.421644in}}%
\pgfpathlineto{\pgfqpoint{2.571980in}{1.420026in}}%
\pgfpathlineto{\pgfqpoint{2.577666in}{1.520352in}}%
\pgfpathlineto{\pgfqpoint{2.583860in}{1.644530in}}%
\pgfpathlineto{\pgfqpoint{2.482179in}{1.650218in}}%
\pgfpathlineto{\pgfqpoint{2.381840in}{1.657804in}}%
\pgfpathlineto{\pgfqpoint{2.304321in}{1.664535in}}%
\pgfpathlineto{\pgfqpoint{2.309468in}{1.723199in}}%
\pgfpathlineto{\pgfqpoint{2.398311in}{1.714720in}}%
\pgfpathlineto{\pgfqpoint{2.499075in}{1.708402in}}%
\pgfpathlineto{\pgfqpoint{2.639410in}{1.700127in}}%
\pgfpathlineto{\pgfqpoint{2.722710in}{1.696913in}}%
\pgfpathlineto{\pgfqpoint{2.822503in}{1.693392in}}%
\pgfpathlineto{\pgfqpoint{2.958063in}{1.691020in}}%
\pgfpathlineto{\pgfqpoint{3.086796in}{1.690427in}}%
\pgfpathlineto{\pgfqpoint{3.087335in}{1.631367in}}%
\pgfpathlineto{\pgfqpoint{3.094561in}{1.586874in}}%
\pgfpathlineto{\pgfqpoint{3.105786in}{1.504693in}}%
\pgfpathlineto{\pgfqpoint{3.103471in}{1.364349in}}%
\pgfpathlineto{\pgfqpoint{3.102403in}{1.298435in}}%
\pgfusepath{stroke}%
\end{pgfscope}%
\begin{pgfscope}%
\pgfpathrectangle{\pgfqpoint{0.100000in}{0.100000in}}{\pgfqpoint{5.307240in}{3.397500in}}%
\pgfusepath{clip}%
\pgfsetbuttcap%
\pgfsetroundjoin%
\pgfsetlinewidth{0.501875pt}%
\definecolor{currentstroke}{rgb}{0.827451,0.827451,0.827451}%
\pgfsetstrokecolor{currentstroke}%
\pgfsetdash{}{0pt}%
\pgfpathmoveto{\pgfqpoint{4.068531in}{1.516389in}}%
\pgfpathlineto{\pgfqpoint{4.068607in}{1.542393in}}%
\pgfpathlineto{\pgfqpoint{4.091204in}{1.552378in}}%
\pgfpathlineto{\pgfqpoint{4.092151in}{1.568341in}}%
\pgfpathlineto{\pgfqpoint{4.112384in}{1.587840in}}%
\pgfpathlineto{\pgfqpoint{4.137675in}{1.591834in}}%
\pgfpathlineto{\pgfqpoint{4.159908in}{1.613324in}}%
\pgfpathlineto{\pgfqpoint{4.183128in}{1.622929in}}%
\pgfpathlineto{\pgfqpoint{4.200555in}{1.651765in}}%
\pgfpathlineto{\pgfqpoint{4.216427in}{1.649672in}}%
\pgfpathlineto{\pgfqpoint{4.249971in}{1.675945in}}%
\pgfpathlineto{\pgfqpoint{4.267685in}{1.676441in}}%
\pgfpathlineto{\pgfqpoint{4.275165in}{1.696460in}}%
\pgfpathlineto{\pgfqpoint{4.289244in}{1.710398in}}%
\pgfpathlineto{\pgfqpoint{4.292466in}{1.734146in}}%
\pgfpathlineto{\pgfqpoint{4.322761in}{1.736703in}}%
\pgfpathlineto{\pgfqpoint{4.445236in}{1.752879in}}%
\pgfpathlineto{\pgfqpoint{4.589445in}{1.777722in}}%
\pgfpathlineto{\pgfqpoint{4.693746in}{1.799070in}}%
\pgfpathlineto{\pgfqpoint{4.810358in}{1.823258in}}%
\pgfpathlineto{\pgfqpoint{4.843679in}{1.777509in}}%
\pgfpathlineto{\pgfqpoint{4.825630in}{1.780430in}}%
\pgfpathlineto{\pgfqpoint{4.802460in}{1.774786in}}%
\pgfpathlineto{\pgfqpoint{4.779630in}{1.751471in}}%
\pgfpathlineto{\pgfqpoint{4.763222in}{1.753147in}}%
\pgfpathlineto{\pgfqpoint{4.761146in}{1.739300in}}%
\pgfpathlineto{\pgfqpoint{4.790804in}{1.750251in}}%
\pgfpathlineto{\pgfqpoint{4.820648in}{1.754766in}}%
\pgfpathlineto{\pgfqpoint{4.831729in}{1.748758in}}%
\pgfpathlineto{\pgfqpoint{4.846641in}{1.755607in}}%
\pgfpathlineto{\pgfqpoint{4.854337in}{1.750808in}}%
\pgfpathlineto{\pgfqpoint{4.861148in}{1.727956in}}%
\pgfpathlineto{\pgfqpoint{4.846973in}{1.720879in}}%
\pgfpathlineto{\pgfqpoint{4.837245in}{1.693014in}}%
\pgfpathlineto{\pgfqpoint{4.827045in}{1.682164in}}%
\pgfpathlineto{\pgfqpoint{4.795777in}{1.684538in}}%
\pgfpathlineto{\pgfqpoint{4.797458in}{1.700900in}}%
\pgfpathlineto{\pgfqpoint{4.780381in}{1.693753in}}%
\pgfpathlineto{\pgfqpoint{4.776667in}{1.680106in}}%
\pgfpathlineto{\pgfqpoint{4.788372in}{1.665971in}}%
\pgfpathlineto{\pgfqpoint{4.792349in}{1.641979in}}%
\pgfpathlineto{\pgfqpoint{4.773195in}{1.628309in}}%
\pgfpathlineto{\pgfqpoint{4.793910in}{1.623500in}}%
\pgfpathlineto{\pgfqpoint{4.803283in}{1.633552in}}%
\pgfpathlineto{\pgfqpoint{4.824009in}{1.634776in}}%
\pgfpathlineto{\pgfqpoint{4.813358in}{1.613070in}}%
\pgfpathlineto{\pgfqpoint{4.800313in}{1.602611in}}%
\pgfpathlineto{\pgfqpoint{4.760853in}{1.592140in}}%
\pgfpathlineto{\pgfqpoint{4.720338in}{1.555394in}}%
\pgfpathlineto{\pgfqpoint{4.695356in}{1.519188in}}%
\pgfpathlineto{\pgfqpoint{4.695208in}{1.504484in}}%
\pgfpathlineto{\pgfqpoint{4.685229in}{1.484194in}}%
\pgfpathlineto{\pgfqpoint{4.634014in}{1.470844in}}%
\pgfpathlineto{\pgfqpoint{4.512216in}{1.558155in}}%
\pgfpathlineto{\pgfqpoint{4.404955in}{1.542395in}}%
\pgfpathlineto{\pgfqpoint{4.404056in}{1.556947in}}%
\pgfpathlineto{\pgfqpoint{4.387753in}{1.573335in}}%
\pgfpathlineto{\pgfqpoint{4.375411in}{1.577345in}}%
\pgfpathlineto{\pgfqpoint{4.258871in}{1.565236in}}%
\pgfpathlineto{\pgfqpoint{4.232085in}{1.556151in}}%
\pgfpathlineto{\pgfqpoint{4.183719in}{1.532092in}}%
\pgfpathlineto{\pgfqpoint{4.141921in}{1.525432in}}%
\pgfpathlineto{\pgfqpoint{4.068531in}{1.516389in}}%
\pgfusepath{stroke}%
\end{pgfscope}%
\begin{pgfscope}%
\pgfpathrectangle{\pgfqpoint{0.100000in}{0.100000in}}{\pgfqpoint{5.307240in}{3.397500in}}%
\pgfusepath{clip}%
\pgfsetbuttcap%
\pgfsetroundjoin%
\pgfsetlinewidth{0.501875pt}%
\definecolor{currentstroke}{rgb}{0.827451,0.827451,0.827451}%
\pgfsetstrokecolor{currentstroke}%
\pgfsetdash{}{0pt}%
\pgfpathmoveto{\pgfqpoint{4.068531in}{1.516389in}}%
\pgfpathlineto{\pgfqpoint{3.946559in}{1.502994in}}%
\pgfpathlineto{\pgfqpoint{3.834978in}{1.492959in}}%
\pgfpathlineto{\pgfqpoint{3.720297in}{1.485543in}}%
\pgfpathlineto{\pgfqpoint{3.700611in}{1.482689in}}%
\pgfpathlineto{\pgfqpoint{3.623304in}{1.476743in}}%
\pgfpathlineto{\pgfqpoint{3.499485in}{1.469518in}}%
\pgfpathlineto{\pgfqpoint{3.521514in}{1.487804in}}%
\pgfpathlineto{\pgfqpoint{3.516487in}{1.507052in}}%
\pgfpathlineto{\pgfqpoint{3.521357in}{1.530385in}}%
\pgfpathlineto{\pgfqpoint{3.528796in}{1.541369in}}%
\pgfpathlineto{\pgfqpoint{3.528516in}{1.556652in}}%
\pgfpathlineto{\pgfqpoint{3.548325in}{1.566264in}}%
\pgfpathlineto{\pgfqpoint{3.548359in}{1.588755in}}%
\pgfpathlineto{\pgfqpoint{3.551081in}{1.614681in}}%
\pgfpathlineto{\pgfqpoint{3.564893in}{1.631250in}}%
\pgfpathlineto{\pgfqpoint{3.562601in}{1.647687in}}%
\pgfpathlineto{\pgfqpoint{3.567955in}{1.648063in}}%
\pgfpathlineto{\pgfqpoint{3.573632in}{1.648885in}}%
\pgfpathlineto{\pgfqpoint{3.701717in}{1.657025in}}%
\pgfpathlineto{\pgfqpoint{3.697428in}{1.678066in}}%
\pgfpathlineto{\pgfqpoint{3.718040in}{1.675212in}}%
\pgfpathlineto{\pgfqpoint{3.842800in}{1.687554in}}%
\pgfpathlineto{\pgfqpoint{3.942941in}{1.693221in}}%
\pgfpathlineto{\pgfqpoint{4.003968in}{1.698353in}}%
\pgfpathlineto{\pgfqpoint{4.105023in}{1.707935in}}%
\pgfpathlineto{\pgfqpoint{4.106708in}{1.709841in}}%
\pgfpathlineto{\pgfqpoint{4.149886in}{1.714117in}}%
\pgfpathlineto{\pgfqpoint{4.292466in}{1.734146in}}%
\pgfpathlineto{\pgfqpoint{4.289244in}{1.710398in}}%
\pgfpathlineto{\pgfqpoint{4.275165in}{1.696460in}}%
\pgfpathlineto{\pgfqpoint{4.267685in}{1.676441in}}%
\pgfpathlineto{\pgfqpoint{4.249971in}{1.675945in}}%
\pgfpathlineto{\pgfqpoint{4.216427in}{1.649672in}}%
\pgfpathlineto{\pgfqpoint{4.200555in}{1.651765in}}%
\pgfpathlineto{\pgfqpoint{4.183128in}{1.622929in}}%
\pgfpathlineto{\pgfqpoint{4.159908in}{1.613324in}}%
\pgfpathlineto{\pgfqpoint{4.137675in}{1.591834in}}%
\pgfpathlineto{\pgfqpoint{4.112384in}{1.587840in}}%
\pgfpathlineto{\pgfqpoint{4.092151in}{1.568341in}}%
\pgfpathlineto{\pgfqpoint{4.091204in}{1.552378in}}%
\pgfpathlineto{\pgfqpoint{4.068607in}{1.542393in}}%
\pgfpathlineto{\pgfqpoint{4.068531in}{1.516389in}}%
\pgfusepath{stroke}%
\end{pgfscope}%
\begin{pgfscope}%
\pgfpathrectangle{\pgfqpoint{0.100000in}{0.100000in}}{\pgfqpoint{5.307240in}{3.397500in}}%
\pgfusepath{clip}%
\pgfsetbuttcap%
\pgfsetroundjoin%
\pgfsetlinewidth{0.501875pt}%
\definecolor{currentstroke}{rgb}{0.827451,0.827451,0.827451}%
\pgfsetstrokecolor{currentstroke}%
\pgfsetdash{}{0pt}%
\pgfpathmoveto{\pgfqpoint{1.905330in}{1.155282in}}%
\pgfpathlineto{\pgfqpoint{1.896630in}{1.169260in}}%
\pgfpathlineto{\pgfqpoint{1.900230in}{1.181301in}}%
\pgfpathlineto{\pgfqpoint{2.075075in}{1.160803in}}%
\pgfpathlineto{\pgfqpoint{2.252235in}{1.143143in}}%
\pgfpathlineto{\pgfqpoint{2.257400in}{1.203031in}}%
\pgfpathlineto{\pgfqpoint{2.284028in}{1.462069in}}%
\pgfpathlineto{\pgfqpoint{2.301528in}{1.664691in}}%
\pgfpathlineto{\pgfqpoint{2.304321in}{1.664535in}}%
\pgfpathlineto{\pgfqpoint{2.482179in}{1.650218in}}%
\pgfpathlineto{\pgfqpoint{2.583860in}{1.644530in}}%
\pgfpathlineto{\pgfqpoint{2.571980in}{1.420026in}}%
\pgfpathlineto{\pgfqpoint{2.578240in}{1.421644in}}%
\pgfpathlineto{\pgfqpoint{2.600709in}{1.396904in}}%
\pgfpathlineto{\pgfqpoint{2.611186in}{1.400045in}}%
\pgfpathlineto{\pgfqpoint{2.627097in}{1.394609in}}%
\pgfpathlineto{\pgfqpoint{2.640988in}{1.397353in}}%
\pgfpathlineto{\pgfqpoint{2.648744in}{1.375325in}}%
\pgfpathlineto{\pgfqpoint{2.666072in}{1.374168in}}%
\pgfpathlineto{\pgfqpoint{2.693738in}{1.363796in}}%
\pgfpathlineto{\pgfqpoint{2.701520in}{1.366719in}}%
\pgfpathlineto{\pgfqpoint{2.718213in}{1.353918in}}%
\pgfpathlineto{\pgfqpoint{2.728038in}{1.363795in}}%
\pgfpathlineto{\pgfqpoint{2.745064in}{1.360061in}}%
\pgfpathlineto{\pgfqpoint{2.752993in}{1.346938in}}%
\pgfpathlineto{\pgfqpoint{2.766011in}{1.342518in}}%
\pgfpathlineto{\pgfqpoint{2.775043in}{1.328907in}}%
\pgfpathlineto{\pgfqpoint{2.793077in}{1.343718in}}%
\pgfpathlineto{\pgfqpoint{2.813716in}{1.325342in}}%
\pgfpathlineto{\pgfqpoint{2.837312in}{1.332849in}}%
\pgfpathlineto{\pgfqpoint{2.849258in}{1.313502in}}%
\pgfpathlineto{\pgfqpoint{2.860939in}{1.335998in}}%
\pgfpathlineto{\pgfqpoint{2.881939in}{1.323242in}}%
\pgfpathlineto{\pgfqpoint{2.895059in}{1.326637in}}%
\pgfpathlineto{\pgfqpoint{2.913792in}{1.317619in}}%
\pgfpathlineto{\pgfqpoint{2.938079in}{1.315058in}}%
\pgfpathlineto{\pgfqpoint{2.961101in}{1.328936in}}%
\pgfpathlineto{\pgfqpoint{2.969429in}{1.323078in}}%
\pgfpathlineto{\pgfqpoint{2.992825in}{1.334974in}}%
\pgfpathlineto{\pgfqpoint{3.007735in}{1.325690in}}%
\pgfpathlineto{\pgfqpoint{3.019520in}{1.326242in}}%
\pgfpathlineto{\pgfqpoint{3.028852in}{1.337178in}}%
\pgfpathlineto{\pgfqpoint{3.043537in}{1.331736in}}%
\pgfpathlineto{\pgfqpoint{3.076600in}{1.306416in}}%
\pgfpathlineto{\pgfqpoint{3.102403in}{1.298435in}}%
\pgfpathlineto{\pgfqpoint{3.112750in}{1.288659in}}%
\pgfpathlineto{\pgfqpoint{3.125702in}{1.293983in}}%
\pgfpathlineto{\pgfqpoint{3.145328in}{1.289906in}}%
\pgfpathlineto{\pgfqpoint{3.147350in}{1.107330in}}%
\pgfpathlineto{\pgfqpoint{3.160981in}{1.095765in}}%
\pgfpathlineto{\pgfqpoint{3.171906in}{1.074454in}}%
\pgfpathlineto{\pgfqpoint{3.168066in}{1.060197in}}%
\pgfpathlineto{\pgfqpoint{3.176645in}{1.054160in}}%
\pgfpathlineto{\pgfqpoint{3.183618in}{1.027869in}}%
\pgfpathlineto{\pgfqpoint{3.196971in}{1.013789in}}%
\pgfpathlineto{\pgfqpoint{3.199963in}{0.983895in}}%
\pgfpathlineto{\pgfqpoint{3.197641in}{0.971238in}}%
\pgfpathlineto{\pgfqpoint{3.179489in}{0.937742in}}%
\pgfpathlineto{\pgfqpoint{3.177402in}{0.913658in}}%
\pgfpathlineto{\pgfqpoint{3.183560in}{0.908631in}}%
\pgfpathlineto{\pgfqpoint{3.184095in}{0.887532in}}%
\pgfpathlineto{\pgfqpoint{3.177836in}{0.874271in}}%
\pgfpathlineto{\pgfqpoint{3.168103in}{0.869276in}}%
\pgfpathlineto{\pgfqpoint{3.158572in}{0.851948in}}%
\pgfpathlineto{\pgfqpoint{3.170719in}{0.835164in}}%
\pgfpathlineto{\pgfqpoint{3.147140in}{0.834833in}}%
\pgfpathlineto{\pgfqpoint{3.084093in}{0.805988in}}%
\pgfpathlineto{\pgfqpoint{3.096130in}{0.823251in}}%
\pgfpathlineto{\pgfqpoint{3.081554in}{0.832558in}}%
\pgfpathlineto{\pgfqpoint{3.078508in}{0.848383in}}%
\pgfpathlineto{\pgfqpoint{3.050059in}{0.820819in}}%
\pgfpathlineto{\pgfqpoint{3.062685in}{0.801980in}}%
\pgfpathlineto{\pgfqpoint{3.044674in}{0.777991in}}%
\pgfpathlineto{\pgfqpoint{3.034982in}{0.778494in}}%
\pgfpathlineto{\pgfqpoint{3.025863in}{0.752369in}}%
\pgfpathlineto{\pgfqpoint{2.997027in}{0.731820in}}%
\pgfpathlineto{\pgfqpoint{2.970085in}{0.724415in}}%
\pgfpathlineto{\pgfqpoint{2.923098in}{0.705148in}}%
\pgfpathlineto{\pgfqpoint{2.918229in}{0.715896in}}%
\pgfpathlineto{\pgfqpoint{2.888959in}{0.706004in}}%
\pgfpathlineto{\pgfqpoint{2.906962in}{0.689161in}}%
\pgfpathlineto{\pgfqpoint{2.878566in}{0.674529in}}%
\pgfpathlineto{\pgfqpoint{2.847936in}{0.652436in}}%
\pgfpathlineto{\pgfqpoint{2.840581in}{0.662706in}}%
\pgfpathlineto{\pgfqpoint{2.815514in}{0.647314in}}%
\pgfpathlineto{\pgfqpoint{2.840083in}{0.641469in}}%
\pgfpathlineto{\pgfqpoint{2.821559in}{0.617267in}}%
\pgfpathlineto{\pgfqpoint{2.812517in}{0.624469in}}%
\pgfpathlineto{\pgfqpoint{2.800372in}{0.612900in}}%
\pgfpathlineto{\pgfqpoint{2.815508in}{0.602796in}}%
\pgfpathlineto{\pgfqpoint{2.806559in}{0.588026in}}%
\pgfpathlineto{\pgfqpoint{2.795560in}{0.553063in}}%
\pgfpathlineto{\pgfqpoint{2.779696in}{0.519388in}}%
\pgfpathlineto{\pgfqpoint{2.786817in}{0.497314in}}%
\pgfpathlineto{\pgfqpoint{2.798924in}{0.445274in}}%
\pgfpathlineto{\pgfqpoint{2.800014in}{0.424213in}}%
\pgfpathlineto{\pgfqpoint{2.818719in}{0.396588in}}%
\pgfpathlineto{\pgfqpoint{2.804295in}{0.398202in}}%
\pgfpathlineto{\pgfqpoint{2.790285in}{0.384266in}}%
\pgfpathlineto{\pgfqpoint{2.773877in}{0.395219in}}%
\pgfpathlineto{\pgfqpoint{2.767987in}{0.406146in}}%
\pgfpathlineto{\pgfqpoint{2.744635in}{0.411229in}}%
\pgfpathlineto{\pgfqpoint{2.708963in}{0.411853in}}%
\pgfpathlineto{\pgfqpoint{2.682680in}{0.432545in}}%
\pgfpathlineto{\pgfqpoint{2.658815in}{0.436002in}}%
\pgfpathlineto{\pgfqpoint{2.644337in}{0.452446in}}%
\pgfpathlineto{\pgfqpoint{2.614016in}{0.459065in}}%
\pgfpathlineto{\pgfqpoint{2.597436in}{0.511941in}}%
\pgfpathlineto{\pgfqpoint{2.580484in}{0.533124in}}%
\pgfpathlineto{\pgfqpoint{2.583364in}{0.553251in}}%
\pgfpathlineto{\pgfqpoint{2.572852in}{0.567960in}}%
\pgfpathlineto{\pgfqpoint{2.579449in}{0.588065in}}%
\pgfpathlineto{\pgfqpoint{2.574007in}{0.602798in}}%
\pgfpathlineto{\pgfqpoint{2.556972in}{0.609473in}}%
\pgfpathlineto{\pgfqpoint{2.541044in}{0.626461in}}%
\pgfpathlineto{\pgfqpoint{2.535239in}{0.649207in}}%
\pgfpathlineto{\pgfqpoint{2.520166in}{0.669877in}}%
\pgfpathlineto{\pgfqpoint{2.500105in}{0.685961in}}%
\pgfpathlineto{\pgfqpoint{2.482035in}{0.732097in}}%
\pgfpathlineto{\pgfqpoint{2.467443in}{0.782554in}}%
\pgfpathlineto{\pgfqpoint{2.455471in}{0.802505in}}%
\pgfpathlineto{\pgfqpoint{2.434717in}{0.819301in}}%
\pgfpathlineto{\pgfqpoint{2.429530in}{0.831493in}}%
\pgfpathlineto{\pgfqpoint{2.410096in}{0.839062in}}%
\pgfpathlineto{\pgfqpoint{2.390915in}{0.871365in}}%
\pgfpathlineto{\pgfqpoint{2.373398in}{0.868885in}}%
\pgfpathlineto{\pgfqpoint{2.355611in}{0.876953in}}%
\pgfpathlineto{\pgfqpoint{2.330399in}{0.875393in}}%
\pgfpathlineto{\pgfqpoint{2.304784in}{0.888707in}}%
\pgfpathlineto{\pgfqpoint{2.297562in}{0.876045in}}%
\pgfpathlineto{\pgfqpoint{2.267628in}{0.875729in}}%
\pgfpathlineto{\pgfqpoint{2.252389in}{0.851724in}}%
\pgfpathlineto{\pgfqpoint{2.239135in}{0.822003in}}%
\pgfpathlineto{\pgfqpoint{2.210950in}{0.790100in}}%
\pgfpathlineto{\pgfqpoint{2.179018in}{0.804101in}}%
\pgfpathlineto{\pgfqpoint{2.174487in}{0.813354in}}%
\pgfpathlineto{\pgfqpoint{2.155081in}{0.820411in}}%
\pgfpathlineto{\pgfqpoint{2.149120in}{0.830065in}}%
\pgfpathlineto{\pgfqpoint{2.123359in}{0.839810in}}%
\pgfpathlineto{\pgfqpoint{2.108947in}{0.859731in}}%
\pgfpathlineto{\pgfqpoint{2.092117in}{0.869342in}}%
\pgfpathlineto{\pgfqpoint{2.077637in}{0.886113in}}%
\pgfpathlineto{\pgfqpoint{2.066356in}{0.914518in}}%
\pgfpathlineto{\pgfqpoint{2.067618in}{0.953332in}}%
\pgfpathlineto{\pgfqpoint{2.054385in}{0.972945in}}%
\pgfpathlineto{\pgfqpoint{2.052862in}{0.994189in}}%
\pgfpathlineto{\pgfqpoint{2.023500in}{1.025994in}}%
\pgfpathlineto{\pgfqpoint{2.006415in}{1.032784in}}%
\pgfpathlineto{\pgfqpoint{1.988286in}{1.062462in}}%
\pgfpathlineto{\pgfqpoint{1.972837in}{1.074253in}}%
\pgfpathlineto{\pgfqpoint{1.953207in}{1.102966in}}%
\pgfpathlineto{\pgfqpoint{1.933122in}{1.115393in}}%
\pgfpathlineto{\pgfqpoint{1.919969in}{1.147233in}}%
\pgfpathlineto{\pgfqpoint{1.905330in}{1.155282in}}%
\pgfpathlineto{\pgfqpoint{1.905330in}{1.155282in}}%
\pgfusepath{stroke}%
\end{pgfscope}%
\begin{pgfscope}%
\pgfpathrectangle{\pgfqpoint{0.100000in}{0.100000in}}{\pgfqpoint{5.307240in}{3.397500in}}%
\pgfusepath{clip}%
\pgfsetbuttcap%
\pgfsetroundjoin%
\pgfsetlinewidth{0.501875pt}%
\definecolor{currentstroke}{rgb}{0.827451,0.827451,0.827451}%
\pgfsetstrokecolor{currentstroke}%
\pgfsetdash{}{0pt}%
\pgfpathmoveto{\pgfqpoint{1.751527in}{1.790989in}}%
\pgfpathlineto{\pgfqpoint{1.896426in}{1.769945in}}%
\pgfpathlineto{\pgfqpoint{1.952561in}{1.761296in}}%
\pgfpathlineto{\pgfqpoint{2.110797in}{1.742476in}}%
\pgfpathlineto{\pgfqpoint{2.217300in}{1.731367in}}%
\pgfpathlineto{\pgfqpoint{2.309468in}{1.723199in}}%
\pgfpathlineto{\pgfqpoint{2.304321in}{1.664535in}}%
\pgfpathlineto{\pgfqpoint{2.301528in}{1.664691in}}%
\pgfpathlineto{\pgfqpoint{2.293085in}{1.564005in}}%
\pgfpathlineto{\pgfqpoint{2.284028in}{1.462069in}}%
\pgfpathlineto{\pgfqpoint{2.273551in}{1.355358in}}%
\pgfpathlineto{\pgfqpoint{2.257400in}{1.203031in}}%
\pgfpathlineto{\pgfqpoint{2.252235in}{1.143143in}}%
\pgfpathlineto{\pgfqpoint{2.157286in}{1.152724in}}%
\pgfpathlineto{\pgfqpoint{2.075075in}{1.160803in}}%
\pgfpathlineto{\pgfqpoint{1.961377in}{1.173717in}}%
\pgfpathlineto{\pgfqpoint{1.900230in}{1.181301in}}%
\pgfpathlineto{\pgfqpoint{1.896630in}{1.169260in}}%
\pgfpathlineto{\pgfqpoint{1.905330in}{1.155282in}}%
\pgfpathlineto{\pgfqpoint{1.831842in}{1.164825in}}%
\pgfpathlineto{\pgfqpoint{1.741174in}{1.177839in}}%
\pgfpathlineto{\pgfqpoint{1.732922in}{1.126557in}}%
\pgfpathlineto{\pgfqpoint{1.650218in}{1.139006in}}%
\pgfpathlineto{\pgfqpoint{1.669651in}{1.266224in}}%
\pgfpathlineto{\pgfqpoint{1.693706in}{1.421119in}}%
\pgfpathlineto{\pgfqpoint{1.715096in}{1.556058in}}%
\pgfpathlineto{\pgfqpoint{1.733669in}{1.676002in}}%
\pgfpathlineto{\pgfqpoint{1.751527in}{1.790989in}}%
\pgfusepath{stroke}%
\end{pgfscope}%
\begin{pgfscope}%
\pgfpathrectangle{\pgfqpoint{0.100000in}{0.100000in}}{\pgfqpoint{5.307240in}{3.397500in}}%
\pgfusepath{clip}%
\pgfsetbuttcap%
\pgfsetroundjoin%
\pgfsetlinewidth{0.501875pt}%
\definecolor{currentstroke}{rgb}{0.827451,0.827451,0.827451}%
\pgfsetstrokecolor{currentstroke}%
\pgfsetdash{}{0pt}%
\pgfpathmoveto{\pgfqpoint{4.055423in}{1.045630in}}%
\pgfpathlineto{\pgfqpoint{3.917188in}{1.030276in}}%
\pgfpathlineto{\pgfqpoint{3.795337in}{1.020817in}}%
\pgfpathlineto{\pgfqpoint{3.793813in}{1.005884in}}%
\pgfpathlineto{\pgfqpoint{3.805005in}{0.991671in}}%
\pgfpathlineto{\pgfqpoint{3.818666in}{0.983322in}}%
\pgfpathlineto{\pgfqpoint{3.818459in}{0.961337in}}%
\pgfpathlineto{\pgfqpoint{3.814840in}{0.946648in}}%
\pgfpathlineto{\pgfqpoint{3.802782in}{0.936063in}}%
\pgfpathlineto{\pgfqpoint{3.785977in}{0.937183in}}%
\pgfpathlineto{\pgfqpoint{3.770088in}{0.950299in}}%
\pgfpathlineto{\pgfqpoint{3.767256in}{0.973639in}}%
\pgfpathlineto{\pgfqpoint{3.755422in}{0.987213in}}%
\pgfpathlineto{\pgfqpoint{3.747367in}{0.938631in}}%
\pgfpathlineto{\pgfqpoint{3.719996in}{0.943247in}}%
\pgfpathlineto{\pgfqpoint{3.710478in}{1.028201in}}%
\pgfpathlineto{\pgfqpoint{3.700171in}{1.117732in}}%
\pgfpathlineto{\pgfqpoint{3.703999in}{1.246905in}}%
\pgfpathlineto{\pgfqpoint{3.706375in}{1.335545in}}%
\pgfpathlineto{\pgfqpoint{3.711429in}{1.470788in}}%
\pgfpathlineto{\pgfqpoint{3.700611in}{1.482689in}}%
\pgfpathlineto{\pgfqpoint{3.720297in}{1.485543in}}%
\pgfpathlineto{\pgfqpoint{3.834978in}{1.492959in}}%
\pgfpathlineto{\pgfqpoint{3.946559in}{1.502994in}}%
\pgfpathlineto{\pgfqpoint{3.975914in}{1.400165in}}%
\pgfpathlineto{\pgfqpoint{3.995936in}{1.324689in}}%
\pgfpathlineto{\pgfqpoint{4.013881in}{1.261502in}}%
\pgfpathlineto{\pgfqpoint{4.024260in}{1.236058in}}%
\pgfpathlineto{\pgfqpoint{4.040602in}{1.212381in}}%
\pgfpathlineto{\pgfqpoint{4.037963in}{1.200331in}}%
\pgfpathlineto{\pgfqpoint{4.049637in}{1.194461in}}%
\pgfpathlineto{\pgfqpoint{4.035841in}{1.176196in}}%
\pgfpathlineto{\pgfqpoint{4.031182in}{1.143668in}}%
\pgfpathlineto{\pgfqpoint{4.044664in}{1.105624in}}%
\pgfpathlineto{\pgfqpoint{4.042791in}{1.067341in}}%
\pgfpathlineto{\pgfqpoint{4.055423in}{1.045630in}}%
\pgfusepath{stroke}%
\end{pgfscope}%
\begin{pgfscope}%
\pgfpathrectangle{\pgfqpoint{0.100000in}{0.100000in}}{\pgfqpoint{5.307240in}{3.397500in}}%
\pgfusepath{clip}%
\pgfsetbuttcap%
\pgfsetroundjoin%
\pgfsetlinewidth{0.501875pt}%
\definecolor{currentstroke}{rgb}{0.827451,0.827451,0.827451}%
\pgfsetstrokecolor{currentstroke}%
\pgfsetdash{}{0pt}%
\pgfpathmoveto{\pgfqpoint{3.719996in}{0.943247in}}%
\pgfpathlineto{\pgfqpoint{3.691909in}{0.935211in}}%
\pgfpathlineto{\pgfqpoint{3.670973in}{0.940371in}}%
\pgfpathlineto{\pgfqpoint{3.632093in}{0.928004in}}%
\pgfpathlineto{\pgfqpoint{3.602770in}{0.912077in}}%
\pgfpathlineto{\pgfqpoint{3.590738in}{0.940857in}}%
\pgfpathlineto{\pgfqpoint{3.578335in}{0.952921in}}%
\pgfpathlineto{\pgfqpoint{3.572352in}{0.967356in}}%
\pgfpathlineto{\pgfqpoint{3.581122in}{1.006428in}}%
\pgfpathlineto{\pgfqpoint{3.498023in}{1.001311in}}%
\pgfpathlineto{\pgfqpoint{3.390269in}{0.996639in}}%
\pgfpathlineto{\pgfqpoint{3.396679in}{1.006365in}}%
\pgfpathlineto{\pgfqpoint{3.388755in}{1.024734in}}%
\pgfpathlineto{\pgfqpoint{3.396651in}{1.028477in}}%
\pgfpathlineto{\pgfqpoint{3.394899in}{1.046125in}}%
\pgfpathlineto{\pgfqpoint{3.408662in}{1.063240in}}%
\pgfpathlineto{\pgfqpoint{3.416191in}{1.096465in}}%
\pgfpathlineto{\pgfqpoint{3.441379in}{1.118355in}}%
\pgfpathlineto{\pgfqpoint{3.432043in}{1.139613in}}%
\pgfpathlineto{\pgfqpoint{3.449904in}{1.151850in}}%
\pgfpathlineto{\pgfqpoint{3.434498in}{1.174003in}}%
\pgfpathlineto{\pgfqpoint{3.423347in}{1.224717in}}%
\pgfpathlineto{\pgfqpoint{3.427576in}{1.233925in}}%
\pgfpathlineto{\pgfqpoint{3.434253in}{1.251588in}}%
\pgfpathlineto{\pgfqpoint{3.428038in}{1.270125in}}%
\pgfpathlineto{\pgfqpoint{3.431077in}{1.278559in}}%
\pgfpathlineto{\pgfqpoint{3.420144in}{1.310420in}}%
\pgfpathlineto{\pgfqpoint{3.452420in}{1.367628in}}%
\pgfpathlineto{\pgfqpoint{3.453921in}{1.382855in}}%
\pgfpathlineto{\pgfqpoint{3.469468in}{1.393926in}}%
\pgfpathlineto{\pgfqpoint{3.486057in}{1.430478in}}%
\pgfpathlineto{\pgfqpoint{3.485269in}{1.445335in}}%
\pgfpathlineto{\pgfqpoint{3.505931in}{1.460522in}}%
\pgfpathlineto{\pgfqpoint{3.499485in}{1.469518in}}%
\pgfpathlineto{\pgfqpoint{3.623304in}{1.476743in}}%
\pgfpathlineto{\pgfqpoint{3.700611in}{1.482689in}}%
\pgfpathlineto{\pgfqpoint{3.711429in}{1.470788in}}%
\pgfpathlineto{\pgfqpoint{3.706375in}{1.335545in}}%
\pgfpathlineto{\pgfqpoint{3.703999in}{1.246905in}}%
\pgfpathlineto{\pgfqpoint{3.700171in}{1.117732in}}%
\pgfpathlineto{\pgfqpoint{3.710478in}{1.028201in}}%
\pgfpathlineto{\pgfqpoint{3.719996in}{0.943247in}}%
\pgfusepath{stroke}%
\end{pgfscope}%
\begin{pgfscope}%
\pgfpathrectangle{\pgfqpoint{0.100000in}{0.100000in}}{\pgfqpoint{5.307240in}{3.397500in}}%
\pgfusepath{clip}%
\pgfsetbuttcap%
\pgfsetroundjoin%
\pgfsetlinewidth{0.501875pt}%
\definecolor{currentstroke}{rgb}{0.827451,0.827451,0.827451}%
\pgfsetstrokecolor{currentstroke}%
\pgfsetdash{}{0pt}%
\pgfpathmoveto{\pgfqpoint{3.946559in}{1.502994in}}%
\pgfpathlineto{\pgfqpoint{4.068531in}{1.516389in}}%
\pgfpathlineto{\pgfqpoint{4.141921in}{1.525432in}}%
\pgfpathlineto{\pgfqpoint{4.183719in}{1.532092in}}%
\pgfpathlineto{\pgfqpoint{4.166404in}{1.504783in}}%
\pgfpathlineto{\pgfqpoint{4.166415in}{1.491907in}}%
\pgfpathlineto{\pgfqpoint{4.184238in}{1.484964in}}%
\pgfpathlineto{\pgfqpoint{4.196358in}{1.473754in}}%
\pgfpathlineto{\pgfqpoint{4.214675in}{1.472368in}}%
\pgfpathlineto{\pgfqpoint{4.231795in}{1.440811in}}%
\pgfpathlineto{\pgfqpoint{4.250365in}{1.418557in}}%
\pgfpathlineto{\pgfqpoint{4.283347in}{1.399854in}}%
\pgfpathlineto{\pgfqpoint{4.292720in}{1.384936in}}%
\pgfpathlineto{\pgfqpoint{4.322131in}{1.368802in}}%
\pgfpathlineto{\pgfqpoint{4.324342in}{1.357103in}}%
\pgfpathlineto{\pgfqpoint{4.348201in}{1.333838in}}%
\pgfpathlineto{\pgfqpoint{4.366477in}{1.327214in}}%
\pgfpathlineto{\pgfqpoint{4.379418in}{1.305246in}}%
\pgfpathlineto{\pgfqpoint{4.385147in}{1.280555in}}%
\pgfpathlineto{\pgfqpoint{4.404130in}{1.271010in}}%
\pgfpathlineto{\pgfqpoint{4.419412in}{1.244469in}}%
\pgfpathlineto{\pgfqpoint{4.424340in}{1.224950in}}%
\pgfpathlineto{\pgfqpoint{4.446734in}{1.216642in}}%
\pgfpathlineto{\pgfqpoint{4.427943in}{1.180661in}}%
\pgfpathlineto{\pgfqpoint{4.420871in}{1.159398in}}%
\pgfpathlineto{\pgfqpoint{4.425485in}{1.149431in}}%
\pgfpathlineto{\pgfqpoint{4.417429in}{1.132866in}}%
\pgfpathlineto{\pgfqpoint{4.414004in}{1.109965in}}%
\pgfpathlineto{\pgfqpoint{4.399677in}{1.105699in}}%
\pgfpathlineto{\pgfqpoint{4.407242in}{1.084731in}}%
\pgfpathlineto{\pgfqpoint{4.406760in}{1.058129in}}%
\pgfpathlineto{\pgfqpoint{4.381294in}{1.059127in}}%
\pgfpathlineto{\pgfqpoint{4.363553in}{1.063980in}}%
\pgfpathlineto{\pgfqpoint{4.355877in}{1.054957in}}%
\pgfpathlineto{\pgfqpoint{4.361605in}{1.033643in}}%
\pgfpathlineto{\pgfqpoint{4.360359in}{1.008863in}}%
\pgfpathlineto{\pgfqpoint{4.349167in}{1.006967in}}%
\pgfpathlineto{\pgfqpoint{4.340113in}{1.030110in}}%
\pgfpathlineto{\pgfqpoint{4.247924in}{1.023966in}}%
\pgfpathlineto{\pgfqpoint{4.131680in}{1.017624in}}%
\pgfpathlineto{\pgfqpoint{4.073054in}{1.013497in}}%
\pgfpathlineto{\pgfqpoint{4.055423in}{1.045630in}}%
\pgfpathlineto{\pgfqpoint{4.042791in}{1.067341in}}%
\pgfpathlineto{\pgfqpoint{4.044664in}{1.105624in}}%
\pgfpathlineto{\pgfqpoint{4.031182in}{1.143668in}}%
\pgfpathlineto{\pgfqpoint{4.035841in}{1.176196in}}%
\pgfpathlineto{\pgfqpoint{4.049637in}{1.194461in}}%
\pgfpathlineto{\pgfqpoint{4.037963in}{1.200331in}}%
\pgfpathlineto{\pgfqpoint{4.040602in}{1.212381in}}%
\pgfpathlineto{\pgfqpoint{4.024260in}{1.236058in}}%
\pgfpathlineto{\pgfqpoint{4.013881in}{1.261502in}}%
\pgfpathlineto{\pgfqpoint{3.995936in}{1.324689in}}%
\pgfpathlineto{\pgfqpoint{3.975914in}{1.400165in}}%
\pgfpathlineto{\pgfqpoint{3.946559in}{1.502994in}}%
\pgfusepath{stroke}%
\end{pgfscope}%
\begin{pgfscope}%
\pgfpathrectangle{\pgfqpoint{0.100000in}{0.100000in}}{\pgfqpoint{5.307240in}{3.397500in}}%
\pgfusepath{clip}%
\pgfsetbuttcap%
\pgfsetroundjoin%
\pgfsetlinewidth{0.501875pt}%
\definecolor{currentstroke}{rgb}{0.827451,0.827451,0.827451}%
\pgfsetstrokecolor{currentstroke}%
\pgfsetdash{}{0pt}%
\pgfpathmoveto{\pgfqpoint{4.183719in}{1.532092in}}%
\pgfpathlineto{\pgfqpoint{4.232085in}{1.556151in}}%
\pgfpathlineto{\pgfqpoint{4.258871in}{1.565236in}}%
\pgfpathlineto{\pgfqpoint{4.375411in}{1.577345in}}%
\pgfpathlineto{\pgfqpoint{4.387753in}{1.573335in}}%
\pgfpathlineto{\pgfqpoint{4.404056in}{1.556947in}}%
\pgfpathlineto{\pgfqpoint{4.404955in}{1.542395in}}%
\pgfpathlineto{\pgfqpoint{4.512216in}{1.558155in}}%
\pgfpathlineto{\pgfqpoint{4.634014in}{1.470844in}}%
\pgfpathlineto{\pgfqpoint{4.611173in}{1.447038in}}%
\pgfpathlineto{\pgfqpoint{4.579831in}{1.391749in}}%
\pgfpathlineto{\pgfqpoint{4.588753in}{1.379867in}}%
\pgfpathlineto{\pgfqpoint{4.572033in}{1.356804in}}%
\pgfpathlineto{\pgfqpoint{4.555402in}{1.354193in}}%
\pgfpathlineto{\pgfqpoint{4.555341in}{1.340334in}}%
\pgfpathlineto{\pgfqpoint{4.543310in}{1.325832in}}%
\pgfpathlineto{\pgfqpoint{4.528338in}{1.322891in}}%
\pgfpathlineto{\pgfqpoint{4.531605in}{1.310005in}}%
\pgfpathlineto{\pgfqpoint{4.523251in}{1.300113in}}%
\pgfpathlineto{\pgfqpoint{4.503265in}{1.291568in}}%
\pgfpathlineto{\pgfqpoint{4.478906in}{1.272129in}}%
\pgfpathlineto{\pgfqpoint{4.483505in}{1.259553in}}%
\pgfpathlineto{\pgfqpoint{4.468174in}{1.251657in}}%
\pgfpathlineto{\pgfqpoint{4.455244in}{1.259960in}}%
\pgfpathlineto{\pgfqpoint{4.445789in}{1.223864in}}%
\pgfpathlineto{\pgfqpoint{4.424340in}{1.224950in}}%
\pgfpathlineto{\pgfqpoint{4.419412in}{1.244469in}}%
\pgfpathlineto{\pgfqpoint{4.404130in}{1.271010in}}%
\pgfpathlineto{\pgfqpoint{4.385147in}{1.280555in}}%
\pgfpathlineto{\pgfqpoint{4.379418in}{1.305246in}}%
\pgfpathlineto{\pgfqpoint{4.366477in}{1.327214in}}%
\pgfpathlineto{\pgfqpoint{4.348201in}{1.333838in}}%
\pgfpathlineto{\pgfqpoint{4.324342in}{1.357103in}}%
\pgfpathlineto{\pgfqpoint{4.322131in}{1.368802in}}%
\pgfpathlineto{\pgfqpoint{4.292720in}{1.384936in}}%
\pgfpathlineto{\pgfqpoint{4.283347in}{1.399854in}}%
\pgfpathlineto{\pgfqpoint{4.250365in}{1.418557in}}%
\pgfpathlineto{\pgfqpoint{4.231795in}{1.440811in}}%
\pgfpathlineto{\pgfqpoint{4.214675in}{1.472368in}}%
\pgfpathlineto{\pgfqpoint{4.196358in}{1.473754in}}%
\pgfpathlineto{\pgfqpoint{4.184238in}{1.484964in}}%
\pgfpathlineto{\pgfqpoint{4.166415in}{1.491907in}}%
\pgfpathlineto{\pgfqpoint{4.166404in}{1.504783in}}%
\pgfpathlineto{\pgfqpoint{4.183719in}{1.532092in}}%
\pgfusepath{stroke}%
\end{pgfscope}%
\begin{pgfscope}%
\pgfpathrectangle{\pgfqpoint{0.100000in}{0.100000in}}{\pgfqpoint{5.307240in}{3.397500in}}%
\pgfusepath{clip}%
\pgfsetbuttcap%
\pgfsetroundjoin%
\pgfsetlinewidth{0.501875pt}%
\definecolor{currentstroke}{rgb}{0.827451,0.827451,0.827451}%
\pgfsetstrokecolor{currentstroke}%
\pgfsetdash{}{0pt}%
\pgfpathmoveto{\pgfqpoint{3.087335in}{1.631367in}}%
\pgfpathlineto{\pgfqpoint{3.158409in}{1.632007in}}%
\pgfpathlineto{\pgfqpoint{3.252439in}{1.633721in}}%
\pgfpathlineto{\pgfqpoint{3.413115in}{1.638822in}}%
\pgfpathlineto{\pgfqpoint{3.505017in}{1.643681in}}%
\pgfpathlineto{\pgfqpoint{3.514874in}{1.631480in}}%
\pgfpathlineto{\pgfqpoint{3.514257in}{1.618593in}}%
\pgfpathlineto{\pgfqpoint{3.491995in}{1.596362in}}%
\pgfpathlineto{\pgfqpoint{3.486612in}{1.584188in}}%
\pgfpathlineto{\pgfqpoint{3.548359in}{1.588755in}}%
\pgfpathlineto{\pgfqpoint{3.548325in}{1.566264in}}%
\pgfpathlineto{\pgfqpoint{3.528516in}{1.556652in}}%
\pgfpathlineto{\pgfqpoint{3.528796in}{1.541369in}}%
\pgfpathlineto{\pgfqpoint{3.521357in}{1.530385in}}%
\pgfpathlineto{\pgfqpoint{3.516487in}{1.507052in}}%
\pgfpathlineto{\pgfqpoint{3.521514in}{1.487804in}}%
\pgfpathlineto{\pgfqpoint{3.499485in}{1.469518in}}%
\pgfpathlineto{\pgfqpoint{3.505931in}{1.460522in}}%
\pgfpathlineto{\pgfqpoint{3.485269in}{1.445335in}}%
\pgfpathlineto{\pgfqpoint{3.486057in}{1.430478in}}%
\pgfpathlineto{\pgfqpoint{3.469468in}{1.393926in}}%
\pgfpathlineto{\pgfqpoint{3.453921in}{1.382855in}}%
\pgfpathlineto{\pgfqpoint{3.452420in}{1.367628in}}%
\pgfpathlineto{\pgfqpoint{3.420144in}{1.310420in}}%
\pgfpathlineto{\pgfqpoint{3.431077in}{1.278559in}}%
\pgfpathlineto{\pgfqpoint{3.428038in}{1.270125in}}%
\pgfpathlineto{\pgfqpoint{3.434253in}{1.251588in}}%
\pgfpathlineto{\pgfqpoint{3.427576in}{1.233925in}}%
\pgfpathlineto{\pgfqpoint{3.339317in}{1.230282in}}%
\pgfpathlineto{\pgfqpoint{3.224741in}{1.228394in}}%
\pgfpathlineto{\pgfqpoint{3.145713in}{1.227681in}}%
\pgfpathlineto{\pgfqpoint{3.145328in}{1.289906in}}%
\pgfpathlineto{\pgfqpoint{3.125702in}{1.293983in}}%
\pgfpathlineto{\pgfqpoint{3.112750in}{1.288659in}}%
\pgfpathlineto{\pgfqpoint{3.102403in}{1.298435in}}%
\pgfpathlineto{\pgfqpoint{3.103471in}{1.364349in}}%
\pgfpathlineto{\pgfqpoint{3.105786in}{1.504693in}}%
\pgfpathlineto{\pgfqpoint{3.094561in}{1.586874in}}%
\pgfpathlineto{\pgfqpoint{3.087335in}{1.631367in}}%
\pgfusepath{stroke}%
\end{pgfscope}%
\begin{pgfscope}%
\pgfpathrectangle{\pgfqpoint{0.100000in}{0.100000in}}{\pgfqpoint{5.307240in}{3.397500in}}%
\pgfusepath{clip}%
\pgfsetbuttcap%
\pgfsetroundjoin%
\pgfsetlinewidth{0.501875pt}%
\definecolor{currentstroke}{rgb}{0.827451,0.827451,0.827451}%
\pgfsetstrokecolor{currentstroke}%
\pgfsetdash{}{0pt}%
\pgfpathmoveto{\pgfqpoint{3.145713in}{1.227681in}}%
\pgfpathlineto{\pgfqpoint{3.224741in}{1.228394in}}%
\pgfpathlineto{\pgfqpoint{3.339317in}{1.230282in}}%
\pgfpathlineto{\pgfqpoint{3.427576in}{1.233925in}}%
\pgfpathlineto{\pgfqpoint{3.423347in}{1.224717in}}%
\pgfpathlineto{\pgfqpoint{3.434498in}{1.174003in}}%
\pgfpathlineto{\pgfqpoint{3.449904in}{1.151850in}}%
\pgfpathlineto{\pgfqpoint{3.432043in}{1.139613in}}%
\pgfpathlineto{\pgfqpoint{3.441379in}{1.118355in}}%
\pgfpathlineto{\pgfqpoint{3.416191in}{1.096465in}}%
\pgfpathlineto{\pgfqpoint{3.408662in}{1.063240in}}%
\pgfpathlineto{\pgfqpoint{3.394899in}{1.046125in}}%
\pgfpathlineto{\pgfqpoint{3.396651in}{1.028477in}}%
\pgfpathlineto{\pgfqpoint{3.388755in}{1.024734in}}%
\pgfpathlineto{\pgfqpoint{3.396679in}{1.006365in}}%
\pgfpathlineto{\pgfqpoint{3.390269in}{0.996639in}}%
\pgfpathlineto{\pgfqpoint{3.498023in}{1.001311in}}%
\pgfpathlineto{\pgfqpoint{3.581122in}{1.006428in}}%
\pgfpathlineto{\pgfqpoint{3.572352in}{0.967356in}}%
\pgfpathlineto{\pgfqpoint{3.578335in}{0.952921in}}%
\pgfpathlineto{\pgfqpoint{3.590738in}{0.940857in}}%
\pgfpathlineto{\pgfqpoint{3.602770in}{0.912077in}}%
\pgfpathlineto{\pgfqpoint{3.564767in}{0.918707in}}%
\pgfpathlineto{\pgfqpoint{3.550748in}{0.929608in}}%
\pgfpathlineto{\pgfqpoint{3.534045in}{0.930127in}}%
\pgfpathlineto{\pgfqpoint{3.516485in}{0.906257in}}%
\pgfpathlineto{\pgfqpoint{3.519985in}{0.895387in}}%
\pgfpathlineto{\pgfqpoint{3.576083in}{0.888811in}}%
\pgfpathlineto{\pgfqpoint{3.590769in}{0.876316in}}%
\pgfpathlineto{\pgfqpoint{3.614667in}{0.869888in}}%
\pgfpathlineto{\pgfqpoint{3.604230in}{0.855096in}}%
\pgfpathlineto{\pgfqpoint{3.586645in}{0.842195in}}%
\pgfpathlineto{\pgfqpoint{3.633060in}{0.813183in}}%
\pgfpathlineto{\pgfqpoint{3.654681in}{0.808650in}}%
\pgfpathlineto{\pgfqpoint{3.663672in}{0.784937in}}%
\pgfpathlineto{\pgfqpoint{3.643316in}{0.780574in}}%
\pgfpathlineto{\pgfqpoint{3.605056in}{0.810400in}}%
\pgfpathlineto{\pgfqpoint{3.590097in}{0.814545in}}%
\pgfpathlineto{\pgfqpoint{3.582858in}{0.826328in}}%
\pgfpathlineto{\pgfqpoint{3.560669in}{0.821492in}}%
\pgfpathlineto{\pgfqpoint{3.553722in}{0.806309in}}%
\pgfpathlineto{\pgfqpoint{3.558122in}{0.789392in}}%
\pgfpathlineto{\pgfqpoint{3.543184in}{0.779408in}}%
\pgfpathlineto{\pgfqpoint{3.529551in}{0.804003in}}%
\pgfpathlineto{\pgfqpoint{3.505673in}{0.796670in}}%
\pgfpathlineto{\pgfqpoint{3.496726in}{0.781995in}}%
\pgfpathlineto{\pgfqpoint{3.479768in}{0.786169in}}%
\pgfpathlineto{\pgfqpoint{3.468927in}{0.804701in}}%
\pgfpathlineto{\pgfqpoint{3.440163in}{0.810821in}}%
\pgfpathlineto{\pgfqpoint{3.434732in}{0.820464in}}%
\pgfpathlineto{\pgfqpoint{3.404763in}{0.837262in}}%
\pgfpathlineto{\pgfqpoint{3.397324in}{0.851949in}}%
\pgfpathlineto{\pgfqpoint{3.372222in}{0.845947in}}%
\pgfpathlineto{\pgfqpoint{3.375411in}{0.859393in}}%
\pgfpathlineto{\pgfqpoint{3.344211in}{0.845599in}}%
\pgfpathlineto{\pgfqpoint{3.352592in}{0.831288in}}%
\pgfpathlineto{\pgfqpoint{3.328498in}{0.822826in}}%
\pgfpathlineto{\pgfqpoint{3.296585in}{0.827477in}}%
\pgfpathlineto{\pgfqpoint{3.231995in}{0.849601in}}%
\pgfpathlineto{\pgfqpoint{3.182156in}{0.845210in}}%
\pgfpathlineto{\pgfqpoint{3.177836in}{0.874271in}}%
\pgfpathlineto{\pgfqpoint{3.184095in}{0.887532in}}%
\pgfpathlineto{\pgfqpoint{3.183560in}{0.908631in}}%
\pgfpathlineto{\pgfqpoint{3.177402in}{0.913658in}}%
\pgfpathlineto{\pgfqpoint{3.179489in}{0.937742in}}%
\pgfpathlineto{\pgfqpoint{3.197641in}{0.971238in}}%
\pgfpathlineto{\pgfqpoint{3.199963in}{0.983895in}}%
\pgfpathlineto{\pgfqpoint{3.196971in}{1.013789in}}%
\pgfpathlineto{\pgfqpoint{3.183618in}{1.027869in}}%
\pgfpathlineto{\pgfqpoint{3.176645in}{1.054160in}}%
\pgfpathlineto{\pgfqpoint{3.168066in}{1.060197in}}%
\pgfpathlineto{\pgfqpoint{3.171906in}{1.074454in}}%
\pgfpathlineto{\pgfqpoint{3.160981in}{1.095765in}}%
\pgfpathlineto{\pgfqpoint{3.147350in}{1.107330in}}%
\pgfpathlineto{\pgfqpoint{3.145713in}{1.227681in}}%
\pgfusepath{stroke}%
\end{pgfscope}%
\begin{pgfscope}%
\pgfpathrectangle{\pgfqpoint{0.100000in}{0.100000in}}{\pgfqpoint{5.307240in}{3.397500in}}%
\pgfusepath{clip}%
\pgfsetbuttcap%
\pgfsetroundjoin%
\pgfsetlinewidth{0.501875pt}%
\definecolor{currentstroke}{rgb}{0.827451,0.827451,0.827451}%
\pgfsetstrokecolor{currentstroke}%
\pgfsetdash{}{0pt}%
\pgfpathmoveto{\pgfqpoint{3.357074in}{0.830192in}}%
\pgfpathlineto{\pgfqpoint{3.368507in}{0.836992in}}%
\pgfpathlineto{\pgfqpoint{3.382384in}{0.828952in}}%
\pgfpathlineto{\pgfqpoint{3.374645in}{0.817890in}}%
\pgfpathlineto{\pgfqpoint{3.357074in}{0.830192in}}%
\pgfusepath{stroke}%
\end{pgfscope}%
\begin{pgfscope}%
\pgfpathrectangle{\pgfqpoint{0.100000in}{0.100000in}}{\pgfqpoint{5.307240in}{3.397500in}}%
\pgfusepath{clip}%
\pgfsetbuttcap%
\pgfsetroundjoin%
\pgfsetlinewidth{0.501875pt}%
\definecolor{currentstroke}{rgb}{0.827451,0.827451,0.827451}%
\pgfsetstrokecolor{currentstroke}%
\pgfsetdash{}{0pt}%
\pgfpathmoveto{\pgfqpoint{3.818459in}{0.961337in}}%
\pgfpathlineto{\pgfqpoint{3.818666in}{0.983322in}}%
\pgfpathlineto{\pgfqpoint{3.805005in}{0.991671in}}%
\pgfpathlineto{\pgfqpoint{3.793813in}{1.005884in}}%
\pgfpathlineto{\pgfqpoint{3.795337in}{1.020817in}}%
\pgfpathlineto{\pgfqpoint{3.917188in}{1.030276in}}%
\pgfpathlineto{\pgfqpoint{4.055423in}{1.045630in}}%
\pgfpathlineto{\pgfqpoint{4.073054in}{1.013497in}}%
\pgfpathlineto{\pgfqpoint{4.131680in}{1.017624in}}%
\pgfpathlineto{\pgfqpoint{4.247924in}{1.023966in}}%
\pgfpathlineto{\pgfqpoint{4.340113in}{1.030110in}}%
\pgfpathlineto{\pgfqpoint{4.349167in}{1.006967in}}%
\pgfpathlineto{\pgfqpoint{4.360359in}{1.008863in}}%
\pgfpathlineto{\pgfqpoint{4.361605in}{1.033643in}}%
\pgfpathlineto{\pgfqpoint{4.355877in}{1.054957in}}%
\pgfpathlineto{\pgfqpoint{4.363553in}{1.063980in}}%
\pgfpathlineto{\pgfqpoint{4.381294in}{1.059127in}}%
\pgfpathlineto{\pgfqpoint{4.406760in}{1.058129in}}%
\pgfpathlineto{\pgfqpoint{4.410675in}{1.039087in}}%
\pgfpathlineto{\pgfqpoint{4.418508in}{1.028205in}}%
\pgfpathlineto{\pgfqpoint{4.424626in}{1.004399in}}%
\pgfpathlineto{\pgfqpoint{4.443579in}{0.967557in}}%
\pgfpathlineto{\pgfqpoint{4.443674in}{0.957593in}}%
\pgfpathlineto{\pgfqpoint{4.471687in}{0.914334in}}%
\pgfpathlineto{\pgfqpoint{4.474397in}{0.905397in}}%
\pgfpathlineto{\pgfqpoint{4.516421in}{0.844300in}}%
\pgfpathlineto{\pgfqpoint{4.509763in}{0.843313in}}%
\pgfpathlineto{\pgfqpoint{4.527480in}{0.799856in}}%
\pgfpathlineto{\pgfqpoint{4.576148in}{0.724308in}}%
\pgfpathlineto{\pgfqpoint{4.604405in}{0.668761in}}%
\pgfpathlineto{\pgfqpoint{4.613850in}{0.659036in}}%
\pgfpathlineto{\pgfqpoint{4.630033in}{0.624170in}}%
\pgfpathlineto{\pgfqpoint{4.635647in}{0.568341in}}%
\pgfpathlineto{\pgfqpoint{4.637881in}{0.526596in}}%
\pgfpathlineto{\pgfqpoint{4.635183in}{0.499868in}}%
\pgfpathlineto{\pgfqpoint{4.626506in}{0.480761in}}%
\pgfpathlineto{\pgfqpoint{4.630537in}{0.455705in}}%
\pgfpathlineto{\pgfqpoint{4.621195in}{0.435872in}}%
\pgfpathlineto{\pgfqpoint{4.593457in}{0.419609in}}%
\pgfpathlineto{\pgfqpoint{4.575417in}{0.420796in}}%
\pgfpathlineto{\pgfqpoint{4.563702in}{0.412295in}}%
\pgfpathlineto{\pgfqpoint{4.560346in}{0.434994in}}%
\pgfpathlineto{\pgfqpoint{4.540954in}{0.440985in}}%
\pgfpathlineto{\pgfqpoint{4.523570in}{0.472657in}}%
\pgfpathlineto{\pgfqpoint{4.521607in}{0.487080in}}%
\pgfpathlineto{\pgfqpoint{4.490496in}{0.495839in}}%
\pgfpathlineto{\pgfqpoint{4.470448in}{0.493942in}}%
\pgfpathlineto{\pgfqpoint{4.459104in}{0.514870in}}%
\pgfpathlineto{\pgfqpoint{4.446117in}{0.552485in}}%
\pgfpathlineto{\pgfqpoint{4.428094in}{0.560110in}}%
\pgfpathlineto{\pgfqpoint{4.418324in}{0.581665in}}%
\pgfpathlineto{\pgfqpoint{4.394428in}{0.594369in}}%
\pgfpathlineto{\pgfqpoint{4.380616in}{0.610260in}}%
\pgfpathlineto{\pgfqpoint{4.356789in}{0.653460in}}%
\pgfpathlineto{\pgfqpoint{4.353987in}{0.683867in}}%
\pgfpathlineto{\pgfqpoint{4.366971in}{0.703321in}}%
\pgfpathlineto{\pgfqpoint{4.365673in}{0.716840in}}%
\pgfpathlineto{\pgfqpoint{4.350668in}{0.718285in}}%
\pgfpathlineto{\pgfqpoint{4.338133in}{0.727688in}}%
\pgfpathlineto{\pgfqpoint{4.331268in}{0.716206in}}%
\pgfpathlineto{\pgfqpoint{4.343299in}{0.706850in}}%
\pgfpathlineto{\pgfqpoint{4.334542in}{0.690026in}}%
\pgfpathlineto{\pgfqpoint{4.320376in}{0.703990in}}%
\pgfpathlineto{\pgfqpoint{4.322019in}{0.742712in}}%
\pgfpathlineto{\pgfqpoint{4.328852in}{0.774127in}}%
\pgfpathlineto{\pgfqpoint{4.325340in}{0.828031in}}%
\pgfpathlineto{\pgfqpoint{4.311219in}{0.840870in}}%
\pgfpathlineto{\pgfqpoint{4.304118in}{0.857346in}}%
\pgfpathlineto{\pgfqpoint{4.279793in}{0.856979in}}%
\pgfpathlineto{\pgfqpoint{4.255748in}{0.884088in}}%
\pgfpathlineto{\pgfqpoint{4.239626in}{0.892213in}}%
\pgfpathlineto{\pgfqpoint{4.234845in}{0.909381in}}%
\pgfpathlineto{\pgfqpoint{4.219076in}{0.915411in}}%
\pgfpathlineto{\pgfqpoint{4.206007in}{0.934366in}}%
\pgfpathlineto{\pgfqpoint{4.171484in}{0.949839in}}%
\pgfpathlineto{\pgfqpoint{4.144645in}{0.950234in}}%
\pgfpathlineto{\pgfqpoint{4.132961in}{0.944291in}}%
\pgfpathlineto{\pgfqpoint{4.135864in}{0.925740in}}%
\pgfpathlineto{\pgfqpoint{4.123681in}{0.926619in}}%
\pgfpathlineto{\pgfqpoint{4.086202in}{0.900682in}}%
\pgfpathlineto{\pgfqpoint{4.041184in}{0.890401in}}%
\pgfpathlineto{\pgfqpoint{4.040445in}{0.903121in}}%
\pgfpathlineto{\pgfqpoint{4.030469in}{0.915553in}}%
\pgfpathlineto{\pgfqpoint{4.003701in}{0.932716in}}%
\pgfpathlineto{\pgfqpoint{3.965279in}{0.950312in}}%
\pgfpathlineto{\pgfqpoint{3.923546in}{0.959627in}}%
\pgfpathlineto{\pgfqpoint{3.915713in}{0.972295in}}%
\pgfpathlineto{\pgfqpoint{3.900665in}{0.961730in}}%
\pgfpathlineto{\pgfqpoint{3.882546in}{0.959399in}}%
\pgfpathlineto{\pgfqpoint{3.819475in}{0.942781in}}%
\pgfpathlineto{\pgfqpoint{3.818459in}{0.961337in}}%
\pgfusepath{stroke}%
\end{pgfscope}%
\begin{pgfscope}%
\pgfpathrectangle{\pgfqpoint{0.100000in}{0.100000in}}{\pgfqpoint{5.307240in}{3.397500in}}%
\pgfusepath{clip}%
\pgfsetbuttcap%
\pgfsetroundjoin%
\pgfsetlinewidth{0.501875pt}%
\definecolor{currentstroke}{rgb}{0.827451,0.827451,0.827451}%
\pgfsetstrokecolor{currentstroke}%
\pgfsetdash{}{0pt}%
\pgfpathmoveto{\pgfqpoint{4.522545in}{0.845169in}}%
\pgfpathlineto{\pgfqpoint{4.538863in}{0.826186in}}%
\pgfpathlineto{\pgfqpoint{4.520013in}{0.824992in}}%
\pgfpathlineto{\pgfqpoint{4.522545in}{0.845169in}}%
\pgfusepath{stroke}%
\end{pgfscope}%
\begin{pgfscope}%
\pgfpathrectangle{\pgfqpoint{0.100000in}{0.100000in}}{\pgfqpoint{5.307240in}{3.397500in}}%
\pgfusepath{clip}%
\pgfsetbuttcap%
\pgfsetroundjoin%
\pgfsetlinewidth{0.501875pt}%
\definecolor{currentstroke}{rgb}{0.827451,0.827451,0.827451}%
\pgfsetstrokecolor{currentstroke}%
\pgfsetdash{}{0pt}%
\pgfpathmoveto{\pgfqpoint{3.561907in}{3.012582in}}%
\pgfpathlineto{\pgfqpoint{3.553053in}{2.995350in}}%
\pgfpathlineto{\pgfqpoint{3.531931in}{2.985293in}}%
\pgfpathlineto{\pgfqpoint{3.522782in}{2.971756in}}%
\pgfpathlineto{\pgfqpoint{3.512066in}{2.981506in}}%
\pgfpathlineto{\pgfqpoint{3.561907in}{3.012582in}}%
\pgfusepath{stroke}%
\end{pgfscope}%
\begin{pgfscope}%
\pgfpathrectangle{\pgfqpoint{0.100000in}{0.100000in}}{\pgfqpoint{5.307240in}{3.397500in}}%
\pgfusepath{clip}%
\pgfsetbuttcap%
\pgfsetroundjoin%
\pgfsetlinewidth{0.501875pt}%
\definecolor{currentstroke}{rgb}{0.827451,0.827451,0.827451}%
\pgfsetstrokecolor{currentstroke}%
\pgfsetdash{}{0pt}%
\pgfpathmoveto{\pgfqpoint{3.569088in}{2.908874in}}%
\pgfpathlineto{\pgfqpoint{3.590700in}{2.929036in}}%
\pgfpathlineto{\pgfqpoint{3.624035in}{2.934316in}}%
\pgfpathlineto{\pgfqpoint{3.614854in}{2.920294in}}%
\pgfpathlineto{\pgfqpoint{3.592001in}{2.900016in}}%
\pgfpathlineto{\pgfqpoint{3.578640in}{2.873989in}}%
\pgfpathlineto{\pgfqpoint{3.572981in}{2.888106in}}%
\pgfpathlineto{\pgfqpoint{3.562934in}{2.890133in}}%
\pgfpathlineto{\pgfqpoint{3.561925in}{2.902877in}}%
\pgfpathlineto{\pgfqpoint{3.569088in}{2.908874in}}%
\pgfusepath{stroke}%
\end{pgfscope}%
\begin{pgfscope}%
\pgfpathrectangle{\pgfqpoint{0.100000in}{0.100000in}}{\pgfqpoint{5.307240in}{3.397500in}}%
\pgfusepath{clip}%
\pgfsetbuttcap%
\pgfsetroundjoin%
\pgfsetlinewidth{0.501875pt}%
\definecolor{currentstroke}{rgb}{0.827451,0.827451,0.827451}%
\pgfsetstrokecolor{currentstroke}%
\pgfsetdash{}{0pt}%
\pgfpathmoveto{\pgfqpoint{3.655168in}{2.662792in}}%
\pgfpathlineto{\pgfqpoint{3.649412in}{2.669179in}}%
\pgfpathlineto{\pgfqpoint{3.655463in}{2.687161in}}%
\pgfpathlineto{\pgfqpoint{3.637501in}{2.688307in}}%
\pgfpathlineto{\pgfqpoint{3.642262in}{2.703811in}}%
\pgfpathlineto{\pgfqpoint{3.639301in}{2.728503in}}%
\pgfpathlineto{\pgfqpoint{3.623185in}{2.737063in}}%
\pgfpathlineto{\pgfqpoint{3.606318in}{2.754388in}}%
\pgfpathlineto{\pgfqpoint{3.580330in}{2.759461in}}%
\pgfpathlineto{\pgfqpoint{3.555037in}{2.759316in}}%
\pgfpathlineto{\pgfqpoint{3.530161in}{2.771611in}}%
\pgfpathlineto{\pgfqpoint{3.446938in}{2.789533in}}%
\pgfpathlineto{\pgfqpoint{3.437843in}{2.808524in}}%
\pgfpathlineto{\pgfqpoint{3.421621in}{2.815004in}}%
\pgfpathlineto{\pgfqpoint{3.452264in}{2.829543in}}%
\pgfpathlineto{\pgfqpoint{3.469574in}{2.847682in}}%
\pgfpathlineto{\pgfqpoint{3.501816in}{2.852602in}}%
\pgfpathlineto{\pgfqpoint{3.521657in}{2.871078in}}%
\pgfpathlineto{\pgfqpoint{3.532067in}{2.871818in}}%
\pgfpathlineto{\pgfqpoint{3.540011in}{2.884994in}}%
\pgfpathlineto{\pgfqpoint{3.559376in}{2.900604in}}%
\pgfpathlineto{\pgfqpoint{3.561064in}{2.889571in}}%
\pgfpathlineto{\pgfqpoint{3.569790in}{2.887292in}}%
\pgfpathlineto{\pgfqpoint{3.576356in}{2.874135in}}%
\pgfpathlineto{\pgfqpoint{3.577538in}{2.851702in}}%
\pgfpathlineto{\pgfqpoint{3.597270in}{2.865108in}}%
\pgfpathlineto{\pgfqpoint{3.620264in}{2.867907in}}%
\pgfpathlineto{\pgfqpoint{3.639890in}{2.860884in}}%
\pgfpathlineto{\pgfqpoint{3.666508in}{2.824369in}}%
\pgfpathlineto{\pgfqpoint{3.695578in}{2.830205in}}%
\pgfpathlineto{\pgfqpoint{3.707391in}{2.820433in}}%
\pgfpathlineto{\pgfqpoint{3.726389in}{2.819560in}}%
\pgfpathlineto{\pgfqpoint{3.739012in}{2.837086in}}%
\pgfpathlineto{\pgfqpoint{3.762963in}{2.852579in}}%
\pgfpathlineto{\pgfqpoint{3.785983in}{2.857431in}}%
\pgfpathlineto{\pgfqpoint{3.814547in}{2.857937in}}%
\pgfpathlineto{\pgfqpoint{3.835420in}{2.869905in}}%
\pgfpathlineto{\pgfqpoint{3.852455in}{2.864392in}}%
\pgfpathlineto{\pgfqpoint{3.856084in}{2.839080in}}%
\pgfpathlineto{\pgfqpoint{3.892629in}{2.835117in}}%
\pgfpathlineto{\pgfqpoint{3.912480in}{2.846965in}}%
\pgfpathlineto{\pgfqpoint{3.926240in}{2.820273in}}%
\pgfpathlineto{\pgfqpoint{3.952496in}{2.789409in}}%
\pgfpathlineto{\pgfqpoint{3.915747in}{2.789587in}}%
\pgfpathlineto{\pgfqpoint{3.904136in}{2.785772in}}%
\pgfpathlineto{\pgfqpoint{3.888220in}{2.790730in}}%
\pgfpathlineto{\pgfqpoint{3.887149in}{2.769363in}}%
\pgfpathlineto{\pgfqpoint{3.858250in}{2.786061in}}%
\pgfpathlineto{\pgfqpoint{3.821085in}{2.791164in}}%
\pgfpathlineto{\pgfqpoint{3.810853in}{2.774902in}}%
\pgfpathlineto{\pgfqpoint{3.762227in}{2.766994in}}%
\pgfpathlineto{\pgfqpoint{3.756632in}{2.753219in}}%
\pgfpathlineto{\pgfqpoint{3.722794in}{2.749085in}}%
\pgfpathlineto{\pgfqpoint{3.712564in}{2.735081in}}%
\pgfpathlineto{\pgfqpoint{3.694666in}{2.731326in}}%
\pgfpathlineto{\pgfqpoint{3.680370in}{2.698118in}}%
\pgfpathlineto{\pgfqpoint{3.662270in}{2.665957in}}%
\pgfpathlineto{\pgfqpoint{3.655168in}{2.662792in}}%
\pgfusepath{stroke}%
\end{pgfscope}%
\begin{pgfscope}%
\pgfpathrectangle{\pgfqpoint{0.100000in}{0.100000in}}{\pgfqpoint{5.307240in}{3.397500in}}%
\pgfusepath{clip}%
\pgfsetbuttcap%
\pgfsetroundjoin%
\pgfsetlinewidth{0.501875pt}%
\definecolor{currentstroke}{rgb}{0.827451,0.827451,0.827451}%
\pgfsetstrokecolor{currentstroke}%
\pgfsetdash{}{0pt}%
\pgfpathmoveto{\pgfqpoint{3.759344in}{2.276672in}}%
\pgfpathlineto{\pgfqpoint{3.776602in}{2.294837in}}%
\pgfpathlineto{\pgfqpoint{3.784463in}{2.321172in}}%
\pgfpathlineto{\pgfqpoint{3.793813in}{2.336436in}}%
\pgfpathlineto{\pgfqpoint{3.799546in}{2.357210in}}%
\pgfpathlineto{\pgfqpoint{3.801328in}{2.398629in}}%
\pgfpathlineto{\pgfqpoint{3.792696in}{2.438324in}}%
\pgfpathlineto{\pgfqpoint{3.764170in}{2.499109in}}%
\pgfpathlineto{\pgfqpoint{3.771895in}{2.518221in}}%
\pgfpathlineto{\pgfqpoint{3.761832in}{2.544639in}}%
\pgfpathlineto{\pgfqpoint{3.779095in}{2.581173in}}%
\pgfpathlineto{\pgfqpoint{3.776282in}{2.621910in}}%
\pgfpathlineto{\pgfqpoint{3.788306in}{2.627037in}}%
\pgfpathlineto{\pgfqpoint{3.789817in}{2.646420in}}%
\pgfpathlineto{\pgfqpoint{3.811161in}{2.658860in}}%
\pgfpathlineto{\pgfqpoint{3.823232in}{2.656852in}}%
\pgfpathlineto{\pgfqpoint{3.826626in}{2.636065in}}%
\pgfpathlineto{\pgfqpoint{3.836047in}{2.635250in}}%
\pgfpathlineto{\pgfqpoint{3.844644in}{2.665234in}}%
\pgfpathlineto{\pgfqpoint{3.841722in}{2.688555in}}%
\pgfpathlineto{\pgfqpoint{3.847311in}{2.701946in}}%
\pgfpathlineto{\pgfqpoint{3.877530in}{2.715780in}}%
\pgfpathlineto{\pgfqpoint{3.863760in}{2.720703in}}%
\pgfpathlineto{\pgfqpoint{3.859274in}{2.732587in}}%
\pgfpathlineto{\pgfqpoint{3.869194in}{2.753456in}}%
\pgfpathlineto{\pgfqpoint{3.888756in}{2.760664in}}%
\pgfpathlineto{\pgfqpoint{3.911473in}{2.748351in}}%
\pgfpathlineto{\pgfqpoint{3.932890in}{2.748197in}}%
\pgfpathlineto{\pgfqpoint{3.942837in}{2.733814in}}%
\pgfpathlineto{\pgfqpoint{3.957836in}{2.734813in}}%
\pgfpathlineto{\pgfqpoint{4.004139in}{2.714819in}}%
\pgfpathlineto{\pgfqpoint{4.013390in}{2.695463in}}%
\pgfpathlineto{\pgfqpoint{4.003287in}{2.688760in}}%
\pgfpathlineto{\pgfqpoint{4.006400in}{2.674254in}}%
\pgfpathlineto{\pgfqpoint{4.016381in}{2.667802in}}%
\pgfpathlineto{\pgfqpoint{4.021943in}{2.649970in}}%
\pgfpathlineto{\pgfqpoint{4.021164in}{2.606528in}}%
\pgfpathlineto{\pgfqpoint{4.007987in}{2.596150in}}%
\pgfpathlineto{\pgfqpoint{4.005029in}{2.573331in}}%
\pgfpathlineto{\pgfqpoint{3.980584in}{2.552242in}}%
\pgfpathlineto{\pgfqpoint{3.982034in}{2.526715in}}%
\pgfpathlineto{\pgfqpoint{4.003459in}{2.517773in}}%
\pgfpathlineto{\pgfqpoint{4.027630in}{2.549622in}}%
\pgfpathlineto{\pgfqpoint{4.029628in}{2.561174in}}%
\pgfpathlineto{\pgfqpoint{4.059771in}{2.580374in}}%
\pgfpathlineto{\pgfqpoint{4.078940in}{2.571480in}}%
\pgfpathlineto{\pgfqpoint{4.090970in}{2.551381in}}%
\pgfpathlineto{\pgfqpoint{4.110375in}{2.481595in}}%
\pgfpathlineto{\pgfqpoint{4.120659in}{2.459517in}}%
\pgfpathlineto{\pgfqpoint{4.117439in}{2.450390in}}%
\pgfpathlineto{\pgfqpoint{4.117752in}{2.419311in}}%
\pgfpathlineto{\pgfqpoint{4.099046in}{2.422294in}}%
\pgfpathlineto{\pgfqpoint{4.088480in}{2.399070in}}%
\pgfpathlineto{\pgfqpoint{4.087016in}{2.383282in}}%
\pgfpathlineto{\pgfqpoint{4.072885in}{2.373146in}}%
\pgfpathlineto{\pgfqpoint{4.069745in}{2.342356in}}%
\pgfpathlineto{\pgfqpoint{4.049197in}{2.303459in}}%
\pgfpathlineto{\pgfqpoint{3.936789in}{2.286718in}}%
\pgfpathlineto{\pgfqpoint{3.936124in}{2.294075in}}%
\pgfpathlineto{\pgfqpoint{3.860932in}{2.286151in}}%
\pgfpathlineto{\pgfqpoint{3.759344in}{2.276672in}}%
\pgfusepath{stroke}%
\end{pgfscope}%
\begin{pgfscope}%
\pgfpathrectangle{\pgfqpoint{0.100000in}{0.100000in}}{\pgfqpoint{5.307240in}{3.397500in}}%
\pgfusepath{clip}%
\pgfsetbuttcap%
\pgfsetroundjoin%
\pgfsetlinewidth{0.501875pt}%
\definecolor{currentstroke}{rgb}{0.827451,0.827451,0.827451}%
\pgfsetstrokecolor{currentstroke}%
\pgfsetdash{}{0pt}%
\pgfpathmoveto{\pgfqpoint{0.000000in}{0.000000in}}%
\pgfusepath{stroke}%
\end{pgfscope}%
\begin{pgfscope}%
\pgfpathrectangle{\pgfqpoint{0.100000in}{0.100000in}}{\pgfqpoint{5.307240in}{3.397500in}}%
\pgfusepath{clip}%
\pgfsetbuttcap%
\pgfsetroundjoin%
\pgfsetlinewidth{0.501875pt}%
\definecolor{currentstroke}{rgb}{0.827451,0.827451,0.827451}%
\pgfsetstrokecolor{currentstroke}%
\pgfsetdash{}{0pt}%
\pgfusepath{stroke}%
\end{pgfscope}%
\begin{pgfscope}%
\pgfpathrectangle{\pgfqpoint{0.100000in}{0.100000in}}{\pgfqpoint{5.307240in}{3.397500in}}%
\pgfusepath{clip}%
\pgfsetbuttcap%
\pgfsetroundjoin%
\pgfsetlinewidth{0.501875pt}%
\definecolor{currentstroke}{rgb}{0.827451,0.827451,0.827451}%
\pgfsetstrokecolor{currentstroke}%
\pgfsetdash{}{0pt}%
\pgfusepath{stroke}%
\end{pgfscope}%
\begin{pgfscope}%
\pgfpathrectangle{\pgfqpoint{0.100000in}{0.100000in}}{\pgfqpoint{5.307240in}{3.397500in}}%
\pgfusepath{clip}%
\pgfsetbuttcap%
\pgfsetroundjoin%
\pgfsetlinewidth{0.501875pt}%
\definecolor{currentstroke}{rgb}{0.827451,0.827451,0.827451}%
\pgfsetstrokecolor{currentstroke}%
\pgfsetdash{}{0pt}%
\pgfusepath{stroke}%
\end{pgfscope}%
\begin{pgfscope}%
\pgfpathrectangle{\pgfqpoint{0.100000in}{0.100000in}}{\pgfqpoint{5.307240in}{3.397500in}}%
\pgfusepath{clip}%
\pgfsetbuttcap%
\pgfsetroundjoin%
\pgfsetlinewidth{0.501875pt}%
\definecolor{currentstroke}{rgb}{0.827451,0.827451,0.827451}%
\pgfsetstrokecolor{currentstroke}%
\pgfsetdash{}{0pt}%
\pgfusepath{stroke}%
\end{pgfscope}%
\begin{pgfscope}%
\pgfpathrectangle{\pgfqpoint{0.100000in}{0.100000in}}{\pgfqpoint{5.307240in}{3.397500in}}%
\pgfusepath{clip}%
\pgfsetbuttcap%
\pgfsetroundjoin%
\pgfsetlinewidth{0.501875pt}%
\definecolor{currentstroke}{rgb}{0.827451,0.827451,0.827451}%
\pgfsetstrokecolor{currentstroke}%
\pgfsetdash{}{0pt}%
\pgfusepath{stroke}%
\end{pgfscope}%
\begin{pgfscope}%
\pgfpathrectangle{\pgfqpoint{0.100000in}{0.100000in}}{\pgfqpoint{5.307240in}{3.397500in}}%
\pgfusepath{clip}%
\pgfsetbuttcap%
\pgfsetroundjoin%
\pgfsetlinewidth{0.501875pt}%
\definecolor{currentstroke}{rgb}{0.827451,0.827451,0.827451}%
\pgfsetstrokecolor{currentstroke}%
\pgfsetdash{}{0pt}%
\pgfusepath{stroke}%
\end{pgfscope}%
\begin{pgfscope}%
\pgfpathrectangle{\pgfqpoint{0.100000in}{0.100000in}}{\pgfqpoint{5.307240in}{3.397500in}}%
\pgfusepath{clip}%
\pgfsetbuttcap%
\pgfsetroundjoin%
\pgfsetlinewidth{0.501875pt}%
\definecolor{currentstroke}{rgb}{0.827451,0.827451,0.827451}%
\pgfsetstrokecolor{currentstroke}%
\pgfsetdash{}{0pt}%
\pgfusepath{stroke}%
\end{pgfscope}%
\begin{pgfscope}%
\pgfpathrectangle{\pgfqpoint{0.100000in}{0.100000in}}{\pgfqpoint{5.307240in}{3.397500in}}%
\pgfusepath{clip}%
\pgfsetbuttcap%
\pgfsetroundjoin%
\pgfsetlinewidth{0.501875pt}%
\definecolor{currentstroke}{rgb}{0.827451,0.827451,0.827451}%
\pgfsetstrokecolor{currentstroke}%
\pgfsetdash{}{0pt}%
\pgfusepath{stroke}%
\end{pgfscope}%
\begin{pgfscope}%
\pgfpathrectangle{\pgfqpoint{0.100000in}{0.100000in}}{\pgfqpoint{5.307240in}{3.397500in}}%
\pgfusepath{clip}%
\pgfsetbuttcap%
\pgfsetroundjoin%
\pgfsetlinewidth{0.501875pt}%
\definecolor{currentstroke}{rgb}{0.827451,0.827451,0.827451}%
\pgfsetstrokecolor{currentstroke}%
\pgfsetdash{}{0pt}%
\pgfusepath{stroke}%
\end{pgfscope}%
\begin{pgfscope}%
\pgfpathrectangle{\pgfqpoint{0.100000in}{0.100000in}}{\pgfqpoint{5.307240in}{3.397500in}}%
\pgfusepath{clip}%
\pgfsetbuttcap%
\pgfsetroundjoin%
\pgfsetlinewidth{0.501875pt}%
\definecolor{currentstroke}{rgb}{0.827451,0.827451,0.827451}%
\pgfsetstrokecolor{currentstroke}%
\pgfsetdash{}{0pt}%
\pgfusepath{stroke}%
\end{pgfscope}%
\begin{pgfscope}%
\pgfpathrectangle{\pgfqpoint{0.100000in}{0.100000in}}{\pgfqpoint{5.307240in}{3.397500in}}%
\pgfusepath{clip}%
\pgfsetbuttcap%
\pgfsetroundjoin%
\pgfsetlinewidth{0.501875pt}%
\definecolor{currentstroke}{rgb}{0.827451,0.827451,0.827451}%
\pgfsetstrokecolor{currentstroke}%
\pgfsetdash{}{0pt}%
\pgfusepath{stroke}%
\end{pgfscope}%
\begin{pgfscope}%
\pgfpathrectangle{\pgfqpoint{0.100000in}{0.100000in}}{\pgfqpoint{5.307240in}{3.397500in}}%
\pgfusepath{clip}%
\pgfsetbuttcap%
\pgfsetroundjoin%
\pgfsetlinewidth{0.501875pt}%
\definecolor{currentstroke}{rgb}{0.827451,0.827451,0.827451}%
\pgfsetstrokecolor{currentstroke}%
\pgfsetdash{}{0pt}%
\pgfusepath{stroke}%
\end{pgfscope}%
\begin{pgfscope}%
\pgfpathrectangle{\pgfqpoint{0.100000in}{0.100000in}}{\pgfqpoint{5.307240in}{3.397500in}}%
\pgfusepath{clip}%
\pgfsetbuttcap%
\pgfsetroundjoin%
\pgfsetlinewidth{0.501875pt}%
\definecolor{currentstroke}{rgb}{0.827451,0.827451,0.827451}%
\pgfsetstrokecolor{currentstroke}%
\pgfsetdash{}{0pt}%
\pgfusepath{stroke}%
\end{pgfscope}%
\begin{pgfscope}%
\pgfpathrectangle{\pgfqpoint{0.100000in}{0.100000in}}{\pgfqpoint{5.307240in}{3.397500in}}%
\pgfusepath{clip}%
\pgfsetbuttcap%
\pgfsetroundjoin%
\pgfsetlinewidth{0.501875pt}%
\definecolor{currentstroke}{rgb}{0.827451,0.827451,0.827451}%
\pgfsetstrokecolor{currentstroke}%
\pgfsetdash{}{0pt}%
\pgfusepath{stroke}%
\end{pgfscope}%
\begin{pgfscope}%
\pgfpathrectangle{\pgfqpoint{0.100000in}{0.100000in}}{\pgfqpoint{5.307240in}{3.397500in}}%
\pgfusepath{clip}%
\pgfsetbuttcap%
\pgfsetroundjoin%
\pgfsetlinewidth{0.501875pt}%
\definecolor{currentstroke}{rgb}{0.827451,0.827451,0.827451}%
\pgfsetstrokecolor{currentstroke}%
\pgfsetdash{}{0pt}%
\pgfusepath{stroke}%
\end{pgfscope}%
\begin{pgfscope}%
\pgfpathrectangle{\pgfqpoint{0.100000in}{0.100000in}}{\pgfqpoint{5.307240in}{3.397500in}}%
\pgfusepath{clip}%
\pgfsetbuttcap%
\pgfsetroundjoin%
\pgfsetlinewidth{0.501875pt}%
\definecolor{currentstroke}{rgb}{0.827451,0.827451,0.827451}%
\pgfsetstrokecolor{currentstroke}%
\pgfsetdash{}{0pt}%
\pgfusepath{stroke}%
\end{pgfscope}%
\begin{pgfscope}%
\pgfpathrectangle{\pgfqpoint{0.100000in}{0.100000in}}{\pgfqpoint{5.307240in}{3.397500in}}%
\pgfusepath{clip}%
\pgfsetbuttcap%
\pgfsetroundjoin%
\pgfsetlinewidth{0.501875pt}%
\definecolor{currentstroke}{rgb}{0.827451,0.827451,0.827451}%
\pgfsetstrokecolor{currentstroke}%
\pgfsetdash{}{0pt}%
\pgfusepath{stroke}%
\end{pgfscope}%
\begin{pgfscope}%
\pgfpathrectangle{\pgfqpoint{0.100000in}{0.100000in}}{\pgfqpoint{5.307240in}{3.397500in}}%
\pgfusepath{clip}%
\pgfsetbuttcap%
\pgfsetroundjoin%
\pgfsetlinewidth{0.501875pt}%
\definecolor{currentstroke}{rgb}{0.827451,0.827451,0.827451}%
\pgfsetstrokecolor{currentstroke}%
\pgfsetdash{}{0pt}%
\pgfusepath{stroke}%
\end{pgfscope}%
\begin{pgfscope}%
\pgfpathrectangle{\pgfqpoint{0.100000in}{0.100000in}}{\pgfqpoint{5.307240in}{3.397500in}}%
\pgfusepath{clip}%
\pgfsetbuttcap%
\pgfsetroundjoin%
\pgfsetlinewidth{0.501875pt}%
\definecolor{currentstroke}{rgb}{0.827451,0.827451,0.827451}%
\pgfsetstrokecolor{currentstroke}%
\pgfsetdash{}{0pt}%
\pgfusepath{stroke}%
\end{pgfscope}%
\begin{pgfscope}%
\pgfpathrectangle{\pgfqpoint{0.100000in}{0.100000in}}{\pgfqpoint{5.307240in}{3.397500in}}%
\pgfusepath{clip}%
\pgfsetbuttcap%
\pgfsetroundjoin%
\pgfsetlinewidth{0.501875pt}%
\definecolor{currentstroke}{rgb}{0.827451,0.827451,0.827451}%
\pgfsetstrokecolor{currentstroke}%
\pgfsetdash{}{0pt}%
\pgfusepath{stroke}%
\end{pgfscope}%
\begin{pgfscope}%
\pgfpathrectangle{\pgfqpoint{0.100000in}{0.100000in}}{\pgfqpoint{5.307240in}{3.397500in}}%
\pgfusepath{clip}%
\pgfsetbuttcap%
\pgfsetroundjoin%
\pgfsetlinewidth{0.501875pt}%
\definecolor{currentstroke}{rgb}{0.827451,0.827451,0.827451}%
\pgfsetstrokecolor{currentstroke}%
\pgfsetdash{}{0pt}%
\pgfusepath{stroke}%
\end{pgfscope}%
\begin{pgfscope}%
\pgfpathrectangle{\pgfqpoint{0.100000in}{0.100000in}}{\pgfqpoint{5.307240in}{3.397500in}}%
\pgfusepath{clip}%
\pgfsetbuttcap%
\pgfsetroundjoin%
\pgfsetlinewidth{0.501875pt}%
\definecolor{currentstroke}{rgb}{0.827451,0.827451,0.827451}%
\pgfsetstrokecolor{currentstroke}%
\pgfsetdash{}{0pt}%
\pgfusepath{stroke}%
\end{pgfscope}%
\begin{pgfscope}%
\pgfpathrectangle{\pgfqpoint{0.100000in}{0.100000in}}{\pgfqpoint{5.307240in}{3.397500in}}%
\pgfusepath{clip}%
\pgfsetbuttcap%
\pgfsetroundjoin%
\pgfsetlinewidth{0.501875pt}%
\definecolor{currentstroke}{rgb}{0.827451,0.827451,0.827451}%
\pgfsetstrokecolor{currentstroke}%
\pgfsetdash{}{0pt}%
\pgfusepath{stroke}%
\end{pgfscope}%
\begin{pgfscope}%
\pgfpathrectangle{\pgfqpoint{0.100000in}{0.100000in}}{\pgfqpoint{5.307240in}{3.397500in}}%
\pgfusepath{clip}%
\pgfsetbuttcap%
\pgfsetroundjoin%
\pgfsetlinewidth{0.501875pt}%
\definecolor{currentstroke}{rgb}{0.827451,0.827451,0.827451}%
\pgfsetstrokecolor{currentstroke}%
\pgfsetdash{}{0pt}%
\pgfusepath{stroke}%
\end{pgfscope}%
\begin{pgfscope}%
\pgfpathrectangle{\pgfqpoint{0.100000in}{0.100000in}}{\pgfqpoint{5.307240in}{3.397500in}}%
\pgfusepath{clip}%
\pgfsetbuttcap%
\pgfsetroundjoin%
\pgfsetlinewidth{0.501875pt}%
\definecolor{currentstroke}{rgb}{0.827451,0.827451,0.827451}%
\pgfsetstrokecolor{currentstroke}%
\pgfsetdash{}{0pt}%
\pgfusepath{stroke}%
\end{pgfscope}%
\begin{pgfscope}%
\pgfpathrectangle{\pgfqpoint{0.100000in}{0.100000in}}{\pgfqpoint{5.307240in}{3.397500in}}%
\pgfusepath{clip}%
\pgfsetbuttcap%
\pgfsetroundjoin%
\pgfsetlinewidth{0.501875pt}%
\definecolor{currentstroke}{rgb}{0.827451,0.827451,0.827451}%
\pgfsetstrokecolor{currentstroke}%
\pgfsetdash{}{0pt}%
\pgfusepath{stroke}%
\end{pgfscope}%
\begin{pgfscope}%
\pgfpathrectangle{\pgfqpoint{0.100000in}{0.100000in}}{\pgfqpoint{5.307240in}{3.397500in}}%
\pgfusepath{clip}%
\pgfsetbuttcap%
\pgfsetroundjoin%
\pgfsetlinewidth{0.501875pt}%
\definecolor{currentstroke}{rgb}{0.827451,0.827451,0.827451}%
\pgfsetstrokecolor{currentstroke}%
\pgfsetdash{}{0pt}%
\pgfusepath{stroke}%
\end{pgfscope}%
\begin{pgfscope}%
\pgfpathrectangle{\pgfqpoint{0.100000in}{0.100000in}}{\pgfqpoint{5.307240in}{3.397500in}}%
\pgfusepath{clip}%
\pgfsetrectcap%
\pgfsetroundjoin%
\pgfsetlinewidth{1.505625pt}%
\definecolor{currentstroke}{rgb}{0.678431,1.000000,0.184314}%
\pgfsetstrokecolor{currentstroke}%
\pgfsetstrokeopacity{0.500000}%
\pgfsetdash{}{0pt}%
\pgfpathmoveto{\pgfqpoint{3.939827in}{1.350487in}}%
\pgfusepath{stroke}%
\end{pgfscope}%
\begin{pgfscope}%
\pgfpathrectangle{\pgfqpoint{0.100000in}{0.100000in}}{\pgfqpoint{5.307240in}{3.397500in}}%
\pgfusepath{clip}%
\pgfsetbuttcap%
\pgfsetroundjoin%
\definecolor{currentfill}{rgb}{0.678431,1.000000,0.184314}%
\pgfsetfillcolor{currentfill}%
\pgfsetfillopacity{0.500000}%
\pgfsetlinewidth{0.250937pt}%
\definecolor{currentstroke}{rgb}{0.000000,0.000000,0.000000}%
\pgfsetstrokecolor{currentstroke}%
\pgfsetstrokeopacity{0.500000}%
\pgfsetdash{}{0pt}%
\pgfsys@defobject{currentmarker}{\pgfqpoint{-0.106250in}{-0.106250in}}{\pgfqpoint{0.106250in}{0.106250in}}{%
\pgfpathmoveto{\pgfqpoint{0.000000in}{-0.106250in}}%
\pgfpathcurveto{\pgfqpoint{0.028178in}{-0.106250in}}{\pgfqpoint{0.055205in}{-0.095055in}}{\pgfqpoint{0.075130in}{-0.075130in}}%
\pgfpathcurveto{\pgfqpoint{0.095055in}{-0.055205in}}{\pgfqpoint{0.106250in}{-0.028178in}}{\pgfqpoint{0.106250in}{0.000000in}}%
\pgfpathcurveto{\pgfqpoint{0.106250in}{0.028178in}}{\pgfqpoint{0.095055in}{0.055205in}}{\pgfqpoint{0.075130in}{0.075130in}}%
\pgfpathcurveto{\pgfqpoint{0.055205in}{0.095055in}}{\pgfqpoint{0.028178in}{0.106250in}}{\pgfqpoint{0.000000in}{0.106250in}}%
\pgfpathcurveto{\pgfqpoint{-0.028178in}{0.106250in}}{\pgfqpoint{-0.055205in}{0.095055in}}{\pgfqpoint{-0.075130in}{0.075130in}}%
\pgfpathcurveto{\pgfqpoint{-0.095055in}{0.055205in}}{\pgfqpoint{-0.106250in}{0.028178in}}{\pgfqpoint{-0.106250in}{0.000000in}}%
\pgfpathcurveto{\pgfqpoint{-0.106250in}{-0.028178in}}{\pgfqpoint{-0.095055in}{-0.055205in}}{\pgfqpoint{-0.075130in}{-0.075130in}}%
\pgfpathcurveto{\pgfqpoint{-0.055205in}{-0.095055in}}{\pgfqpoint{-0.028178in}{-0.106250in}}{\pgfqpoint{0.000000in}{-0.106250in}}%
\pgfpathclose%
\pgfusepath{stroke,fill}%
}%
\begin{pgfscope}%
\pgfsys@transformshift{3.939827in}{1.350487in}%
\pgfsys@useobject{currentmarker}{}%
\end{pgfscope}%
\end{pgfscope}%
\begin{pgfscope}%
\pgfpathrectangle{\pgfqpoint{0.100000in}{0.100000in}}{\pgfqpoint{5.307240in}{3.397500in}}%
\pgfusepath{clip}%
\pgfsetrectcap%
\pgfsetroundjoin%
\pgfsetlinewidth{1.505625pt}%
\definecolor{currentstroke}{rgb}{0.678431,1.000000,0.184314}%
\pgfsetstrokecolor{currentstroke}%
\pgfsetstrokeopacity{0.500000}%
\pgfsetdash{}{0pt}%
\pgfpathmoveto{\pgfqpoint{3.984993in}{1.232932in}}%
\pgfusepath{stroke}%
\end{pgfscope}%
\begin{pgfscope}%
\pgfpathrectangle{\pgfqpoint{0.100000in}{0.100000in}}{\pgfqpoint{5.307240in}{3.397500in}}%
\pgfusepath{clip}%
\pgfsetbuttcap%
\pgfsetroundjoin%
\definecolor{currentfill}{rgb}{0.678431,1.000000,0.184314}%
\pgfsetfillcolor{currentfill}%
\pgfsetfillopacity{0.500000}%
\pgfsetlinewidth{0.250937pt}%
\definecolor{currentstroke}{rgb}{0.000000,0.000000,0.000000}%
\pgfsetstrokecolor{currentstroke}%
\pgfsetstrokeopacity{0.500000}%
\pgfsetdash{}{0pt}%
\pgfsys@defobject{currentmarker}{\pgfqpoint{-0.070139in}{-0.070139in}}{\pgfqpoint{0.070139in}{0.070139in}}{%
\pgfpathmoveto{\pgfqpoint{0.000000in}{-0.070139in}}%
\pgfpathcurveto{\pgfqpoint{0.018601in}{-0.070139in}}{\pgfqpoint{0.036443in}{-0.062749in}}{\pgfqpoint{0.049596in}{-0.049596in}}%
\pgfpathcurveto{\pgfqpoint{0.062749in}{-0.036443in}}{\pgfqpoint{0.070139in}{-0.018601in}}{\pgfqpoint{0.070139in}{0.000000in}}%
\pgfpathcurveto{\pgfqpoint{0.070139in}{0.018601in}}{\pgfqpoint{0.062749in}{0.036443in}}{\pgfqpoint{0.049596in}{0.049596in}}%
\pgfpathcurveto{\pgfqpoint{0.036443in}{0.062749in}}{\pgfqpoint{0.018601in}{0.070139in}}{\pgfqpoint{0.000000in}{0.070139in}}%
\pgfpathcurveto{\pgfqpoint{-0.018601in}{0.070139in}}{\pgfqpoint{-0.036443in}{0.062749in}}{\pgfqpoint{-0.049596in}{0.049596in}}%
\pgfpathcurveto{\pgfqpoint{-0.062749in}{0.036443in}}{\pgfqpoint{-0.070139in}{0.018601in}}{\pgfqpoint{-0.070139in}{0.000000in}}%
\pgfpathcurveto{\pgfqpoint{-0.070139in}{-0.018601in}}{\pgfqpoint{-0.062749in}{-0.036443in}}{\pgfqpoint{-0.049596in}{-0.049596in}}%
\pgfpathcurveto{\pgfqpoint{-0.036443in}{-0.062749in}}{\pgfqpoint{-0.018601in}{-0.070139in}}{\pgfqpoint{0.000000in}{-0.070139in}}%
\pgfpathclose%
\pgfusepath{stroke,fill}%
}%
\begin{pgfscope}%
\pgfsys@transformshift{3.984993in}{1.232932in}%
\pgfsys@useobject{currentmarker}{}%
\end{pgfscope}%
\end{pgfscope}%
\begin{pgfscope}%
\pgfpathrectangle{\pgfqpoint{0.100000in}{0.100000in}}{\pgfqpoint{5.307240in}{3.397500in}}%
\pgfusepath{clip}%
\pgfsetrectcap%
\pgfsetroundjoin%
\pgfsetlinewidth{1.505625pt}%
\definecolor{currentstroke}{rgb}{0.678431,1.000000,0.184314}%
\pgfsetstrokecolor{currentstroke}%
\pgfsetstrokeopacity{0.500000}%
\pgfsetdash{}{0pt}%
\pgfpathmoveto{\pgfqpoint{3.848586in}{1.321254in}}%
\pgfusepath{stroke}%
\end{pgfscope}%
\begin{pgfscope}%
\pgfpathrectangle{\pgfqpoint{0.100000in}{0.100000in}}{\pgfqpoint{5.307240in}{3.397500in}}%
\pgfusepath{clip}%
\pgfsetbuttcap%
\pgfsetroundjoin%
\definecolor{currentfill}{rgb}{0.678431,1.000000,0.184314}%
\pgfsetfillcolor{currentfill}%
\pgfsetfillopacity{0.500000}%
\pgfsetlinewidth{0.250937pt}%
\definecolor{currentstroke}{rgb}{0.000000,0.000000,0.000000}%
\pgfsetstrokecolor{currentstroke}%
\pgfsetstrokeopacity{0.500000}%
\pgfsetdash{}{0pt}%
\pgfsys@defobject{currentmarker}{\pgfqpoint{-0.065972in}{-0.065972in}}{\pgfqpoint{0.065972in}{0.065972in}}{%
\pgfpathmoveto{\pgfqpoint{0.000000in}{-0.065972in}}%
\pgfpathcurveto{\pgfqpoint{0.017496in}{-0.065972in}}{\pgfqpoint{0.034278in}{-0.059021in}}{\pgfqpoint{0.046649in}{-0.046649in}}%
\pgfpathcurveto{\pgfqpoint{0.059021in}{-0.034278in}}{\pgfqpoint{0.065972in}{-0.017496in}}{\pgfqpoint{0.065972in}{0.000000in}}%
\pgfpathcurveto{\pgfqpoint{0.065972in}{0.017496in}}{\pgfqpoint{0.059021in}{0.034278in}}{\pgfqpoint{0.046649in}{0.046649in}}%
\pgfpathcurveto{\pgfqpoint{0.034278in}{0.059021in}}{\pgfqpoint{0.017496in}{0.065972in}}{\pgfqpoint{0.000000in}{0.065972in}}%
\pgfpathcurveto{\pgfqpoint{-0.017496in}{0.065972in}}{\pgfqpoint{-0.034278in}{0.059021in}}{\pgfqpoint{-0.046649in}{0.046649in}}%
\pgfpathcurveto{\pgfqpoint{-0.059021in}{0.034278in}}{\pgfqpoint{-0.065972in}{0.017496in}}{\pgfqpoint{-0.065972in}{0.000000in}}%
\pgfpathcurveto{\pgfqpoint{-0.065972in}{-0.017496in}}{\pgfqpoint{-0.059021in}{-0.034278in}}{\pgfqpoint{-0.046649in}{-0.046649in}}%
\pgfpathcurveto{\pgfqpoint{-0.034278in}{-0.059021in}}{\pgfqpoint{-0.017496in}{-0.065972in}}{\pgfqpoint{0.000000in}{-0.065972in}}%
\pgfpathclose%
\pgfusepath{stroke,fill}%
}%
\begin{pgfscope}%
\pgfsys@transformshift{3.848586in}{1.321254in}%
\pgfsys@useobject{currentmarker}{}%
\end{pgfscope}%
\end{pgfscope}%
\begin{pgfscope}%
\pgfpathrectangle{\pgfqpoint{0.100000in}{0.100000in}}{\pgfqpoint{5.307240in}{3.397500in}}%
\pgfusepath{clip}%
\pgfsetrectcap%
\pgfsetroundjoin%
\pgfsetlinewidth{1.505625pt}%
\definecolor{currentstroke}{rgb}{0.678431,1.000000,0.184314}%
\pgfsetstrokecolor{currentstroke}%
\pgfsetstrokeopacity{0.500000}%
\pgfsetdash{}{0pt}%
\pgfpathmoveto{\pgfqpoint{3.773536in}{0.968317in}}%
\pgfusepath{stroke}%
\end{pgfscope}%
\begin{pgfscope}%
\pgfpathrectangle{\pgfqpoint{0.100000in}{0.100000in}}{\pgfqpoint{5.307240in}{3.397500in}}%
\pgfusepath{clip}%
\pgfsetbuttcap%
\pgfsetroundjoin%
\definecolor{currentfill}{rgb}{0.678431,1.000000,0.184314}%
\pgfsetfillcolor{currentfill}%
\pgfsetfillopacity{0.500000}%
\pgfsetlinewidth{0.250937pt}%
\definecolor{currentstroke}{rgb}{0.000000,0.000000,0.000000}%
\pgfsetstrokecolor{currentstroke}%
\pgfsetstrokeopacity{0.500000}%
\pgfsetdash{}{0pt}%
\pgfsys@defobject{currentmarker}{\pgfqpoint{-0.091667in}{-0.091667in}}{\pgfqpoint{0.091667in}{0.091667in}}{%
\pgfpathmoveto{\pgfqpoint{0.000000in}{-0.091667in}}%
\pgfpathcurveto{\pgfqpoint{0.024310in}{-0.091667in}}{\pgfqpoint{0.047628in}{-0.082008in}}{\pgfqpoint{0.064818in}{-0.064818in}}%
\pgfpathcurveto{\pgfqpoint{0.082008in}{-0.047628in}}{\pgfqpoint{0.091667in}{-0.024310in}}{\pgfqpoint{0.091667in}{0.000000in}}%
\pgfpathcurveto{\pgfqpoint{0.091667in}{0.024310in}}{\pgfqpoint{0.082008in}{0.047628in}}{\pgfqpoint{0.064818in}{0.064818in}}%
\pgfpathcurveto{\pgfqpoint{0.047628in}{0.082008in}}{\pgfqpoint{0.024310in}{0.091667in}}{\pgfqpoint{0.000000in}{0.091667in}}%
\pgfpathcurveto{\pgfqpoint{-0.024310in}{0.091667in}}{\pgfqpoint{-0.047628in}{0.082008in}}{\pgfqpoint{-0.064818in}{0.064818in}}%
\pgfpathcurveto{\pgfqpoint{-0.082008in}{0.047628in}}{\pgfqpoint{-0.091667in}{0.024310in}}{\pgfqpoint{-0.091667in}{0.000000in}}%
\pgfpathcurveto{\pgfqpoint{-0.091667in}{-0.024310in}}{\pgfqpoint{-0.082008in}{-0.047628in}}{\pgfqpoint{-0.064818in}{-0.064818in}}%
\pgfpathcurveto{\pgfqpoint{-0.047628in}{-0.082008in}}{\pgfqpoint{-0.024310in}{-0.091667in}}{\pgfqpoint{0.000000in}{-0.091667in}}%
\pgfpathclose%
\pgfusepath{stroke,fill}%
}%
\begin{pgfscope}%
\pgfsys@transformshift{3.773536in}{0.968317in}%
\pgfsys@useobject{currentmarker}{}%
\end{pgfscope}%
\end{pgfscope}%
\begin{pgfscope}%
\pgfpathrectangle{\pgfqpoint{0.100000in}{0.100000in}}{\pgfqpoint{5.307240in}{3.397500in}}%
\pgfusepath{clip}%
\pgfsetrectcap%
\pgfsetroundjoin%
\pgfsetlinewidth{1.505625pt}%
\definecolor{currentstroke}{rgb}{0.678431,1.000000,0.184314}%
\pgfsetstrokecolor{currentstroke}%
\pgfsetstrokeopacity{0.500000}%
\pgfsetdash{}{0pt}%
\pgfpathmoveto{\pgfqpoint{3.812572in}{1.447303in}}%
\pgfusepath{stroke}%
\end{pgfscope}%
\begin{pgfscope}%
\pgfpathrectangle{\pgfqpoint{0.100000in}{0.100000in}}{\pgfqpoint{5.307240in}{3.397500in}}%
\pgfusepath{clip}%
\pgfsetbuttcap%
\pgfsetroundjoin%
\definecolor{currentfill}{rgb}{0.678431,1.000000,0.184314}%
\pgfsetfillcolor{currentfill}%
\pgfsetfillopacity{0.500000}%
\pgfsetlinewidth{0.250937pt}%
\definecolor{currentstroke}{rgb}{0.000000,0.000000,0.000000}%
\pgfsetstrokecolor{currentstroke}%
\pgfsetstrokeopacity{0.500000}%
\pgfsetdash{}{0pt}%
\pgfsys@defobject{currentmarker}{\pgfqpoint{-0.063194in}{-0.063194in}}{\pgfqpoint{0.063194in}{0.063194in}}{%
\pgfpathmoveto{\pgfqpoint{0.000000in}{-0.063194in}}%
\pgfpathcurveto{\pgfqpoint{0.016759in}{-0.063194in}}{\pgfqpoint{0.032835in}{-0.056536in}}{\pgfqpoint{0.044685in}{-0.044685in}}%
\pgfpathcurveto{\pgfqpoint{0.056536in}{-0.032835in}}{\pgfqpoint{0.063194in}{-0.016759in}}{\pgfqpoint{0.063194in}{0.000000in}}%
\pgfpathcurveto{\pgfqpoint{0.063194in}{0.016759in}}{\pgfqpoint{0.056536in}{0.032835in}}{\pgfqpoint{0.044685in}{0.044685in}}%
\pgfpathcurveto{\pgfqpoint{0.032835in}{0.056536in}}{\pgfqpoint{0.016759in}{0.063194in}}{\pgfqpoint{0.000000in}{0.063194in}}%
\pgfpathcurveto{\pgfqpoint{-0.016759in}{0.063194in}}{\pgfqpoint{-0.032835in}{0.056536in}}{\pgfqpoint{-0.044685in}{0.044685in}}%
\pgfpathcurveto{\pgfqpoint{-0.056536in}{0.032835in}}{\pgfqpoint{-0.063194in}{0.016759in}}{\pgfqpoint{-0.063194in}{0.000000in}}%
\pgfpathcurveto{\pgfqpoint{-0.063194in}{-0.016759in}}{\pgfqpoint{-0.056536in}{-0.032835in}}{\pgfqpoint{-0.044685in}{-0.044685in}}%
\pgfpathcurveto{\pgfqpoint{-0.032835in}{-0.056536in}}{\pgfqpoint{-0.016759in}{-0.063194in}}{\pgfqpoint{0.000000in}{-0.063194in}}%
\pgfpathclose%
\pgfusepath{stroke,fill}%
}%
\begin{pgfscope}%
\pgfsys@transformshift{3.812572in}{1.447303in}%
\pgfsys@useobject{currentmarker}{}%
\end{pgfscope}%
\end{pgfscope}%
\begin{pgfscope}%
\pgfpathrectangle{\pgfqpoint{0.100000in}{0.100000in}}{\pgfqpoint{5.307240in}{3.397500in}}%
\pgfusepath{clip}%
\pgfsetrectcap%
\pgfsetroundjoin%
\pgfsetlinewidth{1.505625pt}%
\definecolor{currentstroke}{rgb}{0.678431,1.000000,0.184314}%
\pgfsetstrokecolor{currentstroke}%
\pgfsetstrokeopacity{0.500000}%
\pgfsetdash{}{0pt}%
\pgfpathmoveto{\pgfqpoint{4.013776in}{1.067319in}}%
\pgfusepath{stroke}%
\end{pgfscope}%
\begin{pgfscope}%
\pgfpathrectangle{\pgfqpoint{0.100000in}{0.100000in}}{\pgfqpoint{5.307240in}{3.397500in}}%
\pgfusepath{clip}%
\pgfsetbuttcap%
\pgfsetroundjoin%
\definecolor{currentfill}{rgb}{0.678431,1.000000,0.184314}%
\pgfsetfillcolor{currentfill}%
\pgfsetfillopacity{0.500000}%
\pgfsetlinewidth{0.250937pt}%
\definecolor{currentstroke}{rgb}{0.000000,0.000000,0.000000}%
\pgfsetstrokecolor{currentstroke}%
\pgfsetstrokeopacity{0.500000}%
\pgfsetdash{}{0pt}%
\pgfsys@defobject{currentmarker}{\pgfqpoint{-0.052778in}{-0.052778in}}{\pgfqpoint{0.052778in}{0.052778in}}{%
\pgfpathmoveto{\pgfqpoint{0.000000in}{-0.052778in}}%
\pgfpathcurveto{\pgfqpoint{0.013997in}{-0.052778in}}{\pgfqpoint{0.027422in}{-0.047217in}}{\pgfqpoint{0.037320in}{-0.037320in}}%
\pgfpathcurveto{\pgfqpoint{0.047217in}{-0.027422in}}{\pgfqpoint{0.052778in}{-0.013997in}}{\pgfqpoint{0.052778in}{0.000000in}}%
\pgfpathcurveto{\pgfqpoint{0.052778in}{0.013997in}}{\pgfqpoint{0.047217in}{0.027422in}}{\pgfqpoint{0.037320in}{0.037320in}}%
\pgfpathcurveto{\pgfqpoint{0.027422in}{0.047217in}}{\pgfqpoint{0.013997in}{0.052778in}}{\pgfqpoint{0.000000in}{0.052778in}}%
\pgfpathcurveto{\pgfqpoint{-0.013997in}{0.052778in}}{\pgfqpoint{-0.027422in}{0.047217in}}{\pgfqpoint{-0.037320in}{0.037320in}}%
\pgfpathcurveto{\pgfqpoint{-0.047217in}{0.027422in}}{\pgfqpoint{-0.052778in}{0.013997in}}{\pgfqpoint{-0.052778in}{0.000000in}}%
\pgfpathcurveto{\pgfqpoint{-0.052778in}{-0.013997in}}{\pgfqpoint{-0.047217in}{-0.027422in}}{\pgfqpoint{-0.037320in}{-0.037320in}}%
\pgfpathcurveto{\pgfqpoint{-0.027422in}{-0.047217in}}{\pgfqpoint{-0.013997in}{-0.052778in}}{\pgfqpoint{0.000000in}{-0.052778in}}%
\pgfpathclose%
\pgfusepath{stroke,fill}%
}%
\begin{pgfscope}%
\pgfsys@transformshift{4.013776in}{1.067319in}%
\pgfsys@useobject{currentmarker}{}%
\end{pgfscope}%
\end{pgfscope}%
\begin{pgfscope}%
\pgfpathrectangle{\pgfqpoint{0.100000in}{0.100000in}}{\pgfqpoint{5.307240in}{3.397500in}}%
\pgfusepath{clip}%
\pgfsetrectcap%
\pgfsetroundjoin%
\pgfsetlinewidth{1.505625pt}%
\definecolor{currentstroke}{rgb}{0.678431,1.000000,0.184314}%
\pgfsetstrokecolor{currentstroke}%
\pgfsetstrokeopacity{0.500000}%
\pgfsetdash{}{0pt}%
\pgfpathmoveto{\pgfqpoint{3.751709in}{1.462745in}}%
\pgfusepath{stroke}%
\end{pgfscope}%
\begin{pgfscope}%
\pgfpathrectangle{\pgfqpoint{0.100000in}{0.100000in}}{\pgfqpoint{5.307240in}{3.397500in}}%
\pgfusepath{clip}%
\pgfsetbuttcap%
\pgfsetroundjoin%
\definecolor{currentfill}{rgb}{0.678431,1.000000,0.184314}%
\pgfsetfillcolor{currentfill}%
\pgfsetfillopacity{0.500000}%
\pgfsetlinewidth{0.250937pt}%
\definecolor{currentstroke}{rgb}{0.000000,0.000000,0.000000}%
\pgfsetstrokecolor{currentstroke}%
\pgfsetstrokeopacity{0.500000}%
\pgfsetdash{}{0pt}%
\pgfsys@defobject{currentmarker}{\pgfqpoint{-0.086111in}{-0.086111in}}{\pgfqpoint{0.086111in}{0.086111in}}{%
\pgfpathmoveto{\pgfqpoint{0.000000in}{-0.086111in}}%
\pgfpathcurveto{\pgfqpoint{0.022837in}{-0.086111in}}{\pgfqpoint{0.044742in}{-0.077038in}}{\pgfqpoint{0.060890in}{-0.060890in}}%
\pgfpathcurveto{\pgfqpoint{0.077038in}{-0.044742in}}{\pgfqpoint{0.086111in}{-0.022837in}}{\pgfqpoint{0.086111in}{0.000000in}}%
\pgfpathcurveto{\pgfqpoint{0.086111in}{0.022837in}}{\pgfqpoint{0.077038in}{0.044742in}}{\pgfqpoint{0.060890in}{0.060890in}}%
\pgfpathcurveto{\pgfqpoint{0.044742in}{0.077038in}}{\pgfqpoint{0.022837in}{0.086111in}}{\pgfqpoint{0.000000in}{0.086111in}}%
\pgfpathcurveto{\pgfqpoint{-0.022837in}{0.086111in}}{\pgfqpoint{-0.044742in}{0.077038in}}{\pgfqpoint{-0.060890in}{0.060890in}}%
\pgfpathcurveto{\pgfqpoint{-0.077038in}{0.044742in}}{\pgfqpoint{-0.086111in}{0.022837in}}{\pgfqpoint{-0.086111in}{0.000000in}}%
\pgfpathcurveto{\pgfqpoint{-0.086111in}{-0.022837in}}{\pgfqpoint{-0.077038in}{-0.044742in}}{\pgfqpoint{-0.060890in}{-0.060890in}}%
\pgfpathcurveto{\pgfqpoint{-0.044742in}{-0.077038in}}{\pgfqpoint{-0.022837in}{-0.086111in}}{\pgfqpoint{0.000000in}{-0.086111in}}%
\pgfpathclose%
\pgfusepath{stroke,fill}%
}%
\begin{pgfscope}%
\pgfsys@transformshift{3.751709in}{1.462745in}%
\pgfsys@useobject{currentmarker}{}%
\end{pgfscope}%
\end{pgfscope}%
\begin{pgfscope}%
\pgfpathrectangle{\pgfqpoint{0.100000in}{0.100000in}}{\pgfqpoint{5.307240in}{3.397500in}}%
\pgfusepath{clip}%
\pgfsetrectcap%
\pgfsetroundjoin%
\pgfsetlinewidth{1.505625pt}%
\definecolor{currentstroke}{rgb}{0.678431,1.000000,0.184314}%
\pgfsetstrokecolor{currentstroke}%
\pgfsetstrokeopacity{0.500000}%
\pgfsetdash{}{0pt}%
\pgfpathmoveto{\pgfqpoint{3.920023in}{1.385924in}}%
\pgfusepath{stroke}%
\end{pgfscope}%
\begin{pgfscope}%
\pgfpathrectangle{\pgfqpoint{0.100000in}{0.100000in}}{\pgfqpoint{5.307240in}{3.397500in}}%
\pgfusepath{clip}%
\pgfsetbuttcap%
\pgfsetroundjoin%
\definecolor{currentfill}{rgb}{0.678431,1.000000,0.184314}%
\pgfsetfillcolor{currentfill}%
\pgfsetfillopacity{0.500000}%
\pgfsetlinewidth{0.250937pt}%
\definecolor{currentstroke}{rgb}{0.000000,0.000000,0.000000}%
\pgfsetstrokecolor{currentstroke}%
\pgfsetstrokeopacity{0.500000}%
\pgfsetdash{}{0pt}%
\pgfsys@defobject{currentmarker}{\pgfqpoint{-0.110417in}{-0.110417in}}{\pgfqpoint{0.110417in}{0.110417in}}{%
\pgfpathmoveto{\pgfqpoint{0.000000in}{-0.110417in}}%
\pgfpathcurveto{\pgfqpoint{0.029283in}{-0.110417in}}{\pgfqpoint{0.057370in}{-0.098782in}}{\pgfqpoint{0.078076in}{-0.078076in}}%
\pgfpathcurveto{\pgfqpoint{0.098782in}{-0.057370in}}{\pgfqpoint{0.110417in}{-0.029283in}}{\pgfqpoint{0.110417in}{0.000000in}}%
\pgfpathcurveto{\pgfqpoint{0.110417in}{0.029283in}}{\pgfqpoint{0.098782in}{0.057370in}}{\pgfqpoint{0.078076in}{0.078076in}}%
\pgfpathcurveto{\pgfqpoint{0.057370in}{0.098782in}}{\pgfqpoint{0.029283in}{0.110417in}}{\pgfqpoint{0.000000in}{0.110417in}}%
\pgfpathcurveto{\pgfqpoint{-0.029283in}{0.110417in}}{\pgfqpoint{-0.057370in}{0.098782in}}{\pgfqpoint{-0.078076in}{0.078076in}}%
\pgfpathcurveto{\pgfqpoint{-0.098782in}{0.057370in}}{\pgfqpoint{-0.110417in}{0.029283in}}{\pgfqpoint{-0.110417in}{0.000000in}}%
\pgfpathcurveto{\pgfqpoint{-0.110417in}{-0.029283in}}{\pgfqpoint{-0.098782in}{-0.057370in}}{\pgfqpoint{-0.078076in}{-0.078076in}}%
\pgfpathcurveto{\pgfqpoint{-0.057370in}{-0.098782in}}{\pgfqpoint{-0.029283in}{-0.110417in}}{\pgfqpoint{0.000000in}{-0.110417in}}%
\pgfpathclose%
\pgfusepath{stroke,fill}%
}%
\begin{pgfscope}%
\pgfsys@transformshift{3.920023in}{1.385924in}%
\pgfsys@useobject{currentmarker}{}%
\end{pgfscope}%
\end{pgfscope}%
\begin{pgfscope}%
\pgfpathrectangle{\pgfqpoint{0.100000in}{0.100000in}}{\pgfqpoint{5.307240in}{3.397500in}}%
\pgfusepath{clip}%
\pgfsetrectcap%
\pgfsetroundjoin%
\pgfsetlinewidth{1.505625pt}%
\definecolor{currentstroke}{rgb}{0.678431,1.000000,0.184314}%
\pgfsetstrokecolor{currentstroke}%
\pgfsetstrokeopacity{0.500000}%
\pgfsetdash{}{0pt}%
\pgfpathmoveto{\pgfqpoint{3.856536in}{1.463658in}}%
\pgfusepath{stroke}%
\end{pgfscope}%
\begin{pgfscope}%
\pgfpathrectangle{\pgfqpoint{0.100000in}{0.100000in}}{\pgfqpoint{5.307240in}{3.397500in}}%
\pgfusepath{clip}%
\pgfsetbuttcap%
\pgfsetroundjoin%
\definecolor{currentfill}{rgb}{0.678431,1.000000,0.184314}%
\pgfsetfillcolor{currentfill}%
\pgfsetfillopacity{0.500000}%
\pgfsetlinewidth{0.250937pt}%
\definecolor{currentstroke}{rgb}{0.000000,0.000000,0.000000}%
\pgfsetstrokecolor{currentstroke}%
\pgfsetstrokeopacity{0.500000}%
\pgfsetdash{}{0pt}%
\pgfsys@defobject{currentmarker}{\pgfqpoint{-0.057639in}{-0.057639in}}{\pgfqpoint{0.057639in}{0.057639in}}{%
\pgfpathmoveto{\pgfqpoint{0.000000in}{-0.057639in}}%
\pgfpathcurveto{\pgfqpoint{0.015286in}{-0.057639in}}{\pgfqpoint{0.029948in}{-0.051566in}}{\pgfqpoint{0.040757in}{-0.040757in}}%
\pgfpathcurveto{\pgfqpoint{0.051566in}{-0.029948in}}{\pgfqpoint{0.057639in}{-0.015286in}}{\pgfqpoint{0.057639in}{0.000000in}}%
\pgfpathcurveto{\pgfqpoint{0.057639in}{0.015286in}}{\pgfqpoint{0.051566in}{0.029948in}}{\pgfqpoint{0.040757in}{0.040757in}}%
\pgfpathcurveto{\pgfqpoint{0.029948in}{0.051566in}}{\pgfqpoint{0.015286in}{0.057639in}}{\pgfqpoint{0.000000in}{0.057639in}}%
\pgfpathcurveto{\pgfqpoint{-0.015286in}{0.057639in}}{\pgfqpoint{-0.029948in}{0.051566in}}{\pgfqpoint{-0.040757in}{0.040757in}}%
\pgfpathcurveto{\pgfqpoint{-0.051566in}{0.029948in}}{\pgfqpoint{-0.057639in}{0.015286in}}{\pgfqpoint{-0.057639in}{0.000000in}}%
\pgfpathcurveto{\pgfqpoint{-0.057639in}{-0.015286in}}{\pgfqpoint{-0.051566in}{-0.029948in}}{\pgfqpoint{-0.040757in}{-0.040757in}}%
\pgfpathcurveto{\pgfqpoint{-0.029948in}{-0.051566in}}{\pgfqpoint{-0.015286in}{-0.057639in}}{\pgfqpoint{0.000000in}{-0.057639in}}%
\pgfpathclose%
\pgfusepath{stroke,fill}%
}%
\begin{pgfscope}%
\pgfsys@transformshift{3.856536in}{1.463658in}%
\pgfsys@useobject{currentmarker}{}%
\end{pgfscope}%
\end{pgfscope}%
\begin{pgfscope}%
\pgfpathrectangle{\pgfqpoint{0.100000in}{0.100000in}}{\pgfqpoint{5.307240in}{3.397500in}}%
\pgfusepath{clip}%
\pgfsetrectcap%
\pgfsetroundjoin%
\pgfsetlinewidth{1.505625pt}%
\definecolor{currentstroke}{rgb}{0.678431,1.000000,0.184314}%
\pgfsetstrokecolor{currentstroke}%
\pgfsetstrokeopacity{0.500000}%
\pgfsetdash{}{0pt}%
\pgfpathmoveto{\pgfqpoint{3.753528in}{0.981197in}}%
\pgfusepath{stroke}%
\end{pgfscope}%
\begin{pgfscope}%
\pgfpathrectangle{\pgfqpoint{0.100000in}{0.100000in}}{\pgfqpoint{5.307240in}{3.397500in}}%
\pgfusepath{clip}%
\pgfsetbuttcap%
\pgfsetroundjoin%
\definecolor{currentfill}{rgb}{0.678431,1.000000,0.184314}%
\pgfsetfillcolor{currentfill}%
\pgfsetfillopacity{0.500000}%
\pgfsetlinewidth{0.250937pt}%
\definecolor{currentstroke}{rgb}{0.000000,0.000000,0.000000}%
\pgfsetstrokecolor{currentstroke}%
\pgfsetstrokeopacity{0.500000}%
\pgfsetdash{}{0pt}%
\pgfsys@defobject{currentmarker}{\pgfqpoint{-0.081944in}{-0.081944in}}{\pgfqpoint{0.081944in}{0.081944in}}{%
\pgfpathmoveto{\pgfqpoint{0.000000in}{-0.081944in}}%
\pgfpathcurveto{\pgfqpoint{0.021732in}{-0.081944in}}{\pgfqpoint{0.042577in}{-0.073310in}}{\pgfqpoint{0.057943in}{-0.057943in}}%
\pgfpathcurveto{\pgfqpoint{0.073310in}{-0.042577in}}{\pgfqpoint{0.081944in}{-0.021732in}}{\pgfqpoint{0.081944in}{0.000000in}}%
\pgfpathcurveto{\pgfqpoint{0.081944in}{0.021732in}}{\pgfqpoint{0.073310in}{0.042577in}}{\pgfqpoint{0.057943in}{0.057943in}}%
\pgfpathcurveto{\pgfqpoint{0.042577in}{0.073310in}}{\pgfqpoint{0.021732in}{0.081944in}}{\pgfqpoint{0.000000in}{0.081944in}}%
\pgfpathcurveto{\pgfqpoint{-0.021732in}{0.081944in}}{\pgfqpoint{-0.042577in}{0.073310in}}{\pgfqpoint{-0.057943in}{0.057943in}}%
\pgfpathcurveto{\pgfqpoint{-0.073310in}{0.042577in}}{\pgfqpoint{-0.081944in}{0.021732in}}{\pgfqpoint{-0.081944in}{0.000000in}}%
\pgfpathcurveto{\pgfqpoint{-0.081944in}{-0.021732in}}{\pgfqpoint{-0.073310in}{-0.042577in}}{\pgfqpoint{-0.057943in}{-0.057943in}}%
\pgfpathcurveto{\pgfqpoint{-0.042577in}{-0.073310in}}{\pgfqpoint{-0.021732in}{-0.081944in}}{\pgfqpoint{0.000000in}{-0.081944in}}%
\pgfpathclose%
\pgfusepath{stroke,fill}%
}%
\begin{pgfscope}%
\pgfsys@transformshift{3.753528in}{0.981197in}%
\pgfsys@useobject{currentmarker}{}%
\end{pgfscope}%
\end{pgfscope}%
\begin{pgfscope}%
\pgfpathrectangle{\pgfqpoint{0.100000in}{0.100000in}}{\pgfqpoint{5.307240in}{3.397500in}}%
\pgfusepath{clip}%
\pgfsetrectcap%
\pgfsetroundjoin%
\pgfsetlinewidth{1.505625pt}%
\definecolor{currentstroke}{rgb}{0.678431,1.000000,0.184314}%
\pgfsetstrokecolor{currentstroke}%
\pgfsetstrokeopacity{0.500000}%
\pgfsetdash{}{0pt}%
\pgfpathmoveto{\pgfqpoint{3.910122in}{1.191533in}}%
\pgfusepath{stroke}%
\end{pgfscope}%
\begin{pgfscope}%
\pgfpathrectangle{\pgfqpoint{0.100000in}{0.100000in}}{\pgfqpoint{5.307240in}{3.397500in}}%
\pgfusepath{clip}%
\pgfsetbuttcap%
\pgfsetroundjoin%
\definecolor{currentfill}{rgb}{0.678431,1.000000,0.184314}%
\pgfsetfillcolor{currentfill}%
\pgfsetfillopacity{0.500000}%
\pgfsetlinewidth{0.250937pt}%
\definecolor{currentstroke}{rgb}{0.000000,0.000000,0.000000}%
\pgfsetstrokecolor{currentstroke}%
\pgfsetstrokeopacity{0.500000}%
\pgfsetdash{}{0pt}%
\pgfsys@defobject{currentmarker}{\pgfqpoint{-0.079167in}{-0.079167in}}{\pgfqpoint{0.079167in}{0.079167in}}{%
\pgfpathmoveto{\pgfqpoint{0.000000in}{-0.079167in}}%
\pgfpathcurveto{\pgfqpoint{0.020995in}{-0.079167in}}{\pgfqpoint{0.041133in}{-0.070825in}}{\pgfqpoint{0.055979in}{-0.055979in}}%
\pgfpathcurveto{\pgfqpoint{0.070825in}{-0.041133in}}{\pgfqpoint{0.079167in}{-0.020995in}}{\pgfqpoint{0.079167in}{0.000000in}}%
\pgfpathcurveto{\pgfqpoint{0.079167in}{0.020995in}}{\pgfqpoint{0.070825in}{0.041133in}}{\pgfqpoint{0.055979in}{0.055979in}}%
\pgfpathcurveto{\pgfqpoint{0.041133in}{0.070825in}}{\pgfqpoint{0.020995in}{0.079167in}}{\pgfqpoint{0.000000in}{0.079167in}}%
\pgfpathcurveto{\pgfqpoint{-0.020995in}{0.079167in}}{\pgfqpoint{-0.041133in}{0.070825in}}{\pgfqpoint{-0.055979in}{0.055979in}}%
\pgfpathcurveto{\pgfqpoint{-0.070825in}{0.041133in}}{\pgfqpoint{-0.079167in}{0.020995in}}{\pgfqpoint{-0.079167in}{0.000000in}}%
\pgfpathcurveto{\pgfqpoint{-0.079167in}{-0.020995in}}{\pgfqpoint{-0.070825in}{-0.041133in}}{\pgfqpoint{-0.055979in}{-0.055979in}}%
\pgfpathcurveto{\pgfqpoint{-0.041133in}{-0.070825in}}{\pgfqpoint{-0.020995in}{-0.079167in}}{\pgfqpoint{0.000000in}{-0.079167in}}%
\pgfpathclose%
\pgfusepath{stroke,fill}%
}%
\begin{pgfscope}%
\pgfsys@transformshift{3.910122in}{1.191533in}%
\pgfsys@useobject{currentmarker}{}%
\end{pgfscope}%
\end{pgfscope}%
\begin{pgfscope}%
\pgfpathrectangle{\pgfqpoint{0.100000in}{0.100000in}}{\pgfqpoint{5.307240in}{3.397500in}}%
\pgfusepath{clip}%
\pgfsetrectcap%
\pgfsetroundjoin%
\pgfsetlinewidth{1.505625pt}%
\definecolor{currentstroke}{rgb}{0.678431,1.000000,0.184314}%
\pgfsetstrokecolor{currentstroke}%
\pgfsetstrokeopacity{0.500000}%
\pgfsetdash{}{0pt}%
\pgfpathmoveto{\pgfqpoint{3.784708in}{1.285956in}}%
\pgfusepath{stroke}%
\end{pgfscope}%
\begin{pgfscope}%
\pgfpathrectangle{\pgfqpoint{0.100000in}{0.100000in}}{\pgfqpoint{5.307240in}{3.397500in}}%
\pgfusepath{clip}%
\pgfsetbuttcap%
\pgfsetroundjoin%
\definecolor{currentfill}{rgb}{0.678431,1.000000,0.184314}%
\pgfsetfillcolor{currentfill}%
\pgfsetfillopacity{0.500000}%
\pgfsetlinewidth{0.250937pt}%
\definecolor{currentstroke}{rgb}{0.000000,0.000000,0.000000}%
\pgfsetstrokecolor{currentstroke}%
\pgfsetstrokeopacity{0.500000}%
\pgfsetdash{}{0pt}%
\pgfsys@defobject{currentmarker}{\pgfqpoint{-0.100694in}{-0.100694in}}{\pgfqpoint{0.100694in}{0.100694in}}{%
\pgfpathmoveto{\pgfqpoint{0.000000in}{-0.100694in}}%
\pgfpathcurveto{\pgfqpoint{0.026704in}{-0.100694in}}{\pgfqpoint{0.052319in}{-0.090085in}}{\pgfqpoint{0.071202in}{-0.071202in}}%
\pgfpathcurveto{\pgfqpoint{0.090085in}{-0.052319in}}{\pgfqpoint{0.100694in}{-0.026704in}}{\pgfqpoint{0.100694in}{0.000000in}}%
\pgfpathcurveto{\pgfqpoint{0.100694in}{0.026704in}}{\pgfqpoint{0.090085in}{0.052319in}}{\pgfqpoint{0.071202in}{0.071202in}}%
\pgfpathcurveto{\pgfqpoint{0.052319in}{0.090085in}}{\pgfqpoint{0.026704in}{0.100694in}}{\pgfqpoint{0.000000in}{0.100694in}}%
\pgfpathcurveto{\pgfqpoint{-0.026704in}{0.100694in}}{\pgfqpoint{-0.052319in}{0.090085in}}{\pgfqpoint{-0.071202in}{0.071202in}}%
\pgfpathcurveto{\pgfqpoint{-0.090085in}{0.052319in}}{\pgfqpoint{-0.100694in}{0.026704in}}{\pgfqpoint{-0.100694in}{0.000000in}}%
\pgfpathcurveto{\pgfqpoint{-0.100694in}{-0.026704in}}{\pgfqpoint{-0.090085in}{-0.052319in}}{\pgfqpoint{-0.071202in}{-0.071202in}}%
\pgfpathcurveto{\pgfqpoint{-0.052319in}{-0.090085in}}{\pgfqpoint{-0.026704in}{-0.100694in}}{\pgfqpoint{0.000000in}{-0.100694in}}%
\pgfpathclose%
\pgfusepath{stroke,fill}%
}%
\begin{pgfscope}%
\pgfsys@transformshift{3.784708in}{1.285956in}%
\pgfsys@useobject{currentmarker}{}%
\end{pgfscope}%
\end{pgfscope}%
\begin{pgfscope}%
\pgfpathrectangle{\pgfqpoint{0.100000in}{0.100000in}}{\pgfqpoint{5.307240in}{3.397500in}}%
\pgfusepath{clip}%
\pgfsetrectcap%
\pgfsetroundjoin%
\pgfsetlinewidth{1.505625pt}%
\definecolor{currentstroke}{rgb}{0.678431,1.000000,0.184314}%
\pgfsetstrokecolor{currentstroke}%
\pgfsetstrokeopacity{0.500000}%
\pgfsetdash{}{0pt}%
\pgfpathmoveto{\pgfqpoint{1.292428in}{0.590564in}}%
\pgfusepath{stroke}%
\end{pgfscope}%
\begin{pgfscope}%
\pgfpathrectangle{\pgfqpoint{0.100000in}{0.100000in}}{\pgfqpoint{5.307240in}{3.397500in}}%
\pgfusepath{clip}%
\pgfsetbuttcap%
\pgfsetroundjoin%
\definecolor{currentfill}{rgb}{0.678431,1.000000,0.184314}%
\pgfsetfillcolor{currentfill}%
\pgfsetfillopacity{0.500000}%
\pgfsetlinewidth{0.250937pt}%
\definecolor{currentstroke}{rgb}{0.000000,0.000000,0.000000}%
\pgfsetstrokecolor{currentstroke}%
\pgfsetstrokeopacity{0.500000}%
\pgfsetdash{}{0pt}%
\pgfsys@defobject{currentmarker}{\pgfqpoint{-0.061111in}{-0.061111in}}{\pgfqpoint{0.061111in}{0.061111in}}{%
\pgfpathmoveto{\pgfqpoint{0.000000in}{-0.061111in}}%
\pgfpathcurveto{\pgfqpoint{0.016207in}{-0.061111in}}{\pgfqpoint{0.031752in}{-0.054672in}}{\pgfqpoint{0.043212in}{-0.043212in}}%
\pgfpathcurveto{\pgfqpoint{0.054672in}{-0.031752in}}{\pgfqpoint{0.061111in}{-0.016207in}}{\pgfqpoint{0.061111in}{0.000000in}}%
\pgfpathcurveto{\pgfqpoint{0.061111in}{0.016207in}}{\pgfqpoint{0.054672in}{0.031752in}}{\pgfqpoint{0.043212in}{0.043212in}}%
\pgfpathcurveto{\pgfqpoint{0.031752in}{0.054672in}}{\pgfqpoint{0.016207in}{0.061111in}}{\pgfqpoint{0.000000in}{0.061111in}}%
\pgfpathcurveto{\pgfqpoint{-0.016207in}{0.061111in}}{\pgfqpoint{-0.031752in}{0.054672in}}{\pgfqpoint{-0.043212in}{0.043212in}}%
\pgfpathcurveto{\pgfqpoint{-0.054672in}{0.031752in}}{\pgfqpoint{-0.061111in}{0.016207in}}{\pgfqpoint{-0.061111in}{0.000000in}}%
\pgfpathcurveto{\pgfqpoint{-0.061111in}{-0.016207in}}{\pgfqpoint{-0.054672in}{-0.031752in}}{\pgfqpoint{-0.043212in}{-0.043212in}}%
\pgfpathcurveto{\pgfqpoint{-0.031752in}{-0.054672in}}{\pgfqpoint{-0.016207in}{-0.061111in}}{\pgfqpoint{0.000000in}{-0.061111in}}%
\pgfpathclose%
\pgfusepath{stroke,fill}%
}%
\begin{pgfscope}%
\pgfsys@transformshift{1.292428in}{0.590564in}%
\pgfsys@useobject{currentmarker}{}%
\end{pgfscope}%
\end{pgfscope}%
\begin{pgfscope}%
\pgfpathrectangle{\pgfqpoint{0.100000in}{0.100000in}}{\pgfqpoint{5.307240in}{3.397500in}}%
\pgfusepath{clip}%
\pgfsetrectcap%
\pgfsetroundjoin%
\pgfsetlinewidth{1.505625pt}%
\definecolor{currentstroke}{rgb}{0.678431,1.000000,0.184314}%
\pgfsetstrokecolor{currentstroke}%
\pgfsetstrokeopacity{0.500000}%
\pgfsetdash{}{0pt}%
\pgfpathmoveto{\pgfqpoint{1.417161in}{0.698339in}}%
\pgfusepath{stroke}%
\end{pgfscope}%
\begin{pgfscope}%
\pgfpathrectangle{\pgfqpoint{0.100000in}{0.100000in}}{\pgfqpoint{5.307240in}{3.397500in}}%
\pgfusepath{clip}%
\pgfsetbuttcap%
\pgfsetroundjoin%
\definecolor{currentfill}{rgb}{0.678431,1.000000,0.184314}%
\pgfsetfillcolor{currentfill}%
\pgfsetfillopacity{0.500000}%
\pgfsetlinewidth{0.250937pt}%
\definecolor{currentstroke}{rgb}{0.000000,0.000000,0.000000}%
\pgfsetstrokecolor{currentstroke}%
\pgfsetstrokeopacity{0.500000}%
\pgfsetdash{}{0pt}%
\pgfsys@defobject{currentmarker}{\pgfqpoint{-0.038889in}{-0.038889in}}{\pgfqpoint{0.038889in}{0.038889in}}{%
\pgfpathmoveto{\pgfqpoint{0.000000in}{-0.038889in}}%
\pgfpathcurveto{\pgfqpoint{0.010313in}{-0.038889in}}{\pgfqpoint{0.020206in}{-0.034791in}}{\pgfqpoint{0.027499in}{-0.027499in}}%
\pgfpathcurveto{\pgfqpoint{0.034791in}{-0.020206in}}{\pgfqpoint{0.038889in}{-0.010313in}}{\pgfqpoint{0.038889in}{0.000000in}}%
\pgfpathcurveto{\pgfqpoint{0.038889in}{0.010313in}}{\pgfqpoint{0.034791in}{0.020206in}}{\pgfqpoint{0.027499in}{0.027499in}}%
\pgfpathcurveto{\pgfqpoint{0.020206in}{0.034791in}}{\pgfqpoint{0.010313in}{0.038889in}}{\pgfqpoint{0.000000in}{0.038889in}}%
\pgfpathcurveto{\pgfqpoint{-0.010313in}{0.038889in}}{\pgfqpoint{-0.020206in}{0.034791in}}{\pgfqpoint{-0.027499in}{0.027499in}}%
\pgfpathcurveto{\pgfqpoint{-0.034791in}{0.020206in}}{\pgfqpoint{-0.038889in}{0.010313in}}{\pgfqpoint{-0.038889in}{0.000000in}}%
\pgfpathcurveto{\pgfqpoint{-0.038889in}{-0.010313in}}{\pgfqpoint{-0.034791in}{-0.020206in}}{\pgfqpoint{-0.027499in}{-0.027499in}}%
\pgfpathcurveto{\pgfqpoint{-0.020206in}{-0.034791in}}{\pgfqpoint{-0.010313in}{-0.038889in}}{\pgfqpoint{0.000000in}{-0.038889in}}%
\pgfpathclose%
\pgfusepath{stroke,fill}%
}%
\begin{pgfscope}%
\pgfsys@transformshift{1.417161in}{0.698339in}%
\pgfsys@useobject{currentmarker}{}%
\end{pgfscope}%
\end{pgfscope}%
\begin{pgfscope}%
\pgfpathrectangle{\pgfqpoint{0.100000in}{0.100000in}}{\pgfqpoint{5.307240in}{3.397500in}}%
\pgfusepath{clip}%
\pgfsetrectcap%
\pgfsetroundjoin%
\pgfsetlinewidth{1.505625pt}%
\definecolor{currentstroke}{rgb}{0.678431,1.000000,0.184314}%
\pgfsetstrokecolor{currentstroke}%
\pgfsetstrokeopacity{0.500000}%
\pgfsetdash{}{0pt}%
\pgfpathmoveto{\pgfqpoint{1.475027in}{1.625932in}}%
\pgfusepath{stroke}%
\end{pgfscope}%
\begin{pgfscope}%
\pgfpathrectangle{\pgfqpoint{0.100000in}{0.100000in}}{\pgfqpoint{5.307240in}{3.397500in}}%
\pgfusepath{clip}%
\pgfsetbuttcap%
\pgfsetroundjoin%
\definecolor{currentfill}{rgb}{0.678431,1.000000,0.184314}%
\pgfsetfillcolor{currentfill}%
\pgfsetfillopacity{0.500000}%
\pgfsetlinewidth{0.250937pt}%
\definecolor{currentstroke}{rgb}{0.000000,0.000000,0.000000}%
\pgfsetstrokecolor{currentstroke}%
\pgfsetstrokeopacity{0.500000}%
\pgfsetdash{}{0pt}%
\pgfsys@defobject{currentmarker}{\pgfqpoint{-0.082639in}{-0.082639in}}{\pgfqpoint{0.082639in}{0.082639in}}{%
\pgfpathmoveto{\pgfqpoint{0.000000in}{-0.082639in}}%
\pgfpathcurveto{\pgfqpoint{0.021916in}{-0.082639in}}{\pgfqpoint{0.042938in}{-0.073932in}}{\pgfqpoint{0.058435in}{-0.058435in}}%
\pgfpathcurveto{\pgfqpoint{0.073932in}{-0.042938in}}{\pgfqpoint{0.082639in}{-0.021916in}}{\pgfqpoint{0.082639in}{0.000000in}}%
\pgfpathcurveto{\pgfqpoint{0.082639in}{0.021916in}}{\pgfqpoint{0.073932in}{0.042938in}}{\pgfqpoint{0.058435in}{0.058435in}}%
\pgfpathcurveto{\pgfqpoint{0.042938in}{0.073932in}}{\pgfqpoint{0.021916in}{0.082639in}}{\pgfqpoint{0.000000in}{0.082639in}}%
\pgfpathcurveto{\pgfqpoint{-0.021916in}{0.082639in}}{\pgfqpoint{-0.042938in}{0.073932in}}{\pgfqpoint{-0.058435in}{0.058435in}}%
\pgfpathcurveto{\pgfqpoint{-0.073932in}{0.042938in}}{\pgfqpoint{-0.082639in}{0.021916in}}{\pgfqpoint{-0.082639in}{0.000000in}}%
\pgfpathcurveto{\pgfqpoint{-0.082639in}{-0.021916in}}{\pgfqpoint{-0.073932in}{-0.042938in}}{\pgfqpoint{-0.058435in}{-0.058435in}}%
\pgfpathcurveto{\pgfqpoint{-0.042938in}{-0.073932in}}{\pgfqpoint{-0.021916in}{-0.082639in}}{\pgfqpoint{0.000000in}{-0.082639in}}%
\pgfpathclose%
\pgfusepath{stroke,fill}%
}%
\begin{pgfscope}%
\pgfsys@transformshift{1.475027in}{1.625932in}%
\pgfsys@useobject{currentmarker}{}%
\end{pgfscope}%
\end{pgfscope}%
\begin{pgfscope}%
\pgfpathrectangle{\pgfqpoint{0.100000in}{0.100000in}}{\pgfqpoint{5.307240in}{3.397500in}}%
\pgfusepath{clip}%
\pgfsetrectcap%
\pgfsetroundjoin%
\pgfsetlinewidth{1.505625pt}%
\definecolor{currentstroke}{rgb}{0.678431,1.000000,0.184314}%
\pgfsetstrokecolor{currentstroke}%
\pgfsetstrokeopacity{0.500000}%
\pgfsetdash{}{0pt}%
\pgfpathmoveto{\pgfqpoint{1.205834in}{1.593747in}}%
\pgfusepath{stroke}%
\end{pgfscope}%
\begin{pgfscope}%
\pgfpathrectangle{\pgfqpoint{0.100000in}{0.100000in}}{\pgfqpoint{5.307240in}{3.397500in}}%
\pgfusepath{clip}%
\pgfsetbuttcap%
\pgfsetroundjoin%
\definecolor{currentfill}{rgb}{0.678431,1.000000,0.184314}%
\pgfsetfillcolor{currentfill}%
\pgfsetfillopacity{0.500000}%
\pgfsetlinewidth{0.250937pt}%
\definecolor{currentstroke}{rgb}{0.000000,0.000000,0.000000}%
\pgfsetstrokecolor{currentstroke}%
\pgfsetstrokeopacity{0.500000}%
\pgfsetdash{}{0pt}%
\pgfsys@defobject{currentmarker}{\pgfqpoint{-0.088889in}{-0.088889in}}{\pgfqpoint{0.088889in}{0.088889in}}{%
\pgfpathmoveto{\pgfqpoint{0.000000in}{-0.088889in}}%
\pgfpathcurveto{\pgfqpoint{0.023574in}{-0.088889in}}{\pgfqpoint{0.046185in}{-0.079523in}}{\pgfqpoint{0.062854in}{-0.062854in}}%
\pgfpathcurveto{\pgfqpoint{0.079523in}{-0.046185in}}{\pgfqpoint{0.088889in}{-0.023574in}}{\pgfqpoint{0.088889in}{0.000000in}}%
\pgfpathcurveto{\pgfqpoint{0.088889in}{0.023574in}}{\pgfqpoint{0.079523in}{0.046185in}}{\pgfqpoint{0.062854in}{0.062854in}}%
\pgfpathcurveto{\pgfqpoint{0.046185in}{0.079523in}}{\pgfqpoint{0.023574in}{0.088889in}}{\pgfqpoint{0.000000in}{0.088889in}}%
\pgfpathcurveto{\pgfqpoint{-0.023574in}{0.088889in}}{\pgfqpoint{-0.046185in}{0.079523in}}{\pgfqpoint{-0.062854in}{0.062854in}}%
\pgfpathcurveto{\pgfqpoint{-0.079523in}{0.046185in}}{\pgfqpoint{-0.088889in}{0.023574in}}{\pgfqpoint{-0.088889in}{0.000000in}}%
\pgfpathcurveto{\pgfqpoint{-0.088889in}{-0.023574in}}{\pgfqpoint{-0.079523in}{-0.046185in}}{\pgfqpoint{-0.062854in}{-0.062854in}}%
\pgfpathcurveto{\pgfqpoint{-0.046185in}{-0.079523in}}{\pgfqpoint{-0.023574in}{-0.088889in}}{\pgfqpoint{0.000000in}{-0.088889in}}%
\pgfpathclose%
\pgfusepath{stroke,fill}%
}%
\begin{pgfscope}%
\pgfsys@transformshift{1.205834in}{1.593747in}%
\pgfsys@useobject{currentmarker}{}%
\end{pgfscope}%
\end{pgfscope}%
\begin{pgfscope}%
\pgfpathrectangle{\pgfqpoint{0.100000in}{0.100000in}}{\pgfqpoint{5.307240in}{3.397500in}}%
\pgfusepath{clip}%
\pgfsetrectcap%
\pgfsetroundjoin%
\pgfsetlinewidth{1.505625pt}%
\definecolor{currentstroke}{rgb}{0.678431,1.000000,0.184314}%
\pgfsetstrokecolor{currentstroke}%
\pgfsetstrokeopacity{0.500000}%
\pgfsetdash{}{0pt}%
\pgfpathmoveto{\pgfqpoint{1.397019in}{1.432775in}}%
\pgfusepath{stroke}%
\end{pgfscope}%
\begin{pgfscope}%
\pgfpathrectangle{\pgfqpoint{0.100000in}{0.100000in}}{\pgfqpoint{5.307240in}{3.397500in}}%
\pgfusepath{clip}%
\pgfsetbuttcap%
\pgfsetroundjoin%
\definecolor{currentfill}{rgb}{0.678431,1.000000,0.184314}%
\pgfsetfillcolor{currentfill}%
\pgfsetfillopacity{0.500000}%
\pgfsetlinewidth{0.250937pt}%
\definecolor{currentstroke}{rgb}{0.000000,0.000000,0.000000}%
\pgfsetstrokecolor{currentstroke}%
\pgfsetstrokeopacity{0.500000}%
\pgfsetdash{}{0pt}%
\pgfsys@defobject{currentmarker}{\pgfqpoint{-0.058333in}{-0.058333in}}{\pgfqpoint{0.058333in}{0.058333in}}{%
\pgfpathmoveto{\pgfqpoint{0.000000in}{-0.058333in}}%
\pgfpathcurveto{\pgfqpoint{0.015470in}{-0.058333in}}{\pgfqpoint{0.030309in}{-0.052187in}}{\pgfqpoint{0.041248in}{-0.041248in}}%
\pgfpathcurveto{\pgfqpoint{0.052187in}{-0.030309in}}{\pgfqpoint{0.058333in}{-0.015470in}}{\pgfqpoint{0.058333in}{0.000000in}}%
\pgfpathcurveto{\pgfqpoint{0.058333in}{0.015470in}}{\pgfqpoint{0.052187in}{0.030309in}}{\pgfqpoint{0.041248in}{0.041248in}}%
\pgfpathcurveto{\pgfqpoint{0.030309in}{0.052187in}}{\pgfqpoint{0.015470in}{0.058333in}}{\pgfqpoint{0.000000in}{0.058333in}}%
\pgfpathcurveto{\pgfqpoint{-0.015470in}{0.058333in}}{\pgfqpoint{-0.030309in}{0.052187in}}{\pgfqpoint{-0.041248in}{0.041248in}}%
\pgfpathcurveto{\pgfqpoint{-0.052187in}{0.030309in}}{\pgfqpoint{-0.058333in}{0.015470in}}{\pgfqpoint{-0.058333in}{0.000000in}}%
\pgfpathcurveto{\pgfqpoint{-0.058333in}{-0.015470in}}{\pgfqpoint{-0.052187in}{-0.030309in}}{\pgfqpoint{-0.041248in}{-0.041248in}}%
\pgfpathcurveto{\pgfqpoint{-0.030309in}{-0.052187in}}{\pgfqpoint{-0.015470in}{-0.058333in}}{\pgfqpoint{0.000000in}{-0.058333in}}%
\pgfpathclose%
\pgfusepath{stroke,fill}%
}%
\begin{pgfscope}%
\pgfsys@transformshift{1.397019in}{1.432775in}%
\pgfsys@useobject{currentmarker}{}%
\end{pgfscope}%
\end{pgfscope}%
\begin{pgfscope}%
\pgfpathrectangle{\pgfqpoint{0.100000in}{0.100000in}}{\pgfqpoint{5.307240in}{3.397500in}}%
\pgfusepath{clip}%
\pgfsetrectcap%
\pgfsetroundjoin%
\pgfsetlinewidth{1.505625pt}%
\definecolor{currentstroke}{rgb}{0.678431,1.000000,0.184314}%
\pgfsetstrokecolor{currentstroke}%
\pgfsetstrokeopacity{0.500000}%
\pgfsetdash{}{0pt}%
\pgfpathmoveto{\pgfqpoint{1.383829in}{1.565051in}}%
\pgfusepath{stroke}%
\end{pgfscope}%
\begin{pgfscope}%
\pgfpathrectangle{\pgfqpoint{0.100000in}{0.100000in}}{\pgfqpoint{5.307240in}{3.397500in}}%
\pgfusepath{clip}%
\pgfsetbuttcap%
\pgfsetroundjoin%
\definecolor{currentfill}{rgb}{0.678431,1.000000,0.184314}%
\pgfsetfillcolor{currentfill}%
\pgfsetfillopacity{0.500000}%
\pgfsetlinewidth{0.250937pt}%
\definecolor{currentstroke}{rgb}{0.000000,0.000000,0.000000}%
\pgfsetstrokecolor{currentstroke}%
\pgfsetstrokeopacity{0.500000}%
\pgfsetdash{}{0pt}%
\pgfsys@defobject{currentmarker}{\pgfqpoint{-0.065972in}{-0.065972in}}{\pgfqpoint{0.065972in}{0.065972in}}{%
\pgfpathmoveto{\pgfqpoint{0.000000in}{-0.065972in}}%
\pgfpathcurveto{\pgfqpoint{0.017496in}{-0.065972in}}{\pgfqpoint{0.034278in}{-0.059021in}}{\pgfqpoint{0.046649in}{-0.046649in}}%
\pgfpathcurveto{\pgfqpoint{0.059021in}{-0.034278in}}{\pgfqpoint{0.065972in}{-0.017496in}}{\pgfqpoint{0.065972in}{0.000000in}}%
\pgfpathcurveto{\pgfqpoint{0.065972in}{0.017496in}}{\pgfqpoint{0.059021in}{0.034278in}}{\pgfqpoint{0.046649in}{0.046649in}}%
\pgfpathcurveto{\pgfqpoint{0.034278in}{0.059021in}}{\pgfqpoint{0.017496in}{0.065972in}}{\pgfqpoint{0.000000in}{0.065972in}}%
\pgfpathcurveto{\pgfqpoint{-0.017496in}{0.065972in}}{\pgfqpoint{-0.034278in}{0.059021in}}{\pgfqpoint{-0.046649in}{0.046649in}}%
\pgfpathcurveto{\pgfqpoint{-0.059021in}{0.034278in}}{\pgfqpoint{-0.065972in}{0.017496in}}{\pgfqpoint{-0.065972in}{0.000000in}}%
\pgfpathcurveto{\pgfqpoint{-0.065972in}{-0.017496in}}{\pgfqpoint{-0.059021in}{-0.034278in}}{\pgfqpoint{-0.046649in}{-0.046649in}}%
\pgfpathcurveto{\pgfqpoint{-0.034278in}{-0.059021in}}{\pgfqpoint{-0.017496in}{-0.065972in}}{\pgfqpoint{0.000000in}{-0.065972in}}%
\pgfpathclose%
\pgfusepath{stroke,fill}%
}%
\begin{pgfscope}%
\pgfsys@transformshift{1.383829in}{1.565051in}%
\pgfsys@useobject{currentmarker}{}%
\end{pgfscope}%
\end{pgfscope}%
\begin{pgfscope}%
\pgfpathrectangle{\pgfqpoint{0.100000in}{0.100000in}}{\pgfqpoint{5.307240in}{3.397500in}}%
\pgfusepath{clip}%
\pgfsetrectcap%
\pgfsetroundjoin%
\pgfsetlinewidth{1.505625pt}%
\definecolor{currentstroke}{rgb}{0.678431,1.000000,0.184314}%
\pgfsetstrokecolor{currentstroke}%
\pgfsetstrokeopacity{0.500000}%
\pgfsetdash{}{0pt}%
\pgfpathmoveto{\pgfqpoint{1.529989in}{1.183667in}}%
\pgfusepath{stroke}%
\end{pgfscope}%
\begin{pgfscope}%
\pgfpathrectangle{\pgfqpoint{0.100000in}{0.100000in}}{\pgfqpoint{5.307240in}{3.397500in}}%
\pgfusepath{clip}%
\pgfsetbuttcap%
\pgfsetroundjoin%
\definecolor{currentfill}{rgb}{0.678431,1.000000,0.184314}%
\pgfsetfillcolor{currentfill}%
\pgfsetfillopacity{0.500000}%
\pgfsetlinewidth{0.250937pt}%
\definecolor{currentstroke}{rgb}{0.000000,0.000000,0.000000}%
\pgfsetstrokecolor{currentstroke}%
\pgfsetstrokeopacity{0.500000}%
\pgfsetdash{}{0pt}%
\pgfsys@defobject{currentmarker}{\pgfqpoint{-0.035417in}{-0.035417in}}{\pgfqpoint{0.035417in}{0.035417in}}{%
\pgfpathmoveto{\pgfqpoint{0.000000in}{-0.035417in}}%
\pgfpathcurveto{\pgfqpoint{0.009393in}{-0.035417in}}{\pgfqpoint{0.018402in}{-0.031685in}}{\pgfqpoint{0.025043in}{-0.025043in}}%
\pgfpathcurveto{\pgfqpoint{0.031685in}{-0.018402in}}{\pgfqpoint{0.035417in}{-0.009393in}}{\pgfqpoint{0.035417in}{0.000000in}}%
\pgfpathcurveto{\pgfqpoint{0.035417in}{0.009393in}}{\pgfqpoint{0.031685in}{0.018402in}}{\pgfqpoint{0.025043in}{0.025043in}}%
\pgfpathcurveto{\pgfqpoint{0.018402in}{0.031685in}}{\pgfqpoint{0.009393in}{0.035417in}}{\pgfqpoint{0.000000in}{0.035417in}}%
\pgfpathcurveto{\pgfqpoint{-0.009393in}{0.035417in}}{\pgfqpoint{-0.018402in}{0.031685in}}{\pgfqpoint{-0.025043in}{0.025043in}}%
\pgfpathcurveto{\pgfqpoint{-0.031685in}{0.018402in}}{\pgfqpoint{-0.035417in}{0.009393in}}{\pgfqpoint{-0.035417in}{0.000000in}}%
\pgfpathcurveto{\pgfqpoint{-0.035417in}{-0.009393in}}{\pgfqpoint{-0.031685in}{-0.018402in}}{\pgfqpoint{-0.025043in}{-0.025043in}}%
\pgfpathcurveto{\pgfqpoint{-0.018402in}{-0.031685in}}{\pgfqpoint{-0.009393in}{-0.035417in}}{\pgfqpoint{0.000000in}{-0.035417in}}%
\pgfpathclose%
\pgfusepath{stroke,fill}%
}%
\begin{pgfscope}%
\pgfsys@transformshift{1.529989in}{1.183667in}%
\pgfsys@useobject{currentmarker}{}%
\end{pgfscope}%
\end{pgfscope}%
\begin{pgfscope}%
\pgfpathrectangle{\pgfqpoint{0.100000in}{0.100000in}}{\pgfqpoint{5.307240in}{3.397500in}}%
\pgfusepath{clip}%
\pgfsetrectcap%
\pgfsetroundjoin%
\pgfsetlinewidth{1.505625pt}%
\definecolor{currentstroke}{rgb}{0.678431,1.000000,0.184314}%
\pgfsetstrokecolor{currentstroke}%
\pgfsetstrokeopacity{0.500000}%
\pgfsetdash{}{0pt}%
\pgfpathmoveto{\pgfqpoint{1.482321in}{1.271266in}}%
\pgfusepath{stroke}%
\end{pgfscope}%
\begin{pgfscope}%
\pgfpathrectangle{\pgfqpoint{0.100000in}{0.100000in}}{\pgfqpoint{5.307240in}{3.397500in}}%
\pgfusepath{clip}%
\pgfsetbuttcap%
\pgfsetroundjoin%
\definecolor{currentfill}{rgb}{0.678431,1.000000,0.184314}%
\pgfsetfillcolor{currentfill}%
\pgfsetfillopacity{0.500000}%
\pgfsetlinewidth{0.250937pt}%
\definecolor{currentstroke}{rgb}{0.000000,0.000000,0.000000}%
\pgfsetstrokecolor{currentstroke}%
\pgfsetstrokeopacity{0.500000}%
\pgfsetdash{}{0pt}%
\pgfsys@defobject{currentmarker}{\pgfqpoint{-0.058333in}{-0.058333in}}{\pgfqpoint{0.058333in}{0.058333in}}{%
\pgfpathmoveto{\pgfqpoint{0.000000in}{-0.058333in}}%
\pgfpathcurveto{\pgfqpoint{0.015470in}{-0.058333in}}{\pgfqpoint{0.030309in}{-0.052187in}}{\pgfqpoint{0.041248in}{-0.041248in}}%
\pgfpathcurveto{\pgfqpoint{0.052187in}{-0.030309in}}{\pgfqpoint{0.058333in}{-0.015470in}}{\pgfqpoint{0.058333in}{0.000000in}}%
\pgfpathcurveto{\pgfqpoint{0.058333in}{0.015470in}}{\pgfqpoint{0.052187in}{0.030309in}}{\pgfqpoint{0.041248in}{0.041248in}}%
\pgfpathcurveto{\pgfqpoint{0.030309in}{0.052187in}}{\pgfqpoint{0.015470in}{0.058333in}}{\pgfqpoint{0.000000in}{0.058333in}}%
\pgfpathcurveto{\pgfqpoint{-0.015470in}{0.058333in}}{\pgfqpoint{-0.030309in}{0.052187in}}{\pgfqpoint{-0.041248in}{0.041248in}}%
\pgfpathcurveto{\pgfqpoint{-0.052187in}{0.030309in}}{\pgfqpoint{-0.058333in}{0.015470in}}{\pgfqpoint{-0.058333in}{0.000000in}}%
\pgfpathcurveto{\pgfqpoint{-0.058333in}{-0.015470in}}{\pgfqpoint{-0.052187in}{-0.030309in}}{\pgfqpoint{-0.041248in}{-0.041248in}}%
\pgfpathcurveto{\pgfqpoint{-0.030309in}{-0.052187in}}{\pgfqpoint{-0.015470in}{-0.058333in}}{\pgfqpoint{0.000000in}{-0.058333in}}%
\pgfpathclose%
\pgfusepath{stroke,fill}%
}%
\begin{pgfscope}%
\pgfsys@transformshift{1.482321in}{1.271266in}%
\pgfsys@useobject{currentmarker}{}%
\end{pgfscope}%
\end{pgfscope}%
\begin{pgfscope}%
\pgfpathrectangle{\pgfqpoint{0.100000in}{0.100000in}}{\pgfqpoint{5.307240in}{3.397500in}}%
\pgfusepath{clip}%
\pgfsetrectcap%
\pgfsetroundjoin%
\pgfsetlinewidth{1.505625pt}%
\definecolor{currentstroke}{rgb}{0.678431,1.000000,0.184314}%
\pgfsetstrokecolor{currentstroke}%
\pgfsetstrokeopacity{0.500000}%
\pgfsetdash{}{0pt}%
\pgfpathmoveto{\pgfqpoint{1.148496in}{1.389996in}}%
\pgfusepath{stroke}%
\end{pgfscope}%
\begin{pgfscope}%
\pgfpathrectangle{\pgfqpoint{0.100000in}{0.100000in}}{\pgfqpoint{5.307240in}{3.397500in}}%
\pgfusepath{clip}%
\pgfsetbuttcap%
\pgfsetroundjoin%
\definecolor{currentfill}{rgb}{0.678431,1.000000,0.184314}%
\pgfsetfillcolor{currentfill}%
\pgfsetfillopacity{0.500000}%
\pgfsetlinewidth{0.250937pt}%
\definecolor{currentstroke}{rgb}{0.000000,0.000000,0.000000}%
\pgfsetstrokecolor{currentstroke}%
\pgfsetstrokeopacity{0.500000}%
\pgfsetdash{}{0pt}%
\pgfsys@defobject{currentmarker}{\pgfqpoint{-0.056250in}{-0.056250in}}{\pgfqpoint{0.056250in}{0.056250in}}{%
\pgfpathmoveto{\pgfqpoint{0.000000in}{-0.056250in}}%
\pgfpathcurveto{\pgfqpoint{0.014918in}{-0.056250in}}{\pgfqpoint{0.029226in}{-0.050323in}}{\pgfqpoint{0.039775in}{-0.039775in}}%
\pgfpathcurveto{\pgfqpoint{0.050323in}{-0.029226in}}{\pgfqpoint{0.056250in}{-0.014918in}}{\pgfqpoint{0.056250in}{0.000000in}}%
\pgfpathcurveto{\pgfqpoint{0.056250in}{0.014918in}}{\pgfqpoint{0.050323in}{0.029226in}}{\pgfqpoint{0.039775in}{0.039775in}}%
\pgfpathcurveto{\pgfqpoint{0.029226in}{0.050323in}}{\pgfqpoint{0.014918in}{0.056250in}}{\pgfqpoint{0.000000in}{0.056250in}}%
\pgfpathcurveto{\pgfqpoint{-0.014918in}{0.056250in}}{\pgfqpoint{-0.029226in}{0.050323in}}{\pgfqpoint{-0.039775in}{0.039775in}}%
\pgfpathcurveto{\pgfqpoint{-0.050323in}{0.029226in}}{\pgfqpoint{-0.056250in}{0.014918in}}{\pgfqpoint{-0.056250in}{0.000000in}}%
\pgfpathcurveto{\pgfqpoint{-0.056250in}{-0.014918in}}{\pgfqpoint{-0.050323in}{-0.029226in}}{\pgfqpoint{-0.039775in}{-0.039775in}}%
\pgfpathcurveto{\pgfqpoint{-0.029226in}{-0.050323in}}{\pgfqpoint{-0.014918in}{-0.056250in}}{\pgfqpoint{0.000000in}{-0.056250in}}%
\pgfpathclose%
\pgfusepath{stroke,fill}%
}%
\begin{pgfscope}%
\pgfsys@transformshift{1.148496in}{1.389996in}%
\pgfsys@useobject{currentmarker}{}%
\end{pgfscope}%
\end{pgfscope}%
\begin{pgfscope}%
\pgfpathrectangle{\pgfqpoint{0.100000in}{0.100000in}}{\pgfqpoint{5.307240in}{3.397500in}}%
\pgfusepath{clip}%
\pgfsetrectcap%
\pgfsetroundjoin%
\pgfsetlinewidth{1.505625pt}%
\definecolor{currentstroke}{rgb}{0.678431,1.000000,0.184314}%
\pgfsetstrokecolor{currentstroke}%
\pgfsetstrokeopacity{0.500000}%
\pgfsetdash{}{0pt}%
\pgfpathmoveto{\pgfqpoint{3.130814in}{1.581982in}}%
\pgfusepath{stroke}%
\end{pgfscope}%
\begin{pgfscope}%
\pgfpathrectangle{\pgfqpoint{0.100000in}{0.100000in}}{\pgfqpoint{5.307240in}{3.397500in}}%
\pgfusepath{clip}%
\pgfsetbuttcap%
\pgfsetroundjoin%
\definecolor{currentfill}{rgb}{0.678431,1.000000,0.184314}%
\pgfsetfillcolor{currentfill}%
\pgfsetfillopacity{0.500000}%
\pgfsetlinewidth{0.250937pt}%
\definecolor{currentstroke}{rgb}{0.000000,0.000000,0.000000}%
\pgfsetstrokecolor{currentstroke}%
\pgfsetstrokeopacity{0.500000}%
\pgfsetdash{}{0pt}%
\pgfsys@defobject{currentmarker}{\pgfqpoint{-0.040278in}{-0.040278in}}{\pgfqpoint{0.040278in}{0.040278in}}{%
\pgfpathmoveto{\pgfqpoint{0.000000in}{-0.040278in}}%
\pgfpathcurveto{\pgfqpoint{0.010682in}{-0.040278in}}{\pgfqpoint{0.020928in}{-0.036034in}}{\pgfqpoint{0.028481in}{-0.028481in}}%
\pgfpathcurveto{\pgfqpoint{0.036034in}{-0.020928in}}{\pgfqpoint{0.040278in}{-0.010682in}}{\pgfqpoint{0.040278in}{0.000000in}}%
\pgfpathcurveto{\pgfqpoint{0.040278in}{0.010682in}}{\pgfqpoint{0.036034in}{0.020928in}}{\pgfqpoint{0.028481in}{0.028481in}}%
\pgfpathcurveto{\pgfqpoint{0.020928in}{0.036034in}}{\pgfqpoint{0.010682in}{0.040278in}}{\pgfqpoint{0.000000in}{0.040278in}}%
\pgfpathcurveto{\pgfqpoint{-0.010682in}{0.040278in}}{\pgfqpoint{-0.020928in}{0.036034in}}{\pgfqpoint{-0.028481in}{0.028481in}}%
\pgfpathcurveto{\pgfqpoint{-0.036034in}{0.020928in}}{\pgfqpoint{-0.040278in}{0.010682in}}{\pgfqpoint{-0.040278in}{0.000000in}}%
\pgfpathcurveto{\pgfqpoint{-0.040278in}{-0.010682in}}{\pgfqpoint{-0.036034in}{-0.020928in}}{\pgfqpoint{-0.028481in}{-0.028481in}}%
\pgfpathcurveto{\pgfqpoint{-0.020928in}{-0.036034in}}{\pgfqpoint{-0.010682in}{-0.040278in}}{\pgfqpoint{0.000000in}{-0.040278in}}%
\pgfpathclose%
\pgfusepath{stroke,fill}%
}%
\begin{pgfscope}%
\pgfsys@transformshift{3.130814in}{1.581982in}%
\pgfsys@useobject{currentmarker}{}%
\end{pgfscope}%
\end{pgfscope}%
\begin{pgfscope}%
\pgfpathrectangle{\pgfqpoint{0.100000in}{0.100000in}}{\pgfqpoint{5.307240in}{3.397500in}}%
\pgfusepath{clip}%
\pgfsetrectcap%
\pgfsetroundjoin%
\pgfsetlinewidth{1.505625pt}%
\definecolor{currentstroke}{rgb}{0.678431,1.000000,0.184314}%
\pgfsetstrokecolor{currentstroke}%
\pgfsetstrokeopacity{0.500000}%
\pgfsetdash{}{0pt}%
\pgfpathmoveto{\pgfqpoint{3.106143in}{1.503143in}}%
\pgfusepath{stroke}%
\end{pgfscope}%
\begin{pgfscope}%
\pgfpathrectangle{\pgfqpoint{0.100000in}{0.100000in}}{\pgfqpoint{5.307240in}{3.397500in}}%
\pgfusepath{clip}%
\pgfsetbuttcap%
\pgfsetroundjoin%
\definecolor{currentfill}{rgb}{0.678431,1.000000,0.184314}%
\pgfsetfillcolor{currentfill}%
\pgfsetfillopacity{0.500000}%
\pgfsetlinewidth{0.250937pt}%
\definecolor{currentstroke}{rgb}{0.000000,0.000000,0.000000}%
\pgfsetstrokecolor{currentstroke}%
\pgfsetstrokeopacity{0.500000}%
\pgfsetdash{}{0pt}%
\pgfsys@defobject{currentmarker}{\pgfqpoint{-0.061806in}{-0.061806in}}{\pgfqpoint{0.061806in}{0.061806in}}{%
\pgfpathmoveto{\pgfqpoint{0.000000in}{-0.061806in}}%
\pgfpathcurveto{\pgfqpoint{0.016391in}{-0.061806in}}{\pgfqpoint{0.032113in}{-0.055293in}}{\pgfqpoint{0.043703in}{-0.043703in}}%
\pgfpathcurveto{\pgfqpoint{0.055293in}{-0.032113in}}{\pgfqpoint{0.061806in}{-0.016391in}}{\pgfqpoint{0.061806in}{0.000000in}}%
\pgfpathcurveto{\pgfqpoint{0.061806in}{0.016391in}}{\pgfqpoint{0.055293in}{0.032113in}}{\pgfqpoint{0.043703in}{0.043703in}}%
\pgfpathcurveto{\pgfqpoint{0.032113in}{0.055293in}}{\pgfqpoint{0.016391in}{0.061806in}}{\pgfqpoint{0.000000in}{0.061806in}}%
\pgfpathcurveto{\pgfqpoint{-0.016391in}{0.061806in}}{\pgfqpoint{-0.032113in}{0.055293in}}{\pgfqpoint{-0.043703in}{0.043703in}}%
\pgfpathcurveto{\pgfqpoint{-0.055293in}{0.032113in}}{\pgfqpoint{-0.061806in}{0.016391in}}{\pgfqpoint{-0.061806in}{0.000000in}}%
\pgfpathcurveto{\pgfqpoint{-0.061806in}{-0.016391in}}{\pgfqpoint{-0.055293in}{-0.032113in}}{\pgfqpoint{-0.043703in}{-0.043703in}}%
\pgfpathcurveto{\pgfqpoint{-0.032113in}{-0.055293in}}{\pgfqpoint{-0.016391in}{-0.061806in}}{\pgfqpoint{0.000000in}{-0.061806in}}%
\pgfpathclose%
\pgfusepath{stroke,fill}%
}%
\begin{pgfscope}%
\pgfsys@transformshift{3.106143in}{1.503143in}%
\pgfsys@useobject{currentmarker}{}%
\end{pgfscope}%
\end{pgfscope}%
\begin{pgfscope}%
\pgfpathrectangle{\pgfqpoint{0.100000in}{0.100000in}}{\pgfqpoint{5.307240in}{3.397500in}}%
\pgfusepath{clip}%
\pgfsetrectcap%
\pgfsetroundjoin%
\pgfsetlinewidth{1.505625pt}%
\definecolor{currentstroke}{rgb}{0.678431,1.000000,0.184314}%
\pgfsetstrokecolor{currentstroke}%
\pgfsetstrokeopacity{0.500000}%
\pgfsetdash{}{0pt}%
\pgfpathmoveto{\pgfqpoint{3.239351in}{1.395526in}}%
\pgfusepath{stroke}%
\end{pgfscope}%
\begin{pgfscope}%
\pgfpathrectangle{\pgfqpoint{0.100000in}{0.100000in}}{\pgfqpoint{5.307240in}{3.397500in}}%
\pgfusepath{clip}%
\pgfsetbuttcap%
\pgfsetroundjoin%
\definecolor{currentfill}{rgb}{0.678431,1.000000,0.184314}%
\pgfsetfillcolor{currentfill}%
\pgfsetfillopacity{0.500000}%
\pgfsetlinewidth{0.250937pt}%
\definecolor{currentstroke}{rgb}{0.000000,0.000000,0.000000}%
\pgfsetstrokecolor{currentstroke}%
\pgfsetstrokeopacity{0.500000}%
\pgfsetdash{}{0pt}%
\pgfsys@defobject{currentmarker}{\pgfqpoint{-0.085417in}{-0.085417in}}{\pgfqpoint{0.085417in}{0.085417in}}{%
\pgfpathmoveto{\pgfqpoint{0.000000in}{-0.085417in}}%
\pgfpathcurveto{\pgfqpoint{0.022653in}{-0.085417in}}{\pgfqpoint{0.044381in}{-0.076417in}}{\pgfqpoint{0.060399in}{-0.060399in}}%
\pgfpathcurveto{\pgfqpoint{0.076417in}{-0.044381in}}{\pgfqpoint{0.085417in}{-0.022653in}}{\pgfqpoint{0.085417in}{0.000000in}}%
\pgfpathcurveto{\pgfqpoint{0.085417in}{0.022653in}}{\pgfqpoint{0.076417in}{0.044381in}}{\pgfqpoint{0.060399in}{0.060399in}}%
\pgfpathcurveto{\pgfqpoint{0.044381in}{0.076417in}}{\pgfqpoint{0.022653in}{0.085417in}}{\pgfqpoint{0.000000in}{0.085417in}}%
\pgfpathcurveto{\pgfqpoint{-0.022653in}{0.085417in}}{\pgfqpoint{-0.044381in}{0.076417in}}{\pgfqpoint{-0.060399in}{0.060399in}}%
\pgfpathcurveto{\pgfqpoint{-0.076417in}{0.044381in}}{\pgfqpoint{-0.085417in}{0.022653in}}{\pgfqpoint{-0.085417in}{0.000000in}}%
\pgfpathcurveto{\pgfqpoint{-0.085417in}{-0.022653in}}{\pgfqpoint{-0.076417in}{-0.044381in}}{\pgfqpoint{-0.060399in}{-0.060399in}}%
\pgfpathcurveto{\pgfqpoint{-0.044381in}{-0.076417in}}{\pgfqpoint{-0.022653in}{-0.085417in}}{\pgfqpoint{0.000000in}{-0.085417in}}%
\pgfpathclose%
\pgfusepath{stroke,fill}%
}%
\begin{pgfscope}%
\pgfsys@transformshift{3.239351in}{1.395526in}%
\pgfsys@useobject{currentmarker}{}%
\end{pgfscope}%
\end{pgfscope}%
\begin{pgfscope}%
\pgfpathrectangle{\pgfqpoint{0.100000in}{0.100000in}}{\pgfqpoint{5.307240in}{3.397500in}}%
\pgfusepath{clip}%
\pgfsetrectcap%
\pgfsetroundjoin%
\pgfsetlinewidth{1.505625pt}%
\definecolor{currentstroke}{rgb}{0.678431,1.000000,0.184314}%
\pgfsetstrokecolor{currentstroke}%
\pgfsetstrokeopacity{0.500000}%
\pgfsetdash{}{0pt}%
\pgfpathmoveto{\pgfqpoint{3.456901in}{1.565619in}}%
\pgfusepath{stroke}%
\end{pgfscope}%
\begin{pgfscope}%
\pgfpathrectangle{\pgfqpoint{0.100000in}{0.100000in}}{\pgfqpoint{5.307240in}{3.397500in}}%
\pgfusepath{clip}%
\pgfsetbuttcap%
\pgfsetroundjoin%
\definecolor{currentfill}{rgb}{0.678431,1.000000,0.184314}%
\pgfsetfillcolor{currentfill}%
\pgfsetfillopacity{0.500000}%
\pgfsetlinewidth{0.250937pt}%
\definecolor{currentstroke}{rgb}{0.000000,0.000000,0.000000}%
\pgfsetstrokecolor{currentstroke}%
\pgfsetstrokeopacity{0.500000}%
\pgfsetdash{}{0pt}%
\pgfsys@defobject{currentmarker}{\pgfqpoint{-0.045833in}{-0.045833in}}{\pgfqpoint{0.045833in}{0.045833in}}{%
\pgfpathmoveto{\pgfqpoint{0.000000in}{-0.045833in}}%
\pgfpathcurveto{\pgfqpoint{0.012155in}{-0.045833in}}{\pgfqpoint{0.023814in}{-0.041004in}}{\pgfqpoint{0.032409in}{-0.032409in}}%
\pgfpathcurveto{\pgfqpoint{0.041004in}{-0.023814in}}{\pgfqpoint{0.045833in}{-0.012155in}}{\pgfqpoint{0.045833in}{0.000000in}}%
\pgfpathcurveto{\pgfqpoint{0.045833in}{0.012155in}}{\pgfqpoint{0.041004in}{0.023814in}}{\pgfqpoint{0.032409in}{0.032409in}}%
\pgfpathcurveto{\pgfqpoint{0.023814in}{0.041004in}}{\pgfqpoint{0.012155in}{0.045833in}}{\pgfqpoint{0.000000in}{0.045833in}}%
\pgfpathcurveto{\pgfqpoint{-0.012155in}{0.045833in}}{\pgfqpoint{-0.023814in}{0.041004in}}{\pgfqpoint{-0.032409in}{0.032409in}}%
\pgfpathcurveto{\pgfqpoint{-0.041004in}{0.023814in}}{\pgfqpoint{-0.045833in}{0.012155in}}{\pgfqpoint{-0.045833in}{0.000000in}}%
\pgfpathcurveto{\pgfqpoint{-0.045833in}{-0.012155in}}{\pgfqpoint{-0.041004in}{-0.023814in}}{\pgfqpoint{-0.032409in}{-0.032409in}}%
\pgfpathcurveto{\pgfqpoint{-0.023814in}{-0.041004in}}{\pgfqpoint{-0.012155in}{-0.045833in}}{\pgfqpoint{0.000000in}{-0.045833in}}%
\pgfpathclose%
\pgfusepath{stroke,fill}%
}%
\begin{pgfscope}%
\pgfsys@transformshift{3.456901in}{1.565619in}%
\pgfsys@useobject{currentmarker}{}%
\end{pgfscope}%
\end{pgfscope}%
\begin{pgfscope}%
\pgfpathrectangle{\pgfqpoint{0.100000in}{0.100000in}}{\pgfqpoint{5.307240in}{3.397500in}}%
\pgfusepath{clip}%
\pgfsetrectcap%
\pgfsetroundjoin%
\pgfsetlinewidth{1.505625pt}%
\definecolor{currentstroke}{rgb}{0.678431,1.000000,0.184314}%
\pgfsetstrokecolor{currentstroke}%
\pgfsetstrokeopacity{0.500000}%
\pgfsetdash{}{0pt}%
\pgfpathmoveto{\pgfqpoint{3.311117in}{1.432175in}}%
\pgfusepath{stroke}%
\end{pgfscope}%
\begin{pgfscope}%
\pgfpathrectangle{\pgfqpoint{0.100000in}{0.100000in}}{\pgfqpoint{5.307240in}{3.397500in}}%
\pgfusepath{clip}%
\pgfsetbuttcap%
\pgfsetroundjoin%
\definecolor{currentfill}{rgb}{0.678431,1.000000,0.184314}%
\pgfsetfillcolor{currentfill}%
\pgfsetfillopacity{0.500000}%
\pgfsetlinewidth{0.250937pt}%
\definecolor{currentstroke}{rgb}{0.000000,0.000000,0.000000}%
\pgfsetstrokecolor{currentstroke}%
\pgfsetstrokeopacity{0.500000}%
\pgfsetdash{}{0pt}%
\pgfsys@defobject{currentmarker}{\pgfqpoint{-0.056944in}{-0.056944in}}{\pgfqpoint{0.056944in}{0.056944in}}{%
\pgfpathmoveto{\pgfqpoint{0.000000in}{-0.056944in}}%
\pgfpathcurveto{\pgfqpoint{0.015102in}{-0.056944in}}{\pgfqpoint{0.029587in}{-0.050944in}}{\pgfqpoint{0.040266in}{-0.040266in}}%
\pgfpathcurveto{\pgfqpoint{0.050944in}{-0.029587in}}{\pgfqpoint{0.056944in}{-0.015102in}}{\pgfqpoint{0.056944in}{0.000000in}}%
\pgfpathcurveto{\pgfqpoint{0.056944in}{0.015102in}}{\pgfqpoint{0.050944in}{0.029587in}}{\pgfqpoint{0.040266in}{0.040266in}}%
\pgfpathcurveto{\pgfqpoint{0.029587in}{0.050944in}}{\pgfqpoint{0.015102in}{0.056944in}}{\pgfqpoint{0.000000in}{0.056944in}}%
\pgfpathcurveto{\pgfqpoint{-0.015102in}{0.056944in}}{\pgfqpoint{-0.029587in}{0.050944in}}{\pgfqpoint{-0.040266in}{0.040266in}}%
\pgfpathcurveto{\pgfqpoint{-0.050944in}{0.029587in}}{\pgfqpoint{-0.056944in}{0.015102in}}{\pgfqpoint{-0.056944in}{0.000000in}}%
\pgfpathcurveto{\pgfqpoint{-0.056944in}{-0.015102in}}{\pgfqpoint{-0.050944in}{-0.029587in}}{\pgfqpoint{-0.040266in}{-0.040266in}}%
\pgfpathcurveto{\pgfqpoint{-0.029587in}{-0.050944in}}{\pgfqpoint{-0.015102in}{-0.056944in}}{\pgfqpoint{0.000000in}{-0.056944in}}%
\pgfpathclose%
\pgfusepath{stroke,fill}%
}%
\begin{pgfscope}%
\pgfsys@transformshift{3.311117in}{1.432175in}%
\pgfsys@useobject{currentmarker}{}%
\end{pgfscope}%
\end{pgfscope}%
\begin{pgfscope}%
\pgfpathrectangle{\pgfqpoint{0.100000in}{0.100000in}}{\pgfqpoint{5.307240in}{3.397500in}}%
\pgfusepath{clip}%
\pgfsetrectcap%
\pgfsetroundjoin%
\pgfsetlinewidth{1.505625pt}%
\definecolor{currentstroke}{rgb}{0.678431,1.000000,0.184314}%
\pgfsetstrokecolor{currentstroke}%
\pgfsetstrokeopacity{0.500000}%
\pgfsetdash{}{0pt}%
\pgfpathmoveto{\pgfqpoint{3.340529in}{1.372655in}}%
\pgfusepath{stroke}%
\end{pgfscope}%
\begin{pgfscope}%
\pgfpathrectangle{\pgfqpoint{0.100000in}{0.100000in}}{\pgfqpoint{5.307240in}{3.397500in}}%
\pgfusepath{clip}%
\pgfsetbuttcap%
\pgfsetroundjoin%
\definecolor{currentfill}{rgb}{0.678431,1.000000,0.184314}%
\pgfsetfillcolor{currentfill}%
\pgfsetfillopacity{0.500000}%
\pgfsetlinewidth{0.250937pt}%
\definecolor{currentstroke}{rgb}{0.000000,0.000000,0.000000}%
\pgfsetstrokecolor{currentstroke}%
\pgfsetstrokeopacity{0.500000}%
\pgfsetdash{}{0pt}%
\pgfsys@defobject{currentmarker}{\pgfqpoint{-0.039583in}{-0.039583in}}{\pgfqpoint{0.039583in}{0.039583in}}{%
\pgfpathmoveto{\pgfqpoint{0.000000in}{-0.039583in}}%
\pgfpathcurveto{\pgfqpoint{0.010498in}{-0.039583in}}{\pgfqpoint{0.020567in}{-0.035413in}}{\pgfqpoint{0.027990in}{-0.027990in}}%
\pgfpathcurveto{\pgfqpoint{0.035413in}{-0.020567in}}{\pgfqpoint{0.039583in}{-0.010498in}}{\pgfqpoint{0.039583in}{0.000000in}}%
\pgfpathcurveto{\pgfqpoint{0.039583in}{0.010498in}}{\pgfqpoint{0.035413in}{0.020567in}}{\pgfqpoint{0.027990in}{0.027990in}}%
\pgfpathcurveto{\pgfqpoint{0.020567in}{0.035413in}}{\pgfqpoint{0.010498in}{0.039583in}}{\pgfqpoint{0.000000in}{0.039583in}}%
\pgfpathcurveto{\pgfqpoint{-0.010498in}{0.039583in}}{\pgfqpoint{-0.020567in}{0.035413in}}{\pgfqpoint{-0.027990in}{0.027990in}}%
\pgfpathcurveto{\pgfqpoint{-0.035413in}{0.020567in}}{\pgfqpoint{-0.039583in}{0.010498in}}{\pgfqpoint{-0.039583in}{0.000000in}}%
\pgfpathcurveto{\pgfqpoint{-0.039583in}{-0.010498in}}{\pgfqpoint{-0.035413in}{-0.020567in}}{\pgfqpoint{-0.027990in}{-0.027990in}}%
\pgfpathcurveto{\pgfqpoint{-0.020567in}{-0.035413in}}{\pgfqpoint{-0.010498in}{-0.039583in}}{\pgfqpoint{0.000000in}{-0.039583in}}%
\pgfpathclose%
\pgfusepath{stroke,fill}%
}%
\begin{pgfscope}%
\pgfsys@transformshift{3.340529in}{1.372655in}%
\pgfsys@useobject{currentmarker}{}%
\end{pgfscope}%
\end{pgfscope}%
\begin{pgfscope}%
\pgfpathrectangle{\pgfqpoint{0.100000in}{0.100000in}}{\pgfqpoint{5.307240in}{3.397500in}}%
\pgfusepath{clip}%
\pgfsetrectcap%
\pgfsetroundjoin%
\pgfsetlinewidth{1.505625pt}%
\definecolor{currentstroke}{rgb}{0.678431,1.000000,0.184314}%
\pgfsetstrokecolor{currentstroke}%
\pgfsetstrokeopacity{0.500000}%
\pgfsetdash{}{0pt}%
\pgfpathmoveto{\pgfqpoint{0.796514in}{1.801171in}}%
\pgfusepath{stroke}%
\end{pgfscope}%
\begin{pgfscope}%
\pgfpathrectangle{\pgfqpoint{0.100000in}{0.100000in}}{\pgfqpoint{5.307240in}{3.397500in}}%
\pgfusepath{clip}%
\pgfsetbuttcap%
\pgfsetroundjoin%
\definecolor{currentfill}{rgb}{0.678431,1.000000,0.184314}%
\pgfsetfillcolor{currentfill}%
\pgfsetfillopacity{0.500000}%
\pgfsetlinewidth{0.250937pt}%
\definecolor{currentstroke}{rgb}{0.000000,0.000000,0.000000}%
\pgfsetstrokecolor{currentstroke}%
\pgfsetstrokeopacity{0.500000}%
\pgfsetdash{}{0pt}%
\pgfsys@defobject{currentmarker}{\pgfqpoint{-0.072222in}{-0.072222in}}{\pgfqpoint{0.072222in}{0.072222in}}{%
\pgfpathmoveto{\pgfqpoint{0.000000in}{-0.072222in}}%
\pgfpathcurveto{\pgfqpoint{0.019154in}{-0.072222in}}{\pgfqpoint{0.037525in}{-0.064612in}}{\pgfqpoint{0.051069in}{-0.051069in}}%
\pgfpathcurveto{\pgfqpoint{0.064612in}{-0.037525in}}{\pgfqpoint{0.072222in}{-0.019154in}}{\pgfqpoint{0.072222in}{0.000000in}}%
\pgfpathcurveto{\pgfqpoint{0.072222in}{0.019154in}}{\pgfqpoint{0.064612in}{0.037525in}}{\pgfqpoint{0.051069in}{0.051069in}}%
\pgfpathcurveto{\pgfqpoint{0.037525in}{0.064612in}}{\pgfqpoint{0.019154in}{0.072222in}}{\pgfqpoint{0.000000in}{0.072222in}}%
\pgfpathcurveto{\pgfqpoint{-0.019154in}{0.072222in}}{\pgfqpoint{-0.037525in}{0.064612in}}{\pgfqpoint{-0.051069in}{0.051069in}}%
\pgfpathcurveto{\pgfqpoint{-0.064612in}{0.037525in}}{\pgfqpoint{-0.072222in}{0.019154in}}{\pgfqpoint{-0.072222in}{0.000000in}}%
\pgfpathcurveto{\pgfqpoint{-0.072222in}{-0.019154in}}{\pgfqpoint{-0.064612in}{-0.037525in}}{\pgfqpoint{-0.051069in}{-0.051069in}}%
\pgfpathcurveto{\pgfqpoint{-0.037525in}{-0.064612in}}{\pgfqpoint{-0.019154in}{-0.072222in}}{\pgfqpoint{0.000000in}{-0.072222in}}%
\pgfpathclose%
\pgfusepath{stroke,fill}%
}%
\begin{pgfscope}%
\pgfsys@transformshift{0.796514in}{1.801171in}%
\pgfsys@useobject{currentmarker}{}%
\end{pgfscope}%
\end{pgfscope}%
\begin{pgfscope}%
\pgfpathrectangle{\pgfqpoint{0.100000in}{0.100000in}}{\pgfqpoint{5.307240in}{3.397500in}}%
\pgfusepath{clip}%
\pgfsetrectcap%
\pgfsetroundjoin%
\pgfsetlinewidth{1.505625pt}%
\definecolor{currentstroke}{rgb}{0.678431,1.000000,0.184314}%
\pgfsetstrokecolor{currentstroke}%
\pgfsetstrokeopacity{0.500000}%
\pgfsetdash{}{0pt}%
\pgfpathmoveto{\pgfqpoint{0.686686in}{2.359687in}}%
\pgfusepath{stroke}%
\end{pgfscope}%
\begin{pgfscope}%
\pgfpathrectangle{\pgfqpoint{0.100000in}{0.100000in}}{\pgfqpoint{5.307240in}{3.397500in}}%
\pgfusepath{clip}%
\pgfsetbuttcap%
\pgfsetroundjoin%
\definecolor{currentfill}{rgb}{0.678431,1.000000,0.184314}%
\pgfsetfillcolor{currentfill}%
\pgfsetfillopacity{0.500000}%
\pgfsetlinewidth{0.250937pt}%
\definecolor{currentstroke}{rgb}{0.000000,0.000000,0.000000}%
\pgfsetstrokecolor{currentstroke}%
\pgfsetstrokeopacity{0.500000}%
\pgfsetdash{}{0pt}%
\pgfsys@defobject{currentmarker}{\pgfqpoint{-0.068750in}{-0.068750in}}{\pgfqpoint{0.068750in}{0.068750in}}{%
\pgfpathmoveto{\pgfqpoint{0.000000in}{-0.068750in}}%
\pgfpathcurveto{\pgfqpoint{0.018233in}{-0.068750in}}{\pgfqpoint{0.035721in}{-0.061506in}}{\pgfqpoint{0.048614in}{-0.048614in}}%
\pgfpathcurveto{\pgfqpoint{0.061506in}{-0.035721in}}{\pgfqpoint{0.068750in}{-0.018233in}}{\pgfqpoint{0.068750in}{0.000000in}}%
\pgfpathcurveto{\pgfqpoint{0.068750in}{0.018233in}}{\pgfqpoint{0.061506in}{0.035721in}}{\pgfqpoint{0.048614in}{0.048614in}}%
\pgfpathcurveto{\pgfqpoint{0.035721in}{0.061506in}}{\pgfqpoint{0.018233in}{0.068750in}}{\pgfqpoint{0.000000in}{0.068750in}}%
\pgfpathcurveto{\pgfqpoint{-0.018233in}{0.068750in}}{\pgfqpoint{-0.035721in}{0.061506in}}{\pgfqpoint{-0.048614in}{0.048614in}}%
\pgfpathcurveto{\pgfqpoint{-0.061506in}{0.035721in}}{\pgfqpoint{-0.068750in}{0.018233in}}{\pgfqpoint{-0.068750in}{0.000000in}}%
\pgfpathcurveto{\pgfqpoint{-0.068750in}{-0.018233in}}{\pgfqpoint{-0.061506in}{-0.035721in}}{\pgfqpoint{-0.048614in}{-0.048614in}}%
\pgfpathcurveto{\pgfqpoint{-0.035721in}{-0.061506in}}{\pgfqpoint{-0.018233in}{-0.068750in}}{\pgfqpoint{0.000000in}{-0.068750in}}%
\pgfpathclose%
\pgfusepath{stroke,fill}%
}%
\begin{pgfscope}%
\pgfsys@transformshift{0.686686in}{2.359687in}%
\pgfsys@useobject{currentmarker}{}%
\end{pgfscope}%
\end{pgfscope}%
\begin{pgfscope}%
\pgfpathrectangle{\pgfqpoint{0.100000in}{0.100000in}}{\pgfqpoint{5.307240in}{3.397500in}}%
\pgfusepath{clip}%
\pgfsetrectcap%
\pgfsetroundjoin%
\pgfsetlinewidth{1.505625pt}%
\definecolor{currentstroke}{rgb}{0.678431,1.000000,0.184314}%
\pgfsetstrokecolor{currentstroke}%
\pgfsetstrokeopacity{0.500000}%
\pgfsetdash{}{0pt}%
\pgfpathmoveto{\pgfqpoint{1.047381in}{1.427843in}}%
\pgfusepath{stroke}%
\end{pgfscope}%
\begin{pgfscope}%
\pgfpathrectangle{\pgfqpoint{0.100000in}{0.100000in}}{\pgfqpoint{5.307240in}{3.397500in}}%
\pgfusepath{clip}%
\pgfsetbuttcap%
\pgfsetroundjoin%
\definecolor{currentfill}{rgb}{0.678431,1.000000,0.184314}%
\pgfsetfillcolor{currentfill}%
\pgfsetfillopacity{0.500000}%
\pgfsetlinewidth{0.250937pt}%
\definecolor{currentstroke}{rgb}{0.000000,0.000000,0.000000}%
\pgfsetstrokecolor{currentstroke}%
\pgfsetstrokeopacity{0.500000}%
\pgfsetdash{}{0pt}%
\pgfsys@defobject{currentmarker}{\pgfqpoint{-0.085417in}{-0.085417in}}{\pgfqpoint{0.085417in}{0.085417in}}{%
\pgfpathmoveto{\pgfqpoint{0.000000in}{-0.085417in}}%
\pgfpathcurveto{\pgfqpoint{0.022653in}{-0.085417in}}{\pgfqpoint{0.044381in}{-0.076417in}}{\pgfqpoint{0.060399in}{-0.060399in}}%
\pgfpathcurveto{\pgfqpoint{0.076417in}{-0.044381in}}{\pgfqpoint{0.085417in}{-0.022653in}}{\pgfqpoint{0.085417in}{0.000000in}}%
\pgfpathcurveto{\pgfqpoint{0.085417in}{0.022653in}}{\pgfqpoint{0.076417in}{0.044381in}}{\pgfqpoint{0.060399in}{0.060399in}}%
\pgfpathcurveto{\pgfqpoint{0.044381in}{0.076417in}}{\pgfqpoint{0.022653in}{0.085417in}}{\pgfqpoint{0.000000in}{0.085417in}}%
\pgfpathcurveto{\pgfqpoint{-0.022653in}{0.085417in}}{\pgfqpoint{-0.044381in}{0.076417in}}{\pgfqpoint{-0.060399in}{0.060399in}}%
\pgfpathcurveto{\pgfqpoint{-0.076417in}{0.044381in}}{\pgfqpoint{-0.085417in}{0.022653in}}{\pgfqpoint{-0.085417in}{0.000000in}}%
\pgfpathcurveto{\pgfqpoint{-0.085417in}{-0.022653in}}{\pgfqpoint{-0.076417in}{-0.044381in}}{\pgfqpoint{-0.060399in}{-0.060399in}}%
\pgfpathcurveto{\pgfqpoint{-0.044381in}{-0.076417in}}{\pgfqpoint{-0.022653in}{-0.085417in}}{\pgfqpoint{0.000000in}{-0.085417in}}%
\pgfpathclose%
\pgfusepath{stroke,fill}%
}%
\begin{pgfscope}%
\pgfsys@transformshift{1.047381in}{1.427843in}%
\pgfsys@useobject{currentmarker}{}%
\end{pgfscope}%
\end{pgfscope}%
\begin{pgfscope}%
\pgfpathrectangle{\pgfqpoint{0.100000in}{0.100000in}}{\pgfqpoint{5.307240in}{3.397500in}}%
\pgfusepath{clip}%
\pgfsetrectcap%
\pgfsetroundjoin%
\pgfsetlinewidth{1.505625pt}%
\definecolor{currentstroke}{rgb}{0.678431,1.000000,0.184314}%
\pgfsetstrokecolor{currentstroke}%
\pgfsetstrokeopacity{0.500000}%
\pgfsetdash{}{0pt}%
\pgfpathmoveto{\pgfqpoint{0.775753in}{1.970536in}}%
\pgfusepath{stroke}%
\end{pgfscope}%
\begin{pgfscope}%
\pgfpathrectangle{\pgfqpoint{0.100000in}{0.100000in}}{\pgfqpoint{5.307240in}{3.397500in}}%
\pgfusepath{clip}%
\pgfsetbuttcap%
\pgfsetroundjoin%
\definecolor{currentfill}{rgb}{0.678431,1.000000,0.184314}%
\pgfsetfillcolor{currentfill}%
\pgfsetfillopacity{0.500000}%
\pgfsetlinewidth{0.250937pt}%
\definecolor{currentstroke}{rgb}{0.000000,0.000000,0.000000}%
\pgfsetstrokecolor{currentstroke}%
\pgfsetstrokeopacity{0.500000}%
\pgfsetdash{}{0pt}%
\pgfsys@defobject{currentmarker}{\pgfqpoint{-0.063194in}{-0.063194in}}{\pgfqpoint{0.063194in}{0.063194in}}{%
\pgfpathmoveto{\pgfqpoint{0.000000in}{-0.063194in}}%
\pgfpathcurveto{\pgfqpoint{0.016759in}{-0.063194in}}{\pgfqpoint{0.032835in}{-0.056536in}}{\pgfqpoint{0.044685in}{-0.044685in}}%
\pgfpathcurveto{\pgfqpoint{0.056536in}{-0.032835in}}{\pgfqpoint{0.063194in}{-0.016759in}}{\pgfqpoint{0.063194in}{0.000000in}}%
\pgfpathcurveto{\pgfqpoint{0.063194in}{0.016759in}}{\pgfqpoint{0.056536in}{0.032835in}}{\pgfqpoint{0.044685in}{0.044685in}}%
\pgfpathcurveto{\pgfqpoint{0.032835in}{0.056536in}}{\pgfqpoint{0.016759in}{0.063194in}}{\pgfqpoint{0.000000in}{0.063194in}}%
\pgfpathcurveto{\pgfqpoint{-0.016759in}{0.063194in}}{\pgfqpoint{-0.032835in}{0.056536in}}{\pgfqpoint{-0.044685in}{0.044685in}}%
\pgfpathcurveto{\pgfqpoint{-0.056536in}{0.032835in}}{\pgfqpoint{-0.063194in}{0.016759in}}{\pgfqpoint{-0.063194in}{0.000000in}}%
\pgfpathcurveto{\pgfqpoint{-0.063194in}{-0.016759in}}{\pgfqpoint{-0.056536in}{-0.032835in}}{\pgfqpoint{-0.044685in}{-0.044685in}}%
\pgfpathcurveto{\pgfqpoint{-0.032835in}{-0.056536in}}{\pgfqpoint{-0.016759in}{-0.063194in}}{\pgfqpoint{0.000000in}{-0.063194in}}%
\pgfpathclose%
\pgfusepath{stroke,fill}%
}%
\begin{pgfscope}%
\pgfsys@transformshift{0.775753in}{1.970536in}%
\pgfsys@useobject{currentmarker}{}%
\end{pgfscope}%
\end{pgfscope}%
\begin{pgfscope}%
\pgfpathrectangle{\pgfqpoint{0.100000in}{0.100000in}}{\pgfqpoint{5.307240in}{3.397500in}}%
\pgfusepath{clip}%
\pgfsetrectcap%
\pgfsetroundjoin%
\pgfsetlinewidth{1.505625pt}%
\definecolor{currentstroke}{rgb}{0.678431,1.000000,0.184314}%
\pgfsetstrokecolor{currentstroke}%
\pgfsetstrokeopacity{0.500000}%
\pgfsetdash{}{0pt}%
\pgfpathmoveto{\pgfqpoint{0.768895in}{1.923878in}}%
\pgfusepath{stroke}%
\end{pgfscope}%
\begin{pgfscope}%
\pgfpathrectangle{\pgfqpoint{0.100000in}{0.100000in}}{\pgfqpoint{5.307240in}{3.397500in}}%
\pgfusepath{clip}%
\pgfsetbuttcap%
\pgfsetroundjoin%
\definecolor{currentfill}{rgb}{0.678431,1.000000,0.184314}%
\pgfsetfillcolor{currentfill}%
\pgfsetfillopacity{0.500000}%
\pgfsetlinewidth{0.250937pt}%
\definecolor{currentstroke}{rgb}{0.000000,0.000000,0.000000}%
\pgfsetstrokecolor{currentstroke}%
\pgfsetstrokeopacity{0.500000}%
\pgfsetdash{}{0pt}%
\pgfsys@defobject{currentmarker}{\pgfqpoint{-0.059028in}{-0.059028in}}{\pgfqpoint{0.059028in}{0.059028in}}{%
\pgfpathmoveto{\pgfqpoint{0.000000in}{-0.059028in}}%
\pgfpathcurveto{\pgfqpoint{0.015654in}{-0.059028in}}{\pgfqpoint{0.030670in}{-0.052808in}}{\pgfqpoint{0.041739in}{-0.041739in}}%
\pgfpathcurveto{\pgfqpoint{0.052808in}{-0.030670in}}{\pgfqpoint{0.059028in}{-0.015654in}}{\pgfqpoint{0.059028in}{0.000000in}}%
\pgfpathcurveto{\pgfqpoint{0.059028in}{0.015654in}}{\pgfqpoint{0.052808in}{0.030670in}}{\pgfqpoint{0.041739in}{0.041739in}}%
\pgfpathcurveto{\pgfqpoint{0.030670in}{0.052808in}}{\pgfqpoint{0.015654in}{0.059028in}}{\pgfqpoint{0.000000in}{0.059028in}}%
\pgfpathcurveto{\pgfqpoint{-0.015654in}{0.059028in}}{\pgfqpoint{-0.030670in}{0.052808in}}{\pgfqpoint{-0.041739in}{0.041739in}}%
\pgfpathcurveto{\pgfqpoint{-0.052808in}{0.030670in}}{\pgfqpoint{-0.059028in}{0.015654in}}{\pgfqpoint{-0.059028in}{0.000000in}}%
\pgfpathcurveto{\pgfqpoint{-0.059028in}{-0.015654in}}{\pgfqpoint{-0.052808in}{-0.030670in}}{\pgfqpoint{-0.041739in}{-0.041739in}}%
\pgfpathcurveto{\pgfqpoint{-0.030670in}{-0.052808in}}{\pgfqpoint{-0.015654in}{-0.059028in}}{\pgfqpoint{0.000000in}{-0.059028in}}%
\pgfpathclose%
\pgfusepath{stroke,fill}%
}%
\begin{pgfscope}%
\pgfsys@transformshift{0.768895in}{1.923878in}%
\pgfsys@useobject{currentmarker}{}%
\end{pgfscope}%
\end{pgfscope}%
\begin{pgfscope}%
\pgfpathrectangle{\pgfqpoint{0.100000in}{0.100000in}}{\pgfqpoint{5.307240in}{3.397500in}}%
\pgfusepath{clip}%
\pgfsetrectcap%
\pgfsetroundjoin%
\pgfsetlinewidth{1.505625pt}%
\definecolor{currentstroke}{rgb}{0.678431,1.000000,0.184314}%
\pgfsetstrokecolor{currentstroke}%
\pgfsetstrokeopacity{0.500000}%
\pgfsetdash{}{0pt}%
\pgfpathmoveto{\pgfqpoint{0.828747in}{1.633479in}}%
\pgfusepath{stroke}%
\end{pgfscope}%
\begin{pgfscope}%
\pgfpathrectangle{\pgfqpoint{0.100000in}{0.100000in}}{\pgfqpoint{5.307240in}{3.397500in}}%
\pgfusepath{clip}%
\pgfsetbuttcap%
\pgfsetroundjoin%
\definecolor{currentfill}{rgb}{0.678431,1.000000,0.184314}%
\pgfsetfillcolor{currentfill}%
\pgfsetfillopacity{0.500000}%
\pgfsetlinewidth{0.250937pt}%
\definecolor{currentstroke}{rgb}{0.000000,0.000000,0.000000}%
\pgfsetstrokecolor{currentstroke}%
\pgfsetstrokeopacity{0.500000}%
\pgfsetdash{}{0pt}%
\pgfsys@defobject{currentmarker}{\pgfqpoint{-0.104861in}{-0.104861in}}{\pgfqpoint{0.104861in}{0.104861in}}{%
\pgfpathmoveto{\pgfqpoint{0.000000in}{-0.104861in}}%
\pgfpathcurveto{\pgfqpoint{0.027809in}{-0.104861in}}{\pgfqpoint{0.054484in}{-0.093812in}}{\pgfqpoint{0.074148in}{-0.074148in}}%
\pgfpathcurveto{\pgfqpoint{0.093812in}{-0.054484in}}{\pgfqpoint{0.104861in}{-0.027809in}}{\pgfqpoint{0.104861in}{0.000000in}}%
\pgfpathcurveto{\pgfqpoint{0.104861in}{0.027809in}}{\pgfqpoint{0.093812in}{0.054484in}}{\pgfqpoint{0.074148in}{0.074148in}}%
\pgfpathcurveto{\pgfqpoint{0.054484in}{0.093812in}}{\pgfqpoint{0.027809in}{0.104861in}}{\pgfqpoint{0.000000in}{0.104861in}}%
\pgfpathcurveto{\pgfqpoint{-0.027809in}{0.104861in}}{\pgfqpoint{-0.054484in}{0.093812in}}{\pgfqpoint{-0.074148in}{0.074148in}}%
\pgfpathcurveto{\pgfqpoint{-0.093812in}{0.054484in}}{\pgfqpoint{-0.104861in}{0.027809in}}{\pgfqpoint{-0.104861in}{0.000000in}}%
\pgfpathcurveto{\pgfqpoint{-0.104861in}{-0.027809in}}{\pgfqpoint{-0.093812in}{-0.054484in}}{\pgfqpoint{-0.074148in}{-0.074148in}}%
\pgfpathcurveto{\pgfqpoint{-0.054484in}{-0.093812in}}{\pgfqpoint{-0.027809in}{-0.104861in}}{\pgfqpoint{0.000000in}{-0.104861in}}%
\pgfpathclose%
\pgfusepath{stroke,fill}%
}%
\begin{pgfscope}%
\pgfsys@transformshift{0.828747in}{1.633479in}%
\pgfsys@useobject{currentmarker}{}%
\end{pgfscope}%
\end{pgfscope}%
\begin{pgfscope}%
\pgfpathrectangle{\pgfqpoint{0.100000in}{0.100000in}}{\pgfqpoint{5.307240in}{3.397500in}}%
\pgfusepath{clip}%
\pgfsetrectcap%
\pgfsetroundjoin%
\pgfsetlinewidth{1.505625pt}%
\definecolor{currentstroke}{rgb}{0.678431,1.000000,0.184314}%
\pgfsetstrokecolor{currentstroke}%
\pgfsetstrokeopacity{0.500000}%
\pgfsetdash{}{0pt}%
\pgfpathmoveto{\pgfqpoint{0.783768in}{2.021694in}}%
\pgfusepath{stroke}%
\end{pgfscope}%
\begin{pgfscope}%
\pgfpathrectangle{\pgfqpoint{0.100000in}{0.100000in}}{\pgfqpoint{5.307240in}{3.397500in}}%
\pgfusepath{clip}%
\pgfsetbuttcap%
\pgfsetroundjoin%
\definecolor{currentfill}{rgb}{0.678431,1.000000,0.184314}%
\pgfsetfillcolor{currentfill}%
\pgfsetfillopacity{0.500000}%
\pgfsetlinewidth{0.250937pt}%
\definecolor{currentstroke}{rgb}{0.000000,0.000000,0.000000}%
\pgfsetstrokecolor{currentstroke}%
\pgfsetstrokeopacity{0.500000}%
\pgfsetdash{}{0pt}%
\pgfsys@defobject{currentmarker}{\pgfqpoint{-0.063889in}{-0.063889in}}{\pgfqpoint{0.063889in}{0.063889in}}{%
\pgfpathmoveto{\pgfqpoint{0.000000in}{-0.063889in}}%
\pgfpathcurveto{\pgfqpoint{0.016944in}{-0.063889in}}{\pgfqpoint{0.033195in}{-0.057157in}}{\pgfqpoint{0.045176in}{-0.045176in}}%
\pgfpathcurveto{\pgfqpoint{0.057157in}{-0.033195in}}{\pgfqpoint{0.063889in}{-0.016944in}}{\pgfqpoint{0.063889in}{0.000000in}}%
\pgfpathcurveto{\pgfqpoint{0.063889in}{0.016944in}}{\pgfqpoint{0.057157in}{0.033195in}}{\pgfqpoint{0.045176in}{0.045176in}}%
\pgfpathcurveto{\pgfqpoint{0.033195in}{0.057157in}}{\pgfqpoint{0.016944in}{0.063889in}}{\pgfqpoint{0.000000in}{0.063889in}}%
\pgfpathcurveto{\pgfqpoint{-0.016944in}{0.063889in}}{\pgfqpoint{-0.033195in}{0.057157in}}{\pgfqpoint{-0.045176in}{0.045176in}}%
\pgfpathcurveto{\pgfqpoint{-0.057157in}{0.033195in}}{\pgfqpoint{-0.063889in}{0.016944in}}{\pgfqpoint{-0.063889in}{0.000000in}}%
\pgfpathcurveto{\pgfqpoint{-0.063889in}{-0.016944in}}{\pgfqpoint{-0.057157in}{-0.033195in}}{\pgfqpoint{-0.045176in}{-0.045176in}}%
\pgfpathcurveto{\pgfqpoint{-0.033195in}{-0.057157in}}{\pgfqpoint{-0.016944in}{-0.063889in}}{\pgfqpoint{0.000000in}{-0.063889in}}%
\pgfpathclose%
\pgfusepath{stroke,fill}%
}%
\begin{pgfscope}%
\pgfsys@transformshift{0.783768in}{2.021694in}%
\pgfsys@useobject{currentmarker}{}%
\end{pgfscope}%
\end{pgfscope}%
\begin{pgfscope}%
\pgfpathrectangle{\pgfqpoint{0.100000in}{0.100000in}}{\pgfqpoint{5.307240in}{3.397500in}}%
\pgfusepath{clip}%
\pgfsetrectcap%
\pgfsetroundjoin%
\pgfsetlinewidth{1.505625pt}%
\definecolor{currentstroke}{rgb}{0.678431,1.000000,0.184314}%
\pgfsetstrokecolor{currentstroke}%
\pgfsetstrokeopacity{0.500000}%
\pgfsetdash{}{0pt}%
\pgfpathmoveto{\pgfqpoint{0.724644in}{2.054351in}}%
\pgfusepath{stroke}%
\end{pgfscope}%
\begin{pgfscope}%
\pgfpathrectangle{\pgfqpoint{0.100000in}{0.100000in}}{\pgfqpoint{5.307240in}{3.397500in}}%
\pgfusepath{clip}%
\pgfsetbuttcap%
\pgfsetroundjoin%
\definecolor{currentfill}{rgb}{0.678431,1.000000,0.184314}%
\pgfsetfillcolor{currentfill}%
\pgfsetfillopacity{0.500000}%
\pgfsetlinewidth{0.250937pt}%
\definecolor{currentstroke}{rgb}{0.000000,0.000000,0.000000}%
\pgfsetstrokecolor{currentstroke}%
\pgfsetstrokeopacity{0.500000}%
\pgfsetdash{}{0pt}%
\pgfsys@defobject{currentmarker}{\pgfqpoint{-0.070139in}{-0.070139in}}{\pgfqpoint{0.070139in}{0.070139in}}{%
\pgfpathmoveto{\pgfqpoint{0.000000in}{-0.070139in}}%
\pgfpathcurveto{\pgfqpoint{0.018601in}{-0.070139in}}{\pgfqpoint{0.036443in}{-0.062749in}}{\pgfqpoint{0.049596in}{-0.049596in}}%
\pgfpathcurveto{\pgfqpoint{0.062749in}{-0.036443in}}{\pgfqpoint{0.070139in}{-0.018601in}}{\pgfqpoint{0.070139in}{0.000000in}}%
\pgfpathcurveto{\pgfqpoint{0.070139in}{0.018601in}}{\pgfqpoint{0.062749in}{0.036443in}}{\pgfqpoint{0.049596in}{0.049596in}}%
\pgfpathcurveto{\pgfqpoint{0.036443in}{0.062749in}}{\pgfqpoint{0.018601in}{0.070139in}}{\pgfqpoint{0.000000in}{0.070139in}}%
\pgfpathcurveto{\pgfqpoint{-0.018601in}{0.070139in}}{\pgfqpoint{-0.036443in}{0.062749in}}{\pgfqpoint{-0.049596in}{0.049596in}}%
\pgfpathcurveto{\pgfqpoint{-0.062749in}{0.036443in}}{\pgfqpoint{-0.070139in}{0.018601in}}{\pgfqpoint{-0.070139in}{0.000000in}}%
\pgfpathcurveto{\pgfqpoint{-0.070139in}{-0.018601in}}{\pgfqpoint{-0.062749in}{-0.036443in}}{\pgfqpoint{-0.049596in}{-0.049596in}}%
\pgfpathcurveto{\pgfqpoint{-0.036443in}{-0.062749in}}{\pgfqpoint{-0.018601in}{-0.070139in}}{\pgfqpoint{0.000000in}{-0.070139in}}%
\pgfpathclose%
\pgfusepath{stroke,fill}%
}%
\begin{pgfscope}%
\pgfsys@transformshift{0.724644in}{2.054351in}%
\pgfsys@useobject{currentmarker}{}%
\end{pgfscope}%
\end{pgfscope}%
\begin{pgfscope}%
\pgfpathrectangle{\pgfqpoint{0.100000in}{0.100000in}}{\pgfqpoint{5.307240in}{3.397500in}}%
\pgfusepath{clip}%
\pgfsetrectcap%
\pgfsetroundjoin%
\pgfsetlinewidth{1.505625pt}%
\definecolor{currentstroke}{rgb}{0.678431,1.000000,0.184314}%
\pgfsetstrokecolor{currentstroke}%
\pgfsetstrokeopacity{0.500000}%
\pgfsetdash{}{0pt}%
\pgfpathmoveto{\pgfqpoint{0.690146in}{2.105106in}}%
\pgfusepath{stroke}%
\end{pgfscope}%
\begin{pgfscope}%
\pgfpathrectangle{\pgfqpoint{0.100000in}{0.100000in}}{\pgfqpoint{5.307240in}{3.397500in}}%
\pgfusepath{clip}%
\pgfsetbuttcap%
\pgfsetroundjoin%
\definecolor{currentfill}{rgb}{0.678431,1.000000,0.184314}%
\pgfsetfillcolor{currentfill}%
\pgfsetfillopacity{0.500000}%
\pgfsetlinewidth{0.250937pt}%
\definecolor{currentstroke}{rgb}{0.000000,0.000000,0.000000}%
\pgfsetstrokecolor{currentstroke}%
\pgfsetstrokeopacity{0.500000}%
\pgfsetdash{}{0pt}%
\pgfsys@defobject{currentmarker}{\pgfqpoint{-0.075000in}{-0.075000in}}{\pgfqpoint{0.075000in}{0.075000in}}{%
\pgfpathmoveto{\pgfqpoint{0.000000in}{-0.075000in}}%
\pgfpathcurveto{\pgfqpoint{0.019890in}{-0.075000in}}{\pgfqpoint{0.038968in}{-0.067098in}}{\pgfqpoint{0.053033in}{-0.053033in}}%
\pgfpathcurveto{\pgfqpoint{0.067098in}{-0.038968in}}{\pgfqpoint{0.075000in}{-0.019890in}}{\pgfqpoint{0.075000in}{0.000000in}}%
\pgfpathcurveto{\pgfqpoint{0.075000in}{0.019890in}}{\pgfqpoint{0.067098in}{0.038968in}}{\pgfqpoint{0.053033in}{0.053033in}}%
\pgfpathcurveto{\pgfqpoint{0.038968in}{0.067098in}}{\pgfqpoint{0.019890in}{0.075000in}}{\pgfqpoint{0.000000in}{0.075000in}}%
\pgfpathcurveto{\pgfqpoint{-0.019890in}{0.075000in}}{\pgfqpoint{-0.038968in}{0.067098in}}{\pgfqpoint{-0.053033in}{0.053033in}}%
\pgfpathcurveto{\pgfqpoint{-0.067098in}{0.038968in}}{\pgfqpoint{-0.075000in}{0.019890in}}{\pgfqpoint{-0.075000in}{0.000000in}}%
\pgfpathcurveto{\pgfqpoint{-0.075000in}{-0.019890in}}{\pgfqpoint{-0.067098in}{-0.038968in}}{\pgfqpoint{-0.053033in}{-0.053033in}}%
\pgfpathcurveto{\pgfqpoint{-0.038968in}{-0.067098in}}{\pgfqpoint{-0.019890in}{-0.075000in}}{\pgfqpoint{0.000000in}{-0.075000in}}%
\pgfpathclose%
\pgfusepath{stroke,fill}%
}%
\begin{pgfscope}%
\pgfsys@transformshift{0.690146in}{2.105106in}%
\pgfsys@useobject{currentmarker}{}%
\end{pgfscope}%
\end{pgfscope}%
\begin{pgfscope}%
\pgfpathrectangle{\pgfqpoint{0.100000in}{0.100000in}}{\pgfqpoint{5.307240in}{3.397500in}}%
\pgfusepath{clip}%
\pgfsetrectcap%
\pgfsetroundjoin%
\pgfsetlinewidth{1.505625pt}%
\definecolor{currentstroke}{rgb}{0.678431,1.000000,0.184314}%
\pgfsetstrokecolor{currentstroke}%
\pgfsetstrokeopacity{0.500000}%
\pgfsetdash{}{0pt}%
\pgfpathmoveto{\pgfqpoint{0.599174in}{2.212486in}}%
\pgfusepath{stroke}%
\end{pgfscope}%
\begin{pgfscope}%
\pgfpathrectangle{\pgfqpoint{0.100000in}{0.100000in}}{\pgfqpoint{5.307240in}{3.397500in}}%
\pgfusepath{clip}%
\pgfsetbuttcap%
\pgfsetroundjoin%
\definecolor{currentfill}{rgb}{0.678431,1.000000,0.184314}%
\pgfsetfillcolor{currentfill}%
\pgfsetfillopacity{0.500000}%
\pgfsetlinewidth{0.250937pt}%
\definecolor{currentstroke}{rgb}{0.000000,0.000000,0.000000}%
\pgfsetstrokecolor{currentstroke}%
\pgfsetstrokeopacity{0.500000}%
\pgfsetdash{}{0pt}%
\pgfsys@defobject{currentmarker}{\pgfqpoint{-0.091667in}{-0.091667in}}{\pgfqpoint{0.091667in}{0.091667in}}{%
\pgfpathmoveto{\pgfqpoint{0.000000in}{-0.091667in}}%
\pgfpathcurveto{\pgfqpoint{0.024310in}{-0.091667in}}{\pgfqpoint{0.047628in}{-0.082008in}}{\pgfqpoint{0.064818in}{-0.064818in}}%
\pgfpathcurveto{\pgfqpoint{0.082008in}{-0.047628in}}{\pgfqpoint{0.091667in}{-0.024310in}}{\pgfqpoint{0.091667in}{0.000000in}}%
\pgfpathcurveto{\pgfqpoint{0.091667in}{0.024310in}}{\pgfqpoint{0.082008in}{0.047628in}}{\pgfqpoint{0.064818in}{0.064818in}}%
\pgfpathcurveto{\pgfqpoint{0.047628in}{0.082008in}}{\pgfqpoint{0.024310in}{0.091667in}}{\pgfqpoint{0.000000in}{0.091667in}}%
\pgfpathcurveto{\pgfqpoint{-0.024310in}{0.091667in}}{\pgfqpoint{-0.047628in}{0.082008in}}{\pgfqpoint{-0.064818in}{0.064818in}}%
\pgfpathcurveto{\pgfqpoint{-0.082008in}{0.047628in}}{\pgfqpoint{-0.091667in}{0.024310in}}{\pgfqpoint{-0.091667in}{0.000000in}}%
\pgfpathcurveto{\pgfqpoint{-0.091667in}{-0.024310in}}{\pgfqpoint{-0.082008in}{-0.047628in}}{\pgfqpoint{-0.064818in}{-0.064818in}}%
\pgfpathcurveto{\pgfqpoint{-0.047628in}{-0.082008in}}{\pgfqpoint{-0.024310in}{-0.091667in}}{\pgfqpoint{0.000000in}{-0.091667in}}%
\pgfpathclose%
\pgfusepath{stroke,fill}%
}%
\begin{pgfscope}%
\pgfsys@transformshift{0.599174in}{2.212486in}%
\pgfsys@useobject{currentmarker}{}%
\end{pgfscope}%
\end{pgfscope}%
\begin{pgfscope}%
\pgfpathrectangle{\pgfqpoint{0.100000in}{0.100000in}}{\pgfqpoint{5.307240in}{3.397500in}}%
\pgfusepath{clip}%
\pgfsetrectcap%
\pgfsetroundjoin%
\pgfsetlinewidth{1.505625pt}%
\definecolor{currentstroke}{rgb}{0.678431,1.000000,0.184314}%
\pgfsetstrokecolor{currentstroke}%
\pgfsetstrokeopacity{0.500000}%
\pgfsetdash{}{0pt}%
\pgfpathmoveto{\pgfqpoint{0.745708in}{1.672884in}}%
\pgfusepath{stroke}%
\end{pgfscope}%
\begin{pgfscope}%
\pgfpathrectangle{\pgfqpoint{0.100000in}{0.100000in}}{\pgfqpoint{5.307240in}{3.397500in}}%
\pgfusepath{clip}%
\pgfsetbuttcap%
\pgfsetroundjoin%
\definecolor{currentfill}{rgb}{0.678431,1.000000,0.184314}%
\pgfsetfillcolor{currentfill}%
\pgfsetfillopacity{0.500000}%
\pgfsetlinewidth{0.250937pt}%
\definecolor{currentstroke}{rgb}{0.000000,0.000000,0.000000}%
\pgfsetstrokecolor{currentstroke}%
\pgfsetstrokeopacity{0.500000}%
\pgfsetdash{}{0pt}%
\pgfsys@defobject{currentmarker}{\pgfqpoint{-0.075694in}{-0.075694in}}{\pgfqpoint{0.075694in}{0.075694in}}{%
\pgfpathmoveto{\pgfqpoint{0.000000in}{-0.075694in}}%
\pgfpathcurveto{\pgfqpoint{0.020074in}{-0.075694in}}{\pgfqpoint{0.039329in}{-0.067719in}}{\pgfqpoint{0.053524in}{-0.053524in}}%
\pgfpathcurveto{\pgfqpoint{0.067719in}{-0.039329in}}{\pgfqpoint{0.075694in}{-0.020074in}}{\pgfqpoint{0.075694in}{0.000000in}}%
\pgfpathcurveto{\pgfqpoint{0.075694in}{0.020074in}}{\pgfqpoint{0.067719in}{0.039329in}}{\pgfqpoint{0.053524in}{0.053524in}}%
\pgfpathcurveto{\pgfqpoint{0.039329in}{0.067719in}}{\pgfqpoint{0.020074in}{0.075694in}}{\pgfqpoint{0.000000in}{0.075694in}}%
\pgfpathcurveto{\pgfqpoint{-0.020074in}{0.075694in}}{\pgfqpoint{-0.039329in}{0.067719in}}{\pgfqpoint{-0.053524in}{0.053524in}}%
\pgfpathcurveto{\pgfqpoint{-0.067719in}{0.039329in}}{\pgfqpoint{-0.075694in}{0.020074in}}{\pgfqpoint{-0.075694in}{0.000000in}}%
\pgfpathcurveto{\pgfqpoint{-0.075694in}{-0.020074in}}{\pgfqpoint{-0.067719in}{-0.039329in}}{\pgfqpoint{-0.053524in}{-0.053524in}}%
\pgfpathcurveto{\pgfqpoint{-0.039329in}{-0.067719in}}{\pgfqpoint{-0.020074in}{-0.075694in}}{\pgfqpoint{0.000000in}{-0.075694in}}%
\pgfpathclose%
\pgfusepath{stroke,fill}%
}%
\begin{pgfscope}%
\pgfsys@transformshift{0.745708in}{1.672884in}%
\pgfsys@useobject{currentmarker}{}%
\end{pgfscope}%
\end{pgfscope}%
\begin{pgfscope}%
\pgfpathrectangle{\pgfqpoint{0.100000in}{0.100000in}}{\pgfqpoint{5.307240in}{3.397500in}}%
\pgfusepath{clip}%
\pgfsetrectcap%
\pgfsetroundjoin%
\pgfsetlinewidth{1.505625pt}%
\definecolor{currentstroke}{rgb}{0.678431,1.000000,0.184314}%
\pgfsetstrokecolor{currentstroke}%
\pgfsetstrokeopacity{0.500000}%
\pgfsetdash{}{0pt}%
\pgfpathmoveto{\pgfqpoint{0.668935in}{2.469501in}}%
\pgfusepath{stroke}%
\end{pgfscope}%
\begin{pgfscope}%
\pgfpathrectangle{\pgfqpoint{0.100000in}{0.100000in}}{\pgfqpoint{5.307240in}{3.397500in}}%
\pgfusepath{clip}%
\pgfsetbuttcap%
\pgfsetroundjoin%
\definecolor{currentfill}{rgb}{0.678431,1.000000,0.184314}%
\pgfsetfillcolor{currentfill}%
\pgfsetfillopacity{0.500000}%
\pgfsetlinewidth{0.250937pt}%
\definecolor{currentstroke}{rgb}{0.000000,0.000000,0.000000}%
\pgfsetstrokecolor{currentstroke}%
\pgfsetstrokeopacity{0.500000}%
\pgfsetdash{}{0pt}%
\pgfsys@defobject{currentmarker}{\pgfqpoint{-0.077778in}{-0.077778in}}{\pgfqpoint{0.077778in}{0.077778in}}{%
\pgfpathmoveto{\pgfqpoint{0.000000in}{-0.077778in}}%
\pgfpathcurveto{\pgfqpoint{0.020627in}{-0.077778in}}{\pgfqpoint{0.040412in}{-0.069583in}}{\pgfqpoint{0.054997in}{-0.054997in}}%
\pgfpathcurveto{\pgfqpoint{0.069583in}{-0.040412in}}{\pgfqpoint{0.077778in}{-0.020627in}}{\pgfqpoint{0.077778in}{0.000000in}}%
\pgfpathcurveto{\pgfqpoint{0.077778in}{0.020627in}}{\pgfqpoint{0.069583in}{0.040412in}}{\pgfqpoint{0.054997in}{0.054997in}}%
\pgfpathcurveto{\pgfqpoint{0.040412in}{0.069583in}}{\pgfqpoint{0.020627in}{0.077778in}}{\pgfqpoint{0.000000in}{0.077778in}}%
\pgfpathcurveto{\pgfqpoint{-0.020627in}{0.077778in}}{\pgfqpoint{-0.040412in}{0.069583in}}{\pgfqpoint{-0.054997in}{0.054997in}}%
\pgfpathcurveto{\pgfqpoint{-0.069583in}{0.040412in}}{\pgfqpoint{-0.077778in}{0.020627in}}{\pgfqpoint{-0.077778in}{0.000000in}}%
\pgfpathcurveto{\pgfqpoint{-0.077778in}{-0.020627in}}{\pgfqpoint{-0.069583in}{-0.040412in}}{\pgfqpoint{-0.054997in}{-0.054997in}}%
\pgfpathcurveto{\pgfqpoint{-0.040412in}{-0.069583in}}{\pgfqpoint{-0.020627in}{-0.077778in}}{\pgfqpoint{0.000000in}{-0.077778in}}%
\pgfpathclose%
\pgfusepath{stroke,fill}%
}%
\begin{pgfscope}%
\pgfsys@transformshift{0.668935in}{2.469501in}%
\pgfsys@useobject{currentmarker}{}%
\end{pgfscope}%
\end{pgfscope}%
\begin{pgfscope}%
\pgfpathrectangle{\pgfqpoint{0.100000in}{0.100000in}}{\pgfqpoint{5.307240in}{3.397500in}}%
\pgfusepath{clip}%
\pgfsetrectcap%
\pgfsetroundjoin%
\pgfsetlinewidth{1.505625pt}%
\definecolor{currentstroke}{rgb}{0.678431,1.000000,0.184314}%
\pgfsetstrokecolor{currentstroke}%
\pgfsetstrokeopacity{0.500000}%
\pgfsetdash{}{0pt}%
\pgfpathmoveto{\pgfqpoint{0.905295in}{1.601604in}}%
\pgfusepath{stroke}%
\end{pgfscope}%
\begin{pgfscope}%
\pgfpathrectangle{\pgfqpoint{0.100000in}{0.100000in}}{\pgfqpoint{5.307240in}{3.397500in}}%
\pgfusepath{clip}%
\pgfsetbuttcap%
\pgfsetroundjoin%
\definecolor{currentfill}{rgb}{0.678431,1.000000,0.184314}%
\pgfsetfillcolor{currentfill}%
\pgfsetfillopacity{0.500000}%
\pgfsetlinewidth{0.250937pt}%
\definecolor{currentstroke}{rgb}{0.000000,0.000000,0.000000}%
\pgfsetstrokecolor{currentstroke}%
\pgfsetstrokeopacity{0.500000}%
\pgfsetdash{}{0pt}%
\pgfsys@defobject{currentmarker}{\pgfqpoint{-0.075000in}{-0.075000in}}{\pgfqpoint{0.075000in}{0.075000in}}{%
\pgfpathmoveto{\pgfqpoint{0.000000in}{-0.075000in}}%
\pgfpathcurveto{\pgfqpoint{0.019890in}{-0.075000in}}{\pgfqpoint{0.038968in}{-0.067098in}}{\pgfqpoint{0.053033in}{-0.053033in}}%
\pgfpathcurveto{\pgfqpoint{0.067098in}{-0.038968in}}{\pgfqpoint{0.075000in}{-0.019890in}}{\pgfqpoint{0.075000in}{0.000000in}}%
\pgfpathcurveto{\pgfqpoint{0.075000in}{0.019890in}}{\pgfqpoint{0.067098in}{0.038968in}}{\pgfqpoint{0.053033in}{0.053033in}}%
\pgfpathcurveto{\pgfqpoint{0.038968in}{0.067098in}}{\pgfqpoint{0.019890in}{0.075000in}}{\pgfqpoint{0.000000in}{0.075000in}}%
\pgfpathcurveto{\pgfqpoint{-0.019890in}{0.075000in}}{\pgfqpoint{-0.038968in}{0.067098in}}{\pgfqpoint{-0.053033in}{0.053033in}}%
\pgfpathcurveto{\pgfqpoint{-0.067098in}{0.038968in}}{\pgfqpoint{-0.075000in}{0.019890in}}{\pgfqpoint{-0.075000in}{0.000000in}}%
\pgfpathcurveto{\pgfqpoint{-0.075000in}{-0.019890in}}{\pgfqpoint{-0.067098in}{-0.038968in}}{\pgfqpoint{-0.053033in}{-0.053033in}}%
\pgfpathcurveto{\pgfqpoint{-0.038968in}{-0.067098in}}{\pgfqpoint{-0.019890in}{-0.075000in}}{\pgfqpoint{0.000000in}{-0.075000in}}%
\pgfpathclose%
\pgfusepath{stroke,fill}%
}%
\begin{pgfscope}%
\pgfsys@transformshift{0.905295in}{1.601604in}%
\pgfsys@useobject{currentmarker}{}%
\end{pgfscope}%
\end{pgfscope}%
\begin{pgfscope}%
\pgfpathrectangle{\pgfqpoint{0.100000in}{0.100000in}}{\pgfqpoint{5.307240in}{3.397500in}}%
\pgfusepath{clip}%
\pgfsetrectcap%
\pgfsetroundjoin%
\pgfsetlinewidth{1.505625pt}%
\definecolor{currentstroke}{rgb}{0.678431,1.000000,0.184314}%
\pgfsetstrokecolor{currentstroke}%
\pgfsetstrokeopacity{0.500000}%
\pgfsetdash{}{0pt}%
\pgfpathmoveto{\pgfqpoint{0.677726in}{2.223109in}}%
\pgfusepath{stroke}%
\end{pgfscope}%
\begin{pgfscope}%
\pgfpathrectangle{\pgfqpoint{0.100000in}{0.100000in}}{\pgfqpoint{5.307240in}{3.397500in}}%
\pgfusepath{clip}%
\pgfsetbuttcap%
\pgfsetroundjoin%
\definecolor{currentfill}{rgb}{0.678431,1.000000,0.184314}%
\pgfsetfillcolor{currentfill}%
\pgfsetfillopacity{0.500000}%
\pgfsetlinewidth{0.250937pt}%
\definecolor{currentstroke}{rgb}{0.000000,0.000000,0.000000}%
\pgfsetstrokecolor{currentstroke}%
\pgfsetstrokeopacity{0.500000}%
\pgfsetdash{}{0pt}%
\pgfsys@defobject{currentmarker}{\pgfqpoint{-0.075000in}{-0.075000in}}{\pgfqpoint{0.075000in}{0.075000in}}{%
\pgfpathmoveto{\pgfqpoint{0.000000in}{-0.075000in}}%
\pgfpathcurveto{\pgfqpoint{0.019890in}{-0.075000in}}{\pgfqpoint{0.038968in}{-0.067098in}}{\pgfqpoint{0.053033in}{-0.053033in}}%
\pgfpathcurveto{\pgfqpoint{0.067098in}{-0.038968in}}{\pgfqpoint{0.075000in}{-0.019890in}}{\pgfqpoint{0.075000in}{0.000000in}}%
\pgfpathcurveto{\pgfqpoint{0.075000in}{0.019890in}}{\pgfqpoint{0.067098in}{0.038968in}}{\pgfqpoint{0.053033in}{0.053033in}}%
\pgfpathcurveto{\pgfqpoint{0.038968in}{0.067098in}}{\pgfqpoint{0.019890in}{0.075000in}}{\pgfqpoint{0.000000in}{0.075000in}}%
\pgfpathcurveto{\pgfqpoint{-0.019890in}{0.075000in}}{\pgfqpoint{-0.038968in}{0.067098in}}{\pgfqpoint{-0.053033in}{0.053033in}}%
\pgfpathcurveto{\pgfqpoint{-0.067098in}{0.038968in}}{\pgfqpoint{-0.075000in}{0.019890in}}{\pgfqpoint{-0.075000in}{0.000000in}}%
\pgfpathcurveto{\pgfqpoint{-0.075000in}{-0.019890in}}{\pgfqpoint{-0.067098in}{-0.038968in}}{\pgfqpoint{-0.053033in}{-0.053033in}}%
\pgfpathcurveto{\pgfqpoint{-0.038968in}{-0.067098in}}{\pgfqpoint{-0.019890in}{-0.075000in}}{\pgfqpoint{0.000000in}{-0.075000in}}%
\pgfpathclose%
\pgfusepath{stroke,fill}%
}%
\begin{pgfscope}%
\pgfsys@transformshift{0.677726in}{2.223109in}%
\pgfsys@useobject{currentmarker}{}%
\end{pgfscope}%
\end{pgfscope}%
\begin{pgfscope}%
\pgfpathrectangle{\pgfqpoint{0.100000in}{0.100000in}}{\pgfqpoint{5.307240in}{3.397500in}}%
\pgfusepath{clip}%
\pgfsetrectcap%
\pgfsetroundjoin%
\pgfsetlinewidth{1.505625pt}%
\definecolor{currentstroke}{rgb}{0.678431,1.000000,0.184314}%
\pgfsetstrokecolor{currentstroke}%
\pgfsetstrokeopacity{0.500000}%
\pgfsetdash{}{0pt}%
\pgfpathmoveto{\pgfqpoint{0.599624in}{2.015002in}}%
\pgfusepath{stroke}%
\end{pgfscope}%
\begin{pgfscope}%
\pgfpathrectangle{\pgfqpoint{0.100000in}{0.100000in}}{\pgfqpoint{5.307240in}{3.397500in}}%
\pgfusepath{clip}%
\pgfsetbuttcap%
\pgfsetroundjoin%
\definecolor{currentfill}{rgb}{0.678431,1.000000,0.184314}%
\pgfsetfillcolor{currentfill}%
\pgfsetfillopacity{0.500000}%
\pgfsetlinewidth{0.250937pt}%
\definecolor{currentstroke}{rgb}{0.000000,0.000000,0.000000}%
\pgfsetstrokecolor{currentstroke}%
\pgfsetstrokeopacity{0.500000}%
\pgfsetdash{}{0pt}%
\pgfsys@defobject{currentmarker}{\pgfqpoint{-0.090972in}{-0.090972in}}{\pgfqpoint{0.090972in}{0.090972in}}{%
\pgfpathmoveto{\pgfqpoint{0.000000in}{-0.090972in}}%
\pgfpathcurveto{\pgfqpoint{0.024126in}{-0.090972in}}{\pgfqpoint{0.047267in}{-0.081387in}}{\pgfqpoint{0.064327in}{-0.064327in}}%
\pgfpathcurveto{\pgfqpoint{0.081387in}{-0.047267in}}{\pgfqpoint{0.090972in}{-0.024126in}}{\pgfqpoint{0.090972in}{0.000000in}}%
\pgfpathcurveto{\pgfqpoint{0.090972in}{0.024126in}}{\pgfqpoint{0.081387in}{0.047267in}}{\pgfqpoint{0.064327in}{0.064327in}}%
\pgfpathcurveto{\pgfqpoint{0.047267in}{0.081387in}}{\pgfqpoint{0.024126in}{0.090972in}}{\pgfqpoint{0.000000in}{0.090972in}}%
\pgfpathcurveto{\pgfqpoint{-0.024126in}{0.090972in}}{\pgfqpoint{-0.047267in}{0.081387in}}{\pgfqpoint{-0.064327in}{0.064327in}}%
\pgfpathcurveto{\pgfqpoint{-0.081387in}{0.047267in}}{\pgfqpoint{-0.090972in}{0.024126in}}{\pgfqpoint{-0.090972in}{0.000000in}}%
\pgfpathcurveto{\pgfqpoint{-0.090972in}{-0.024126in}}{\pgfqpoint{-0.081387in}{-0.047267in}}{\pgfqpoint{-0.064327in}{-0.064327in}}%
\pgfpathcurveto{\pgfqpoint{-0.047267in}{-0.081387in}}{\pgfqpoint{-0.024126in}{-0.090972in}}{\pgfqpoint{0.000000in}{-0.090972in}}%
\pgfpathclose%
\pgfusepath{stroke,fill}%
}%
\begin{pgfscope}%
\pgfsys@transformshift{0.599624in}{2.015002in}%
\pgfsys@useobject{currentmarker}{}%
\end{pgfscope}%
\end{pgfscope}%
\begin{pgfscope}%
\pgfpathrectangle{\pgfqpoint{0.100000in}{0.100000in}}{\pgfqpoint{5.307240in}{3.397500in}}%
\pgfusepath{clip}%
\pgfsetrectcap%
\pgfsetroundjoin%
\pgfsetlinewidth{1.505625pt}%
\definecolor{currentstroke}{rgb}{0.678431,1.000000,0.184314}%
\pgfsetstrokecolor{currentstroke}%
\pgfsetstrokeopacity{0.500000}%
\pgfsetdash{}{0pt}%
\pgfpathmoveto{\pgfqpoint{0.892349in}{1.456006in}}%
\pgfusepath{stroke}%
\end{pgfscope}%
\begin{pgfscope}%
\pgfpathrectangle{\pgfqpoint{0.100000in}{0.100000in}}{\pgfqpoint{5.307240in}{3.397500in}}%
\pgfusepath{clip}%
\pgfsetbuttcap%
\pgfsetroundjoin%
\definecolor{currentfill}{rgb}{0.678431,1.000000,0.184314}%
\pgfsetfillcolor{currentfill}%
\pgfsetfillopacity{0.500000}%
\pgfsetlinewidth{0.250937pt}%
\definecolor{currentstroke}{rgb}{0.000000,0.000000,0.000000}%
\pgfsetstrokecolor{currentstroke}%
\pgfsetstrokeopacity{0.500000}%
\pgfsetdash{}{0pt}%
\pgfsys@defobject{currentmarker}{\pgfqpoint{-0.084028in}{-0.084028in}}{\pgfqpoint{0.084028in}{0.084028in}}{%
\pgfpathmoveto{\pgfqpoint{0.000000in}{-0.084028in}}%
\pgfpathcurveto{\pgfqpoint{0.022284in}{-0.084028in}}{\pgfqpoint{0.043659in}{-0.075174in}}{\pgfqpoint{0.059417in}{-0.059417in}}%
\pgfpathcurveto{\pgfqpoint{0.075174in}{-0.043659in}}{\pgfqpoint{0.084028in}{-0.022284in}}{\pgfqpoint{0.084028in}{0.000000in}}%
\pgfpathcurveto{\pgfqpoint{0.084028in}{0.022284in}}{\pgfqpoint{0.075174in}{0.043659in}}{\pgfqpoint{0.059417in}{0.059417in}}%
\pgfpathcurveto{\pgfqpoint{0.043659in}{0.075174in}}{\pgfqpoint{0.022284in}{0.084028in}}{\pgfqpoint{0.000000in}{0.084028in}}%
\pgfpathcurveto{\pgfqpoint{-0.022284in}{0.084028in}}{\pgfqpoint{-0.043659in}{0.075174in}}{\pgfqpoint{-0.059417in}{0.059417in}}%
\pgfpathcurveto{\pgfqpoint{-0.075174in}{0.043659in}}{\pgfqpoint{-0.084028in}{0.022284in}}{\pgfqpoint{-0.084028in}{0.000000in}}%
\pgfpathcurveto{\pgfqpoint{-0.084028in}{-0.022284in}}{\pgfqpoint{-0.075174in}{-0.043659in}}{\pgfqpoint{-0.059417in}{-0.059417in}}%
\pgfpathcurveto{\pgfqpoint{-0.043659in}{-0.075174in}}{\pgfqpoint{-0.022284in}{-0.084028in}}{\pgfqpoint{0.000000in}{-0.084028in}}%
\pgfpathclose%
\pgfusepath{stroke,fill}%
}%
\begin{pgfscope}%
\pgfsys@transformshift{0.892349in}{1.456006in}%
\pgfsys@useobject{currentmarker}{}%
\end{pgfscope}%
\end{pgfscope}%
\begin{pgfscope}%
\pgfpathrectangle{\pgfqpoint{0.100000in}{0.100000in}}{\pgfqpoint{5.307240in}{3.397500in}}%
\pgfusepath{clip}%
\pgfsetrectcap%
\pgfsetroundjoin%
\pgfsetlinewidth{1.505625pt}%
\definecolor{currentstroke}{rgb}{0.678431,1.000000,0.184314}%
\pgfsetstrokecolor{currentstroke}%
\pgfsetstrokeopacity{0.500000}%
\pgfsetdash{}{0pt}%
\pgfpathmoveto{\pgfqpoint{0.569631in}{2.158612in}}%
\pgfusepath{stroke}%
\end{pgfscope}%
\begin{pgfscope}%
\pgfpathrectangle{\pgfqpoint{0.100000in}{0.100000in}}{\pgfqpoint{5.307240in}{3.397500in}}%
\pgfusepath{clip}%
\pgfsetbuttcap%
\pgfsetroundjoin%
\definecolor{currentfill}{rgb}{0.678431,1.000000,0.184314}%
\pgfsetfillcolor{currentfill}%
\pgfsetfillopacity{0.500000}%
\pgfsetlinewidth{0.250937pt}%
\definecolor{currentstroke}{rgb}{0.000000,0.000000,0.000000}%
\pgfsetstrokecolor{currentstroke}%
\pgfsetstrokeopacity{0.500000}%
\pgfsetdash{}{0pt}%
\pgfsys@defobject{currentmarker}{\pgfqpoint{-0.075000in}{-0.075000in}}{\pgfqpoint{0.075000in}{0.075000in}}{%
\pgfpathmoveto{\pgfqpoint{0.000000in}{-0.075000in}}%
\pgfpathcurveto{\pgfqpoint{0.019890in}{-0.075000in}}{\pgfqpoint{0.038968in}{-0.067098in}}{\pgfqpoint{0.053033in}{-0.053033in}}%
\pgfpathcurveto{\pgfqpoint{0.067098in}{-0.038968in}}{\pgfqpoint{0.075000in}{-0.019890in}}{\pgfqpoint{0.075000in}{0.000000in}}%
\pgfpathcurveto{\pgfqpoint{0.075000in}{0.019890in}}{\pgfqpoint{0.067098in}{0.038968in}}{\pgfqpoint{0.053033in}{0.053033in}}%
\pgfpathcurveto{\pgfqpoint{0.038968in}{0.067098in}}{\pgfqpoint{0.019890in}{0.075000in}}{\pgfqpoint{0.000000in}{0.075000in}}%
\pgfpathcurveto{\pgfqpoint{-0.019890in}{0.075000in}}{\pgfqpoint{-0.038968in}{0.067098in}}{\pgfqpoint{-0.053033in}{0.053033in}}%
\pgfpathcurveto{\pgfqpoint{-0.067098in}{0.038968in}}{\pgfqpoint{-0.075000in}{0.019890in}}{\pgfqpoint{-0.075000in}{0.000000in}}%
\pgfpathcurveto{\pgfqpoint{-0.075000in}{-0.019890in}}{\pgfqpoint{-0.067098in}{-0.038968in}}{\pgfqpoint{-0.053033in}{-0.053033in}}%
\pgfpathcurveto{\pgfqpoint{-0.038968in}{-0.067098in}}{\pgfqpoint{-0.019890in}{-0.075000in}}{\pgfqpoint{0.000000in}{-0.075000in}}%
\pgfpathclose%
\pgfusepath{stroke,fill}%
}%
\begin{pgfscope}%
\pgfsys@transformshift{0.569631in}{2.158612in}%
\pgfsys@useobject{currentmarker}{}%
\end{pgfscope}%
\end{pgfscope}%
\begin{pgfscope}%
\pgfpathrectangle{\pgfqpoint{0.100000in}{0.100000in}}{\pgfqpoint{5.307240in}{3.397500in}}%
\pgfusepath{clip}%
\pgfsetrectcap%
\pgfsetroundjoin%
\pgfsetlinewidth{1.505625pt}%
\definecolor{currentstroke}{rgb}{0.678431,1.000000,0.184314}%
\pgfsetstrokecolor{currentstroke}%
\pgfsetstrokeopacity{0.500000}%
\pgfsetdash{}{0pt}%
\pgfpathmoveto{\pgfqpoint{0.601038in}{2.095021in}}%
\pgfusepath{stroke}%
\end{pgfscope}%
\begin{pgfscope}%
\pgfpathrectangle{\pgfqpoint{0.100000in}{0.100000in}}{\pgfqpoint{5.307240in}{3.397500in}}%
\pgfusepath{clip}%
\pgfsetbuttcap%
\pgfsetroundjoin%
\definecolor{currentfill}{rgb}{0.678431,1.000000,0.184314}%
\pgfsetfillcolor{currentfill}%
\pgfsetfillopacity{0.500000}%
\pgfsetlinewidth{0.250937pt}%
\definecolor{currentstroke}{rgb}{0.000000,0.000000,0.000000}%
\pgfsetstrokecolor{currentstroke}%
\pgfsetstrokeopacity{0.500000}%
\pgfsetdash{}{0pt}%
\pgfsys@defobject{currentmarker}{\pgfqpoint{-0.067361in}{-0.067361in}}{\pgfqpoint{0.067361in}{0.067361in}}{%
\pgfpathmoveto{\pgfqpoint{0.000000in}{-0.067361in}}%
\pgfpathcurveto{\pgfqpoint{0.017864in}{-0.067361in}}{\pgfqpoint{0.034999in}{-0.060264in}}{\pgfqpoint{0.047631in}{-0.047631in}}%
\pgfpathcurveto{\pgfqpoint{0.060264in}{-0.034999in}}{\pgfqpoint{0.067361in}{-0.017864in}}{\pgfqpoint{0.067361in}{0.000000in}}%
\pgfpathcurveto{\pgfqpoint{0.067361in}{0.017864in}}{\pgfqpoint{0.060264in}{0.034999in}}{\pgfqpoint{0.047631in}{0.047631in}}%
\pgfpathcurveto{\pgfqpoint{0.034999in}{0.060264in}}{\pgfqpoint{0.017864in}{0.067361in}}{\pgfqpoint{0.000000in}{0.067361in}}%
\pgfpathcurveto{\pgfqpoint{-0.017864in}{0.067361in}}{\pgfqpoint{-0.034999in}{0.060264in}}{\pgfqpoint{-0.047631in}{0.047631in}}%
\pgfpathcurveto{\pgfqpoint{-0.060264in}{0.034999in}}{\pgfqpoint{-0.067361in}{0.017864in}}{\pgfqpoint{-0.067361in}{0.000000in}}%
\pgfpathcurveto{\pgfqpoint{-0.067361in}{-0.017864in}}{\pgfqpoint{-0.060264in}{-0.034999in}}{\pgfqpoint{-0.047631in}{-0.047631in}}%
\pgfpathcurveto{\pgfqpoint{-0.034999in}{-0.060264in}}{\pgfqpoint{-0.017864in}{-0.067361in}}{\pgfqpoint{0.000000in}{-0.067361in}}%
\pgfpathclose%
\pgfusepath{stroke,fill}%
}%
\begin{pgfscope}%
\pgfsys@transformshift{0.601038in}{2.095021in}%
\pgfsys@useobject{currentmarker}{}%
\end{pgfscope}%
\end{pgfscope}%
\begin{pgfscope}%
\pgfpathrectangle{\pgfqpoint{0.100000in}{0.100000in}}{\pgfqpoint{5.307240in}{3.397500in}}%
\pgfusepath{clip}%
\pgfsetrectcap%
\pgfsetroundjoin%
\pgfsetlinewidth{1.505625pt}%
\definecolor{currentstroke}{rgb}{0.678431,1.000000,0.184314}%
\pgfsetstrokecolor{currentstroke}%
\pgfsetstrokeopacity{0.500000}%
\pgfsetdash{}{0pt}%
\pgfpathmoveto{\pgfqpoint{0.643552in}{1.833030in}}%
\pgfusepath{stroke}%
\end{pgfscope}%
\begin{pgfscope}%
\pgfpathrectangle{\pgfqpoint{0.100000in}{0.100000in}}{\pgfqpoint{5.307240in}{3.397500in}}%
\pgfusepath{clip}%
\pgfsetbuttcap%
\pgfsetroundjoin%
\definecolor{currentfill}{rgb}{0.678431,1.000000,0.184314}%
\pgfsetfillcolor{currentfill}%
\pgfsetfillopacity{0.500000}%
\pgfsetlinewidth{0.250937pt}%
\definecolor{currentstroke}{rgb}{0.000000,0.000000,0.000000}%
\pgfsetstrokecolor{currentstroke}%
\pgfsetstrokeopacity{0.500000}%
\pgfsetdash{}{0pt}%
\pgfsys@defobject{currentmarker}{\pgfqpoint{-0.077083in}{-0.077083in}}{\pgfqpoint{0.077083in}{0.077083in}}{%
\pgfpathmoveto{\pgfqpoint{0.000000in}{-0.077083in}}%
\pgfpathcurveto{\pgfqpoint{0.020443in}{-0.077083in}}{\pgfqpoint{0.040051in}{-0.068961in}}{\pgfqpoint{0.054506in}{-0.054506in}}%
\pgfpathcurveto{\pgfqpoint{0.068961in}{-0.040051in}}{\pgfqpoint{0.077083in}{-0.020443in}}{\pgfqpoint{0.077083in}{0.000000in}}%
\pgfpathcurveto{\pgfqpoint{0.077083in}{0.020443in}}{\pgfqpoint{0.068961in}{0.040051in}}{\pgfqpoint{0.054506in}{0.054506in}}%
\pgfpathcurveto{\pgfqpoint{0.040051in}{0.068961in}}{\pgfqpoint{0.020443in}{0.077083in}}{\pgfqpoint{0.000000in}{0.077083in}}%
\pgfpathcurveto{\pgfqpoint{-0.020443in}{0.077083in}}{\pgfqpoint{-0.040051in}{0.068961in}}{\pgfqpoint{-0.054506in}{0.054506in}}%
\pgfpathcurveto{\pgfqpoint{-0.068961in}{0.040051in}}{\pgfqpoint{-0.077083in}{0.020443in}}{\pgfqpoint{-0.077083in}{0.000000in}}%
\pgfpathcurveto{\pgfqpoint{-0.077083in}{-0.020443in}}{\pgfqpoint{-0.068961in}{-0.040051in}}{\pgfqpoint{-0.054506in}{-0.054506in}}%
\pgfpathcurveto{\pgfqpoint{-0.040051in}{-0.068961in}}{\pgfqpoint{-0.020443in}{-0.077083in}}{\pgfqpoint{0.000000in}{-0.077083in}}%
\pgfpathclose%
\pgfusepath{stroke,fill}%
}%
\begin{pgfscope}%
\pgfsys@transformshift{0.643552in}{1.833030in}%
\pgfsys@useobject{currentmarker}{}%
\end{pgfscope}%
\end{pgfscope}%
\begin{pgfscope}%
\pgfpathrectangle{\pgfqpoint{0.100000in}{0.100000in}}{\pgfqpoint{5.307240in}{3.397500in}}%
\pgfusepath{clip}%
\pgfsetrectcap%
\pgfsetroundjoin%
\pgfsetlinewidth{1.505625pt}%
\definecolor{currentstroke}{rgb}{0.678431,1.000000,0.184314}%
\pgfsetstrokecolor{currentstroke}%
\pgfsetstrokeopacity{0.500000}%
\pgfsetdash{}{0pt}%
\pgfpathmoveto{\pgfqpoint{0.576699in}{2.058358in}}%
\pgfusepath{stroke}%
\end{pgfscope}%
\begin{pgfscope}%
\pgfpathrectangle{\pgfqpoint{0.100000in}{0.100000in}}{\pgfqpoint{5.307240in}{3.397500in}}%
\pgfusepath{clip}%
\pgfsetbuttcap%
\pgfsetroundjoin%
\definecolor{currentfill}{rgb}{0.678431,1.000000,0.184314}%
\pgfsetfillcolor{currentfill}%
\pgfsetfillopacity{0.500000}%
\pgfsetlinewidth{0.250937pt}%
\definecolor{currentstroke}{rgb}{0.000000,0.000000,0.000000}%
\pgfsetstrokecolor{currentstroke}%
\pgfsetstrokeopacity{0.500000}%
\pgfsetdash{}{0pt}%
\pgfsys@defobject{currentmarker}{\pgfqpoint{-0.084028in}{-0.084028in}}{\pgfqpoint{0.084028in}{0.084028in}}{%
\pgfpathmoveto{\pgfqpoint{0.000000in}{-0.084028in}}%
\pgfpathcurveto{\pgfqpoint{0.022284in}{-0.084028in}}{\pgfqpoint{0.043659in}{-0.075174in}}{\pgfqpoint{0.059417in}{-0.059417in}}%
\pgfpathcurveto{\pgfqpoint{0.075174in}{-0.043659in}}{\pgfqpoint{0.084028in}{-0.022284in}}{\pgfqpoint{0.084028in}{0.000000in}}%
\pgfpathcurveto{\pgfqpoint{0.084028in}{0.022284in}}{\pgfqpoint{0.075174in}{0.043659in}}{\pgfqpoint{0.059417in}{0.059417in}}%
\pgfpathcurveto{\pgfqpoint{0.043659in}{0.075174in}}{\pgfqpoint{0.022284in}{0.084028in}}{\pgfqpoint{0.000000in}{0.084028in}}%
\pgfpathcurveto{\pgfqpoint{-0.022284in}{0.084028in}}{\pgfqpoint{-0.043659in}{0.075174in}}{\pgfqpoint{-0.059417in}{0.059417in}}%
\pgfpathcurveto{\pgfqpoint{-0.075174in}{0.043659in}}{\pgfqpoint{-0.084028in}{0.022284in}}{\pgfqpoint{-0.084028in}{0.000000in}}%
\pgfpathcurveto{\pgfqpoint{-0.084028in}{-0.022284in}}{\pgfqpoint{-0.075174in}{-0.043659in}}{\pgfqpoint{-0.059417in}{-0.059417in}}%
\pgfpathcurveto{\pgfqpoint{-0.043659in}{-0.075174in}}{\pgfqpoint{-0.022284in}{-0.084028in}}{\pgfqpoint{0.000000in}{-0.084028in}}%
\pgfpathclose%
\pgfusepath{stroke,fill}%
}%
\begin{pgfscope}%
\pgfsys@transformshift{0.576699in}{2.058358in}%
\pgfsys@useobject{currentmarker}{}%
\end{pgfscope}%
\end{pgfscope}%
\begin{pgfscope}%
\pgfpathrectangle{\pgfqpoint{0.100000in}{0.100000in}}{\pgfqpoint{5.307240in}{3.397500in}}%
\pgfusepath{clip}%
\pgfsetrectcap%
\pgfsetroundjoin%
\pgfsetlinewidth{1.505625pt}%
\definecolor{currentstroke}{rgb}{0.678431,1.000000,0.184314}%
\pgfsetstrokecolor{currentstroke}%
\pgfsetstrokeopacity{0.500000}%
\pgfsetdash{}{0pt}%
\pgfpathmoveto{\pgfqpoint{0.653386in}{1.790212in}}%
\pgfusepath{stroke}%
\end{pgfscope}%
\begin{pgfscope}%
\pgfpathrectangle{\pgfqpoint{0.100000in}{0.100000in}}{\pgfqpoint{5.307240in}{3.397500in}}%
\pgfusepath{clip}%
\pgfsetbuttcap%
\pgfsetroundjoin%
\definecolor{currentfill}{rgb}{0.678431,1.000000,0.184314}%
\pgfsetfillcolor{currentfill}%
\pgfsetfillopacity{0.500000}%
\pgfsetlinewidth{0.250937pt}%
\definecolor{currentstroke}{rgb}{0.000000,0.000000,0.000000}%
\pgfsetstrokecolor{currentstroke}%
\pgfsetstrokeopacity{0.500000}%
\pgfsetdash{}{0pt}%
\pgfsys@defobject{currentmarker}{\pgfqpoint{-0.071528in}{-0.071528in}}{\pgfqpoint{0.071528in}{0.071528in}}{%
\pgfpathmoveto{\pgfqpoint{0.000000in}{-0.071528in}}%
\pgfpathcurveto{\pgfqpoint{0.018969in}{-0.071528in}}{\pgfqpoint{0.037164in}{-0.063991in}}{\pgfqpoint{0.050578in}{-0.050578in}}%
\pgfpathcurveto{\pgfqpoint{0.063991in}{-0.037164in}}{\pgfqpoint{0.071528in}{-0.018969in}}{\pgfqpoint{0.071528in}{0.000000in}}%
\pgfpathcurveto{\pgfqpoint{0.071528in}{0.018969in}}{\pgfqpoint{0.063991in}{0.037164in}}{\pgfqpoint{0.050578in}{0.050578in}}%
\pgfpathcurveto{\pgfqpoint{0.037164in}{0.063991in}}{\pgfqpoint{0.018969in}{0.071528in}}{\pgfqpoint{0.000000in}{0.071528in}}%
\pgfpathcurveto{\pgfqpoint{-0.018969in}{0.071528in}}{\pgfqpoint{-0.037164in}{0.063991in}}{\pgfqpoint{-0.050578in}{0.050578in}}%
\pgfpathcurveto{\pgfqpoint{-0.063991in}{0.037164in}}{\pgfqpoint{-0.071528in}{0.018969in}}{\pgfqpoint{-0.071528in}{0.000000in}}%
\pgfpathcurveto{\pgfqpoint{-0.071528in}{-0.018969in}}{\pgfqpoint{-0.063991in}{-0.037164in}}{\pgfqpoint{-0.050578in}{-0.050578in}}%
\pgfpathcurveto{\pgfqpoint{-0.037164in}{-0.063991in}}{\pgfqpoint{-0.018969in}{-0.071528in}}{\pgfqpoint{0.000000in}{-0.071528in}}%
\pgfpathclose%
\pgfusepath{stroke,fill}%
}%
\begin{pgfscope}%
\pgfsys@transformshift{0.653386in}{1.790212in}%
\pgfsys@useobject{currentmarker}{}%
\end{pgfscope}%
\end{pgfscope}%
\begin{pgfscope}%
\pgfpathrectangle{\pgfqpoint{0.100000in}{0.100000in}}{\pgfqpoint{5.307240in}{3.397500in}}%
\pgfusepath{clip}%
\pgfsetrectcap%
\pgfsetroundjoin%
\pgfsetlinewidth{1.505625pt}%
\definecolor{currentstroke}{rgb}{0.678431,1.000000,0.184314}%
\pgfsetstrokecolor{currentstroke}%
\pgfsetstrokeopacity{0.500000}%
\pgfsetdash{}{0pt}%
\pgfpathmoveto{\pgfqpoint{0.566822in}{2.240043in}}%
\pgfusepath{stroke}%
\end{pgfscope}%
\begin{pgfscope}%
\pgfpathrectangle{\pgfqpoint{0.100000in}{0.100000in}}{\pgfqpoint{5.307240in}{3.397500in}}%
\pgfusepath{clip}%
\pgfsetbuttcap%
\pgfsetroundjoin%
\definecolor{currentfill}{rgb}{0.678431,1.000000,0.184314}%
\pgfsetfillcolor{currentfill}%
\pgfsetfillopacity{0.500000}%
\pgfsetlinewidth{0.250937pt}%
\definecolor{currentstroke}{rgb}{0.000000,0.000000,0.000000}%
\pgfsetstrokecolor{currentstroke}%
\pgfsetstrokeopacity{0.500000}%
\pgfsetdash{}{0pt}%
\pgfsys@defobject{currentmarker}{\pgfqpoint{-0.088194in}{-0.088194in}}{\pgfqpoint{0.088194in}{0.088194in}}{%
\pgfpathmoveto{\pgfqpoint{0.000000in}{-0.088194in}}%
\pgfpathcurveto{\pgfqpoint{0.023389in}{-0.088194in}}{\pgfqpoint{0.045824in}{-0.078902in}}{\pgfqpoint{0.062363in}{-0.062363in}}%
\pgfpathcurveto{\pgfqpoint{0.078902in}{-0.045824in}}{\pgfqpoint{0.088194in}{-0.023389in}}{\pgfqpoint{0.088194in}{0.000000in}}%
\pgfpathcurveto{\pgfqpoint{0.088194in}{0.023389in}}{\pgfqpoint{0.078902in}{0.045824in}}{\pgfqpoint{0.062363in}{0.062363in}}%
\pgfpathcurveto{\pgfqpoint{0.045824in}{0.078902in}}{\pgfqpoint{0.023389in}{0.088194in}}{\pgfqpoint{0.000000in}{0.088194in}}%
\pgfpathcurveto{\pgfqpoint{-0.023389in}{0.088194in}}{\pgfqpoint{-0.045824in}{0.078902in}}{\pgfqpoint{-0.062363in}{0.062363in}}%
\pgfpathcurveto{\pgfqpoint{-0.078902in}{0.045824in}}{\pgfqpoint{-0.088194in}{0.023389in}}{\pgfqpoint{-0.088194in}{0.000000in}}%
\pgfpathcurveto{\pgfqpoint{-0.088194in}{-0.023389in}}{\pgfqpoint{-0.078902in}{-0.045824in}}{\pgfqpoint{-0.062363in}{-0.062363in}}%
\pgfpathcurveto{\pgfqpoint{-0.045824in}{-0.078902in}}{\pgfqpoint{-0.023389in}{-0.088194in}}{\pgfqpoint{0.000000in}{-0.088194in}}%
\pgfpathclose%
\pgfusepath{stroke,fill}%
}%
\begin{pgfscope}%
\pgfsys@transformshift{0.566822in}{2.240043in}%
\pgfsys@useobject{currentmarker}{}%
\end{pgfscope}%
\end{pgfscope}%
\begin{pgfscope}%
\pgfpathrectangle{\pgfqpoint{0.100000in}{0.100000in}}{\pgfqpoint{5.307240in}{3.397500in}}%
\pgfusepath{clip}%
\pgfsetrectcap%
\pgfsetroundjoin%
\pgfsetlinewidth{1.505625pt}%
\definecolor{currentstroke}{rgb}{0.678431,1.000000,0.184314}%
\pgfsetstrokecolor{currentstroke}%
\pgfsetstrokeopacity{0.500000}%
\pgfsetdash{}{0pt}%
\pgfpathmoveto{\pgfqpoint{0.674764in}{2.148292in}}%
\pgfusepath{stroke}%
\end{pgfscope}%
\begin{pgfscope}%
\pgfpathrectangle{\pgfqpoint{0.100000in}{0.100000in}}{\pgfqpoint{5.307240in}{3.397500in}}%
\pgfusepath{clip}%
\pgfsetbuttcap%
\pgfsetroundjoin%
\definecolor{currentfill}{rgb}{0.678431,1.000000,0.184314}%
\pgfsetfillcolor{currentfill}%
\pgfsetfillopacity{0.500000}%
\pgfsetlinewidth{0.250937pt}%
\definecolor{currentstroke}{rgb}{0.000000,0.000000,0.000000}%
\pgfsetstrokecolor{currentstroke}%
\pgfsetstrokeopacity{0.500000}%
\pgfsetdash{}{0pt}%
\pgfsys@defobject{currentmarker}{\pgfqpoint{-0.086806in}{-0.086806in}}{\pgfqpoint{0.086806in}{0.086806in}}{%
\pgfpathmoveto{\pgfqpoint{0.000000in}{-0.086806in}}%
\pgfpathcurveto{\pgfqpoint{0.023021in}{-0.086806in}}{\pgfqpoint{0.045102in}{-0.077659in}}{\pgfqpoint{0.061381in}{-0.061381in}}%
\pgfpathcurveto{\pgfqpoint{0.077659in}{-0.045102in}}{\pgfqpoint{0.086806in}{-0.023021in}}{\pgfqpoint{0.086806in}{0.000000in}}%
\pgfpathcurveto{\pgfqpoint{0.086806in}{0.023021in}}{\pgfqpoint{0.077659in}{0.045102in}}{\pgfqpoint{0.061381in}{0.061381in}}%
\pgfpathcurveto{\pgfqpoint{0.045102in}{0.077659in}}{\pgfqpoint{0.023021in}{0.086806in}}{\pgfqpoint{0.000000in}{0.086806in}}%
\pgfpathcurveto{\pgfqpoint{-0.023021in}{0.086806in}}{\pgfqpoint{-0.045102in}{0.077659in}}{\pgfqpoint{-0.061381in}{0.061381in}}%
\pgfpathcurveto{\pgfqpoint{-0.077659in}{0.045102in}}{\pgfqpoint{-0.086806in}{0.023021in}}{\pgfqpoint{-0.086806in}{0.000000in}}%
\pgfpathcurveto{\pgfqpoint{-0.086806in}{-0.023021in}}{\pgfqpoint{-0.077659in}{-0.045102in}}{\pgfqpoint{-0.061381in}{-0.061381in}}%
\pgfpathcurveto{\pgfqpoint{-0.045102in}{-0.077659in}}{\pgfqpoint{-0.023021in}{-0.086806in}}{\pgfqpoint{0.000000in}{-0.086806in}}%
\pgfpathclose%
\pgfusepath{stroke,fill}%
}%
\begin{pgfscope}%
\pgfsys@transformshift{0.674764in}{2.148292in}%
\pgfsys@useobject{currentmarker}{}%
\end{pgfscope}%
\end{pgfscope}%
\begin{pgfscope}%
\pgfpathrectangle{\pgfqpoint{0.100000in}{0.100000in}}{\pgfqpoint{5.307240in}{3.397500in}}%
\pgfusepath{clip}%
\pgfsetrectcap%
\pgfsetroundjoin%
\pgfsetlinewidth{1.505625pt}%
\definecolor{currentstroke}{rgb}{0.678431,1.000000,0.184314}%
\pgfsetstrokecolor{currentstroke}%
\pgfsetstrokeopacity{0.500000}%
\pgfsetdash{}{0pt}%
\pgfpathmoveto{\pgfqpoint{0.595062in}{2.190264in}}%
\pgfusepath{stroke}%
\end{pgfscope}%
\begin{pgfscope}%
\pgfpathrectangle{\pgfqpoint{0.100000in}{0.100000in}}{\pgfqpoint{5.307240in}{3.397500in}}%
\pgfusepath{clip}%
\pgfsetbuttcap%
\pgfsetroundjoin%
\definecolor{currentfill}{rgb}{0.678431,1.000000,0.184314}%
\pgfsetfillcolor{currentfill}%
\pgfsetfillopacity{0.500000}%
\pgfsetlinewidth{0.250937pt}%
\definecolor{currentstroke}{rgb}{0.000000,0.000000,0.000000}%
\pgfsetstrokecolor{currentstroke}%
\pgfsetstrokeopacity{0.500000}%
\pgfsetdash{}{0pt}%
\pgfsys@defobject{currentmarker}{\pgfqpoint{-0.079167in}{-0.079167in}}{\pgfqpoint{0.079167in}{0.079167in}}{%
\pgfpathmoveto{\pgfqpoint{0.000000in}{-0.079167in}}%
\pgfpathcurveto{\pgfqpoint{0.020995in}{-0.079167in}}{\pgfqpoint{0.041133in}{-0.070825in}}{\pgfqpoint{0.055979in}{-0.055979in}}%
\pgfpathcurveto{\pgfqpoint{0.070825in}{-0.041133in}}{\pgfqpoint{0.079167in}{-0.020995in}}{\pgfqpoint{0.079167in}{0.000000in}}%
\pgfpathcurveto{\pgfqpoint{0.079167in}{0.020995in}}{\pgfqpoint{0.070825in}{0.041133in}}{\pgfqpoint{0.055979in}{0.055979in}}%
\pgfpathcurveto{\pgfqpoint{0.041133in}{0.070825in}}{\pgfqpoint{0.020995in}{0.079167in}}{\pgfqpoint{0.000000in}{0.079167in}}%
\pgfpathcurveto{\pgfqpoint{-0.020995in}{0.079167in}}{\pgfqpoint{-0.041133in}{0.070825in}}{\pgfqpoint{-0.055979in}{0.055979in}}%
\pgfpathcurveto{\pgfqpoint{-0.070825in}{0.041133in}}{\pgfqpoint{-0.079167in}{0.020995in}}{\pgfqpoint{-0.079167in}{0.000000in}}%
\pgfpathcurveto{\pgfqpoint{-0.079167in}{-0.020995in}}{\pgfqpoint{-0.070825in}{-0.041133in}}{\pgfqpoint{-0.055979in}{-0.055979in}}%
\pgfpathcurveto{\pgfqpoint{-0.041133in}{-0.070825in}}{\pgfqpoint{-0.020995in}{-0.079167in}}{\pgfqpoint{0.000000in}{-0.079167in}}%
\pgfpathclose%
\pgfusepath{stroke,fill}%
}%
\begin{pgfscope}%
\pgfsys@transformshift{0.595062in}{2.190264in}%
\pgfsys@useobject{currentmarker}{}%
\end{pgfscope}%
\end{pgfscope}%
\begin{pgfscope}%
\pgfpathrectangle{\pgfqpoint{0.100000in}{0.100000in}}{\pgfqpoint{5.307240in}{3.397500in}}%
\pgfusepath{clip}%
\pgfsetrectcap%
\pgfsetroundjoin%
\pgfsetlinewidth{1.505625pt}%
\definecolor{currentstroke}{rgb}{0.678431,1.000000,0.184314}%
\pgfsetstrokecolor{currentstroke}%
\pgfsetstrokeopacity{0.500000}%
\pgfsetdash{}{0pt}%
\pgfpathmoveto{\pgfqpoint{0.800943in}{1.915370in}}%
\pgfusepath{stroke}%
\end{pgfscope}%
\begin{pgfscope}%
\pgfpathrectangle{\pgfqpoint{0.100000in}{0.100000in}}{\pgfqpoint{5.307240in}{3.397500in}}%
\pgfusepath{clip}%
\pgfsetbuttcap%
\pgfsetroundjoin%
\definecolor{currentfill}{rgb}{0.678431,1.000000,0.184314}%
\pgfsetfillcolor{currentfill}%
\pgfsetfillopacity{0.500000}%
\pgfsetlinewidth{0.250937pt}%
\definecolor{currentstroke}{rgb}{0.000000,0.000000,0.000000}%
\pgfsetstrokecolor{currentstroke}%
\pgfsetstrokeopacity{0.500000}%
\pgfsetdash{}{0pt}%
\pgfsys@defobject{currentmarker}{\pgfqpoint{-0.063194in}{-0.063194in}}{\pgfqpoint{0.063194in}{0.063194in}}{%
\pgfpathmoveto{\pgfqpoint{0.000000in}{-0.063194in}}%
\pgfpathcurveto{\pgfqpoint{0.016759in}{-0.063194in}}{\pgfqpoint{0.032835in}{-0.056536in}}{\pgfqpoint{0.044685in}{-0.044685in}}%
\pgfpathcurveto{\pgfqpoint{0.056536in}{-0.032835in}}{\pgfqpoint{0.063194in}{-0.016759in}}{\pgfqpoint{0.063194in}{0.000000in}}%
\pgfpathcurveto{\pgfqpoint{0.063194in}{0.016759in}}{\pgfqpoint{0.056536in}{0.032835in}}{\pgfqpoint{0.044685in}{0.044685in}}%
\pgfpathcurveto{\pgfqpoint{0.032835in}{0.056536in}}{\pgfqpoint{0.016759in}{0.063194in}}{\pgfqpoint{0.000000in}{0.063194in}}%
\pgfpathcurveto{\pgfqpoint{-0.016759in}{0.063194in}}{\pgfqpoint{-0.032835in}{0.056536in}}{\pgfqpoint{-0.044685in}{0.044685in}}%
\pgfpathcurveto{\pgfqpoint{-0.056536in}{0.032835in}}{\pgfqpoint{-0.063194in}{0.016759in}}{\pgfqpoint{-0.063194in}{0.000000in}}%
\pgfpathcurveto{\pgfqpoint{-0.063194in}{-0.016759in}}{\pgfqpoint{-0.056536in}{-0.032835in}}{\pgfqpoint{-0.044685in}{-0.044685in}}%
\pgfpathcurveto{\pgfqpoint{-0.032835in}{-0.056536in}}{\pgfqpoint{-0.016759in}{-0.063194in}}{\pgfqpoint{0.000000in}{-0.063194in}}%
\pgfpathclose%
\pgfusepath{stroke,fill}%
}%
\begin{pgfscope}%
\pgfsys@transformshift{0.800943in}{1.915370in}%
\pgfsys@useobject{currentmarker}{}%
\end{pgfscope}%
\end{pgfscope}%
\begin{pgfscope}%
\pgfpathrectangle{\pgfqpoint{0.100000in}{0.100000in}}{\pgfqpoint{5.307240in}{3.397500in}}%
\pgfusepath{clip}%
\pgfsetrectcap%
\pgfsetroundjoin%
\pgfsetlinewidth{1.505625pt}%
\definecolor{currentstroke}{rgb}{0.678431,1.000000,0.184314}%
\pgfsetstrokecolor{currentstroke}%
\pgfsetstrokeopacity{0.500000}%
\pgfsetdash{}{0pt}%
\pgfpathmoveto{\pgfqpoint{0.685823in}{2.288520in}}%
\pgfusepath{stroke}%
\end{pgfscope}%
\begin{pgfscope}%
\pgfpathrectangle{\pgfqpoint{0.100000in}{0.100000in}}{\pgfqpoint{5.307240in}{3.397500in}}%
\pgfusepath{clip}%
\pgfsetbuttcap%
\pgfsetroundjoin%
\definecolor{currentfill}{rgb}{0.678431,1.000000,0.184314}%
\pgfsetfillcolor{currentfill}%
\pgfsetfillopacity{0.500000}%
\pgfsetlinewidth{0.250937pt}%
\definecolor{currentstroke}{rgb}{0.000000,0.000000,0.000000}%
\pgfsetstrokecolor{currentstroke}%
\pgfsetstrokeopacity{0.500000}%
\pgfsetdash{}{0pt}%
\pgfsys@defobject{currentmarker}{\pgfqpoint{-0.066667in}{-0.066667in}}{\pgfqpoint{0.066667in}{0.066667in}}{%
\pgfpathmoveto{\pgfqpoint{0.000000in}{-0.066667in}}%
\pgfpathcurveto{\pgfqpoint{0.017680in}{-0.066667in}}{\pgfqpoint{0.034639in}{-0.059642in}}{\pgfqpoint{0.047140in}{-0.047140in}}%
\pgfpathcurveto{\pgfqpoint{0.059642in}{-0.034639in}}{\pgfqpoint{0.066667in}{-0.017680in}}{\pgfqpoint{0.066667in}{0.000000in}}%
\pgfpathcurveto{\pgfqpoint{0.066667in}{0.017680in}}{\pgfqpoint{0.059642in}{0.034639in}}{\pgfqpoint{0.047140in}{0.047140in}}%
\pgfpathcurveto{\pgfqpoint{0.034639in}{0.059642in}}{\pgfqpoint{0.017680in}{0.066667in}}{\pgfqpoint{0.000000in}{0.066667in}}%
\pgfpathcurveto{\pgfqpoint{-0.017680in}{0.066667in}}{\pgfqpoint{-0.034639in}{0.059642in}}{\pgfqpoint{-0.047140in}{0.047140in}}%
\pgfpathcurveto{\pgfqpoint{-0.059642in}{0.034639in}}{\pgfqpoint{-0.066667in}{0.017680in}}{\pgfqpoint{-0.066667in}{0.000000in}}%
\pgfpathcurveto{\pgfqpoint{-0.066667in}{-0.017680in}}{\pgfqpoint{-0.059642in}{-0.034639in}}{\pgfqpoint{-0.047140in}{-0.047140in}}%
\pgfpathcurveto{\pgfqpoint{-0.034639in}{-0.059642in}}{\pgfqpoint{-0.017680in}{-0.066667in}}{\pgfqpoint{0.000000in}{-0.066667in}}%
\pgfpathclose%
\pgfusepath{stroke,fill}%
}%
\begin{pgfscope}%
\pgfsys@transformshift{0.685823in}{2.288520in}%
\pgfsys@useobject{currentmarker}{}%
\end{pgfscope}%
\end{pgfscope}%
\begin{pgfscope}%
\pgfpathrectangle{\pgfqpoint{0.100000in}{0.100000in}}{\pgfqpoint{5.307240in}{3.397500in}}%
\pgfusepath{clip}%
\pgfsetrectcap%
\pgfsetroundjoin%
\pgfsetlinewidth{1.505625pt}%
\definecolor{currentstroke}{rgb}{0.678431,1.000000,0.184314}%
\pgfsetstrokecolor{currentstroke}%
\pgfsetstrokeopacity{0.500000}%
\pgfsetdash{}{0pt}%
\pgfpathmoveto{\pgfqpoint{2.138936in}{2.092869in}}%
\pgfusepath{stroke}%
\end{pgfscope}%
\begin{pgfscope}%
\pgfpathrectangle{\pgfqpoint{0.100000in}{0.100000in}}{\pgfqpoint{5.307240in}{3.397500in}}%
\pgfusepath{clip}%
\pgfsetbuttcap%
\pgfsetroundjoin%
\definecolor{currentfill}{rgb}{0.678431,1.000000,0.184314}%
\pgfsetfillcolor{currentfill}%
\pgfsetfillopacity{0.500000}%
\pgfsetlinewidth{0.250937pt}%
\definecolor{currentstroke}{rgb}{0.000000,0.000000,0.000000}%
\pgfsetstrokecolor{currentstroke}%
\pgfsetstrokeopacity{0.500000}%
\pgfsetdash{}{0pt}%
\pgfsys@defobject{currentmarker}{\pgfqpoint{-0.050694in}{-0.050694in}}{\pgfqpoint{0.050694in}{0.050694in}}{%
\pgfpathmoveto{\pgfqpoint{0.000000in}{-0.050694in}}%
\pgfpathcurveto{\pgfqpoint{0.013444in}{-0.050694in}}{\pgfqpoint{0.026340in}{-0.045353in}}{\pgfqpoint{0.035846in}{-0.035846in}}%
\pgfpathcurveto{\pgfqpoint{0.045353in}{-0.026340in}}{\pgfqpoint{0.050694in}{-0.013444in}}{\pgfqpoint{0.050694in}{0.000000in}}%
\pgfpathcurveto{\pgfqpoint{0.050694in}{0.013444in}}{\pgfqpoint{0.045353in}{0.026340in}}{\pgfqpoint{0.035846in}{0.035846in}}%
\pgfpathcurveto{\pgfqpoint{0.026340in}{0.045353in}}{\pgfqpoint{0.013444in}{0.050694in}}{\pgfqpoint{0.000000in}{0.050694in}}%
\pgfpathcurveto{\pgfqpoint{-0.013444in}{0.050694in}}{\pgfqpoint{-0.026340in}{0.045353in}}{\pgfqpoint{-0.035846in}{0.035846in}}%
\pgfpathcurveto{\pgfqpoint{-0.045353in}{0.026340in}}{\pgfqpoint{-0.050694in}{0.013444in}}{\pgfqpoint{-0.050694in}{0.000000in}}%
\pgfpathcurveto{\pgfqpoint{-0.050694in}{-0.013444in}}{\pgfqpoint{-0.045353in}{-0.026340in}}{\pgfqpoint{-0.035846in}{-0.035846in}}%
\pgfpathcurveto{\pgfqpoint{-0.026340in}{-0.045353in}}{\pgfqpoint{-0.013444in}{-0.050694in}}{\pgfqpoint{0.000000in}{-0.050694in}}%
\pgfpathclose%
\pgfusepath{stroke,fill}%
}%
\begin{pgfscope}%
\pgfsys@transformshift{2.138936in}{2.092869in}%
\pgfsys@useobject{currentmarker}{}%
\end{pgfscope}%
\end{pgfscope}%
\begin{pgfscope}%
\pgfpathrectangle{\pgfqpoint{0.100000in}{0.100000in}}{\pgfqpoint{5.307240in}{3.397500in}}%
\pgfusepath{clip}%
\pgfsetrectcap%
\pgfsetroundjoin%
\pgfsetlinewidth{1.505625pt}%
\definecolor{currentstroke}{rgb}{0.678431,1.000000,0.184314}%
\pgfsetstrokecolor{currentstroke}%
\pgfsetstrokeopacity{0.500000}%
\pgfsetdash{}{0pt}%
\pgfpathmoveto{\pgfqpoint{2.163521in}{1.951998in}}%
\pgfusepath{stroke}%
\end{pgfscope}%
\begin{pgfscope}%
\pgfpathrectangle{\pgfqpoint{0.100000in}{0.100000in}}{\pgfqpoint{5.307240in}{3.397500in}}%
\pgfusepath{clip}%
\pgfsetbuttcap%
\pgfsetroundjoin%
\definecolor{currentfill}{rgb}{0.678431,1.000000,0.184314}%
\pgfsetfillcolor{currentfill}%
\pgfsetfillopacity{0.500000}%
\pgfsetlinewidth{0.250937pt}%
\definecolor{currentstroke}{rgb}{0.000000,0.000000,0.000000}%
\pgfsetstrokecolor{currentstroke}%
\pgfsetstrokeopacity{0.500000}%
\pgfsetdash{}{0pt}%
\pgfsys@defobject{currentmarker}{\pgfqpoint{-0.062500in}{-0.062500in}}{\pgfqpoint{0.062500in}{0.062500in}}{%
\pgfpathmoveto{\pgfqpoint{0.000000in}{-0.062500in}}%
\pgfpathcurveto{\pgfqpoint{0.016575in}{-0.062500in}}{\pgfqpoint{0.032474in}{-0.055915in}}{\pgfqpoint{0.044194in}{-0.044194in}}%
\pgfpathcurveto{\pgfqpoint{0.055915in}{-0.032474in}}{\pgfqpoint{0.062500in}{-0.016575in}}{\pgfqpoint{0.062500in}{0.000000in}}%
\pgfpathcurveto{\pgfqpoint{0.062500in}{0.016575in}}{\pgfqpoint{0.055915in}{0.032474in}}{\pgfqpoint{0.044194in}{0.044194in}}%
\pgfpathcurveto{\pgfqpoint{0.032474in}{0.055915in}}{\pgfqpoint{0.016575in}{0.062500in}}{\pgfqpoint{0.000000in}{0.062500in}}%
\pgfpathcurveto{\pgfqpoint{-0.016575in}{0.062500in}}{\pgfqpoint{-0.032474in}{0.055915in}}{\pgfqpoint{-0.044194in}{0.044194in}}%
\pgfpathcurveto{\pgfqpoint{-0.055915in}{0.032474in}}{\pgfqpoint{-0.062500in}{0.016575in}}{\pgfqpoint{-0.062500in}{0.000000in}}%
\pgfpathcurveto{\pgfqpoint{-0.062500in}{-0.016575in}}{\pgfqpoint{-0.055915in}{-0.032474in}}{\pgfqpoint{-0.044194in}{-0.044194in}}%
\pgfpathcurveto{\pgfqpoint{-0.032474in}{-0.055915in}}{\pgfqpoint{-0.016575in}{-0.062500in}}{\pgfqpoint{0.000000in}{-0.062500in}}%
\pgfpathclose%
\pgfusepath{stroke,fill}%
}%
\begin{pgfscope}%
\pgfsys@transformshift{2.163521in}{1.951998in}%
\pgfsys@useobject{currentmarker}{}%
\end{pgfscope}%
\end{pgfscope}%
\begin{pgfscope}%
\pgfpathrectangle{\pgfqpoint{0.100000in}{0.100000in}}{\pgfqpoint{5.307240in}{3.397500in}}%
\pgfusepath{clip}%
\pgfsetrectcap%
\pgfsetroundjoin%
\pgfsetlinewidth{1.505625pt}%
\definecolor{currentstroke}{rgb}{0.678431,1.000000,0.184314}%
\pgfsetstrokecolor{currentstroke}%
\pgfsetstrokeopacity{0.500000}%
\pgfsetdash{}{0pt}%
\pgfpathmoveto{\pgfqpoint{2.160715in}{2.058163in}}%
\pgfusepath{stroke}%
\end{pgfscope}%
\begin{pgfscope}%
\pgfpathrectangle{\pgfqpoint{0.100000in}{0.100000in}}{\pgfqpoint{5.307240in}{3.397500in}}%
\pgfusepath{clip}%
\pgfsetbuttcap%
\pgfsetroundjoin%
\definecolor{currentfill}{rgb}{0.678431,1.000000,0.184314}%
\pgfsetfillcolor{currentfill}%
\pgfsetfillopacity{0.500000}%
\pgfsetlinewidth{0.250937pt}%
\definecolor{currentstroke}{rgb}{0.000000,0.000000,0.000000}%
\pgfsetstrokecolor{currentstroke}%
\pgfsetstrokeopacity{0.500000}%
\pgfsetdash{}{0pt}%
\pgfsys@defobject{currentmarker}{\pgfqpoint{-0.065278in}{-0.065278in}}{\pgfqpoint{0.065278in}{0.065278in}}{%
\pgfpathmoveto{\pgfqpoint{0.000000in}{-0.065278in}}%
\pgfpathcurveto{\pgfqpoint{0.017312in}{-0.065278in}}{\pgfqpoint{0.033917in}{-0.058400in}}{\pgfqpoint{0.046158in}{-0.046158in}}%
\pgfpathcurveto{\pgfqpoint{0.058400in}{-0.033917in}}{\pgfqpoint{0.065278in}{-0.017312in}}{\pgfqpoint{0.065278in}{0.000000in}}%
\pgfpathcurveto{\pgfqpoint{0.065278in}{0.017312in}}{\pgfqpoint{0.058400in}{0.033917in}}{\pgfqpoint{0.046158in}{0.046158in}}%
\pgfpathcurveto{\pgfqpoint{0.033917in}{0.058400in}}{\pgfqpoint{0.017312in}{0.065278in}}{\pgfqpoint{0.000000in}{0.065278in}}%
\pgfpathcurveto{\pgfqpoint{-0.017312in}{0.065278in}}{\pgfqpoint{-0.033917in}{0.058400in}}{\pgfqpoint{-0.046158in}{0.046158in}}%
\pgfpathcurveto{\pgfqpoint{-0.058400in}{0.033917in}}{\pgfqpoint{-0.065278in}{0.017312in}}{\pgfqpoint{-0.065278in}{0.000000in}}%
\pgfpathcurveto{\pgfqpoint{-0.065278in}{-0.017312in}}{\pgfqpoint{-0.058400in}{-0.033917in}}{\pgfqpoint{-0.046158in}{-0.046158in}}%
\pgfpathcurveto{\pgfqpoint{-0.033917in}{-0.058400in}}{\pgfqpoint{-0.017312in}{-0.065278in}}{\pgfqpoint{0.000000in}{-0.065278in}}%
\pgfpathclose%
\pgfusepath{stroke,fill}%
}%
\begin{pgfscope}%
\pgfsys@transformshift{2.160715in}{2.058163in}%
\pgfsys@useobject{currentmarker}{}%
\end{pgfscope}%
\end{pgfscope}%
\begin{pgfscope}%
\pgfpathrectangle{\pgfqpoint{0.100000in}{0.100000in}}{\pgfqpoint{5.307240in}{3.397500in}}%
\pgfusepath{clip}%
\pgfsetrectcap%
\pgfsetroundjoin%
\pgfsetlinewidth{1.505625pt}%
\definecolor{currentstroke}{rgb}{0.678431,1.000000,0.184314}%
\pgfsetstrokecolor{currentstroke}%
\pgfsetstrokeopacity{0.500000}%
\pgfsetdash{}{0pt}%
\pgfpathmoveto{\pgfqpoint{2.163853in}{2.152770in}}%
\pgfusepath{stroke}%
\end{pgfscope}%
\begin{pgfscope}%
\pgfpathrectangle{\pgfqpoint{0.100000in}{0.100000in}}{\pgfqpoint{5.307240in}{3.397500in}}%
\pgfusepath{clip}%
\pgfsetbuttcap%
\pgfsetroundjoin%
\definecolor{currentfill}{rgb}{0.678431,1.000000,0.184314}%
\pgfsetfillcolor{currentfill}%
\pgfsetfillopacity{0.500000}%
\pgfsetlinewidth{0.250937pt}%
\definecolor{currentstroke}{rgb}{0.000000,0.000000,0.000000}%
\pgfsetstrokecolor{currentstroke}%
\pgfsetstrokeopacity{0.500000}%
\pgfsetdash{}{0pt}%
\pgfsys@defobject{currentmarker}{\pgfqpoint{-0.059722in}{-0.059722in}}{\pgfqpoint{0.059722in}{0.059722in}}{%
\pgfpathmoveto{\pgfqpoint{0.000000in}{-0.059722in}}%
\pgfpathcurveto{\pgfqpoint{0.015839in}{-0.059722in}}{\pgfqpoint{0.031030in}{-0.053430in}}{\pgfqpoint{0.042230in}{-0.042230in}}%
\pgfpathcurveto{\pgfqpoint{0.053430in}{-0.031030in}}{\pgfqpoint{0.059722in}{-0.015839in}}{\pgfqpoint{0.059722in}{0.000000in}}%
\pgfpathcurveto{\pgfqpoint{0.059722in}{0.015839in}}{\pgfqpoint{0.053430in}{0.031030in}}{\pgfqpoint{0.042230in}{0.042230in}}%
\pgfpathcurveto{\pgfqpoint{0.031030in}{0.053430in}}{\pgfqpoint{0.015839in}{0.059722in}}{\pgfqpoint{0.000000in}{0.059722in}}%
\pgfpathcurveto{\pgfqpoint{-0.015839in}{0.059722in}}{\pgfqpoint{-0.031030in}{0.053430in}}{\pgfqpoint{-0.042230in}{0.042230in}}%
\pgfpathcurveto{\pgfqpoint{-0.053430in}{0.031030in}}{\pgfqpoint{-0.059722in}{0.015839in}}{\pgfqpoint{-0.059722in}{0.000000in}}%
\pgfpathcurveto{\pgfqpoint{-0.059722in}{-0.015839in}}{\pgfqpoint{-0.053430in}{-0.031030in}}{\pgfqpoint{-0.042230in}{-0.042230in}}%
\pgfpathcurveto{\pgfqpoint{-0.031030in}{-0.053430in}}{\pgfqpoint{-0.015839in}{-0.059722in}}{\pgfqpoint{0.000000in}{-0.059722in}}%
\pgfpathclose%
\pgfusepath{stroke,fill}%
}%
\begin{pgfscope}%
\pgfsys@transformshift{2.163853in}{2.152770in}%
\pgfsys@useobject{currentmarker}{}%
\end{pgfscope}%
\end{pgfscope}%
\begin{pgfscope}%
\pgfpathrectangle{\pgfqpoint{0.100000in}{0.100000in}}{\pgfqpoint{5.307240in}{3.397500in}}%
\pgfusepath{clip}%
\pgfsetrectcap%
\pgfsetroundjoin%
\pgfsetlinewidth{1.505625pt}%
\definecolor{currentstroke}{rgb}{0.678431,1.000000,0.184314}%
\pgfsetstrokecolor{currentstroke}%
\pgfsetstrokeopacity{0.500000}%
\pgfsetdash{}{0pt}%
\pgfpathmoveto{\pgfqpoint{1.832949in}{2.021704in}}%
\pgfusepath{stroke}%
\end{pgfscope}%
\begin{pgfscope}%
\pgfpathrectangle{\pgfqpoint{0.100000in}{0.100000in}}{\pgfqpoint{5.307240in}{3.397500in}}%
\pgfusepath{clip}%
\pgfsetbuttcap%
\pgfsetroundjoin%
\definecolor{currentfill}{rgb}{0.678431,1.000000,0.184314}%
\pgfsetfillcolor{currentfill}%
\pgfsetfillopacity{0.500000}%
\pgfsetlinewidth{0.250937pt}%
\definecolor{currentstroke}{rgb}{0.000000,0.000000,0.000000}%
\pgfsetstrokecolor{currentstroke}%
\pgfsetstrokeopacity{0.500000}%
\pgfsetdash{}{0pt}%
\pgfsys@defobject{currentmarker}{\pgfqpoint{-0.064583in}{-0.064583in}}{\pgfqpoint{0.064583in}{0.064583in}}{%
\pgfpathmoveto{\pgfqpoint{0.000000in}{-0.064583in}}%
\pgfpathcurveto{\pgfqpoint{0.017128in}{-0.064583in}}{\pgfqpoint{0.033556in}{-0.057778in}}{\pgfqpoint{0.045667in}{-0.045667in}}%
\pgfpathcurveto{\pgfqpoint{0.057778in}{-0.033556in}}{\pgfqpoint{0.064583in}{-0.017128in}}{\pgfqpoint{0.064583in}{0.000000in}}%
\pgfpathcurveto{\pgfqpoint{0.064583in}{0.017128in}}{\pgfqpoint{0.057778in}{0.033556in}}{\pgfqpoint{0.045667in}{0.045667in}}%
\pgfpathcurveto{\pgfqpoint{0.033556in}{0.057778in}}{\pgfqpoint{0.017128in}{0.064583in}}{\pgfqpoint{0.000000in}{0.064583in}}%
\pgfpathcurveto{\pgfqpoint{-0.017128in}{0.064583in}}{\pgfqpoint{-0.033556in}{0.057778in}}{\pgfqpoint{-0.045667in}{0.045667in}}%
\pgfpathcurveto{\pgfqpoint{-0.057778in}{0.033556in}}{\pgfqpoint{-0.064583in}{0.017128in}}{\pgfqpoint{-0.064583in}{0.000000in}}%
\pgfpathcurveto{\pgfqpoint{-0.064583in}{-0.017128in}}{\pgfqpoint{-0.057778in}{-0.033556in}}{\pgfqpoint{-0.045667in}{-0.045667in}}%
\pgfpathcurveto{\pgfqpoint{-0.033556in}{-0.057778in}}{\pgfqpoint{-0.017128in}{-0.064583in}}{\pgfqpoint{0.000000in}{-0.064583in}}%
\pgfpathclose%
\pgfusepath{stroke,fill}%
}%
\begin{pgfscope}%
\pgfsys@transformshift{1.832949in}{2.021704in}%
\pgfsys@useobject{currentmarker}{}%
\end{pgfscope}%
\end{pgfscope}%
\begin{pgfscope}%
\pgfpathrectangle{\pgfqpoint{0.100000in}{0.100000in}}{\pgfqpoint{5.307240in}{3.397500in}}%
\pgfusepath{clip}%
\pgfsetrectcap%
\pgfsetroundjoin%
\pgfsetlinewidth{1.505625pt}%
\definecolor{currentstroke}{rgb}{0.678431,1.000000,0.184314}%
\pgfsetstrokecolor{currentstroke}%
\pgfsetstrokeopacity{0.500000}%
\pgfsetdash{}{0pt}%
\pgfpathmoveto{\pgfqpoint{2.193688in}{2.134591in}}%
\pgfusepath{stroke}%
\end{pgfscope}%
\begin{pgfscope}%
\pgfpathrectangle{\pgfqpoint{0.100000in}{0.100000in}}{\pgfqpoint{5.307240in}{3.397500in}}%
\pgfusepath{clip}%
\pgfsetbuttcap%
\pgfsetroundjoin%
\definecolor{currentfill}{rgb}{0.678431,1.000000,0.184314}%
\pgfsetfillcolor{currentfill}%
\pgfsetfillopacity{0.500000}%
\pgfsetlinewidth{0.250937pt}%
\definecolor{currentstroke}{rgb}{0.000000,0.000000,0.000000}%
\pgfsetstrokecolor{currentstroke}%
\pgfsetstrokeopacity{0.500000}%
\pgfsetdash{}{0pt}%
\pgfsys@defobject{currentmarker}{\pgfqpoint{-0.050694in}{-0.050694in}}{\pgfqpoint{0.050694in}{0.050694in}}{%
\pgfpathmoveto{\pgfqpoint{0.000000in}{-0.050694in}}%
\pgfpathcurveto{\pgfqpoint{0.013444in}{-0.050694in}}{\pgfqpoint{0.026340in}{-0.045353in}}{\pgfqpoint{0.035846in}{-0.035846in}}%
\pgfpathcurveto{\pgfqpoint{0.045353in}{-0.026340in}}{\pgfqpoint{0.050694in}{-0.013444in}}{\pgfqpoint{0.050694in}{0.000000in}}%
\pgfpathcurveto{\pgfqpoint{0.050694in}{0.013444in}}{\pgfqpoint{0.045353in}{0.026340in}}{\pgfqpoint{0.035846in}{0.035846in}}%
\pgfpathcurveto{\pgfqpoint{0.026340in}{0.045353in}}{\pgfqpoint{0.013444in}{0.050694in}}{\pgfqpoint{0.000000in}{0.050694in}}%
\pgfpathcurveto{\pgfqpoint{-0.013444in}{0.050694in}}{\pgfqpoint{-0.026340in}{0.045353in}}{\pgfqpoint{-0.035846in}{0.035846in}}%
\pgfpathcurveto{\pgfqpoint{-0.045353in}{0.026340in}}{\pgfqpoint{-0.050694in}{0.013444in}}{\pgfqpoint{-0.050694in}{0.000000in}}%
\pgfpathcurveto{\pgfqpoint{-0.050694in}{-0.013444in}}{\pgfqpoint{-0.045353in}{-0.026340in}}{\pgfqpoint{-0.035846in}{-0.035846in}}%
\pgfpathcurveto{\pgfqpoint{-0.026340in}{-0.045353in}}{\pgfqpoint{-0.013444in}{-0.050694in}}{\pgfqpoint{0.000000in}{-0.050694in}}%
\pgfpathclose%
\pgfusepath{stroke,fill}%
}%
\begin{pgfscope}%
\pgfsys@transformshift{2.193688in}{2.134591in}%
\pgfsys@useobject{currentmarker}{}%
\end{pgfscope}%
\end{pgfscope}%
\begin{pgfscope}%
\pgfpathrectangle{\pgfqpoint{0.100000in}{0.100000in}}{\pgfqpoint{5.307240in}{3.397500in}}%
\pgfusepath{clip}%
\pgfsetrectcap%
\pgfsetroundjoin%
\pgfsetlinewidth{1.505625pt}%
\definecolor{currentstroke}{rgb}{0.678431,1.000000,0.184314}%
\pgfsetstrokecolor{currentstroke}%
\pgfsetstrokeopacity{0.500000}%
\pgfsetdash{}{0pt}%
\pgfpathmoveto{\pgfqpoint{2.175878in}{1.882913in}}%
\pgfusepath{stroke}%
\end{pgfscope}%
\begin{pgfscope}%
\pgfpathrectangle{\pgfqpoint{0.100000in}{0.100000in}}{\pgfqpoint{5.307240in}{3.397500in}}%
\pgfusepath{clip}%
\pgfsetbuttcap%
\pgfsetroundjoin%
\definecolor{currentfill}{rgb}{0.678431,1.000000,0.184314}%
\pgfsetfillcolor{currentfill}%
\pgfsetfillopacity{0.500000}%
\pgfsetlinewidth{0.250937pt}%
\definecolor{currentstroke}{rgb}{0.000000,0.000000,0.000000}%
\pgfsetstrokecolor{currentstroke}%
\pgfsetstrokeopacity{0.500000}%
\pgfsetdash{}{0pt}%
\pgfsys@defobject{currentmarker}{\pgfqpoint{-0.052778in}{-0.052778in}}{\pgfqpoint{0.052778in}{0.052778in}}{%
\pgfpathmoveto{\pgfqpoint{0.000000in}{-0.052778in}}%
\pgfpathcurveto{\pgfqpoint{0.013997in}{-0.052778in}}{\pgfqpoint{0.027422in}{-0.047217in}}{\pgfqpoint{0.037320in}{-0.037320in}}%
\pgfpathcurveto{\pgfqpoint{0.047217in}{-0.027422in}}{\pgfqpoint{0.052778in}{-0.013997in}}{\pgfqpoint{0.052778in}{0.000000in}}%
\pgfpathcurveto{\pgfqpoint{0.052778in}{0.013997in}}{\pgfqpoint{0.047217in}{0.027422in}}{\pgfqpoint{0.037320in}{0.037320in}}%
\pgfpathcurveto{\pgfqpoint{0.027422in}{0.047217in}}{\pgfqpoint{0.013997in}{0.052778in}}{\pgfqpoint{0.000000in}{0.052778in}}%
\pgfpathcurveto{\pgfqpoint{-0.013997in}{0.052778in}}{\pgfqpoint{-0.027422in}{0.047217in}}{\pgfqpoint{-0.037320in}{0.037320in}}%
\pgfpathcurveto{\pgfqpoint{-0.047217in}{0.027422in}}{\pgfqpoint{-0.052778in}{0.013997in}}{\pgfqpoint{-0.052778in}{0.000000in}}%
\pgfpathcurveto{\pgfqpoint{-0.052778in}{-0.013997in}}{\pgfqpoint{-0.047217in}{-0.027422in}}{\pgfqpoint{-0.037320in}{-0.037320in}}%
\pgfpathcurveto{\pgfqpoint{-0.027422in}{-0.047217in}}{\pgfqpoint{-0.013997in}{-0.052778in}}{\pgfqpoint{0.000000in}{-0.052778in}}%
\pgfpathclose%
\pgfusepath{stroke,fill}%
}%
\begin{pgfscope}%
\pgfsys@transformshift{2.175878in}{1.882913in}%
\pgfsys@useobject{currentmarker}{}%
\end{pgfscope}%
\end{pgfscope}%
\begin{pgfscope}%
\pgfpathrectangle{\pgfqpoint{0.100000in}{0.100000in}}{\pgfqpoint{5.307240in}{3.397500in}}%
\pgfusepath{clip}%
\pgfsetrectcap%
\pgfsetroundjoin%
\pgfsetlinewidth{1.505625pt}%
\definecolor{currentstroke}{rgb}{0.678431,1.000000,0.184314}%
\pgfsetstrokecolor{currentstroke}%
\pgfsetstrokeopacity{0.500000}%
\pgfsetdash{}{0pt}%
\pgfpathmoveto{\pgfqpoint{4.941893in}{2.402998in}}%
\pgfusepath{stroke}%
\end{pgfscope}%
\begin{pgfscope}%
\pgfpathrectangle{\pgfqpoint{0.100000in}{0.100000in}}{\pgfqpoint{5.307240in}{3.397500in}}%
\pgfusepath{clip}%
\pgfsetbuttcap%
\pgfsetroundjoin%
\definecolor{currentfill}{rgb}{0.678431,1.000000,0.184314}%
\pgfsetfillcolor{currentfill}%
\pgfsetfillopacity{0.500000}%
\pgfsetlinewidth{0.250937pt}%
\definecolor{currentstroke}{rgb}{0.000000,0.000000,0.000000}%
\pgfsetstrokecolor{currentstroke}%
\pgfsetstrokeopacity{0.500000}%
\pgfsetdash{}{0pt}%
\pgfsys@defobject{currentmarker}{\pgfqpoint{-0.031944in}{-0.031944in}}{\pgfqpoint{0.031944in}{0.031944in}}{%
\pgfpathmoveto{\pgfqpoint{0.000000in}{-0.031944in}}%
\pgfpathcurveto{\pgfqpoint{0.008472in}{-0.031944in}}{\pgfqpoint{0.016598in}{-0.028579in}}{\pgfqpoint{0.022588in}{-0.022588in}}%
\pgfpathcurveto{\pgfqpoint{0.028579in}{-0.016598in}}{\pgfqpoint{0.031944in}{-0.008472in}}{\pgfqpoint{0.031944in}{0.000000in}}%
\pgfpathcurveto{\pgfqpoint{0.031944in}{0.008472in}}{\pgfqpoint{0.028579in}{0.016598in}}{\pgfqpoint{0.022588in}{0.022588in}}%
\pgfpathcurveto{\pgfqpoint{0.016598in}{0.028579in}}{\pgfqpoint{0.008472in}{0.031944in}}{\pgfqpoint{0.000000in}{0.031944in}}%
\pgfpathcurveto{\pgfqpoint{-0.008472in}{0.031944in}}{\pgfqpoint{-0.016598in}{0.028579in}}{\pgfqpoint{-0.022588in}{0.022588in}}%
\pgfpathcurveto{\pgfqpoint{-0.028579in}{0.016598in}}{\pgfqpoint{-0.031944in}{0.008472in}}{\pgfqpoint{-0.031944in}{0.000000in}}%
\pgfpathcurveto{\pgfqpoint{-0.031944in}{-0.008472in}}{\pgfqpoint{-0.028579in}{-0.016598in}}{\pgfqpoint{-0.022588in}{-0.022588in}}%
\pgfpathcurveto{\pgfqpoint{-0.016598in}{-0.028579in}}{\pgfqpoint{-0.008472in}{-0.031944in}}{\pgfqpoint{0.000000in}{-0.031944in}}%
\pgfpathclose%
\pgfusepath{stroke,fill}%
}%
\begin{pgfscope}%
\pgfsys@transformshift{4.941893in}{2.402998in}%
\pgfsys@useobject{currentmarker}{}%
\end{pgfscope}%
\end{pgfscope}%
\begin{pgfscope}%
\pgfpathrectangle{\pgfqpoint{0.100000in}{0.100000in}}{\pgfqpoint{5.307240in}{3.397500in}}%
\pgfusepath{clip}%
\pgfsetrectcap%
\pgfsetroundjoin%
\pgfsetlinewidth{1.505625pt}%
\definecolor{currentstroke}{rgb}{0.678431,1.000000,0.184314}%
\pgfsetstrokecolor{currentstroke}%
\pgfsetstrokeopacity{0.500000}%
\pgfsetdash{}{0pt}%
\pgfpathmoveto{\pgfqpoint{4.914466in}{2.423601in}}%
\pgfusepath{stroke}%
\end{pgfscope}%
\begin{pgfscope}%
\pgfpathrectangle{\pgfqpoint{0.100000in}{0.100000in}}{\pgfqpoint{5.307240in}{3.397500in}}%
\pgfusepath{clip}%
\pgfsetbuttcap%
\pgfsetroundjoin%
\definecolor{currentfill}{rgb}{0.678431,1.000000,0.184314}%
\pgfsetfillcolor{currentfill}%
\pgfsetfillopacity{0.500000}%
\pgfsetlinewidth{0.250937pt}%
\definecolor{currentstroke}{rgb}{0.000000,0.000000,0.000000}%
\pgfsetstrokecolor{currentstroke}%
\pgfsetstrokeopacity{0.500000}%
\pgfsetdash{}{0pt}%
\pgfsys@defobject{currentmarker}{\pgfqpoint{-0.035417in}{-0.035417in}}{\pgfqpoint{0.035417in}{0.035417in}}{%
\pgfpathmoveto{\pgfqpoint{0.000000in}{-0.035417in}}%
\pgfpathcurveto{\pgfqpoint{0.009393in}{-0.035417in}}{\pgfqpoint{0.018402in}{-0.031685in}}{\pgfqpoint{0.025043in}{-0.025043in}}%
\pgfpathcurveto{\pgfqpoint{0.031685in}{-0.018402in}}{\pgfqpoint{0.035417in}{-0.009393in}}{\pgfqpoint{0.035417in}{0.000000in}}%
\pgfpathcurveto{\pgfqpoint{0.035417in}{0.009393in}}{\pgfqpoint{0.031685in}{0.018402in}}{\pgfqpoint{0.025043in}{0.025043in}}%
\pgfpathcurveto{\pgfqpoint{0.018402in}{0.031685in}}{\pgfqpoint{0.009393in}{0.035417in}}{\pgfqpoint{0.000000in}{0.035417in}}%
\pgfpathcurveto{\pgfqpoint{-0.009393in}{0.035417in}}{\pgfqpoint{-0.018402in}{0.031685in}}{\pgfqpoint{-0.025043in}{0.025043in}}%
\pgfpathcurveto{\pgfqpoint{-0.031685in}{0.018402in}}{\pgfqpoint{-0.035417in}{0.009393in}}{\pgfqpoint{-0.035417in}{0.000000in}}%
\pgfpathcurveto{\pgfqpoint{-0.035417in}{-0.009393in}}{\pgfqpoint{-0.031685in}{-0.018402in}}{\pgfqpoint{-0.025043in}{-0.025043in}}%
\pgfpathcurveto{\pgfqpoint{-0.018402in}{-0.031685in}}{\pgfqpoint{-0.009393in}{-0.035417in}}{\pgfqpoint{0.000000in}{-0.035417in}}%
\pgfpathclose%
\pgfusepath{stroke,fill}%
}%
\begin{pgfscope}%
\pgfsys@transformshift{4.914466in}{2.423601in}%
\pgfsys@useobject{currentmarker}{}%
\end{pgfscope}%
\end{pgfscope}%
\begin{pgfscope}%
\pgfpathrectangle{\pgfqpoint{0.100000in}{0.100000in}}{\pgfqpoint{5.307240in}{3.397500in}}%
\pgfusepath{clip}%
\pgfsetrectcap%
\pgfsetroundjoin%
\pgfsetlinewidth{1.505625pt}%
\definecolor{currentstroke}{rgb}{0.678431,1.000000,0.184314}%
\pgfsetstrokecolor{currentstroke}%
\pgfsetstrokeopacity{0.500000}%
\pgfsetdash{}{0pt}%
\pgfpathmoveto{\pgfqpoint{4.969374in}{2.481295in}}%
\pgfusepath{stroke}%
\end{pgfscope}%
\begin{pgfscope}%
\pgfpathrectangle{\pgfqpoint{0.100000in}{0.100000in}}{\pgfqpoint{5.307240in}{3.397500in}}%
\pgfusepath{clip}%
\pgfsetbuttcap%
\pgfsetroundjoin%
\definecolor{currentfill}{rgb}{0.678431,1.000000,0.184314}%
\pgfsetfillcolor{currentfill}%
\pgfsetfillopacity{0.500000}%
\pgfsetlinewidth{0.250937pt}%
\definecolor{currentstroke}{rgb}{0.000000,0.000000,0.000000}%
\pgfsetstrokecolor{currentstroke}%
\pgfsetstrokeopacity{0.500000}%
\pgfsetdash{}{0pt}%
\pgfsys@defobject{currentmarker}{\pgfqpoint{-0.028472in}{-0.028472in}}{\pgfqpoint{0.028472in}{0.028472in}}{%
\pgfpathmoveto{\pgfqpoint{0.000000in}{-0.028472in}}%
\pgfpathcurveto{\pgfqpoint{0.007551in}{-0.028472in}}{\pgfqpoint{0.014794in}{-0.025472in}}{\pgfqpoint{0.020133in}{-0.020133in}}%
\pgfpathcurveto{\pgfqpoint{0.025472in}{-0.014794in}}{\pgfqpoint{0.028472in}{-0.007551in}}{\pgfqpoint{0.028472in}{0.000000in}}%
\pgfpathcurveto{\pgfqpoint{0.028472in}{0.007551in}}{\pgfqpoint{0.025472in}{0.014794in}}{\pgfqpoint{0.020133in}{0.020133in}}%
\pgfpathcurveto{\pgfqpoint{0.014794in}{0.025472in}}{\pgfqpoint{0.007551in}{0.028472in}}{\pgfqpoint{0.000000in}{0.028472in}}%
\pgfpathcurveto{\pgfqpoint{-0.007551in}{0.028472in}}{\pgfqpoint{-0.014794in}{0.025472in}}{\pgfqpoint{-0.020133in}{0.020133in}}%
\pgfpathcurveto{\pgfqpoint{-0.025472in}{0.014794in}}{\pgfqpoint{-0.028472in}{0.007551in}}{\pgfqpoint{-0.028472in}{0.000000in}}%
\pgfpathcurveto{\pgfqpoint{-0.028472in}{-0.007551in}}{\pgfqpoint{-0.025472in}{-0.014794in}}{\pgfqpoint{-0.020133in}{-0.020133in}}%
\pgfpathcurveto{\pgfqpoint{-0.014794in}{-0.025472in}}{\pgfqpoint{-0.007551in}{-0.028472in}}{\pgfqpoint{0.000000in}{-0.028472in}}%
\pgfpathclose%
\pgfusepath{stroke,fill}%
}%
\begin{pgfscope}%
\pgfsys@transformshift{4.969374in}{2.481295in}%
\pgfsys@useobject{currentmarker}{}%
\end{pgfscope}%
\end{pgfscope}%
\begin{pgfscope}%
\pgfpathrectangle{\pgfqpoint{0.100000in}{0.100000in}}{\pgfqpoint{5.307240in}{3.397500in}}%
\pgfusepath{clip}%
\pgfsetrectcap%
\pgfsetroundjoin%
\pgfsetlinewidth{1.505625pt}%
\definecolor{currentstroke}{rgb}{0.678431,1.000000,0.184314}%
\pgfsetstrokecolor{currentstroke}%
\pgfsetstrokeopacity{0.500000}%
\pgfsetdash{}{0pt}%
\pgfpathmoveto{\pgfqpoint{4.961733in}{2.424795in}}%
\pgfusepath{stroke}%
\end{pgfscope}%
\begin{pgfscope}%
\pgfpathrectangle{\pgfqpoint{0.100000in}{0.100000in}}{\pgfqpoint{5.307240in}{3.397500in}}%
\pgfusepath{clip}%
\pgfsetbuttcap%
\pgfsetroundjoin%
\definecolor{currentfill}{rgb}{0.678431,1.000000,0.184314}%
\pgfsetfillcolor{currentfill}%
\pgfsetfillopacity{0.500000}%
\pgfsetlinewidth{0.250937pt}%
\definecolor{currentstroke}{rgb}{0.000000,0.000000,0.000000}%
\pgfsetstrokecolor{currentstroke}%
\pgfsetstrokeopacity{0.500000}%
\pgfsetdash{}{0pt}%
\pgfsys@defobject{currentmarker}{\pgfqpoint{-0.027778in}{-0.027778in}}{\pgfqpoint{0.027778in}{0.027778in}}{%
\pgfpathmoveto{\pgfqpoint{0.000000in}{-0.027778in}}%
\pgfpathcurveto{\pgfqpoint{0.007367in}{-0.027778in}}{\pgfqpoint{0.014433in}{-0.024851in}}{\pgfqpoint{0.019642in}{-0.019642in}}%
\pgfpathcurveto{\pgfqpoint{0.024851in}{-0.014433in}}{\pgfqpoint{0.027778in}{-0.007367in}}{\pgfqpoint{0.027778in}{0.000000in}}%
\pgfpathcurveto{\pgfqpoint{0.027778in}{0.007367in}}{\pgfqpoint{0.024851in}{0.014433in}}{\pgfqpoint{0.019642in}{0.019642in}}%
\pgfpathcurveto{\pgfqpoint{0.014433in}{0.024851in}}{\pgfqpoint{0.007367in}{0.027778in}}{\pgfqpoint{0.000000in}{0.027778in}}%
\pgfpathcurveto{\pgfqpoint{-0.007367in}{0.027778in}}{\pgfqpoint{-0.014433in}{0.024851in}}{\pgfqpoint{-0.019642in}{0.019642in}}%
\pgfpathcurveto{\pgfqpoint{-0.024851in}{0.014433in}}{\pgfqpoint{-0.027778in}{0.007367in}}{\pgfqpoint{-0.027778in}{0.000000in}}%
\pgfpathcurveto{\pgfqpoint{-0.027778in}{-0.007367in}}{\pgfqpoint{-0.024851in}{-0.014433in}}{\pgfqpoint{-0.019642in}{-0.019642in}}%
\pgfpathcurveto{\pgfqpoint{-0.014433in}{-0.024851in}}{\pgfqpoint{-0.007367in}{-0.027778in}}{\pgfqpoint{0.000000in}{-0.027778in}}%
\pgfpathclose%
\pgfusepath{stroke,fill}%
}%
\begin{pgfscope}%
\pgfsys@transformshift{4.961733in}{2.424795in}%
\pgfsys@useobject{currentmarker}{}%
\end{pgfscope}%
\end{pgfscope}%
\begin{pgfscope}%
\pgfpathrectangle{\pgfqpoint{0.100000in}{0.100000in}}{\pgfqpoint{5.307240in}{3.397500in}}%
\pgfusepath{clip}%
\pgfsetrectcap%
\pgfsetroundjoin%
\pgfsetlinewidth{1.505625pt}%
\definecolor{currentstroke}{rgb}{0.678431,1.000000,0.184314}%
\pgfsetstrokecolor{currentstroke}%
\pgfsetstrokeopacity{0.500000}%
\pgfsetdash{}{0pt}%
\pgfpathmoveto{\pgfqpoint{5.027402in}{2.467322in}}%
\pgfusepath{stroke}%
\end{pgfscope}%
\begin{pgfscope}%
\pgfpathrectangle{\pgfqpoint{0.100000in}{0.100000in}}{\pgfqpoint{5.307240in}{3.397500in}}%
\pgfusepath{clip}%
\pgfsetbuttcap%
\pgfsetroundjoin%
\definecolor{currentfill}{rgb}{0.678431,1.000000,0.184314}%
\pgfsetfillcolor{currentfill}%
\pgfsetfillopacity{0.500000}%
\pgfsetlinewidth{0.250937pt}%
\definecolor{currentstroke}{rgb}{0.000000,0.000000,0.000000}%
\pgfsetstrokecolor{currentstroke}%
\pgfsetstrokeopacity{0.500000}%
\pgfsetdash{}{0pt}%
\pgfsys@defobject{currentmarker}{\pgfqpoint{-0.072917in}{-0.072917in}}{\pgfqpoint{0.072917in}{0.072917in}}{%
\pgfpathmoveto{\pgfqpoint{0.000000in}{-0.072917in}}%
\pgfpathcurveto{\pgfqpoint{0.019338in}{-0.072917in}}{\pgfqpoint{0.037886in}{-0.065234in}}{\pgfqpoint{0.051560in}{-0.051560in}}%
\pgfpathcurveto{\pgfqpoint{0.065234in}{-0.037886in}}{\pgfqpoint{0.072917in}{-0.019338in}}{\pgfqpoint{0.072917in}{0.000000in}}%
\pgfpathcurveto{\pgfqpoint{0.072917in}{0.019338in}}{\pgfqpoint{0.065234in}{0.037886in}}{\pgfqpoint{0.051560in}{0.051560in}}%
\pgfpathcurveto{\pgfqpoint{0.037886in}{0.065234in}}{\pgfqpoint{0.019338in}{0.072917in}}{\pgfqpoint{0.000000in}{0.072917in}}%
\pgfpathcurveto{\pgfqpoint{-0.019338in}{0.072917in}}{\pgfqpoint{-0.037886in}{0.065234in}}{\pgfqpoint{-0.051560in}{0.051560in}}%
\pgfpathcurveto{\pgfqpoint{-0.065234in}{0.037886in}}{\pgfqpoint{-0.072917in}{0.019338in}}{\pgfqpoint{-0.072917in}{0.000000in}}%
\pgfpathcurveto{\pgfqpoint{-0.072917in}{-0.019338in}}{\pgfqpoint{-0.065234in}{-0.037886in}}{\pgfqpoint{-0.051560in}{-0.051560in}}%
\pgfpathcurveto{\pgfqpoint{-0.037886in}{-0.065234in}}{\pgfqpoint{-0.019338in}{-0.072917in}}{\pgfqpoint{0.000000in}{-0.072917in}}%
\pgfpathclose%
\pgfusepath{stroke,fill}%
}%
\begin{pgfscope}%
\pgfsys@transformshift{5.027402in}{2.467322in}%
\pgfsys@useobject{currentmarker}{}%
\end{pgfscope}%
\end{pgfscope}%
\begin{pgfscope}%
\pgfpathrectangle{\pgfqpoint{0.100000in}{0.100000in}}{\pgfqpoint{5.307240in}{3.397500in}}%
\pgfusepath{clip}%
\pgfsetrectcap%
\pgfsetroundjoin%
\pgfsetlinewidth{1.505625pt}%
\definecolor{currentstroke}{rgb}{0.678431,1.000000,0.184314}%
\pgfsetstrokecolor{currentstroke}%
\pgfsetstrokeopacity{0.500000}%
\pgfsetdash{}{0pt}%
\pgfpathmoveto{\pgfqpoint{4.944050in}{2.450381in}}%
\pgfusepath{stroke}%
\end{pgfscope}%
\begin{pgfscope}%
\pgfpathrectangle{\pgfqpoint{0.100000in}{0.100000in}}{\pgfqpoint{5.307240in}{3.397500in}}%
\pgfusepath{clip}%
\pgfsetbuttcap%
\pgfsetroundjoin%
\definecolor{currentfill}{rgb}{0.678431,1.000000,0.184314}%
\pgfsetfillcolor{currentfill}%
\pgfsetfillopacity{0.500000}%
\pgfsetlinewidth{0.250937pt}%
\definecolor{currentstroke}{rgb}{0.000000,0.000000,0.000000}%
\pgfsetstrokecolor{currentstroke}%
\pgfsetstrokeopacity{0.500000}%
\pgfsetdash{}{0pt}%
\pgfsys@defobject{currentmarker}{\pgfqpoint{-0.033333in}{-0.033333in}}{\pgfqpoint{0.033333in}{0.033333in}}{%
\pgfpathmoveto{\pgfqpoint{0.000000in}{-0.033333in}}%
\pgfpathcurveto{\pgfqpoint{0.008840in}{-0.033333in}}{\pgfqpoint{0.017319in}{-0.029821in}}{\pgfqpoint{0.023570in}{-0.023570in}}%
\pgfpathcurveto{\pgfqpoint{0.029821in}{-0.017319in}}{\pgfqpoint{0.033333in}{-0.008840in}}{\pgfqpoint{0.033333in}{0.000000in}}%
\pgfpathcurveto{\pgfqpoint{0.033333in}{0.008840in}}{\pgfqpoint{0.029821in}{0.017319in}}{\pgfqpoint{0.023570in}{0.023570in}}%
\pgfpathcurveto{\pgfqpoint{0.017319in}{0.029821in}}{\pgfqpoint{0.008840in}{0.033333in}}{\pgfqpoint{0.000000in}{0.033333in}}%
\pgfpathcurveto{\pgfqpoint{-0.008840in}{0.033333in}}{\pgfqpoint{-0.017319in}{0.029821in}}{\pgfqpoint{-0.023570in}{0.023570in}}%
\pgfpathcurveto{\pgfqpoint{-0.029821in}{0.017319in}}{\pgfqpoint{-0.033333in}{0.008840in}}{\pgfqpoint{-0.033333in}{0.000000in}}%
\pgfpathcurveto{\pgfqpoint{-0.033333in}{-0.008840in}}{\pgfqpoint{-0.029821in}{-0.017319in}}{\pgfqpoint{-0.023570in}{-0.023570in}}%
\pgfpathcurveto{\pgfqpoint{-0.017319in}{-0.029821in}}{\pgfqpoint{-0.008840in}{-0.033333in}}{\pgfqpoint{0.000000in}{-0.033333in}}%
\pgfpathclose%
\pgfusepath{stroke,fill}%
}%
\begin{pgfscope}%
\pgfsys@transformshift{4.944050in}{2.450381in}%
\pgfsys@useobject{currentmarker}{}%
\end{pgfscope}%
\end{pgfscope}%
\begin{pgfscope}%
\pgfpathrectangle{\pgfqpoint{0.100000in}{0.100000in}}{\pgfqpoint{5.307240in}{3.397500in}}%
\pgfusepath{clip}%
\pgfsetrectcap%
\pgfsetroundjoin%
\pgfsetlinewidth{1.505625pt}%
\definecolor{currentstroke}{rgb}{0.678431,1.000000,0.184314}%
\pgfsetstrokecolor{currentstroke}%
\pgfsetstrokeopacity{0.500000}%
\pgfsetdash{}{0pt}%
\pgfpathmoveto{\pgfqpoint{4.793666in}{2.129031in}}%
\pgfusepath{stroke}%
\end{pgfscope}%
\begin{pgfscope}%
\pgfpathrectangle{\pgfqpoint{0.100000in}{0.100000in}}{\pgfqpoint{5.307240in}{3.397500in}}%
\pgfusepath{clip}%
\pgfsetbuttcap%
\pgfsetroundjoin%
\definecolor{currentfill}{rgb}{0.678431,1.000000,0.184314}%
\pgfsetfillcolor{currentfill}%
\pgfsetfillopacity{0.500000}%
\pgfsetlinewidth{0.250937pt}%
\definecolor{currentstroke}{rgb}{0.000000,0.000000,0.000000}%
\pgfsetstrokecolor{currentstroke}%
\pgfsetstrokeopacity{0.500000}%
\pgfsetdash{}{0pt}%
\pgfsys@defobject{currentmarker}{\pgfqpoint{-0.084028in}{-0.084028in}}{\pgfqpoint{0.084028in}{0.084028in}}{%
\pgfpathmoveto{\pgfqpoint{0.000000in}{-0.084028in}}%
\pgfpathcurveto{\pgfqpoint{0.022284in}{-0.084028in}}{\pgfqpoint{0.043659in}{-0.075174in}}{\pgfqpoint{0.059417in}{-0.059417in}}%
\pgfpathcurveto{\pgfqpoint{0.075174in}{-0.043659in}}{\pgfqpoint{0.084028in}{-0.022284in}}{\pgfqpoint{0.084028in}{0.000000in}}%
\pgfpathcurveto{\pgfqpoint{0.084028in}{0.022284in}}{\pgfqpoint{0.075174in}{0.043659in}}{\pgfqpoint{0.059417in}{0.059417in}}%
\pgfpathcurveto{\pgfqpoint{0.043659in}{0.075174in}}{\pgfqpoint{0.022284in}{0.084028in}}{\pgfqpoint{0.000000in}{0.084028in}}%
\pgfpathcurveto{\pgfqpoint{-0.022284in}{0.084028in}}{\pgfqpoint{-0.043659in}{0.075174in}}{\pgfqpoint{-0.059417in}{0.059417in}}%
\pgfpathcurveto{\pgfqpoint{-0.075174in}{0.043659in}}{\pgfqpoint{-0.084028in}{0.022284in}}{\pgfqpoint{-0.084028in}{0.000000in}}%
\pgfpathcurveto{\pgfqpoint{-0.084028in}{-0.022284in}}{\pgfqpoint{-0.075174in}{-0.043659in}}{\pgfqpoint{-0.059417in}{-0.059417in}}%
\pgfpathcurveto{\pgfqpoint{-0.043659in}{-0.075174in}}{\pgfqpoint{-0.022284in}{-0.084028in}}{\pgfqpoint{0.000000in}{-0.084028in}}%
\pgfpathclose%
\pgfusepath{stroke,fill}%
}%
\begin{pgfscope}%
\pgfsys@transformshift{4.793666in}{2.129031in}%
\pgfsys@useobject{currentmarker}{}%
\end{pgfscope}%
\end{pgfscope}%
\begin{pgfscope}%
\pgfpathrectangle{\pgfqpoint{0.100000in}{0.100000in}}{\pgfqpoint{5.307240in}{3.397500in}}%
\pgfusepath{clip}%
\pgfsetrectcap%
\pgfsetroundjoin%
\pgfsetlinewidth{1.505625pt}%
\definecolor{currentstroke}{rgb}{0.678431,1.000000,0.184314}%
\pgfsetstrokecolor{currentstroke}%
\pgfsetstrokeopacity{0.500000}%
\pgfsetdash{}{0pt}%
\pgfpathmoveto{\pgfqpoint{4.806430in}{2.037590in}}%
\pgfusepath{stroke}%
\end{pgfscope}%
\begin{pgfscope}%
\pgfpathrectangle{\pgfqpoint{0.100000in}{0.100000in}}{\pgfqpoint{5.307240in}{3.397500in}}%
\pgfusepath{clip}%
\pgfsetbuttcap%
\pgfsetroundjoin%
\definecolor{currentfill}{rgb}{0.678431,1.000000,0.184314}%
\pgfsetfillcolor{currentfill}%
\pgfsetfillopacity{0.500000}%
\pgfsetlinewidth{0.250937pt}%
\definecolor{currentstroke}{rgb}{0.000000,0.000000,0.000000}%
\pgfsetstrokecolor{currentstroke}%
\pgfsetstrokeopacity{0.500000}%
\pgfsetdash{}{0pt}%
\pgfsys@defobject{currentmarker}{\pgfqpoint{-0.079861in}{-0.079861in}}{\pgfqpoint{0.079861in}{0.079861in}}{%
\pgfpathmoveto{\pgfqpoint{0.000000in}{-0.079861in}}%
\pgfpathcurveto{\pgfqpoint{0.021179in}{-0.079861in}}{\pgfqpoint{0.041494in}{-0.071446in}}{\pgfqpoint{0.056470in}{-0.056470in}}%
\pgfpathcurveto{\pgfqpoint{0.071446in}{-0.041494in}}{\pgfqpoint{0.079861in}{-0.021179in}}{\pgfqpoint{0.079861in}{0.000000in}}%
\pgfpathcurveto{\pgfqpoint{0.079861in}{0.021179in}}{\pgfqpoint{0.071446in}{0.041494in}}{\pgfqpoint{0.056470in}{0.056470in}}%
\pgfpathcurveto{\pgfqpoint{0.041494in}{0.071446in}}{\pgfqpoint{0.021179in}{0.079861in}}{\pgfqpoint{0.000000in}{0.079861in}}%
\pgfpathcurveto{\pgfqpoint{-0.021179in}{0.079861in}}{\pgfqpoint{-0.041494in}{0.071446in}}{\pgfqpoint{-0.056470in}{0.056470in}}%
\pgfpathcurveto{\pgfqpoint{-0.071446in}{0.041494in}}{\pgfqpoint{-0.079861in}{0.021179in}}{\pgfqpoint{-0.079861in}{0.000000in}}%
\pgfpathcurveto{\pgfqpoint{-0.079861in}{-0.021179in}}{\pgfqpoint{-0.071446in}{-0.041494in}}{\pgfqpoint{-0.056470in}{-0.056470in}}%
\pgfpathcurveto{\pgfqpoint{-0.041494in}{-0.071446in}}{\pgfqpoint{-0.021179in}{-0.079861in}}{\pgfqpoint{0.000000in}{-0.079861in}}%
\pgfpathclose%
\pgfusepath{stroke,fill}%
}%
\begin{pgfscope}%
\pgfsys@transformshift{4.806430in}{2.037590in}%
\pgfsys@useobject{currentmarker}{}%
\end{pgfscope}%
\end{pgfscope}%
\begin{pgfscope}%
\pgfpathrectangle{\pgfqpoint{0.100000in}{0.100000in}}{\pgfqpoint{5.307240in}{3.397500in}}%
\pgfusepath{clip}%
\pgfsetrectcap%
\pgfsetroundjoin%
\pgfsetlinewidth{1.505625pt}%
\definecolor{currentstroke}{rgb}{0.678431,1.000000,0.184314}%
\pgfsetstrokecolor{currentstroke}%
\pgfsetstrokeopacity{0.500000}%
\pgfsetdash{}{0pt}%
\pgfpathmoveto{\pgfqpoint{4.666241in}{2.071147in}}%
\pgfusepath{stroke}%
\end{pgfscope}%
\begin{pgfscope}%
\pgfpathrectangle{\pgfqpoint{0.100000in}{0.100000in}}{\pgfqpoint{5.307240in}{3.397500in}}%
\pgfusepath{clip}%
\pgfsetbuttcap%
\pgfsetroundjoin%
\definecolor{currentfill}{rgb}{0.678431,1.000000,0.184314}%
\pgfsetfillcolor{currentfill}%
\pgfsetfillopacity{0.500000}%
\pgfsetlinewidth{0.250937pt}%
\definecolor{currentstroke}{rgb}{0.000000,0.000000,0.000000}%
\pgfsetstrokecolor{currentstroke}%
\pgfsetstrokeopacity{0.500000}%
\pgfsetdash{}{0pt}%
\pgfsys@defobject{currentmarker}{\pgfqpoint{-0.049306in}{-0.049306in}}{\pgfqpoint{0.049306in}{0.049306in}}{%
\pgfpathmoveto{\pgfqpoint{0.000000in}{-0.049306in}}%
\pgfpathcurveto{\pgfqpoint{0.013076in}{-0.049306in}}{\pgfqpoint{0.025618in}{-0.044110in}}{\pgfqpoint{0.034864in}{-0.034864in}}%
\pgfpathcurveto{\pgfqpoint{0.044110in}{-0.025618in}}{\pgfqpoint{0.049306in}{-0.013076in}}{\pgfqpoint{0.049306in}{0.000000in}}%
\pgfpathcurveto{\pgfqpoint{0.049306in}{0.013076in}}{\pgfqpoint{0.044110in}{0.025618in}}{\pgfqpoint{0.034864in}{0.034864in}}%
\pgfpathcurveto{\pgfqpoint{0.025618in}{0.044110in}}{\pgfqpoint{0.013076in}{0.049306in}}{\pgfqpoint{0.000000in}{0.049306in}}%
\pgfpathcurveto{\pgfqpoint{-0.013076in}{0.049306in}}{\pgfqpoint{-0.025618in}{0.044110in}}{\pgfqpoint{-0.034864in}{0.034864in}}%
\pgfpathcurveto{\pgfqpoint{-0.044110in}{0.025618in}}{\pgfqpoint{-0.049306in}{0.013076in}}{\pgfqpoint{-0.049306in}{0.000000in}}%
\pgfpathcurveto{\pgfqpoint{-0.049306in}{-0.013076in}}{\pgfqpoint{-0.044110in}{-0.025618in}}{\pgfqpoint{-0.034864in}{-0.034864in}}%
\pgfpathcurveto{\pgfqpoint{-0.025618in}{-0.044110in}}{\pgfqpoint{-0.013076in}{-0.049306in}}{\pgfqpoint{0.000000in}{-0.049306in}}%
\pgfpathclose%
\pgfusepath{stroke,fill}%
}%
\begin{pgfscope}%
\pgfsys@transformshift{4.666241in}{2.071147in}%
\pgfsys@useobject{currentmarker}{}%
\end{pgfscope}%
\end{pgfscope}%
\begin{pgfscope}%
\pgfpathrectangle{\pgfqpoint{0.100000in}{0.100000in}}{\pgfqpoint{5.307240in}{3.397500in}}%
\pgfusepath{clip}%
\pgfsetrectcap%
\pgfsetroundjoin%
\pgfsetlinewidth{1.505625pt}%
\definecolor{currentstroke}{rgb}{0.678431,1.000000,0.184314}%
\pgfsetstrokecolor{currentstroke}%
\pgfsetstrokeopacity{0.500000}%
\pgfsetdash{}{0pt}%
\pgfpathmoveto{\pgfqpoint{4.431006in}{0.570289in}}%
\pgfusepath{stroke}%
\end{pgfscope}%
\begin{pgfscope}%
\pgfpathrectangle{\pgfqpoint{0.100000in}{0.100000in}}{\pgfqpoint{5.307240in}{3.397500in}}%
\pgfusepath{clip}%
\pgfsetbuttcap%
\pgfsetroundjoin%
\definecolor{currentfill}{rgb}{0.678431,1.000000,0.184314}%
\pgfsetfillcolor{currentfill}%
\pgfsetfillopacity{0.500000}%
\pgfsetlinewidth{0.250937pt}%
\definecolor{currentstroke}{rgb}{0.000000,0.000000,0.000000}%
\pgfsetstrokecolor{currentstroke}%
\pgfsetstrokeopacity{0.500000}%
\pgfsetdash{}{0pt}%
\pgfsys@defobject{currentmarker}{\pgfqpoint{-0.081250in}{-0.081250in}}{\pgfqpoint{0.081250in}{0.081250in}}{%
\pgfpathmoveto{\pgfqpoint{0.000000in}{-0.081250in}}%
\pgfpathcurveto{\pgfqpoint{0.021548in}{-0.081250in}}{\pgfqpoint{0.042216in}{-0.072689in}}{\pgfqpoint{0.057452in}{-0.057452in}}%
\pgfpathcurveto{\pgfqpoint{0.072689in}{-0.042216in}}{\pgfqpoint{0.081250in}{-0.021548in}}{\pgfqpoint{0.081250in}{0.000000in}}%
\pgfpathcurveto{\pgfqpoint{0.081250in}{0.021548in}}{\pgfqpoint{0.072689in}{0.042216in}}{\pgfqpoint{0.057452in}{0.057452in}}%
\pgfpathcurveto{\pgfqpoint{0.042216in}{0.072689in}}{\pgfqpoint{0.021548in}{0.081250in}}{\pgfqpoint{0.000000in}{0.081250in}}%
\pgfpathcurveto{\pgfqpoint{-0.021548in}{0.081250in}}{\pgfqpoint{-0.042216in}{0.072689in}}{\pgfqpoint{-0.057452in}{0.057452in}}%
\pgfpathcurveto{\pgfqpoint{-0.072689in}{0.042216in}}{\pgfqpoint{-0.081250in}{0.021548in}}{\pgfqpoint{-0.081250in}{0.000000in}}%
\pgfpathcurveto{\pgfqpoint{-0.081250in}{-0.021548in}}{\pgfqpoint{-0.072689in}{-0.042216in}}{\pgfqpoint{-0.057452in}{-0.057452in}}%
\pgfpathcurveto{\pgfqpoint{-0.042216in}{-0.072689in}}{\pgfqpoint{-0.021548in}{-0.081250in}}{\pgfqpoint{0.000000in}{-0.081250in}}%
\pgfpathclose%
\pgfusepath{stroke,fill}%
}%
\begin{pgfscope}%
\pgfsys@transformshift{4.431006in}{0.570289in}%
\pgfsys@useobject{currentmarker}{}%
\end{pgfscope}%
\end{pgfscope}%
\begin{pgfscope}%
\pgfpathrectangle{\pgfqpoint{0.100000in}{0.100000in}}{\pgfqpoint{5.307240in}{3.397500in}}%
\pgfusepath{clip}%
\pgfsetrectcap%
\pgfsetroundjoin%
\pgfsetlinewidth{1.505625pt}%
\definecolor{currentstroke}{rgb}{0.678431,1.000000,0.184314}%
\pgfsetstrokecolor{currentstroke}%
\pgfsetstrokeopacity{0.500000}%
\pgfsetdash{}{0pt}%
\pgfpathmoveto{\pgfqpoint{3.900991in}{1.001699in}}%
\pgfusepath{stroke}%
\end{pgfscope}%
\begin{pgfscope}%
\pgfpathrectangle{\pgfqpoint{0.100000in}{0.100000in}}{\pgfqpoint{5.307240in}{3.397500in}}%
\pgfusepath{clip}%
\pgfsetbuttcap%
\pgfsetroundjoin%
\definecolor{currentfill}{rgb}{0.678431,1.000000,0.184314}%
\pgfsetfillcolor{currentfill}%
\pgfsetfillopacity{0.500000}%
\pgfsetlinewidth{0.250937pt}%
\definecolor{currentstroke}{rgb}{0.000000,0.000000,0.000000}%
\pgfsetstrokecolor{currentstroke}%
\pgfsetstrokeopacity{0.500000}%
\pgfsetdash{}{0pt}%
\pgfsys@defobject{currentmarker}{\pgfqpoint{-0.076389in}{-0.076389in}}{\pgfqpoint{0.076389in}{0.076389in}}{%
\pgfpathmoveto{\pgfqpoint{0.000000in}{-0.076389in}}%
\pgfpathcurveto{\pgfqpoint{0.020259in}{-0.076389in}}{\pgfqpoint{0.039690in}{-0.068340in}}{\pgfqpoint{0.054015in}{-0.054015in}}%
\pgfpathcurveto{\pgfqpoint{0.068340in}{-0.039690in}}{\pgfqpoint{0.076389in}{-0.020259in}}{\pgfqpoint{0.076389in}{0.000000in}}%
\pgfpathcurveto{\pgfqpoint{0.076389in}{0.020259in}}{\pgfqpoint{0.068340in}{0.039690in}}{\pgfqpoint{0.054015in}{0.054015in}}%
\pgfpathcurveto{\pgfqpoint{0.039690in}{0.068340in}}{\pgfqpoint{0.020259in}{0.076389in}}{\pgfqpoint{0.000000in}{0.076389in}}%
\pgfpathcurveto{\pgfqpoint{-0.020259in}{0.076389in}}{\pgfqpoint{-0.039690in}{0.068340in}}{\pgfqpoint{-0.054015in}{0.054015in}}%
\pgfpathcurveto{\pgfqpoint{-0.068340in}{0.039690in}}{\pgfqpoint{-0.076389in}{0.020259in}}{\pgfqpoint{-0.076389in}{0.000000in}}%
\pgfpathcurveto{\pgfqpoint{-0.076389in}{-0.020259in}}{\pgfqpoint{-0.068340in}{-0.039690in}}{\pgfqpoint{-0.054015in}{-0.054015in}}%
\pgfpathcurveto{\pgfqpoint{-0.039690in}{-0.068340in}}{\pgfqpoint{-0.020259in}{-0.076389in}}{\pgfqpoint{0.000000in}{-0.076389in}}%
\pgfpathclose%
\pgfusepath{stroke,fill}%
}%
\begin{pgfscope}%
\pgfsys@transformshift{3.900991in}{1.001699in}%
\pgfsys@useobject{currentmarker}{}%
\end{pgfscope}%
\end{pgfscope}%
\begin{pgfscope}%
\pgfpathrectangle{\pgfqpoint{0.100000in}{0.100000in}}{\pgfqpoint{5.307240in}{3.397500in}}%
\pgfusepath{clip}%
\pgfsetrectcap%
\pgfsetroundjoin%
\pgfsetlinewidth{1.505625pt}%
\definecolor{currentstroke}{rgb}{0.678431,1.000000,0.184314}%
\pgfsetstrokecolor{currentstroke}%
\pgfsetstrokeopacity{0.500000}%
\pgfsetdash{}{0pt}%
\pgfpathmoveto{\pgfqpoint{3.899990in}{0.959555in}}%
\pgfusepath{stroke}%
\end{pgfscope}%
\begin{pgfscope}%
\pgfpathrectangle{\pgfqpoint{0.100000in}{0.100000in}}{\pgfqpoint{5.307240in}{3.397500in}}%
\pgfusepath{clip}%
\pgfsetbuttcap%
\pgfsetroundjoin%
\definecolor{currentfill}{rgb}{0.678431,1.000000,0.184314}%
\pgfsetfillcolor{currentfill}%
\pgfsetfillopacity{0.500000}%
\pgfsetlinewidth{0.250937pt}%
\definecolor{currentstroke}{rgb}{0.000000,0.000000,0.000000}%
\pgfsetstrokecolor{currentstroke}%
\pgfsetstrokeopacity{0.500000}%
\pgfsetdash{}{0pt}%
\pgfsys@defobject{currentmarker}{\pgfqpoint{-0.076389in}{-0.076389in}}{\pgfqpoint{0.076389in}{0.076389in}}{%
\pgfpathmoveto{\pgfqpoint{0.000000in}{-0.076389in}}%
\pgfpathcurveto{\pgfqpoint{0.020259in}{-0.076389in}}{\pgfqpoint{0.039690in}{-0.068340in}}{\pgfqpoint{0.054015in}{-0.054015in}}%
\pgfpathcurveto{\pgfqpoint{0.068340in}{-0.039690in}}{\pgfqpoint{0.076389in}{-0.020259in}}{\pgfqpoint{0.076389in}{0.000000in}}%
\pgfpathcurveto{\pgfqpoint{0.076389in}{0.020259in}}{\pgfqpoint{0.068340in}{0.039690in}}{\pgfqpoint{0.054015in}{0.054015in}}%
\pgfpathcurveto{\pgfqpoint{0.039690in}{0.068340in}}{\pgfqpoint{0.020259in}{0.076389in}}{\pgfqpoint{0.000000in}{0.076389in}}%
\pgfpathcurveto{\pgfqpoint{-0.020259in}{0.076389in}}{\pgfqpoint{-0.039690in}{0.068340in}}{\pgfqpoint{-0.054015in}{0.054015in}}%
\pgfpathcurveto{\pgfqpoint{-0.068340in}{0.039690in}}{\pgfqpoint{-0.076389in}{0.020259in}}{\pgfqpoint{-0.076389in}{0.000000in}}%
\pgfpathcurveto{\pgfqpoint{-0.076389in}{-0.020259in}}{\pgfqpoint{-0.068340in}{-0.039690in}}{\pgfqpoint{-0.054015in}{-0.054015in}}%
\pgfpathcurveto{\pgfqpoint{-0.039690in}{-0.068340in}}{\pgfqpoint{-0.020259in}{-0.076389in}}{\pgfqpoint{0.000000in}{-0.076389in}}%
\pgfpathclose%
\pgfusepath{stroke,fill}%
}%
\begin{pgfscope}%
\pgfsys@transformshift{3.899990in}{0.959555in}%
\pgfsys@useobject{currentmarker}{}%
\end{pgfscope}%
\end{pgfscope}%
\begin{pgfscope}%
\pgfpathrectangle{\pgfqpoint{0.100000in}{0.100000in}}{\pgfqpoint{5.307240in}{3.397500in}}%
\pgfusepath{clip}%
\pgfsetrectcap%
\pgfsetroundjoin%
\pgfsetlinewidth{1.505625pt}%
\definecolor{currentstroke}{rgb}{0.678431,1.000000,0.184314}%
\pgfsetstrokecolor{currentstroke}%
\pgfsetstrokeopacity{0.500000}%
\pgfsetdash{}{0pt}%
\pgfpathmoveto{\pgfqpoint{4.465490in}{0.850185in}}%
\pgfusepath{stroke}%
\end{pgfscope}%
\begin{pgfscope}%
\pgfpathrectangle{\pgfqpoint{0.100000in}{0.100000in}}{\pgfqpoint{5.307240in}{3.397500in}}%
\pgfusepath{clip}%
\pgfsetbuttcap%
\pgfsetroundjoin%
\definecolor{currentfill}{rgb}{0.678431,1.000000,0.184314}%
\pgfsetfillcolor{currentfill}%
\pgfsetfillopacity{0.500000}%
\pgfsetlinewidth{0.250937pt}%
\definecolor{currentstroke}{rgb}{0.000000,0.000000,0.000000}%
\pgfsetstrokecolor{currentstroke}%
\pgfsetstrokeopacity{0.500000}%
\pgfsetdash{}{0pt}%
\pgfsys@defobject{currentmarker}{\pgfqpoint{-0.079167in}{-0.079167in}}{\pgfqpoint{0.079167in}{0.079167in}}{%
\pgfpathmoveto{\pgfqpoint{0.000000in}{-0.079167in}}%
\pgfpathcurveto{\pgfqpoint{0.020995in}{-0.079167in}}{\pgfqpoint{0.041133in}{-0.070825in}}{\pgfqpoint{0.055979in}{-0.055979in}}%
\pgfpathcurveto{\pgfqpoint{0.070825in}{-0.041133in}}{\pgfqpoint{0.079167in}{-0.020995in}}{\pgfqpoint{0.079167in}{0.000000in}}%
\pgfpathcurveto{\pgfqpoint{0.079167in}{0.020995in}}{\pgfqpoint{0.070825in}{0.041133in}}{\pgfqpoint{0.055979in}{0.055979in}}%
\pgfpathcurveto{\pgfqpoint{0.041133in}{0.070825in}}{\pgfqpoint{0.020995in}{0.079167in}}{\pgfqpoint{0.000000in}{0.079167in}}%
\pgfpathcurveto{\pgfqpoint{-0.020995in}{0.079167in}}{\pgfqpoint{-0.041133in}{0.070825in}}{\pgfqpoint{-0.055979in}{0.055979in}}%
\pgfpathcurveto{\pgfqpoint{-0.070825in}{0.041133in}}{\pgfqpoint{-0.079167in}{0.020995in}}{\pgfqpoint{-0.079167in}{0.000000in}}%
\pgfpathcurveto{\pgfqpoint{-0.079167in}{-0.020995in}}{\pgfqpoint{-0.070825in}{-0.041133in}}{\pgfqpoint{-0.055979in}{-0.055979in}}%
\pgfpathcurveto{\pgfqpoint{-0.041133in}{-0.070825in}}{\pgfqpoint{-0.020995in}{-0.079167in}}{\pgfqpoint{0.000000in}{-0.079167in}}%
\pgfpathclose%
\pgfusepath{stroke,fill}%
}%
\begin{pgfscope}%
\pgfsys@transformshift{4.465490in}{0.850185in}%
\pgfsys@useobject{currentmarker}{}%
\end{pgfscope}%
\end{pgfscope}%
\begin{pgfscope}%
\pgfpathrectangle{\pgfqpoint{0.100000in}{0.100000in}}{\pgfqpoint{5.307240in}{3.397500in}}%
\pgfusepath{clip}%
\pgfsetrectcap%
\pgfsetroundjoin%
\pgfsetlinewidth{1.505625pt}%
\definecolor{currentstroke}{rgb}{0.678431,1.000000,0.184314}%
\pgfsetstrokecolor{currentstroke}%
\pgfsetstrokeopacity{0.500000}%
\pgfsetdash{}{0pt}%
\pgfpathmoveto{\pgfqpoint{4.344665in}{0.922035in}}%
\pgfusepath{stroke}%
\end{pgfscope}%
\begin{pgfscope}%
\pgfpathrectangle{\pgfqpoint{0.100000in}{0.100000in}}{\pgfqpoint{5.307240in}{3.397500in}}%
\pgfusepath{clip}%
\pgfsetbuttcap%
\pgfsetroundjoin%
\definecolor{currentfill}{rgb}{0.678431,1.000000,0.184314}%
\pgfsetfillcolor{currentfill}%
\pgfsetfillopacity{0.500000}%
\pgfsetlinewidth{0.250937pt}%
\definecolor{currentstroke}{rgb}{0.000000,0.000000,0.000000}%
\pgfsetstrokecolor{currentstroke}%
\pgfsetstrokeopacity{0.500000}%
\pgfsetdash{}{0pt}%
\pgfsys@defobject{currentmarker}{\pgfqpoint{-0.043056in}{-0.043056in}}{\pgfqpoint{0.043056in}{0.043056in}}{%
\pgfpathmoveto{\pgfqpoint{0.000000in}{-0.043056in}}%
\pgfpathcurveto{\pgfqpoint{0.011418in}{-0.043056in}}{\pgfqpoint{0.022371in}{-0.038519in}}{\pgfqpoint{0.030445in}{-0.030445in}}%
\pgfpathcurveto{\pgfqpoint{0.038519in}{-0.022371in}}{\pgfqpoint{0.043056in}{-0.011418in}}{\pgfqpoint{0.043056in}{0.000000in}}%
\pgfpathcurveto{\pgfqpoint{0.043056in}{0.011418in}}{\pgfqpoint{0.038519in}{0.022371in}}{\pgfqpoint{0.030445in}{0.030445in}}%
\pgfpathcurveto{\pgfqpoint{0.022371in}{0.038519in}}{\pgfqpoint{0.011418in}{0.043056in}}{\pgfqpoint{0.000000in}{0.043056in}}%
\pgfpathcurveto{\pgfqpoint{-0.011418in}{0.043056in}}{\pgfqpoint{-0.022371in}{0.038519in}}{\pgfqpoint{-0.030445in}{0.030445in}}%
\pgfpathcurveto{\pgfqpoint{-0.038519in}{0.022371in}}{\pgfqpoint{-0.043056in}{0.011418in}}{\pgfqpoint{-0.043056in}{0.000000in}}%
\pgfpathcurveto{\pgfqpoint{-0.043056in}{-0.011418in}}{\pgfqpoint{-0.038519in}{-0.022371in}}{\pgfqpoint{-0.030445in}{-0.030445in}}%
\pgfpathcurveto{\pgfqpoint{-0.022371in}{-0.038519in}}{\pgfqpoint{-0.011418in}{-0.043056in}}{\pgfqpoint{0.000000in}{-0.043056in}}%
\pgfpathclose%
\pgfusepath{stroke,fill}%
}%
\begin{pgfscope}%
\pgfsys@transformshift{4.344665in}{0.922035in}%
\pgfsys@useobject{currentmarker}{}%
\end{pgfscope}%
\end{pgfscope}%
\begin{pgfscope}%
\pgfpathrectangle{\pgfqpoint{0.100000in}{0.100000in}}{\pgfqpoint{5.307240in}{3.397500in}}%
\pgfusepath{clip}%
\pgfsetrectcap%
\pgfsetroundjoin%
\pgfsetlinewidth{1.505625pt}%
\definecolor{currentstroke}{rgb}{0.678431,1.000000,0.184314}%
\pgfsetstrokecolor{currentstroke}%
\pgfsetstrokeopacity{0.500000}%
\pgfsetdash{}{0pt}%
\pgfpathmoveto{\pgfqpoint{4.332859in}{0.819398in}}%
\pgfusepath{stroke}%
\end{pgfscope}%
\begin{pgfscope}%
\pgfpathrectangle{\pgfqpoint{0.100000in}{0.100000in}}{\pgfqpoint{5.307240in}{3.397500in}}%
\pgfusepath{clip}%
\pgfsetbuttcap%
\pgfsetroundjoin%
\definecolor{currentfill}{rgb}{0.678431,1.000000,0.184314}%
\pgfsetfillcolor{currentfill}%
\pgfsetfillopacity{0.500000}%
\pgfsetlinewidth{0.250937pt}%
\definecolor{currentstroke}{rgb}{0.000000,0.000000,0.000000}%
\pgfsetstrokecolor{currentstroke}%
\pgfsetstrokeopacity{0.500000}%
\pgfsetdash{}{0pt}%
\pgfsys@defobject{currentmarker}{\pgfqpoint{-0.078472in}{-0.078472in}}{\pgfqpoint{0.078472in}{0.078472in}}{%
\pgfpathmoveto{\pgfqpoint{0.000000in}{-0.078472in}}%
\pgfpathcurveto{\pgfqpoint{0.020811in}{-0.078472in}}{\pgfqpoint{0.040773in}{-0.070204in}}{\pgfqpoint{0.055488in}{-0.055488in}}%
\pgfpathcurveto{\pgfqpoint{0.070204in}{-0.040773in}}{\pgfqpoint{0.078472in}{-0.020811in}}{\pgfqpoint{0.078472in}{0.000000in}}%
\pgfpathcurveto{\pgfqpoint{0.078472in}{0.020811in}}{\pgfqpoint{0.070204in}{0.040773in}}{\pgfqpoint{0.055488in}{0.055488in}}%
\pgfpathcurveto{\pgfqpoint{0.040773in}{0.070204in}}{\pgfqpoint{0.020811in}{0.078472in}}{\pgfqpoint{0.000000in}{0.078472in}}%
\pgfpathcurveto{\pgfqpoint{-0.020811in}{0.078472in}}{\pgfqpoint{-0.040773in}{0.070204in}}{\pgfqpoint{-0.055488in}{0.055488in}}%
\pgfpathcurveto{\pgfqpoint{-0.070204in}{0.040773in}}{\pgfqpoint{-0.078472in}{0.020811in}}{\pgfqpoint{-0.078472in}{0.000000in}}%
\pgfpathcurveto{\pgfqpoint{-0.078472in}{-0.020811in}}{\pgfqpoint{-0.070204in}{-0.040773in}}{\pgfqpoint{-0.055488in}{-0.055488in}}%
\pgfpathcurveto{\pgfqpoint{-0.040773in}{-0.070204in}}{\pgfqpoint{-0.020811in}{-0.078472in}}{\pgfqpoint{0.000000in}{-0.078472in}}%
\pgfpathclose%
\pgfusepath{stroke,fill}%
}%
\begin{pgfscope}%
\pgfsys@transformshift{4.332859in}{0.819398in}%
\pgfsys@useobject{currentmarker}{}%
\end{pgfscope}%
\end{pgfscope}%
\begin{pgfscope}%
\pgfpathrectangle{\pgfqpoint{0.100000in}{0.100000in}}{\pgfqpoint{5.307240in}{3.397500in}}%
\pgfusepath{clip}%
\pgfsetrectcap%
\pgfsetroundjoin%
\pgfsetlinewidth{1.505625pt}%
\definecolor{currentstroke}{rgb}{0.678431,1.000000,0.184314}%
\pgfsetstrokecolor{currentstroke}%
\pgfsetstrokeopacity{0.500000}%
\pgfsetdash{}{0pt}%
\pgfpathmoveto{\pgfqpoint{4.400763in}{1.010993in}}%
\pgfusepath{stroke}%
\end{pgfscope}%
\begin{pgfscope}%
\pgfpathrectangle{\pgfqpoint{0.100000in}{0.100000in}}{\pgfqpoint{5.307240in}{3.397500in}}%
\pgfusepath{clip}%
\pgfsetbuttcap%
\pgfsetroundjoin%
\definecolor{currentfill}{rgb}{0.678431,1.000000,0.184314}%
\pgfsetfillcolor{currentfill}%
\pgfsetfillopacity{0.500000}%
\pgfsetlinewidth{0.250937pt}%
\definecolor{currentstroke}{rgb}{0.000000,0.000000,0.000000}%
\pgfsetstrokecolor{currentstroke}%
\pgfsetstrokeopacity{0.500000}%
\pgfsetdash{}{0pt}%
\pgfsys@defobject{currentmarker}{\pgfqpoint{-0.057639in}{-0.057639in}}{\pgfqpoint{0.057639in}{0.057639in}}{%
\pgfpathmoveto{\pgfqpoint{0.000000in}{-0.057639in}}%
\pgfpathcurveto{\pgfqpoint{0.015286in}{-0.057639in}}{\pgfqpoint{0.029948in}{-0.051566in}}{\pgfqpoint{0.040757in}{-0.040757in}}%
\pgfpathcurveto{\pgfqpoint{0.051566in}{-0.029948in}}{\pgfqpoint{0.057639in}{-0.015286in}}{\pgfqpoint{0.057639in}{0.000000in}}%
\pgfpathcurveto{\pgfqpoint{0.057639in}{0.015286in}}{\pgfqpoint{0.051566in}{0.029948in}}{\pgfqpoint{0.040757in}{0.040757in}}%
\pgfpathcurveto{\pgfqpoint{0.029948in}{0.051566in}}{\pgfqpoint{0.015286in}{0.057639in}}{\pgfqpoint{0.000000in}{0.057639in}}%
\pgfpathcurveto{\pgfqpoint{-0.015286in}{0.057639in}}{\pgfqpoint{-0.029948in}{0.051566in}}{\pgfqpoint{-0.040757in}{0.040757in}}%
\pgfpathcurveto{\pgfqpoint{-0.051566in}{0.029948in}}{\pgfqpoint{-0.057639in}{0.015286in}}{\pgfqpoint{-0.057639in}{0.000000in}}%
\pgfpathcurveto{\pgfqpoint{-0.057639in}{-0.015286in}}{\pgfqpoint{-0.051566in}{-0.029948in}}{\pgfqpoint{-0.040757in}{-0.040757in}}%
\pgfpathcurveto{\pgfqpoint{-0.029948in}{-0.051566in}}{\pgfqpoint{-0.015286in}{-0.057639in}}{\pgfqpoint{0.000000in}{-0.057639in}}%
\pgfpathclose%
\pgfusepath{stroke,fill}%
}%
\begin{pgfscope}%
\pgfsys@transformshift{4.400763in}{1.010993in}%
\pgfsys@useobject{currentmarker}{}%
\end{pgfscope}%
\end{pgfscope}%
\begin{pgfscope}%
\pgfpathrectangle{\pgfqpoint{0.100000in}{0.100000in}}{\pgfqpoint{5.307240in}{3.397500in}}%
\pgfusepath{clip}%
\pgfsetrectcap%
\pgfsetroundjoin%
\pgfsetlinewidth{1.505625pt}%
\definecolor{currentstroke}{rgb}{0.678431,1.000000,0.184314}%
\pgfsetstrokecolor{currentstroke}%
\pgfsetstrokeopacity{0.500000}%
\pgfsetdash{}{0pt}%
\pgfpathmoveto{\pgfqpoint{4.409925in}{0.739057in}}%
\pgfusepath{stroke}%
\end{pgfscope}%
\begin{pgfscope}%
\pgfpathrectangle{\pgfqpoint{0.100000in}{0.100000in}}{\pgfqpoint{5.307240in}{3.397500in}}%
\pgfusepath{clip}%
\pgfsetbuttcap%
\pgfsetroundjoin%
\definecolor{currentfill}{rgb}{0.678431,1.000000,0.184314}%
\pgfsetfillcolor{currentfill}%
\pgfsetfillopacity{0.500000}%
\pgfsetlinewidth{0.250937pt}%
\definecolor{currentstroke}{rgb}{0.000000,0.000000,0.000000}%
\pgfsetstrokecolor{currentstroke}%
\pgfsetstrokeopacity{0.500000}%
\pgfsetdash{}{0pt}%
\pgfsys@defobject{currentmarker}{\pgfqpoint{-0.069444in}{-0.069444in}}{\pgfqpoint{0.069444in}{0.069444in}}{%
\pgfpathmoveto{\pgfqpoint{0.000000in}{-0.069444in}}%
\pgfpathcurveto{\pgfqpoint{0.018417in}{-0.069444in}}{\pgfqpoint{0.036082in}{-0.062127in}}{\pgfqpoint{0.049105in}{-0.049105in}}%
\pgfpathcurveto{\pgfqpoint{0.062127in}{-0.036082in}}{\pgfqpoint{0.069444in}{-0.018417in}}{\pgfqpoint{0.069444in}{0.000000in}}%
\pgfpathcurveto{\pgfqpoint{0.069444in}{0.018417in}}{\pgfqpoint{0.062127in}{0.036082in}}{\pgfqpoint{0.049105in}{0.049105in}}%
\pgfpathcurveto{\pgfqpoint{0.036082in}{0.062127in}}{\pgfqpoint{0.018417in}{0.069444in}}{\pgfqpoint{0.000000in}{0.069444in}}%
\pgfpathcurveto{\pgfqpoint{-0.018417in}{0.069444in}}{\pgfqpoint{-0.036082in}{0.062127in}}{\pgfqpoint{-0.049105in}{0.049105in}}%
\pgfpathcurveto{\pgfqpoint{-0.062127in}{0.036082in}}{\pgfqpoint{-0.069444in}{0.018417in}}{\pgfqpoint{-0.069444in}{0.000000in}}%
\pgfpathcurveto{\pgfqpoint{-0.069444in}{-0.018417in}}{\pgfqpoint{-0.062127in}{-0.036082in}}{\pgfqpoint{-0.049105in}{-0.049105in}}%
\pgfpathcurveto{\pgfqpoint{-0.036082in}{-0.062127in}}{\pgfqpoint{-0.018417in}{-0.069444in}}{\pgfqpoint{0.000000in}{-0.069444in}}%
\pgfpathclose%
\pgfusepath{stroke,fill}%
}%
\begin{pgfscope}%
\pgfsys@transformshift{4.409925in}{0.739057in}%
\pgfsys@useobject{currentmarker}{}%
\end{pgfscope}%
\end{pgfscope}%
\begin{pgfscope}%
\pgfpathrectangle{\pgfqpoint{0.100000in}{0.100000in}}{\pgfqpoint{5.307240in}{3.397500in}}%
\pgfusepath{clip}%
\pgfsetrectcap%
\pgfsetroundjoin%
\pgfsetlinewidth{1.505625pt}%
\definecolor{currentstroke}{rgb}{0.678431,1.000000,0.184314}%
\pgfsetstrokecolor{currentstroke}%
\pgfsetstrokeopacity{0.500000}%
\pgfsetdash{}{0pt}%
\pgfpathmoveto{\pgfqpoint{4.634848in}{0.501534in}}%
\pgfusepath{stroke}%
\end{pgfscope}%
\begin{pgfscope}%
\pgfpathrectangle{\pgfqpoint{0.100000in}{0.100000in}}{\pgfqpoint{5.307240in}{3.397500in}}%
\pgfusepath{clip}%
\pgfsetbuttcap%
\pgfsetroundjoin%
\definecolor{currentfill}{rgb}{0.678431,1.000000,0.184314}%
\pgfsetfillcolor{currentfill}%
\pgfsetfillopacity{0.500000}%
\pgfsetlinewidth{0.250937pt}%
\definecolor{currentstroke}{rgb}{0.000000,0.000000,0.000000}%
\pgfsetstrokecolor{currentstroke}%
\pgfsetstrokeopacity{0.500000}%
\pgfsetdash{}{0pt}%
\pgfsys@defobject{currentmarker}{\pgfqpoint{-0.072917in}{-0.072917in}}{\pgfqpoint{0.072917in}{0.072917in}}{%
\pgfpathmoveto{\pgfqpoint{0.000000in}{-0.072917in}}%
\pgfpathcurveto{\pgfqpoint{0.019338in}{-0.072917in}}{\pgfqpoint{0.037886in}{-0.065234in}}{\pgfqpoint{0.051560in}{-0.051560in}}%
\pgfpathcurveto{\pgfqpoint{0.065234in}{-0.037886in}}{\pgfqpoint{0.072917in}{-0.019338in}}{\pgfqpoint{0.072917in}{0.000000in}}%
\pgfpathcurveto{\pgfqpoint{0.072917in}{0.019338in}}{\pgfqpoint{0.065234in}{0.037886in}}{\pgfqpoint{0.051560in}{0.051560in}}%
\pgfpathcurveto{\pgfqpoint{0.037886in}{0.065234in}}{\pgfqpoint{0.019338in}{0.072917in}}{\pgfqpoint{0.000000in}{0.072917in}}%
\pgfpathcurveto{\pgfqpoint{-0.019338in}{0.072917in}}{\pgfqpoint{-0.037886in}{0.065234in}}{\pgfqpoint{-0.051560in}{0.051560in}}%
\pgfpathcurveto{\pgfqpoint{-0.065234in}{0.037886in}}{\pgfqpoint{-0.072917in}{0.019338in}}{\pgfqpoint{-0.072917in}{0.000000in}}%
\pgfpathcurveto{\pgfqpoint{-0.072917in}{-0.019338in}}{\pgfqpoint{-0.065234in}{-0.037886in}}{\pgfqpoint{-0.051560in}{-0.051560in}}%
\pgfpathcurveto{\pgfqpoint{-0.037886in}{-0.065234in}}{\pgfqpoint{-0.019338in}{-0.072917in}}{\pgfqpoint{0.000000in}{-0.072917in}}%
\pgfpathclose%
\pgfusepath{stroke,fill}%
}%
\begin{pgfscope}%
\pgfsys@transformshift{4.634848in}{0.501534in}%
\pgfsys@useobject{currentmarker}{}%
\end{pgfscope}%
\end{pgfscope}%
\begin{pgfscope}%
\pgfpathrectangle{\pgfqpoint{0.100000in}{0.100000in}}{\pgfqpoint{5.307240in}{3.397500in}}%
\pgfusepath{clip}%
\pgfsetrectcap%
\pgfsetroundjoin%
\pgfsetlinewidth{1.505625pt}%
\definecolor{currentstroke}{rgb}{0.678431,1.000000,0.184314}%
\pgfsetstrokecolor{currentstroke}%
\pgfsetstrokeopacity{0.500000}%
\pgfsetdash{}{0pt}%
\pgfpathmoveto{\pgfqpoint{4.458562in}{0.518508in}}%
\pgfusepath{stroke}%
\end{pgfscope}%
\begin{pgfscope}%
\pgfpathrectangle{\pgfqpoint{0.100000in}{0.100000in}}{\pgfqpoint{5.307240in}{3.397500in}}%
\pgfusepath{clip}%
\pgfsetbuttcap%
\pgfsetroundjoin%
\definecolor{currentfill}{rgb}{0.678431,1.000000,0.184314}%
\pgfsetfillcolor{currentfill}%
\pgfsetfillopacity{0.500000}%
\pgfsetlinewidth{0.250937pt}%
\definecolor{currentstroke}{rgb}{0.000000,0.000000,0.000000}%
\pgfsetstrokecolor{currentstroke}%
\pgfsetstrokeopacity{0.500000}%
\pgfsetdash{}{0pt}%
\pgfsys@defobject{currentmarker}{\pgfqpoint{-0.074306in}{-0.074306in}}{\pgfqpoint{0.074306in}{0.074306in}}{%
\pgfpathmoveto{\pgfqpoint{0.000000in}{-0.074306in}}%
\pgfpathcurveto{\pgfqpoint{0.019706in}{-0.074306in}}{\pgfqpoint{0.038608in}{-0.066476in}}{\pgfqpoint{0.052542in}{-0.052542in}}%
\pgfpathcurveto{\pgfqpoint{0.066476in}{-0.038608in}}{\pgfqpoint{0.074306in}{-0.019706in}}{\pgfqpoint{0.074306in}{0.000000in}}%
\pgfpathcurveto{\pgfqpoint{0.074306in}{0.019706in}}{\pgfqpoint{0.066476in}{0.038608in}}{\pgfqpoint{0.052542in}{0.052542in}}%
\pgfpathcurveto{\pgfqpoint{0.038608in}{0.066476in}}{\pgfqpoint{0.019706in}{0.074306in}}{\pgfqpoint{0.000000in}{0.074306in}}%
\pgfpathcurveto{\pgfqpoint{-0.019706in}{0.074306in}}{\pgfqpoint{-0.038608in}{0.066476in}}{\pgfqpoint{-0.052542in}{0.052542in}}%
\pgfpathcurveto{\pgfqpoint{-0.066476in}{0.038608in}}{\pgfqpoint{-0.074306in}{0.019706in}}{\pgfqpoint{-0.074306in}{0.000000in}}%
\pgfpathcurveto{\pgfqpoint{-0.074306in}{-0.019706in}}{\pgfqpoint{-0.066476in}{-0.038608in}}{\pgfqpoint{-0.052542in}{-0.052542in}}%
\pgfpathcurveto{\pgfqpoint{-0.038608in}{-0.066476in}}{\pgfqpoint{-0.019706in}{-0.074306in}}{\pgfqpoint{0.000000in}{-0.074306in}}%
\pgfpathclose%
\pgfusepath{stroke,fill}%
}%
\begin{pgfscope}%
\pgfsys@transformshift{4.458562in}{0.518508in}%
\pgfsys@useobject{currentmarker}{}%
\end{pgfscope}%
\end{pgfscope}%
\begin{pgfscope}%
\pgfpathrectangle{\pgfqpoint{0.100000in}{0.100000in}}{\pgfqpoint{5.307240in}{3.397500in}}%
\pgfusepath{clip}%
\pgfsetrectcap%
\pgfsetroundjoin%
\pgfsetlinewidth{1.505625pt}%
\definecolor{currentstroke}{rgb}{0.678431,1.000000,0.184314}%
\pgfsetstrokecolor{currentstroke}%
\pgfsetstrokeopacity{0.500000}%
\pgfsetdash{}{0pt}%
\pgfpathmoveto{\pgfqpoint{4.396856in}{0.618022in}}%
\pgfusepath{stroke}%
\end{pgfscope}%
\begin{pgfscope}%
\pgfpathrectangle{\pgfqpoint{0.100000in}{0.100000in}}{\pgfqpoint{5.307240in}{3.397500in}}%
\pgfusepath{clip}%
\pgfsetbuttcap%
\pgfsetroundjoin%
\definecolor{currentfill}{rgb}{0.678431,1.000000,0.184314}%
\pgfsetfillcolor{currentfill}%
\pgfsetfillopacity{0.500000}%
\pgfsetlinewidth{0.250937pt}%
\definecolor{currentstroke}{rgb}{0.000000,0.000000,0.000000}%
\pgfsetstrokecolor{currentstroke}%
\pgfsetstrokeopacity{0.500000}%
\pgfsetdash{}{0pt}%
\pgfsys@defobject{currentmarker}{\pgfqpoint{-0.078472in}{-0.078472in}}{\pgfqpoint{0.078472in}{0.078472in}}{%
\pgfpathmoveto{\pgfqpoint{0.000000in}{-0.078472in}}%
\pgfpathcurveto{\pgfqpoint{0.020811in}{-0.078472in}}{\pgfqpoint{0.040773in}{-0.070204in}}{\pgfqpoint{0.055488in}{-0.055488in}}%
\pgfpathcurveto{\pgfqpoint{0.070204in}{-0.040773in}}{\pgfqpoint{0.078472in}{-0.020811in}}{\pgfqpoint{0.078472in}{0.000000in}}%
\pgfpathcurveto{\pgfqpoint{0.078472in}{0.020811in}}{\pgfqpoint{0.070204in}{0.040773in}}{\pgfqpoint{0.055488in}{0.055488in}}%
\pgfpathcurveto{\pgfqpoint{0.040773in}{0.070204in}}{\pgfqpoint{0.020811in}{0.078472in}}{\pgfqpoint{0.000000in}{0.078472in}}%
\pgfpathcurveto{\pgfqpoint{-0.020811in}{0.078472in}}{\pgfqpoint{-0.040773in}{0.070204in}}{\pgfqpoint{-0.055488in}{0.055488in}}%
\pgfpathcurveto{\pgfqpoint{-0.070204in}{0.040773in}}{\pgfqpoint{-0.078472in}{0.020811in}}{\pgfqpoint{-0.078472in}{0.000000in}}%
\pgfpathcurveto{\pgfqpoint{-0.078472in}{-0.020811in}}{\pgfqpoint{-0.070204in}{-0.040773in}}{\pgfqpoint{-0.055488in}{-0.055488in}}%
\pgfpathcurveto{\pgfqpoint{-0.040773in}{-0.070204in}}{\pgfqpoint{-0.020811in}{-0.078472in}}{\pgfqpoint{0.000000in}{-0.078472in}}%
\pgfpathclose%
\pgfusepath{stroke,fill}%
}%
\begin{pgfscope}%
\pgfsys@transformshift{4.396856in}{0.618022in}%
\pgfsys@useobject{currentmarker}{}%
\end{pgfscope}%
\end{pgfscope}%
\begin{pgfscope}%
\pgfpathrectangle{\pgfqpoint{0.100000in}{0.100000in}}{\pgfqpoint{5.307240in}{3.397500in}}%
\pgfusepath{clip}%
\pgfsetrectcap%
\pgfsetroundjoin%
\pgfsetlinewidth{1.505625pt}%
\definecolor{currentstroke}{rgb}{0.678431,1.000000,0.184314}%
\pgfsetstrokecolor{currentstroke}%
\pgfsetstrokeopacity{0.500000}%
\pgfsetdash{}{0pt}%
\pgfpathmoveto{\pgfqpoint{4.371129in}{0.870461in}}%
\pgfusepath{stroke}%
\end{pgfscope}%
\begin{pgfscope}%
\pgfpathrectangle{\pgfqpoint{0.100000in}{0.100000in}}{\pgfqpoint{5.307240in}{3.397500in}}%
\pgfusepath{clip}%
\pgfsetbuttcap%
\pgfsetroundjoin%
\definecolor{currentfill}{rgb}{0.678431,1.000000,0.184314}%
\pgfsetfillcolor{currentfill}%
\pgfsetfillopacity{0.500000}%
\pgfsetlinewidth{0.250937pt}%
\definecolor{currentstroke}{rgb}{0.000000,0.000000,0.000000}%
\pgfsetstrokecolor{currentstroke}%
\pgfsetstrokeopacity{0.500000}%
\pgfsetdash{}{0pt}%
\pgfsys@defobject{currentmarker}{\pgfqpoint{-0.059028in}{-0.059028in}}{\pgfqpoint{0.059028in}{0.059028in}}{%
\pgfpathmoveto{\pgfqpoint{0.000000in}{-0.059028in}}%
\pgfpathcurveto{\pgfqpoint{0.015654in}{-0.059028in}}{\pgfqpoint{0.030670in}{-0.052808in}}{\pgfqpoint{0.041739in}{-0.041739in}}%
\pgfpathcurveto{\pgfqpoint{0.052808in}{-0.030670in}}{\pgfqpoint{0.059028in}{-0.015654in}}{\pgfqpoint{0.059028in}{0.000000in}}%
\pgfpathcurveto{\pgfqpoint{0.059028in}{0.015654in}}{\pgfqpoint{0.052808in}{0.030670in}}{\pgfqpoint{0.041739in}{0.041739in}}%
\pgfpathcurveto{\pgfqpoint{0.030670in}{0.052808in}}{\pgfqpoint{0.015654in}{0.059028in}}{\pgfqpoint{0.000000in}{0.059028in}}%
\pgfpathcurveto{\pgfqpoint{-0.015654in}{0.059028in}}{\pgfqpoint{-0.030670in}{0.052808in}}{\pgfqpoint{-0.041739in}{0.041739in}}%
\pgfpathcurveto{\pgfqpoint{-0.052808in}{0.030670in}}{\pgfqpoint{-0.059028in}{0.015654in}}{\pgfqpoint{-0.059028in}{0.000000in}}%
\pgfpathcurveto{\pgfqpoint{-0.059028in}{-0.015654in}}{\pgfqpoint{-0.052808in}{-0.030670in}}{\pgfqpoint{-0.041739in}{-0.041739in}}%
\pgfpathcurveto{\pgfqpoint{-0.030670in}{-0.052808in}}{\pgfqpoint{-0.015654in}{-0.059028in}}{\pgfqpoint{0.000000in}{-0.059028in}}%
\pgfpathclose%
\pgfusepath{stroke,fill}%
}%
\begin{pgfscope}%
\pgfsys@transformshift{4.371129in}{0.870461in}%
\pgfsys@useobject{currentmarker}{}%
\end{pgfscope}%
\end{pgfscope}%
\begin{pgfscope}%
\pgfpathrectangle{\pgfqpoint{0.100000in}{0.100000in}}{\pgfqpoint{5.307240in}{3.397500in}}%
\pgfusepath{clip}%
\pgfsetrectcap%
\pgfsetroundjoin%
\pgfsetlinewidth{1.505625pt}%
\definecolor{currentstroke}{rgb}{0.678431,1.000000,0.184314}%
\pgfsetstrokecolor{currentstroke}%
\pgfsetstrokeopacity{0.500000}%
\pgfsetdash{}{0pt}%
\pgfpathmoveto{\pgfqpoint{4.460024in}{0.806544in}}%
\pgfusepath{stroke}%
\end{pgfscope}%
\begin{pgfscope}%
\pgfpathrectangle{\pgfqpoint{0.100000in}{0.100000in}}{\pgfqpoint{5.307240in}{3.397500in}}%
\pgfusepath{clip}%
\pgfsetbuttcap%
\pgfsetroundjoin%
\definecolor{currentfill}{rgb}{0.678431,1.000000,0.184314}%
\pgfsetfillcolor{currentfill}%
\pgfsetfillopacity{0.500000}%
\pgfsetlinewidth{0.250937pt}%
\definecolor{currentstroke}{rgb}{0.000000,0.000000,0.000000}%
\pgfsetstrokecolor{currentstroke}%
\pgfsetstrokeopacity{0.500000}%
\pgfsetdash{}{0pt}%
\pgfsys@defobject{currentmarker}{\pgfqpoint{-0.093056in}{-0.093056in}}{\pgfqpoint{0.093056in}{0.093056in}}{%
\pgfpathmoveto{\pgfqpoint{0.000000in}{-0.093056in}}%
\pgfpathcurveto{\pgfqpoint{0.024679in}{-0.093056in}}{\pgfqpoint{0.048350in}{-0.083251in}}{\pgfqpoint{0.065800in}{-0.065800in}}%
\pgfpathcurveto{\pgfqpoint{0.083251in}{-0.048350in}}{\pgfqpoint{0.093056in}{-0.024679in}}{\pgfqpoint{0.093056in}{0.000000in}}%
\pgfpathcurveto{\pgfqpoint{0.093056in}{0.024679in}}{\pgfqpoint{0.083251in}{0.048350in}}{\pgfqpoint{0.065800in}{0.065800in}}%
\pgfpathcurveto{\pgfqpoint{0.048350in}{0.083251in}}{\pgfqpoint{0.024679in}{0.093056in}}{\pgfqpoint{0.000000in}{0.093056in}}%
\pgfpathcurveto{\pgfqpoint{-0.024679in}{0.093056in}}{\pgfqpoint{-0.048350in}{0.083251in}}{\pgfqpoint{-0.065800in}{0.065800in}}%
\pgfpathcurveto{\pgfqpoint{-0.083251in}{0.048350in}}{\pgfqpoint{-0.093056in}{0.024679in}}{\pgfqpoint{-0.093056in}{0.000000in}}%
\pgfpathcurveto{\pgfqpoint{-0.093056in}{-0.024679in}}{\pgfqpoint{-0.083251in}{-0.048350in}}{\pgfqpoint{-0.065800in}{-0.065800in}}%
\pgfpathcurveto{\pgfqpoint{-0.048350in}{-0.083251in}}{\pgfqpoint{-0.024679in}{-0.093056in}}{\pgfqpoint{0.000000in}{-0.093056in}}%
\pgfpathclose%
\pgfusepath{stroke,fill}%
}%
\begin{pgfscope}%
\pgfsys@transformshift{4.460024in}{0.806544in}%
\pgfsys@useobject{currentmarker}{}%
\end{pgfscope}%
\end{pgfscope}%
\begin{pgfscope}%
\pgfpathrectangle{\pgfqpoint{0.100000in}{0.100000in}}{\pgfqpoint{5.307240in}{3.397500in}}%
\pgfusepath{clip}%
\pgfsetrectcap%
\pgfsetroundjoin%
\pgfsetlinewidth{1.505625pt}%
\definecolor{currentstroke}{rgb}{0.678431,1.000000,0.184314}%
\pgfsetstrokecolor{currentstroke}%
\pgfsetstrokeopacity{0.500000}%
\pgfsetdash{}{0pt}%
\pgfpathmoveto{\pgfqpoint{4.544989in}{0.758829in}}%
\pgfusepath{stroke}%
\end{pgfscope}%
\begin{pgfscope}%
\pgfpathrectangle{\pgfqpoint{0.100000in}{0.100000in}}{\pgfqpoint{5.307240in}{3.397500in}}%
\pgfusepath{clip}%
\pgfsetbuttcap%
\pgfsetroundjoin%
\definecolor{currentfill}{rgb}{0.678431,1.000000,0.184314}%
\pgfsetfillcolor{currentfill}%
\pgfsetfillopacity{0.500000}%
\pgfsetlinewidth{0.250937pt}%
\definecolor{currentstroke}{rgb}{0.000000,0.000000,0.000000}%
\pgfsetstrokecolor{currentstroke}%
\pgfsetstrokeopacity{0.500000}%
\pgfsetdash{}{0pt}%
\pgfsys@defobject{currentmarker}{\pgfqpoint{-0.068750in}{-0.068750in}}{\pgfqpoint{0.068750in}{0.068750in}}{%
\pgfpathmoveto{\pgfqpoint{0.000000in}{-0.068750in}}%
\pgfpathcurveto{\pgfqpoint{0.018233in}{-0.068750in}}{\pgfqpoint{0.035721in}{-0.061506in}}{\pgfqpoint{0.048614in}{-0.048614in}}%
\pgfpathcurveto{\pgfqpoint{0.061506in}{-0.035721in}}{\pgfqpoint{0.068750in}{-0.018233in}}{\pgfqpoint{0.068750in}{0.000000in}}%
\pgfpathcurveto{\pgfqpoint{0.068750in}{0.018233in}}{\pgfqpoint{0.061506in}{0.035721in}}{\pgfqpoint{0.048614in}{0.048614in}}%
\pgfpathcurveto{\pgfqpoint{0.035721in}{0.061506in}}{\pgfqpoint{0.018233in}{0.068750in}}{\pgfqpoint{0.000000in}{0.068750in}}%
\pgfpathcurveto{\pgfqpoint{-0.018233in}{0.068750in}}{\pgfqpoint{-0.035721in}{0.061506in}}{\pgfqpoint{-0.048614in}{0.048614in}}%
\pgfpathcurveto{\pgfqpoint{-0.061506in}{0.035721in}}{\pgfqpoint{-0.068750in}{0.018233in}}{\pgfqpoint{-0.068750in}{0.000000in}}%
\pgfpathcurveto{\pgfqpoint{-0.068750in}{-0.018233in}}{\pgfqpoint{-0.061506in}{-0.035721in}}{\pgfqpoint{-0.048614in}{-0.048614in}}%
\pgfpathcurveto{\pgfqpoint{-0.035721in}{-0.061506in}}{\pgfqpoint{-0.018233in}{-0.068750in}}{\pgfqpoint{0.000000in}{-0.068750in}}%
\pgfpathclose%
\pgfusepath{stroke,fill}%
}%
\begin{pgfscope}%
\pgfsys@transformshift{4.544989in}{0.758829in}%
\pgfsys@useobject{currentmarker}{}%
\end{pgfscope}%
\end{pgfscope}%
\begin{pgfscope}%
\pgfpathrectangle{\pgfqpoint{0.100000in}{0.100000in}}{\pgfqpoint{5.307240in}{3.397500in}}%
\pgfusepath{clip}%
\pgfsetrectcap%
\pgfsetroundjoin%
\pgfsetlinewidth{1.505625pt}%
\definecolor{currentstroke}{rgb}{0.678431,1.000000,0.184314}%
\pgfsetstrokecolor{currentstroke}%
\pgfsetstrokeopacity{0.500000}%
\pgfsetdash{}{0pt}%
\pgfpathmoveto{\pgfqpoint{4.000246in}{0.940340in}}%
\pgfusepath{stroke}%
\end{pgfscope}%
\begin{pgfscope}%
\pgfpathrectangle{\pgfqpoint{0.100000in}{0.100000in}}{\pgfqpoint{5.307240in}{3.397500in}}%
\pgfusepath{clip}%
\pgfsetbuttcap%
\pgfsetroundjoin%
\definecolor{currentfill}{rgb}{0.678431,1.000000,0.184314}%
\pgfsetfillcolor{currentfill}%
\pgfsetfillopacity{0.500000}%
\pgfsetlinewidth{0.250937pt}%
\definecolor{currentstroke}{rgb}{0.000000,0.000000,0.000000}%
\pgfsetstrokecolor{currentstroke}%
\pgfsetstrokeopacity{0.500000}%
\pgfsetdash{}{0pt}%
\pgfsys@defobject{currentmarker}{\pgfqpoint{-0.060417in}{-0.060417in}}{\pgfqpoint{0.060417in}{0.060417in}}{%
\pgfpathmoveto{\pgfqpoint{0.000000in}{-0.060417in}}%
\pgfpathcurveto{\pgfqpoint{0.016023in}{-0.060417in}}{\pgfqpoint{0.031391in}{-0.054051in}}{\pgfqpoint{0.042721in}{-0.042721in}}%
\pgfpathcurveto{\pgfqpoint{0.054051in}{-0.031391in}}{\pgfqpoint{0.060417in}{-0.016023in}}{\pgfqpoint{0.060417in}{0.000000in}}%
\pgfpathcurveto{\pgfqpoint{0.060417in}{0.016023in}}{\pgfqpoint{0.054051in}{0.031391in}}{\pgfqpoint{0.042721in}{0.042721in}}%
\pgfpathcurveto{\pgfqpoint{0.031391in}{0.054051in}}{\pgfqpoint{0.016023in}{0.060417in}}{\pgfqpoint{0.000000in}{0.060417in}}%
\pgfpathcurveto{\pgfqpoint{-0.016023in}{0.060417in}}{\pgfqpoint{-0.031391in}{0.054051in}}{\pgfqpoint{-0.042721in}{0.042721in}}%
\pgfpathcurveto{\pgfqpoint{-0.054051in}{0.031391in}}{\pgfqpoint{-0.060417in}{0.016023in}}{\pgfqpoint{-0.060417in}{0.000000in}}%
\pgfpathcurveto{\pgfqpoint{-0.060417in}{-0.016023in}}{\pgfqpoint{-0.054051in}{-0.031391in}}{\pgfqpoint{-0.042721in}{-0.042721in}}%
\pgfpathcurveto{\pgfqpoint{-0.031391in}{-0.054051in}}{\pgfqpoint{-0.016023in}{-0.060417in}}{\pgfqpoint{0.000000in}{-0.060417in}}%
\pgfpathclose%
\pgfusepath{stroke,fill}%
}%
\begin{pgfscope}%
\pgfsys@transformshift{4.000246in}{0.940340in}%
\pgfsys@useobject{currentmarker}{}%
\end{pgfscope}%
\end{pgfscope}%
\begin{pgfscope}%
\pgfpathrectangle{\pgfqpoint{0.100000in}{0.100000in}}{\pgfqpoint{5.307240in}{3.397500in}}%
\pgfusepath{clip}%
\pgfsetrectcap%
\pgfsetroundjoin%
\pgfsetlinewidth{1.505625pt}%
\definecolor{currentstroke}{rgb}{0.678431,1.000000,0.184314}%
\pgfsetstrokecolor{currentstroke}%
\pgfsetstrokeopacity{0.500000}%
\pgfsetdash{}{0pt}%
\pgfpathmoveto{\pgfqpoint{3.839442in}{0.955991in}}%
\pgfusepath{stroke}%
\end{pgfscope}%
\begin{pgfscope}%
\pgfpathrectangle{\pgfqpoint{0.100000in}{0.100000in}}{\pgfqpoint{5.307240in}{3.397500in}}%
\pgfusepath{clip}%
\pgfsetbuttcap%
\pgfsetroundjoin%
\definecolor{currentfill}{rgb}{0.678431,1.000000,0.184314}%
\pgfsetfillcolor{currentfill}%
\pgfsetfillopacity{0.500000}%
\pgfsetlinewidth{0.250937pt}%
\definecolor{currentstroke}{rgb}{0.000000,0.000000,0.000000}%
\pgfsetstrokecolor{currentstroke}%
\pgfsetstrokeopacity{0.500000}%
\pgfsetdash{}{0pt}%
\pgfsys@defobject{currentmarker}{\pgfqpoint{-0.063194in}{-0.063194in}}{\pgfqpoint{0.063194in}{0.063194in}}{%
\pgfpathmoveto{\pgfqpoint{0.000000in}{-0.063194in}}%
\pgfpathcurveto{\pgfqpoint{0.016759in}{-0.063194in}}{\pgfqpoint{0.032835in}{-0.056536in}}{\pgfqpoint{0.044685in}{-0.044685in}}%
\pgfpathcurveto{\pgfqpoint{0.056536in}{-0.032835in}}{\pgfqpoint{0.063194in}{-0.016759in}}{\pgfqpoint{0.063194in}{0.000000in}}%
\pgfpathcurveto{\pgfqpoint{0.063194in}{0.016759in}}{\pgfqpoint{0.056536in}{0.032835in}}{\pgfqpoint{0.044685in}{0.044685in}}%
\pgfpathcurveto{\pgfqpoint{0.032835in}{0.056536in}}{\pgfqpoint{0.016759in}{0.063194in}}{\pgfqpoint{0.000000in}{0.063194in}}%
\pgfpathcurveto{\pgfqpoint{-0.016759in}{0.063194in}}{\pgfqpoint{-0.032835in}{0.056536in}}{\pgfqpoint{-0.044685in}{0.044685in}}%
\pgfpathcurveto{\pgfqpoint{-0.056536in}{0.032835in}}{\pgfqpoint{-0.063194in}{0.016759in}}{\pgfqpoint{-0.063194in}{0.000000in}}%
\pgfpathcurveto{\pgfqpoint{-0.063194in}{-0.016759in}}{\pgfqpoint{-0.056536in}{-0.032835in}}{\pgfqpoint{-0.044685in}{-0.044685in}}%
\pgfpathcurveto{\pgfqpoint{-0.032835in}{-0.056536in}}{\pgfqpoint{-0.016759in}{-0.063194in}}{\pgfqpoint{0.000000in}{-0.063194in}}%
\pgfpathclose%
\pgfusepath{stroke,fill}%
}%
\begin{pgfscope}%
\pgfsys@transformshift{3.839442in}{0.955991in}%
\pgfsys@useobject{currentmarker}{}%
\end{pgfscope}%
\end{pgfscope}%
\begin{pgfscope}%
\pgfpathrectangle{\pgfqpoint{0.100000in}{0.100000in}}{\pgfqpoint{5.307240in}{3.397500in}}%
\pgfusepath{clip}%
\pgfsetrectcap%
\pgfsetroundjoin%
\pgfsetlinewidth{1.505625pt}%
\definecolor{currentstroke}{rgb}{0.678431,1.000000,0.184314}%
\pgfsetstrokecolor{currentstroke}%
\pgfsetstrokeopacity{0.500000}%
\pgfsetdash{}{0pt}%
\pgfpathmoveto{\pgfqpoint{4.589228in}{0.677233in}}%
\pgfusepath{stroke}%
\end{pgfscope}%
\begin{pgfscope}%
\pgfpathrectangle{\pgfqpoint{0.100000in}{0.100000in}}{\pgfqpoint{5.307240in}{3.397500in}}%
\pgfusepath{clip}%
\pgfsetbuttcap%
\pgfsetroundjoin%
\definecolor{currentfill}{rgb}{0.678431,1.000000,0.184314}%
\pgfsetfillcolor{currentfill}%
\pgfsetfillopacity{0.500000}%
\pgfsetlinewidth{0.250937pt}%
\definecolor{currentstroke}{rgb}{0.000000,0.000000,0.000000}%
\pgfsetstrokecolor{currentstroke}%
\pgfsetstrokeopacity{0.500000}%
\pgfsetdash{}{0pt}%
\pgfsys@defobject{currentmarker}{\pgfqpoint{-0.070139in}{-0.070139in}}{\pgfqpoint{0.070139in}{0.070139in}}{%
\pgfpathmoveto{\pgfqpoint{0.000000in}{-0.070139in}}%
\pgfpathcurveto{\pgfqpoint{0.018601in}{-0.070139in}}{\pgfqpoint{0.036443in}{-0.062749in}}{\pgfqpoint{0.049596in}{-0.049596in}}%
\pgfpathcurveto{\pgfqpoint{0.062749in}{-0.036443in}}{\pgfqpoint{0.070139in}{-0.018601in}}{\pgfqpoint{0.070139in}{0.000000in}}%
\pgfpathcurveto{\pgfqpoint{0.070139in}{0.018601in}}{\pgfqpoint{0.062749in}{0.036443in}}{\pgfqpoint{0.049596in}{0.049596in}}%
\pgfpathcurveto{\pgfqpoint{0.036443in}{0.062749in}}{\pgfqpoint{0.018601in}{0.070139in}}{\pgfqpoint{0.000000in}{0.070139in}}%
\pgfpathcurveto{\pgfqpoint{-0.018601in}{0.070139in}}{\pgfqpoint{-0.036443in}{0.062749in}}{\pgfqpoint{-0.049596in}{0.049596in}}%
\pgfpathcurveto{\pgfqpoint{-0.062749in}{0.036443in}}{\pgfqpoint{-0.070139in}{0.018601in}}{\pgfqpoint{-0.070139in}{0.000000in}}%
\pgfpathcurveto{\pgfqpoint{-0.070139in}{-0.018601in}}{\pgfqpoint{-0.062749in}{-0.036443in}}{\pgfqpoint{-0.049596in}{-0.049596in}}%
\pgfpathcurveto{\pgfqpoint{-0.036443in}{-0.062749in}}{\pgfqpoint{-0.018601in}{-0.070139in}}{\pgfqpoint{0.000000in}{-0.070139in}}%
\pgfpathclose%
\pgfusepath{stroke,fill}%
}%
\begin{pgfscope}%
\pgfsys@transformshift{4.589228in}{0.677233in}%
\pgfsys@useobject{currentmarker}{}%
\end{pgfscope}%
\end{pgfscope}%
\begin{pgfscope}%
\pgfpathrectangle{\pgfqpoint{0.100000in}{0.100000in}}{\pgfqpoint{5.307240in}{3.397500in}}%
\pgfusepath{clip}%
\pgfsetrectcap%
\pgfsetroundjoin%
\pgfsetlinewidth{1.505625pt}%
\definecolor{currentstroke}{rgb}{0.678431,1.000000,0.184314}%
\pgfsetstrokecolor{currentstroke}%
\pgfsetstrokeopacity{0.500000}%
\pgfsetdash{}{0pt}%
\pgfpathmoveto{\pgfqpoint{4.418736in}{0.607411in}}%
\pgfusepath{stroke}%
\end{pgfscope}%
\begin{pgfscope}%
\pgfpathrectangle{\pgfqpoint{0.100000in}{0.100000in}}{\pgfqpoint{5.307240in}{3.397500in}}%
\pgfusepath{clip}%
\pgfsetbuttcap%
\pgfsetroundjoin%
\definecolor{currentfill}{rgb}{0.678431,1.000000,0.184314}%
\pgfsetfillcolor{currentfill}%
\pgfsetfillopacity{0.500000}%
\pgfsetlinewidth{0.250937pt}%
\definecolor{currentstroke}{rgb}{0.000000,0.000000,0.000000}%
\pgfsetstrokecolor{currentstroke}%
\pgfsetstrokeopacity{0.500000}%
\pgfsetdash{}{0pt}%
\pgfsys@defobject{currentmarker}{\pgfqpoint{-0.084722in}{-0.084722in}}{\pgfqpoint{0.084722in}{0.084722in}}{%
\pgfpathmoveto{\pgfqpoint{0.000000in}{-0.084722in}}%
\pgfpathcurveto{\pgfqpoint{0.022469in}{-0.084722in}}{\pgfqpoint{0.044020in}{-0.075795in}}{\pgfqpoint{0.059908in}{-0.059908in}}%
\pgfpathcurveto{\pgfqpoint{0.075795in}{-0.044020in}}{\pgfqpoint{0.084722in}{-0.022469in}}{\pgfqpoint{0.084722in}{0.000000in}}%
\pgfpathcurveto{\pgfqpoint{0.084722in}{0.022469in}}{\pgfqpoint{0.075795in}{0.044020in}}{\pgfqpoint{0.059908in}{0.059908in}}%
\pgfpathcurveto{\pgfqpoint{0.044020in}{0.075795in}}{\pgfqpoint{0.022469in}{0.084722in}}{\pgfqpoint{0.000000in}{0.084722in}}%
\pgfpathcurveto{\pgfqpoint{-0.022469in}{0.084722in}}{\pgfqpoint{-0.044020in}{0.075795in}}{\pgfqpoint{-0.059908in}{0.059908in}}%
\pgfpathcurveto{\pgfqpoint{-0.075795in}{0.044020in}}{\pgfqpoint{-0.084722in}{0.022469in}}{\pgfqpoint{-0.084722in}{0.000000in}}%
\pgfpathcurveto{\pgfqpoint{-0.084722in}{-0.022469in}}{\pgfqpoint{-0.075795in}{-0.044020in}}{\pgfqpoint{-0.059908in}{-0.059908in}}%
\pgfpathcurveto{\pgfqpoint{-0.044020in}{-0.075795in}}{\pgfqpoint{-0.022469in}{-0.084722in}}{\pgfqpoint{0.000000in}{-0.084722in}}%
\pgfpathclose%
\pgfusepath{stroke,fill}%
}%
\begin{pgfscope}%
\pgfsys@transformshift{4.418736in}{0.607411in}%
\pgfsys@useobject{currentmarker}{}%
\end{pgfscope}%
\end{pgfscope}%
\begin{pgfscope}%
\pgfpathrectangle{\pgfqpoint{0.100000in}{0.100000in}}{\pgfqpoint{5.307240in}{3.397500in}}%
\pgfusepath{clip}%
\pgfsetrectcap%
\pgfsetroundjoin%
\pgfsetlinewidth{1.505625pt}%
\definecolor{currentstroke}{rgb}{0.678431,1.000000,0.184314}%
\pgfsetstrokecolor{currentstroke}%
\pgfsetstrokeopacity{0.500000}%
\pgfsetdash{}{0pt}%
\pgfpathmoveto{\pgfqpoint{4.566852in}{0.736343in}}%
\pgfusepath{stroke}%
\end{pgfscope}%
\begin{pgfscope}%
\pgfpathrectangle{\pgfqpoint{0.100000in}{0.100000in}}{\pgfqpoint{5.307240in}{3.397500in}}%
\pgfusepath{clip}%
\pgfsetbuttcap%
\pgfsetroundjoin%
\definecolor{currentfill}{rgb}{0.678431,1.000000,0.184314}%
\pgfsetfillcolor{currentfill}%
\pgfsetfillopacity{0.500000}%
\pgfsetlinewidth{0.250937pt}%
\definecolor{currentstroke}{rgb}{0.000000,0.000000,0.000000}%
\pgfsetstrokecolor{currentstroke}%
\pgfsetstrokeopacity{0.500000}%
\pgfsetdash{}{0pt}%
\pgfsys@defobject{currentmarker}{\pgfqpoint{-0.073611in}{-0.073611in}}{\pgfqpoint{0.073611in}{0.073611in}}{%
\pgfpathmoveto{\pgfqpoint{0.000000in}{-0.073611in}}%
\pgfpathcurveto{\pgfqpoint{0.019522in}{-0.073611in}}{\pgfqpoint{0.038247in}{-0.065855in}}{\pgfqpoint{0.052051in}{-0.052051in}}%
\pgfpathcurveto{\pgfqpoint{0.065855in}{-0.038247in}}{\pgfqpoint{0.073611in}{-0.019522in}}{\pgfqpoint{0.073611in}{0.000000in}}%
\pgfpathcurveto{\pgfqpoint{0.073611in}{0.019522in}}{\pgfqpoint{0.065855in}{0.038247in}}{\pgfqpoint{0.052051in}{0.052051in}}%
\pgfpathcurveto{\pgfqpoint{0.038247in}{0.065855in}}{\pgfqpoint{0.019522in}{0.073611in}}{\pgfqpoint{0.000000in}{0.073611in}}%
\pgfpathcurveto{\pgfqpoint{-0.019522in}{0.073611in}}{\pgfqpoint{-0.038247in}{0.065855in}}{\pgfqpoint{-0.052051in}{0.052051in}}%
\pgfpathcurveto{\pgfqpoint{-0.065855in}{0.038247in}}{\pgfqpoint{-0.073611in}{0.019522in}}{\pgfqpoint{-0.073611in}{0.000000in}}%
\pgfpathcurveto{\pgfqpoint{-0.073611in}{-0.019522in}}{\pgfqpoint{-0.065855in}{-0.038247in}}{\pgfqpoint{-0.052051in}{-0.052051in}}%
\pgfpathcurveto{\pgfqpoint{-0.038247in}{-0.065855in}}{\pgfqpoint{-0.019522in}{-0.073611in}}{\pgfqpoint{0.000000in}{-0.073611in}}%
\pgfpathclose%
\pgfusepath{stroke,fill}%
}%
\begin{pgfscope}%
\pgfsys@transformshift{4.566852in}{0.736343in}%
\pgfsys@useobject{currentmarker}{}%
\end{pgfscope}%
\end{pgfscope}%
\begin{pgfscope}%
\pgfpathrectangle{\pgfqpoint{0.100000in}{0.100000in}}{\pgfqpoint{5.307240in}{3.397500in}}%
\pgfusepath{clip}%
\pgfsetrectcap%
\pgfsetroundjoin%
\pgfsetlinewidth{1.505625pt}%
\definecolor{currentstroke}{rgb}{0.678431,1.000000,0.184314}%
\pgfsetstrokecolor{currentstroke}%
\pgfsetstrokeopacity{0.500000}%
\pgfsetdash{}{0pt}%
\pgfpathmoveto{\pgfqpoint{4.472069in}{0.683125in}}%
\pgfusepath{stroke}%
\end{pgfscope}%
\begin{pgfscope}%
\pgfpathrectangle{\pgfqpoint{0.100000in}{0.100000in}}{\pgfqpoint{5.307240in}{3.397500in}}%
\pgfusepath{clip}%
\pgfsetbuttcap%
\pgfsetroundjoin%
\definecolor{currentfill}{rgb}{0.678431,1.000000,0.184314}%
\pgfsetfillcolor{currentfill}%
\pgfsetfillopacity{0.500000}%
\pgfsetlinewidth{0.250937pt}%
\definecolor{currentstroke}{rgb}{0.000000,0.000000,0.000000}%
\pgfsetstrokecolor{currentstroke}%
\pgfsetstrokeopacity{0.500000}%
\pgfsetdash{}{0pt}%
\pgfsys@defobject{currentmarker}{\pgfqpoint{-0.050694in}{-0.050694in}}{\pgfqpoint{0.050694in}{0.050694in}}{%
\pgfpathmoveto{\pgfqpoint{0.000000in}{-0.050694in}}%
\pgfpathcurveto{\pgfqpoint{0.013444in}{-0.050694in}}{\pgfqpoint{0.026340in}{-0.045353in}}{\pgfqpoint{0.035846in}{-0.035846in}}%
\pgfpathcurveto{\pgfqpoint{0.045353in}{-0.026340in}}{\pgfqpoint{0.050694in}{-0.013444in}}{\pgfqpoint{0.050694in}{0.000000in}}%
\pgfpathcurveto{\pgfqpoint{0.050694in}{0.013444in}}{\pgfqpoint{0.045353in}{0.026340in}}{\pgfqpoint{0.035846in}{0.035846in}}%
\pgfpathcurveto{\pgfqpoint{0.026340in}{0.045353in}}{\pgfqpoint{0.013444in}{0.050694in}}{\pgfqpoint{0.000000in}{0.050694in}}%
\pgfpathcurveto{\pgfqpoint{-0.013444in}{0.050694in}}{\pgfqpoint{-0.026340in}{0.045353in}}{\pgfqpoint{-0.035846in}{0.035846in}}%
\pgfpathcurveto{\pgfqpoint{-0.045353in}{0.026340in}}{\pgfqpoint{-0.050694in}{0.013444in}}{\pgfqpoint{-0.050694in}{0.000000in}}%
\pgfpathcurveto{\pgfqpoint{-0.050694in}{-0.013444in}}{\pgfqpoint{-0.045353in}{-0.026340in}}{\pgfqpoint{-0.035846in}{-0.035846in}}%
\pgfpathcurveto{\pgfqpoint{-0.026340in}{-0.045353in}}{\pgfqpoint{-0.013444in}{-0.050694in}}{\pgfqpoint{0.000000in}{-0.050694in}}%
\pgfpathclose%
\pgfusepath{stroke,fill}%
}%
\begin{pgfscope}%
\pgfsys@transformshift{4.472069in}{0.683125in}%
\pgfsys@useobject{currentmarker}{}%
\end{pgfscope}%
\end{pgfscope}%
\begin{pgfscope}%
\pgfpathrectangle{\pgfqpoint{0.100000in}{0.100000in}}{\pgfqpoint{5.307240in}{3.397500in}}%
\pgfusepath{clip}%
\pgfsetrectcap%
\pgfsetroundjoin%
\pgfsetlinewidth{1.505625pt}%
\definecolor{currentstroke}{rgb}{0.678431,1.000000,0.184314}%
\pgfsetstrokecolor{currentstroke}%
\pgfsetstrokeopacity{0.500000}%
\pgfsetdash{}{0pt}%
\pgfpathmoveto{\pgfqpoint{4.135242in}{0.988189in}}%
\pgfusepath{stroke}%
\end{pgfscope}%
\begin{pgfscope}%
\pgfpathrectangle{\pgfqpoint{0.100000in}{0.100000in}}{\pgfqpoint{5.307240in}{3.397500in}}%
\pgfusepath{clip}%
\pgfsetbuttcap%
\pgfsetroundjoin%
\definecolor{currentfill}{rgb}{0.678431,1.000000,0.184314}%
\pgfsetfillcolor{currentfill}%
\pgfsetfillopacity{0.500000}%
\pgfsetlinewidth{0.250937pt}%
\definecolor{currentstroke}{rgb}{0.000000,0.000000,0.000000}%
\pgfsetstrokecolor{currentstroke}%
\pgfsetstrokeopacity{0.500000}%
\pgfsetdash{}{0pt}%
\pgfsys@defobject{currentmarker}{\pgfqpoint{-0.036111in}{-0.036111in}}{\pgfqpoint{0.036111in}{0.036111in}}{%
\pgfpathmoveto{\pgfqpoint{0.000000in}{-0.036111in}}%
\pgfpathcurveto{\pgfqpoint{0.009577in}{-0.036111in}}{\pgfqpoint{0.018763in}{-0.032306in}}{\pgfqpoint{0.025534in}{-0.025534in}}%
\pgfpathcurveto{\pgfqpoint{0.032306in}{-0.018763in}}{\pgfqpoint{0.036111in}{-0.009577in}}{\pgfqpoint{0.036111in}{0.000000in}}%
\pgfpathcurveto{\pgfqpoint{0.036111in}{0.009577in}}{\pgfqpoint{0.032306in}{0.018763in}}{\pgfqpoint{0.025534in}{0.025534in}}%
\pgfpathcurveto{\pgfqpoint{0.018763in}{0.032306in}}{\pgfqpoint{0.009577in}{0.036111in}}{\pgfqpoint{0.000000in}{0.036111in}}%
\pgfpathcurveto{\pgfqpoint{-0.009577in}{0.036111in}}{\pgfqpoint{-0.018763in}{0.032306in}}{\pgfqpoint{-0.025534in}{0.025534in}}%
\pgfpathcurveto{\pgfqpoint{-0.032306in}{0.018763in}}{\pgfqpoint{-0.036111in}{0.009577in}}{\pgfqpoint{-0.036111in}{0.000000in}}%
\pgfpathcurveto{\pgfqpoint{-0.036111in}{-0.009577in}}{\pgfqpoint{-0.032306in}{-0.018763in}}{\pgfqpoint{-0.025534in}{-0.025534in}}%
\pgfpathcurveto{\pgfqpoint{-0.018763in}{-0.032306in}}{\pgfqpoint{-0.009577in}{-0.036111in}}{\pgfqpoint{0.000000in}{-0.036111in}}%
\pgfpathclose%
\pgfusepath{stroke,fill}%
}%
\begin{pgfscope}%
\pgfsys@transformshift{4.135242in}{0.988189in}%
\pgfsys@useobject{currentmarker}{}%
\end{pgfscope}%
\end{pgfscope}%
\begin{pgfscope}%
\pgfpathrectangle{\pgfqpoint{0.100000in}{0.100000in}}{\pgfqpoint{5.307240in}{3.397500in}}%
\pgfusepath{clip}%
\pgfsetrectcap%
\pgfsetroundjoin%
\pgfsetlinewidth{1.505625pt}%
\definecolor{currentstroke}{rgb}{0.678431,1.000000,0.184314}%
\pgfsetstrokecolor{currentstroke}%
\pgfsetstrokeopacity{0.500000}%
\pgfsetdash{}{0pt}%
\pgfpathmoveto{\pgfqpoint{4.358817in}{0.720851in}}%
\pgfusepath{stroke}%
\end{pgfscope}%
\begin{pgfscope}%
\pgfpathrectangle{\pgfqpoint{0.100000in}{0.100000in}}{\pgfqpoint{5.307240in}{3.397500in}}%
\pgfusepath{clip}%
\pgfsetbuttcap%
\pgfsetroundjoin%
\definecolor{currentfill}{rgb}{0.678431,1.000000,0.184314}%
\pgfsetfillcolor{currentfill}%
\pgfsetfillopacity{0.500000}%
\pgfsetlinewidth{0.250937pt}%
\definecolor{currentstroke}{rgb}{0.000000,0.000000,0.000000}%
\pgfsetstrokecolor{currentstroke}%
\pgfsetstrokeopacity{0.500000}%
\pgfsetdash{}{0pt}%
\pgfsys@defobject{currentmarker}{\pgfqpoint{-0.070139in}{-0.070139in}}{\pgfqpoint{0.070139in}{0.070139in}}{%
\pgfpathmoveto{\pgfqpoint{0.000000in}{-0.070139in}}%
\pgfpathcurveto{\pgfqpoint{0.018601in}{-0.070139in}}{\pgfqpoint{0.036443in}{-0.062749in}}{\pgfqpoint{0.049596in}{-0.049596in}}%
\pgfpathcurveto{\pgfqpoint{0.062749in}{-0.036443in}}{\pgfqpoint{0.070139in}{-0.018601in}}{\pgfqpoint{0.070139in}{0.000000in}}%
\pgfpathcurveto{\pgfqpoint{0.070139in}{0.018601in}}{\pgfqpoint{0.062749in}{0.036443in}}{\pgfqpoint{0.049596in}{0.049596in}}%
\pgfpathcurveto{\pgfqpoint{0.036443in}{0.062749in}}{\pgfqpoint{0.018601in}{0.070139in}}{\pgfqpoint{0.000000in}{0.070139in}}%
\pgfpathcurveto{\pgfqpoint{-0.018601in}{0.070139in}}{\pgfqpoint{-0.036443in}{0.062749in}}{\pgfqpoint{-0.049596in}{0.049596in}}%
\pgfpathcurveto{\pgfqpoint{-0.062749in}{0.036443in}}{\pgfqpoint{-0.070139in}{0.018601in}}{\pgfqpoint{-0.070139in}{0.000000in}}%
\pgfpathcurveto{\pgfqpoint{-0.070139in}{-0.018601in}}{\pgfqpoint{-0.062749in}{-0.036443in}}{\pgfqpoint{-0.049596in}{-0.049596in}}%
\pgfpathcurveto{\pgfqpoint{-0.036443in}{-0.062749in}}{\pgfqpoint{-0.018601in}{-0.070139in}}{\pgfqpoint{0.000000in}{-0.070139in}}%
\pgfpathclose%
\pgfusepath{stroke,fill}%
}%
\begin{pgfscope}%
\pgfsys@transformshift{4.358817in}{0.720851in}%
\pgfsys@useobject{currentmarker}{}%
\end{pgfscope}%
\end{pgfscope}%
\begin{pgfscope}%
\pgfpathrectangle{\pgfqpoint{0.100000in}{0.100000in}}{\pgfqpoint{5.307240in}{3.397500in}}%
\pgfusepath{clip}%
\pgfsetrectcap%
\pgfsetroundjoin%
\pgfsetlinewidth{1.505625pt}%
\definecolor{currentstroke}{rgb}{0.678431,1.000000,0.184314}%
\pgfsetstrokecolor{currentstroke}%
\pgfsetstrokeopacity{0.500000}%
\pgfsetdash{}{0pt}%
\pgfpathmoveto{\pgfqpoint{4.394218in}{0.845130in}}%
\pgfusepath{stroke}%
\end{pgfscope}%
\begin{pgfscope}%
\pgfpathrectangle{\pgfqpoint{0.100000in}{0.100000in}}{\pgfqpoint{5.307240in}{3.397500in}}%
\pgfusepath{clip}%
\pgfsetbuttcap%
\pgfsetroundjoin%
\definecolor{currentfill}{rgb}{0.678431,1.000000,0.184314}%
\pgfsetfillcolor{currentfill}%
\pgfsetfillopacity{0.500000}%
\pgfsetlinewidth{0.250937pt}%
\definecolor{currentstroke}{rgb}{0.000000,0.000000,0.000000}%
\pgfsetstrokecolor{currentstroke}%
\pgfsetstrokeopacity{0.500000}%
\pgfsetdash{}{0pt}%
\pgfsys@defobject{currentmarker}{\pgfqpoint{-0.054861in}{-0.054861in}}{\pgfqpoint{0.054861in}{0.054861in}}{%
\pgfpathmoveto{\pgfqpoint{0.000000in}{-0.054861in}}%
\pgfpathcurveto{\pgfqpoint{0.014549in}{-0.054861in}}{\pgfqpoint{0.028505in}{-0.049081in}}{\pgfqpoint{0.038793in}{-0.038793in}}%
\pgfpathcurveto{\pgfqpoint{0.049081in}{-0.028505in}}{\pgfqpoint{0.054861in}{-0.014549in}}{\pgfqpoint{0.054861in}{0.000000in}}%
\pgfpathcurveto{\pgfqpoint{0.054861in}{0.014549in}}{\pgfqpoint{0.049081in}{0.028505in}}{\pgfqpoint{0.038793in}{0.038793in}}%
\pgfpathcurveto{\pgfqpoint{0.028505in}{0.049081in}}{\pgfqpoint{0.014549in}{0.054861in}}{\pgfqpoint{0.000000in}{0.054861in}}%
\pgfpathcurveto{\pgfqpoint{-0.014549in}{0.054861in}}{\pgfqpoint{-0.028505in}{0.049081in}}{\pgfqpoint{-0.038793in}{0.038793in}}%
\pgfpathcurveto{\pgfqpoint{-0.049081in}{0.028505in}}{\pgfqpoint{-0.054861in}{0.014549in}}{\pgfqpoint{-0.054861in}{0.000000in}}%
\pgfpathcurveto{\pgfqpoint{-0.054861in}{-0.014549in}}{\pgfqpoint{-0.049081in}{-0.028505in}}{\pgfqpoint{-0.038793in}{-0.038793in}}%
\pgfpathcurveto{\pgfqpoint{-0.028505in}{-0.049081in}}{\pgfqpoint{-0.014549in}{-0.054861in}}{\pgfqpoint{0.000000in}{-0.054861in}}%
\pgfpathclose%
\pgfusepath{stroke,fill}%
}%
\begin{pgfscope}%
\pgfsys@transformshift{4.394218in}{0.845130in}%
\pgfsys@useobject{currentmarker}{}%
\end{pgfscope}%
\end{pgfscope}%
\begin{pgfscope}%
\pgfpathrectangle{\pgfqpoint{0.100000in}{0.100000in}}{\pgfqpoint{5.307240in}{3.397500in}}%
\pgfusepath{clip}%
\pgfsetrectcap%
\pgfsetroundjoin%
\pgfsetlinewidth{1.505625pt}%
\definecolor{currentstroke}{rgb}{0.678431,1.000000,0.184314}%
\pgfsetstrokecolor{currentstroke}%
\pgfsetstrokeopacity{0.500000}%
\pgfsetdash{}{0pt}%
\pgfpathmoveto{\pgfqpoint{4.131914in}{1.122562in}}%
\pgfusepath{stroke}%
\end{pgfscope}%
\begin{pgfscope}%
\pgfpathrectangle{\pgfqpoint{0.100000in}{0.100000in}}{\pgfqpoint{5.307240in}{3.397500in}}%
\pgfusepath{clip}%
\pgfsetbuttcap%
\pgfsetroundjoin%
\definecolor{currentfill}{rgb}{0.678431,1.000000,0.184314}%
\pgfsetfillcolor{currentfill}%
\pgfsetfillopacity{0.500000}%
\pgfsetlinewidth{0.250937pt}%
\definecolor{currentstroke}{rgb}{0.000000,0.000000,0.000000}%
\pgfsetstrokecolor{currentstroke}%
\pgfsetstrokeopacity{0.500000}%
\pgfsetdash{}{0pt}%
\pgfsys@defobject{currentmarker}{\pgfqpoint{-0.052083in}{-0.052083in}}{\pgfqpoint{0.052083in}{0.052083in}}{%
\pgfpathmoveto{\pgfqpoint{0.000000in}{-0.052083in}}%
\pgfpathcurveto{\pgfqpoint{0.013813in}{-0.052083in}}{\pgfqpoint{0.027061in}{-0.046596in}}{\pgfqpoint{0.036828in}{-0.036828in}}%
\pgfpathcurveto{\pgfqpoint{0.046596in}{-0.027061in}}{\pgfqpoint{0.052083in}{-0.013813in}}{\pgfqpoint{0.052083in}{0.000000in}}%
\pgfpathcurveto{\pgfqpoint{0.052083in}{0.013813in}}{\pgfqpoint{0.046596in}{0.027061in}}{\pgfqpoint{0.036828in}{0.036828in}}%
\pgfpathcurveto{\pgfqpoint{0.027061in}{0.046596in}}{\pgfqpoint{0.013813in}{0.052083in}}{\pgfqpoint{0.000000in}{0.052083in}}%
\pgfpathcurveto{\pgfqpoint{-0.013813in}{0.052083in}}{\pgfqpoint{-0.027061in}{0.046596in}}{\pgfqpoint{-0.036828in}{0.036828in}}%
\pgfpathcurveto{\pgfqpoint{-0.046596in}{0.027061in}}{\pgfqpoint{-0.052083in}{0.013813in}}{\pgfqpoint{-0.052083in}{0.000000in}}%
\pgfpathcurveto{\pgfqpoint{-0.052083in}{-0.013813in}}{\pgfqpoint{-0.046596in}{-0.027061in}}{\pgfqpoint{-0.036828in}{-0.036828in}}%
\pgfpathcurveto{\pgfqpoint{-0.027061in}{-0.046596in}}{\pgfqpoint{-0.013813in}{-0.052083in}}{\pgfqpoint{0.000000in}{-0.052083in}}%
\pgfpathclose%
\pgfusepath{stroke,fill}%
}%
\begin{pgfscope}%
\pgfsys@transformshift{4.131914in}{1.122562in}%
\pgfsys@useobject{currentmarker}{}%
\end{pgfscope}%
\end{pgfscope}%
\begin{pgfscope}%
\pgfpathrectangle{\pgfqpoint{0.100000in}{0.100000in}}{\pgfqpoint{5.307240in}{3.397500in}}%
\pgfusepath{clip}%
\pgfsetrectcap%
\pgfsetroundjoin%
\pgfsetlinewidth{1.505625pt}%
\definecolor{currentstroke}{rgb}{0.678431,1.000000,0.184314}%
\pgfsetstrokecolor{currentstroke}%
\pgfsetstrokeopacity{0.500000}%
\pgfsetdash{}{0pt}%
\pgfpathmoveto{\pgfqpoint{4.172554in}{1.407713in}}%
\pgfusepath{stroke}%
\end{pgfscope}%
\begin{pgfscope}%
\pgfpathrectangle{\pgfqpoint{0.100000in}{0.100000in}}{\pgfqpoint{5.307240in}{3.397500in}}%
\pgfusepath{clip}%
\pgfsetbuttcap%
\pgfsetroundjoin%
\definecolor{currentfill}{rgb}{0.678431,1.000000,0.184314}%
\pgfsetfillcolor{currentfill}%
\pgfsetfillopacity{0.500000}%
\pgfsetlinewidth{0.250937pt}%
\definecolor{currentstroke}{rgb}{0.000000,0.000000,0.000000}%
\pgfsetstrokecolor{currentstroke}%
\pgfsetstrokeopacity{0.500000}%
\pgfsetdash{}{0pt}%
\pgfsys@defobject{currentmarker}{\pgfqpoint{-0.058333in}{-0.058333in}}{\pgfqpoint{0.058333in}{0.058333in}}{%
\pgfpathmoveto{\pgfqpoint{0.000000in}{-0.058333in}}%
\pgfpathcurveto{\pgfqpoint{0.015470in}{-0.058333in}}{\pgfqpoint{0.030309in}{-0.052187in}}{\pgfqpoint{0.041248in}{-0.041248in}}%
\pgfpathcurveto{\pgfqpoint{0.052187in}{-0.030309in}}{\pgfqpoint{0.058333in}{-0.015470in}}{\pgfqpoint{0.058333in}{0.000000in}}%
\pgfpathcurveto{\pgfqpoint{0.058333in}{0.015470in}}{\pgfqpoint{0.052187in}{0.030309in}}{\pgfqpoint{0.041248in}{0.041248in}}%
\pgfpathcurveto{\pgfqpoint{0.030309in}{0.052187in}}{\pgfqpoint{0.015470in}{0.058333in}}{\pgfqpoint{0.000000in}{0.058333in}}%
\pgfpathcurveto{\pgfqpoint{-0.015470in}{0.058333in}}{\pgfqpoint{-0.030309in}{0.052187in}}{\pgfqpoint{-0.041248in}{0.041248in}}%
\pgfpathcurveto{\pgfqpoint{-0.052187in}{0.030309in}}{\pgfqpoint{-0.058333in}{0.015470in}}{\pgfqpoint{-0.058333in}{0.000000in}}%
\pgfpathcurveto{\pgfqpoint{-0.058333in}{-0.015470in}}{\pgfqpoint{-0.052187in}{-0.030309in}}{\pgfqpoint{-0.041248in}{-0.041248in}}%
\pgfpathcurveto{\pgfqpoint{-0.030309in}{-0.052187in}}{\pgfqpoint{-0.015470in}{-0.058333in}}{\pgfqpoint{0.000000in}{-0.058333in}}%
\pgfpathclose%
\pgfusepath{stroke,fill}%
}%
\begin{pgfscope}%
\pgfsys@transformshift{4.172554in}{1.407713in}%
\pgfsys@useobject{currentmarker}{}%
\end{pgfscope}%
\end{pgfscope}%
\begin{pgfscope}%
\pgfpathrectangle{\pgfqpoint{0.100000in}{0.100000in}}{\pgfqpoint{5.307240in}{3.397500in}}%
\pgfusepath{clip}%
\pgfsetrectcap%
\pgfsetroundjoin%
\pgfsetlinewidth{1.505625pt}%
\definecolor{currentstroke}{rgb}{0.678431,1.000000,0.184314}%
\pgfsetstrokecolor{currentstroke}%
\pgfsetstrokeopacity{0.500000}%
\pgfsetdash{}{0pt}%
\pgfpathmoveto{\pgfqpoint{4.079080in}{1.372003in}}%
\pgfusepath{stroke}%
\end{pgfscope}%
\begin{pgfscope}%
\pgfpathrectangle{\pgfqpoint{0.100000in}{0.100000in}}{\pgfqpoint{5.307240in}{3.397500in}}%
\pgfusepath{clip}%
\pgfsetbuttcap%
\pgfsetroundjoin%
\definecolor{currentfill}{rgb}{0.678431,1.000000,0.184314}%
\pgfsetfillcolor{currentfill}%
\pgfsetfillopacity{0.500000}%
\pgfsetlinewidth{0.250937pt}%
\definecolor{currentstroke}{rgb}{0.000000,0.000000,0.000000}%
\pgfsetstrokecolor{currentstroke}%
\pgfsetstrokeopacity{0.500000}%
\pgfsetdash{}{0pt}%
\pgfsys@defobject{currentmarker}{\pgfqpoint{-0.067361in}{-0.067361in}}{\pgfqpoint{0.067361in}{0.067361in}}{%
\pgfpathmoveto{\pgfqpoint{0.000000in}{-0.067361in}}%
\pgfpathcurveto{\pgfqpoint{0.017864in}{-0.067361in}}{\pgfqpoint{0.034999in}{-0.060264in}}{\pgfqpoint{0.047631in}{-0.047631in}}%
\pgfpathcurveto{\pgfqpoint{0.060264in}{-0.034999in}}{\pgfqpoint{0.067361in}{-0.017864in}}{\pgfqpoint{0.067361in}{0.000000in}}%
\pgfpathcurveto{\pgfqpoint{0.067361in}{0.017864in}}{\pgfqpoint{0.060264in}{0.034999in}}{\pgfqpoint{0.047631in}{0.047631in}}%
\pgfpathcurveto{\pgfqpoint{0.034999in}{0.060264in}}{\pgfqpoint{0.017864in}{0.067361in}}{\pgfqpoint{0.000000in}{0.067361in}}%
\pgfpathcurveto{\pgfqpoint{-0.017864in}{0.067361in}}{\pgfqpoint{-0.034999in}{0.060264in}}{\pgfqpoint{-0.047631in}{0.047631in}}%
\pgfpathcurveto{\pgfqpoint{-0.060264in}{0.034999in}}{\pgfqpoint{-0.067361in}{0.017864in}}{\pgfqpoint{-0.067361in}{0.000000in}}%
\pgfpathcurveto{\pgfqpoint{-0.067361in}{-0.017864in}}{\pgfqpoint{-0.060264in}{-0.034999in}}{\pgfqpoint{-0.047631in}{-0.047631in}}%
\pgfpathcurveto{\pgfqpoint{-0.034999in}{-0.060264in}}{\pgfqpoint{-0.017864in}{-0.067361in}}{\pgfqpoint{0.000000in}{-0.067361in}}%
\pgfpathclose%
\pgfusepath{stroke,fill}%
}%
\begin{pgfscope}%
\pgfsys@transformshift{4.079080in}{1.372003in}%
\pgfsys@useobject{currentmarker}{}%
\end{pgfscope}%
\end{pgfscope}%
\begin{pgfscope}%
\pgfpathrectangle{\pgfqpoint{0.100000in}{0.100000in}}{\pgfqpoint{5.307240in}{3.397500in}}%
\pgfusepath{clip}%
\pgfsetrectcap%
\pgfsetroundjoin%
\pgfsetlinewidth{1.505625pt}%
\definecolor{currentstroke}{rgb}{0.678431,1.000000,0.184314}%
\pgfsetstrokecolor{currentstroke}%
\pgfsetstrokeopacity{0.500000}%
\pgfsetdash{}{0pt}%
\pgfpathmoveto{\pgfqpoint{4.316189in}{1.370261in}}%
\pgfusepath{stroke}%
\end{pgfscope}%
\begin{pgfscope}%
\pgfpathrectangle{\pgfqpoint{0.100000in}{0.100000in}}{\pgfqpoint{5.307240in}{3.397500in}}%
\pgfusepath{clip}%
\pgfsetbuttcap%
\pgfsetroundjoin%
\definecolor{currentfill}{rgb}{0.678431,1.000000,0.184314}%
\pgfsetfillcolor{currentfill}%
\pgfsetfillopacity{0.500000}%
\pgfsetlinewidth{0.250937pt}%
\definecolor{currentstroke}{rgb}{0.000000,0.000000,0.000000}%
\pgfsetstrokecolor{currentstroke}%
\pgfsetstrokeopacity{0.500000}%
\pgfsetdash{}{0pt}%
\pgfsys@defobject{currentmarker}{\pgfqpoint{-0.051389in}{-0.051389in}}{\pgfqpoint{0.051389in}{0.051389in}}{%
\pgfpathmoveto{\pgfqpoint{0.000000in}{-0.051389in}}%
\pgfpathcurveto{\pgfqpoint{0.013628in}{-0.051389in}}{\pgfqpoint{0.026701in}{-0.045974in}}{\pgfqpoint{0.036337in}{-0.036337in}}%
\pgfpathcurveto{\pgfqpoint{0.045974in}{-0.026701in}}{\pgfqpoint{0.051389in}{-0.013628in}}{\pgfqpoint{0.051389in}{0.000000in}}%
\pgfpathcurveto{\pgfqpoint{0.051389in}{0.013628in}}{\pgfqpoint{0.045974in}{0.026701in}}{\pgfqpoint{0.036337in}{0.036337in}}%
\pgfpathcurveto{\pgfqpoint{0.026701in}{0.045974in}}{\pgfqpoint{0.013628in}{0.051389in}}{\pgfqpoint{0.000000in}{0.051389in}}%
\pgfpathcurveto{\pgfqpoint{-0.013628in}{0.051389in}}{\pgfqpoint{-0.026701in}{0.045974in}}{\pgfqpoint{-0.036337in}{0.036337in}}%
\pgfpathcurveto{\pgfqpoint{-0.045974in}{0.026701in}}{\pgfqpoint{-0.051389in}{0.013628in}}{\pgfqpoint{-0.051389in}{0.000000in}}%
\pgfpathcurveto{\pgfqpoint{-0.051389in}{-0.013628in}}{\pgfqpoint{-0.045974in}{-0.026701in}}{\pgfqpoint{-0.036337in}{-0.036337in}}%
\pgfpathcurveto{\pgfqpoint{-0.026701in}{-0.045974in}}{\pgfqpoint{-0.013628in}{-0.051389in}}{\pgfqpoint{0.000000in}{-0.051389in}}%
\pgfpathclose%
\pgfusepath{stroke,fill}%
}%
\begin{pgfscope}%
\pgfsys@transformshift{4.316189in}{1.370261in}%
\pgfsys@useobject{currentmarker}{}%
\end{pgfscope}%
\end{pgfscope}%
\begin{pgfscope}%
\pgfpathrectangle{\pgfqpoint{0.100000in}{0.100000in}}{\pgfqpoint{5.307240in}{3.397500in}}%
\pgfusepath{clip}%
\pgfsetrectcap%
\pgfsetroundjoin%
\pgfsetlinewidth{1.505625pt}%
\definecolor{currentstroke}{rgb}{0.678431,1.000000,0.184314}%
\pgfsetstrokecolor{currentstroke}%
\pgfsetstrokeopacity{0.500000}%
\pgfsetdash{}{0pt}%
\pgfpathmoveto{\pgfqpoint{4.403015in}{1.108432in}}%
\pgfusepath{stroke}%
\end{pgfscope}%
\begin{pgfscope}%
\pgfpathrectangle{\pgfqpoint{0.100000in}{0.100000in}}{\pgfqpoint{5.307240in}{3.397500in}}%
\pgfusepath{clip}%
\pgfsetbuttcap%
\pgfsetroundjoin%
\definecolor{currentfill}{rgb}{0.678431,1.000000,0.184314}%
\pgfsetfillcolor{currentfill}%
\pgfsetfillopacity{0.500000}%
\pgfsetlinewidth{0.250937pt}%
\definecolor{currentstroke}{rgb}{0.000000,0.000000,0.000000}%
\pgfsetstrokecolor{currentstroke}%
\pgfsetstrokeopacity{0.500000}%
\pgfsetdash{}{0pt}%
\pgfsys@defobject{currentmarker}{\pgfqpoint{-0.087500in}{-0.087500in}}{\pgfqpoint{0.087500in}{0.087500in}}{%
\pgfpathmoveto{\pgfqpoint{0.000000in}{-0.087500in}}%
\pgfpathcurveto{\pgfqpoint{0.023205in}{-0.087500in}}{\pgfqpoint{0.045463in}{-0.078280in}}{\pgfqpoint{0.061872in}{-0.061872in}}%
\pgfpathcurveto{\pgfqpoint{0.078280in}{-0.045463in}}{\pgfqpoint{0.087500in}{-0.023205in}}{\pgfqpoint{0.087500in}{0.000000in}}%
\pgfpathcurveto{\pgfqpoint{0.087500in}{0.023205in}}{\pgfqpoint{0.078280in}{0.045463in}}{\pgfqpoint{0.061872in}{0.061872in}}%
\pgfpathcurveto{\pgfqpoint{0.045463in}{0.078280in}}{\pgfqpoint{0.023205in}{0.087500in}}{\pgfqpoint{0.000000in}{0.087500in}}%
\pgfpathcurveto{\pgfqpoint{-0.023205in}{0.087500in}}{\pgfqpoint{-0.045463in}{0.078280in}}{\pgfqpoint{-0.061872in}{0.061872in}}%
\pgfpathcurveto{\pgfqpoint{-0.078280in}{0.045463in}}{\pgfqpoint{-0.087500in}{0.023205in}}{\pgfqpoint{-0.087500in}{0.000000in}}%
\pgfpathcurveto{\pgfqpoint{-0.087500in}{-0.023205in}}{\pgfqpoint{-0.078280in}{-0.045463in}}{\pgfqpoint{-0.061872in}{-0.061872in}}%
\pgfpathcurveto{\pgfqpoint{-0.045463in}{-0.078280in}}{\pgfqpoint{-0.023205in}{-0.087500in}}{\pgfqpoint{0.000000in}{-0.087500in}}%
\pgfpathclose%
\pgfusepath{stroke,fill}%
}%
\begin{pgfscope}%
\pgfsys@transformshift{4.403015in}{1.108432in}%
\pgfsys@useobject{currentmarker}{}%
\end{pgfscope}%
\end{pgfscope}%
\begin{pgfscope}%
\pgfpathrectangle{\pgfqpoint{0.100000in}{0.100000in}}{\pgfqpoint{5.307240in}{3.397500in}}%
\pgfusepath{clip}%
\pgfsetrectcap%
\pgfsetroundjoin%
\pgfsetlinewidth{1.505625pt}%
\definecolor{currentstroke}{rgb}{0.678431,1.000000,0.184314}%
\pgfsetstrokecolor{currentstroke}%
\pgfsetstrokeopacity{0.500000}%
\pgfsetdash{}{0pt}%
\pgfpathmoveto{\pgfqpoint{4.038004in}{1.215804in}}%
\pgfusepath{stroke}%
\end{pgfscope}%
\begin{pgfscope}%
\pgfpathrectangle{\pgfqpoint{0.100000in}{0.100000in}}{\pgfqpoint{5.307240in}{3.397500in}}%
\pgfusepath{clip}%
\pgfsetbuttcap%
\pgfsetroundjoin%
\definecolor{currentfill}{rgb}{0.678431,1.000000,0.184314}%
\pgfsetfillcolor{currentfill}%
\pgfsetfillopacity{0.500000}%
\pgfsetlinewidth{0.250937pt}%
\definecolor{currentstroke}{rgb}{0.000000,0.000000,0.000000}%
\pgfsetstrokecolor{currentstroke}%
\pgfsetstrokeopacity{0.500000}%
\pgfsetdash{}{0pt}%
\pgfsys@defobject{currentmarker}{\pgfqpoint{-0.059722in}{-0.059722in}}{\pgfqpoint{0.059722in}{0.059722in}}{%
\pgfpathmoveto{\pgfqpoint{0.000000in}{-0.059722in}}%
\pgfpathcurveto{\pgfqpoint{0.015839in}{-0.059722in}}{\pgfqpoint{0.031030in}{-0.053430in}}{\pgfqpoint{0.042230in}{-0.042230in}}%
\pgfpathcurveto{\pgfqpoint{0.053430in}{-0.031030in}}{\pgfqpoint{0.059722in}{-0.015839in}}{\pgfqpoint{0.059722in}{0.000000in}}%
\pgfpathcurveto{\pgfqpoint{0.059722in}{0.015839in}}{\pgfqpoint{0.053430in}{0.031030in}}{\pgfqpoint{0.042230in}{0.042230in}}%
\pgfpathcurveto{\pgfqpoint{0.031030in}{0.053430in}}{\pgfqpoint{0.015839in}{0.059722in}}{\pgfqpoint{0.000000in}{0.059722in}}%
\pgfpathcurveto{\pgfqpoint{-0.015839in}{0.059722in}}{\pgfqpoint{-0.031030in}{0.053430in}}{\pgfqpoint{-0.042230in}{0.042230in}}%
\pgfpathcurveto{\pgfqpoint{-0.053430in}{0.031030in}}{\pgfqpoint{-0.059722in}{0.015839in}}{\pgfqpoint{-0.059722in}{0.000000in}}%
\pgfpathcurveto{\pgfqpoint{-0.059722in}{-0.015839in}}{\pgfqpoint{-0.053430in}{-0.031030in}}{\pgfqpoint{-0.042230in}{-0.042230in}}%
\pgfpathcurveto{\pgfqpoint{-0.031030in}{-0.053430in}}{\pgfqpoint{-0.015839in}{-0.059722in}}{\pgfqpoint{0.000000in}{-0.059722in}}%
\pgfpathclose%
\pgfusepath{stroke,fill}%
}%
\begin{pgfscope}%
\pgfsys@transformshift{4.038004in}{1.215804in}%
\pgfsys@useobject{currentmarker}{}%
\end{pgfscope}%
\end{pgfscope}%
\begin{pgfscope}%
\pgfpathrectangle{\pgfqpoint{0.100000in}{0.100000in}}{\pgfqpoint{5.307240in}{3.397500in}}%
\pgfusepath{clip}%
\pgfsetrectcap%
\pgfsetroundjoin%
\pgfsetlinewidth{1.505625pt}%
\definecolor{currentstroke}{rgb}{0.678431,1.000000,0.184314}%
\pgfsetstrokecolor{currentstroke}%
\pgfsetstrokeopacity{0.500000}%
\pgfsetdash{}{0pt}%
\pgfpathmoveto{\pgfqpoint{4.020302in}{1.479709in}}%
\pgfusepath{stroke}%
\end{pgfscope}%
\begin{pgfscope}%
\pgfpathrectangle{\pgfqpoint{0.100000in}{0.100000in}}{\pgfqpoint{5.307240in}{3.397500in}}%
\pgfusepath{clip}%
\pgfsetbuttcap%
\pgfsetroundjoin%
\definecolor{currentfill}{rgb}{0.678431,1.000000,0.184314}%
\pgfsetfillcolor{currentfill}%
\pgfsetfillopacity{0.500000}%
\pgfsetlinewidth{0.250937pt}%
\definecolor{currentstroke}{rgb}{0.000000,0.000000,0.000000}%
\pgfsetstrokecolor{currentstroke}%
\pgfsetstrokeopacity{0.500000}%
\pgfsetdash{}{0pt}%
\pgfsys@defobject{currentmarker}{\pgfqpoint{-0.113889in}{-0.113889in}}{\pgfqpoint{0.113889in}{0.113889in}}{%
\pgfpathmoveto{\pgfqpoint{0.000000in}{-0.113889in}}%
\pgfpathcurveto{\pgfqpoint{0.030204in}{-0.113889in}}{\pgfqpoint{0.059174in}{-0.101889in}}{\pgfqpoint{0.080532in}{-0.080532in}}%
\pgfpathcurveto{\pgfqpoint{0.101889in}{-0.059174in}}{\pgfqpoint{0.113889in}{-0.030204in}}{\pgfqpoint{0.113889in}{0.000000in}}%
\pgfpathcurveto{\pgfqpoint{0.113889in}{0.030204in}}{\pgfqpoint{0.101889in}{0.059174in}}{\pgfqpoint{0.080532in}{0.080532in}}%
\pgfpathcurveto{\pgfqpoint{0.059174in}{0.101889in}}{\pgfqpoint{0.030204in}{0.113889in}}{\pgfqpoint{0.000000in}{0.113889in}}%
\pgfpathcurveto{\pgfqpoint{-0.030204in}{0.113889in}}{\pgfqpoint{-0.059174in}{0.101889in}}{\pgfqpoint{-0.080532in}{0.080532in}}%
\pgfpathcurveto{\pgfqpoint{-0.101889in}{0.059174in}}{\pgfqpoint{-0.113889in}{0.030204in}}{\pgfqpoint{-0.113889in}{0.000000in}}%
\pgfpathcurveto{\pgfqpoint{-0.113889in}{-0.030204in}}{\pgfqpoint{-0.101889in}{-0.059174in}}{\pgfqpoint{-0.080532in}{-0.080532in}}%
\pgfpathcurveto{\pgfqpoint{-0.059174in}{-0.101889in}}{\pgfqpoint{-0.030204in}{-0.113889in}}{\pgfqpoint{0.000000in}{-0.113889in}}%
\pgfpathclose%
\pgfusepath{stroke,fill}%
}%
\begin{pgfscope}%
\pgfsys@transformshift{4.020302in}{1.479709in}%
\pgfsys@useobject{currentmarker}{}%
\end{pgfscope}%
\end{pgfscope}%
\begin{pgfscope}%
\pgfpathrectangle{\pgfqpoint{0.100000in}{0.100000in}}{\pgfqpoint{5.307240in}{3.397500in}}%
\pgfusepath{clip}%
\pgfsetrectcap%
\pgfsetroundjoin%
\pgfsetlinewidth{1.505625pt}%
\definecolor{currentstroke}{rgb}{0.678431,1.000000,0.184314}%
\pgfsetstrokecolor{currentstroke}%
\pgfsetstrokeopacity{0.500000}%
\pgfsetdash{}{0pt}%
\pgfpathmoveto{\pgfqpoint{4.125783in}{1.442130in}}%
\pgfusepath{stroke}%
\end{pgfscope}%
\begin{pgfscope}%
\pgfpathrectangle{\pgfqpoint{0.100000in}{0.100000in}}{\pgfqpoint{5.307240in}{3.397500in}}%
\pgfusepath{clip}%
\pgfsetbuttcap%
\pgfsetroundjoin%
\definecolor{currentfill}{rgb}{0.678431,1.000000,0.184314}%
\pgfsetfillcolor{currentfill}%
\pgfsetfillopacity{0.500000}%
\pgfsetlinewidth{0.250937pt}%
\definecolor{currentstroke}{rgb}{0.000000,0.000000,0.000000}%
\pgfsetstrokecolor{currentstroke}%
\pgfsetstrokeopacity{0.500000}%
\pgfsetdash{}{0pt}%
\pgfsys@defobject{currentmarker}{\pgfqpoint{-0.055556in}{-0.055556in}}{\pgfqpoint{0.055556in}{0.055556in}}{%
\pgfpathmoveto{\pgfqpoint{0.000000in}{-0.055556in}}%
\pgfpathcurveto{\pgfqpoint{0.014734in}{-0.055556in}}{\pgfqpoint{0.028866in}{-0.049702in}}{\pgfqpoint{0.039284in}{-0.039284in}}%
\pgfpathcurveto{\pgfqpoint{0.049702in}{-0.028866in}}{\pgfqpoint{0.055556in}{-0.014734in}}{\pgfqpoint{0.055556in}{0.000000in}}%
\pgfpathcurveto{\pgfqpoint{0.055556in}{0.014734in}}{\pgfqpoint{0.049702in}{0.028866in}}{\pgfqpoint{0.039284in}{0.039284in}}%
\pgfpathcurveto{\pgfqpoint{0.028866in}{0.049702in}}{\pgfqpoint{0.014734in}{0.055556in}}{\pgfqpoint{0.000000in}{0.055556in}}%
\pgfpathcurveto{\pgfqpoint{-0.014734in}{0.055556in}}{\pgfqpoint{-0.028866in}{0.049702in}}{\pgfqpoint{-0.039284in}{0.039284in}}%
\pgfpathcurveto{\pgfqpoint{-0.049702in}{0.028866in}}{\pgfqpoint{-0.055556in}{0.014734in}}{\pgfqpoint{-0.055556in}{0.000000in}}%
\pgfpathcurveto{\pgfqpoint{-0.055556in}{-0.014734in}}{\pgfqpoint{-0.049702in}{-0.028866in}}{\pgfqpoint{-0.039284in}{-0.039284in}}%
\pgfpathcurveto{\pgfqpoint{-0.028866in}{-0.049702in}}{\pgfqpoint{-0.014734in}{-0.055556in}}{\pgfqpoint{0.000000in}{-0.055556in}}%
\pgfpathclose%
\pgfusepath{stroke,fill}%
}%
\begin{pgfscope}%
\pgfsys@transformshift{4.125783in}{1.442130in}%
\pgfsys@useobject{currentmarker}{}%
\end{pgfscope}%
\end{pgfscope}%
\begin{pgfscope}%
\pgfpathrectangle{\pgfqpoint{0.100000in}{0.100000in}}{\pgfqpoint{5.307240in}{3.397500in}}%
\pgfusepath{clip}%
\pgfsetrectcap%
\pgfsetroundjoin%
\pgfsetlinewidth{1.505625pt}%
\definecolor{currentstroke}{rgb}{0.678431,1.000000,0.184314}%
\pgfsetstrokecolor{currentstroke}%
\pgfsetstrokeopacity{0.500000}%
\pgfsetdash{}{0pt}%
\pgfpathmoveto{\pgfqpoint{4.380609in}{1.187804in}}%
\pgfusepath{stroke}%
\end{pgfscope}%
\begin{pgfscope}%
\pgfpathrectangle{\pgfqpoint{0.100000in}{0.100000in}}{\pgfqpoint{5.307240in}{3.397500in}}%
\pgfusepath{clip}%
\pgfsetbuttcap%
\pgfsetroundjoin%
\definecolor{currentfill}{rgb}{0.678431,1.000000,0.184314}%
\pgfsetfillcolor{currentfill}%
\pgfsetfillopacity{0.500000}%
\pgfsetlinewidth{0.250937pt}%
\definecolor{currentstroke}{rgb}{0.000000,0.000000,0.000000}%
\pgfsetstrokecolor{currentstroke}%
\pgfsetstrokeopacity{0.500000}%
\pgfsetdash{}{0pt}%
\pgfsys@defobject{currentmarker}{\pgfqpoint{-0.044444in}{-0.044444in}}{\pgfqpoint{0.044444in}{0.044444in}}{%
\pgfpathmoveto{\pgfqpoint{0.000000in}{-0.044444in}}%
\pgfpathcurveto{\pgfqpoint{0.011787in}{-0.044444in}}{\pgfqpoint{0.023092in}{-0.039761in}}{\pgfqpoint{0.031427in}{-0.031427in}}%
\pgfpathcurveto{\pgfqpoint{0.039761in}{-0.023092in}}{\pgfqpoint{0.044444in}{-0.011787in}}{\pgfqpoint{0.044444in}{0.000000in}}%
\pgfpathcurveto{\pgfqpoint{0.044444in}{0.011787in}}{\pgfqpoint{0.039761in}{0.023092in}}{\pgfqpoint{0.031427in}{0.031427in}}%
\pgfpathcurveto{\pgfqpoint{0.023092in}{0.039761in}}{\pgfqpoint{0.011787in}{0.044444in}}{\pgfqpoint{0.000000in}{0.044444in}}%
\pgfpathcurveto{\pgfqpoint{-0.011787in}{0.044444in}}{\pgfqpoint{-0.023092in}{0.039761in}}{\pgfqpoint{-0.031427in}{0.031427in}}%
\pgfpathcurveto{\pgfqpoint{-0.039761in}{0.023092in}}{\pgfqpoint{-0.044444in}{0.011787in}}{\pgfqpoint{-0.044444in}{0.000000in}}%
\pgfpathcurveto{\pgfqpoint{-0.044444in}{-0.011787in}}{\pgfqpoint{-0.039761in}{-0.023092in}}{\pgfqpoint{-0.031427in}{-0.031427in}}%
\pgfpathcurveto{\pgfqpoint{-0.023092in}{-0.039761in}}{\pgfqpoint{-0.011787in}{-0.044444in}}{\pgfqpoint{0.000000in}{-0.044444in}}%
\pgfpathclose%
\pgfusepath{stroke,fill}%
}%
\begin{pgfscope}%
\pgfsys@transformshift{4.380609in}{1.187804in}%
\pgfsys@useobject{currentmarker}{}%
\end{pgfscope}%
\end{pgfscope}%
\begin{pgfscope}%
\pgfpathrectangle{\pgfqpoint{0.100000in}{0.100000in}}{\pgfqpoint{5.307240in}{3.397500in}}%
\pgfusepath{clip}%
\pgfsetrectcap%
\pgfsetroundjoin%
\pgfsetlinewidth{1.505625pt}%
\definecolor{currentstroke}{rgb}{0.678431,1.000000,0.184314}%
\pgfsetstrokecolor{currentstroke}%
\pgfsetstrokeopacity{0.500000}%
\pgfsetdash{}{0pt}%
\pgfpathmoveto{\pgfqpoint{4.165342in}{1.275574in}}%
\pgfusepath{stroke}%
\end{pgfscope}%
\begin{pgfscope}%
\pgfpathrectangle{\pgfqpoint{0.100000in}{0.100000in}}{\pgfqpoint{5.307240in}{3.397500in}}%
\pgfusepath{clip}%
\pgfsetbuttcap%
\pgfsetroundjoin%
\definecolor{currentfill}{rgb}{0.678431,1.000000,0.184314}%
\pgfsetfillcolor{currentfill}%
\pgfsetfillopacity{0.500000}%
\pgfsetlinewidth{0.250937pt}%
\definecolor{currentstroke}{rgb}{0.000000,0.000000,0.000000}%
\pgfsetstrokecolor{currentstroke}%
\pgfsetstrokeopacity{0.500000}%
\pgfsetdash{}{0pt}%
\pgfsys@defobject{currentmarker}{\pgfqpoint{-0.050694in}{-0.050694in}}{\pgfqpoint{0.050694in}{0.050694in}}{%
\pgfpathmoveto{\pgfqpoint{0.000000in}{-0.050694in}}%
\pgfpathcurveto{\pgfqpoint{0.013444in}{-0.050694in}}{\pgfqpoint{0.026340in}{-0.045353in}}{\pgfqpoint{0.035846in}{-0.035846in}}%
\pgfpathcurveto{\pgfqpoint{0.045353in}{-0.026340in}}{\pgfqpoint{0.050694in}{-0.013444in}}{\pgfqpoint{0.050694in}{0.000000in}}%
\pgfpathcurveto{\pgfqpoint{0.050694in}{0.013444in}}{\pgfqpoint{0.045353in}{0.026340in}}{\pgfqpoint{0.035846in}{0.035846in}}%
\pgfpathcurveto{\pgfqpoint{0.026340in}{0.045353in}}{\pgfqpoint{0.013444in}{0.050694in}}{\pgfqpoint{0.000000in}{0.050694in}}%
\pgfpathcurveto{\pgfqpoint{-0.013444in}{0.050694in}}{\pgfqpoint{-0.026340in}{0.045353in}}{\pgfqpoint{-0.035846in}{0.035846in}}%
\pgfpathcurveto{\pgfqpoint{-0.045353in}{0.026340in}}{\pgfqpoint{-0.050694in}{0.013444in}}{\pgfqpoint{-0.050694in}{0.000000in}}%
\pgfpathcurveto{\pgfqpoint{-0.050694in}{-0.013444in}}{\pgfqpoint{-0.045353in}{-0.026340in}}{\pgfqpoint{-0.035846in}{-0.035846in}}%
\pgfpathcurveto{\pgfqpoint{-0.026340in}{-0.045353in}}{\pgfqpoint{-0.013444in}{-0.050694in}}{\pgfqpoint{0.000000in}{-0.050694in}}%
\pgfpathclose%
\pgfusepath{stroke,fill}%
}%
\begin{pgfscope}%
\pgfsys@transformshift{4.165342in}{1.275574in}%
\pgfsys@useobject{currentmarker}{}%
\end{pgfscope}%
\end{pgfscope}%
\begin{pgfscope}%
\pgfpathrectangle{\pgfqpoint{0.100000in}{0.100000in}}{\pgfqpoint{5.307240in}{3.397500in}}%
\pgfusepath{clip}%
\pgfsetrectcap%
\pgfsetroundjoin%
\pgfsetlinewidth{1.505625pt}%
\definecolor{currentstroke}{rgb}{0.678431,1.000000,0.184314}%
\pgfsetstrokecolor{currentstroke}%
\pgfsetstrokeopacity{0.500000}%
\pgfsetdash{}{0pt}%
\pgfpathmoveto{\pgfqpoint{3.997974in}{1.422493in}}%
\pgfusepath{stroke}%
\end{pgfscope}%
\begin{pgfscope}%
\pgfpathrectangle{\pgfqpoint{0.100000in}{0.100000in}}{\pgfqpoint{5.307240in}{3.397500in}}%
\pgfusepath{clip}%
\pgfsetbuttcap%
\pgfsetroundjoin%
\definecolor{currentfill}{rgb}{0.678431,1.000000,0.184314}%
\pgfsetfillcolor{currentfill}%
\pgfsetfillopacity{0.500000}%
\pgfsetlinewidth{0.250937pt}%
\definecolor{currentstroke}{rgb}{0.000000,0.000000,0.000000}%
\pgfsetstrokecolor{currentstroke}%
\pgfsetstrokeopacity{0.500000}%
\pgfsetdash{}{0pt}%
\pgfsys@defobject{currentmarker}{\pgfqpoint{-0.071528in}{-0.071528in}}{\pgfqpoint{0.071528in}{0.071528in}}{%
\pgfpathmoveto{\pgfqpoint{0.000000in}{-0.071528in}}%
\pgfpathcurveto{\pgfqpoint{0.018969in}{-0.071528in}}{\pgfqpoint{0.037164in}{-0.063991in}}{\pgfqpoint{0.050578in}{-0.050578in}}%
\pgfpathcurveto{\pgfqpoint{0.063991in}{-0.037164in}}{\pgfqpoint{0.071528in}{-0.018969in}}{\pgfqpoint{0.071528in}{0.000000in}}%
\pgfpathcurveto{\pgfqpoint{0.071528in}{0.018969in}}{\pgfqpoint{0.063991in}{0.037164in}}{\pgfqpoint{0.050578in}{0.050578in}}%
\pgfpathcurveto{\pgfqpoint{0.037164in}{0.063991in}}{\pgfqpoint{0.018969in}{0.071528in}}{\pgfqpoint{0.000000in}{0.071528in}}%
\pgfpathcurveto{\pgfqpoint{-0.018969in}{0.071528in}}{\pgfqpoint{-0.037164in}{0.063991in}}{\pgfqpoint{-0.050578in}{0.050578in}}%
\pgfpathcurveto{\pgfqpoint{-0.063991in}{0.037164in}}{\pgfqpoint{-0.071528in}{0.018969in}}{\pgfqpoint{-0.071528in}{0.000000in}}%
\pgfpathcurveto{\pgfqpoint{-0.071528in}{-0.018969in}}{\pgfqpoint{-0.063991in}{-0.037164in}}{\pgfqpoint{-0.050578in}{-0.050578in}}%
\pgfpathcurveto{\pgfqpoint{-0.037164in}{-0.063991in}}{\pgfqpoint{-0.018969in}{-0.071528in}}{\pgfqpoint{0.000000in}{-0.071528in}}%
\pgfpathclose%
\pgfusepath{stroke,fill}%
}%
\begin{pgfscope}%
\pgfsys@transformshift{3.997974in}{1.422493in}%
\pgfsys@useobject{currentmarker}{}%
\end{pgfscope}%
\end{pgfscope}%
\begin{pgfscope}%
\pgfpathrectangle{\pgfqpoint{0.100000in}{0.100000in}}{\pgfqpoint{5.307240in}{3.397500in}}%
\pgfusepath{clip}%
\pgfsetrectcap%
\pgfsetroundjoin%
\pgfsetlinewidth{1.505625pt}%
\definecolor{currentstroke}{rgb}{0.678431,1.000000,0.184314}%
\pgfsetstrokecolor{currentstroke}%
\pgfsetstrokeopacity{0.500000}%
\pgfsetdash{}{0pt}%
\pgfpathmoveto{\pgfqpoint{4.425225in}{1.222491in}}%
\pgfusepath{stroke}%
\end{pgfscope}%
\begin{pgfscope}%
\pgfpathrectangle{\pgfqpoint{0.100000in}{0.100000in}}{\pgfqpoint{5.307240in}{3.397500in}}%
\pgfusepath{clip}%
\pgfsetbuttcap%
\pgfsetroundjoin%
\definecolor{currentfill}{rgb}{0.678431,1.000000,0.184314}%
\pgfsetfillcolor{currentfill}%
\pgfsetfillopacity{0.500000}%
\pgfsetlinewidth{0.250937pt}%
\definecolor{currentstroke}{rgb}{0.000000,0.000000,0.000000}%
\pgfsetstrokecolor{currentstroke}%
\pgfsetstrokeopacity{0.500000}%
\pgfsetdash{}{0pt}%
\pgfsys@defobject{currentmarker}{\pgfqpoint{-0.086111in}{-0.086111in}}{\pgfqpoint{0.086111in}{0.086111in}}{%
\pgfpathmoveto{\pgfqpoint{0.000000in}{-0.086111in}}%
\pgfpathcurveto{\pgfqpoint{0.022837in}{-0.086111in}}{\pgfqpoint{0.044742in}{-0.077038in}}{\pgfqpoint{0.060890in}{-0.060890in}}%
\pgfpathcurveto{\pgfqpoint{0.077038in}{-0.044742in}}{\pgfqpoint{0.086111in}{-0.022837in}}{\pgfqpoint{0.086111in}{0.000000in}}%
\pgfpathcurveto{\pgfqpoint{0.086111in}{0.022837in}}{\pgfqpoint{0.077038in}{0.044742in}}{\pgfqpoint{0.060890in}{0.060890in}}%
\pgfpathcurveto{\pgfqpoint{0.044742in}{0.077038in}}{\pgfqpoint{0.022837in}{0.086111in}}{\pgfqpoint{0.000000in}{0.086111in}}%
\pgfpathcurveto{\pgfqpoint{-0.022837in}{0.086111in}}{\pgfqpoint{-0.044742in}{0.077038in}}{\pgfqpoint{-0.060890in}{0.060890in}}%
\pgfpathcurveto{\pgfqpoint{-0.077038in}{0.044742in}}{\pgfqpoint{-0.086111in}{0.022837in}}{\pgfqpoint{-0.086111in}{0.000000in}}%
\pgfpathcurveto{\pgfqpoint{-0.086111in}{-0.022837in}}{\pgfqpoint{-0.077038in}{-0.044742in}}{\pgfqpoint{-0.060890in}{-0.060890in}}%
\pgfpathcurveto{\pgfqpoint{-0.044742in}{-0.077038in}}{\pgfqpoint{-0.022837in}{-0.086111in}}{\pgfqpoint{0.000000in}{-0.086111in}}%
\pgfpathclose%
\pgfusepath{stroke,fill}%
}%
\begin{pgfscope}%
\pgfsys@transformshift{4.425225in}{1.222491in}%
\pgfsys@useobject{currentmarker}{}%
\end{pgfscope}%
\end{pgfscope}%
\begin{pgfscope}%
\pgfpathrectangle{\pgfqpoint{0.100000in}{0.100000in}}{\pgfqpoint{5.307240in}{3.397500in}}%
\pgfusepath{clip}%
\pgfsetrectcap%
\pgfsetroundjoin%
\pgfsetlinewidth{1.505625pt}%
\definecolor{currentstroke}{rgb}{0.678431,1.000000,0.184314}%
\pgfsetstrokecolor{currentstroke}%
\pgfsetstrokeopacity{0.500000}%
\pgfsetdash{}{0pt}%
\pgfpathmoveto{\pgfqpoint{4.230090in}{1.046674in}}%
\pgfusepath{stroke}%
\end{pgfscope}%
\begin{pgfscope}%
\pgfpathrectangle{\pgfqpoint{0.100000in}{0.100000in}}{\pgfqpoint{5.307240in}{3.397500in}}%
\pgfusepath{clip}%
\pgfsetbuttcap%
\pgfsetroundjoin%
\definecolor{currentfill}{rgb}{0.678431,1.000000,0.184314}%
\pgfsetfillcolor{currentfill}%
\pgfsetfillopacity{0.500000}%
\pgfsetlinewidth{0.250937pt}%
\definecolor{currentstroke}{rgb}{0.000000,0.000000,0.000000}%
\pgfsetstrokecolor{currentstroke}%
\pgfsetstrokeopacity{0.500000}%
\pgfsetdash{}{0pt}%
\pgfsys@defobject{currentmarker}{\pgfqpoint{-0.047222in}{-0.047222in}}{\pgfqpoint{0.047222in}{0.047222in}}{%
\pgfpathmoveto{\pgfqpoint{0.000000in}{-0.047222in}}%
\pgfpathcurveto{\pgfqpoint{0.012523in}{-0.047222in}}{\pgfqpoint{0.024536in}{-0.042247in}}{\pgfqpoint{0.033391in}{-0.033391in}}%
\pgfpathcurveto{\pgfqpoint{0.042247in}{-0.024536in}}{\pgfqpoint{0.047222in}{-0.012523in}}{\pgfqpoint{0.047222in}{0.000000in}}%
\pgfpathcurveto{\pgfqpoint{0.047222in}{0.012523in}}{\pgfqpoint{0.042247in}{0.024536in}}{\pgfqpoint{0.033391in}{0.033391in}}%
\pgfpathcurveto{\pgfqpoint{0.024536in}{0.042247in}}{\pgfqpoint{0.012523in}{0.047222in}}{\pgfqpoint{0.000000in}{0.047222in}}%
\pgfpathcurveto{\pgfqpoint{-0.012523in}{0.047222in}}{\pgfqpoint{-0.024536in}{0.042247in}}{\pgfqpoint{-0.033391in}{0.033391in}}%
\pgfpathcurveto{\pgfqpoint{-0.042247in}{0.024536in}}{\pgfqpoint{-0.047222in}{0.012523in}}{\pgfqpoint{-0.047222in}{0.000000in}}%
\pgfpathcurveto{\pgfqpoint{-0.047222in}{-0.012523in}}{\pgfqpoint{-0.042247in}{-0.024536in}}{\pgfqpoint{-0.033391in}{-0.033391in}}%
\pgfpathcurveto{\pgfqpoint{-0.024536in}{-0.042247in}}{\pgfqpoint{-0.012523in}{-0.047222in}}{\pgfqpoint{0.000000in}{-0.047222in}}%
\pgfpathclose%
\pgfusepath{stroke,fill}%
}%
\begin{pgfscope}%
\pgfsys@transformshift{4.230090in}{1.046674in}%
\pgfsys@useobject{currentmarker}{}%
\end{pgfscope}%
\end{pgfscope}%
\begin{pgfscope}%
\pgfpathrectangle{\pgfqpoint{0.100000in}{0.100000in}}{\pgfqpoint{5.307240in}{3.397500in}}%
\pgfusepath{clip}%
\pgfsetrectcap%
\pgfsetroundjoin%
\pgfsetlinewidth{1.505625pt}%
\definecolor{currentstroke}{rgb}{0.678431,1.000000,0.184314}%
\pgfsetstrokecolor{currentstroke}%
\pgfsetstrokeopacity{0.500000}%
\pgfsetdash{}{0pt}%
\pgfpathmoveto{\pgfqpoint{4.169536in}{1.247532in}}%
\pgfusepath{stroke}%
\end{pgfscope}%
\begin{pgfscope}%
\pgfpathrectangle{\pgfqpoint{0.100000in}{0.100000in}}{\pgfqpoint{5.307240in}{3.397500in}}%
\pgfusepath{clip}%
\pgfsetbuttcap%
\pgfsetroundjoin%
\definecolor{currentfill}{rgb}{0.678431,1.000000,0.184314}%
\pgfsetfillcolor{currentfill}%
\pgfsetfillopacity{0.500000}%
\pgfsetlinewidth{0.250937pt}%
\definecolor{currentstroke}{rgb}{0.000000,0.000000,0.000000}%
\pgfsetstrokecolor{currentstroke}%
\pgfsetstrokeopacity{0.500000}%
\pgfsetdash{}{0pt}%
\pgfsys@defobject{currentmarker}{\pgfqpoint{-0.048611in}{-0.048611in}}{\pgfqpoint{0.048611in}{0.048611in}}{%
\pgfpathmoveto{\pgfqpoint{0.000000in}{-0.048611in}}%
\pgfpathcurveto{\pgfqpoint{0.012892in}{-0.048611in}}{\pgfqpoint{0.025257in}{-0.043489in}}{\pgfqpoint{0.034373in}{-0.034373in}}%
\pgfpathcurveto{\pgfqpoint{0.043489in}{-0.025257in}}{\pgfqpoint{0.048611in}{-0.012892in}}{\pgfqpoint{0.048611in}{0.000000in}}%
\pgfpathcurveto{\pgfqpoint{0.048611in}{0.012892in}}{\pgfqpoint{0.043489in}{0.025257in}}{\pgfqpoint{0.034373in}{0.034373in}}%
\pgfpathcurveto{\pgfqpoint{0.025257in}{0.043489in}}{\pgfqpoint{0.012892in}{0.048611in}}{\pgfqpoint{0.000000in}{0.048611in}}%
\pgfpathcurveto{\pgfqpoint{-0.012892in}{0.048611in}}{\pgfqpoint{-0.025257in}{0.043489in}}{\pgfqpoint{-0.034373in}{0.034373in}}%
\pgfpathcurveto{\pgfqpoint{-0.043489in}{0.025257in}}{\pgfqpoint{-0.048611in}{0.012892in}}{\pgfqpoint{-0.048611in}{0.000000in}}%
\pgfpathcurveto{\pgfqpoint{-0.048611in}{-0.012892in}}{\pgfqpoint{-0.043489in}{-0.025257in}}{\pgfqpoint{-0.034373in}{-0.034373in}}%
\pgfpathcurveto{\pgfqpoint{-0.025257in}{-0.043489in}}{\pgfqpoint{-0.012892in}{-0.048611in}}{\pgfqpoint{0.000000in}{-0.048611in}}%
\pgfpathclose%
\pgfusepath{stroke,fill}%
}%
\begin{pgfscope}%
\pgfsys@transformshift{4.169536in}{1.247532in}%
\pgfsys@useobject{currentmarker}{}%
\end{pgfscope}%
\end{pgfscope}%
\begin{pgfscope}%
\pgfpathrectangle{\pgfqpoint{0.100000in}{0.100000in}}{\pgfqpoint{5.307240in}{3.397500in}}%
\pgfusepath{clip}%
\pgfsetrectcap%
\pgfsetroundjoin%
\pgfsetlinewidth{1.505625pt}%
\definecolor{currentstroke}{rgb}{0.678431,1.000000,0.184314}%
\pgfsetstrokecolor{currentstroke}%
\pgfsetstrokeopacity{0.500000}%
\pgfsetdash{}{0pt}%
\pgfpathmoveto{\pgfqpoint{2.038698in}{0.590797in}}%
\pgfusepath{stroke}%
\end{pgfscope}%
\begin{pgfscope}%
\pgfpathrectangle{\pgfqpoint{0.100000in}{0.100000in}}{\pgfqpoint{5.307240in}{3.397500in}}%
\pgfusepath{clip}%
\pgfsetbuttcap%
\pgfsetroundjoin%
\definecolor{currentfill}{rgb}{0.678431,1.000000,0.184314}%
\pgfsetfillcolor{currentfill}%
\pgfsetfillopacity{0.500000}%
\pgfsetlinewidth{0.250937pt}%
\definecolor{currentstroke}{rgb}{0.000000,0.000000,0.000000}%
\pgfsetstrokecolor{currentstroke}%
\pgfsetstrokeopacity{0.500000}%
\pgfsetdash{}{0pt}%
\pgfsys@defobject{currentmarker}{\pgfqpoint{-0.225694in}{-0.225694in}}{\pgfqpoint{0.225694in}{0.225694in}}{%
\pgfpathmoveto{\pgfqpoint{0.000000in}{-0.225694in}}%
\pgfpathcurveto{\pgfqpoint{0.059855in}{-0.225694in}}{\pgfqpoint{0.117266in}{-0.201914in}}{\pgfqpoint{0.159590in}{-0.159590in}}%
\pgfpathcurveto{\pgfqpoint{0.201914in}{-0.117266in}}{\pgfqpoint{0.225694in}{-0.059855in}}{\pgfqpoint{0.225694in}{0.000000in}}%
\pgfpathcurveto{\pgfqpoint{0.225694in}{0.059855in}}{\pgfqpoint{0.201914in}{0.117266in}}{\pgfqpoint{0.159590in}{0.159590in}}%
\pgfpathcurveto{\pgfqpoint{0.117266in}{0.201914in}}{\pgfqpoint{0.059855in}{0.225694in}}{\pgfqpoint{0.000000in}{0.225694in}}%
\pgfpathcurveto{\pgfqpoint{-0.059855in}{0.225694in}}{\pgfqpoint{-0.117266in}{0.201914in}}{\pgfqpoint{-0.159590in}{0.159590in}}%
\pgfpathcurveto{\pgfqpoint{-0.201914in}{0.117266in}}{\pgfqpoint{-0.225694in}{0.059855in}}{\pgfqpoint{-0.225694in}{0.000000in}}%
\pgfpathcurveto{\pgfqpoint{-0.225694in}{-0.059855in}}{\pgfqpoint{-0.201914in}{-0.117266in}}{\pgfqpoint{-0.159590in}{-0.159590in}}%
\pgfpathcurveto{\pgfqpoint{-0.117266in}{-0.201914in}}{\pgfqpoint{-0.059855in}{-0.225694in}}{\pgfqpoint{0.000000in}{-0.225694in}}%
\pgfpathclose%
\pgfusepath{stroke,fill}%
}%
\begin{pgfscope}%
\pgfsys@transformshift{2.038698in}{0.590797in}%
\pgfsys@useobject{currentmarker}{}%
\end{pgfscope}%
\end{pgfscope}%
\begin{pgfscope}%
\pgfpathrectangle{\pgfqpoint{0.100000in}{0.100000in}}{\pgfqpoint{5.307240in}{3.397500in}}%
\pgfusepath{clip}%
\pgfsetrectcap%
\pgfsetroundjoin%
\pgfsetlinewidth{1.505625pt}%
\definecolor{currentstroke}{rgb}{0.678431,1.000000,0.184314}%
\pgfsetstrokecolor{currentstroke}%
\pgfsetstrokeopacity{0.500000}%
\pgfsetdash{}{0pt}%
\pgfpathmoveto{\pgfqpoint{1.951431in}{0.730510in}}%
\pgfusepath{stroke}%
\end{pgfscope}%
\begin{pgfscope}%
\pgfpathrectangle{\pgfqpoint{0.100000in}{0.100000in}}{\pgfqpoint{5.307240in}{3.397500in}}%
\pgfusepath{clip}%
\pgfsetbuttcap%
\pgfsetroundjoin%
\definecolor{currentfill}{rgb}{0.678431,1.000000,0.184314}%
\pgfsetfillcolor{currentfill}%
\pgfsetfillopacity{0.500000}%
\pgfsetlinewidth{0.250937pt}%
\definecolor{currentstroke}{rgb}{0.000000,0.000000,0.000000}%
\pgfsetstrokecolor{currentstroke}%
\pgfsetstrokeopacity{0.500000}%
\pgfsetdash{}{0pt}%
\pgfsys@defobject{currentmarker}{\pgfqpoint{-0.121528in}{-0.121528in}}{\pgfqpoint{0.121528in}{0.121528in}}{%
\pgfpathmoveto{\pgfqpoint{0.000000in}{-0.121528in}}%
\pgfpathcurveto{\pgfqpoint{0.032230in}{-0.121528in}}{\pgfqpoint{0.063143in}{-0.108723in}}{\pgfqpoint{0.085933in}{-0.085933in}}%
\pgfpathcurveto{\pgfqpoint{0.108723in}{-0.063143in}}{\pgfqpoint{0.121528in}{-0.032230in}}{\pgfqpoint{0.121528in}{0.000000in}}%
\pgfpathcurveto{\pgfqpoint{0.121528in}{0.032230in}}{\pgfqpoint{0.108723in}{0.063143in}}{\pgfqpoint{0.085933in}{0.085933in}}%
\pgfpathcurveto{\pgfqpoint{0.063143in}{0.108723in}}{\pgfqpoint{0.032230in}{0.121528in}}{\pgfqpoint{0.000000in}{0.121528in}}%
\pgfpathcurveto{\pgfqpoint{-0.032230in}{0.121528in}}{\pgfqpoint{-0.063143in}{0.108723in}}{\pgfqpoint{-0.085933in}{0.085933in}}%
\pgfpathcurveto{\pgfqpoint{-0.108723in}{0.063143in}}{\pgfqpoint{-0.121528in}{0.032230in}}{\pgfqpoint{-0.121528in}{0.000000in}}%
\pgfpathcurveto{\pgfqpoint{-0.121528in}{-0.032230in}}{\pgfqpoint{-0.108723in}{-0.063143in}}{\pgfqpoint{-0.085933in}{-0.085933in}}%
\pgfpathcurveto{\pgfqpoint{-0.063143in}{-0.108723in}}{\pgfqpoint{-0.032230in}{-0.121528in}}{\pgfqpoint{0.000000in}{-0.121528in}}%
\pgfpathclose%
\pgfusepath{stroke,fill}%
}%
\begin{pgfscope}%
\pgfsys@transformshift{1.951431in}{0.730510in}%
\pgfsys@useobject{currentmarker}{}%
\end{pgfscope}%
\end{pgfscope}%
\begin{pgfscope}%
\pgfpathrectangle{\pgfqpoint{0.100000in}{0.100000in}}{\pgfqpoint{5.307240in}{3.397500in}}%
\pgfusepath{clip}%
\pgfsetrectcap%
\pgfsetroundjoin%
\pgfsetlinewidth{1.505625pt}%
\definecolor{currentstroke}{rgb}{0.678431,1.000000,0.184314}%
\pgfsetstrokecolor{currentstroke}%
\pgfsetstrokeopacity{0.500000}%
\pgfsetdash{}{0pt}%
\pgfpathmoveto{\pgfqpoint{1.273742in}{2.667152in}}%
\pgfusepath{stroke}%
\end{pgfscope}%
\begin{pgfscope}%
\pgfpathrectangle{\pgfqpoint{0.100000in}{0.100000in}}{\pgfqpoint{5.307240in}{3.397500in}}%
\pgfusepath{clip}%
\pgfsetbuttcap%
\pgfsetroundjoin%
\definecolor{currentfill}{rgb}{0.678431,1.000000,0.184314}%
\pgfsetfillcolor{currentfill}%
\pgfsetfillopacity{0.500000}%
\pgfsetlinewidth{0.250937pt}%
\definecolor{currentstroke}{rgb}{0.000000,0.000000,0.000000}%
\pgfsetstrokecolor{currentstroke}%
\pgfsetstrokeopacity{0.500000}%
\pgfsetdash{}{0pt}%
\pgfsys@defobject{currentmarker}{\pgfqpoint{-0.066667in}{-0.066667in}}{\pgfqpoint{0.066667in}{0.066667in}}{%
\pgfpathmoveto{\pgfqpoint{0.000000in}{-0.066667in}}%
\pgfpathcurveto{\pgfqpoint{0.017680in}{-0.066667in}}{\pgfqpoint{0.034639in}{-0.059642in}}{\pgfqpoint{0.047140in}{-0.047140in}}%
\pgfpathcurveto{\pgfqpoint{0.059642in}{-0.034639in}}{\pgfqpoint{0.066667in}{-0.017680in}}{\pgfqpoint{0.066667in}{0.000000in}}%
\pgfpathcurveto{\pgfqpoint{0.066667in}{0.017680in}}{\pgfqpoint{0.059642in}{0.034639in}}{\pgfqpoint{0.047140in}{0.047140in}}%
\pgfpathcurveto{\pgfqpoint{0.034639in}{0.059642in}}{\pgfqpoint{0.017680in}{0.066667in}}{\pgfqpoint{0.000000in}{0.066667in}}%
\pgfpathcurveto{\pgfqpoint{-0.017680in}{0.066667in}}{\pgfqpoint{-0.034639in}{0.059642in}}{\pgfqpoint{-0.047140in}{0.047140in}}%
\pgfpathcurveto{\pgfqpoint{-0.059642in}{0.034639in}}{\pgfqpoint{-0.066667in}{0.017680in}}{\pgfqpoint{-0.066667in}{0.000000in}}%
\pgfpathcurveto{\pgfqpoint{-0.066667in}{-0.017680in}}{\pgfqpoint{-0.059642in}{-0.034639in}}{\pgfqpoint{-0.047140in}{-0.047140in}}%
\pgfpathcurveto{\pgfqpoint{-0.034639in}{-0.059642in}}{\pgfqpoint{-0.017680in}{-0.066667in}}{\pgfqpoint{0.000000in}{-0.066667in}}%
\pgfpathclose%
\pgfusepath{stroke,fill}%
}%
\begin{pgfscope}%
\pgfsys@transformshift{1.273742in}{2.667152in}%
\pgfsys@useobject{currentmarker}{}%
\end{pgfscope}%
\end{pgfscope}%
\begin{pgfscope}%
\pgfpathrectangle{\pgfqpoint{0.100000in}{0.100000in}}{\pgfqpoint{5.307240in}{3.397500in}}%
\pgfusepath{clip}%
\pgfsetrectcap%
\pgfsetroundjoin%
\pgfsetlinewidth{1.505625pt}%
\definecolor{currentstroke}{rgb}{0.678431,1.000000,0.184314}%
\pgfsetstrokecolor{currentstroke}%
\pgfsetstrokeopacity{0.500000}%
\pgfsetdash{}{0pt}%
\pgfpathmoveto{\pgfqpoint{1.342988in}{3.141645in}}%
\pgfusepath{stroke}%
\end{pgfscope}%
\begin{pgfscope}%
\pgfpathrectangle{\pgfqpoint{0.100000in}{0.100000in}}{\pgfqpoint{5.307240in}{3.397500in}}%
\pgfusepath{clip}%
\pgfsetbuttcap%
\pgfsetroundjoin%
\definecolor{currentfill}{rgb}{0.678431,1.000000,0.184314}%
\pgfsetfillcolor{currentfill}%
\pgfsetfillopacity{0.500000}%
\pgfsetlinewidth{0.250937pt}%
\definecolor{currentstroke}{rgb}{0.000000,0.000000,0.000000}%
\pgfsetstrokecolor{currentstroke}%
\pgfsetstrokeopacity{0.500000}%
\pgfsetdash{}{0pt}%
\pgfsys@defobject{currentmarker}{\pgfqpoint{-0.093056in}{-0.093056in}}{\pgfqpoint{0.093056in}{0.093056in}}{%
\pgfpathmoveto{\pgfqpoint{0.000000in}{-0.093056in}}%
\pgfpathcurveto{\pgfqpoint{0.024679in}{-0.093056in}}{\pgfqpoint{0.048350in}{-0.083251in}}{\pgfqpoint{0.065800in}{-0.065800in}}%
\pgfpathcurveto{\pgfqpoint{0.083251in}{-0.048350in}}{\pgfqpoint{0.093056in}{-0.024679in}}{\pgfqpoint{0.093056in}{0.000000in}}%
\pgfpathcurveto{\pgfqpoint{0.093056in}{0.024679in}}{\pgfqpoint{0.083251in}{0.048350in}}{\pgfqpoint{0.065800in}{0.065800in}}%
\pgfpathcurveto{\pgfqpoint{0.048350in}{0.083251in}}{\pgfqpoint{0.024679in}{0.093056in}}{\pgfqpoint{0.000000in}{0.093056in}}%
\pgfpathcurveto{\pgfqpoint{-0.024679in}{0.093056in}}{\pgfqpoint{-0.048350in}{0.083251in}}{\pgfqpoint{-0.065800in}{0.065800in}}%
\pgfpathcurveto{\pgfqpoint{-0.083251in}{0.048350in}}{\pgfqpoint{-0.093056in}{0.024679in}}{\pgfqpoint{-0.093056in}{0.000000in}}%
\pgfpathcurveto{\pgfqpoint{-0.093056in}{-0.024679in}}{\pgfqpoint{-0.083251in}{-0.048350in}}{\pgfqpoint{-0.065800in}{-0.065800in}}%
\pgfpathcurveto{\pgfqpoint{-0.048350in}{-0.083251in}}{\pgfqpoint{-0.024679in}{-0.093056in}}{\pgfqpoint{0.000000in}{-0.093056in}}%
\pgfpathclose%
\pgfusepath{stroke,fill}%
}%
\begin{pgfscope}%
\pgfsys@transformshift{1.342988in}{3.141645in}%
\pgfsys@useobject{currentmarker}{}%
\end{pgfscope}%
\end{pgfscope}%
\begin{pgfscope}%
\pgfpathrectangle{\pgfqpoint{0.100000in}{0.100000in}}{\pgfqpoint{5.307240in}{3.397500in}}%
\pgfusepath{clip}%
\pgfsetrectcap%
\pgfsetroundjoin%
\pgfsetlinewidth{1.505625pt}%
\definecolor{currentstroke}{rgb}{0.678431,1.000000,0.184314}%
\pgfsetstrokecolor{currentstroke}%
\pgfsetstrokeopacity{0.500000}%
\pgfsetdash{}{0pt}%
\pgfpathmoveto{\pgfqpoint{1.618551in}{2.580876in}}%
\pgfusepath{stroke}%
\end{pgfscope}%
\begin{pgfscope}%
\pgfpathrectangle{\pgfqpoint{0.100000in}{0.100000in}}{\pgfqpoint{5.307240in}{3.397500in}}%
\pgfusepath{clip}%
\pgfsetbuttcap%
\pgfsetroundjoin%
\definecolor{currentfill}{rgb}{0.678431,1.000000,0.184314}%
\pgfsetfillcolor{currentfill}%
\pgfsetfillopacity{0.500000}%
\pgfsetlinewidth{0.250937pt}%
\definecolor{currentstroke}{rgb}{0.000000,0.000000,0.000000}%
\pgfsetstrokecolor{currentstroke}%
\pgfsetstrokeopacity{0.500000}%
\pgfsetdash{}{0pt}%
\pgfsys@defobject{currentmarker}{\pgfqpoint{-0.038194in}{-0.038194in}}{\pgfqpoint{0.038194in}{0.038194in}}{%
\pgfpathmoveto{\pgfqpoint{0.000000in}{-0.038194in}}%
\pgfpathcurveto{\pgfqpoint{0.010129in}{-0.038194in}}{\pgfqpoint{0.019845in}{-0.034170in}}{\pgfqpoint{0.027008in}{-0.027008in}}%
\pgfpathcurveto{\pgfqpoint{0.034170in}{-0.019845in}}{\pgfqpoint{0.038194in}{-0.010129in}}{\pgfqpoint{0.038194in}{0.000000in}}%
\pgfpathcurveto{\pgfqpoint{0.038194in}{0.010129in}}{\pgfqpoint{0.034170in}{0.019845in}}{\pgfqpoint{0.027008in}{0.027008in}}%
\pgfpathcurveto{\pgfqpoint{0.019845in}{0.034170in}}{\pgfqpoint{0.010129in}{0.038194in}}{\pgfqpoint{0.000000in}{0.038194in}}%
\pgfpathcurveto{\pgfqpoint{-0.010129in}{0.038194in}}{\pgfqpoint{-0.019845in}{0.034170in}}{\pgfqpoint{-0.027008in}{0.027008in}}%
\pgfpathcurveto{\pgfqpoint{-0.034170in}{0.019845in}}{\pgfqpoint{-0.038194in}{0.010129in}}{\pgfqpoint{-0.038194in}{0.000000in}}%
\pgfpathcurveto{\pgfqpoint{-0.038194in}{-0.010129in}}{\pgfqpoint{-0.034170in}{-0.019845in}}{\pgfqpoint{-0.027008in}{-0.027008in}}%
\pgfpathcurveto{\pgfqpoint{-0.019845in}{-0.034170in}}{\pgfqpoint{-0.010129in}{-0.038194in}}{\pgfqpoint{0.000000in}{-0.038194in}}%
\pgfpathclose%
\pgfusepath{stroke,fill}%
}%
\begin{pgfscope}%
\pgfsys@transformshift{1.618551in}{2.580876in}%
\pgfsys@useobject{currentmarker}{}%
\end{pgfscope}%
\end{pgfscope}%
\begin{pgfscope}%
\pgfpathrectangle{\pgfqpoint{0.100000in}{0.100000in}}{\pgfqpoint{5.307240in}{3.397500in}}%
\pgfusepath{clip}%
\pgfsetrectcap%
\pgfsetroundjoin%
\pgfsetlinewidth{1.505625pt}%
\definecolor{currentstroke}{rgb}{0.678431,1.000000,0.184314}%
\pgfsetstrokecolor{currentstroke}%
\pgfsetstrokeopacity{0.500000}%
\pgfsetdash{}{0pt}%
\pgfpathmoveto{\pgfqpoint{1.289215in}{3.002357in}}%
\pgfusepath{stroke}%
\end{pgfscope}%
\begin{pgfscope}%
\pgfpathrectangle{\pgfqpoint{0.100000in}{0.100000in}}{\pgfqpoint{5.307240in}{3.397500in}}%
\pgfusepath{clip}%
\pgfsetbuttcap%
\pgfsetroundjoin%
\definecolor{currentfill}{rgb}{0.678431,1.000000,0.184314}%
\pgfsetfillcolor{currentfill}%
\pgfsetfillopacity{0.500000}%
\pgfsetlinewidth{0.250937pt}%
\definecolor{currentstroke}{rgb}{0.000000,0.000000,0.000000}%
\pgfsetstrokecolor{currentstroke}%
\pgfsetstrokeopacity{0.500000}%
\pgfsetdash{}{0pt}%
\pgfsys@defobject{currentmarker}{\pgfqpoint{-0.052778in}{-0.052778in}}{\pgfqpoint{0.052778in}{0.052778in}}{%
\pgfpathmoveto{\pgfqpoint{0.000000in}{-0.052778in}}%
\pgfpathcurveto{\pgfqpoint{0.013997in}{-0.052778in}}{\pgfqpoint{0.027422in}{-0.047217in}}{\pgfqpoint{0.037320in}{-0.037320in}}%
\pgfpathcurveto{\pgfqpoint{0.047217in}{-0.027422in}}{\pgfqpoint{0.052778in}{-0.013997in}}{\pgfqpoint{0.052778in}{0.000000in}}%
\pgfpathcurveto{\pgfqpoint{0.052778in}{0.013997in}}{\pgfqpoint{0.047217in}{0.027422in}}{\pgfqpoint{0.037320in}{0.037320in}}%
\pgfpathcurveto{\pgfqpoint{0.027422in}{0.047217in}}{\pgfqpoint{0.013997in}{0.052778in}}{\pgfqpoint{0.000000in}{0.052778in}}%
\pgfpathcurveto{\pgfqpoint{-0.013997in}{0.052778in}}{\pgfqpoint{-0.027422in}{0.047217in}}{\pgfqpoint{-0.037320in}{0.037320in}}%
\pgfpathcurveto{\pgfqpoint{-0.047217in}{0.027422in}}{\pgfqpoint{-0.052778in}{0.013997in}}{\pgfqpoint{-0.052778in}{0.000000in}}%
\pgfpathcurveto{\pgfqpoint{-0.052778in}{-0.013997in}}{\pgfqpoint{-0.047217in}{-0.027422in}}{\pgfqpoint{-0.037320in}{-0.037320in}}%
\pgfpathcurveto{\pgfqpoint{-0.027422in}{-0.047217in}}{\pgfqpoint{-0.013997in}{-0.052778in}}{\pgfqpoint{0.000000in}{-0.052778in}}%
\pgfpathclose%
\pgfusepath{stroke,fill}%
}%
\begin{pgfscope}%
\pgfsys@transformshift{1.289215in}{3.002357in}%
\pgfsys@useobject{currentmarker}{}%
\end{pgfscope}%
\end{pgfscope}%
\begin{pgfscope}%
\pgfpathrectangle{\pgfqpoint{0.100000in}{0.100000in}}{\pgfqpoint{5.307240in}{3.397500in}}%
\pgfusepath{clip}%
\pgfsetrectcap%
\pgfsetroundjoin%
\pgfsetlinewidth{1.505625pt}%
\definecolor{currentstroke}{rgb}{0.678431,1.000000,0.184314}%
\pgfsetstrokecolor{currentstroke}%
\pgfsetstrokeopacity{0.500000}%
\pgfsetdash{}{0pt}%
\pgfpathmoveto{\pgfqpoint{1.571434in}{2.515786in}}%
\pgfusepath{stroke}%
\end{pgfscope}%
\begin{pgfscope}%
\pgfpathrectangle{\pgfqpoint{0.100000in}{0.100000in}}{\pgfqpoint{5.307240in}{3.397500in}}%
\pgfusepath{clip}%
\pgfsetbuttcap%
\pgfsetroundjoin%
\definecolor{currentfill}{rgb}{0.678431,1.000000,0.184314}%
\pgfsetfillcolor{currentfill}%
\pgfsetfillopacity{0.500000}%
\pgfsetlinewidth{0.250937pt}%
\definecolor{currentstroke}{rgb}{0.000000,0.000000,0.000000}%
\pgfsetstrokecolor{currentstroke}%
\pgfsetstrokeopacity{0.500000}%
\pgfsetdash{}{0pt}%
\pgfsys@defobject{currentmarker}{\pgfqpoint{-0.050000in}{-0.050000in}}{\pgfqpoint{0.050000in}{0.050000in}}{%
\pgfpathmoveto{\pgfqpoint{0.000000in}{-0.050000in}}%
\pgfpathcurveto{\pgfqpoint{0.013260in}{-0.050000in}}{\pgfqpoint{0.025979in}{-0.044732in}}{\pgfqpoint{0.035355in}{-0.035355in}}%
\pgfpathcurveto{\pgfqpoint{0.044732in}{-0.025979in}}{\pgfqpoint{0.050000in}{-0.013260in}}{\pgfqpoint{0.050000in}{0.000000in}}%
\pgfpathcurveto{\pgfqpoint{0.050000in}{0.013260in}}{\pgfqpoint{0.044732in}{0.025979in}}{\pgfqpoint{0.035355in}{0.035355in}}%
\pgfpathcurveto{\pgfqpoint{0.025979in}{0.044732in}}{\pgfqpoint{0.013260in}{0.050000in}}{\pgfqpoint{0.000000in}{0.050000in}}%
\pgfpathcurveto{\pgfqpoint{-0.013260in}{0.050000in}}{\pgfqpoint{-0.025979in}{0.044732in}}{\pgfqpoint{-0.035355in}{0.035355in}}%
\pgfpathcurveto{\pgfqpoint{-0.044732in}{0.025979in}}{\pgfqpoint{-0.050000in}{0.013260in}}{\pgfqpoint{-0.050000in}{0.000000in}}%
\pgfpathcurveto{\pgfqpoint{-0.050000in}{-0.013260in}}{\pgfqpoint{-0.044732in}{-0.025979in}}{\pgfqpoint{-0.035355in}{-0.035355in}}%
\pgfpathcurveto{\pgfqpoint{-0.025979in}{-0.044732in}}{\pgfqpoint{-0.013260in}{-0.050000in}}{\pgfqpoint{0.000000in}{-0.050000in}}%
\pgfpathclose%
\pgfusepath{stroke,fill}%
}%
\begin{pgfscope}%
\pgfsys@transformshift{1.571434in}{2.515786in}%
\pgfsys@useobject{currentmarker}{}%
\end{pgfscope}%
\end{pgfscope}%
\begin{pgfscope}%
\pgfpathrectangle{\pgfqpoint{0.100000in}{0.100000in}}{\pgfqpoint{5.307240in}{3.397500in}}%
\pgfusepath{clip}%
\pgfsetrectcap%
\pgfsetroundjoin%
\pgfsetlinewidth{1.505625pt}%
\definecolor{currentstroke}{rgb}{0.678431,1.000000,0.184314}%
\pgfsetstrokecolor{currentstroke}%
\pgfsetstrokeopacity{0.500000}%
\pgfsetdash{}{0pt}%
\pgfpathmoveto{\pgfqpoint{3.582409in}{2.112062in}}%
\pgfusepath{stroke}%
\end{pgfscope}%
\begin{pgfscope}%
\pgfpathrectangle{\pgfqpoint{0.100000in}{0.100000in}}{\pgfqpoint{5.307240in}{3.397500in}}%
\pgfusepath{clip}%
\pgfsetbuttcap%
\pgfsetroundjoin%
\definecolor{currentfill}{rgb}{0.678431,1.000000,0.184314}%
\pgfsetfillcolor{currentfill}%
\pgfsetfillopacity{0.500000}%
\pgfsetlinewidth{0.250937pt}%
\definecolor{currentstroke}{rgb}{0.000000,0.000000,0.000000}%
\pgfsetstrokecolor{currentstroke}%
\pgfsetstrokeopacity{0.500000}%
\pgfsetdash{}{0pt}%
\pgfsys@defobject{currentmarker}{\pgfqpoint{-0.065972in}{-0.065972in}}{\pgfqpoint{0.065972in}{0.065972in}}{%
\pgfpathmoveto{\pgfqpoint{0.000000in}{-0.065972in}}%
\pgfpathcurveto{\pgfqpoint{0.017496in}{-0.065972in}}{\pgfqpoint{0.034278in}{-0.059021in}}{\pgfqpoint{0.046649in}{-0.046649in}}%
\pgfpathcurveto{\pgfqpoint{0.059021in}{-0.034278in}}{\pgfqpoint{0.065972in}{-0.017496in}}{\pgfqpoint{0.065972in}{0.000000in}}%
\pgfpathcurveto{\pgfqpoint{0.065972in}{0.017496in}}{\pgfqpoint{0.059021in}{0.034278in}}{\pgfqpoint{0.046649in}{0.046649in}}%
\pgfpathcurveto{\pgfqpoint{0.034278in}{0.059021in}}{\pgfqpoint{0.017496in}{0.065972in}}{\pgfqpoint{0.000000in}{0.065972in}}%
\pgfpathcurveto{\pgfqpoint{-0.017496in}{0.065972in}}{\pgfqpoint{-0.034278in}{0.059021in}}{\pgfqpoint{-0.046649in}{0.046649in}}%
\pgfpathcurveto{\pgfqpoint{-0.059021in}{0.034278in}}{\pgfqpoint{-0.065972in}{0.017496in}}{\pgfqpoint{-0.065972in}{0.000000in}}%
\pgfpathcurveto{\pgfqpoint{-0.065972in}{-0.017496in}}{\pgfqpoint{-0.059021in}{-0.034278in}}{\pgfqpoint{-0.046649in}{-0.046649in}}%
\pgfpathcurveto{\pgfqpoint{-0.034278in}{-0.059021in}}{\pgfqpoint{-0.017496in}{-0.065972in}}{\pgfqpoint{0.000000in}{-0.065972in}}%
\pgfpathclose%
\pgfusepath{stroke,fill}%
}%
\begin{pgfscope}%
\pgfsys@transformshift{3.582409in}{2.112062in}%
\pgfsys@useobject{currentmarker}{}%
\end{pgfscope}%
\end{pgfscope}%
\begin{pgfscope}%
\pgfpathrectangle{\pgfqpoint{0.100000in}{0.100000in}}{\pgfqpoint{5.307240in}{3.397500in}}%
\pgfusepath{clip}%
\pgfsetrectcap%
\pgfsetroundjoin%
\pgfsetlinewidth{1.505625pt}%
\definecolor{currentstroke}{rgb}{0.678431,1.000000,0.184314}%
\pgfsetstrokecolor{currentstroke}%
\pgfsetstrokeopacity{0.500000}%
\pgfsetdash{}{0pt}%
\pgfpathmoveto{\pgfqpoint{3.583068in}{1.792222in}}%
\pgfusepath{stroke}%
\end{pgfscope}%
\begin{pgfscope}%
\pgfpathrectangle{\pgfqpoint{0.100000in}{0.100000in}}{\pgfqpoint{5.307240in}{3.397500in}}%
\pgfusepath{clip}%
\pgfsetbuttcap%
\pgfsetroundjoin%
\definecolor{currentfill}{rgb}{0.678431,1.000000,0.184314}%
\pgfsetfillcolor{currentfill}%
\pgfsetfillopacity{0.500000}%
\pgfsetlinewidth{0.250937pt}%
\definecolor{currentstroke}{rgb}{0.000000,0.000000,0.000000}%
\pgfsetstrokecolor{currentstroke}%
\pgfsetstrokeopacity{0.500000}%
\pgfsetdash{}{0pt}%
\pgfsys@defobject{currentmarker}{\pgfqpoint{-0.095139in}{-0.095139in}}{\pgfqpoint{0.095139in}{0.095139in}}{%
\pgfpathmoveto{\pgfqpoint{0.000000in}{-0.095139in}}%
\pgfpathcurveto{\pgfqpoint{0.025231in}{-0.095139in}}{\pgfqpoint{0.049432in}{-0.085114in}}{\pgfqpoint{0.067273in}{-0.067273in}}%
\pgfpathcurveto{\pgfqpoint{0.085114in}{-0.049432in}}{\pgfqpoint{0.095139in}{-0.025231in}}{\pgfqpoint{0.095139in}{0.000000in}}%
\pgfpathcurveto{\pgfqpoint{0.095139in}{0.025231in}}{\pgfqpoint{0.085114in}{0.049432in}}{\pgfqpoint{0.067273in}{0.067273in}}%
\pgfpathcurveto{\pgfqpoint{0.049432in}{0.085114in}}{\pgfqpoint{0.025231in}{0.095139in}}{\pgfqpoint{0.000000in}{0.095139in}}%
\pgfpathcurveto{\pgfqpoint{-0.025231in}{0.095139in}}{\pgfqpoint{-0.049432in}{0.085114in}}{\pgfqpoint{-0.067273in}{0.067273in}}%
\pgfpathcurveto{\pgfqpoint{-0.085114in}{0.049432in}}{\pgfqpoint{-0.095139in}{0.025231in}}{\pgfqpoint{-0.095139in}{0.000000in}}%
\pgfpathcurveto{\pgfqpoint{-0.095139in}{-0.025231in}}{\pgfqpoint{-0.085114in}{-0.049432in}}{\pgfqpoint{-0.067273in}{-0.067273in}}%
\pgfpathcurveto{\pgfqpoint{-0.049432in}{-0.085114in}}{\pgfqpoint{-0.025231in}{-0.095139in}}{\pgfqpoint{0.000000in}{-0.095139in}}%
\pgfpathclose%
\pgfusepath{stroke,fill}%
}%
\begin{pgfscope}%
\pgfsys@transformshift{3.583068in}{1.792222in}%
\pgfsys@useobject{currentmarker}{}%
\end{pgfscope}%
\end{pgfscope}%
\begin{pgfscope}%
\pgfpathrectangle{\pgfqpoint{0.100000in}{0.100000in}}{\pgfqpoint{5.307240in}{3.397500in}}%
\pgfusepath{clip}%
\pgfsetrectcap%
\pgfsetroundjoin%
\pgfsetlinewidth{1.505625pt}%
\definecolor{currentstroke}{rgb}{0.678431,1.000000,0.184314}%
\pgfsetstrokecolor{currentstroke}%
\pgfsetstrokeopacity{0.500000}%
\pgfsetdash{}{0pt}%
\pgfpathmoveto{\pgfqpoint{3.651737in}{2.075362in}}%
\pgfusepath{stroke}%
\end{pgfscope}%
\begin{pgfscope}%
\pgfpathrectangle{\pgfqpoint{0.100000in}{0.100000in}}{\pgfqpoint{5.307240in}{3.397500in}}%
\pgfusepath{clip}%
\pgfsetbuttcap%
\pgfsetroundjoin%
\definecolor{currentfill}{rgb}{0.678431,1.000000,0.184314}%
\pgfsetfillcolor{currentfill}%
\pgfsetfillopacity{0.500000}%
\pgfsetlinewidth{0.250937pt}%
\definecolor{currentstroke}{rgb}{0.000000,0.000000,0.000000}%
\pgfsetstrokecolor{currentstroke}%
\pgfsetstrokeopacity{0.500000}%
\pgfsetdash{}{0pt}%
\pgfsys@defobject{currentmarker}{\pgfqpoint{-0.052778in}{-0.052778in}}{\pgfqpoint{0.052778in}{0.052778in}}{%
\pgfpathmoveto{\pgfqpoint{0.000000in}{-0.052778in}}%
\pgfpathcurveto{\pgfqpoint{0.013997in}{-0.052778in}}{\pgfqpoint{0.027422in}{-0.047217in}}{\pgfqpoint{0.037320in}{-0.037320in}}%
\pgfpathcurveto{\pgfqpoint{0.047217in}{-0.027422in}}{\pgfqpoint{0.052778in}{-0.013997in}}{\pgfqpoint{0.052778in}{0.000000in}}%
\pgfpathcurveto{\pgfqpoint{0.052778in}{0.013997in}}{\pgfqpoint{0.047217in}{0.027422in}}{\pgfqpoint{0.037320in}{0.037320in}}%
\pgfpathcurveto{\pgfqpoint{0.027422in}{0.047217in}}{\pgfqpoint{0.013997in}{0.052778in}}{\pgfqpoint{0.000000in}{0.052778in}}%
\pgfpathcurveto{\pgfqpoint{-0.013997in}{0.052778in}}{\pgfqpoint{-0.027422in}{0.047217in}}{\pgfqpoint{-0.037320in}{0.037320in}}%
\pgfpathcurveto{\pgfqpoint{-0.047217in}{0.027422in}}{\pgfqpoint{-0.052778in}{0.013997in}}{\pgfqpoint{-0.052778in}{0.000000in}}%
\pgfpathcurveto{\pgfqpoint{-0.052778in}{-0.013997in}}{\pgfqpoint{-0.047217in}{-0.027422in}}{\pgfqpoint{-0.037320in}{-0.037320in}}%
\pgfpathcurveto{\pgfqpoint{-0.027422in}{-0.047217in}}{\pgfqpoint{-0.013997in}{-0.052778in}}{\pgfqpoint{0.000000in}{-0.052778in}}%
\pgfpathclose%
\pgfusepath{stroke,fill}%
}%
\begin{pgfscope}%
\pgfsys@transformshift{3.651737in}{2.075362in}%
\pgfsys@useobject{currentmarker}{}%
\end{pgfscope}%
\end{pgfscope}%
\begin{pgfscope}%
\pgfpathrectangle{\pgfqpoint{0.100000in}{0.100000in}}{\pgfqpoint{5.307240in}{3.397500in}}%
\pgfusepath{clip}%
\pgfsetrectcap%
\pgfsetroundjoin%
\pgfsetlinewidth{1.505625pt}%
\definecolor{currentstroke}{rgb}{0.678431,1.000000,0.184314}%
\pgfsetstrokecolor{currentstroke}%
\pgfsetstrokeopacity{0.500000}%
\pgfsetdash{}{0pt}%
\pgfpathmoveto{\pgfqpoint{3.690017in}{2.283585in}}%
\pgfusepath{stroke}%
\end{pgfscope}%
\begin{pgfscope}%
\pgfpathrectangle{\pgfqpoint{0.100000in}{0.100000in}}{\pgfqpoint{5.307240in}{3.397500in}}%
\pgfusepath{clip}%
\pgfsetbuttcap%
\pgfsetroundjoin%
\definecolor{currentfill}{rgb}{0.678431,1.000000,0.184314}%
\pgfsetfillcolor{currentfill}%
\pgfsetfillopacity{0.500000}%
\pgfsetlinewidth{0.250937pt}%
\definecolor{currentstroke}{rgb}{0.000000,0.000000,0.000000}%
\pgfsetstrokecolor{currentstroke}%
\pgfsetstrokeopacity{0.500000}%
\pgfsetdash{}{0pt}%
\pgfsys@defobject{currentmarker}{\pgfqpoint{-0.096528in}{-0.096528in}}{\pgfqpoint{0.096528in}{0.096528in}}{%
\pgfpathmoveto{\pgfqpoint{0.000000in}{-0.096528in}}%
\pgfpathcurveto{\pgfqpoint{0.025599in}{-0.096528in}}{\pgfqpoint{0.050154in}{-0.086357in}}{\pgfqpoint{0.068255in}{-0.068255in}}%
\pgfpathcurveto{\pgfqpoint{0.086357in}{-0.050154in}}{\pgfqpoint{0.096528in}{-0.025599in}}{\pgfqpoint{0.096528in}{0.000000in}}%
\pgfpathcurveto{\pgfqpoint{0.096528in}{0.025599in}}{\pgfqpoint{0.086357in}{0.050154in}}{\pgfqpoint{0.068255in}{0.068255in}}%
\pgfpathcurveto{\pgfqpoint{0.050154in}{0.086357in}}{\pgfqpoint{0.025599in}{0.096528in}}{\pgfqpoint{0.000000in}{0.096528in}}%
\pgfpathcurveto{\pgfqpoint{-0.025599in}{0.096528in}}{\pgfqpoint{-0.050154in}{0.086357in}}{\pgfqpoint{-0.068255in}{0.068255in}}%
\pgfpathcurveto{\pgfqpoint{-0.086357in}{0.050154in}}{\pgfqpoint{-0.096528in}{0.025599in}}{\pgfqpoint{-0.096528in}{0.000000in}}%
\pgfpathcurveto{\pgfqpoint{-0.096528in}{-0.025599in}}{\pgfqpoint{-0.086357in}{-0.050154in}}{\pgfqpoint{-0.068255in}{-0.068255in}}%
\pgfpathcurveto{\pgfqpoint{-0.050154in}{-0.086357in}}{\pgfqpoint{-0.025599in}{-0.096528in}}{\pgfqpoint{0.000000in}{-0.096528in}}%
\pgfpathclose%
\pgfusepath{stroke,fill}%
}%
\begin{pgfscope}%
\pgfsys@transformshift{3.690017in}{2.283585in}%
\pgfsys@useobject{currentmarker}{}%
\end{pgfscope}%
\end{pgfscope}%
\begin{pgfscope}%
\pgfpathrectangle{\pgfqpoint{0.100000in}{0.100000in}}{\pgfqpoint{5.307240in}{3.397500in}}%
\pgfusepath{clip}%
\pgfsetrectcap%
\pgfsetroundjoin%
\pgfsetlinewidth{1.505625pt}%
\definecolor{currentstroke}{rgb}{0.678431,1.000000,0.184314}%
\pgfsetstrokecolor{currentstroke}%
\pgfsetstrokeopacity{0.500000}%
\pgfsetdash{}{0pt}%
\pgfpathmoveto{\pgfqpoint{3.705987in}{2.080613in}}%
\pgfusepath{stroke}%
\end{pgfscope}%
\begin{pgfscope}%
\pgfpathrectangle{\pgfqpoint{0.100000in}{0.100000in}}{\pgfqpoint{5.307240in}{3.397500in}}%
\pgfusepath{clip}%
\pgfsetbuttcap%
\pgfsetroundjoin%
\definecolor{currentfill}{rgb}{0.678431,1.000000,0.184314}%
\pgfsetfillcolor{currentfill}%
\pgfsetfillopacity{0.500000}%
\pgfsetlinewidth{0.250937pt}%
\definecolor{currentstroke}{rgb}{0.000000,0.000000,0.000000}%
\pgfsetstrokecolor{currentstroke}%
\pgfsetstrokeopacity{0.500000}%
\pgfsetdash{}{0pt}%
\pgfsys@defobject{currentmarker}{\pgfqpoint{-0.089583in}{-0.089583in}}{\pgfqpoint{0.089583in}{0.089583in}}{%
\pgfpathmoveto{\pgfqpoint{0.000000in}{-0.089583in}}%
\pgfpathcurveto{\pgfqpoint{0.023758in}{-0.089583in}}{\pgfqpoint{0.046546in}{-0.080144in}}{\pgfqpoint{0.063345in}{-0.063345in}}%
\pgfpathcurveto{\pgfqpoint{0.080144in}{-0.046546in}}{\pgfqpoint{0.089583in}{-0.023758in}}{\pgfqpoint{0.089583in}{0.000000in}}%
\pgfpathcurveto{\pgfqpoint{0.089583in}{0.023758in}}{\pgfqpoint{0.080144in}{0.046546in}}{\pgfqpoint{0.063345in}{0.063345in}}%
\pgfpathcurveto{\pgfqpoint{0.046546in}{0.080144in}}{\pgfqpoint{0.023758in}{0.089583in}}{\pgfqpoint{0.000000in}{0.089583in}}%
\pgfpathcurveto{\pgfqpoint{-0.023758in}{0.089583in}}{\pgfqpoint{-0.046546in}{0.080144in}}{\pgfqpoint{-0.063345in}{0.063345in}}%
\pgfpathcurveto{\pgfqpoint{-0.080144in}{0.046546in}}{\pgfqpoint{-0.089583in}{0.023758in}}{\pgfqpoint{-0.089583in}{0.000000in}}%
\pgfpathcurveto{\pgfqpoint{-0.089583in}{-0.023758in}}{\pgfqpoint{-0.080144in}{-0.046546in}}{\pgfqpoint{-0.063345in}{-0.063345in}}%
\pgfpathcurveto{\pgfqpoint{-0.046546in}{-0.080144in}}{\pgfqpoint{-0.023758in}{-0.089583in}}{\pgfqpoint{0.000000in}{-0.089583in}}%
\pgfpathclose%
\pgfusepath{stroke,fill}%
}%
\begin{pgfscope}%
\pgfsys@transformshift{3.705987in}{2.080613in}%
\pgfsys@useobject{currentmarker}{}%
\end{pgfscope}%
\end{pgfscope}%
\begin{pgfscope}%
\pgfpathrectangle{\pgfqpoint{0.100000in}{0.100000in}}{\pgfqpoint{5.307240in}{3.397500in}}%
\pgfusepath{clip}%
\pgfsetrectcap%
\pgfsetroundjoin%
\pgfsetlinewidth{1.505625pt}%
\definecolor{currentstroke}{rgb}{0.678431,1.000000,0.184314}%
\pgfsetstrokecolor{currentstroke}%
\pgfsetstrokeopacity{0.500000}%
\pgfsetdash{}{0pt}%
\pgfpathmoveto{\pgfqpoint{3.436624in}{2.226080in}}%
\pgfusepath{stroke}%
\end{pgfscope}%
\begin{pgfscope}%
\pgfpathrectangle{\pgfqpoint{0.100000in}{0.100000in}}{\pgfqpoint{5.307240in}{3.397500in}}%
\pgfusepath{clip}%
\pgfsetbuttcap%
\pgfsetroundjoin%
\definecolor{currentfill}{rgb}{0.678431,1.000000,0.184314}%
\pgfsetfillcolor{currentfill}%
\pgfsetfillopacity{0.500000}%
\pgfsetlinewidth{0.250937pt}%
\definecolor{currentstroke}{rgb}{0.000000,0.000000,0.000000}%
\pgfsetstrokecolor{currentstroke}%
\pgfsetstrokeopacity{0.500000}%
\pgfsetdash{}{0pt}%
\pgfsys@defobject{currentmarker}{\pgfqpoint{-0.080556in}{-0.080556in}}{\pgfqpoint{0.080556in}{0.080556in}}{%
\pgfpathmoveto{\pgfqpoint{0.000000in}{-0.080556in}}%
\pgfpathcurveto{\pgfqpoint{0.021364in}{-0.080556in}}{\pgfqpoint{0.041855in}{-0.072068in}}{\pgfqpoint{0.056961in}{-0.056961in}}%
\pgfpathcurveto{\pgfqpoint{0.072068in}{-0.041855in}}{\pgfqpoint{0.080556in}{-0.021364in}}{\pgfqpoint{0.080556in}{0.000000in}}%
\pgfpathcurveto{\pgfqpoint{0.080556in}{0.021364in}}{\pgfqpoint{0.072068in}{0.041855in}}{\pgfqpoint{0.056961in}{0.056961in}}%
\pgfpathcurveto{\pgfqpoint{0.041855in}{0.072068in}}{\pgfqpoint{0.021364in}{0.080556in}}{\pgfqpoint{0.000000in}{0.080556in}}%
\pgfpathcurveto{\pgfqpoint{-0.021364in}{0.080556in}}{\pgfqpoint{-0.041855in}{0.072068in}}{\pgfqpoint{-0.056961in}{0.056961in}}%
\pgfpathcurveto{\pgfqpoint{-0.072068in}{0.041855in}}{\pgfqpoint{-0.080556in}{0.021364in}}{\pgfqpoint{-0.080556in}{0.000000in}}%
\pgfpathcurveto{\pgfqpoint{-0.080556in}{-0.021364in}}{\pgfqpoint{-0.072068in}{-0.041855in}}{\pgfqpoint{-0.056961in}{-0.056961in}}%
\pgfpathcurveto{\pgfqpoint{-0.041855in}{-0.072068in}}{\pgfqpoint{-0.021364in}{-0.080556in}}{\pgfqpoint{0.000000in}{-0.080556in}}%
\pgfpathclose%
\pgfusepath{stroke,fill}%
}%
\begin{pgfscope}%
\pgfsys@transformshift{3.436624in}{2.226080in}%
\pgfsys@useobject{currentmarker}{}%
\end{pgfscope}%
\end{pgfscope}%
\begin{pgfscope}%
\pgfpathrectangle{\pgfqpoint{0.100000in}{0.100000in}}{\pgfqpoint{5.307240in}{3.397500in}}%
\pgfusepath{clip}%
\pgfsetrectcap%
\pgfsetroundjoin%
\pgfsetlinewidth{1.505625pt}%
\definecolor{currentstroke}{rgb}{0.678431,1.000000,0.184314}%
\pgfsetstrokecolor{currentstroke}%
\pgfsetstrokeopacity{0.500000}%
\pgfsetdash{}{0pt}%
\pgfpathmoveto{\pgfqpoint{3.590945in}{2.039486in}}%
\pgfusepath{stroke}%
\end{pgfscope}%
\begin{pgfscope}%
\pgfpathrectangle{\pgfqpoint{0.100000in}{0.100000in}}{\pgfqpoint{5.307240in}{3.397500in}}%
\pgfusepath{clip}%
\pgfsetbuttcap%
\pgfsetroundjoin%
\definecolor{currentfill}{rgb}{0.678431,1.000000,0.184314}%
\pgfsetfillcolor{currentfill}%
\pgfsetfillopacity{0.500000}%
\pgfsetlinewidth{0.250937pt}%
\definecolor{currentstroke}{rgb}{0.000000,0.000000,0.000000}%
\pgfsetstrokecolor{currentstroke}%
\pgfsetstrokeopacity{0.500000}%
\pgfsetdash{}{0pt}%
\pgfsys@defobject{currentmarker}{\pgfqpoint{-0.081250in}{-0.081250in}}{\pgfqpoint{0.081250in}{0.081250in}}{%
\pgfpathmoveto{\pgfqpoint{0.000000in}{-0.081250in}}%
\pgfpathcurveto{\pgfqpoint{0.021548in}{-0.081250in}}{\pgfqpoint{0.042216in}{-0.072689in}}{\pgfqpoint{0.057452in}{-0.057452in}}%
\pgfpathcurveto{\pgfqpoint{0.072689in}{-0.042216in}}{\pgfqpoint{0.081250in}{-0.021548in}}{\pgfqpoint{0.081250in}{0.000000in}}%
\pgfpathcurveto{\pgfqpoint{0.081250in}{0.021548in}}{\pgfqpoint{0.072689in}{0.042216in}}{\pgfqpoint{0.057452in}{0.057452in}}%
\pgfpathcurveto{\pgfqpoint{0.042216in}{0.072689in}}{\pgfqpoint{0.021548in}{0.081250in}}{\pgfqpoint{0.000000in}{0.081250in}}%
\pgfpathcurveto{\pgfqpoint{-0.021548in}{0.081250in}}{\pgfqpoint{-0.042216in}{0.072689in}}{\pgfqpoint{-0.057452in}{0.057452in}}%
\pgfpathcurveto{\pgfqpoint{-0.072689in}{0.042216in}}{\pgfqpoint{-0.081250in}{0.021548in}}{\pgfqpoint{-0.081250in}{0.000000in}}%
\pgfpathcurveto{\pgfqpoint{-0.081250in}{-0.021548in}}{\pgfqpoint{-0.072689in}{-0.042216in}}{\pgfqpoint{-0.057452in}{-0.057452in}}%
\pgfpathcurveto{\pgfqpoint{-0.042216in}{-0.072689in}}{\pgfqpoint{-0.021548in}{-0.081250in}}{\pgfqpoint{0.000000in}{-0.081250in}}%
\pgfpathclose%
\pgfusepath{stroke,fill}%
}%
\begin{pgfscope}%
\pgfsys@transformshift{3.590945in}{2.039486in}%
\pgfsys@useobject{currentmarker}{}%
\end{pgfscope}%
\end{pgfscope}%
\begin{pgfscope}%
\pgfpathrectangle{\pgfqpoint{0.100000in}{0.100000in}}{\pgfqpoint{5.307240in}{3.397500in}}%
\pgfusepath{clip}%
\pgfsetrectcap%
\pgfsetroundjoin%
\pgfsetlinewidth{1.505625pt}%
\definecolor{currentstroke}{rgb}{0.678431,1.000000,0.184314}%
\pgfsetstrokecolor{currentstroke}%
\pgfsetstrokeopacity{0.500000}%
\pgfsetdash{}{0pt}%
\pgfpathmoveto{\pgfqpoint{3.676465in}{2.194311in}}%
\pgfusepath{stroke}%
\end{pgfscope}%
\begin{pgfscope}%
\pgfpathrectangle{\pgfqpoint{0.100000in}{0.100000in}}{\pgfqpoint{5.307240in}{3.397500in}}%
\pgfusepath{clip}%
\pgfsetbuttcap%
\pgfsetroundjoin%
\definecolor{currentfill}{rgb}{0.678431,1.000000,0.184314}%
\pgfsetfillcolor{currentfill}%
\pgfsetfillopacity{0.500000}%
\pgfsetlinewidth{0.250937pt}%
\definecolor{currentstroke}{rgb}{0.000000,0.000000,0.000000}%
\pgfsetstrokecolor{currentstroke}%
\pgfsetstrokeopacity{0.500000}%
\pgfsetdash{}{0pt}%
\pgfsys@defobject{currentmarker}{\pgfqpoint{-0.079861in}{-0.079861in}}{\pgfqpoint{0.079861in}{0.079861in}}{%
\pgfpathmoveto{\pgfqpoint{0.000000in}{-0.079861in}}%
\pgfpathcurveto{\pgfqpoint{0.021179in}{-0.079861in}}{\pgfqpoint{0.041494in}{-0.071446in}}{\pgfqpoint{0.056470in}{-0.056470in}}%
\pgfpathcurveto{\pgfqpoint{0.071446in}{-0.041494in}}{\pgfqpoint{0.079861in}{-0.021179in}}{\pgfqpoint{0.079861in}{0.000000in}}%
\pgfpathcurveto{\pgfqpoint{0.079861in}{0.021179in}}{\pgfqpoint{0.071446in}{0.041494in}}{\pgfqpoint{0.056470in}{0.056470in}}%
\pgfpathcurveto{\pgfqpoint{0.041494in}{0.071446in}}{\pgfqpoint{0.021179in}{0.079861in}}{\pgfqpoint{0.000000in}{0.079861in}}%
\pgfpathcurveto{\pgfqpoint{-0.021179in}{0.079861in}}{\pgfqpoint{-0.041494in}{0.071446in}}{\pgfqpoint{-0.056470in}{0.056470in}}%
\pgfpathcurveto{\pgfqpoint{-0.071446in}{0.041494in}}{\pgfqpoint{-0.079861in}{0.021179in}}{\pgfqpoint{-0.079861in}{0.000000in}}%
\pgfpathcurveto{\pgfqpoint{-0.079861in}{-0.021179in}}{\pgfqpoint{-0.071446in}{-0.041494in}}{\pgfqpoint{-0.056470in}{-0.056470in}}%
\pgfpathcurveto{\pgfqpoint{-0.041494in}{-0.071446in}}{\pgfqpoint{-0.021179in}{-0.079861in}}{\pgfqpoint{0.000000in}{-0.079861in}}%
\pgfpathclose%
\pgfusepath{stroke,fill}%
}%
\begin{pgfscope}%
\pgfsys@transformshift{3.676465in}{2.194311in}%
\pgfsys@useobject{currentmarker}{}%
\end{pgfscope}%
\end{pgfscope}%
\begin{pgfscope}%
\pgfpathrectangle{\pgfqpoint{0.100000in}{0.100000in}}{\pgfqpoint{5.307240in}{3.397500in}}%
\pgfusepath{clip}%
\pgfsetrectcap%
\pgfsetroundjoin%
\pgfsetlinewidth{1.505625pt}%
\definecolor{currentstroke}{rgb}{0.678431,1.000000,0.184314}%
\pgfsetstrokecolor{currentstroke}%
\pgfsetstrokeopacity{0.500000}%
\pgfsetdash{}{0pt}%
\pgfpathmoveto{\pgfqpoint{3.528356in}{2.134384in}}%
\pgfusepath{stroke}%
\end{pgfscope}%
\begin{pgfscope}%
\pgfpathrectangle{\pgfqpoint{0.100000in}{0.100000in}}{\pgfqpoint{5.307240in}{3.397500in}}%
\pgfusepath{clip}%
\pgfsetbuttcap%
\pgfsetroundjoin%
\definecolor{currentfill}{rgb}{0.678431,1.000000,0.184314}%
\pgfsetfillcolor{currentfill}%
\pgfsetfillopacity{0.500000}%
\pgfsetlinewidth{0.250937pt}%
\definecolor{currentstroke}{rgb}{0.000000,0.000000,0.000000}%
\pgfsetstrokecolor{currentstroke}%
\pgfsetstrokeopacity{0.500000}%
\pgfsetdash{}{0pt}%
\pgfsys@defobject{currentmarker}{\pgfqpoint{-0.095139in}{-0.095139in}}{\pgfqpoint{0.095139in}{0.095139in}}{%
\pgfpathmoveto{\pgfqpoint{0.000000in}{-0.095139in}}%
\pgfpathcurveto{\pgfqpoint{0.025231in}{-0.095139in}}{\pgfqpoint{0.049432in}{-0.085114in}}{\pgfqpoint{0.067273in}{-0.067273in}}%
\pgfpathcurveto{\pgfqpoint{0.085114in}{-0.049432in}}{\pgfqpoint{0.095139in}{-0.025231in}}{\pgfqpoint{0.095139in}{0.000000in}}%
\pgfpathcurveto{\pgfqpoint{0.095139in}{0.025231in}}{\pgfqpoint{0.085114in}{0.049432in}}{\pgfqpoint{0.067273in}{0.067273in}}%
\pgfpathcurveto{\pgfqpoint{0.049432in}{0.085114in}}{\pgfqpoint{0.025231in}{0.095139in}}{\pgfqpoint{0.000000in}{0.095139in}}%
\pgfpathcurveto{\pgfqpoint{-0.025231in}{0.095139in}}{\pgfqpoint{-0.049432in}{0.085114in}}{\pgfqpoint{-0.067273in}{0.067273in}}%
\pgfpathcurveto{\pgfqpoint{-0.085114in}{0.049432in}}{\pgfqpoint{-0.095139in}{0.025231in}}{\pgfqpoint{-0.095139in}{0.000000in}}%
\pgfpathcurveto{\pgfqpoint{-0.095139in}{-0.025231in}}{\pgfqpoint{-0.085114in}{-0.049432in}}{\pgfqpoint{-0.067273in}{-0.067273in}}%
\pgfpathcurveto{\pgfqpoint{-0.049432in}{-0.085114in}}{\pgfqpoint{-0.025231in}{-0.095139in}}{\pgfqpoint{0.000000in}{-0.095139in}}%
\pgfpathclose%
\pgfusepath{stroke,fill}%
}%
\begin{pgfscope}%
\pgfsys@transformshift{3.528356in}{2.134384in}%
\pgfsys@useobject{currentmarker}{}%
\end{pgfscope}%
\end{pgfscope}%
\begin{pgfscope}%
\pgfpathrectangle{\pgfqpoint{0.100000in}{0.100000in}}{\pgfqpoint{5.307240in}{3.397500in}}%
\pgfusepath{clip}%
\pgfsetrectcap%
\pgfsetroundjoin%
\pgfsetlinewidth{1.505625pt}%
\definecolor{currentstroke}{rgb}{0.678431,1.000000,0.184314}%
\pgfsetstrokecolor{currentstroke}%
\pgfsetstrokeopacity{0.500000}%
\pgfsetdash{}{0pt}%
\pgfpathmoveto{\pgfqpoint{3.560010in}{2.320269in}}%
\pgfusepath{stroke}%
\end{pgfscope}%
\begin{pgfscope}%
\pgfpathrectangle{\pgfqpoint{0.100000in}{0.100000in}}{\pgfqpoint{5.307240in}{3.397500in}}%
\pgfusepath{clip}%
\pgfsetbuttcap%
\pgfsetroundjoin%
\definecolor{currentfill}{rgb}{0.678431,1.000000,0.184314}%
\pgfsetfillcolor{currentfill}%
\pgfsetfillopacity{0.500000}%
\pgfsetlinewidth{0.250937pt}%
\definecolor{currentstroke}{rgb}{0.000000,0.000000,0.000000}%
\pgfsetstrokecolor{currentstroke}%
\pgfsetstrokeopacity{0.500000}%
\pgfsetdash{}{0pt}%
\pgfsys@defobject{currentmarker}{\pgfqpoint{-0.124306in}{-0.124306in}}{\pgfqpoint{0.124306in}{0.124306in}}{%
\pgfpathmoveto{\pgfqpoint{0.000000in}{-0.124306in}}%
\pgfpathcurveto{\pgfqpoint{0.032966in}{-0.124306in}}{\pgfqpoint{0.064587in}{-0.111208in}}{\pgfqpoint{0.087897in}{-0.087897in}}%
\pgfpathcurveto{\pgfqpoint{0.111208in}{-0.064587in}}{\pgfqpoint{0.124306in}{-0.032966in}}{\pgfqpoint{0.124306in}{0.000000in}}%
\pgfpathcurveto{\pgfqpoint{0.124306in}{0.032966in}}{\pgfqpoint{0.111208in}{0.064587in}}{\pgfqpoint{0.087897in}{0.087897in}}%
\pgfpathcurveto{\pgfqpoint{0.064587in}{0.111208in}}{\pgfqpoint{0.032966in}{0.124306in}}{\pgfqpoint{0.000000in}{0.124306in}}%
\pgfpathcurveto{\pgfqpoint{-0.032966in}{0.124306in}}{\pgfqpoint{-0.064587in}{0.111208in}}{\pgfqpoint{-0.087897in}{0.087897in}}%
\pgfpathcurveto{\pgfqpoint{-0.111208in}{0.064587in}}{\pgfqpoint{-0.124306in}{0.032966in}}{\pgfqpoint{-0.124306in}{0.000000in}}%
\pgfpathcurveto{\pgfqpoint{-0.124306in}{-0.032966in}}{\pgfqpoint{-0.111208in}{-0.064587in}}{\pgfqpoint{-0.087897in}{-0.087897in}}%
\pgfpathcurveto{\pgfqpoint{-0.064587in}{-0.111208in}}{\pgfqpoint{-0.032966in}{-0.124306in}}{\pgfqpoint{0.000000in}{-0.124306in}}%
\pgfpathclose%
\pgfusepath{stroke,fill}%
}%
\begin{pgfscope}%
\pgfsys@transformshift{3.560010in}{2.320269in}%
\pgfsys@useobject{currentmarker}{}%
\end{pgfscope}%
\end{pgfscope}%
\begin{pgfscope}%
\pgfpathrectangle{\pgfqpoint{0.100000in}{0.100000in}}{\pgfqpoint{5.307240in}{3.397500in}}%
\pgfusepath{clip}%
\pgfsetrectcap%
\pgfsetroundjoin%
\pgfsetlinewidth{1.505625pt}%
\definecolor{currentstroke}{rgb}{0.678431,1.000000,0.184314}%
\pgfsetstrokecolor{currentstroke}%
\pgfsetstrokeopacity{0.500000}%
\pgfsetdash{}{0pt}%
\pgfpathmoveto{\pgfqpoint{3.529620in}{2.030223in}}%
\pgfusepath{stroke}%
\end{pgfscope}%
\begin{pgfscope}%
\pgfpathrectangle{\pgfqpoint{0.100000in}{0.100000in}}{\pgfqpoint{5.307240in}{3.397500in}}%
\pgfusepath{clip}%
\pgfsetbuttcap%
\pgfsetroundjoin%
\definecolor{currentfill}{rgb}{0.678431,1.000000,0.184314}%
\pgfsetfillcolor{currentfill}%
\pgfsetfillopacity{0.500000}%
\pgfsetlinewidth{0.250937pt}%
\definecolor{currentstroke}{rgb}{0.000000,0.000000,0.000000}%
\pgfsetstrokecolor{currentstroke}%
\pgfsetstrokeopacity{0.500000}%
\pgfsetdash{}{0pt}%
\pgfsys@defobject{currentmarker}{\pgfqpoint{-0.074306in}{-0.074306in}}{\pgfqpoint{0.074306in}{0.074306in}}{%
\pgfpathmoveto{\pgfqpoint{0.000000in}{-0.074306in}}%
\pgfpathcurveto{\pgfqpoint{0.019706in}{-0.074306in}}{\pgfqpoint{0.038608in}{-0.066476in}}{\pgfqpoint{0.052542in}{-0.052542in}}%
\pgfpathcurveto{\pgfqpoint{0.066476in}{-0.038608in}}{\pgfqpoint{0.074306in}{-0.019706in}}{\pgfqpoint{0.074306in}{0.000000in}}%
\pgfpathcurveto{\pgfqpoint{0.074306in}{0.019706in}}{\pgfqpoint{0.066476in}{0.038608in}}{\pgfqpoint{0.052542in}{0.052542in}}%
\pgfpathcurveto{\pgfqpoint{0.038608in}{0.066476in}}{\pgfqpoint{0.019706in}{0.074306in}}{\pgfqpoint{0.000000in}{0.074306in}}%
\pgfpathcurveto{\pgfqpoint{-0.019706in}{0.074306in}}{\pgfqpoint{-0.038608in}{0.066476in}}{\pgfqpoint{-0.052542in}{0.052542in}}%
\pgfpathcurveto{\pgfqpoint{-0.066476in}{0.038608in}}{\pgfqpoint{-0.074306in}{0.019706in}}{\pgfqpoint{-0.074306in}{0.000000in}}%
\pgfpathcurveto{\pgfqpoint{-0.074306in}{-0.019706in}}{\pgfqpoint{-0.066476in}{-0.038608in}}{\pgfqpoint{-0.052542in}{-0.052542in}}%
\pgfpathcurveto{\pgfqpoint{-0.038608in}{-0.066476in}}{\pgfqpoint{-0.019706in}{-0.074306in}}{\pgfqpoint{0.000000in}{-0.074306in}}%
\pgfpathclose%
\pgfusepath{stroke,fill}%
}%
\begin{pgfscope}%
\pgfsys@transformshift{3.529620in}{2.030223in}%
\pgfsys@useobject{currentmarker}{}%
\end{pgfscope}%
\end{pgfscope}%
\begin{pgfscope}%
\pgfpathrectangle{\pgfqpoint{0.100000in}{0.100000in}}{\pgfqpoint{5.307240in}{3.397500in}}%
\pgfusepath{clip}%
\pgfsetrectcap%
\pgfsetroundjoin%
\pgfsetlinewidth{1.505625pt}%
\definecolor{currentstroke}{rgb}{0.678431,1.000000,0.184314}%
\pgfsetstrokecolor{currentstroke}%
\pgfsetstrokeopacity{0.500000}%
\pgfsetdash{}{0pt}%
\pgfpathmoveto{\pgfqpoint{3.813433in}{1.978181in}}%
\pgfusepath{stroke}%
\end{pgfscope}%
\begin{pgfscope}%
\pgfpathrectangle{\pgfqpoint{0.100000in}{0.100000in}}{\pgfqpoint{5.307240in}{3.397500in}}%
\pgfusepath{clip}%
\pgfsetbuttcap%
\pgfsetroundjoin%
\definecolor{currentfill}{rgb}{0.678431,1.000000,0.184314}%
\pgfsetfillcolor{currentfill}%
\pgfsetfillopacity{0.500000}%
\pgfsetlinewidth{0.250937pt}%
\definecolor{currentstroke}{rgb}{0.000000,0.000000,0.000000}%
\pgfsetstrokecolor{currentstroke}%
\pgfsetstrokeopacity{0.500000}%
\pgfsetdash{}{0pt}%
\pgfsys@defobject{currentmarker}{\pgfqpoint{-0.053472in}{-0.053472in}}{\pgfqpoint{0.053472in}{0.053472in}}{%
\pgfpathmoveto{\pgfqpoint{0.000000in}{-0.053472in}}%
\pgfpathcurveto{\pgfqpoint{0.014181in}{-0.053472in}}{\pgfqpoint{0.027783in}{-0.047838in}}{\pgfqpoint{0.037811in}{-0.037811in}}%
\pgfpathcurveto{\pgfqpoint{0.047838in}{-0.027783in}}{\pgfqpoint{0.053472in}{-0.014181in}}{\pgfqpoint{0.053472in}{0.000000in}}%
\pgfpathcurveto{\pgfqpoint{0.053472in}{0.014181in}}{\pgfqpoint{0.047838in}{0.027783in}}{\pgfqpoint{0.037811in}{0.037811in}}%
\pgfpathcurveto{\pgfqpoint{0.027783in}{0.047838in}}{\pgfqpoint{0.014181in}{0.053472in}}{\pgfqpoint{0.000000in}{0.053472in}}%
\pgfpathcurveto{\pgfqpoint{-0.014181in}{0.053472in}}{\pgfqpoint{-0.027783in}{0.047838in}}{\pgfqpoint{-0.037811in}{0.037811in}}%
\pgfpathcurveto{\pgfqpoint{-0.047838in}{0.027783in}}{\pgfqpoint{-0.053472in}{0.014181in}}{\pgfqpoint{-0.053472in}{0.000000in}}%
\pgfpathcurveto{\pgfqpoint{-0.053472in}{-0.014181in}}{\pgfqpoint{-0.047838in}{-0.027783in}}{\pgfqpoint{-0.037811in}{-0.037811in}}%
\pgfpathcurveto{\pgfqpoint{-0.027783in}{-0.047838in}}{\pgfqpoint{-0.014181in}{-0.053472in}}{\pgfqpoint{0.000000in}{-0.053472in}}%
\pgfpathclose%
\pgfusepath{stroke,fill}%
}%
\begin{pgfscope}%
\pgfsys@transformshift{3.813433in}{1.978181in}%
\pgfsys@useobject{currentmarker}{}%
\end{pgfscope}%
\end{pgfscope}%
\begin{pgfscope}%
\pgfpathrectangle{\pgfqpoint{0.100000in}{0.100000in}}{\pgfqpoint{5.307240in}{3.397500in}}%
\pgfusepath{clip}%
\pgfsetrectcap%
\pgfsetroundjoin%
\pgfsetlinewidth{1.505625pt}%
\definecolor{currentstroke}{rgb}{0.678431,1.000000,0.184314}%
\pgfsetstrokecolor{currentstroke}%
\pgfsetstrokeopacity{0.500000}%
\pgfsetdash{}{0pt}%
\pgfpathmoveto{\pgfqpoint{3.868018in}{1.987484in}}%
\pgfusepath{stroke}%
\end{pgfscope}%
\begin{pgfscope}%
\pgfpathrectangle{\pgfqpoint{0.100000in}{0.100000in}}{\pgfqpoint{5.307240in}{3.397500in}}%
\pgfusepath{clip}%
\pgfsetbuttcap%
\pgfsetroundjoin%
\definecolor{currentfill}{rgb}{0.678431,1.000000,0.184314}%
\pgfsetfillcolor{currentfill}%
\pgfsetfillopacity{0.500000}%
\pgfsetlinewidth{0.250937pt}%
\definecolor{currentstroke}{rgb}{0.000000,0.000000,0.000000}%
\pgfsetstrokecolor{currentstroke}%
\pgfsetstrokeopacity{0.500000}%
\pgfsetdash{}{0pt}%
\pgfsys@defobject{currentmarker}{\pgfqpoint{-0.113889in}{-0.113889in}}{\pgfqpoint{0.113889in}{0.113889in}}{%
\pgfpathmoveto{\pgfqpoint{0.000000in}{-0.113889in}}%
\pgfpathcurveto{\pgfqpoint{0.030204in}{-0.113889in}}{\pgfqpoint{0.059174in}{-0.101889in}}{\pgfqpoint{0.080532in}{-0.080532in}}%
\pgfpathcurveto{\pgfqpoint{0.101889in}{-0.059174in}}{\pgfqpoint{0.113889in}{-0.030204in}}{\pgfqpoint{0.113889in}{0.000000in}}%
\pgfpathcurveto{\pgfqpoint{0.113889in}{0.030204in}}{\pgfqpoint{0.101889in}{0.059174in}}{\pgfqpoint{0.080532in}{0.080532in}}%
\pgfpathcurveto{\pgfqpoint{0.059174in}{0.101889in}}{\pgfqpoint{0.030204in}{0.113889in}}{\pgfqpoint{0.000000in}{0.113889in}}%
\pgfpathcurveto{\pgfqpoint{-0.030204in}{0.113889in}}{\pgfqpoint{-0.059174in}{0.101889in}}{\pgfqpoint{-0.080532in}{0.080532in}}%
\pgfpathcurveto{\pgfqpoint{-0.101889in}{0.059174in}}{\pgfqpoint{-0.113889in}{0.030204in}}{\pgfqpoint{-0.113889in}{0.000000in}}%
\pgfpathcurveto{\pgfqpoint{-0.113889in}{-0.030204in}}{\pgfqpoint{-0.101889in}{-0.059174in}}{\pgfqpoint{-0.080532in}{-0.080532in}}%
\pgfpathcurveto{\pgfqpoint{-0.059174in}{-0.101889in}}{\pgfqpoint{-0.030204in}{-0.113889in}}{\pgfqpoint{0.000000in}{-0.113889in}}%
\pgfpathclose%
\pgfusepath{stroke,fill}%
}%
\begin{pgfscope}%
\pgfsys@transformshift{3.868018in}{1.987484in}%
\pgfsys@useobject{currentmarker}{}%
\end{pgfscope}%
\end{pgfscope}%
\begin{pgfscope}%
\pgfpathrectangle{\pgfqpoint{0.100000in}{0.100000in}}{\pgfqpoint{5.307240in}{3.397500in}}%
\pgfusepath{clip}%
\pgfsetrectcap%
\pgfsetroundjoin%
\pgfsetlinewidth{1.505625pt}%
\definecolor{currentstroke}{rgb}{0.678431,1.000000,0.184314}%
\pgfsetstrokecolor{currentstroke}%
\pgfsetstrokeopacity{0.500000}%
\pgfsetdash{}{0pt}%
\pgfpathmoveto{\pgfqpoint{3.834467in}{2.274036in}}%
\pgfusepath{stroke}%
\end{pgfscope}%
\begin{pgfscope}%
\pgfpathrectangle{\pgfqpoint{0.100000in}{0.100000in}}{\pgfqpoint{5.307240in}{3.397500in}}%
\pgfusepath{clip}%
\pgfsetbuttcap%
\pgfsetroundjoin%
\definecolor{currentfill}{rgb}{0.678431,1.000000,0.184314}%
\pgfsetfillcolor{currentfill}%
\pgfsetfillopacity{0.500000}%
\pgfsetlinewidth{0.250937pt}%
\definecolor{currentstroke}{rgb}{0.000000,0.000000,0.000000}%
\pgfsetstrokecolor{currentstroke}%
\pgfsetstrokeopacity{0.500000}%
\pgfsetdash{}{0pt}%
\pgfsys@defobject{currentmarker}{\pgfqpoint{-0.186111in}{-0.186111in}}{\pgfqpoint{0.186111in}{0.186111in}}{%
\pgfpathmoveto{\pgfqpoint{0.000000in}{-0.186111in}}%
\pgfpathcurveto{\pgfqpoint{0.049357in}{-0.186111in}}{\pgfqpoint{0.096700in}{-0.166501in}}{\pgfqpoint{0.131600in}{-0.131600in}}%
\pgfpathcurveto{\pgfqpoint{0.166501in}{-0.096700in}}{\pgfqpoint{0.186111in}{-0.049357in}}{\pgfqpoint{0.186111in}{0.000000in}}%
\pgfpathcurveto{\pgfqpoint{0.186111in}{0.049357in}}{\pgfqpoint{0.166501in}{0.096700in}}{\pgfqpoint{0.131600in}{0.131600in}}%
\pgfpathcurveto{\pgfqpoint{0.096700in}{0.166501in}}{\pgfqpoint{0.049357in}{0.186111in}}{\pgfqpoint{0.000000in}{0.186111in}}%
\pgfpathcurveto{\pgfqpoint{-0.049357in}{0.186111in}}{\pgfqpoint{-0.096700in}{0.166501in}}{\pgfqpoint{-0.131600in}{0.131600in}}%
\pgfpathcurveto{\pgfqpoint{-0.166501in}{0.096700in}}{\pgfqpoint{-0.186111in}{0.049357in}}{\pgfqpoint{-0.186111in}{0.000000in}}%
\pgfpathcurveto{\pgfqpoint{-0.186111in}{-0.049357in}}{\pgfqpoint{-0.166501in}{-0.096700in}}{\pgfqpoint{-0.131600in}{-0.131600in}}%
\pgfpathcurveto{\pgfqpoint{-0.096700in}{-0.166501in}}{\pgfqpoint{-0.049357in}{-0.186111in}}{\pgfqpoint{0.000000in}{-0.186111in}}%
\pgfpathclose%
\pgfusepath{stroke,fill}%
}%
\begin{pgfscope}%
\pgfsys@transformshift{3.834467in}{2.274036in}%
\pgfsys@useobject{currentmarker}{}%
\end{pgfscope}%
\end{pgfscope}%
\begin{pgfscope}%
\pgfpathrectangle{\pgfqpoint{0.100000in}{0.100000in}}{\pgfqpoint{5.307240in}{3.397500in}}%
\pgfusepath{clip}%
\pgfsetrectcap%
\pgfsetroundjoin%
\pgfsetlinewidth{1.505625pt}%
\definecolor{currentstroke}{rgb}{0.678431,1.000000,0.184314}%
\pgfsetstrokecolor{currentstroke}%
\pgfsetstrokeopacity{0.500000}%
\pgfsetdash{}{0pt}%
\pgfpathmoveto{\pgfqpoint{3.733055in}{1.831987in}}%
\pgfusepath{stroke}%
\end{pgfscope}%
\begin{pgfscope}%
\pgfpathrectangle{\pgfqpoint{0.100000in}{0.100000in}}{\pgfqpoint{5.307240in}{3.397500in}}%
\pgfusepath{clip}%
\pgfsetbuttcap%
\pgfsetroundjoin%
\definecolor{currentfill}{rgb}{0.678431,1.000000,0.184314}%
\pgfsetfillcolor{currentfill}%
\pgfsetfillopacity{0.500000}%
\pgfsetlinewidth{0.250937pt}%
\definecolor{currentstroke}{rgb}{0.000000,0.000000,0.000000}%
\pgfsetstrokecolor{currentstroke}%
\pgfsetstrokeopacity{0.500000}%
\pgfsetdash{}{0pt}%
\pgfsys@defobject{currentmarker}{\pgfqpoint{-0.084028in}{-0.084028in}}{\pgfqpoint{0.084028in}{0.084028in}}{%
\pgfpathmoveto{\pgfqpoint{0.000000in}{-0.084028in}}%
\pgfpathcurveto{\pgfqpoint{0.022284in}{-0.084028in}}{\pgfqpoint{0.043659in}{-0.075174in}}{\pgfqpoint{0.059417in}{-0.059417in}}%
\pgfpathcurveto{\pgfqpoint{0.075174in}{-0.043659in}}{\pgfqpoint{0.084028in}{-0.022284in}}{\pgfqpoint{0.084028in}{0.000000in}}%
\pgfpathcurveto{\pgfqpoint{0.084028in}{0.022284in}}{\pgfqpoint{0.075174in}{0.043659in}}{\pgfqpoint{0.059417in}{0.059417in}}%
\pgfpathcurveto{\pgfqpoint{0.043659in}{0.075174in}}{\pgfqpoint{0.022284in}{0.084028in}}{\pgfqpoint{0.000000in}{0.084028in}}%
\pgfpathcurveto{\pgfqpoint{-0.022284in}{0.084028in}}{\pgfqpoint{-0.043659in}{0.075174in}}{\pgfqpoint{-0.059417in}{0.059417in}}%
\pgfpathcurveto{\pgfqpoint{-0.075174in}{0.043659in}}{\pgfqpoint{-0.084028in}{0.022284in}}{\pgfqpoint{-0.084028in}{0.000000in}}%
\pgfpathcurveto{\pgfqpoint{-0.084028in}{-0.022284in}}{\pgfqpoint{-0.075174in}{-0.043659in}}{\pgfqpoint{-0.059417in}{-0.059417in}}%
\pgfpathcurveto{\pgfqpoint{-0.043659in}{-0.075174in}}{\pgfqpoint{-0.022284in}{-0.084028in}}{\pgfqpoint{0.000000in}{-0.084028in}}%
\pgfpathclose%
\pgfusepath{stroke,fill}%
}%
\begin{pgfscope}%
\pgfsys@transformshift{3.733055in}{1.831987in}%
\pgfsys@useobject{currentmarker}{}%
\end{pgfscope}%
\end{pgfscope}%
\begin{pgfscope}%
\pgfpathrectangle{\pgfqpoint{0.100000in}{0.100000in}}{\pgfqpoint{5.307240in}{3.397500in}}%
\pgfusepath{clip}%
\pgfsetrectcap%
\pgfsetroundjoin%
\pgfsetlinewidth{1.505625pt}%
\definecolor{currentstroke}{rgb}{0.678431,1.000000,0.184314}%
\pgfsetstrokecolor{currentstroke}%
\pgfsetstrokeopacity{0.500000}%
\pgfsetdash{}{0pt}%
\pgfpathmoveto{\pgfqpoint{3.914496in}{2.211962in}}%
\pgfusepath{stroke}%
\end{pgfscope}%
\begin{pgfscope}%
\pgfpathrectangle{\pgfqpoint{0.100000in}{0.100000in}}{\pgfqpoint{5.307240in}{3.397500in}}%
\pgfusepath{clip}%
\pgfsetbuttcap%
\pgfsetroundjoin%
\definecolor{currentfill}{rgb}{0.678431,1.000000,0.184314}%
\pgfsetfillcolor{currentfill}%
\pgfsetfillopacity{0.500000}%
\pgfsetlinewidth{0.250937pt}%
\definecolor{currentstroke}{rgb}{0.000000,0.000000,0.000000}%
\pgfsetstrokecolor{currentstroke}%
\pgfsetstrokeopacity{0.500000}%
\pgfsetdash{}{0pt}%
\pgfsys@defobject{currentmarker}{\pgfqpoint{-0.118750in}{-0.118750in}}{\pgfqpoint{0.118750in}{0.118750in}}{%
\pgfpathmoveto{\pgfqpoint{0.000000in}{-0.118750in}}%
\pgfpathcurveto{\pgfqpoint{0.031493in}{-0.118750in}}{\pgfqpoint{0.061700in}{-0.106238in}}{\pgfqpoint{0.083969in}{-0.083969in}}%
\pgfpathcurveto{\pgfqpoint{0.106238in}{-0.061700in}}{\pgfqpoint{0.118750in}{-0.031493in}}{\pgfqpoint{0.118750in}{0.000000in}}%
\pgfpathcurveto{\pgfqpoint{0.118750in}{0.031493in}}{\pgfqpoint{0.106238in}{0.061700in}}{\pgfqpoint{0.083969in}{0.083969in}}%
\pgfpathcurveto{\pgfqpoint{0.061700in}{0.106238in}}{\pgfqpoint{0.031493in}{0.118750in}}{\pgfqpoint{0.000000in}{0.118750in}}%
\pgfpathcurveto{\pgfqpoint{-0.031493in}{0.118750in}}{\pgfqpoint{-0.061700in}{0.106238in}}{\pgfqpoint{-0.083969in}{0.083969in}}%
\pgfpathcurveto{\pgfqpoint{-0.106238in}{0.061700in}}{\pgfqpoint{-0.118750in}{0.031493in}}{\pgfqpoint{-0.118750in}{0.000000in}}%
\pgfpathcurveto{\pgfqpoint{-0.118750in}{-0.031493in}}{\pgfqpoint{-0.106238in}{-0.061700in}}{\pgfqpoint{-0.083969in}{-0.083969in}}%
\pgfpathcurveto{\pgfqpoint{-0.061700in}{-0.106238in}}{\pgfqpoint{-0.031493in}{-0.118750in}}{\pgfqpoint{0.000000in}{-0.118750in}}%
\pgfpathclose%
\pgfusepath{stroke,fill}%
}%
\begin{pgfscope}%
\pgfsys@transformshift{3.914496in}{2.211962in}%
\pgfsys@useobject{currentmarker}{}%
\end{pgfscope}%
\end{pgfscope}%
\begin{pgfscope}%
\pgfpathrectangle{\pgfqpoint{0.100000in}{0.100000in}}{\pgfqpoint{5.307240in}{3.397500in}}%
\pgfusepath{clip}%
\pgfsetrectcap%
\pgfsetroundjoin%
\pgfsetlinewidth{1.505625pt}%
\definecolor{currentstroke}{rgb}{0.678431,1.000000,0.184314}%
\pgfsetstrokecolor{currentstroke}%
\pgfsetstrokeopacity{0.500000}%
\pgfsetdash{}{0pt}%
\pgfpathmoveto{\pgfqpoint{3.878871in}{2.094209in}}%
\pgfusepath{stroke}%
\end{pgfscope}%
\begin{pgfscope}%
\pgfpathrectangle{\pgfqpoint{0.100000in}{0.100000in}}{\pgfqpoint{5.307240in}{3.397500in}}%
\pgfusepath{clip}%
\pgfsetbuttcap%
\pgfsetroundjoin%
\definecolor{currentfill}{rgb}{0.678431,1.000000,0.184314}%
\pgfsetfillcolor{currentfill}%
\pgfsetfillopacity{0.500000}%
\pgfsetlinewidth{0.250937pt}%
\definecolor{currentstroke}{rgb}{0.000000,0.000000,0.000000}%
\pgfsetstrokecolor{currentstroke}%
\pgfsetstrokeopacity{0.500000}%
\pgfsetdash{}{0pt}%
\pgfsys@defobject{currentmarker}{\pgfqpoint{-0.074306in}{-0.074306in}}{\pgfqpoint{0.074306in}{0.074306in}}{%
\pgfpathmoveto{\pgfqpoint{0.000000in}{-0.074306in}}%
\pgfpathcurveto{\pgfqpoint{0.019706in}{-0.074306in}}{\pgfqpoint{0.038608in}{-0.066476in}}{\pgfqpoint{0.052542in}{-0.052542in}}%
\pgfpathcurveto{\pgfqpoint{0.066476in}{-0.038608in}}{\pgfqpoint{0.074306in}{-0.019706in}}{\pgfqpoint{0.074306in}{0.000000in}}%
\pgfpathcurveto{\pgfqpoint{0.074306in}{0.019706in}}{\pgfqpoint{0.066476in}{0.038608in}}{\pgfqpoint{0.052542in}{0.052542in}}%
\pgfpathcurveto{\pgfqpoint{0.038608in}{0.066476in}}{\pgfqpoint{0.019706in}{0.074306in}}{\pgfqpoint{0.000000in}{0.074306in}}%
\pgfpathcurveto{\pgfqpoint{-0.019706in}{0.074306in}}{\pgfqpoint{-0.038608in}{0.066476in}}{\pgfqpoint{-0.052542in}{0.052542in}}%
\pgfpathcurveto{\pgfqpoint{-0.066476in}{0.038608in}}{\pgfqpoint{-0.074306in}{0.019706in}}{\pgfqpoint{-0.074306in}{0.000000in}}%
\pgfpathcurveto{\pgfqpoint{-0.074306in}{-0.019706in}}{\pgfqpoint{-0.066476in}{-0.038608in}}{\pgfqpoint{-0.052542in}{-0.052542in}}%
\pgfpathcurveto{\pgfqpoint{-0.038608in}{-0.066476in}}{\pgfqpoint{-0.019706in}{-0.074306in}}{\pgfqpoint{0.000000in}{-0.074306in}}%
\pgfpathclose%
\pgfusepath{stroke,fill}%
}%
\begin{pgfscope}%
\pgfsys@transformshift{3.878871in}{2.094209in}%
\pgfsys@useobject{currentmarker}{}%
\end{pgfscope}%
\end{pgfscope}%
\begin{pgfscope}%
\pgfpathrectangle{\pgfqpoint{0.100000in}{0.100000in}}{\pgfqpoint{5.307240in}{3.397500in}}%
\pgfusepath{clip}%
\pgfsetrectcap%
\pgfsetroundjoin%
\pgfsetlinewidth{1.505625pt}%
\definecolor{currentstroke}{rgb}{0.678431,1.000000,0.184314}%
\pgfsetstrokecolor{currentstroke}%
\pgfsetstrokeopacity{0.500000}%
\pgfsetdash{}{0pt}%
\pgfpathmoveto{\pgfqpoint{3.840372in}{2.050974in}}%
\pgfusepath{stroke}%
\end{pgfscope}%
\begin{pgfscope}%
\pgfpathrectangle{\pgfqpoint{0.100000in}{0.100000in}}{\pgfqpoint{5.307240in}{3.397500in}}%
\pgfusepath{clip}%
\pgfsetbuttcap%
\pgfsetroundjoin%
\definecolor{currentfill}{rgb}{0.678431,1.000000,0.184314}%
\pgfsetfillcolor{currentfill}%
\pgfsetfillopacity{0.500000}%
\pgfsetlinewidth{0.250937pt}%
\definecolor{currentstroke}{rgb}{0.000000,0.000000,0.000000}%
\pgfsetstrokecolor{currentstroke}%
\pgfsetstrokeopacity{0.500000}%
\pgfsetdash{}{0pt}%
\pgfsys@defobject{currentmarker}{\pgfqpoint{-0.074306in}{-0.074306in}}{\pgfqpoint{0.074306in}{0.074306in}}{%
\pgfpathmoveto{\pgfqpoint{0.000000in}{-0.074306in}}%
\pgfpathcurveto{\pgfqpoint{0.019706in}{-0.074306in}}{\pgfqpoint{0.038608in}{-0.066476in}}{\pgfqpoint{0.052542in}{-0.052542in}}%
\pgfpathcurveto{\pgfqpoint{0.066476in}{-0.038608in}}{\pgfqpoint{0.074306in}{-0.019706in}}{\pgfqpoint{0.074306in}{0.000000in}}%
\pgfpathcurveto{\pgfqpoint{0.074306in}{0.019706in}}{\pgfqpoint{0.066476in}{0.038608in}}{\pgfqpoint{0.052542in}{0.052542in}}%
\pgfpathcurveto{\pgfqpoint{0.038608in}{0.066476in}}{\pgfqpoint{0.019706in}{0.074306in}}{\pgfqpoint{0.000000in}{0.074306in}}%
\pgfpathcurveto{\pgfqpoint{-0.019706in}{0.074306in}}{\pgfqpoint{-0.038608in}{0.066476in}}{\pgfqpoint{-0.052542in}{0.052542in}}%
\pgfpathcurveto{\pgfqpoint{-0.066476in}{0.038608in}}{\pgfqpoint{-0.074306in}{0.019706in}}{\pgfqpoint{-0.074306in}{0.000000in}}%
\pgfpathcurveto{\pgfqpoint{-0.074306in}{-0.019706in}}{\pgfqpoint{-0.066476in}{-0.038608in}}{\pgfqpoint{-0.052542in}{-0.052542in}}%
\pgfpathcurveto{\pgfqpoint{-0.038608in}{-0.066476in}}{\pgfqpoint{-0.019706in}{-0.074306in}}{\pgfqpoint{0.000000in}{-0.074306in}}%
\pgfpathclose%
\pgfusepath{stroke,fill}%
}%
\begin{pgfscope}%
\pgfsys@transformshift{3.840372in}{2.050974in}%
\pgfsys@useobject{currentmarker}{}%
\end{pgfscope}%
\end{pgfscope}%
\begin{pgfscope}%
\pgfpathrectangle{\pgfqpoint{0.100000in}{0.100000in}}{\pgfqpoint{5.307240in}{3.397500in}}%
\pgfusepath{clip}%
\pgfsetrectcap%
\pgfsetroundjoin%
\pgfsetlinewidth{1.505625pt}%
\definecolor{currentstroke}{rgb}{0.678431,1.000000,0.184314}%
\pgfsetstrokecolor{currentstroke}%
\pgfsetstrokeopacity{0.500000}%
\pgfsetdash{}{0pt}%
\pgfpathmoveto{\pgfqpoint{3.834437in}{2.134286in}}%
\pgfusepath{stroke}%
\end{pgfscope}%
\begin{pgfscope}%
\pgfpathrectangle{\pgfqpoint{0.100000in}{0.100000in}}{\pgfqpoint{5.307240in}{3.397500in}}%
\pgfusepath{clip}%
\pgfsetbuttcap%
\pgfsetroundjoin%
\definecolor{currentfill}{rgb}{0.678431,1.000000,0.184314}%
\pgfsetfillcolor{currentfill}%
\pgfsetfillopacity{0.500000}%
\pgfsetlinewidth{0.250937pt}%
\definecolor{currentstroke}{rgb}{0.000000,0.000000,0.000000}%
\pgfsetstrokecolor{currentstroke}%
\pgfsetstrokeopacity{0.500000}%
\pgfsetdash{}{0pt}%
\pgfsys@defobject{currentmarker}{\pgfqpoint{-0.190972in}{-0.190972in}}{\pgfqpoint{0.190972in}{0.190972in}}{%
\pgfpathmoveto{\pgfqpoint{0.000000in}{-0.190972in}}%
\pgfpathcurveto{\pgfqpoint{0.050646in}{-0.190972in}}{\pgfqpoint{0.099225in}{-0.170850in}}{\pgfqpoint{0.135038in}{-0.135038in}}%
\pgfpathcurveto{\pgfqpoint{0.170850in}{-0.099225in}}{\pgfqpoint{0.190972in}{-0.050646in}}{\pgfqpoint{0.190972in}{0.000000in}}%
\pgfpathcurveto{\pgfqpoint{0.190972in}{0.050646in}}{\pgfqpoint{0.170850in}{0.099225in}}{\pgfqpoint{0.135038in}{0.135038in}}%
\pgfpathcurveto{\pgfqpoint{0.099225in}{0.170850in}}{\pgfqpoint{0.050646in}{0.190972in}}{\pgfqpoint{0.000000in}{0.190972in}}%
\pgfpathcurveto{\pgfqpoint{-0.050646in}{0.190972in}}{\pgfqpoint{-0.099225in}{0.170850in}}{\pgfqpoint{-0.135038in}{0.135038in}}%
\pgfpathcurveto{\pgfqpoint{-0.170850in}{0.099225in}}{\pgfqpoint{-0.190972in}{0.050646in}}{\pgfqpoint{-0.190972in}{0.000000in}}%
\pgfpathcurveto{\pgfqpoint{-0.190972in}{-0.050646in}}{\pgfqpoint{-0.170850in}{-0.099225in}}{\pgfqpoint{-0.135038in}{-0.135038in}}%
\pgfpathcurveto{\pgfqpoint{-0.099225in}{-0.170850in}}{\pgfqpoint{-0.050646in}{-0.190972in}}{\pgfqpoint{0.000000in}{-0.190972in}}%
\pgfpathclose%
\pgfusepath{stroke,fill}%
}%
\begin{pgfscope}%
\pgfsys@transformshift{3.834437in}{2.134286in}%
\pgfsys@useobject{currentmarker}{}%
\end{pgfscope}%
\end{pgfscope}%
\begin{pgfscope}%
\pgfpathrectangle{\pgfqpoint{0.100000in}{0.100000in}}{\pgfqpoint{5.307240in}{3.397500in}}%
\pgfusepath{clip}%
\pgfsetrectcap%
\pgfsetroundjoin%
\pgfsetlinewidth{1.505625pt}%
\definecolor{currentstroke}{rgb}{0.678431,1.000000,0.184314}%
\pgfsetstrokecolor{currentstroke}%
\pgfsetstrokeopacity{0.500000}%
\pgfsetdash{}{0pt}%
\pgfpathmoveto{\pgfqpoint{3.769855in}{2.120086in}}%
\pgfusepath{stroke}%
\end{pgfscope}%
\begin{pgfscope}%
\pgfpathrectangle{\pgfqpoint{0.100000in}{0.100000in}}{\pgfqpoint{5.307240in}{3.397500in}}%
\pgfusepath{clip}%
\pgfsetbuttcap%
\pgfsetroundjoin%
\definecolor{currentfill}{rgb}{0.678431,1.000000,0.184314}%
\pgfsetfillcolor{currentfill}%
\pgfsetfillopacity{0.500000}%
\pgfsetlinewidth{0.250937pt}%
\definecolor{currentstroke}{rgb}{0.000000,0.000000,0.000000}%
\pgfsetstrokecolor{currentstroke}%
\pgfsetstrokeopacity{0.500000}%
\pgfsetdash{}{0pt}%
\pgfsys@defobject{currentmarker}{\pgfqpoint{-0.068750in}{-0.068750in}}{\pgfqpoint{0.068750in}{0.068750in}}{%
\pgfpathmoveto{\pgfqpoint{0.000000in}{-0.068750in}}%
\pgfpathcurveto{\pgfqpoint{0.018233in}{-0.068750in}}{\pgfqpoint{0.035721in}{-0.061506in}}{\pgfqpoint{0.048614in}{-0.048614in}}%
\pgfpathcurveto{\pgfqpoint{0.061506in}{-0.035721in}}{\pgfqpoint{0.068750in}{-0.018233in}}{\pgfqpoint{0.068750in}{0.000000in}}%
\pgfpathcurveto{\pgfqpoint{0.068750in}{0.018233in}}{\pgfqpoint{0.061506in}{0.035721in}}{\pgfqpoint{0.048614in}{0.048614in}}%
\pgfpathcurveto{\pgfqpoint{0.035721in}{0.061506in}}{\pgfqpoint{0.018233in}{0.068750in}}{\pgfqpoint{0.000000in}{0.068750in}}%
\pgfpathcurveto{\pgfqpoint{-0.018233in}{0.068750in}}{\pgfqpoint{-0.035721in}{0.061506in}}{\pgfqpoint{-0.048614in}{0.048614in}}%
\pgfpathcurveto{\pgfqpoint{-0.061506in}{0.035721in}}{\pgfqpoint{-0.068750in}{0.018233in}}{\pgfqpoint{-0.068750in}{0.000000in}}%
\pgfpathcurveto{\pgfqpoint{-0.068750in}{-0.018233in}}{\pgfqpoint{-0.061506in}{-0.035721in}}{\pgfqpoint{-0.048614in}{-0.048614in}}%
\pgfpathcurveto{\pgfqpoint{-0.035721in}{-0.061506in}}{\pgfqpoint{-0.018233in}{-0.068750in}}{\pgfqpoint{0.000000in}{-0.068750in}}%
\pgfpathclose%
\pgfusepath{stroke,fill}%
}%
\begin{pgfscope}%
\pgfsys@transformshift{3.769855in}{2.120086in}%
\pgfsys@useobject{currentmarker}{}%
\end{pgfscope}%
\end{pgfscope}%
\begin{pgfscope}%
\pgfpathrectangle{\pgfqpoint{0.100000in}{0.100000in}}{\pgfqpoint{5.307240in}{3.397500in}}%
\pgfusepath{clip}%
\pgfsetrectcap%
\pgfsetroundjoin%
\pgfsetlinewidth{1.505625pt}%
\definecolor{currentstroke}{rgb}{0.678431,1.000000,0.184314}%
\pgfsetstrokecolor{currentstroke}%
\pgfsetstrokeopacity{0.500000}%
\pgfsetdash{}{0pt}%
\pgfpathmoveto{\pgfqpoint{3.754740in}{2.269486in}}%
\pgfusepath{stroke}%
\end{pgfscope}%
\begin{pgfscope}%
\pgfpathrectangle{\pgfqpoint{0.100000in}{0.100000in}}{\pgfqpoint{5.307240in}{3.397500in}}%
\pgfusepath{clip}%
\pgfsetbuttcap%
\pgfsetroundjoin%
\definecolor{currentfill}{rgb}{0.678431,1.000000,0.184314}%
\pgfsetfillcolor{currentfill}%
\pgfsetfillopacity{0.500000}%
\pgfsetlinewidth{0.250937pt}%
\definecolor{currentstroke}{rgb}{0.000000,0.000000,0.000000}%
\pgfsetstrokecolor{currentstroke}%
\pgfsetstrokeopacity{0.500000}%
\pgfsetdash{}{0pt}%
\pgfsys@defobject{currentmarker}{\pgfqpoint{-0.117361in}{-0.117361in}}{\pgfqpoint{0.117361in}{0.117361in}}{%
\pgfpathmoveto{\pgfqpoint{0.000000in}{-0.117361in}}%
\pgfpathcurveto{\pgfqpoint{0.031125in}{-0.117361in}}{\pgfqpoint{0.060978in}{-0.104995in}}{\pgfqpoint{0.082987in}{-0.082987in}}%
\pgfpathcurveto{\pgfqpoint{0.104995in}{-0.060978in}}{\pgfqpoint{0.117361in}{-0.031125in}}{\pgfqpoint{0.117361in}{0.000000in}}%
\pgfpathcurveto{\pgfqpoint{0.117361in}{0.031125in}}{\pgfqpoint{0.104995in}{0.060978in}}{\pgfqpoint{0.082987in}{0.082987in}}%
\pgfpathcurveto{\pgfqpoint{0.060978in}{0.104995in}}{\pgfqpoint{0.031125in}{0.117361in}}{\pgfqpoint{0.000000in}{0.117361in}}%
\pgfpathcurveto{\pgfqpoint{-0.031125in}{0.117361in}}{\pgfqpoint{-0.060978in}{0.104995in}}{\pgfqpoint{-0.082987in}{0.082987in}}%
\pgfpathcurveto{\pgfqpoint{-0.104995in}{0.060978in}}{\pgfqpoint{-0.117361in}{0.031125in}}{\pgfqpoint{-0.117361in}{0.000000in}}%
\pgfpathcurveto{\pgfqpoint{-0.117361in}{-0.031125in}}{\pgfqpoint{-0.104995in}{-0.060978in}}{\pgfqpoint{-0.082987in}{-0.082987in}}%
\pgfpathcurveto{\pgfqpoint{-0.060978in}{-0.104995in}}{\pgfqpoint{-0.031125in}{-0.117361in}}{\pgfqpoint{0.000000in}{-0.117361in}}%
\pgfpathclose%
\pgfusepath{stroke,fill}%
}%
\begin{pgfscope}%
\pgfsys@transformshift{3.754740in}{2.269486in}%
\pgfsys@useobject{currentmarker}{}%
\end{pgfscope}%
\end{pgfscope}%
\begin{pgfscope}%
\pgfpathrectangle{\pgfqpoint{0.100000in}{0.100000in}}{\pgfqpoint{5.307240in}{3.397500in}}%
\pgfusepath{clip}%
\pgfsetrectcap%
\pgfsetroundjoin%
\pgfsetlinewidth{1.505625pt}%
\definecolor{currentstroke}{rgb}{0.678431,1.000000,0.184314}%
\pgfsetstrokecolor{currentstroke}%
\pgfsetstrokeopacity{0.500000}%
\pgfsetdash{}{0pt}%
\pgfpathmoveto{\pgfqpoint{3.903767in}{2.107140in}}%
\pgfusepath{stroke}%
\end{pgfscope}%
\begin{pgfscope}%
\pgfpathrectangle{\pgfqpoint{0.100000in}{0.100000in}}{\pgfqpoint{5.307240in}{3.397500in}}%
\pgfusepath{clip}%
\pgfsetbuttcap%
\pgfsetroundjoin%
\definecolor{currentfill}{rgb}{0.678431,1.000000,0.184314}%
\pgfsetfillcolor{currentfill}%
\pgfsetfillopacity{0.500000}%
\pgfsetlinewidth{0.250937pt}%
\definecolor{currentstroke}{rgb}{0.000000,0.000000,0.000000}%
\pgfsetstrokecolor{currentstroke}%
\pgfsetstrokeopacity{0.500000}%
\pgfsetdash{}{0pt}%
\pgfsys@defobject{currentmarker}{\pgfqpoint{-0.092361in}{-0.092361in}}{\pgfqpoint{0.092361in}{0.092361in}}{%
\pgfpathmoveto{\pgfqpoint{0.000000in}{-0.092361in}}%
\pgfpathcurveto{\pgfqpoint{0.024494in}{-0.092361in}}{\pgfqpoint{0.047989in}{-0.082629in}}{\pgfqpoint{0.065309in}{-0.065309in}}%
\pgfpathcurveto{\pgfqpoint{0.082629in}{-0.047989in}}{\pgfqpoint{0.092361in}{-0.024494in}}{\pgfqpoint{0.092361in}{0.000000in}}%
\pgfpathcurveto{\pgfqpoint{0.092361in}{0.024494in}}{\pgfqpoint{0.082629in}{0.047989in}}{\pgfqpoint{0.065309in}{0.065309in}}%
\pgfpathcurveto{\pgfqpoint{0.047989in}{0.082629in}}{\pgfqpoint{0.024494in}{0.092361in}}{\pgfqpoint{0.000000in}{0.092361in}}%
\pgfpathcurveto{\pgfqpoint{-0.024494in}{0.092361in}}{\pgfqpoint{-0.047989in}{0.082629in}}{\pgfqpoint{-0.065309in}{0.065309in}}%
\pgfpathcurveto{\pgfqpoint{-0.082629in}{0.047989in}}{\pgfqpoint{-0.092361in}{0.024494in}}{\pgfqpoint{-0.092361in}{0.000000in}}%
\pgfpathcurveto{\pgfqpoint{-0.092361in}{-0.024494in}}{\pgfqpoint{-0.082629in}{-0.047989in}}{\pgfqpoint{-0.065309in}{-0.065309in}}%
\pgfpathcurveto{\pgfqpoint{-0.047989in}{-0.082629in}}{\pgfqpoint{-0.024494in}{-0.092361in}}{\pgfqpoint{0.000000in}{-0.092361in}}%
\pgfpathclose%
\pgfusepath{stroke,fill}%
}%
\begin{pgfscope}%
\pgfsys@transformshift{3.903767in}{2.107140in}%
\pgfsys@useobject{currentmarker}{}%
\end{pgfscope}%
\end{pgfscope}%
\begin{pgfscope}%
\pgfpathrectangle{\pgfqpoint{0.100000in}{0.100000in}}{\pgfqpoint{5.307240in}{3.397500in}}%
\pgfusepath{clip}%
\pgfsetrectcap%
\pgfsetroundjoin%
\pgfsetlinewidth{1.505625pt}%
\definecolor{currentstroke}{rgb}{0.678431,1.000000,0.184314}%
\pgfsetstrokecolor{currentstroke}%
\pgfsetstrokeopacity{0.500000}%
\pgfsetdash{}{0pt}%
\pgfpathmoveto{\pgfqpoint{3.810808in}{2.271877in}}%
\pgfusepath{stroke}%
\end{pgfscope}%
\begin{pgfscope}%
\pgfpathrectangle{\pgfqpoint{0.100000in}{0.100000in}}{\pgfqpoint{5.307240in}{3.397500in}}%
\pgfusepath{clip}%
\pgfsetbuttcap%
\pgfsetroundjoin%
\definecolor{currentfill}{rgb}{0.678431,1.000000,0.184314}%
\pgfsetfillcolor{currentfill}%
\pgfsetfillopacity{0.500000}%
\pgfsetlinewidth{0.250937pt}%
\definecolor{currentstroke}{rgb}{0.000000,0.000000,0.000000}%
\pgfsetstrokecolor{currentstroke}%
\pgfsetstrokeopacity{0.500000}%
\pgfsetdash{}{0pt}%
\pgfsys@defobject{currentmarker}{\pgfqpoint{-0.125000in}{-0.125000in}}{\pgfqpoint{0.125000in}{0.125000in}}{%
\pgfpathmoveto{\pgfqpoint{0.000000in}{-0.125000in}}%
\pgfpathcurveto{\pgfqpoint{0.033150in}{-0.125000in}}{\pgfqpoint{0.064947in}{-0.111829in}}{\pgfqpoint{0.088388in}{-0.088388in}}%
\pgfpathcurveto{\pgfqpoint{0.111829in}{-0.064947in}}{\pgfqpoint{0.125000in}{-0.033150in}}{\pgfqpoint{0.125000in}{0.000000in}}%
\pgfpathcurveto{\pgfqpoint{0.125000in}{0.033150in}}{\pgfqpoint{0.111829in}{0.064947in}}{\pgfqpoint{0.088388in}{0.088388in}}%
\pgfpathcurveto{\pgfqpoint{0.064947in}{0.111829in}}{\pgfqpoint{0.033150in}{0.125000in}}{\pgfqpoint{0.000000in}{0.125000in}}%
\pgfpathcurveto{\pgfqpoint{-0.033150in}{0.125000in}}{\pgfqpoint{-0.064947in}{0.111829in}}{\pgfqpoint{-0.088388in}{0.088388in}}%
\pgfpathcurveto{\pgfqpoint{-0.111829in}{0.064947in}}{\pgfqpoint{-0.125000in}{0.033150in}}{\pgfqpoint{-0.125000in}{0.000000in}}%
\pgfpathcurveto{\pgfqpoint{-0.125000in}{-0.033150in}}{\pgfqpoint{-0.111829in}{-0.064947in}}{\pgfqpoint{-0.088388in}{-0.088388in}}%
\pgfpathcurveto{\pgfqpoint{-0.064947in}{-0.111829in}}{\pgfqpoint{-0.033150in}{-0.125000in}}{\pgfqpoint{0.000000in}{-0.125000in}}%
\pgfpathclose%
\pgfusepath{stroke,fill}%
}%
\begin{pgfscope}%
\pgfsys@transformshift{3.810808in}{2.271877in}%
\pgfsys@useobject{currentmarker}{}%
\end{pgfscope}%
\end{pgfscope}%
\begin{pgfscope}%
\pgfpathrectangle{\pgfqpoint{0.100000in}{0.100000in}}{\pgfqpoint{5.307240in}{3.397500in}}%
\pgfusepath{clip}%
\pgfsetrectcap%
\pgfsetroundjoin%
\pgfsetlinewidth{1.505625pt}%
\definecolor{currentstroke}{rgb}{0.678431,1.000000,0.184314}%
\pgfsetstrokecolor{currentstroke}%
\pgfsetstrokeopacity{0.500000}%
\pgfsetdash{}{0pt}%
\pgfpathmoveto{\pgfqpoint{3.731558in}{2.005903in}}%
\pgfusepath{stroke}%
\end{pgfscope}%
\begin{pgfscope}%
\pgfpathrectangle{\pgfqpoint{0.100000in}{0.100000in}}{\pgfqpoint{5.307240in}{3.397500in}}%
\pgfusepath{clip}%
\pgfsetbuttcap%
\pgfsetroundjoin%
\definecolor{currentfill}{rgb}{0.678431,1.000000,0.184314}%
\pgfsetfillcolor{currentfill}%
\pgfsetfillopacity{0.500000}%
\pgfsetlinewidth{0.250937pt}%
\definecolor{currentstroke}{rgb}{0.000000,0.000000,0.000000}%
\pgfsetstrokecolor{currentstroke}%
\pgfsetstrokeopacity{0.500000}%
\pgfsetdash{}{0pt}%
\pgfsys@defobject{currentmarker}{\pgfqpoint{-0.088889in}{-0.088889in}}{\pgfqpoint{0.088889in}{0.088889in}}{%
\pgfpathmoveto{\pgfqpoint{0.000000in}{-0.088889in}}%
\pgfpathcurveto{\pgfqpoint{0.023574in}{-0.088889in}}{\pgfqpoint{0.046185in}{-0.079523in}}{\pgfqpoint{0.062854in}{-0.062854in}}%
\pgfpathcurveto{\pgfqpoint{0.079523in}{-0.046185in}}{\pgfqpoint{0.088889in}{-0.023574in}}{\pgfqpoint{0.088889in}{0.000000in}}%
\pgfpathcurveto{\pgfqpoint{0.088889in}{0.023574in}}{\pgfqpoint{0.079523in}{0.046185in}}{\pgfqpoint{0.062854in}{0.062854in}}%
\pgfpathcurveto{\pgfqpoint{0.046185in}{0.079523in}}{\pgfqpoint{0.023574in}{0.088889in}}{\pgfqpoint{0.000000in}{0.088889in}}%
\pgfpathcurveto{\pgfqpoint{-0.023574in}{0.088889in}}{\pgfqpoint{-0.046185in}{0.079523in}}{\pgfqpoint{-0.062854in}{0.062854in}}%
\pgfpathcurveto{\pgfqpoint{-0.079523in}{0.046185in}}{\pgfqpoint{-0.088889in}{0.023574in}}{\pgfqpoint{-0.088889in}{0.000000in}}%
\pgfpathcurveto{\pgfqpoint{-0.088889in}{-0.023574in}}{\pgfqpoint{-0.079523in}{-0.046185in}}{\pgfqpoint{-0.062854in}{-0.062854in}}%
\pgfpathcurveto{\pgfqpoint{-0.046185in}{-0.079523in}}{\pgfqpoint{-0.023574in}{-0.088889in}}{\pgfqpoint{0.000000in}{-0.088889in}}%
\pgfpathclose%
\pgfusepath{stroke,fill}%
}%
\begin{pgfscope}%
\pgfsys@transformshift{3.731558in}{2.005903in}%
\pgfsys@useobject{currentmarker}{}%
\end{pgfscope}%
\end{pgfscope}%
\begin{pgfscope}%
\pgfpathrectangle{\pgfqpoint{0.100000in}{0.100000in}}{\pgfqpoint{5.307240in}{3.397500in}}%
\pgfusepath{clip}%
\pgfsetrectcap%
\pgfsetroundjoin%
\pgfsetlinewidth{1.505625pt}%
\definecolor{currentstroke}{rgb}{0.678431,1.000000,0.184314}%
\pgfsetstrokecolor{currentstroke}%
\pgfsetstrokeopacity{0.500000}%
\pgfsetdash{}{0pt}%
\pgfpathmoveto{\pgfqpoint{3.171105in}{2.276171in}}%
\pgfusepath{stroke}%
\end{pgfscope}%
\begin{pgfscope}%
\pgfpathrectangle{\pgfqpoint{0.100000in}{0.100000in}}{\pgfqpoint{5.307240in}{3.397500in}}%
\pgfusepath{clip}%
\pgfsetbuttcap%
\pgfsetroundjoin%
\definecolor{currentfill}{rgb}{0.678431,1.000000,0.184314}%
\pgfsetfillcolor{currentfill}%
\pgfsetfillopacity{0.500000}%
\pgfsetlinewidth{0.250937pt}%
\definecolor{currentstroke}{rgb}{0.000000,0.000000,0.000000}%
\pgfsetstrokecolor{currentstroke}%
\pgfsetstrokeopacity{0.500000}%
\pgfsetdash{}{0pt}%
\pgfsys@defobject{currentmarker}{\pgfqpoint{-0.047917in}{-0.047917in}}{\pgfqpoint{0.047917in}{0.047917in}}{%
\pgfpathmoveto{\pgfqpoint{0.000000in}{-0.047917in}}%
\pgfpathcurveto{\pgfqpoint{0.012708in}{-0.047917in}}{\pgfqpoint{0.024897in}{-0.042868in}}{\pgfqpoint{0.033882in}{-0.033882in}}%
\pgfpathcurveto{\pgfqpoint{0.042868in}{-0.024897in}}{\pgfqpoint{0.047917in}{-0.012708in}}{\pgfqpoint{0.047917in}{0.000000in}}%
\pgfpathcurveto{\pgfqpoint{0.047917in}{0.012708in}}{\pgfqpoint{0.042868in}{0.024897in}}{\pgfqpoint{0.033882in}{0.033882in}}%
\pgfpathcurveto{\pgfqpoint{0.024897in}{0.042868in}}{\pgfqpoint{0.012708in}{0.047917in}}{\pgfqpoint{0.000000in}{0.047917in}}%
\pgfpathcurveto{\pgfqpoint{-0.012708in}{0.047917in}}{\pgfqpoint{-0.024897in}{0.042868in}}{\pgfqpoint{-0.033882in}{0.033882in}}%
\pgfpathcurveto{\pgfqpoint{-0.042868in}{0.024897in}}{\pgfqpoint{-0.047917in}{0.012708in}}{\pgfqpoint{-0.047917in}{0.000000in}}%
\pgfpathcurveto{\pgfqpoint{-0.047917in}{-0.012708in}}{\pgfqpoint{-0.042868in}{-0.024897in}}{\pgfqpoint{-0.033882in}{-0.033882in}}%
\pgfpathcurveto{\pgfqpoint{-0.024897in}{-0.042868in}}{\pgfqpoint{-0.012708in}{-0.047917in}}{\pgfqpoint{0.000000in}{-0.047917in}}%
\pgfpathclose%
\pgfusepath{stroke,fill}%
}%
\begin{pgfscope}%
\pgfsys@transformshift{3.171105in}{2.276171in}%
\pgfsys@useobject{currentmarker}{}%
\end{pgfscope}%
\end{pgfscope}%
\begin{pgfscope}%
\pgfpathrectangle{\pgfqpoint{0.100000in}{0.100000in}}{\pgfqpoint{5.307240in}{3.397500in}}%
\pgfusepath{clip}%
\pgfsetrectcap%
\pgfsetroundjoin%
\pgfsetlinewidth{1.505625pt}%
\definecolor{currentstroke}{rgb}{0.678431,1.000000,0.184314}%
\pgfsetstrokecolor{currentstroke}%
\pgfsetstrokeopacity{0.500000}%
\pgfsetdash{}{0pt}%
\pgfpathmoveto{\pgfqpoint{3.339588in}{2.274616in}}%
\pgfusepath{stroke}%
\end{pgfscope}%
\begin{pgfscope}%
\pgfpathrectangle{\pgfqpoint{0.100000in}{0.100000in}}{\pgfqpoint{5.307240in}{3.397500in}}%
\pgfusepath{clip}%
\pgfsetbuttcap%
\pgfsetroundjoin%
\definecolor{currentfill}{rgb}{0.678431,1.000000,0.184314}%
\pgfsetfillcolor{currentfill}%
\pgfsetfillopacity{0.500000}%
\pgfsetlinewidth{0.250937pt}%
\definecolor{currentstroke}{rgb}{0.000000,0.000000,0.000000}%
\pgfsetstrokecolor{currentstroke}%
\pgfsetstrokeopacity{0.500000}%
\pgfsetdash{}{0pt}%
\pgfsys@defobject{currentmarker}{\pgfqpoint{-0.066667in}{-0.066667in}}{\pgfqpoint{0.066667in}{0.066667in}}{%
\pgfpathmoveto{\pgfqpoint{0.000000in}{-0.066667in}}%
\pgfpathcurveto{\pgfqpoint{0.017680in}{-0.066667in}}{\pgfqpoint{0.034639in}{-0.059642in}}{\pgfqpoint{0.047140in}{-0.047140in}}%
\pgfpathcurveto{\pgfqpoint{0.059642in}{-0.034639in}}{\pgfqpoint{0.066667in}{-0.017680in}}{\pgfqpoint{0.066667in}{0.000000in}}%
\pgfpathcurveto{\pgfqpoint{0.066667in}{0.017680in}}{\pgfqpoint{0.059642in}{0.034639in}}{\pgfqpoint{0.047140in}{0.047140in}}%
\pgfpathcurveto{\pgfqpoint{0.034639in}{0.059642in}}{\pgfqpoint{0.017680in}{0.066667in}}{\pgfqpoint{0.000000in}{0.066667in}}%
\pgfpathcurveto{\pgfqpoint{-0.017680in}{0.066667in}}{\pgfqpoint{-0.034639in}{0.059642in}}{\pgfqpoint{-0.047140in}{0.047140in}}%
\pgfpathcurveto{\pgfqpoint{-0.059642in}{0.034639in}}{\pgfqpoint{-0.066667in}{0.017680in}}{\pgfqpoint{-0.066667in}{0.000000in}}%
\pgfpathcurveto{\pgfqpoint{-0.066667in}{-0.017680in}}{\pgfqpoint{-0.059642in}{-0.034639in}}{\pgfqpoint{-0.047140in}{-0.047140in}}%
\pgfpathcurveto{\pgfqpoint{-0.034639in}{-0.059642in}}{\pgfqpoint{-0.017680in}{-0.066667in}}{\pgfqpoint{0.000000in}{-0.066667in}}%
\pgfpathclose%
\pgfusepath{stroke,fill}%
}%
\begin{pgfscope}%
\pgfsys@transformshift{3.339588in}{2.274616in}%
\pgfsys@useobject{currentmarker}{}%
\end{pgfscope}%
\end{pgfscope}%
\begin{pgfscope}%
\pgfpathrectangle{\pgfqpoint{0.100000in}{0.100000in}}{\pgfqpoint{5.307240in}{3.397500in}}%
\pgfusepath{clip}%
\pgfsetrectcap%
\pgfsetroundjoin%
\pgfsetlinewidth{1.505625pt}%
\definecolor{currentstroke}{rgb}{0.678431,1.000000,0.184314}%
\pgfsetstrokecolor{currentstroke}%
\pgfsetstrokeopacity{0.500000}%
\pgfsetdash{}{0pt}%
\pgfpathmoveto{\pgfqpoint{3.173038in}{2.225486in}}%
\pgfusepath{stroke}%
\end{pgfscope}%
\begin{pgfscope}%
\pgfpathrectangle{\pgfqpoint{0.100000in}{0.100000in}}{\pgfqpoint{5.307240in}{3.397500in}}%
\pgfusepath{clip}%
\pgfsetbuttcap%
\pgfsetroundjoin%
\definecolor{currentfill}{rgb}{0.678431,1.000000,0.184314}%
\pgfsetfillcolor{currentfill}%
\pgfsetfillopacity{0.500000}%
\pgfsetlinewidth{0.250937pt}%
\definecolor{currentstroke}{rgb}{0.000000,0.000000,0.000000}%
\pgfsetstrokecolor{currentstroke}%
\pgfsetstrokeopacity{0.500000}%
\pgfsetdash{}{0pt}%
\pgfsys@defobject{currentmarker}{\pgfqpoint{-0.062500in}{-0.062500in}}{\pgfqpoint{0.062500in}{0.062500in}}{%
\pgfpathmoveto{\pgfqpoint{0.000000in}{-0.062500in}}%
\pgfpathcurveto{\pgfqpoint{0.016575in}{-0.062500in}}{\pgfqpoint{0.032474in}{-0.055915in}}{\pgfqpoint{0.044194in}{-0.044194in}}%
\pgfpathcurveto{\pgfqpoint{0.055915in}{-0.032474in}}{\pgfqpoint{0.062500in}{-0.016575in}}{\pgfqpoint{0.062500in}{0.000000in}}%
\pgfpathcurveto{\pgfqpoint{0.062500in}{0.016575in}}{\pgfqpoint{0.055915in}{0.032474in}}{\pgfqpoint{0.044194in}{0.044194in}}%
\pgfpathcurveto{\pgfqpoint{0.032474in}{0.055915in}}{\pgfqpoint{0.016575in}{0.062500in}}{\pgfqpoint{0.000000in}{0.062500in}}%
\pgfpathcurveto{\pgfqpoint{-0.016575in}{0.062500in}}{\pgfqpoint{-0.032474in}{0.055915in}}{\pgfqpoint{-0.044194in}{0.044194in}}%
\pgfpathcurveto{\pgfqpoint{-0.055915in}{0.032474in}}{\pgfqpoint{-0.062500in}{0.016575in}}{\pgfqpoint{-0.062500in}{0.000000in}}%
\pgfpathcurveto{\pgfqpoint{-0.062500in}{-0.016575in}}{\pgfqpoint{-0.055915in}{-0.032474in}}{\pgfqpoint{-0.044194in}{-0.044194in}}%
\pgfpathcurveto{\pgfqpoint{-0.032474in}{-0.055915in}}{\pgfqpoint{-0.016575in}{-0.062500in}}{\pgfqpoint{0.000000in}{-0.062500in}}%
\pgfpathclose%
\pgfusepath{stroke,fill}%
}%
\begin{pgfscope}%
\pgfsys@transformshift{3.173038in}{2.225486in}%
\pgfsys@useobject{currentmarker}{}%
\end{pgfscope}%
\end{pgfscope}%
\begin{pgfscope}%
\pgfpathrectangle{\pgfqpoint{0.100000in}{0.100000in}}{\pgfqpoint{5.307240in}{3.397500in}}%
\pgfusepath{clip}%
\pgfsetrectcap%
\pgfsetroundjoin%
\pgfsetlinewidth{1.505625pt}%
\definecolor{currentstroke}{rgb}{0.678431,1.000000,0.184314}%
\pgfsetstrokecolor{currentstroke}%
\pgfsetstrokeopacity{0.500000}%
\pgfsetdash{}{0pt}%
\pgfpathmoveto{\pgfqpoint{3.423617in}{2.339310in}}%
\pgfusepath{stroke}%
\end{pgfscope}%
\begin{pgfscope}%
\pgfpathrectangle{\pgfqpoint{0.100000in}{0.100000in}}{\pgfqpoint{5.307240in}{3.397500in}}%
\pgfusepath{clip}%
\pgfsetbuttcap%
\pgfsetroundjoin%
\definecolor{currentfill}{rgb}{0.678431,1.000000,0.184314}%
\pgfsetfillcolor{currentfill}%
\pgfsetfillopacity{0.500000}%
\pgfsetlinewidth{0.250937pt}%
\definecolor{currentstroke}{rgb}{0.000000,0.000000,0.000000}%
\pgfsetstrokecolor{currentstroke}%
\pgfsetstrokeopacity{0.500000}%
\pgfsetdash{}{0pt}%
\pgfsys@defobject{currentmarker}{\pgfqpoint{-0.072222in}{-0.072222in}}{\pgfqpoint{0.072222in}{0.072222in}}{%
\pgfpathmoveto{\pgfqpoint{0.000000in}{-0.072222in}}%
\pgfpathcurveto{\pgfqpoint{0.019154in}{-0.072222in}}{\pgfqpoint{0.037525in}{-0.064612in}}{\pgfqpoint{0.051069in}{-0.051069in}}%
\pgfpathcurveto{\pgfqpoint{0.064612in}{-0.037525in}}{\pgfqpoint{0.072222in}{-0.019154in}}{\pgfqpoint{0.072222in}{0.000000in}}%
\pgfpathcurveto{\pgfqpoint{0.072222in}{0.019154in}}{\pgfqpoint{0.064612in}{0.037525in}}{\pgfqpoint{0.051069in}{0.051069in}}%
\pgfpathcurveto{\pgfqpoint{0.037525in}{0.064612in}}{\pgfqpoint{0.019154in}{0.072222in}}{\pgfqpoint{0.000000in}{0.072222in}}%
\pgfpathcurveto{\pgfqpoint{-0.019154in}{0.072222in}}{\pgfqpoint{-0.037525in}{0.064612in}}{\pgfqpoint{-0.051069in}{0.051069in}}%
\pgfpathcurveto{\pgfqpoint{-0.064612in}{0.037525in}}{\pgfqpoint{-0.072222in}{0.019154in}}{\pgfqpoint{-0.072222in}{0.000000in}}%
\pgfpathcurveto{\pgfqpoint{-0.072222in}{-0.019154in}}{\pgfqpoint{-0.064612in}{-0.037525in}}{\pgfqpoint{-0.051069in}{-0.051069in}}%
\pgfpathcurveto{\pgfqpoint{-0.037525in}{-0.064612in}}{\pgfqpoint{-0.019154in}{-0.072222in}}{\pgfqpoint{0.000000in}{-0.072222in}}%
\pgfpathclose%
\pgfusepath{stroke,fill}%
}%
\begin{pgfscope}%
\pgfsys@transformshift{3.423617in}{2.339310in}%
\pgfsys@useobject{currentmarker}{}%
\end{pgfscope}%
\end{pgfscope}%
\begin{pgfscope}%
\pgfpathrectangle{\pgfqpoint{0.100000in}{0.100000in}}{\pgfqpoint{5.307240in}{3.397500in}}%
\pgfusepath{clip}%
\pgfsetrectcap%
\pgfsetroundjoin%
\pgfsetlinewidth{1.505625pt}%
\definecolor{currentstroke}{rgb}{0.678431,1.000000,0.184314}%
\pgfsetstrokecolor{currentstroke}%
\pgfsetstrokeopacity{0.500000}%
\pgfsetdash{}{0pt}%
\pgfpathmoveto{\pgfqpoint{3.353135in}{2.238480in}}%
\pgfusepath{stroke}%
\end{pgfscope}%
\begin{pgfscope}%
\pgfpathrectangle{\pgfqpoint{0.100000in}{0.100000in}}{\pgfqpoint{5.307240in}{3.397500in}}%
\pgfusepath{clip}%
\pgfsetbuttcap%
\pgfsetroundjoin%
\definecolor{currentfill}{rgb}{0.678431,1.000000,0.184314}%
\pgfsetfillcolor{currentfill}%
\pgfsetfillopacity{0.500000}%
\pgfsetlinewidth{0.250937pt}%
\definecolor{currentstroke}{rgb}{0.000000,0.000000,0.000000}%
\pgfsetstrokecolor{currentstroke}%
\pgfsetstrokeopacity{0.500000}%
\pgfsetdash{}{0pt}%
\pgfsys@defobject{currentmarker}{\pgfqpoint{-0.052778in}{-0.052778in}}{\pgfqpoint{0.052778in}{0.052778in}}{%
\pgfpathmoveto{\pgfqpoint{0.000000in}{-0.052778in}}%
\pgfpathcurveto{\pgfqpoint{0.013997in}{-0.052778in}}{\pgfqpoint{0.027422in}{-0.047217in}}{\pgfqpoint{0.037320in}{-0.037320in}}%
\pgfpathcurveto{\pgfqpoint{0.047217in}{-0.027422in}}{\pgfqpoint{0.052778in}{-0.013997in}}{\pgfqpoint{0.052778in}{0.000000in}}%
\pgfpathcurveto{\pgfqpoint{0.052778in}{0.013997in}}{\pgfqpoint{0.047217in}{0.027422in}}{\pgfqpoint{0.037320in}{0.037320in}}%
\pgfpathcurveto{\pgfqpoint{0.027422in}{0.047217in}}{\pgfqpoint{0.013997in}{0.052778in}}{\pgfqpoint{0.000000in}{0.052778in}}%
\pgfpathcurveto{\pgfqpoint{-0.013997in}{0.052778in}}{\pgfqpoint{-0.027422in}{0.047217in}}{\pgfqpoint{-0.037320in}{0.037320in}}%
\pgfpathcurveto{\pgfqpoint{-0.047217in}{0.027422in}}{\pgfqpoint{-0.052778in}{0.013997in}}{\pgfqpoint{-0.052778in}{0.000000in}}%
\pgfpathcurveto{\pgfqpoint{-0.052778in}{-0.013997in}}{\pgfqpoint{-0.047217in}{-0.027422in}}{\pgfqpoint{-0.037320in}{-0.037320in}}%
\pgfpathcurveto{\pgfqpoint{-0.027422in}{-0.047217in}}{\pgfqpoint{-0.013997in}{-0.052778in}}{\pgfqpoint{0.000000in}{-0.052778in}}%
\pgfpathclose%
\pgfusepath{stroke,fill}%
}%
\begin{pgfscope}%
\pgfsys@transformshift{3.353135in}{2.238480in}%
\pgfsys@useobject{currentmarker}{}%
\end{pgfscope}%
\end{pgfscope}%
\begin{pgfscope}%
\pgfpathrectangle{\pgfqpoint{0.100000in}{0.100000in}}{\pgfqpoint{5.307240in}{3.397500in}}%
\pgfusepath{clip}%
\pgfsetrectcap%
\pgfsetroundjoin%
\pgfsetlinewidth{1.505625pt}%
\definecolor{currentstroke}{rgb}{0.678431,1.000000,0.184314}%
\pgfsetstrokecolor{currentstroke}%
\pgfsetstrokeopacity{0.500000}%
\pgfsetdash{}{0pt}%
\pgfpathmoveto{\pgfqpoint{2.930813in}{2.330908in}}%
\pgfusepath{stroke}%
\end{pgfscope}%
\begin{pgfscope}%
\pgfpathrectangle{\pgfqpoint{0.100000in}{0.100000in}}{\pgfqpoint{5.307240in}{3.397500in}}%
\pgfusepath{clip}%
\pgfsetbuttcap%
\pgfsetroundjoin%
\definecolor{currentfill}{rgb}{0.678431,1.000000,0.184314}%
\pgfsetfillcolor{currentfill}%
\pgfsetfillopacity{0.500000}%
\pgfsetlinewidth{0.250937pt}%
\definecolor{currentstroke}{rgb}{0.000000,0.000000,0.000000}%
\pgfsetstrokecolor{currentstroke}%
\pgfsetstrokeopacity{0.500000}%
\pgfsetdash{}{0pt}%
\pgfsys@defobject{currentmarker}{\pgfqpoint{-0.049306in}{-0.049306in}}{\pgfqpoint{0.049306in}{0.049306in}}{%
\pgfpathmoveto{\pgfqpoint{0.000000in}{-0.049306in}}%
\pgfpathcurveto{\pgfqpoint{0.013076in}{-0.049306in}}{\pgfqpoint{0.025618in}{-0.044110in}}{\pgfqpoint{0.034864in}{-0.034864in}}%
\pgfpathcurveto{\pgfqpoint{0.044110in}{-0.025618in}}{\pgfqpoint{0.049306in}{-0.013076in}}{\pgfqpoint{0.049306in}{0.000000in}}%
\pgfpathcurveto{\pgfqpoint{0.049306in}{0.013076in}}{\pgfqpoint{0.044110in}{0.025618in}}{\pgfqpoint{0.034864in}{0.034864in}}%
\pgfpathcurveto{\pgfqpoint{0.025618in}{0.044110in}}{\pgfqpoint{0.013076in}{0.049306in}}{\pgfqpoint{0.000000in}{0.049306in}}%
\pgfpathcurveto{\pgfqpoint{-0.013076in}{0.049306in}}{\pgfqpoint{-0.025618in}{0.044110in}}{\pgfqpoint{-0.034864in}{0.034864in}}%
\pgfpathcurveto{\pgfqpoint{-0.044110in}{0.025618in}}{\pgfqpoint{-0.049306in}{0.013076in}}{\pgfqpoint{-0.049306in}{0.000000in}}%
\pgfpathcurveto{\pgfqpoint{-0.049306in}{-0.013076in}}{\pgfqpoint{-0.044110in}{-0.025618in}}{\pgfqpoint{-0.034864in}{-0.034864in}}%
\pgfpathcurveto{\pgfqpoint{-0.025618in}{-0.044110in}}{\pgfqpoint{-0.013076in}{-0.049306in}}{\pgfqpoint{0.000000in}{-0.049306in}}%
\pgfpathclose%
\pgfusepath{stroke,fill}%
}%
\begin{pgfscope}%
\pgfsys@transformshift{2.930813in}{2.330908in}%
\pgfsys@useobject{currentmarker}{}%
\end{pgfscope}%
\end{pgfscope}%
\begin{pgfscope}%
\pgfpathrectangle{\pgfqpoint{0.100000in}{0.100000in}}{\pgfqpoint{5.307240in}{3.397500in}}%
\pgfusepath{clip}%
\pgfsetrectcap%
\pgfsetroundjoin%
\pgfsetlinewidth{1.505625pt}%
\definecolor{currentstroke}{rgb}{0.678431,1.000000,0.184314}%
\pgfsetstrokecolor{currentstroke}%
\pgfsetstrokeopacity{0.500000}%
\pgfsetdash{}{0pt}%
\pgfpathmoveto{\pgfqpoint{3.280165in}{2.333511in}}%
\pgfusepath{stroke}%
\end{pgfscope}%
\begin{pgfscope}%
\pgfpathrectangle{\pgfqpoint{0.100000in}{0.100000in}}{\pgfqpoint{5.307240in}{3.397500in}}%
\pgfusepath{clip}%
\pgfsetbuttcap%
\pgfsetroundjoin%
\definecolor{currentfill}{rgb}{0.678431,1.000000,0.184314}%
\pgfsetfillcolor{currentfill}%
\pgfsetfillopacity{0.500000}%
\pgfsetlinewidth{0.250937pt}%
\definecolor{currentstroke}{rgb}{0.000000,0.000000,0.000000}%
\pgfsetstrokecolor{currentstroke}%
\pgfsetstrokeopacity{0.500000}%
\pgfsetdash{}{0pt}%
\pgfsys@defobject{currentmarker}{\pgfqpoint{-0.059028in}{-0.059028in}}{\pgfqpoint{0.059028in}{0.059028in}}{%
\pgfpathmoveto{\pgfqpoint{0.000000in}{-0.059028in}}%
\pgfpathcurveto{\pgfqpoint{0.015654in}{-0.059028in}}{\pgfqpoint{0.030670in}{-0.052808in}}{\pgfqpoint{0.041739in}{-0.041739in}}%
\pgfpathcurveto{\pgfqpoint{0.052808in}{-0.030670in}}{\pgfqpoint{0.059028in}{-0.015654in}}{\pgfqpoint{0.059028in}{0.000000in}}%
\pgfpathcurveto{\pgfqpoint{0.059028in}{0.015654in}}{\pgfqpoint{0.052808in}{0.030670in}}{\pgfqpoint{0.041739in}{0.041739in}}%
\pgfpathcurveto{\pgfqpoint{0.030670in}{0.052808in}}{\pgfqpoint{0.015654in}{0.059028in}}{\pgfqpoint{0.000000in}{0.059028in}}%
\pgfpathcurveto{\pgfqpoint{-0.015654in}{0.059028in}}{\pgfqpoint{-0.030670in}{0.052808in}}{\pgfqpoint{-0.041739in}{0.041739in}}%
\pgfpathcurveto{\pgfqpoint{-0.052808in}{0.030670in}}{\pgfqpoint{-0.059028in}{0.015654in}}{\pgfqpoint{-0.059028in}{0.000000in}}%
\pgfpathcurveto{\pgfqpoint{-0.059028in}{-0.015654in}}{\pgfqpoint{-0.052808in}{-0.030670in}}{\pgfqpoint{-0.041739in}{-0.041739in}}%
\pgfpathcurveto{\pgfqpoint{-0.030670in}{-0.052808in}}{\pgfqpoint{-0.015654in}{-0.059028in}}{\pgfqpoint{0.000000in}{-0.059028in}}%
\pgfpathclose%
\pgfusepath{stroke,fill}%
}%
\begin{pgfscope}%
\pgfsys@transformshift{3.280165in}{2.333511in}%
\pgfsys@useobject{currentmarker}{}%
\end{pgfscope}%
\end{pgfscope}%
\begin{pgfscope}%
\pgfpathrectangle{\pgfqpoint{0.100000in}{0.100000in}}{\pgfqpoint{5.307240in}{3.397500in}}%
\pgfusepath{clip}%
\pgfsetrectcap%
\pgfsetroundjoin%
\pgfsetlinewidth{1.505625pt}%
\definecolor{currentstroke}{rgb}{0.678431,1.000000,0.184314}%
\pgfsetstrokecolor{currentstroke}%
\pgfsetstrokeopacity{0.500000}%
\pgfsetdash{}{0pt}%
\pgfpathmoveto{\pgfqpoint{3.030202in}{1.920014in}}%
\pgfusepath{stroke}%
\end{pgfscope}%
\begin{pgfscope}%
\pgfpathrectangle{\pgfqpoint{0.100000in}{0.100000in}}{\pgfqpoint{5.307240in}{3.397500in}}%
\pgfusepath{clip}%
\pgfsetbuttcap%
\pgfsetroundjoin%
\definecolor{currentfill}{rgb}{0.678431,1.000000,0.184314}%
\pgfsetfillcolor{currentfill}%
\pgfsetfillopacity{0.500000}%
\pgfsetlinewidth{0.250937pt}%
\definecolor{currentstroke}{rgb}{0.000000,0.000000,0.000000}%
\pgfsetstrokecolor{currentstroke}%
\pgfsetstrokeopacity{0.500000}%
\pgfsetdash{}{0pt}%
\pgfsys@defobject{currentmarker}{\pgfqpoint{-0.060417in}{-0.060417in}}{\pgfqpoint{0.060417in}{0.060417in}}{%
\pgfpathmoveto{\pgfqpoint{0.000000in}{-0.060417in}}%
\pgfpathcurveto{\pgfqpoint{0.016023in}{-0.060417in}}{\pgfqpoint{0.031391in}{-0.054051in}}{\pgfqpoint{0.042721in}{-0.042721in}}%
\pgfpathcurveto{\pgfqpoint{0.054051in}{-0.031391in}}{\pgfqpoint{0.060417in}{-0.016023in}}{\pgfqpoint{0.060417in}{0.000000in}}%
\pgfpathcurveto{\pgfqpoint{0.060417in}{0.016023in}}{\pgfqpoint{0.054051in}{0.031391in}}{\pgfqpoint{0.042721in}{0.042721in}}%
\pgfpathcurveto{\pgfqpoint{0.031391in}{0.054051in}}{\pgfqpoint{0.016023in}{0.060417in}}{\pgfqpoint{0.000000in}{0.060417in}}%
\pgfpathcurveto{\pgfqpoint{-0.016023in}{0.060417in}}{\pgfqpoint{-0.031391in}{0.054051in}}{\pgfqpoint{-0.042721in}{0.042721in}}%
\pgfpathcurveto{\pgfqpoint{-0.054051in}{0.031391in}}{\pgfqpoint{-0.060417in}{0.016023in}}{\pgfqpoint{-0.060417in}{0.000000in}}%
\pgfpathcurveto{\pgfqpoint{-0.060417in}{-0.016023in}}{\pgfqpoint{-0.054051in}{-0.031391in}}{\pgfqpoint{-0.042721in}{-0.042721in}}%
\pgfpathcurveto{\pgfqpoint{-0.031391in}{-0.054051in}}{\pgfqpoint{-0.016023in}{-0.060417in}}{\pgfqpoint{0.000000in}{-0.060417in}}%
\pgfpathclose%
\pgfusepath{stroke,fill}%
}%
\begin{pgfscope}%
\pgfsys@transformshift{3.030202in}{1.920014in}%
\pgfsys@useobject{currentmarker}{}%
\end{pgfscope}%
\end{pgfscope}%
\begin{pgfscope}%
\pgfpathrectangle{\pgfqpoint{0.100000in}{0.100000in}}{\pgfqpoint{5.307240in}{3.397500in}}%
\pgfusepath{clip}%
\pgfsetrectcap%
\pgfsetroundjoin%
\pgfsetlinewidth{1.505625pt}%
\definecolor{currentstroke}{rgb}{0.678431,1.000000,0.184314}%
\pgfsetstrokecolor{currentstroke}%
\pgfsetstrokeopacity{0.500000}%
\pgfsetdash{}{0pt}%
\pgfpathmoveto{\pgfqpoint{2.909914in}{1.945818in}}%
\pgfusepath{stroke}%
\end{pgfscope}%
\begin{pgfscope}%
\pgfpathrectangle{\pgfqpoint{0.100000in}{0.100000in}}{\pgfqpoint{5.307240in}{3.397500in}}%
\pgfusepath{clip}%
\pgfsetbuttcap%
\pgfsetroundjoin%
\definecolor{currentfill}{rgb}{0.678431,1.000000,0.184314}%
\pgfsetfillcolor{currentfill}%
\pgfsetfillopacity{0.500000}%
\pgfsetlinewidth{0.250937pt}%
\definecolor{currentstroke}{rgb}{0.000000,0.000000,0.000000}%
\pgfsetstrokecolor{currentstroke}%
\pgfsetstrokeopacity{0.500000}%
\pgfsetdash{}{0pt}%
\pgfsys@defobject{currentmarker}{\pgfqpoint{-0.041667in}{-0.041667in}}{\pgfqpoint{0.041667in}{0.041667in}}{%
\pgfpathmoveto{\pgfqpoint{0.000000in}{-0.041667in}}%
\pgfpathcurveto{\pgfqpoint{0.011050in}{-0.041667in}}{\pgfqpoint{0.021649in}{-0.037276in}}{\pgfqpoint{0.029463in}{-0.029463in}}%
\pgfpathcurveto{\pgfqpoint{0.037276in}{-0.021649in}}{\pgfqpoint{0.041667in}{-0.011050in}}{\pgfqpoint{0.041667in}{0.000000in}}%
\pgfpathcurveto{\pgfqpoint{0.041667in}{0.011050in}}{\pgfqpoint{0.037276in}{0.021649in}}{\pgfqpoint{0.029463in}{0.029463in}}%
\pgfpathcurveto{\pgfqpoint{0.021649in}{0.037276in}}{\pgfqpoint{0.011050in}{0.041667in}}{\pgfqpoint{0.000000in}{0.041667in}}%
\pgfpathcurveto{\pgfqpoint{-0.011050in}{0.041667in}}{\pgfqpoint{-0.021649in}{0.037276in}}{\pgfqpoint{-0.029463in}{0.029463in}}%
\pgfpathcurveto{\pgfqpoint{-0.037276in}{0.021649in}}{\pgfqpoint{-0.041667in}{0.011050in}}{\pgfqpoint{-0.041667in}{0.000000in}}%
\pgfpathcurveto{\pgfqpoint{-0.041667in}{-0.011050in}}{\pgfqpoint{-0.037276in}{-0.021649in}}{\pgfqpoint{-0.029463in}{-0.029463in}}%
\pgfpathcurveto{\pgfqpoint{-0.021649in}{-0.037276in}}{\pgfqpoint{-0.011050in}{-0.041667in}}{\pgfqpoint{0.000000in}{-0.041667in}}%
\pgfpathclose%
\pgfusepath{stroke,fill}%
}%
\begin{pgfscope}%
\pgfsys@transformshift{2.909914in}{1.945818in}%
\pgfsys@useobject{currentmarker}{}%
\end{pgfscope}%
\end{pgfscope}%
\begin{pgfscope}%
\pgfpathrectangle{\pgfqpoint{0.100000in}{0.100000in}}{\pgfqpoint{5.307240in}{3.397500in}}%
\pgfusepath{clip}%
\pgfsetrectcap%
\pgfsetroundjoin%
\pgfsetlinewidth{1.505625pt}%
\definecolor{currentstroke}{rgb}{0.678431,1.000000,0.184314}%
\pgfsetstrokecolor{currentstroke}%
\pgfsetstrokeopacity{0.500000}%
\pgfsetdash{}{0pt}%
\pgfpathmoveto{\pgfqpoint{2.990358in}{1.929174in}}%
\pgfusepath{stroke}%
\end{pgfscope}%
\begin{pgfscope}%
\pgfpathrectangle{\pgfqpoint{0.100000in}{0.100000in}}{\pgfqpoint{5.307240in}{3.397500in}}%
\pgfusepath{clip}%
\pgfsetbuttcap%
\pgfsetroundjoin%
\definecolor{currentfill}{rgb}{0.678431,1.000000,0.184314}%
\pgfsetfillcolor{currentfill}%
\pgfsetfillopacity{0.500000}%
\pgfsetlinewidth{0.250937pt}%
\definecolor{currentstroke}{rgb}{0.000000,0.000000,0.000000}%
\pgfsetstrokecolor{currentstroke}%
\pgfsetstrokeopacity{0.500000}%
\pgfsetdash{}{0pt}%
\pgfsys@defobject{currentmarker}{\pgfqpoint{-0.061806in}{-0.061806in}}{\pgfqpoint{0.061806in}{0.061806in}}{%
\pgfpathmoveto{\pgfqpoint{0.000000in}{-0.061806in}}%
\pgfpathcurveto{\pgfqpoint{0.016391in}{-0.061806in}}{\pgfqpoint{0.032113in}{-0.055293in}}{\pgfqpoint{0.043703in}{-0.043703in}}%
\pgfpathcurveto{\pgfqpoint{0.055293in}{-0.032113in}}{\pgfqpoint{0.061806in}{-0.016391in}}{\pgfqpoint{0.061806in}{0.000000in}}%
\pgfpathcurveto{\pgfqpoint{0.061806in}{0.016391in}}{\pgfqpoint{0.055293in}{0.032113in}}{\pgfqpoint{0.043703in}{0.043703in}}%
\pgfpathcurveto{\pgfqpoint{0.032113in}{0.055293in}}{\pgfqpoint{0.016391in}{0.061806in}}{\pgfqpoint{0.000000in}{0.061806in}}%
\pgfpathcurveto{\pgfqpoint{-0.016391in}{0.061806in}}{\pgfqpoint{-0.032113in}{0.055293in}}{\pgfqpoint{-0.043703in}{0.043703in}}%
\pgfpathcurveto{\pgfqpoint{-0.055293in}{0.032113in}}{\pgfqpoint{-0.061806in}{0.016391in}}{\pgfqpoint{-0.061806in}{0.000000in}}%
\pgfpathcurveto{\pgfqpoint{-0.061806in}{-0.016391in}}{\pgfqpoint{-0.055293in}{-0.032113in}}{\pgfqpoint{-0.043703in}{-0.043703in}}%
\pgfpathcurveto{\pgfqpoint{-0.032113in}{-0.055293in}}{\pgfqpoint{-0.016391in}{-0.061806in}}{\pgfqpoint{0.000000in}{-0.061806in}}%
\pgfpathclose%
\pgfusepath{stroke,fill}%
}%
\begin{pgfscope}%
\pgfsys@transformshift{2.990358in}{1.929174in}%
\pgfsys@useobject{currentmarker}{}%
\end{pgfscope}%
\end{pgfscope}%
\begin{pgfscope}%
\pgfpathrectangle{\pgfqpoint{0.100000in}{0.100000in}}{\pgfqpoint{5.307240in}{3.397500in}}%
\pgfusepath{clip}%
\pgfsetrectcap%
\pgfsetroundjoin%
\pgfsetlinewidth{1.505625pt}%
\definecolor{currentstroke}{rgb}{0.678431,1.000000,0.184314}%
\pgfsetstrokecolor{currentstroke}%
\pgfsetstrokeopacity{0.500000}%
\pgfsetdash{}{0pt}%
\pgfpathmoveto{\pgfqpoint{2.836459in}{1.773972in}}%
\pgfusepath{stroke}%
\end{pgfscope}%
\begin{pgfscope}%
\pgfpathrectangle{\pgfqpoint{0.100000in}{0.100000in}}{\pgfqpoint{5.307240in}{3.397500in}}%
\pgfusepath{clip}%
\pgfsetbuttcap%
\pgfsetroundjoin%
\definecolor{currentfill}{rgb}{0.678431,1.000000,0.184314}%
\pgfsetfillcolor{currentfill}%
\pgfsetfillopacity{0.500000}%
\pgfsetlinewidth{0.250937pt}%
\definecolor{currentstroke}{rgb}{0.000000,0.000000,0.000000}%
\pgfsetstrokecolor{currentstroke}%
\pgfsetstrokeopacity{0.500000}%
\pgfsetdash{}{0pt}%
\pgfsys@defobject{currentmarker}{\pgfqpoint{-0.102778in}{-0.102778in}}{\pgfqpoint{0.102778in}{0.102778in}}{%
\pgfpathmoveto{\pgfqpoint{0.000000in}{-0.102778in}}%
\pgfpathcurveto{\pgfqpoint{0.027257in}{-0.102778in}}{\pgfqpoint{0.053401in}{-0.091948in}}{\pgfqpoint{0.072675in}{-0.072675in}}%
\pgfpathcurveto{\pgfqpoint{0.091948in}{-0.053401in}}{\pgfqpoint{0.102778in}{-0.027257in}}{\pgfqpoint{0.102778in}{0.000000in}}%
\pgfpathcurveto{\pgfqpoint{0.102778in}{0.027257in}}{\pgfqpoint{0.091948in}{0.053401in}}{\pgfqpoint{0.072675in}{0.072675in}}%
\pgfpathcurveto{\pgfqpoint{0.053401in}{0.091948in}}{\pgfqpoint{0.027257in}{0.102778in}}{\pgfqpoint{0.000000in}{0.102778in}}%
\pgfpathcurveto{\pgfqpoint{-0.027257in}{0.102778in}}{\pgfqpoint{-0.053401in}{0.091948in}}{\pgfqpoint{-0.072675in}{0.072675in}}%
\pgfpathcurveto{\pgfqpoint{-0.091948in}{0.053401in}}{\pgfqpoint{-0.102778in}{0.027257in}}{\pgfqpoint{-0.102778in}{0.000000in}}%
\pgfpathcurveto{\pgfqpoint{-0.102778in}{-0.027257in}}{\pgfqpoint{-0.091948in}{-0.053401in}}{\pgfqpoint{-0.072675in}{-0.072675in}}%
\pgfpathcurveto{\pgfqpoint{-0.053401in}{-0.091948in}}{\pgfqpoint{-0.027257in}{-0.102778in}}{\pgfqpoint{0.000000in}{-0.102778in}}%
\pgfpathclose%
\pgfusepath{stroke,fill}%
}%
\begin{pgfscope}%
\pgfsys@transformshift{2.836459in}{1.773972in}%
\pgfsys@useobject{currentmarker}{}%
\end{pgfscope}%
\end{pgfscope}%
\begin{pgfscope}%
\pgfpathrectangle{\pgfqpoint{0.100000in}{0.100000in}}{\pgfqpoint{5.307240in}{3.397500in}}%
\pgfusepath{clip}%
\pgfsetrectcap%
\pgfsetroundjoin%
\pgfsetlinewidth{1.505625pt}%
\definecolor{currentstroke}{rgb}{0.678431,1.000000,0.184314}%
\pgfsetstrokecolor{currentstroke}%
\pgfsetstrokeopacity{0.500000}%
\pgfsetdash{}{0pt}%
\pgfpathmoveto{\pgfqpoint{3.845366in}{1.726970in}}%
\pgfusepath{stroke}%
\end{pgfscope}%
\begin{pgfscope}%
\pgfpathrectangle{\pgfqpoint{0.100000in}{0.100000in}}{\pgfqpoint{5.307240in}{3.397500in}}%
\pgfusepath{clip}%
\pgfsetbuttcap%
\pgfsetroundjoin%
\definecolor{currentfill}{rgb}{0.678431,1.000000,0.184314}%
\pgfsetfillcolor{currentfill}%
\pgfsetfillopacity{0.500000}%
\pgfsetlinewidth{0.250937pt}%
\definecolor{currentstroke}{rgb}{0.000000,0.000000,0.000000}%
\pgfsetstrokecolor{currentstroke}%
\pgfsetstrokeopacity{0.500000}%
\pgfsetdash{}{0pt}%
\pgfsys@defobject{currentmarker}{\pgfqpoint{-0.098611in}{-0.098611in}}{\pgfqpoint{0.098611in}{0.098611in}}{%
\pgfpathmoveto{\pgfqpoint{0.000000in}{-0.098611in}}%
\pgfpathcurveto{\pgfqpoint{0.026152in}{-0.098611in}}{\pgfqpoint{0.051236in}{-0.088221in}}{\pgfqpoint{0.069729in}{-0.069729in}}%
\pgfpathcurveto{\pgfqpoint{0.088221in}{-0.051236in}}{\pgfqpoint{0.098611in}{-0.026152in}}{\pgfqpoint{0.098611in}{0.000000in}}%
\pgfpathcurveto{\pgfqpoint{0.098611in}{0.026152in}}{\pgfqpoint{0.088221in}{0.051236in}}{\pgfqpoint{0.069729in}{0.069729in}}%
\pgfpathcurveto{\pgfqpoint{0.051236in}{0.088221in}}{\pgfqpoint{0.026152in}{0.098611in}}{\pgfqpoint{0.000000in}{0.098611in}}%
\pgfpathcurveto{\pgfqpoint{-0.026152in}{0.098611in}}{\pgfqpoint{-0.051236in}{0.088221in}}{\pgfqpoint{-0.069729in}{0.069729in}}%
\pgfpathcurveto{\pgfqpoint{-0.088221in}{0.051236in}}{\pgfqpoint{-0.098611in}{0.026152in}}{\pgfqpoint{-0.098611in}{0.000000in}}%
\pgfpathcurveto{\pgfqpoint{-0.098611in}{-0.026152in}}{\pgfqpoint{-0.088221in}{-0.051236in}}{\pgfqpoint{-0.069729in}{-0.069729in}}%
\pgfpathcurveto{\pgfqpoint{-0.051236in}{-0.088221in}}{\pgfqpoint{-0.026152in}{-0.098611in}}{\pgfqpoint{0.000000in}{-0.098611in}}%
\pgfpathclose%
\pgfusepath{stroke,fill}%
}%
\begin{pgfscope}%
\pgfsys@transformshift{3.845366in}{1.726970in}%
\pgfsys@useobject{currentmarker}{}%
\end{pgfscope}%
\end{pgfscope}%
\begin{pgfscope}%
\pgfpathrectangle{\pgfqpoint{0.100000in}{0.100000in}}{\pgfqpoint{5.307240in}{3.397500in}}%
\pgfusepath{clip}%
\pgfsetrectcap%
\pgfsetroundjoin%
\pgfsetlinewidth{1.505625pt}%
\definecolor{currentstroke}{rgb}{0.678431,1.000000,0.184314}%
\pgfsetstrokecolor{currentstroke}%
\pgfsetstrokeopacity{0.500000}%
\pgfsetdash{}{0pt}%
\pgfpathmoveto{\pgfqpoint{3.891197in}{1.813670in}}%
\pgfusepath{stroke}%
\end{pgfscope}%
\begin{pgfscope}%
\pgfpathrectangle{\pgfqpoint{0.100000in}{0.100000in}}{\pgfqpoint{5.307240in}{3.397500in}}%
\pgfusepath{clip}%
\pgfsetbuttcap%
\pgfsetroundjoin%
\definecolor{currentfill}{rgb}{0.678431,1.000000,0.184314}%
\pgfsetfillcolor{currentfill}%
\pgfsetfillopacity{0.500000}%
\pgfsetlinewidth{0.250937pt}%
\definecolor{currentstroke}{rgb}{0.000000,0.000000,0.000000}%
\pgfsetstrokecolor{currentstroke}%
\pgfsetstrokeopacity{0.500000}%
\pgfsetdash{}{0pt}%
\pgfsys@defobject{currentmarker}{\pgfqpoint{-0.096528in}{-0.096528in}}{\pgfqpoint{0.096528in}{0.096528in}}{%
\pgfpathmoveto{\pgfqpoint{0.000000in}{-0.096528in}}%
\pgfpathcurveto{\pgfqpoint{0.025599in}{-0.096528in}}{\pgfqpoint{0.050154in}{-0.086357in}}{\pgfqpoint{0.068255in}{-0.068255in}}%
\pgfpathcurveto{\pgfqpoint{0.086357in}{-0.050154in}}{\pgfqpoint{0.096528in}{-0.025599in}}{\pgfqpoint{0.096528in}{0.000000in}}%
\pgfpathcurveto{\pgfqpoint{0.096528in}{0.025599in}}{\pgfqpoint{0.086357in}{0.050154in}}{\pgfqpoint{0.068255in}{0.068255in}}%
\pgfpathcurveto{\pgfqpoint{0.050154in}{0.086357in}}{\pgfqpoint{0.025599in}{0.096528in}}{\pgfqpoint{0.000000in}{0.096528in}}%
\pgfpathcurveto{\pgfqpoint{-0.025599in}{0.096528in}}{\pgfqpoint{-0.050154in}{0.086357in}}{\pgfqpoint{-0.068255in}{0.068255in}}%
\pgfpathcurveto{\pgfqpoint{-0.086357in}{0.050154in}}{\pgfqpoint{-0.096528in}{0.025599in}}{\pgfqpoint{-0.096528in}{0.000000in}}%
\pgfpathcurveto{\pgfqpoint{-0.096528in}{-0.025599in}}{\pgfqpoint{-0.086357in}{-0.050154in}}{\pgfqpoint{-0.068255in}{-0.068255in}}%
\pgfpathcurveto{\pgfqpoint{-0.050154in}{-0.086357in}}{\pgfqpoint{-0.025599in}{-0.096528in}}{\pgfqpoint{0.000000in}{-0.096528in}}%
\pgfpathclose%
\pgfusepath{stroke,fill}%
}%
\begin{pgfscope}%
\pgfsys@transformshift{3.891197in}{1.813670in}%
\pgfsys@useobject{currentmarker}{}%
\end{pgfscope}%
\end{pgfscope}%
\begin{pgfscope}%
\pgfpathrectangle{\pgfqpoint{0.100000in}{0.100000in}}{\pgfqpoint{5.307240in}{3.397500in}}%
\pgfusepath{clip}%
\pgfsetrectcap%
\pgfsetroundjoin%
\pgfsetlinewidth{1.505625pt}%
\definecolor{currentstroke}{rgb}{0.678431,1.000000,0.184314}%
\pgfsetstrokecolor{currentstroke}%
\pgfsetstrokeopacity{0.500000}%
\pgfsetdash{}{0pt}%
\pgfpathmoveto{\pgfqpoint{4.011069in}{1.867898in}}%
\pgfusepath{stroke}%
\end{pgfscope}%
\begin{pgfscope}%
\pgfpathrectangle{\pgfqpoint{0.100000in}{0.100000in}}{\pgfqpoint{5.307240in}{3.397500in}}%
\pgfusepath{clip}%
\pgfsetbuttcap%
\pgfsetroundjoin%
\definecolor{currentfill}{rgb}{0.678431,1.000000,0.184314}%
\pgfsetfillcolor{currentfill}%
\pgfsetfillopacity{0.500000}%
\pgfsetlinewidth{0.250937pt}%
\definecolor{currentstroke}{rgb}{0.000000,0.000000,0.000000}%
\pgfsetstrokecolor{currentstroke}%
\pgfsetstrokeopacity{0.500000}%
\pgfsetdash{}{0pt}%
\pgfsys@defobject{currentmarker}{\pgfqpoint{-0.081250in}{-0.081250in}}{\pgfqpoint{0.081250in}{0.081250in}}{%
\pgfpathmoveto{\pgfqpoint{0.000000in}{-0.081250in}}%
\pgfpathcurveto{\pgfqpoint{0.021548in}{-0.081250in}}{\pgfqpoint{0.042216in}{-0.072689in}}{\pgfqpoint{0.057452in}{-0.057452in}}%
\pgfpathcurveto{\pgfqpoint{0.072689in}{-0.042216in}}{\pgfqpoint{0.081250in}{-0.021548in}}{\pgfqpoint{0.081250in}{0.000000in}}%
\pgfpathcurveto{\pgfqpoint{0.081250in}{0.021548in}}{\pgfqpoint{0.072689in}{0.042216in}}{\pgfqpoint{0.057452in}{0.057452in}}%
\pgfpathcurveto{\pgfqpoint{0.042216in}{0.072689in}}{\pgfqpoint{0.021548in}{0.081250in}}{\pgfqpoint{0.000000in}{0.081250in}}%
\pgfpathcurveto{\pgfqpoint{-0.021548in}{0.081250in}}{\pgfqpoint{-0.042216in}{0.072689in}}{\pgfqpoint{-0.057452in}{0.057452in}}%
\pgfpathcurveto{\pgfqpoint{-0.072689in}{0.042216in}}{\pgfqpoint{-0.081250in}{0.021548in}}{\pgfqpoint{-0.081250in}{0.000000in}}%
\pgfpathcurveto{\pgfqpoint{-0.081250in}{-0.021548in}}{\pgfqpoint{-0.072689in}{-0.042216in}}{\pgfqpoint{-0.057452in}{-0.057452in}}%
\pgfpathcurveto{\pgfqpoint{-0.042216in}{-0.072689in}}{\pgfqpoint{-0.021548in}{-0.081250in}}{\pgfqpoint{0.000000in}{-0.081250in}}%
\pgfpathclose%
\pgfusepath{stroke,fill}%
}%
\begin{pgfscope}%
\pgfsys@transformshift{4.011069in}{1.867898in}%
\pgfsys@useobject{currentmarker}{}%
\end{pgfscope}%
\end{pgfscope}%
\begin{pgfscope}%
\pgfpathrectangle{\pgfqpoint{0.100000in}{0.100000in}}{\pgfqpoint{5.307240in}{3.397500in}}%
\pgfusepath{clip}%
\pgfsetrectcap%
\pgfsetroundjoin%
\pgfsetlinewidth{1.505625pt}%
\definecolor{currentstroke}{rgb}{0.678431,1.000000,0.184314}%
\pgfsetstrokecolor{currentstroke}%
\pgfsetstrokeopacity{0.500000}%
\pgfsetdash{}{0pt}%
\pgfpathmoveto{\pgfqpoint{3.893713in}{1.879406in}}%
\pgfusepath{stroke}%
\end{pgfscope}%
\begin{pgfscope}%
\pgfpathrectangle{\pgfqpoint{0.100000in}{0.100000in}}{\pgfqpoint{5.307240in}{3.397500in}}%
\pgfusepath{clip}%
\pgfsetbuttcap%
\pgfsetroundjoin%
\definecolor{currentfill}{rgb}{0.678431,1.000000,0.184314}%
\pgfsetfillcolor{currentfill}%
\pgfsetfillopacity{0.500000}%
\pgfsetlinewidth{0.250937pt}%
\definecolor{currentstroke}{rgb}{0.000000,0.000000,0.000000}%
\pgfsetstrokecolor{currentstroke}%
\pgfsetstrokeopacity{0.500000}%
\pgfsetdash{}{0pt}%
\pgfsys@defobject{currentmarker}{\pgfqpoint{-0.090972in}{-0.090972in}}{\pgfqpoint{0.090972in}{0.090972in}}{%
\pgfpathmoveto{\pgfqpoint{0.000000in}{-0.090972in}}%
\pgfpathcurveto{\pgfqpoint{0.024126in}{-0.090972in}}{\pgfqpoint{0.047267in}{-0.081387in}}{\pgfqpoint{0.064327in}{-0.064327in}}%
\pgfpathcurveto{\pgfqpoint{0.081387in}{-0.047267in}}{\pgfqpoint{0.090972in}{-0.024126in}}{\pgfqpoint{0.090972in}{0.000000in}}%
\pgfpathcurveto{\pgfqpoint{0.090972in}{0.024126in}}{\pgfqpoint{0.081387in}{0.047267in}}{\pgfqpoint{0.064327in}{0.064327in}}%
\pgfpathcurveto{\pgfqpoint{0.047267in}{0.081387in}}{\pgfqpoint{0.024126in}{0.090972in}}{\pgfqpoint{0.000000in}{0.090972in}}%
\pgfpathcurveto{\pgfqpoint{-0.024126in}{0.090972in}}{\pgfqpoint{-0.047267in}{0.081387in}}{\pgfqpoint{-0.064327in}{0.064327in}}%
\pgfpathcurveto{\pgfqpoint{-0.081387in}{0.047267in}}{\pgfqpoint{-0.090972in}{0.024126in}}{\pgfqpoint{-0.090972in}{0.000000in}}%
\pgfpathcurveto{\pgfqpoint{-0.090972in}{-0.024126in}}{\pgfqpoint{-0.081387in}{-0.047267in}}{\pgfqpoint{-0.064327in}{-0.064327in}}%
\pgfpathcurveto{\pgfqpoint{-0.047267in}{-0.081387in}}{\pgfqpoint{-0.024126in}{-0.090972in}}{\pgfqpoint{0.000000in}{-0.090972in}}%
\pgfpathclose%
\pgfusepath{stroke,fill}%
}%
\begin{pgfscope}%
\pgfsys@transformshift{3.893713in}{1.879406in}%
\pgfsys@useobject{currentmarker}{}%
\end{pgfscope}%
\end{pgfscope}%
\begin{pgfscope}%
\pgfpathrectangle{\pgfqpoint{0.100000in}{0.100000in}}{\pgfqpoint{5.307240in}{3.397500in}}%
\pgfusepath{clip}%
\pgfsetrectcap%
\pgfsetroundjoin%
\pgfsetlinewidth{1.505625pt}%
\definecolor{currentstroke}{rgb}{0.678431,1.000000,0.184314}%
\pgfsetstrokecolor{currentstroke}%
\pgfsetstrokeopacity{0.500000}%
\pgfsetdash{}{0pt}%
\pgfpathmoveto{\pgfqpoint{3.775491in}{1.812173in}}%
\pgfusepath{stroke}%
\end{pgfscope}%
\begin{pgfscope}%
\pgfpathrectangle{\pgfqpoint{0.100000in}{0.100000in}}{\pgfqpoint{5.307240in}{3.397500in}}%
\pgfusepath{clip}%
\pgfsetbuttcap%
\pgfsetroundjoin%
\definecolor{currentfill}{rgb}{0.678431,1.000000,0.184314}%
\pgfsetfillcolor{currentfill}%
\pgfsetfillopacity{0.500000}%
\pgfsetlinewidth{0.250937pt}%
\definecolor{currentstroke}{rgb}{0.000000,0.000000,0.000000}%
\pgfsetstrokecolor{currentstroke}%
\pgfsetstrokeopacity{0.500000}%
\pgfsetdash{}{0pt}%
\pgfsys@defobject{currentmarker}{\pgfqpoint{-0.073611in}{-0.073611in}}{\pgfqpoint{0.073611in}{0.073611in}}{%
\pgfpathmoveto{\pgfqpoint{0.000000in}{-0.073611in}}%
\pgfpathcurveto{\pgfqpoint{0.019522in}{-0.073611in}}{\pgfqpoint{0.038247in}{-0.065855in}}{\pgfqpoint{0.052051in}{-0.052051in}}%
\pgfpathcurveto{\pgfqpoint{0.065855in}{-0.038247in}}{\pgfqpoint{0.073611in}{-0.019522in}}{\pgfqpoint{0.073611in}{0.000000in}}%
\pgfpathcurveto{\pgfqpoint{0.073611in}{0.019522in}}{\pgfqpoint{0.065855in}{0.038247in}}{\pgfqpoint{0.052051in}{0.052051in}}%
\pgfpathcurveto{\pgfqpoint{0.038247in}{0.065855in}}{\pgfqpoint{0.019522in}{0.073611in}}{\pgfqpoint{0.000000in}{0.073611in}}%
\pgfpathcurveto{\pgfqpoint{-0.019522in}{0.073611in}}{\pgfqpoint{-0.038247in}{0.065855in}}{\pgfqpoint{-0.052051in}{0.052051in}}%
\pgfpathcurveto{\pgfqpoint{-0.065855in}{0.038247in}}{\pgfqpoint{-0.073611in}{0.019522in}}{\pgfqpoint{-0.073611in}{0.000000in}}%
\pgfpathcurveto{\pgfqpoint{-0.073611in}{-0.019522in}}{\pgfqpoint{-0.065855in}{-0.038247in}}{\pgfqpoint{-0.052051in}{-0.052051in}}%
\pgfpathcurveto{\pgfqpoint{-0.038247in}{-0.065855in}}{\pgfqpoint{-0.019522in}{-0.073611in}}{\pgfqpoint{0.000000in}{-0.073611in}}%
\pgfpathclose%
\pgfusepath{stroke,fill}%
}%
\begin{pgfscope}%
\pgfsys@transformshift{3.775491in}{1.812173in}%
\pgfsys@useobject{currentmarker}{}%
\end{pgfscope}%
\end{pgfscope}%
\begin{pgfscope}%
\pgfpathrectangle{\pgfqpoint{0.100000in}{0.100000in}}{\pgfqpoint{5.307240in}{3.397500in}}%
\pgfusepath{clip}%
\pgfsetrectcap%
\pgfsetroundjoin%
\pgfsetlinewidth{1.505625pt}%
\definecolor{currentstroke}{rgb}{0.678431,1.000000,0.184314}%
\pgfsetstrokecolor{currentstroke}%
\pgfsetstrokeopacity{0.500000}%
\pgfsetdash{}{0pt}%
\pgfpathmoveto{\pgfqpoint{3.307882in}{1.017158in}}%
\pgfusepath{stroke}%
\end{pgfscope}%
\begin{pgfscope}%
\pgfpathrectangle{\pgfqpoint{0.100000in}{0.100000in}}{\pgfqpoint{5.307240in}{3.397500in}}%
\pgfusepath{clip}%
\pgfsetbuttcap%
\pgfsetroundjoin%
\definecolor{currentfill}{rgb}{0.678431,1.000000,0.184314}%
\pgfsetfillcolor{currentfill}%
\pgfsetfillopacity{0.500000}%
\pgfsetlinewidth{0.250937pt}%
\definecolor{currentstroke}{rgb}{0.000000,0.000000,0.000000}%
\pgfsetstrokecolor{currentstroke}%
\pgfsetstrokeopacity{0.500000}%
\pgfsetdash{}{0pt}%
\pgfsys@defobject{currentmarker}{\pgfqpoint{-0.040972in}{-0.040972in}}{\pgfqpoint{0.040972in}{0.040972in}}{%
\pgfpathmoveto{\pgfqpoint{0.000000in}{-0.040972in}}%
\pgfpathcurveto{\pgfqpoint{0.010866in}{-0.040972in}}{\pgfqpoint{0.021288in}{-0.036655in}}{\pgfqpoint{0.028972in}{-0.028972in}}%
\pgfpathcurveto{\pgfqpoint{0.036655in}{-0.021288in}}{\pgfqpoint{0.040972in}{-0.010866in}}{\pgfqpoint{0.040972in}{0.000000in}}%
\pgfpathcurveto{\pgfqpoint{0.040972in}{0.010866in}}{\pgfqpoint{0.036655in}{0.021288in}}{\pgfqpoint{0.028972in}{0.028972in}}%
\pgfpathcurveto{\pgfqpoint{0.021288in}{0.036655in}}{\pgfqpoint{0.010866in}{0.040972in}}{\pgfqpoint{0.000000in}{0.040972in}}%
\pgfpathcurveto{\pgfqpoint{-0.010866in}{0.040972in}}{\pgfqpoint{-0.021288in}{0.036655in}}{\pgfqpoint{-0.028972in}{0.028972in}}%
\pgfpathcurveto{\pgfqpoint{-0.036655in}{0.021288in}}{\pgfqpoint{-0.040972in}{0.010866in}}{\pgfqpoint{-0.040972in}{0.000000in}}%
\pgfpathcurveto{\pgfqpoint{-0.040972in}{-0.010866in}}{\pgfqpoint{-0.036655in}{-0.021288in}}{\pgfqpoint{-0.028972in}{-0.028972in}}%
\pgfpathcurveto{\pgfqpoint{-0.021288in}{-0.036655in}}{\pgfqpoint{-0.010866in}{-0.040972in}}{\pgfqpoint{0.000000in}{-0.040972in}}%
\pgfpathclose%
\pgfusepath{stroke,fill}%
}%
\begin{pgfscope}%
\pgfsys@transformshift{3.307882in}{1.017158in}%
\pgfsys@useobject{currentmarker}{}%
\end{pgfscope}%
\end{pgfscope}%
\begin{pgfscope}%
\pgfpathrectangle{\pgfqpoint{0.100000in}{0.100000in}}{\pgfqpoint{5.307240in}{3.397500in}}%
\pgfusepath{clip}%
\pgfsetrectcap%
\pgfsetroundjoin%
\pgfsetlinewidth{1.505625pt}%
\definecolor{currentstroke}{rgb}{0.678431,1.000000,0.184314}%
\pgfsetstrokecolor{currentstroke}%
\pgfsetstrokeopacity{0.500000}%
\pgfsetdash{}{0pt}%
\pgfpathmoveto{\pgfqpoint{3.441032in}{0.933933in}}%
\pgfusepath{stroke}%
\end{pgfscope}%
\begin{pgfscope}%
\pgfpathrectangle{\pgfqpoint{0.100000in}{0.100000in}}{\pgfqpoint{5.307240in}{3.397500in}}%
\pgfusepath{clip}%
\pgfsetbuttcap%
\pgfsetroundjoin%
\definecolor{currentfill}{rgb}{0.678431,1.000000,0.184314}%
\pgfsetfillcolor{currentfill}%
\pgfsetfillopacity{0.500000}%
\pgfsetlinewidth{0.250937pt}%
\definecolor{currentstroke}{rgb}{0.000000,0.000000,0.000000}%
\pgfsetstrokecolor{currentstroke}%
\pgfsetstrokeopacity{0.500000}%
\pgfsetdash{}{0pt}%
\pgfsys@defobject{currentmarker}{\pgfqpoint{-0.066667in}{-0.066667in}}{\pgfqpoint{0.066667in}{0.066667in}}{%
\pgfpathmoveto{\pgfqpoint{0.000000in}{-0.066667in}}%
\pgfpathcurveto{\pgfqpoint{0.017680in}{-0.066667in}}{\pgfqpoint{0.034639in}{-0.059642in}}{\pgfqpoint{0.047140in}{-0.047140in}}%
\pgfpathcurveto{\pgfqpoint{0.059642in}{-0.034639in}}{\pgfqpoint{0.066667in}{-0.017680in}}{\pgfqpoint{0.066667in}{0.000000in}}%
\pgfpathcurveto{\pgfqpoint{0.066667in}{0.017680in}}{\pgfqpoint{0.059642in}{0.034639in}}{\pgfqpoint{0.047140in}{0.047140in}}%
\pgfpathcurveto{\pgfqpoint{0.034639in}{0.059642in}}{\pgfqpoint{0.017680in}{0.066667in}}{\pgfqpoint{0.000000in}{0.066667in}}%
\pgfpathcurveto{\pgfqpoint{-0.017680in}{0.066667in}}{\pgfqpoint{-0.034639in}{0.059642in}}{\pgfqpoint{-0.047140in}{0.047140in}}%
\pgfpathcurveto{\pgfqpoint{-0.059642in}{0.034639in}}{\pgfqpoint{-0.066667in}{0.017680in}}{\pgfqpoint{-0.066667in}{0.000000in}}%
\pgfpathcurveto{\pgfqpoint{-0.066667in}{-0.017680in}}{\pgfqpoint{-0.059642in}{-0.034639in}}{\pgfqpoint{-0.047140in}{-0.047140in}}%
\pgfpathcurveto{\pgfqpoint{-0.034639in}{-0.059642in}}{\pgfqpoint{-0.017680in}{-0.066667in}}{\pgfqpoint{0.000000in}{-0.066667in}}%
\pgfpathclose%
\pgfusepath{stroke,fill}%
}%
\begin{pgfscope}%
\pgfsys@transformshift{3.441032in}{0.933933in}%
\pgfsys@useobject{currentmarker}{}%
\end{pgfscope}%
\end{pgfscope}%
\begin{pgfscope}%
\pgfpathrectangle{\pgfqpoint{0.100000in}{0.100000in}}{\pgfqpoint{5.307240in}{3.397500in}}%
\pgfusepath{clip}%
\pgfsetrectcap%
\pgfsetroundjoin%
\pgfsetlinewidth{1.505625pt}%
\definecolor{currentstroke}{rgb}{0.678431,1.000000,0.184314}%
\pgfsetstrokecolor{currentstroke}%
\pgfsetstrokeopacity{0.500000}%
\pgfsetdash{}{0pt}%
\pgfpathmoveto{\pgfqpoint{3.703563in}{2.250578in}}%
\pgfusepath{stroke}%
\end{pgfscope}%
\begin{pgfscope}%
\pgfpathrectangle{\pgfqpoint{0.100000in}{0.100000in}}{\pgfqpoint{5.307240in}{3.397500in}}%
\pgfusepath{clip}%
\pgfsetbuttcap%
\pgfsetroundjoin%
\definecolor{currentfill}{rgb}{0.678431,1.000000,0.184314}%
\pgfsetfillcolor{currentfill}%
\pgfsetfillopacity{0.500000}%
\pgfsetlinewidth{0.250937pt}%
\definecolor{currentstroke}{rgb}{0.000000,0.000000,0.000000}%
\pgfsetstrokecolor{currentstroke}%
\pgfsetstrokeopacity{0.500000}%
\pgfsetdash{}{0pt}%
\pgfsys@defobject{currentmarker}{\pgfqpoint{-0.088889in}{-0.088889in}}{\pgfqpoint{0.088889in}{0.088889in}}{%
\pgfpathmoveto{\pgfqpoint{0.000000in}{-0.088889in}}%
\pgfpathcurveto{\pgfqpoint{0.023574in}{-0.088889in}}{\pgfqpoint{0.046185in}{-0.079523in}}{\pgfqpoint{0.062854in}{-0.062854in}}%
\pgfpathcurveto{\pgfqpoint{0.079523in}{-0.046185in}}{\pgfqpoint{0.088889in}{-0.023574in}}{\pgfqpoint{0.088889in}{0.000000in}}%
\pgfpathcurveto{\pgfqpoint{0.088889in}{0.023574in}}{\pgfqpoint{0.079523in}{0.046185in}}{\pgfqpoint{0.062854in}{0.062854in}}%
\pgfpathcurveto{\pgfqpoint{0.046185in}{0.079523in}}{\pgfqpoint{0.023574in}{0.088889in}}{\pgfqpoint{0.000000in}{0.088889in}}%
\pgfpathcurveto{\pgfqpoint{-0.023574in}{0.088889in}}{\pgfqpoint{-0.046185in}{0.079523in}}{\pgfqpoint{-0.062854in}{0.062854in}}%
\pgfpathcurveto{\pgfqpoint{-0.079523in}{0.046185in}}{\pgfqpoint{-0.088889in}{0.023574in}}{\pgfqpoint{-0.088889in}{0.000000in}}%
\pgfpathcurveto{\pgfqpoint{-0.088889in}{-0.023574in}}{\pgfqpoint{-0.079523in}{-0.046185in}}{\pgfqpoint{-0.062854in}{-0.062854in}}%
\pgfpathcurveto{\pgfqpoint{-0.046185in}{-0.079523in}}{\pgfqpoint{-0.023574in}{-0.088889in}}{\pgfqpoint{0.000000in}{-0.088889in}}%
\pgfpathclose%
\pgfusepath{stroke,fill}%
}%
\begin{pgfscope}%
\pgfsys@transformshift{3.703563in}{2.250578in}%
\pgfsys@useobject{currentmarker}{}%
\end{pgfscope}%
\end{pgfscope}%
\begin{pgfscope}%
\pgfpathrectangle{\pgfqpoint{0.100000in}{0.100000in}}{\pgfqpoint{5.307240in}{3.397500in}}%
\pgfusepath{clip}%
\pgfsetrectcap%
\pgfsetroundjoin%
\pgfsetlinewidth{1.505625pt}%
\definecolor{currentstroke}{rgb}{0.678431,1.000000,0.184314}%
\pgfsetstrokecolor{currentstroke}%
\pgfsetstrokeopacity{0.500000}%
\pgfsetdash{}{0pt}%
\pgfpathmoveto{\pgfqpoint{3.489800in}{0.835416in}}%
\pgfusepath{stroke}%
\end{pgfscope}%
\begin{pgfscope}%
\pgfpathrectangle{\pgfqpoint{0.100000in}{0.100000in}}{\pgfqpoint{5.307240in}{3.397500in}}%
\pgfusepath{clip}%
\pgfsetbuttcap%
\pgfsetroundjoin%
\definecolor{currentfill}{rgb}{0.678431,1.000000,0.184314}%
\pgfsetfillcolor{currentfill}%
\pgfsetfillopacity{0.500000}%
\pgfsetlinewidth{0.250937pt}%
\definecolor{currentstroke}{rgb}{0.000000,0.000000,0.000000}%
\pgfsetstrokecolor{currentstroke}%
\pgfsetstrokeopacity{0.500000}%
\pgfsetdash{}{0pt}%
\pgfsys@defobject{currentmarker}{\pgfqpoint{-0.060417in}{-0.060417in}}{\pgfqpoint{0.060417in}{0.060417in}}{%
\pgfpathmoveto{\pgfqpoint{0.000000in}{-0.060417in}}%
\pgfpathcurveto{\pgfqpoint{0.016023in}{-0.060417in}}{\pgfqpoint{0.031391in}{-0.054051in}}{\pgfqpoint{0.042721in}{-0.042721in}}%
\pgfpathcurveto{\pgfqpoint{0.054051in}{-0.031391in}}{\pgfqpoint{0.060417in}{-0.016023in}}{\pgfqpoint{0.060417in}{0.000000in}}%
\pgfpathcurveto{\pgfqpoint{0.060417in}{0.016023in}}{\pgfqpoint{0.054051in}{0.031391in}}{\pgfqpoint{0.042721in}{0.042721in}}%
\pgfpathcurveto{\pgfqpoint{0.031391in}{0.054051in}}{\pgfqpoint{0.016023in}{0.060417in}}{\pgfqpoint{0.000000in}{0.060417in}}%
\pgfpathcurveto{\pgfqpoint{-0.016023in}{0.060417in}}{\pgfqpoint{-0.031391in}{0.054051in}}{\pgfqpoint{-0.042721in}{0.042721in}}%
\pgfpathcurveto{\pgfqpoint{-0.054051in}{0.031391in}}{\pgfqpoint{-0.060417in}{0.016023in}}{\pgfqpoint{-0.060417in}{0.000000in}}%
\pgfpathcurveto{\pgfqpoint{-0.060417in}{-0.016023in}}{\pgfqpoint{-0.054051in}{-0.031391in}}{\pgfqpoint{-0.042721in}{-0.042721in}}%
\pgfpathcurveto{\pgfqpoint{-0.031391in}{-0.054051in}}{\pgfqpoint{-0.016023in}{-0.060417in}}{\pgfqpoint{0.000000in}{-0.060417in}}%
\pgfpathclose%
\pgfusepath{stroke,fill}%
}%
\begin{pgfscope}%
\pgfsys@transformshift{3.489800in}{0.835416in}%
\pgfsys@useobject{currentmarker}{}%
\end{pgfscope}%
\end{pgfscope}%
\begin{pgfscope}%
\pgfpathrectangle{\pgfqpoint{0.100000in}{0.100000in}}{\pgfqpoint{5.307240in}{3.397500in}}%
\pgfusepath{clip}%
\pgfsetrectcap%
\pgfsetroundjoin%
\pgfsetlinewidth{1.505625pt}%
\definecolor{currentstroke}{rgb}{0.678431,1.000000,0.184314}%
\pgfsetstrokecolor{currentstroke}%
\pgfsetstrokeopacity{0.500000}%
\pgfsetdash{}{0pt}%
\pgfpathmoveto{\pgfqpoint{3.354296in}{0.904008in}}%
\pgfusepath{stroke}%
\end{pgfscope}%
\begin{pgfscope}%
\pgfpathrectangle{\pgfqpoint{0.100000in}{0.100000in}}{\pgfqpoint{5.307240in}{3.397500in}}%
\pgfusepath{clip}%
\pgfsetbuttcap%
\pgfsetroundjoin%
\definecolor{currentfill}{rgb}{0.678431,1.000000,0.184314}%
\pgfsetfillcolor{currentfill}%
\pgfsetfillopacity{0.500000}%
\pgfsetlinewidth{0.250937pt}%
\definecolor{currentstroke}{rgb}{0.000000,0.000000,0.000000}%
\pgfsetstrokecolor{currentstroke}%
\pgfsetstrokeopacity{0.500000}%
\pgfsetdash{}{0pt}%
\pgfsys@defobject{currentmarker}{\pgfqpoint{-0.065278in}{-0.065278in}}{\pgfqpoint{0.065278in}{0.065278in}}{%
\pgfpathmoveto{\pgfqpoint{0.000000in}{-0.065278in}}%
\pgfpathcurveto{\pgfqpoint{0.017312in}{-0.065278in}}{\pgfqpoint{0.033917in}{-0.058400in}}{\pgfqpoint{0.046158in}{-0.046158in}}%
\pgfpathcurveto{\pgfqpoint{0.058400in}{-0.033917in}}{\pgfqpoint{0.065278in}{-0.017312in}}{\pgfqpoint{0.065278in}{0.000000in}}%
\pgfpathcurveto{\pgfqpoint{0.065278in}{0.017312in}}{\pgfqpoint{0.058400in}{0.033917in}}{\pgfqpoint{0.046158in}{0.046158in}}%
\pgfpathcurveto{\pgfqpoint{0.033917in}{0.058400in}}{\pgfqpoint{0.017312in}{0.065278in}}{\pgfqpoint{0.000000in}{0.065278in}}%
\pgfpathcurveto{\pgfqpoint{-0.017312in}{0.065278in}}{\pgfqpoint{-0.033917in}{0.058400in}}{\pgfqpoint{-0.046158in}{0.046158in}}%
\pgfpathcurveto{\pgfqpoint{-0.058400in}{0.033917in}}{\pgfqpoint{-0.065278in}{0.017312in}}{\pgfqpoint{-0.065278in}{0.000000in}}%
\pgfpathcurveto{\pgfqpoint{-0.065278in}{-0.017312in}}{\pgfqpoint{-0.058400in}{-0.033917in}}{\pgfqpoint{-0.046158in}{-0.046158in}}%
\pgfpathcurveto{\pgfqpoint{-0.033917in}{-0.058400in}}{\pgfqpoint{-0.017312in}{-0.065278in}}{\pgfqpoint{0.000000in}{-0.065278in}}%
\pgfpathclose%
\pgfusepath{stroke,fill}%
}%
\begin{pgfscope}%
\pgfsys@transformshift{3.354296in}{0.904008in}%
\pgfsys@useobject{currentmarker}{}%
\end{pgfscope}%
\end{pgfscope}%
\begin{pgfscope}%
\pgfpathrectangle{\pgfqpoint{0.100000in}{0.100000in}}{\pgfqpoint{5.307240in}{3.397500in}}%
\pgfusepath{clip}%
\pgfsetrectcap%
\pgfsetroundjoin%
\pgfsetlinewidth{1.505625pt}%
\definecolor{currentstroke}{rgb}{0.678431,1.000000,0.184314}%
\pgfsetstrokecolor{currentstroke}%
\pgfsetstrokeopacity{0.500000}%
\pgfsetdash{}{0pt}%
\pgfpathmoveto{\pgfqpoint{3.232648in}{0.901115in}}%
\pgfusepath{stroke}%
\end{pgfscope}%
\begin{pgfscope}%
\pgfpathrectangle{\pgfqpoint{0.100000in}{0.100000in}}{\pgfqpoint{5.307240in}{3.397500in}}%
\pgfusepath{clip}%
\pgfsetbuttcap%
\pgfsetroundjoin%
\definecolor{currentfill}{rgb}{0.678431,1.000000,0.184314}%
\pgfsetfillcolor{currentfill}%
\pgfsetfillopacity{0.500000}%
\pgfsetlinewidth{0.250937pt}%
\definecolor{currentstroke}{rgb}{0.000000,0.000000,0.000000}%
\pgfsetstrokecolor{currentstroke}%
\pgfsetstrokeopacity{0.500000}%
\pgfsetdash{}{0pt}%
\pgfsys@defobject{currentmarker}{\pgfqpoint{-0.079167in}{-0.079167in}}{\pgfqpoint{0.079167in}{0.079167in}}{%
\pgfpathmoveto{\pgfqpoint{0.000000in}{-0.079167in}}%
\pgfpathcurveto{\pgfqpoint{0.020995in}{-0.079167in}}{\pgfqpoint{0.041133in}{-0.070825in}}{\pgfqpoint{0.055979in}{-0.055979in}}%
\pgfpathcurveto{\pgfqpoint{0.070825in}{-0.041133in}}{\pgfqpoint{0.079167in}{-0.020995in}}{\pgfqpoint{0.079167in}{0.000000in}}%
\pgfpathcurveto{\pgfqpoint{0.079167in}{0.020995in}}{\pgfqpoint{0.070825in}{0.041133in}}{\pgfqpoint{0.055979in}{0.055979in}}%
\pgfpathcurveto{\pgfqpoint{0.041133in}{0.070825in}}{\pgfqpoint{0.020995in}{0.079167in}}{\pgfqpoint{0.000000in}{0.079167in}}%
\pgfpathcurveto{\pgfqpoint{-0.020995in}{0.079167in}}{\pgfqpoint{-0.041133in}{0.070825in}}{\pgfqpoint{-0.055979in}{0.055979in}}%
\pgfpathcurveto{\pgfqpoint{-0.070825in}{0.041133in}}{\pgfqpoint{-0.079167in}{0.020995in}}{\pgfqpoint{-0.079167in}{0.000000in}}%
\pgfpathcurveto{\pgfqpoint{-0.079167in}{-0.020995in}}{\pgfqpoint{-0.070825in}{-0.041133in}}{\pgfqpoint{-0.055979in}{-0.055979in}}%
\pgfpathcurveto{\pgfqpoint{-0.041133in}{-0.070825in}}{\pgfqpoint{-0.020995in}{-0.079167in}}{\pgfqpoint{0.000000in}{-0.079167in}}%
\pgfpathclose%
\pgfusepath{stroke,fill}%
}%
\begin{pgfscope}%
\pgfsys@transformshift{3.232648in}{0.901115in}%
\pgfsys@useobject{currentmarker}{}%
\end{pgfscope}%
\end{pgfscope}%
\begin{pgfscope}%
\pgfpathrectangle{\pgfqpoint{0.100000in}{0.100000in}}{\pgfqpoint{5.307240in}{3.397500in}}%
\pgfusepath{clip}%
\pgfsetrectcap%
\pgfsetroundjoin%
\pgfsetlinewidth{1.505625pt}%
\definecolor{currentstroke}{rgb}{0.678431,1.000000,0.184314}%
\pgfsetstrokecolor{currentstroke}%
\pgfsetstrokeopacity{0.500000}%
\pgfsetdash{}{0pt}%
\pgfpathmoveto{\pgfqpoint{3.491330in}{1.850311in}}%
\pgfusepath{stroke}%
\end{pgfscope}%
\begin{pgfscope}%
\pgfpathrectangle{\pgfqpoint{0.100000in}{0.100000in}}{\pgfqpoint{5.307240in}{3.397500in}}%
\pgfusepath{clip}%
\pgfsetbuttcap%
\pgfsetroundjoin%
\definecolor{currentfill}{rgb}{0.678431,1.000000,0.184314}%
\pgfsetfillcolor{currentfill}%
\pgfsetfillopacity{0.500000}%
\pgfsetlinewidth{0.250937pt}%
\definecolor{currentstroke}{rgb}{0.000000,0.000000,0.000000}%
\pgfsetstrokecolor{currentstroke}%
\pgfsetstrokeopacity{0.500000}%
\pgfsetdash{}{0pt}%
\pgfsys@defobject{currentmarker}{\pgfqpoint{-0.050694in}{-0.050694in}}{\pgfqpoint{0.050694in}{0.050694in}}{%
\pgfpathmoveto{\pgfqpoint{0.000000in}{-0.050694in}}%
\pgfpathcurveto{\pgfqpoint{0.013444in}{-0.050694in}}{\pgfqpoint{0.026340in}{-0.045353in}}{\pgfqpoint{0.035846in}{-0.035846in}}%
\pgfpathcurveto{\pgfqpoint{0.045353in}{-0.026340in}}{\pgfqpoint{0.050694in}{-0.013444in}}{\pgfqpoint{0.050694in}{0.000000in}}%
\pgfpathcurveto{\pgfqpoint{0.050694in}{0.013444in}}{\pgfqpoint{0.045353in}{0.026340in}}{\pgfqpoint{0.035846in}{0.035846in}}%
\pgfpathcurveto{\pgfqpoint{0.026340in}{0.045353in}}{\pgfqpoint{0.013444in}{0.050694in}}{\pgfqpoint{0.000000in}{0.050694in}}%
\pgfpathcurveto{\pgfqpoint{-0.013444in}{0.050694in}}{\pgfqpoint{-0.026340in}{0.045353in}}{\pgfqpoint{-0.035846in}{0.035846in}}%
\pgfpathcurveto{\pgfqpoint{-0.045353in}{0.026340in}}{\pgfqpoint{-0.050694in}{0.013444in}}{\pgfqpoint{-0.050694in}{0.000000in}}%
\pgfpathcurveto{\pgfqpoint{-0.050694in}{-0.013444in}}{\pgfqpoint{-0.045353in}{-0.026340in}}{\pgfqpoint{-0.035846in}{-0.035846in}}%
\pgfpathcurveto{\pgfqpoint{-0.026340in}{-0.045353in}}{\pgfqpoint{-0.013444in}{-0.050694in}}{\pgfqpoint{0.000000in}{-0.050694in}}%
\pgfpathclose%
\pgfusepath{stroke,fill}%
}%
\begin{pgfscope}%
\pgfsys@transformshift{3.491330in}{1.850311in}%
\pgfsys@useobject{currentmarker}{}%
\end{pgfscope}%
\end{pgfscope}%
\begin{pgfscope}%
\pgfpathrectangle{\pgfqpoint{0.100000in}{0.100000in}}{\pgfqpoint{5.307240in}{3.397500in}}%
\pgfusepath{clip}%
\pgfsetrectcap%
\pgfsetroundjoin%
\pgfsetlinewidth{1.505625pt}%
\definecolor{currentstroke}{rgb}{0.678431,1.000000,0.184314}%
\pgfsetstrokecolor{currentstroke}%
\pgfsetstrokeopacity{0.500000}%
\pgfsetdash{}{0pt}%
\pgfpathmoveto{\pgfqpoint{3.553972in}{0.880423in}}%
\pgfusepath{stroke}%
\end{pgfscope}%
\begin{pgfscope}%
\pgfpathrectangle{\pgfqpoint{0.100000in}{0.100000in}}{\pgfqpoint{5.307240in}{3.397500in}}%
\pgfusepath{clip}%
\pgfsetbuttcap%
\pgfsetroundjoin%
\definecolor{currentfill}{rgb}{0.678431,1.000000,0.184314}%
\pgfsetfillcolor{currentfill}%
\pgfsetfillopacity{0.500000}%
\pgfsetlinewidth{0.250937pt}%
\definecolor{currentstroke}{rgb}{0.000000,0.000000,0.000000}%
\pgfsetstrokecolor{currentstroke}%
\pgfsetstrokeopacity{0.500000}%
\pgfsetdash{}{0pt}%
\pgfsys@defobject{currentmarker}{\pgfqpoint{-0.106250in}{-0.106250in}}{\pgfqpoint{0.106250in}{0.106250in}}{%
\pgfpathmoveto{\pgfqpoint{0.000000in}{-0.106250in}}%
\pgfpathcurveto{\pgfqpoint{0.028178in}{-0.106250in}}{\pgfqpoint{0.055205in}{-0.095055in}}{\pgfqpoint{0.075130in}{-0.075130in}}%
\pgfpathcurveto{\pgfqpoint{0.095055in}{-0.055205in}}{\pgfqpoint{0.106250in}{-0.028178in}}{\pgfqpoint{0.106250in}{0.000000in}}%
\pgfpathcurveto{\pgfqpoint{0.106250in}{0.028178in}}{\pgfqpoint{0.095055in}{0.055205in}}{\pgfqpoint{0.075130in}{0.075130in}}%
\pgfpathcurveto{\pgfqpoint{0.055205in}{0.095055in}}{\pgfqpoint{0.028178in}{0.106250in}}{\pgfqpoint{0.000000in}{0.106250in}}%
\pgfpathcurveto{\pgfqpoint{-0.028178in}{0.106250in}}{\pgfqpoint{-0.055205in}{0.095055in}}{\pgfqpoint{-0.075130in}{0.075130in}}%
\pgfpathcurveto{\pgfqpoint{-0.095055in}{0.055205in}}{\pgfqpoint{-0.106250in}{0.028178in}}{\pgfqpoint{-0.106250in}{0.000000in}}%
\pgfpathcurveto{\pgfqpoint{-0.106250in}{-0.028178in}}{\pgfqpoint{-0.095055in}{-0.055205in}}{\pgfqpoint{-0.075130in}{-0.075130in}}%
\pgfpathcurveto{\pgfqpoint{-0.055205in}{-0.095055in}}{\pgfqpoint{-0.028178in}{-0.106250in}}{\pgfqpoint{0.000000in}{-0.106250in}}%
\pgfpathclose%
\pgfusepath{stroke,fill}%
}%
\begin{pgfscope}%
\pgfsys@transformshift{3.553972in}{0.880423in}%
\pgfsys@useobject{currentmarker}{}%
\end{pgfscope}%
\end{pgfscope}%
\begin{pgfscope}%
\pgfpathrectangle{\pgfqpoint{0.100000in}{0.100000in}}{\pgfqpoint{5.307240in}{3.397500in}}%
\pgfusepath{clip}%
\pgfsetrectcap%
\pgfsetroundjoin%
\pgfsetlinewidth{1.505625pt}%
\definecolor{currentstroke}{rgb}{0.678431,1.000000,0.184314}%
\pgfsetstrokecolor{currentstroke}%
\pgfsetstrokeopacity{0.500000}%
\pgfsetdash{}{0pt}%
\pgfpathmoveto{\pgfqpoint{3.173334in}{1.169396in}}%
\pgfusepath{stroke}%
\end{pgfscope}%
\begin{pgfscope}%
\pgfpathrectangle{\pgfqpoint{0.100000in}{0.100000in}}{\pgfqpoint{5.307240in}{3.397500in}}%
\pgfusepath{clip}%
\pgfsetbuttcap%
\pgfsetroundjoin%
\definecolor{currentfill}{rgb}{0.678431,1.000000,0.184314}%
\pgfsetfillcolor{currentfill}%
\pgfsetfillopacity{0.500000}%
\pgfsetlinewidth{0.250937pt}%
\definecolor{currentstroke}{rgb}{0.000000,0.000000,0.000000}%
\pgfsetstrokecolor{currentstroke}%
\pgfsetstrokeopacity{0.500000}%
\pgfsetdash{}{0pt}%
\pgfsys@defobject{currentmarker}{\pgfqpoint{-0.061806in}{-0.061806in}}{\pgfqpoint{0.061806in}{0.061806in}}{%
\pgfpathmoveto{\pgfqpoint{0.000000in}{-0.061806in}}%
\pgfpathcurveto{\pgfqpoint{0.016391in}{-0.061806in}}{\pgfqpoint{0.032113in}{-0.055293in}}{\pgfqpoint{0.043703in}{-0.043703in}}%
\pgfpathcurveto{\pgfqpoint{0.055293in}{-0.032113in}}{\pgfqpoint{0.061806in}{-0.016391in}}{\pgfqpoint{0.061806in}{0.000000in}}%
\pgfpathcurveto{\pgfqpoint{0.061806in}{0.016391in}}{\pgfqpoint{0.055293in}{0.032113in}}{\pgfqpoint{0.043703in}{0.043703in}}%
\pgfpathcurveto{\pgfqpoint{0.032113in}{0.055293in}}{\pgfqpoint{0.016391in}{0.061806in}}{\pgfqpoint{0.000000in}{0.061806in}}%
\pgfpathcurveto{\pgfqpoint{-0.016391in}{0.061806in}}{\pgfqpoint{-0.032113in}{0.055293in}}{\pgfqpoint{-0.043703in}{0.043703in}}%
\pgfpathcurveto{\pgfqpoint{-0.055293in}{0.032113in}}{\pgfqpoint{-0.061806in}{0.016391in}}{\pgfqpoint{-0.061806in}{0.000000in}}%
\pgfpathcurveto{\pgfqpoint{-0.061806in}{-0.016391in}}{\pgfqpoint{-0.055293in}{-0.032113in}}{\pgfqpoint{-0.043703in}{-0.043703in}}%
\pgfpathcurveto{\pgfqpoint{-0.032113in}{-0.055293in}}{\pgfqpoint{-0.016391in}{-0.061806in}}{\pgfqpoint{0.000000in}{-0.061806in}}%
\pgfpathclose%
\pgfusepath{stroke,fill}%
}%
\begin{pgfscope}%
\pgfsys@transformshift{3.173334in}{1.169396in}%
\pgfsys@useobject{currentmarker}{}%
\end{pgfscope}%
\end{pgfscope}%
\begin{pgfscope}%
\pgfpathrectangle{\pgfqpoint{0.100000in}{0.100000in}}{\pgfqpoint{5.307240in}{3.397500in}}%
\pgfusepath{clip}%
\pgfsetrectcap%
\pgfsetroundjoin%
\pgfsetlinewidth{1.505625pt}%
\definecolor{currentstroke}{rgb}{0.678431,1.000000,0.184314}%
\pgfsetstrokecolor{currentstroke}%
\pgfsetstrokeopacity{0.500000}%
\pgfsetdash{}{0pt}%
\pgfpathmoveto{\pgfqpoint{5.195558in}{2.910253in}}%
\pgfusepath{stroke}%
\end{pgfscope}%
\begin{pgfscope}%
\pgfpathrectangle{\pgfqpoint{0.100000in}{0.100000in}}{\pgfqpoint{5.307240in}{3.397500in}}%
\pgfusepath{clip}%
\pgfsetbuttcap%
\pgfsetroundjoin%
\definecolor{currentfill}{rgb}{0.678431,1.000000,0.184314}%
\pgfsetfillcolor{currentfill}%
\pgfsetfillopacity{0.500000}%
\pgfsetlinewidth{0.250937pt}%
\definecolor{currentstroke}{rgb}{0.000000,0.000000,0.000000}%
\pgfsetstrokecolor{currentstroke}%
\pgfsetstrokeopacity{0.500000}%
\pgfsetdash{}{0pt}%
\pgfsys@defobject{currentmarker}{\pgfqpoint{-0.049306in}{-0.049306in}}{\pgfqpoint{0.049306in}{0.049306in}}{%
\pgfpathmoveto{\pgfqpoint{0.000000in}{-0.049306in}}%
\pgfpathcurveto{\pgfqpoint{0.013076in}{-0.049306in}}{\pgfqpoint{0.025618in}{-0.044110in}}{\pgfqpoint{0.034864in}{-0.034864in}}%
\pgfpathcurveto{\pgfqpoint{0.044110in}{-0.025618in}}{\pgfqpoint{0.049306in}{-0.013076in}}{\pgfqpoint{0.049306in}{0.000000in}}%
\pgfpathcurveto{\pgfqpoint{0.049306in}{0.013076in}}{\pgfqpoint{0.044110in}{0.025618in}}{\pgfqpoint{0.034864in}{0.034864in}}%
\pgfpathcurveto{\pgfqpoint{0.025618in}{0.044110in}}{\pgfqpoint{0.013076in}{0.049306in}}{\pgfqpoint{0.000000in}{0.049306in}}%
\pgfpathcurveto{\pgfqpoint{-0.013076in}{0.049306in}}{\pgfqpoint{-0.025618in}{0.044110in}}{\pgfqpoint{-0.034864in}{0.034864in}}%
\pgfpathcurveto{\pgfqpoint{-0.044110in}{0.025618in}}{\pgfqpoint{-0.049306in}{0.013076in}}{\pgfqpoint{-0.049306in}{0.000000in}}%
\pgfpathcurveto{\pgfqpoint{-0.049306in}{-0.013076in}}{\pgfqpoint{-0.044110in}{-0.025618in}}{\pgfqpoint{-0.034864in}{-0.034864in}}%
\pgfpathcurveto{\pgfqpoint{-0.025618in}{-0.044110in}}{\pgfqpoint{-0.013076in}{-0.049306in}}{\pgfqpoint{0.000000in}{-0.049306in}}%
\pgfpathclose%
\pgfusepath{stroke,fill}%
}%
\begin{pgfscope}%
\pgfsys@transformshift{5.195558in}{2.910253in}%
\pgfsys@useobject{currentmarker}{}%
\end{pgfscope}%
\end{pgfscope}%
\begin{pgfscope}%
\pgfpathrectangle{\pgfqpoint{0.100000in}{0.100000in}}{\pgfqpoint{5.307240in}{3.397500in}}%
\pgfusepath{clip}%
\pgfsetrectcap%
\pgfsetroundjoin%
\pgfsetlinewidth{1.505625pt}%
\definecolor{currentstroke}{rgb}{0.678431,1.000000,0.184314}%
\pgfsetstrokecolor{currentstroke}%
\pgfsetstrokeopacity{0.500000}%
\pgfsetdash{}{0pt}%
\pgfpathmoveto{\pgfqpoint{5.103127in}{2.798440in}}%
\pgfusepath{stroke}%
\end{pgfscope}%
\begin{pgfscope}%
\pgfpathrectangle{\pgfqpoint{0.100000in}{0.100000in}}{\pgfqpoint{5.307240in}{3.397500in}}%
\pgfusepath{clip}%
\pgfsetbuttcap%
\pgfsetroundjoin%
\definecolor{currentfill}{rgb}{0.678431,1.000000,0.184314}%
\pgfsetfillcolor{currentfill}%
\pgfsetfillopacity{0.500000}%
\pgfsetlinewidth{0.250937pt}%
\definecolor{currentstroke}{rgb}{0.000000,0.000000,0.000000}%
\pgfsetstrokecolor{currentstroke}%
\pgfsetstrokeopacity{0.500000}%
\pgfsetdash{}{0pt}%
\pgfsys@defobject{currentmarker}{\pgfqpoint{-0.052778in}{-0.052778in}}{\pgfqpoint{0.052778in}{0.052778in}}{%
\pgfpathmoveto{\pgfqpoint{0.000000in}{-0.052778in}}%
\pgfpathcurveto{\pgfqpoint{0.013997in}{-0.052778in}}{\pgfqpoint{0.027422in}{-0.047217in}}{\pgfqpoint{0.037320in}{-0.037320in}}%
\pgfpathcurveto{\pgfqpoint{0.047217in}{-0.027422in}}{\pgfqpoint{0.052778in}{-0.013997in}}{\pgfqpoint{0.052778in}{0.000000in}}%
\pgfpathcurveto{\pgfqpoint{0.052778in}{0.013997in}}{\pgfqpoint{0.047217in}{0.027422in}}{\pgfqpoint{0.037320in}{0.037320in}}%
\pgfpathcurveto{\pgfqpoint{0.027422in}{0.047217in}}{\pgfqpoint{0.013997in}{0.052778in}}{\pgfqpoint{0.000000in}{0.052778in}}%
\pgfpathcurveto{\pgfqpoint{-0.013997in}{0.052778in}}{\pgfqpoint{-0.027422in}{0.047217in}}{\pgfqpoint{-0.037320in}{0.037320in}}%
\pgfpathcurveto{\pgfqpoint{-0.047217in}{0.027422in}}{\pgfqpoint{-0.052778in}{0.013997in}}{\pgfqpoint{-0.052778in}{0.000000in}}%
\pgfpathcurveto{\pgfqpoint{-0.052778in}{-0.013997in}}{\pgfqpoint{-0.047217in}{-0.027422in}}{\pgfqpoint{-0.037320in}{-0.037320in}}%
\pgfpathcurveto{\pgfqpoint{-0.027422in}{-0.047217in}}{\pgfqpoint{-0.013997in}{-0.052778in}}{\pgfqpoint{0.000000in}{-0.052778in}}%
\pgfpathclose%
\pgfusepath{stroke,fill}%
}%
\begin{pgfscope}%
\pgfsys@transformshift{5.103127in}{2.798440in}%
\pgfsys@useobject{currentmarker}{}%
\end{pgfscope}%
\end{pgfscope}%
\begin{pgfscope}%
\pgfpathrectangle{\pgfqpoint{0.100000in}{0.100000in}}{\pgfqpoint{5.307240in}{3.397500in}}%
\pgfusepath{clip}%
\pgfsetrectcap%
\pgfsetroundjoin%
\pgfsetlinewidth{1.505625pt}%
\definecolor{currentstroke}{rgb}{0.678431,1.000000,0.184314}%
\pgfsetstrokecolor{currentstroke}%
\pgfsetstrokeopacity{0.500000}%
\pgfsetdash{}{0pt}%
\pgfpathmoveto{\pgfqpoint{5.113676in}{2.748170in}}%
\pgfusepath{stroke}%
\end{pgfscope}%
\begin{pgfscope}%
\pgfpathrectangle{\pgfqpoint{0.100000in}{0.100000in}}{\pgfqpoint{5.307240in}{3.397500in}}%
\pgfusepath{clip}%
\pgfsetbuttcap%
\pgfsetroundjoin%
\definecolor{currentfill}{rgb}{0.678431,1.000000,0.184314}%
\pgfsetfillcolor{currentfill}%
\pgfsetfillopacity{0.500000}%
\pgfsetlinewidth{0.250937pt}%
\definecolor{currentstroke}{rgb}{0.000000,0.000000,0.000000}%
\pgfsetstrokecolor{currentstroke}%
\pgfsetstrokeopacity{0.500000}%
\pgfsetdash{}{0pt}%
\pgfsys@defobject{currentmarker}{\pgfqpoint{-0.059028in}{-0.059028in}}{\pgfqpoint{0.059028in}{0.059028in}}{%
\pgfpathmoveto{\pgfqpoint{0.000000in}{-0.059028in}}%
\pgfpathcurveto{\pgfqpoint{0.015654in}{-0.059028in}}{\pgfqpoint{0.030670in}{-0.052808in}}{\pgfqpoint{0.041739in}{-0.041739in}}%
\pgfpathcurveto{\pgfqpoint{0.052808in}{-0.030670in}}{\pgfqpoint{0.059028in}{-0.015654in}}{\pgfqpoint{0.059028in}{0.000000in}}%
\pgfpathcurveto{\pgfqpoint{0.059028in}{0.015654in}}{\pgfqpoint{0.052808in}{0.030670in}}{\pgfqpoint{0.041739in}{0.041739in}}%
\pgfpathcurveto{\pgfqpoint{0.030670in}{0.052808in}}{\pgfqpoint{0.015654in}{0.059028in}}{\pgfqpoint{0.000000in}{0.059028in}}%
\pgfpathcurveto{\pgfqpoint{-0.015654in}{0.059028in}}{\pgfqpoint{-0.030670in}{0.052808in}}{\pgfqpoint{-0.041739in}{0.041739in}}%
\pgfpathcurveto{\pgfqpoint{-0.052808in}{0.030670in}}{\pgfqpoint{-0.059028in}{0.015654in}}{\pgfqpoint{-0.059028in}{0.000000in}}%
\pgfpathcurveto{\pgfqpoint{-0.059028in}{-0.015654in}}{\pgfqpoint{-0.052808in}{-0.030670in}}{\pgfqpoint{-0.041739in}{-0.041739in}}%
\pgfpathcurveto{\pgfqpoint{-0.030670in}{-0.052808in}}{\pgfqpoint{-0.015654in}{-0.059028in}}{\pgfqpoint{0.000000in}{-0.059028in}}%
\pgfpathclose%
\pgfusepath{stroke,fill}%
}%
\begin{pgfscope}%
\pgfsys@transformshift{5.113676in}{2.748170in}%
\pgfsys@useobject{currentmarker}{}%
\end{pgfscope}%
\end{pgfscope}%
\begin{pgfscope}%
\pgfpathrectangle{\pgfqpoint{0.100000in}{0.100000in}}{\pgfqpoint{5.307240in}{3.397500in}}%
\pgfusepath{clip}%
\pgfsetrectcap%
\pgfsetroundjoin%
\pgfsetlinewidth{1.505625pt}%
\definecolor{currentstroke}{rgb}{0.678431,1.000000,0.184314}%
\pgfsetstrokecolor{currentstroke}%
\pgfsetstrokeopacity{0.500000}%
\pgfsetdash{}{0pt}%
\pgfpathmoveto{\pgfqpoint{4.694756in}{2.123890in}}%
\pgfusepath{stroke}%
\end{pgfscope}%
\begin{pgfscope}%
\pgfpathrectangle{\pgfqpoint{0.100000in}{0.100000in}}{\pgfqpoint{5.307240in}{3.397500in}}%
\pgfusepath{clip}%
\pgfsetbuttcap%
\pgfsetroundjoin%
\definecolor{currentfill}{rgb}{0.678431,1.000000,0.184314}%
\pgfsetfillcolor{currentfill}%
\pgfsetfillopacity{0.500000}%
\pgfsetlinewidth{0.250937pt}%
\definecolor{currentstroke}{rgb}{0.000000,0.000000,0.000000}%
\pgfsetstrokecolor{currentstroke}%
\pgfsetstrokeopacity{0.500000}%
\pgfsetdash{}{0pt}%
\pgfsys@defobject{currentmarker}{\pgfqpoint{-0.049306in}{-0.049306in}}{\pgfqpoint{0.049306in}{0.049306in}}{%
\pgfpathmoveto{\pgfqpoint{0.000000in}{-0.049306in}}%
\pgfpathcurveto{\pgfqpoint{0.013076in}{-0.049306in}}{\pgfqpoint{0.025618in}{-0.044110in}}{\pgfqpoint{0.034864in}{-0.034864in}}%
\pgfpathcurveto{\pgfqpoint{0.044110in}{-0.025618in}}{\pgfqpoint{0.049306in}{-0.013076in}}{\pgfqpoint{0.049306in}{0.000000in}}%
\pgfpathcurveto{\pgfqpoint{0.049306in}{0.013076in}}{\pgfqpoint{0.044110in}{0.025618in}}{\pgfqpoint{0.034864in}{0.034864in}}%
\pgfpathcurveto{\pgfqpoint{0.025618in}{0.044110in}}{\pgfqpoint{0.013076in}{0.049306in}}{\pgfqpoint{0.000000in}{0.049306in}}%
\pgfpathcurveto{\pgfqpoint{-0.013076in}{0.049306in}}{\pgfqpoint{-0.025618in}{0.044110in}}{\pgfqpoint{-0.034864in}{0.034864in}}%
\pgfpathcurveto{\pgfqpoint{-0.044110in}{0.025618in}}{\pgfqpoint{-0.049306in}{0.013076in}}{\pgfqpoint{-0.049306in}{0.000000in}}%
\pgfpathcurveto{\pgfqpoint{-0.049306in}{-0.013076in}}{\pgfqpoint{-0.044110in}{-0.025618in}}{\pgfqpoint{-0.034864in}{-0.034864in}}%
\pgfpathcurveto{\pgfqpoint{-0.025618in}{-0.044110in}}{\pgfqpoint{-0.013076in}{-0.049306in}}{\pgfqpoint{0.000000in}{-0.049306in}}%
\pgfpathclose%
\pgfusepath{stroke,fill}%
}%
\begin{pgfscope}%
\pgfsys@transformshift{4.694756in}{2.123890in}%
\pgfsys@useobject{currentmarker}{}%
\end{pgfscope}%
\end{pgfscope}%
\begin{pgfscope}%
\pgfpathrectangle{\pgfqpoint{0.100000in}{0.100000in}}{\pgfqpoint{5.307240in}{3.397500in}}%
\pgfusepath{clip}%
\pgfsetrectcap%
\pgfsetroundjoin%
\pgfsetlinewidth{1.505625pt}%
\definecolor{currentstroke}{rgb}{0.678431,1.000000,0.184314}%
\pgfsetstrokecolor{currentstroke}%
\pgfsetstrokeopacity{0.500000}%
\pgfsetdash{}{0pt}%
\pgfpathmoveto{\pgfqpoint{4.727129in}{2.012993in}}%
\pgfusepath{stroke}%
\end{pgfscope}%
\begin{pgfscope}%
\pgfpathrectangle{\pgfqpoint{0.100000in}{0.100000in}}{\pgfqpoint{5.307240in}{3.397500in}}%
\pgfusepath{clip}%
\pgfsetbuttcap%
\pgfsetroundjoin%
\definecolor{currentfill}{rgb}{0.678431,1.000000,0.184314}%
\pgfsetfillcolor{currentfill}%
\pgfsetfillopacity{0.500000}%
\pgfsetlinewidth{0.250937pt}%
\definecolor{currentstroke}{rgb}{0.000000,0.000000,0.000000}%
\pgfsetstrokecolor{currentstroke}%
\pgfsetstrokeopacity{0.500000}%
\pgfsetdash{}{0pt}%
\pgfsys@defobject{currentmarker}{\pgfqpoint{-0.031250in}{-0.031250in}}{\pgfqpoint{0.031250in}{0.031250in}}{%
\pgfpathmoveto{\pgfqpoint{0.000000in}{-0.031250in}}%
\pgfpathcurveto{\pgfqpoint{0.008288in}{-0.031250in}}{\pgfqpoint{0.016237in}{-0.027957in}}{\pgfqpoint{0.022097in}{-0.022097in}}%
\pgfpathcurveto{\pgfqpoint{0.027957in}{-0.016237in}}{\pgfqpoint{0.031250in}{-0.008288in}}{\pgfqpoint{0.031250in}{0.000000in}}%
\pgfpathcurveto{\pgfqpoint{0.031250in}{0.008288in}}{\pgfqpoint{0.027957in}{0.016237in}}{\pgfqpoint{0.022097in}{0.022097in}}%
\pgfpathcurveto{\pgfqpoint{0.016237in}{0.027957in}}{\pgfqpoint{0.008288in}{0.031250in}}{\pgfqpoint{0.000000in}{0.031250in}}%
\pgfpathcurveto{\pgfqpoint{-0.008288in}{0.031250in}}{\pgfqpoint{-0.016237in}{0.027957in}}{\pgfqpoint{-0.022097in}{0.022097in}}%
\pgfpathcurveto{\pgfqpoint{-0.027957in}{0.016237in}}{\pgfqpoint{-0.031250in}{0.008288in}}{\pgfqpoint{-0.031250in}{0.000000in}}%
\pgfpathcurveto{\pgfqpoint{-0.031250in}{-0.008288in}}{\pgfqpoint{-0.027957in}{-0.016237in}}{\pgfqpoint{-0.022097in}{-0.022097in}}%
\pgfpathcurveto{\pgfqpoint{-0.016237in}{-0.027957in}}{\pgfqpoint{-0.008288in}{-0.031250in}}{\pgfqpoint{0.000000in}{-0.031250in}}%
\pgfpathclose%
\pgfusepath{stroke,fill}%
}%
\begin{pgfscope}%
\pgfsys@transformshift{4.727129in}{2.012993in}%
\pgfsys@useobject{currentmarker}{}%
\end{pgfscope}%
\end{pgfscope}%
\begin{pgfscope}%
\pgfpathrectangle{\pgfqpoint{0.100000in}{0.100000in}}{\pgfqpoint{5.307240in}{3.397500in}}%
\pgfusepath{clip}%
\pgfsetrectcap%
\pgfsetroundjoin%
\pgfsetlinewidth{1.505625pt}%
\definecolor{currentstroke}{rgb}{0.678431,1.000000,0.184314}%
\pgfsetstrokecolor{currentstroke}%
\pgfsetstrokeopacity{0.500000}%
\pgfsetdash{}{0pt}%
\pgfpathmoveto{\pgfqpoint{4.497193in}{2.128641in}}%
\pgfusepath{stroke}%
\end{pgfscope}%
\begin{pgfscope}%
\pgfpathrectangle{\pgfqpoint{0.100000in}{0.100000in}}{\pgfqpoint{5.307240in}{3.397500in}}%
\pgfusepath{clip}%
\pgfsetbuttcap%
\pgfsetroundjoin%
\definecolor{currentfill}{rgb}{0.678431,1.000000,0.184314}%
\pgfsetfillcolor{currentfill}%
\pgfsetfillopacity{0.500000}%
\pgfsetlinewidth{0.250937pt}%
\definecolor{currentstroke}{rgb}{0.000000,0.000000,0.000000}%
\pgfsetstrokecolor{currentstroke}%
\pgfsetstrokeopacity{0.500000}%
\pgfsetdash{}{0pt}%
\pgfsys@defobject{currentmarker}{\pgfqpoint{-0.066667in}{-0.066667in}}{\pgfqpoint{0.066667in}{0.066667in}}{%
\pgfpathmoveto{\pgfqpoint{0.000000in}{-0.066667in}}%
\pgfpathcurveto{\pgfqpoint{0.017680in}{-0.066667in}}{\pgfqpoint{0.034639in}{-0.059642in}}{\pgfqpoint{0.047140in}{-0.047140in}}%
\pgfpathcurveto{\pgfqpoint{0.059642in}{-0.034639in}}{\pgfqpoint{0.066667in}{-0.017680in}}{\pgfqpoint{0.066667in}{0.000000in}}%
\pgfpathcurveto{\pgfqpoint{0.066667in}{0.017680in}}{\pgfqpoint{0.059642in}{0.034639in}}{\pgfqpoint{0.047140in}{0.047140in}}%
\pgfpathcurveto{\pgfqpoint{0.034639in}{0.059642in}}{\pgfqpoint{0.017680in}{0.066667in}}{\pgfqpoint{0.000000in}{0.066667in}}%
\pgfpathcurveto{\pgfqpoint{-0.017680in}{0.066667in}}{\pgfqpoint{-0.034639in}{0.059642in}}{\pgfqpoint{-0.047140in}{0.047140in}}%
\pgfpathcurveto{\pgfqpoint{-0.059642in}{0.034639in}}{\pgfqpoint{-0.066667in}{0.017680in}}{\pgfqpoint{-0.066667in}{0.000000in}}%
\pgfpathcurveto{\pgfqpoint{-0.066667in}{-0.017680in}}{\pgfqpoint{-0.059642in}{-0.034639in}}{\pgfqpoint{-0.047140in}{-0.047140in}}%
\pgfpathcurveto{\pgfqpoint{-0.034639in}{-0.059642in}}{\pgfqpoint{-0.017680in}{-0.066667in}}{\pgfqpoint{0.000000in}{-0.066667in}}%
\pgfpathclose%
\pgfusepath{stroke,fill}%
}%
\begin{pgfscope}%
\pgfsys@transformshift{4.497193in}{2.128641in}%
\pgfsys@useobject{currentmarker}{}%
\end{pgfscope}%
\end{pgfscope}%
\begin{pgfscope}%
\pgfpathrectangle{\pgfqpoint{0.100000in}{0.100000in}}{\pgfqpoint{5.307240in}{3.397500in}}%
\pgfusepath{clip}%
\pgfsetrectcap%
\pgfsetroundjoin%
\pgfsetlinewidth{1.505625pt}%
\definecolor{currentstroke}{rgb}{0.678431,1.000000,0.184314}%
\pgfsetstrokecolor{currentstroke}%
\pgfsetstrokeopacity{0.500000}%
\pgfsetdash{}{0pt}%
\pgfpathmoveto{\pgfqpoint{4.589136in}{2.144499in}}%
\pgfusepath{stroke}%
\end{pgfscope}%
\begin{pgfscope}%
\pgfpathrectangle{\pgfqpoint{0.100000in}{0.100000in}}{\pgfqpoint{5.307240in}{3.397500in}}%
\pgfusepath{clip}%
\pgfsetbuttcap%
\pgfsetroundjoin%
\definecolor{currentfill}{rgb}{0.678431,1.000000,0.184314}%
\pgfsetfillcolor{currentfill}%
\pgfsetfillopacity{0.500000}%
\pgfsetlinewidth{0.250937pt}%
\definecolor{currentstroke}{rgb}{0.000000,0.000000,0.000000}%
\pgfsetstrokecolor{currentstroke}%
\pgfsetstrokeopacity{0.500000}%
\pgfsetdash{}{0pt}%
\pgfsys@defobject{currentmarker}{\pgfqpoint{-0.060417in}{-0.060417in}}{\pgfqpoint{0.060417in}{0.060417in}}{%
\pgfpathmoveto{\pgfqpoint{0.000000in}{-0.060417in}}%
\pgfpathcurveto{\pgfqpoint{0.016023in}{-0.060417in}}{\pgfqpoint{0.031391in}{-0.054051in}}{\pgfqpoint{0.042721in}{-0.042721in}}%
\pgfpathcurveto{\pgfqpoint{0.054051in}{-0.031391in}}{\pgfqpoint{0.060417in}{-0.016023in}}{\pgfqpoint{0.060417in}{0.000000in}}%
\pgfpathcurveto{\pgfqpoint{0.060417in}{0.016023in}}{\pgfqpoint{0.054051in}{0.031391in}}{\pgfqpoint{0.042721in}{0.042721in}}%
\pgfpathcurveto{\pgfqpoint{0.031391in}{0.054051in}}{\pgfqpoint{0.016023in}{0.060417in}}{\pgfqpoint{0.000000in}{0.060417in}}%
\pgfpathcurveto{\pgfqpoint{-0.016023in}{0.060417in}}{\pgfqpoint{-0.031391in}{0.054051in}}{\pgfqpoint{-0.042721in}{0.042721in}}%
\pgfpathcurveto{\pgfqpoint{-0.054051in}{0.031391in}}{\pgfqpoint{-0.060417in}{0.016023in}}{\pgfqpoint{-0.060417in}{0.000000in}}%
\pgfpathcurveto{\pgfqpoint{-0.060417in}{-0.016023in}}{\pgfqpoint{-0.054051in}{-0.031391in}}{\pgfqpoint{-0.042721in}{-0.042721in}}%
\pgfpathcurveto{\pgfqpoint{-0.031391in}{-0.054051in}}{\pgfqpoint{-0.016023in}{-0.060417in}}{\pgfqpoint{0.000000in}{-0.060417in}}%
\pgfpathclose%
\pgfusepath{stroke,fill}%
}%
\begin{pgfscope}%
\pgfsys@transformshift{4.589136in}{2.144499in}%
\pgfsys@useobject{currentmarker}{}%
\end{pgfscope}%
\end{pgfscope}%
\begin{pgfscope}%
\pgfpathrectangle{\pgfqpoint{0.100000in}{0.100000in}}{\pgfqpoint{5.307240in}{3.397500in}}%
\pgfusepath{clip}%
\pgfsetrectcap%
\pgfsetroundjoin%
\pgfsetlinewidth{1.505625pt}%
\definecolor{currentstroke}{rgb}{0.678431,1.000000,0.184314}%
\pgfsetstrokecolor{currentstroke}%
\pgfsetstrokeopacity{0.500000}%
\pgfsetdash{}{0pt}%
\pgfpathmoveto{\pgfqpoint{5.174229in}{2.522216in}}%
\pgfusepath{stroke}%
\end{pgfscope}%
\begin{pgfscope}%
\pgfpathrectangle{\pgfqpoint{0.100000in}{0.100000in}}{\pgfqpoint{5.307240in}{3.397500in}}%
\pgfusepath{clip}%
\pgfsetbuttcap%
\pgfsetroundjoin%
\definecolor{currentfill}{rgb}{0.678431,1.000000,0.184314}%
\pgfsetfillcolor{currentfill}%
\pgfsetfillopacity{0.500000}%
\pgfsetlinewidth{0.250937pt}%
\definecolor{currentstroke}{rgb}{0.000000,0.000000,0.000000}%
\pgfsetstrokecolor{currentstroke}%
\pgfsetstrokeopacity{0.500000}%
\pgfsetdash{}{0pt}%
\pgfsys@defobject{currentmarker}{\pgfqpoint{-0.120833in}{-0.120833in}}{\pgfqpoint{0.120833in}{0.120833in}}{%
\pgfpathmoveto{\pgfqpoint{0.000000in}{-0.120833in}}%
\pgfpathcurveto{\pgfqpoint{0.032045in}{-0.120833in}}{\pgfqpoint{0.062783in}{-0.108102in}}{\pgfqpoint{0.085442in}{-0.085442in}}%
\pgfpathcurveto{\pgfqpoint{0.108102in}{-0.062783in}}{\pgfqpoint{0.120833in}{-0.032045in}}{\pgfqpoint{0.120833in}{0.000000in}}%
\pgfpathcurveto{\pgfqpoint{0.120833in}{0.032045in}}{\pgfqpoint{0.108102in}{0.062783in}}{\pgfqpoint{0.085442in}{0.085442in}}%
\pgfpathcurveto{\pgfqpoint{0.062783in}{0.108102in}}{\pgfqpoint{0.032045in}{0.120833in}}{\pgfqpoint{0.000000in}{0.120833in}}%
\pgfpathcurveto{\pgfqpoint{-0.032045in}{0.120833in}}{\pgfqpoint{-0.062783in}{0.108102in}}{\pgfqpoint{-0.085442in}{0.085442in}}%
\pgfpathcurveto{\pgfqpoint{-0.108102in}{0.062783in}}{\pgfqpoint{-0.120833in}{0.032045in}}{\pgfqpoint{-0.120833in}{0.000000in}}%
\pgfpathcurveto{\pgfqpoint{-0.120833in}{-0.032045in}}{\pgfqpoint{-0.108102in}{-0.062783in}}{\pgfqpoint{-0.085442in}{-0.085442in}}%
\pgfpathcurveto{\pgfqpoint{-0.062783in}{-0.108102in}}{\pgfqpoint{-0.032045in}{-0.120833in}}{\pgfqpoint{0.000000in}{-0.120833in}}%
\pgfpathclose%
\pgfusepath{stroke,fill}%
}%
\begin{pgfscope}%
\pgfsys@transformshift{5.174229in}{2.522216in}%
\pgfsys@useobject{currentmarker}{}%
\end{pgfscope}%
\end{pgfscope}%
\begin{pgfscope}%
\pgfpathrectangle{\pgfqpoint{0.100000in}{0.100000in}}{\pgfqpoint{5.307240in}{3.397500in}}%
\pgfusepath{clip}%
\pgfsetrectcap%
\pgfsetroundjoin%
\pgfsetlinewidth{1.505625pt}%
\definecolor{currentstroke}{rgb}{0.678431,1.000000,0.184314}%
\pgfsetstrokecolor{currentstroke}%
\pgfsetstrokeopacity{0.500000}%
\pgfsetdash{}{0pt}%
\pgfpathmoveto{\pgfqpoint{5.087666in}{2.583874in}}%
\pgfusepath{stroke}%
\end{pgfscope}%
\begin{pgfscope}%
\pgfpathrectangle{\pgfqpoint{0.100000in}{0.100000in}}{\pgfqpoint{5.307240in}{3.397500in}}%
\pgfusepath{clip}%
\pgfsetbuttcap%
\pgfsetroundjoin%
\definecolor{currentfill}{rgb}{0.678431,1.000000,0.184314}%
\pgfsetfillcolor{currentfill}%
\pgfsetfillopacity{0.500000}%
\pgfsetlinewidth{0.250937pt}%
\definecolor{currentstroke}{rgb}{0.000000,0.000000,0.000000}%
\pgfsetstrokecolor{currentstroke}%
\pgfsetstrokeopacity{0.500000}%
\pgfsetdash{}{0pt}%
\pgfsys@defobject{currentmarker}{\pgfqpoint{-0.089583in}{-0.089583in}}{\pgfqpoint{0.089583in}{0.089583in}}{%
\pgfpathmoveto{\pgfqpoint{0.000000in}{-0.089583in}}%
\pgfpathcurveto{\pgfqpoint{0.023758in}{-0.089583in}}{\pgfqpoint{0.046546in}{-0.080144in}}{\pgfqpoint{0.063345in}{-0.063345in}}%
\pgfpathcurveto{\pgfqpoint{0.080144in}{-0.046546in}}{\pgfqpoint{0.089583in}{-0.023758in}}{\pgfqpoint{0.089583in}{0.000000in}}%
\pgfpathcurveto{\pgfqpoint{0.089583in}{0.023758in}}{\pgfqpoint{0.080144in}{0.046546in}}{\pgfqpoint{0.063345in}{0.063345in}}%
\pgfpathcurveto{\pgfqpoint{0.046546in}{0.080144in}}{\pgfqpoint{0.023758in}{0.089583in}}{\pgfqpoint{0.000000in}{0.089583in}}%
\pgfpathcurveto{\pgfqpoint{-0.023758in}{0.089583in}}{\pgfqpoint{-0.046546in}{0.080144in}}{\pgfqpoint{-0.063345in}{0.063345in}}%
\pgfpathcurveto{\pgfqpoint{-0.080144in}{0.046546in}}{\pgfqpoint{-0.089583in}{0.023758in}}{\pgfqpoint{-0.089583in}{0.000000in}}%
\pgfpathcurveto{\pgfqpoint{-0.089583in}{-0.023758in}}{\pgfqpoint{-0.080144in}{-0.046546in}}{\pgfqpoint{-0.063345in}{-0.063345in}}%
\pgfpathcurveto{\pgfqpoint{-0.046546in}{-0.080144in}}{\pgfqpoint{-0.023758in}{-0.089583in}}{\pgfqpoint{0.000000in}{-0.089583in}}%
\pgfpathclose%
\pgfusepath{stroke,fill}%
}%
\begin{pgfscope}%
\pgfsys@transformshift{5.087666in}{2.583874in}%
\pgfsys@useobject{currentmarker}{}%
\end{pgfscope}%
\end{pgfscope}%
\begin{pgfscope}%
\pgfpathrectangle{\pgfqpoint{0.100000in}{0.100000in}}{\pgfqpoint{5.307240in}{3.397500in}}%
\pgfusepath{clip}%
\pgfsetrectcap%
\pgfsetroundjoin%
\pgfsetlinewidth{1.505625pt}%
\definecolor{currentstroke}{rgb}{0.678431,1.000000,0.184314}%
\pgfsetstrokecolor{currentstroke}%
\pgfsetstrokeopacity{0.500000}%
\pgfsetdash{}{0pt}%
\pgfpathmoveto{\pgfqpoint{5.024587in}{2.586955in}}%
\pgfusepath{stroke}%
\end{pgfscope}%
\begin{pgfscope}%
\pgfpathrectangle{\pgfqpoint{0.100000in}{0.100000in}}{\pgfqpoint{5.307240in}{3.397500in}}%
\pgfusepath{clip}%
\pgfsetbuttcap%
\pgfsetroundjoin%
\definecolor{currentfill}{rgb}{0.678431,1.000000,0.184314}%
\pgfsetfillcolor{currentfill}%
\pgfsetfillopacity{0.500000}%
\pgfsetlinewidth{0.250937pt}%
\definecolor{currentstroke}{rgb}{0.000000,0.000000,0.000000}%
\pgfsetstrokecolor{currentstroke}%
\pgfsetstrokeopacity{0.500000}%
\pgfsetdash{}{0pt}%
\pgfsys@defobject{currentmarker}{\pgfqpoint{-0.096528in}{-0.096528in}}{\pgfqpoint{0.096528in}{0.096528in}}{%
\pgfpathmoveto{\pgfqpoint{0.000000in}{-0.096528in}}%
\pgfpathcurveto{\pgfqpoint{0.025599in}{-0.096528in}}{\pgfqpoint{0.050154in}{-0.086357in}}{\pgfqpoint{0.068255in}{-0.068255in}}%
\pgfpathcurveto{\pgfqpoint{0.086357in}{-0.050154in}}{\pgfqpoint{0.096528in}{-0.025599in}}{\pgfqpoint{0.096528in}{0.000000in}}%
\pgfpathcurveto{\pgfqpoint{0.096528in}{0.025599in}}{\pgfqpoint{0.086357in}{0.050154in}}{\pgfqpoint{0.068255in}{0.068255in}}%
\pgfpathcurveto{\pgfqpoint{0.050154in}{0.086357in}}{\pgfqpoint{0.025599in}{0.096528in}}{\pgfqpoint{0.000000in}{0.096528in}}%
\pgfpathcurveto{\pgfqpoint{-0.025599in}{0.096528in}}{\pgfqpoint{-0.050154in}{0.086357in}}{\pgfqpoint{-0.068255in}{0.068255in}}%
\pgfpathcurveto{\pgfqpoint{-0.086357in}{0.050154in}}{\pgfqpoint{-0.096528in}{0.025599in}}{\pgfqpoint{-0.096528in}{0.000000in}}%
\pgfpathcurveto{\pgfqpoint{-0.096528in}{-0.025599in}}{\pgfqpoint{-0.086357in}{-0.050154in}}{\pgfqpoint{-0.068255in}{-0.068255in}}%
\pgfpathcurveto{\pgfqpoint{-0.050154in}{-0.086357in}}{\pgfqpoint{-0.025599in}{-0.096528in}}{\pgfqpoint{0.000000in}{-0.096528in}}%
\pgfpathclose%
\pgfusepath{stroke,fill}%
}%
\begin{pgfscope}%
\pgfsys@transformshift{5.024587in}{2.586955in}%
\pgfsys@useobject{currentmarker}{}%
\end{pgfscope}%
\end{pgfscope}%
\begin{pgfscope}%
\pgfpathrectangle{\pgfqpoint{0.100000in}{0.100000in}}{\pgfqpoint{5.307240in}{3.397500in}}%
\pgfusepath{clip}%
\pgfsetrectcap%
\pgfsetroundjoin%
\pgfsetlinewidth{1.505625pt}%
\definecolor{currentstroke}{rgb}{0.678431,1.000000,0.184314}%
\pgfsetstrokecolor{currentstroke}%
\pgfsetstrokeopacity{0.500000}%
\pgfsetdash{}{0pt}%
\pgfpathmoveto{\pgfqpoint{5.120144in}{2.505322in}}%
\pgfusepath{stroke}%
\end{pgfscope}%
\begin{pgfscope}%
\pgfpathrectangle{\pgfqpoint{0.100000in}{0.100000in}}{\pgfqpoint{5.307240in}{3.397500in}}%
\pgfusepath{clip}%
\pgfsetbuttcap%
\pgfsetroundjoin%
\definecolor{currentfill}{rgb}{0.678431,1.000000,0.184314}%
\pgfsetfillcolor{currentfill}%
\pgfsetfillopacity{0.500000}%
\pgfsetlinewidth{0.250937pt}%
\definecolor{currentstroke}{rgb}{0.000000,0.000000,0.000000}%
\pgfsetstrokecolor{currentstroke}%
\pgfsetstrokeopacity{0.500000}%
\pgfsetdash{}{0pt}%
\pgfsys@defobject{currentmarker}{\pgfqpoint{-0.124306in}{-0.124306in}}{\pgfqpoint{0.124306in}{0.124306in}}{%
\pgfpathmoveto{\pgfqpoint{0.000000in}{-0.124306in}}%
\pgfpathcurveto{\pgfqpoint{0.032966in}{-0.124306in}}{\pgfqpoint{0.064587in}{-0.111208in}}{\pgfqpoint{0.087897in}{-0.087897in}}%
\pgfpathcurveto{\pgfqpoint{0.111208in}{-0.064587in}}{\pgfqpoint{0.124306in}{-0.032966in}}{\pgfqpoint{0.124306in}{0.000000in}}%
\pgfpathcurveto{\pgfqpoint{0.124306in}{0.032966in}}{\pgfqpoint{0.111208in}{0.064587in}}{\pgfqpoint{0.087897in}{0.087897in}}%
\pgfpathcurveto{\pgfqpoint{0.064587in}{0.111208in}}{\pgfqpoint{0.032966in}{0.124306in}}{\pgfqpoint{0.000000in}{0.124306in}}%
\pgfpathcurveto{\pgfqpoint{-0.032966in}{0.124306in}}{\pgfqpoint{-0.064587in}{0.111208in}}{\pgfqpoint{-0.087897in}{0.087897in}}%
\pgfpathcurveto{\pgfqpoint{-0.111208in}{0.064587in}}{\pgfqpoint{-0.124306in}{0.032966in}}{\pgfqpoint{-0.124306in}{0.000000in}}%
\pgfpathcurveto{\pgfqpoint{-0.124306in}{-0.032966in}}{\pgfqpoint{-0.111208in}{-0.064587in}}{\pgfqpoint{-0.087897in}{-0.087897in}}%
\pgfpathcurveto{\pgfqpoint{-0.064587in}{-0.111208in}}{\pgfqpoint{-0.032966in}{-0.124306in}}{\pgfqpoint{0.000000in}{-0.124306in}}%
\pgfpathclose%
\pgfusepath{stroke,fill}%
}%
\begin{pgfscope}%
\pgfsys@transformshift{5.120144in}{2.505322in}%
\pgfsys@useobject{currentmarker}{}%
\end{pgfscope}%
\end{pgfscope}%
\begin{pgfscope}%
\pgfpathrectangle{\pgfqpoint{0.100000in}{0.100000in}}{\pgfqpoint{5.307240in}{3.397500in}}%
\pgfusepath{clip}%
\pgfsetrectcap%
\pgfsetroundjoin%
\pgfsetlinewidth{1.505625pt}%
\definecolor{currentstroke}{rgb}{0.678431,1.000000,0.184314}%
\pgfsetstrokecolor{currentstroke}%
\pgfsetstrokeopacity{0.500000}%
\pgfsetdash{}{0pt}%
\pgfpathmoveto{\pgfqpoint{4.903037in}{2.547226in}}%
\pgfusepath{stroke}%
\end{pgfscope}%
\begin{pgfscope}%
\pgfpathrectangle{\pgfqpoint{0.100000in}{0.100000in}}{\pgfqpoint{5.307240in}{3.397500in}}%
\pgfusepath{clip}%
\pgfsetbuttcap%
\pgfsetroundjoin%
\definecolor{currentfill}{rgb}{0.678431,1.000000,0.184314}%
\pgfsetfillcolor{currentfill}%
\pgfsetfillopacity{0.500000}%
\pgfsetlinewidth{0.250937pt}%
\definecolor{currentstroke}{rgb}{0.000000,0.000000,0.000000}%
\pgfsetstrokecolor{currentstroke}%
\pgfsetstrokeopacity{0.500000}%
\pgfsetdash{}{0pt}%
\pgfsys@defobject{currentmarker}{\pgfqpoint{-0.098611in}{-0.098611in}}{\pgfqpoint{0.098611in}{0.098611in}}{%
\pgfpathmoveto{\pgfqpoint{0.000000in}{-0.098611in}}%
\pgfpathcurveto{\pgfqpoint{0.026152in}{-0.098611in}}{\pgfqpoint{0.051236in}{-0.088221in}}{\pgfqpoint{0.069729in}{-0.069729in}}%
\pgfpathcurveto{\pgfqpoint{0.088221in}{-0.051236in}}{\pgfqpoint{0.098611in}{-0.026152in}}{\pgfqpoint{0.098611in}{0.000000in}}%
\pgfpathcurveto{\pgfqpoint{0.098611in}{0.026152in}}{\pgfqpoint{0.088221in}{0.051236in}}{\pgfqpoint{0.069729in}{0.069729in}}%
\pgfpathcurveto{\pgfqpoint{0.051236in}{0.088221in}}{\pgfqpoint{0.026152in}{0.098611in}}{\pgfqpoint{0.000000in}{0.098611in}}%
\pgfpathcurveto{\pgfqpoint{-0.026152in}{0.098611in}}{\pgfqpoint{-0.051236in}{0.088221in}}{\pgfqpoint{-0.069729in}{0.069729in}}%
\pgfpathcurveto{\pgfqpoint{-0.088221in}{0.051236in}}{\pgfqpoint{-0.098611in}{0.026152in}}{\pgfqpoint{-0.098611in}{0.000000in}}%
\pgfpathcurveto{\pgfqpoint{-0.098611in}{-0.026152in}}{\pgfqpoint{-0.088221in}{-0.051236in}}{\pgfqpoint{-0.069729in}{-0.069729in}}%
\pgfpathcurveto{\pgfqpoint{-0.051236in}{-0.088221in}}{\pgfqpoint{-0.026152in}{-0.098611in}}{\pgfqpoint{0.000000in}{-0.098611in}}%
\pgfpathclose%
\pgfusepath{stroke,fill}%
}%
\begin{pgfscope}%
\pgfsys@transformshift{4.903037in}{2.547226in}%
\pgfsys@useobject{currentmarker}{}%
\end{pgfscope}%
\end{pgfscope}%
\begin{pgfscope}%
\pgfpathrectangle{\pgfqpoint{0.100000in}{0.100000in}}{\pgfqpoint{5.307240in}{3.397500in}}%
\pgfusepath{clip}%
\pgfsetrectcap%
\pgfsetroundjoin%
\pgfsetlinewidth{1.505625pt}%
\definecolor{currentstroke}{rgb}{0.678431,1.000000,0.184314}%
\pgfsetstrokecolor{currentstroke}%
\pgfsetstrokeopacity{0.500000}%
\pgfsetdash{}{0pt}%
\pgfpathmoveto{\pgfqpoint{4.967621in}{2.521418in}}%
\pgfusepath{stroke}%
\end{pgfscope}%
\begin{pgfscope}%
\pgfpathrectangle{\pgfqpoint{0.100000in}{0.100000in}}{\pgfqpoint{5.307240in}{3.397500in}}%
\pgfusepath{clip}%
\pgfsetbuttcap%
\pgfsetroundjoin%
\definecolor{currentfill}{rgb}{0.678431,1.000000,0.184314}%
\pgfsetfillcolor{currentfill}%
\pgfsetfillopacity{0.500000}%
\pgfsetlinewidth{0.250937pt}%
\definecolor{currentstroke}{rgb}{0.000000,0.000000,0.000000}%
\pgfsetstrokecolor{currentstroke}%
\pgfsetstrokeopacity{0.500000}%
\pgfsetdash{}{0pt}%
\pgfsys@defobject{currentmarker}{\pgfqpoint{-0.079167in}{-0.079167in}}{\pgfqpoint{0.079167in}{0.079167in}}{%
\pgfpathmoveto{\pgfqpoint{0.000000in}{-0.079167in}}%
\pgfpathcurveto{\pgfqpoint{0.020995in}{-0.079167in}}{\pgfqpoint{0.041133in}{-0.070825in}}{\pgfqpoint{0.055979in}{-0.055979in}}%
\pgfpathcurveto{\pgfqpoint{0.070825in}{-0.041133in}}{\pgfqpoint{0.079167in}{-0.020995in}}{\pgfqpoint{0.079167in}{0.000000in}}%
\pgfpathcurveto{\pgfqpoint{0.079167in}{0.020995in}}{\pgfqpoint{0.070825in}{0.041133in}}{\pgfqpoint{0.055979in}{0.055979in}}%
\pgfpathcurveto{\pgfqpoint{0.041133in}{0.070825in}}{\pgfqpoint{0.020995in}{0.079167in}}{\pgfqpoint{0.000000in}{0.079167in}}%
\pgfpathcurveto{\pgfqpoint{-0.020995in}{0.079167in}}{\pgfqpoint{-0.041133in}{0.070825in}}{\pgfqpoint{-0.055979in}{0.055979in}}%
\pgfpathcurveto{\pgfqpoint{-0.070825in}{0.041133in}}{\pgfqpoint{-0.079167in}{0.020995in}}{\pgfqpoint{-0.079167in}{0.000000in}}%
\pgfpathcurveto{\pgfqpoint{-0.079167in}{-0.020995in}}{\pgfqpoint{-0.070825in}{-0.041133in}}{\pgfqpoint{-0.055979in}{-0.055979in}}%
\pgfpathcurveto{\pgfqpoint{-0.041133in}{-0.070825in}}{\pgfqpoint{-0.020995in}{-0.079167in}}{\pgfqpoint{0.000000in}{-0.079167in}}%
\pgfpathclose%
\pgfusepath{stroke,fill}%
}%
\begin{pgfscope}%
\pgfsys@transformshift{4.967621in}{2.521418in}%
\pgfsys@useobject{currentmarker}{}%
\end{pgfscope}%
\end{pgfscope}%
\begin{pgfscope}%
\pgfpathrectangle{\pgfqpoint{0.100000in}{0.100000in}}{\pgfqpoint{5.307240in}{3.397500in}}%
\pgfusepath{clip}%
\pgfsetrectcap%
\pgfsetroundjoin%
\pgfsetlinewidth{1.505625pt}%
\definecolor{currentstroke}{rgb}{0.678431,1.000000,0.184314}%
\pgfsetstrokecolor{currentstroke}%
\pgfsetstrokeopacity{0.500000}%
\pgfsetdash{}{0pt}%
\pgfpathmoveto{\pgfqpoint{5.028775in}{2.556457in}}%
\pgfusepath{stroke}%
\end{pgfscope}%
\begin{pgfscope}%
\pgfpathrectangle{\pgfqpoint{0.100000in}{0.100000in}}{\pgfqpoint{5.307240in}{3.397500in}}%
\pgfusepath{clip}%
\pgfsetbuttcap%
\pgfsetroundjoin%
\definecolor{currentfill}{rgb}{0.678431,1.000000,0.184314}%
\pgfsetfillcolor{currentfill}%
\pgfsetfillopacity{0.500000}%
\pgfsetlinewidth{0.250937pt}%
\definecolor{currentstroke}{rgb}{0.000000,0.000000,0.000000}%
\pgfsetstrokecolor{currentstroke}%
\pgfsetstrokeopacity{0.500000}%
\pgfsetdash{}{0pt}%
\pgfsys@defobject{currentmarker}{\pgfqpoint{-0.079167in}{-0.079167in}}{\pgfqpoint{0.079167in}{0.079167in}}{%
\pgfpathmoveto{\pgfqpoint{0.000000in}{-0.079167in}}%
\pgfpathcurveto{\pgfqpoint{0.020995in}{-0.079167in}}{\pgfqpoint{0.041133in}{-0.070825in}}{\pgfqpoint{0.055979in}{-0.055979in}}%
\pgfpathcurveto{\pgfqpoint{0.070825in}{-0.041133in}}{\pgfqpoint{0.079167in}{-0.020995in}}{\pgfqpoint{0.079167in}{0.000000in}}%
\pgfpathcurveto{\pgfqpoint{0.079167in}{0.020995in}}{\pgfqpoint{0.070825in}{0.041133in}}{\pgfqpoint{0.055979in}{0.055979in}}%
\pgfpathcurveto{\pgfqpoint{0.041133in}{0.070825in}}{\pgfqpoint{0.020995in}{0.079167in}}{\pgfqpoint{0.000000in}{0.079167in}}%
\pgfpathcurveto{\pgfqpoint{-0.020995in}{0.079167in}}{\pgfqpoint{-0.041133in}{0.070825in}}{\pgfqpoint{-0.055979in}{0.055979in}}%
\pgfpathcurveto{\pgfqpoint{-0.070825in}{0.041133in}}{\pgfqpoint{-0.079167in}{0.020995in}}{\pgfqpoint{-0.079167in}{0.000000in}}%
\pgfpathcurveto{\pgfqpoint{-0.079167in}{-0.020995in}}{\pgfqpoint{-0.070825in}{-0.041133in}}{\pgfqpoint{-0.055979in}{-0.055979in}}%
\pgfpathcurveto{\pgfqpoint{-0.041133in}{-0.070825in}}{\pgfqpoint{-0.020995in}{-0.079167in}}{\pgfqpoint{0.000000in}{-0.079167in}}%
\pgfpathclose%
\pgfusepath{stroke,fill}%
}%
\begin{pgfscope}%
\pgfsys@transformshift{5.028775in}{2.556457in}%
\pgfsys@useobject{currentmarker}{}%
\end{pgfscope}%
\end{pgfscope}%
\begin{pgfscope}%
\pgfpathrectangle{\pgfqpoint{0.100000in}{0.100000in}}{\pgfqpoint{5.307240in}{3.397500in}}%
\pgfusepath{clip}%
\pgfsetrectcap%
\pgfsetroundjoin%
\pgfsetlinewidth{1.505625pt}%
\definecolor{currentstroke}{rgb}{0.678431,1.000000,0.184314}%
\pgfsetstrokecolor{currentstroke}%
\pgfsetstrokeopacity{0.500000}%
\pgfsetdash{}{0pt}%
\pgfpathmoveto{\pgfqpoint{4.019961in}{2.363484in}}%
\pgfusepath{stroke}%
\end{pgfscope}%
\begin{pgfscope}%
\pgfpathrectangle{\pgfqpoint{0.100000in}{0.100000in}}{\pgfqpoint{5.307240in}{3.397500in}}%
\pgfusepath{clip}%
\pgfsetbuttcap%
\pgfsetroundjoin%
\definecolor{currentfill}{rgb}{0.678431,1.000000,0.184314}%
\pgfsetfillcolor{currentfill}%
\pgfsetfillopacity{0.500000}%
\pgfsetlinewidth{0.250937pt}%
\definecolor{currentstroke}{rgb}{0.000000,0.000000,0.000000}%
\pgfsetstrokecolor{currentstroke}%
\pgfsetstrokeopacity{0.500000}%
\pgfsetdash{}{0pt}%
\pgfsys@defobject{currentmarker}{\pgfqpoint{-0.085417in}{-0.085417in}}{\pgfqpoint{0.085417in}{0.085417in}}{%
\pgfpathmoveto{\pgfqpoint{0.000000in}{-0.085417in}}%
\pgfpathcurveto{\pgfqpoint{0.022653in}{-0.085417in}}{\pgfqpoint{0.044381in}{-0.076417in}}{\pgfqpoint{0.060399in}{-0.060399in}}%
\pgfpathcurveto{\pgfqpoint{0.076417in}{-0.044381in}}{\pgfqpoint{0.085417in}{-0.022653in}}{\pgfqpoint{0.085417in}{0.000000in}}%
\pgfpathcurveto{\pgfqpoint{0.085417in}{0.022653in}}{\pgfqpoint{0.076417in}{0.044381in}}{\pgfqpoint{0.060399in}{0.060399in}}%
\pgfpathcurveto{\pgfqpoint{0.044381in}{0.076417in}}{\pgfqpoint{0.022653in}{0.085417in}}{\pgfqpoint{0.000000in}{0.085417in}}%
\pgfpathcurveto{\pgfqpoint{-0.022653in}{0.085417in}}{\pgfqpoint{-0.044381in}{0.076417in}}{\pgfqpoint{-0.060399in}{0.060399in}}%
\pgfpathcurveto{\pgfqpoint{-0.076417in}{0.044381in}}{\pgfqpoint{-0.085417in}{0.022653in}}{\pgfqpoint{-0.085417in}{0.000000in}}%
\pgfpathcurveto{\pgfqpoint{-0.085417in}{-0.022653in}}{\pgfqpoint{-0.076417in}{-0.044381in}}{\pgfqpoint{-0.060399in}{-0.060399in}}%
\pgfpathcurveto{\pgfqpoint{-0.044381in}{-0.076417in}}{\pgfqpoint{-0.022653in}{-0.085417in}}{\pgfqpoint{0.000000in}{-0.085417in}}%
\pgfpathclose%
\pgfusepath{stroke,fill}%
}%
\begin{pgfscope}%
\pgfsys@transformshift{4.019961in}{2.363484in}%
\pgfsys@useobject{currentmarker}{}%
\end{pgfscope}%
\end{pgfscope}%
\begin{pgfscope}%
\pgfpathrectangle{\pgfqpoint{0.100000in}{0.100000in}}{\pgfqpoint{5.307240in}{3.397500in}}%
\pgfusepath{clip}%
\pgfsetrectcap%
\pgfsetroundjoin%
\pgfsetlinewidth{1.505625pt}%
\definecolor{currentstroke}{rgb}{0.678431,1.000000,0.184314}%
\pgfsetstrokecolor{currentstroke}%
\pgfsetstrokeopacity{0.500000}%
\pgfsetdash{}{0pt}%
\pgfpathmoveto{\pgfqpoint{3.895126in}{2.354925in}}%
\pgfusepath{stroke}%
\end{pgfscope}%
\begin{pgfscope}%
\pgfpathrectangle{\pgfqpoint{0.100000in}{0.100000in}}{\pgfqpoint{5.307240in}{3.397500in}}%
\pgfusepath{clip}%
\pgfsetbuttcap%
\pgfsetroundjoin%
\definecolor{currentfill}{rgb}{0.678431,1.000000,0.184314}%
\pgfsetfillcolor{currentfill}%
\pgfsetfillopacity{0.500000}%
\pgfsetlinewidth{0.250937pt}%
\definecolor{currentstroke}{rgb}{0.000000,0.000000,0.000000}%
\pgfsetstrokecolor{currentstroke}%
\pgfsetstrokeopacity{0.500000}%
\pgfsetdash{}{0pt}%
\pgfsys@defobject{currentmarker}{\pgfqpoint{-0.144444in}{-0.144444in}}{\pgfqpoint{0.144444in}{0.144444in}}{%
\pgfpathmoveto{\pgfqpoint{0.000000in}{-0.144444in}}%
\pgfpathcurveto{\pgfqpoint{0.038307in}{-0.144444in}}{\pgfqpoint{0.075050in}{-0.129225in}}{\pgfqpoint{0.102138in}{-0.102138in}}%
\pgfpathcurveto{\pgfqpoint{0.129225in}{-0.075050in}}{\pgfqpoint{0.144444in}{-0.038307in}}{\pgfqpoint{0.144444in}{0.000000in}}%
\pgfpathcurveto{\pgfqpoint{0.144444in}{0.038307in}}{\pgfqpoint{0.129225in}{0.075050in}}{\pgfqpoint{0.102138in}{0.102138in}}%
\pgfpathcurveto{\pgfqpoint{0.075050in}{0.129225in}}{\pgfqpoint{0.038307in}{0.144444in}}{\pgfqpoint{0.000000in}{0.144444in}}%
\pgfpathcurveto{\pgfqpoint{-0.038307in}{0.144444in}}{\pgfqpoint{-0.075050in}{0.129225in}}{\pgfqpoint{-0.102138in}{0.102138in}}%
\pgfpathcurveto{\pgfqpoint{-0.129225in}{0.075050in}}{\pgfqpoint{-0.144444in}{0.038307in}}{\pgfqpoint{-0.144444in}{0.000000in}}%
\pgfpathcurveto{\pgfqpoint{-0.144444in}{-0.038307in}}{\pgfqpoint{-0.129225in}{-0.075050in}}{\pgfqpoint{-0.102138in}{-0.102138in}}%
\pgfpathcurveto{\pgfqpoint{-0.075050in}{-0.129225in}}{\pgfqpoint{-0.038307in}{-0.144444in}}{\pgfqpoint{0.000000in}{-0.144444in}}%
\pgfpathclose%
\pgfusepath{stroke,fill}%
}%
\begin{pgfscope}%
\pgfsys@transformshift{3.895126in}{2.354925in}%
\pgfsys@useobject{currentmarker}{}%
\end{pgfscope}%
\end{pgfscope}%
\begin{pgfscope}%
\pgfpathrectangle{\pgfqpoint{0.100000in}{0.100000in}}{\pgfqpoint{5.307240in}{3.397500in}}%
\pgfusepath{clip}%
\pgfsetrectcap%
\pgfsetroundjoin%
\pgfsetlinewidth{1.505625pt}%
\definecolor{currentstroke}{rgb}{0.678431,1.000000,0.184314}%
\pgfsetstrokecolor{currentstroke}%
\pgfsetstrokeopacity{0.500000}%
\pgfsetdash{}{0pt}%
\pgfpathmoveto{\pgfqpoint{3.987630in}{2.515207in}}%
\pgfusepath{stroke}%
\end{pgfscope}%
\begin{pgfscope}%
\pgfpathrectangle{\pgfqpoint{0.100000in}{0.100000in}}{\pgfqpoint{5.307240in}{3.397500in}}%
\pgfusepath{clip}%
\pgfsetbuttcap%
\pgfsetroundjoin%
\definecolor{currentfill}{rgb}{0.678431,1.000000,0.184314}%
\pgfsetfillcolor{currentfill}%
\pgfsetfillopacity{0.500000}%
\pgfsetlinewidth{0.250937pt}%
\definecolor{currentstroke}{rgb}{0.000000,0.000000,0.000000}%
\pgfsetstrokecolor{currentstroke}%
\pgfsetstrokeopacity{0.500000}%
\pgfsetdash{}{0pt}%
\pgfsys@defobject{currentmarker}{\pgfqpoint{-0.145833in}{-0.145833in}}{\pgfqpoint{0.145833in}{0.145833in}}{%
\pgfpathmoveto{\pgfqpoint{0.000000in}{-0.145833in}}%
\pgfpathcurveto{\pgfqpoint{0.038675in}{-0.145833in}}{\pgfqpoint{0.075772in}{-0.130467in}}{\pgfqpoint{0.103120in}{-0.103120in}}%
\pgfpathcurveto{\pgfqpoint{0.130467in}{-0.075772in}}{\pgfqpoint{0.145833in}{-0.038675in}}{\pgfqpoint{0.145833in}{0.000000in}}%
\pgfpathcurveto{\pgfqpoint{0.145833in}{0.038675in}}{\pgfqpoint{0.130467in}{0.075772in}}{\pgfqpoint{0.103120in}{0.103120in}}%
\pgfpathcurveto{\pgfqpoint{0.075772in}{0.130467in}}{\pgfqpoint{0.038675in}{0.145833in}}{\pgfqpoint{0.000000in}{0.145833in}}%
\pgfpathcurveto{\pgfqpoint{-0.038675in}{0.145833in}}{\pgfqpoint{-0.075772in}{0.130467in}}{\pgfqpoint{-0.103120in}{0.103120in}}%
\pgfpathcurveto{\pgfqpoint{-0.130467in}{0.075772in}}{\pgfqpoint{-0.145833in}{0.038675in}}{\pgfqpoint{-0.145833in}{0.000000in}}%
\pgfpathcurveto{\pgfqpoint{-0.145833in}{-0.038675in}}{\pgfqpoint{-0.130467in}{-0.075772in}}{\pgfqpoint{-0.103120in}{-0.103120in}}%
\pgfpathcurveto{\pgfqpoint{-0.075772in}{-0.130467in}}{\pgfqpoint{-0.038675in}{-0.145833in}}{\pgfqpoint{0.000000in}{-0.145833in}}%
\pgfpathclose%
\pgfusepath{stroke,fill}%
}%
\begin{pgfscope}%
\pgfsys@transformshift{3.987630in}{2.515207in}%
\pgfsys@useobject{currentmarker}{}%
\end{pgfscope}%
\end{pgfscope}%
\begin{pgfscope}%
\pgfpathrectangle{\pgfqpoint{0.100000in}{0.100000in}}{\pgfqpoint{5.307240in}{3.397500in}}%
\pgfusepath{clip}%
\pgfsetrectcap%
\pgfsetroundjoin%
\pgfsetlinewidth{1.505625pt}%
\definecolor{currentstroke}{rgb}{0.678431,1.000000,0.184314}%
\pgfsetstrokecolor{currentstroke}%
\pgfsetstrokeopacity{0.500000}%
\pgfsetdash{}{0pt}%
\pgfpathmoveto{\pgfqpoint{4.076374in}{2.380178in}}%
\pgfusepath{stroke}%
\end{pgfscope}%
\begin{pgfscope}%
\pgfpathrectangle{\pgfqpoint{0.100000in}{0.100000in}}{\pgfqpoint{5.307240in}{3.397500in}}%
\pgfusepath{clip}%
\pgfsetbuttcap%
\pgfsetroundjoin%
\definecolor{currentfill}{rgb}{0.678431,1.000000,0.184314}%
\pgfsetfillcolor{currentfill}%
\pgfsetfillopacity{0.500000}%
\pgfsetlinewidth{0.250937pt}%
\definecolor{currentstroke}{rgb}{0.000000,0.000000,0.000000}%
\pgfsetstrokecolor{currentstroke}%
\pgfsetstrokeopacity{0.500000}%
\pgfsetdash{}{0pt}%
\pgfsys@defobject{currentmarker}{\pgfqpoint{-0.142361in}{-0.142361in}}{\pgfqpoint{0.142361in}{0.142361in}}{%
\pgfpathmoveto{\pgfqpoint{0.000000in}{-0.142361in}}%
\pgfpathcurveto{\pgfqpoint{0.037755in}{-0.142361in}}{\pgfqpoint{0.073968in}{-0.127361in}}{\pgfqpoint{0.100665in}{-0.100665in}}%
\pgfpathcurveto{\pgfqpoint{0.127361in}{-0.073968in}}{\pgfqpoint{0.142361in}{-0.037755in}}{\pgfqpoint{0.142361in}{0.000000in}}%
\pgfpathcurveto{\pgfqpoint{0.142361in}{0.037755in}}{\pgfqpoint{0.127361in}{0.073968in}}{\pgfqpoint{0.100665in}{0.100665in}}%
\pgfpathcurveto{\pgfqpoint{0.073968in}{0.127361in}}{\pgfqpoint{0.037755in}{0.142361in}}{\pgfqpoint{0.000000in}{0.142361in}}%
\pgfpathcurveto{\pgfqpoint{-0.037755in}{0.142361in}}{\pgfqpoint{-0.073968in}{0.127361in}}{\pgfqpoint{-0.100665in}{0.100665in}}%
\pgfpathcurveto{\pgfqpoint{-0.127361in}{0.073968in}}{\pgfqpoint{-0.142361in}{0.037755in}}{\pgfqpoint{-0.142361in}{0.000000in}}%
\pgfpathcurveto{\pgfqpoint{-0.142361in}{-0.037755in}}{\pgfqpoint{-0.127361in}{-0.073968in}}{\pgfqpoint{-0.100665in}{-0.100665in}}%
\pgfpathcurveto{\pgfqpoint{-0.073968in}{-0.127361in}}{\pgfqpoint{-0.037755in}{-0.142361in}}{\pgfqpoint{0.000000in}{-0.142361in}}%
\pgfpathclose%
\pgfusepath{stroke,fill}%
}%
\begin{pgfscope}%
\pgfsys@transformshift{4.076374in}{2.380178in}%
\pgfsys@useobject{currentmarker}{}%
\end{pgfscope}%
\end{pgfscope}%
\begin{pgfscope}%
\pgfpathrectangle{\pgfqpoint{0.100000in}{0.100000in}}{\pgfqpoint{5.307240in}{3.397500in}}%
\pgfusepath{clip}%
\pgfsetrectcap%
\pgfsetroundjoin%
\pgfsetlinewidth{1.505625pt}%
\definecolor{currentstroke}{rgb}{0.678431,1.000000,0.184314}%
\pgfsetstrokecolor{currentstroke}%
\pgfsetstrokeopacity{0.500000}%
\pgfsetdash{}{0pt}%
\pgfpathmoveto{\pgfqpoint{4.009815in}{2.450139in}}%
\pgfusepath{stroke}%
\end{pgfscope}%
\begin{pgfscope}%
\pgfpathrectangle{\pgfqpoint{0.100000in}{0.100000in}}{\pgfqpoint{5.307240in}{3.397500in}}%
\pgfusepath{clip}%
\pgfsetbuttcap%
\pgfsetroundjoin%
\definecolor{currentfill}{rgb}{0.678431,1.000000,0.184314}%
\pgfsetfillcolor{currentfill}%
\pgfsetfillopacity{0.500000}%
\pgfsetlinewidth{0.250937pt}%
\definecolor{currentstroke}{rgb}{0.000000,0.000000,0.000000}%
\pgfsetstrokecolor{currentstroke}%
\pgfsetstrokeopacity{0.500000}%
\pgfsetdash{}{0pt}%
\pgfsys@defobject{currentmarker}{\pgfqpoint{-0.178472in}{-0.178472in}}{\pgfqpoint{0.178472in}{0.178472in}}{%
\pgfpathmoveto{\pgfqpoint{0.000000in}{-0.178472in}}%
\pgfpathcurveto{\pgfqpoint{0.047331in}{-0.178472in}}{\pgfqpoint{0.092731in}{-0.159667in}}{\pgfqpoint{0.126199in}{-0.126199in}}%
\pgfpathcurveto{\pgfqpoint{0.159667in}{-0.092731in}}{\pgfqpoint{0.178472in}{-0.047331in}}{\pgfqpoint{0.178472in}{0.000000in}}%
\pgfpathcurveto{\pgfqpoint{0.178472in}{0.047331in}}{\pgfqpoint{0.159667in}{0.092731in}}{\pgfqpoint{0.126199in}{0.126199in}}%
\pgfpathcurveto{\pgfqpoint{0.092731in}{0.159667in}}{\pgfqpoint{0.047331in}{0.178472in}}{\pgfqpoint{0.000000in}{0.178472in}}%
\pgfpathcurveto{\pgfqpoint{-0.047331in}{0.178472in}}{\pgfqpoint{-0.092731in}{0.159667in}}{\pgfqpoint{-0.126199in}{0.126199in}}%
\pgfpathcurveto{\pgfqpoint{-0.159667in}{0.092731in}}{\pgfqpoint{-0.178472in}{0.047331in}}{\pgfqpoint{-0.178472in}{0.000000in}}%
\pgfpathcurveto{\pgfqpoint{-0.178472in}{-0.047331in}}{\pgfqpoint{-0.159667in}{-0.092731in}}{\pgfqpoint{-0.126199in}{-0.126199in}}%
\pgfpathcurveto{\pgfqpoint{-0.092731in}{-0.159667in}}{\pgfqpoint{-0.047331in}{-0.178472in}}{\pgfqpoint{0.000000in}{-0.178472in}}%
\pgfpathclose%
\pgfusepath{stroke,fill}%
}%
\begin{pgfscope}%
\pgfsys@transformshift{4.009815in}{2.450139in}%
\pgfsys@useobject{currentmarker}{}%
\end{pgfscope}%
\end{pgfscope}%
\begin{pgfscope}%
\pgfpathrectangle{\pgfqpoint{0.100000in}{0.100000in}}{\pgfqpoint{5.307240in}{3.397500in}}%
\pgfusepath{clip}%
\pgfsetrectcap%
\pgfsetroundjoin%
\pgfsetlinewidth{1.505625pt}%
\definecolor{currentstroke}{rgb}{0.678431,1.000000,0.184314}%
\pgfsetstrokecolor{currentstroke}%
\pgfsetstrokeopacity{0.500000}%
\pgfsetdash{}{0pt}%
\pgfpathmoveto{\pgfqpoint{3.845875in}{2.425145in}}%
\pgfusepath{stroke}%
\end{pgfscope}%
\begin{pgfscope}%
\pgfpathrectangle{\pgfqpoint{0.100000in}{0.100000in}}{\pgfqpoint{5.307240in}{3.397500in}}%
\pgfusepath{clip}%
\pgfsetbuttcap%
\pgfsetroundjoin%
\definecolor{currentfill}{rgb}{0.678431,1.000000,0.184314}%
\pgfsetfillcolor{currentfill}%
\pgfsetfillopacity{0.500000}%
\pgfsetlinewidth{0.250937pt}%
\definecolor{currentstroke}{rgb}{0.000000,0.000000,0.000000}%
\pgfsetstrokecolor{currentstroke}%
\pgfsetstrokeopacity{0.500000}%
\pgfsetdash{}{0pt}%
\pgfsys@defobject{currentmarker}{\pgfqpoint{-0.133333in}{-0.133333in}}{\pgfqpoint{0.133333in}{0.133333in}}{%
\pgfpathmoveto{\pgfqpoint{0.000000in}{-0.133333in}}%
\pgfpathcurveto{\pgfqpoint{0.035360in}{-0.133333in}}{\pgfqpoint{0.069277in}{-0.119284in}}{\pgfqpoint{0.094281in}{-0.094281in}}%
\pgfpathcurveto{\pgfqpoint{0.119284in}{-0.069277in}}{\pgfqpoint{0.133333in}{-0.035360in}}{\pgfqpoint{0.133333in}{0.000000in}}%
\pgfpathcurveto{\pgfqpoint{0.133333in}{0.035360in}}{\pgfqpoint{0.119284in}{0.069277in}}{\pgfqpoint{0.094281in}{0.094281in}}%
\pgfpathcurveto{\pgfqpoint{0.069277in}{0.119284in}}{\pgfqpoint{0.035360in}{0.133333in}}{\pgfqpoint{0.000000in}{0.133333in}}%
\pgfpathcurveto{\pgfqpoint{-0.035360in}{0.133333in}}{\pgfqpoint{-0.069277in}{0.119284in}}{\pgfqpoint{-0.094281in}{0.094281in}}%
\pgfpathcurveto{\pgfqpoint{-0.119284in}{0.069277in}}{\pgfqpoint{-0.133333in}{0.035360in}}{\pgfqpoint{-0.133333in}{0.000000in}}%
\pgfpathcurveto{\pgfqpoint{-0.133333in}{-0.035360in}}{\pgfqpoint{-0.119284in}{-0.069277in}}{\pgfqpoint{-0.094281in}{-0.094281in}}%
\pgfpathcurveto{\pgfqpoint{-0.069277in}{-0.119284in}}{\pgfqpoint{-0.035360in}{-0.133333in}}{\pgfqpoint{0.000000in}{-0.133333in}}%
\pgfpathclose%
\pgfusepath{stroke,fill}%
}%
\begin{pgfscope}%
\pgfsys@transformshift{3.845875in}{2.425145in}%
\pgfsys@useobject{currentmarker}{}%
\end{pgfscope}%
\end{pgfscope}%
\begin{pgfscope}%
\pgfpathrectangle{\pgfqpoint{0.100000in}{0.100000in}}{\pgfqpoint{5.307240in}{3.397500in}}%
\pgfusepath{clip}%
\pgfsetrectcap%
\pgfsetroundjoin%
\pgfsetlinewidth{1.505625pt}%
\definecolor{currentstroke}{rgb}{0.678431,1.000000,0.184314}%
\pgfsetstrokecolor{currentstroke}%
\pgfsetstrokeopacity{0.500000}%
\pgfsetdash{}{0pt}%
\pgfpathmoveto{\pgfqpoint{3.962549in}{2.353919in}}%
\pgfusepath{stroke}%
\end{pgfscope}%
\begin{pgfscope}%
\pgfpathrectangle{\pgfqpoint{0.100000in}{0.100000in}}{\pgfqpoint{5.307240in}{3.397500in}}%
\pgfusepath{clip}%
\pgfsetbuttcap%
\pgfsetroundjoin%
\definecolor{currentfill}{rgb}{0.678431,1.000000,0.184314}%
\pgfsetfillcolor{currentfill}%
\pgfsetfillopacity{0.500000}%
\pgfsetlinewidth{0.250937pt}%
\definecolor{currentstroke}{rgb}{0.000000,0.000000,0.000000}%
\pgfsetstrokecolor{currentstroke}%
\pgfsetstrokeopacity{0.500000}%
\pgfsetdash{}{0pt}%
\pgfsys@defobject{currentmarker}{\pgfqpoint{-0.145833in}{-0.145833in}}{\pgfqpoint{0.145833in}{0.145833in}}{%
\pgfpathmoveto{\pgfqpoint{0.000000in}{-0.145833in}}%
\pgfpathcurveto{\pgfqpoint{0.038675in}{-0.145833in}}{\pgfqpoint{0.075772in}{-0.130467in}}{\pgfqpoint{0.103120in}{-0.103120in}}%
\pgfpathcurveto{\pgfqpoint{0.130467in}{-0.075772in}}{\pgfqpoint{0.145833in}{-0.038675in}}{\pgfqpoint{0.145833in}{0.000000in}}%
\pgfpathcurveto{\pgfqpoint{0.145833in}{0.038675in}}{\pgfqpoint{0.130467in}{0.075772in}}{\pgfqpoint{0.103120in}{0.103120in}}%
\pgfpathcurveto{\pgfqpoint{0.075772in}{0.130467in}}{\pgfqpoint{0.038675in}{0.145833in}}{\pgfqpoint{0.000000in}{0.145833in}}%
\pgfpathcurveto{\pgfqpoint{-0.038675in}{0.145833in}}{\pgfqpoint{-0.075772in}{0.130467in}}{\pgfqpoint{-0.103120in}{0.103120in}}%
\pgfpathcurveto{\pgfqpoint{-0.130467in}{0.075772in}}{\pgfqpoint{-0.145833in}{0.038675in}}{\pgfqpoint{-0.145833in}{0.000000in}}%
\pgfpathcurveto{\pgfqpoint{-0.145833in}{-0.038675in}}{\pgfqpoint{-0.130467in}{-0.075772in}}{\pgfqpoint{-0.103120in}{-0.103120in}}%
\pgfpathcurveto{\pgfqpoint{-0.075772in}{-0.130467in}}{\pgfqpoint{-0.038675in}{-0.145833in}}{\pgfqpoint{0.000000in}{-0.145833in}}%
\pgfpathclose%
\pgfusepath{stroke,fill}%
}%
\begin{pgfscope}%
\pgfsys@transformshift{3.962549in}{2.353919in}%
\pgfsys@useobject{currentmarker}{}%
\end{pgfscope}%
\end{pgfscope}%
\begin{pgfscope}%
\pgfpathrectangle{\pgfqpoint{0.100000in}{0.100000in}}{\pgfqpoint{5.307240in}{3.397500in}}%
\pgfusepath{clip}%
\pgfsetrectcap%
\pgfsetroundjoin%
\pgfsetlinewidth{1.505625pt}%
\definecolor{currentstroke}{rgb}{0.678431,1.000000,0.184314}%
\pgfsetstrokecolor{currentstroke}%
\pgfsetstrokeopacity{0.500000}%
\pgfsetdash{}{0pt}%
\pgfpathmoveto{\pgfqpoint{3.860801in}{2.348054in}}%
\pgfusepath{stroke}%
\end{pgfscope}%
\begin{pgfscope}%
\pgfpathrectangle{\pgfqpoint{0.100000in}{0.100000in}}{\pgfqpoint{5.307240in}{3.397500in}}%
\pgfusepath{clip}%
\pgfsetbuttcap%
\pgfsetroundjoin%
\definecolor{currentfill}{rgb}{0.678431,1.000000,0.184314}%
\pgfsetfillcolor{currentfill}%
\pgfsetfillopacity{0.500000}%
\pgfsetlinewidth{0.250937pt}%
\definecolor{currentstroke}{rgb}{0.000000,0.000000,0.000000}%
\pgfsetstrokecolor{currentstroke}%
\pgfsetstrokeopacity{0.500000}%
\pgfsetdash{}{0pt}%
\pgfsys@defobject{currentmarker}{\pgfqpoint{-0.103472in}{-0.103472in}}{\pgfqpoint{0.103472in}{0.103472in}}{%
\pgfpathmoveto{\pgfqpoint{0.000000in}{-0.103472in}}%
\pgfpathcurveto{\pgfqpoint{0.027441in}{-0.103472in}}{\pgfqpoint{0.053762in}{-0.092570in}}{\pgfqpoint{0.073166in}{-0.073166in}}%
\pgfpathcurveto{\pgfqpoint{0.092570in}{-0.053762in}}{\pgfqpoint{0.103472in}{-0.027441in}}{\pgfqpoint{0.103472in}{0.000000in}}%
\pgfpathcurveto{\pgfqpoint{0.103472in}{0.027441in}}{\pgfqpoint{0.092570in}{0.053762in}}{\pgfqpoint{0.073166in}{0.073166in}}%
\pgfpathcurveto{\pgfqpoint{0.053762in}{0.092570in}}{\pgfqpoint{0.027441in}{0.103472in}}{\pgfqpoint{0.000000in}{0.103472in}}%
\pgfpathcurveto{\pgfqpoint{-0.027441in}{0.103472in}}{\pgfqpoint{-0.053762in}{0.092570in}}{\pgfqpoint{-0.073166in}{0.073166in}}%
\pgfpathcurveto{\pgfqpoint{-0.092570in}{0.053762in}}{\pgfqpoint{-0.103472in}{0.027441in}}{\pgfqpoint{-0.103472in}{0.000000in}}%
\pgfpathcurveto{\pgfqpoint{-0.103472in}{-0.027441in}}{\pgfqpoint{-0.092570in}{-0.053762in}}{\pgfqpoint{-0.073166in}{-0.073166in}}%
\pgfpathcurveto{\pgfqpoint{-0.053762in}{-0.092570in}}{\pgfqpoint{-0.027441in}{-0.103472in}}{\pgfqpoint{0.000000in}{-0.103472in}}%
\pgfpathclose%
\pgfusepath{stroke,fill}%
}%
\begin{pgfscope}%
\pgfsys@transformshift{3.860801in}{2.348054in}%
\pgfsys@useobject{currentmarker}{}%
\end{pgfscope}%
\end{pgfscope}%
\begin{pgfscope}%
\pgfpathrectangle{\pgfqpoint{0.100000in}{0.100000in}}{\pgfqpoint{5.307240in}{3.397500in}}%
\pgfusepath{clip}%
\pgfsetrectcap%
\pgfsetroundjoin%
\pgfsetlinewidth{1.505625pt}%
\definecolor{currentstroke}{rgb}{0.678431,1.000000,0.184314}%
\pgfsetstrokecolor{currentstroke}%
\pgfsetstrokeopacity{0.500000}%
\pgfsetdash{}{0pt}%
\pgfpathmoveto{\pgfqpoint{3.943240in}{2.408888in}}%
\pgfusepath{stroke}%
\end{pgfscope}%
\begin{pgfscope}%
\pgfpathrectangle{\pgfqpoint{0.100000in}{0.100000in}}{\pgfqpoint{5.307240in}{3.397500in}}%
\pgfusepath{clip}%
\pgfsetbuttcap%
\pgfsetroundjoin%
\definecolor{currentfill}{rgb}{0.678431,1.000000,0.184314}%
\pgfsetfillcolor{currentfill}%
\pgfsetfillopacity{0.500000}%
\pgfsetlinewidth{0.250937pt}%
\definecolor{currentstroke}{rgb}{0.000000,0.000000,0.000000}%
\pgfsetstrokecolor{currentstroke}%
\pgfsetstrokeopacity{0.500000}%
\pgfsetdash{}{0pt}%
\pgfsys@defobject{currentmarker}{\pgfqpoint{-0.112500in}{-0.112500in}}{\pgfqpoint{0.112500in}{0.112500in}}{%
\pgfpathmoveto{\pgfqpoint{0.000000in}{-0.112500in}}%
\pgfpathcurveto{\pgfqpoint{0.029835in}{-0.112500in}}{\pgfqpoint{0.058453in}{-0.100646in}}{\pgfqpoint{0.079550in}{-0.079550in}}%
\pgfpathcurveto{\pgfqpoint{0.100646in}{-0.058453in}}{\pgfqpoint{0.112500in}{-0.029835in}}{\pgfqpoint{0.112500in}{0.000000in}}%
\pgfpathcurveto{\pgfqpoint{0.112500in}{0.029835in}}{\pgfqpoint{0.100646in}{0.058453in}}{\pgfqpoint{0.079550in}{0.079550in}}%
\pgfpathcurveto{\pgfqpoint{0.058453in}{0.100646in}}{\pgfqpoint{0.029835in}{0.112500in}}{\pgfqpoint{0.000000in}{0.112500in}}%
\pgfpathcurveto{\pgfqpoint{-0.029835in}{0.112500in}}{\pgfqpoint{-0.058453in}{0.100646in}}{\pgfqpoint{-0.079550in}{0.079550in}}%
\pgfpathcurveto{\pgfqpoint{-0.100646in}{0.058453in}}{\pgfqpoint{-0.112500in}{0.029835in}}{\pgfqpoint{-0.112500in}{0.000000in}}%
\pgfpathcurveto{\pgfqpoint{-0.112500in}{-0.029835in}}{\pgfqpoint{-0.100646in}{-0.058453in}}{\pgfqpoint{-0.079550in}{-0.079550in}}%
\pgfpathcurveto{\pgfqpoint{-0.058453in}{-0.100646in}}{\pgfqpoint{-0.029835in}{-0.112500in}}{\pgfqpoint{0.000000in}{-0.112500in}}%
\pgfpathclose%
\pgfusepath{stroke,fill}%
}%
\begin{pgfscope}%
\pgfsys@transformshift{3.943240in}{2.408888in}%
\pgfsys@useobject{currentmarker}{}%
\end{pgfscope}%
\end{pgfscope}%
\begin{pgfscope}%
\pgfpathrectangle{\pgfqpoint{0.100000in}{0.100000in}}{\pgfqpoint{5.307240in}{3.397500in}}%
\pgfusepath{clip}%
\pgfsetrectcap%
\pgfsetroundjoin%
\pgfsetlinewidth{1.505625pt}%
\definecolor{currentstroke}{rgb}{0.678431,1.000000,0.184314}%
\pgfsetstrokecolor{currentstroke}%
\pgfsetstrokeopacity{0.500000}%
\pgfsetdash{}{0pt}%
\pgfpathmoveto{\pgfqpoint{3.957296in}{2.514016in}}%
\pgfusepath{stroke}%
\end{pgfscope}%
\begin{pgfscope}%
\pgfpathrectangle{\pgfqpoint{0.100000in}{0.100000in}}{\pgfqpoint{5.307240in}{3.397500in}}%
\pgfusepath{clip}%
\pgfsetbuttcap%
\pgfsetroundjoin%
\definecolor{currentfill}{rgb}{0.678431,1.000000,0.184314}%
\pgfsetfillcolor{currentfill}%
\pgfsetfillopacity{0.500000}%
\pgfsetlinewidth{0.250937pt}%
\definecolor{currentstroke}{rgb}{0.000000,0.000000,0.000000}%
\pgfsetstrokecolor{currentstroke}%
\pgfsetstrokeopacity{0.500000}%
\pgfsetdash{}{0pt}%
\pgfsys@defobject{currentmarker}{\pgfqpoint{-0.114583in}{-0.114583in}}{\pgfqpoint{0.114583in}{0.114583in}}{%
\pgfpathmoveto{\pgfqpoint{0.000000in}{-0.114583in}}%
\pgfpathcurveto{\pgfqpoint{0.030388in}{-0.114583in}}{\pgfqpoint{0.059535in}{-0.102510in}}{\pgfqpoint{0.081023in}{-0.081023in}}%
\pgfpathcurveto{\pgfqpoint{0.102510in}{-0.059535in}}{\pgfqpoint{0.114583in}{-0.030388in}}{\pgfqpoint{0.114583in}{0.000000in}}%
\pgfpathcurveto{\pgfqpoint{0.114583in}{0.030388in}}{\pgfqpoint{0.102510in}{0.059535in}}{\pgfqpoint{0.081023in}{0.081023in}}%
\pgfpathcurveto{\pgfqpoint{0.059535in}{0.102510in}}{\pgfqpoint{0.030388in}{0.114583in}}{\pgfqpoint{0.000000in}{0.114583in}}%
\pgfpathcurveto{\pgfqpoint{-0.030388in}{0.114583in}}{\pgfqpoint{-0.059535in}{0.102510in}}{\pgfqpoint{-0.081023in}{0.081023in}}%
\pgfpathcurveto{\pgfqpoint{-0.102510in}{0.059535in}}{\pgfqpoint{-0.114583in}{0.030388in}}{\pgfqpoint{-0.114583in}{0.000000in}}%
\pgfpathcurveto{\pgfqpoint{-0.114583in}{-0.030388in}}{\pgfqpoint{-0.102510in}{-0.059535in}}{\pgfqpoint{-0.081023in}{-0.081023in}}%
\pgfpathcurveto{\pgfqpoint{-0.059535in}{-0.102510in}}{\pgfqpoint{-0.030388in}{-0.114583in}}{\pgfqpoint{0.000000in}{-0.114583in}}%
\pgfpathclose%
\pgfusepath{stroke,fill}%
}%
\begin{pgfscope}%
\pgfsys@transformshift{3.957296in}{2.514016in}%
\pgfsys@useobject{currentmarker}{}%
\end{pgfscope}%
\end{pgfscope}%
\begin{pgfscope}%
\pgfpathrectangle{\pgfqpoint{0.100000in}{0.100000in}}{\pgfqpoint{5.307240in}{3.397500in}}%
\pgfusepath{clip}%
\pgfsetrectcap%
\pgfsetroundjoin%
\pgfsetlinewidth{1.505625pt}%
\definecolor{currentstroke}{rgb}{0.678431,1.000000,0.184314}%
\pgfsetstrokecolor{currentstroke}%
\pgfsetstrokeopacity{0.500000}%
\pgfsetdash{}{0pt}%
\pgfpathmoveto{\pgfqpoint{4.053788in}{2.326265in}}%
\pgfusepath{stroke}%
\end{pgfscope}%
\begin{pgfscope}%
\pgfpathrectangle{\pgfqpoint{0.100000in}{0.100000in}}{\pgfqpoint{5.307240in}{3.397500in}}%
\pgfusepath{clip}%
\pgfsetbuttcap%
\pgfsetroundjoin%
\definecolor{currentfill}{rgb}{0.678431,1.000000,0.184314}%
\pgfsetfillcolor{currentfill}%
\pgfsetfillopacity{0.500000}%
\pgfsetlinewidth{0.250937pt}%
\definecolor{currentstroke}{rgb}{0.000000,0.000000,0.000000}%
\pgfsetstrokecolor{currentstroke}%
\pgfsetstrokeopacity{0.500000}%
\pgfsetdash{}{0pt}%
\pgfsys@defobject{currentmarker}{\pgfqpoint{-0.154167in}{-0.154167in}}{\pgfqpoint{0.154167in}{0.154167in}}{%
\pgfpathmoveto{\pgfqpoint{0.000000in}{-0.154167in}}%
\pgfpathcurveto{\pgfqpoint{0.040885in}{-0.154167in}}{\pgfqpoint{0.080102in}{-0.137923in}}{\pgfqpoint{0.109012in}{-0.109012in}}%
\pgfpathcurveto{\pgfqpoint{0.137923in}{-0.080102in}}{\pgfqpoint{0.154167in}{-0.040885in}}{\pgfqpoint{0.154167in}{0.000000in}}%
\pgfpathcurveto{\pgfqpoint{0.154167in}{0.040885in}}{\pgfqpoint{0.137923in}{0.080102in}}{\pgfqpoint{0.109012in}{0.109012in}}%
\pgfpathcurveto{\pgfqpoint{0.080102in}{0.137923in}}{\pgfqpoint{0.040885in}{0.154167in}}{\pgfqpoint{0.000000in}{0.154167in}}%
\pgfpathcurveto{\pgfqpoint{-0.040885in}{0.154167in}}{\pgfqpoint{-0.080102in}{0.137923in}}{\pgfqpoint{-0.109012in}{0.109012in}}%
\pgfpathcurveto{\pgfqpoint{-0.137923in}{0.080102in}}{\pgfqpoint{-0.154167in}{0.040885in}}{\pgfqpoint{-0.154167in}{0.000000in}}%
\pgfpathcurveto{\pgfqpoint{-0.154167in}{-0.040885in}}{\pgfqpoint{-0.137923in}{-0.080102in}}{\pgfqpoint{-0.109012in}{-0.109012in}}%
\pgfpathcurveto{\pgfqpoint{-0.080102in}{-0.137923in}}{\pgfqpoint{-0.040885in}{-0.154167in}}{\pgfqpoint{0.000000in}{-0.154167in}}%
\pgfpathclose%
\pgfusepath{stroke,fill}%
}%
\begin{pgfscope}%
\pgfsys@transformshift{4.053788in}{2.326265in}%
\pgfsys@useobject{currentmarker}{}%
\end{pgfscope}%
\end{pgfscope}%
\begin{pgfscope}%
\pgfpathrectangle{\pgfqpoint{0.100000in}{0.100000in}}{\pgfqpoint{5.307240in}{3.397500in}}%
\pgfusepath{clip}%
\pgfsetrectcap%
\pgfsetroundjoin%
\pgfsetlinewidth{1.505625pt}%
\definecolor{currentstroke}{rgb}{0.678431,1.000000,0.184314}%
\pgfsetstrokecolor{currentstroke}%
\pgfsetstrokeopacity{0.500000}%
\pgfsetdash{}{0pt}%
\pgfpathmoveto{\pgfqpoint{3.793582in}{2.451657in}}%
\pgfusepath{stroke}%
\end{pgfscope}%
\begin{pgfscope}%
\pgfpathrectangle{\pgfqpoint{0.100000in}{0.100000in}}{\pgfqpoint{5.307240in}{3.397500in}}%
\pgfusepath{clip}%
\pgfsetbuttcap%
\pgfsetroundjoin%
\definecolor{currentfill}{rgb}{0.678431,1.000000,0.184314}%
\pgfsetfillcolor{currentfill}%
\pgfsetfillopacity{0.500000}%
\pgfsetlinewidth{0.250937pt}%
\definecolor{currentstroke}{rgb}{0.000000,0.000000,0.000000}%
\pgfsetstrokecolor{currentstroke}%
\pgfsetstrokeopacity{0.500000}%
\pgfsetdash{}{0pt}%
\pgfsys@defobject{currentmarker}{\pgfqpoint{-0.177083in}{-0.177083in}}{\pgfqpoint{0.177083in}{0.177083in}}{%
\pgfpathmoveto{\pgfqpoint{0.000000in}{-0.177083in}}%
\pgfpathcurveto{\pgfqpoint{0.046963in}{-0.177083in}}{\pgfqpoint{0.092009in}{-0.158425in}}{\pgfqpoint{0.125217in}{-0.125217in}}%
\pgfpathcurveto{\pgfqpoint{0.158425in}{-0.092009in}}{\pgfqpoint{0.177083in}{-0.046963in}}{\pgfqpoint{0.177083in}{0.000000in}}%
\pgfpathcurveto{\pgfqpoint{0.177083in}{0.046963in}}{\pgfqpoint{0.158425in}{0.092009in}}{\pgfqpoint{0.125217in}{0.125217in}}%
\pgfpathcurveto{\pgfqpoint{0.092009in}{0.158425in}}{\pgfqpoint{0.046963in}{0.177083in}}{\pgfqpoint{0.000000in}{0.177083in}}%
\pgfpathcurveto{\pgfqpoint{-0.046963in}{0.177083in}}{\pgfqpoint{-0.092009in}{0.158425in}}{\pgfqpoint{-0.125217in}{0.125217in}}%
\pgfpathcurveto{\pgfqpoint{-0.158425in}{0.092009in}}{\pgfqpoint{-0.177083in}{0.046963in}}{\pgfqpoint{-0.177083in}{0.000000in}}%
\pgfpathcurveto{\pgfqpoint{-0.177083in}{-0.046963in}}{\pgfqpoint{-0.158425in}{-0.092009in}}{\pgfqpoint{-0.125217in}{-0.125217in}}%
\pgfpathcurveto{\pgfqpoint{-0.092009in}{-0.158425in}}{\pgfqpoint{-0.046963in}{-0.177083in}}{\pgfqpoint{0.000000in}{-0.177083in}}%
\pgfpathclose%
\pgfusepath{stroke,fill}%
}%
\begin{pgfscope}%
\pgfsys@transformshift{3.793582in}{2.451657in}%
\pgfsys@useobject{currentmarker}{}%
\end{pgfscope}%
\end{pgfscope}%
\begin{pgfscope}%
\pgfpathrectangle{\pgfqpoint{0.100000in}{0.100000in}}{\pgfqpoint{5.307240in}{3.397500in}}%
\pgfusepath{clip}%
\pgfsetrectcap%
\pgfsetroundjoin%
\pgfsetlinewidth{1.505625pt}%
\definecolor{currentstroke}{rgb}{0.678431,1.000000,0.184314}%
\pgfsetstrokecolor{currentstroke}%
\pgfsetstrokeopacity{0.500000}%
\pgfsetdash{}{0pt}%
\pgfpathmoveto{\pgfqpoint{3.808811in}{2.288800in}}%
\pgfusepath{stroke}%
\end{pgfscope}%
\begin{pgfscope}%
\pgfpathrectangle{\pgfqpoint{0.100000in}{0.100000in}}{\pgfqpoint{5.307240in}{3.397500in}}%
\pgfusepath{clip}%
\pgfsetbuttcap%
\pgfsetroundjoin%
\definecolor{currentfill}{rgb}{0.678431,1.000000,0.184314}%
\pgfsetfillcolor{currentfill}%
\pgfsetfillopacity{0.500000}%
\pgfsetlinewidth{0.250937pt}%
\definecolor{currentstroke}{rgb}{0.000000,0.000000,0.000000}%
\pgfsetstrokecolor{currentstroke}%
\pgfsetstrokeopacity{0.500000}%
\pgfsetdash{}{0pt}%
\pgfsys@defobject{currentmarker}{\pgfqpoint{-0.113889in}{-0.113889in}}{\pgfqpoint{0.113889in}{0.113889in}}{%
\pgfpathmoveto{\pgfqpoint{0.000000in}{-0.113889in}}%
\pgfpathcurveto{\pgfqpoint{0.030204in}{-0.113889in}}{\pgfqpoint{0.059174in}{-0.101889in}}{\pgfqpoint{0.080532in}{-0.080532in}}%
\pgfpathcurveto{\pgfqpoint{0.101889in}{-0.059174in}}{\pgfqpoint{0.113889in}{-0.030204in}}{\pgfqpoint{0.113889in}{0.000000in}}%
\pgfpathcurveto{\pgfqpoint{0.113889in}{0.030204in}}{\pgfqpoint{0.101889in}{0.059174in}}{\pgfqpoint{0.080532in}{0.080532in}}%
\pgfpathcurveto{\pgfqpoint{0.059174in}{0.101889in}}{\pgfqpoint{0.030204in}{0.113889in}}{\pgfqpoint{0.000000in}{0.113889in}}%
\pgfpathcurveto{\pgfqpoint{-0.030204in}{0.113889in}}{\pgfqpoint{-0.059174in}{0.101889in}}{\pgfqpoint{-0.080532in}{0.080532in}}%
\pgfpathcurveto{\pgfqpoint{-0.101889in}{0.059174in}}{\pgfqpoint{-0.113889in}{0.030204in}}{\pgfqpoint{-0.113889in}{0.000000in}}%
\pgfpathcurveto{\pgfqpoint{-0.113889in}{-0.030204in}}{\pgfqpoint{-0.101889in}{-0.059174in}}{\pgfqpoint{-0.080532in}{-0.080532in}}%
\pgfpathcurveto{\pgfqpoint{-0.059174in}{-0.101889in}}{\pgfqpoint{-0.030204in}{-0.113889in}}{\pgfqpoint{0.000000in}{-0.113889in}}%
\pgfpathclose%
\pgfusepath{stroke,fill}%
}%
\begin{pgfscope}%
\pgfsys@transformshift{3.808811in}{2.288800in}%
\pgfsys@useobject{currentmarker}{}%
\end{pgfscope}%
\end{pgfscope}%
\begin{pgfscope}%
\pgfpathrectangle{\pgfqpoint{0.100000in}{0.100000in}}{\pgfqpoint{5.307240in}{3.397500in}}%
\pgfusepath{clip}%
\pgfsetrectcap%
\pgfsetroundjoin%
\pgfsetlinewidth{1.505625pt}%
\definecolor{currentstroke}{rgb}{0.678431,1.000000,0.184314}%
\pgfsetstrokecolor{currentstroke}%
\pgfsetstrokeopacity{0.500000}%
\pgfsetdash{}{0pt}%
\pgfpathmoveto{\pgfqpoint{3.985057in}{2.494396in}}%
\pgfusepath{stroke}%
\end{pgfscope}%
\begin{pgfscope}%
\pgfpathrectangle{\pgfqpoint{0.100000in}{0.100000in}}{\pgfqpoint{5.307240in}{3.397500in}}%
\pgfusepath{clip}%
\pgfsetbuttcap%
\pgfsetroundjoin%
\definecolor{currentfill}{rgb}{0.678431,1.000000,0.184314}%
\pgfsetfillcolor{currentfill}%
\pgfsetfillopacity{0.500000}%
\pgfsetlinewidth{0.250937pt}%
\definecolor{currentstroke}{rgb}{0.000000,0.000000,0.000000}%
\pgfsetstrokecolor{currentstroke}%
\pgfsetstrokeopacity{0.500000}%
\pgfsetdash{}{0pt}%
\pgfsys@defobject{currentmarker}{\pgfqpoint{-0.147917in}{-0.147917in}}{\pgfqpoint{0.147917in}{0.147917in}}{%
\pgfpathmoveto{\pgfqpoint{0.000000in}{-0.147917in}}%
\pgfpathcurveto{\pgfqpoint{0.039228in}{-0.147917in}}{\pgfqpoint{0.076855in}{-0.132331in}}{\pgfqpoint{0.104593in}{-0.104593in}}%
\pgfpathcurveto{\pgfqpoint{0.132331in}{-0.076855in}}{\pgfqpoint{0.147917in}{-0.039228in}}{\pgfqpoint{0.147917in}{0.000000in}}%
\pgfpathcurveto{\pgfqpoint{0.147917in}{0.039228in}}{\pgfqpoint{0.132331in}{0.076855in}}{\pgfqpoint{0.104593in}{0.104593in}}%
\pgfpathcurveto{\pgfqpoint{0.076855in}{0.132331in}}{\pgfqpoint{0.039228in}{0.147917in}}{\pgfqpoint{0.000000in}{0.147917in}}%
\pgfpathcurveto{\pgfqpoint{-0.039228in}{0.147917in}}{\pgfqpoint{-0.076855in}{0.132331in}}{\pgfqpoint{-0.104593in}{0.104593in}}%
\pgfpathcurveto{\pgfqpoint{-0.132331in}{0.076855in}}{\pgfqpoint{-0.147917in}{0.039228in}}{\pgfqpoint{-0.147917in}{0.000000in}}%
\pgfpathcurveto{\pgfqpoint{-0.147917in}{-0.039228in}}{\pgfqpoint{-0.132331in}{-0.076855in}}{\pgfqpoint{-0.104593in}{-0.104593in}}%
\pgfpathcurveto{\pgfqpoint{-0.076855in}{-0.132331in}}{\pgfqpoint{-0.039228in}{-0.147917in}}{\pgfqpoint{0.000000in}{-0.147917in}}%
\pgfpathclose%
\pgfusepath{stroke,fill}%
}%
\begin{pgfscope}%
\pgfsys@transformshift{3.985057in}{2.494396in}%
\pgfsys@useobject{currentmarker}{}%
\end{pgfscope}%
\end{pgfscope}%
\begin{pgfscope}%
\pgfpathrectangle{\pgfqpoint{0.100000in}{0.100000in}}{\pgfqpoint{5.307240in}{3.397500in}}%
\pgfusepath{clip}%
\pgfsetrectcap%
\pgfsetroundjoin%
\pgfsetlinewidth{1.505625pt}%
\definecolor{currentstroke}{rgb}{0.678431,1.000000,0.184314}%
\pgfsetstrokecolor{currentstroke}%
\pgfsetstrokeopacity{0.500000}%
\pgfsetdash{}{0pt}%
\pgfpathmoveto{\pgfqpoint{3.282541in}{2.833276in}}%
\pgfusepath{stroke}%
\end{pgfscope}%
\begin{pgfscope}%
\pgfpathrectangle{\pgfqpoint{0.100000in}{0.100000in}}{\pgfqpoint{5.307240in}{3.397500in}}%
\pgfusepath{clip}%
\pgfsetbuttcap%
\pgfsetroundjoin%
\definecolor{currentfill}{rgb}{0.678431,1.000000,0.184314}%
\pgfsetfillcolor{currentfill}%
\pgfsetfillopacity{0.500000}%
\pgfsetlinewidth{0.250937pt}%
\definecolor{currentstroke}{rgb}{0.000000,0.000000,0.000000}%
\pgfsetstrokecolor{currentstroke}%
\pgfsetstrokeopacity{0.500000}%
\pgfsetdash{}{0pt}%
\pgfsys@defobject{currentmarker}{\pgfqpoint{-0.061111in}{-0.061111in}}{\pgfqpoint{0.061111in}{0.061111in}}{%
\pgfpathmoveto{\pgfqpoint{0.000000in}{-0.061111in}}%
\pgfpathcurveto{\pgfqpoint{0.016207in}{-0.061111in}}{\pgfqpoint{0.031752in}{-0.054672in}}{\pgfqpoint{0.043212in}{-0.043212in}}%
\pgfpathcurveto{\pgfqpoint{0.054672in}{-0.031752in}}{\pgfqpoint{0.061111in}{-0.016207in}}{\pgfqpoint{0.061111in}{0.000000in}}%
\pgfpathcurveto{\pgfqpoint{0.061111in}{0.016207in}}{\pgfqpoint{0.054672in}{0.031752in}}{\pgfqpoint{0.043212in}{0.043212in}}%
\pgfpathcurveto{\pgfqpoint{0.031752in}{0.054672in}}{\pgfqpoint{0.016207in}{0.061111in}}{\pgfqpoint{0.000000in}{0.061111in}}%
\pgfpathcurveto{\pgfqpoint{-0.016207in}{0.061111in}}{\pgfqpoint{-0.031752in}{0.054672in}}{\pgfqpoint{-0.043212in}{0.043212in}}%
\pgfpathcurveto{\pgfqpoint{-0.054672in}{0.031752in}}{\pgfqpoint{-0.061111in}{0.016207in}}{\pgfqpoint{-0.061111in}{0.000000in}}%
\pgfpathcurveto{\pgfqpoint{-0.061111in}{-0.016207in}}{\pgfqpoint{-0.054672in}{-0.031752in}}{\pgfqpoint{-0.043212in}{-0.043212in}}%
\pgfpathcurveto{\pgfqpoint{-0.031752in}{-0.054672in}}{\pgfqpoint{-0.016207in}{-0.061111in}}{\pgfqpoint{0.000000in}{-0.061111in}}%
\pgfpathclose%
\pgfusepath{stroke,fill}%
}%
\begin{pgfscope}%
\pgfsys@transformshift{3.282541in}{2.833276in}%
\pgfsys@useobject{currentmarker}{}%
\end{pgfscope}%
\end{pgfscope}%
\begin{pgfscope}%
\pgfpathrectangle{\pgfqpoint{0.100000in}{0.100000in}}{\pgfqpoint{5.307240in}{3.397500in}}%
\pgfusepath{clip}%
\pgfsetrectcap%
\pgfsetroundjoin%
\pgfsetlinewidth{1.505625pt}%
\definecolor{currentstroke}{rgb}{0.678431,1.000000,0.184314}%
\pgfsetstrokecolor{currentstroke}%
\pgfsetstrokeopacity{0.500000}%
\pgfsetdash{}{0pt}%
\pgfpathmoveto{\pgfqpoint{3.135309in}{2.524750in}}%
\pgfusepath{stroke}%
\end{pgfscope}%
\begin{pgfscope}%
\pgfpathrectangle{\pgfqpoint{0.100000in}{0.100000in}}{\pgfqpoint{5.307240in}{3.397500in}}%
\pgfusepath{clip}%
\pgfsetbuttcap%
\pgfsetroundjoin%
\definecolor{currentfill}{rgb}{0.678431,1.000000,0.184314}%
\pgfsetfillcolor{currentfill}%
\pgfsetfillopacity{0.500000}%
\pgfsetlinewidth{0.250937pt}%
\definecolor{currentstroke}{rgb}{0.000000,0.000000,0.000000}%
\pgfsetstrokecolor{currentstroke}%
\pgfsetstrokeopacity{0.500000}%
\pgfsetdash{}{0pt}%
\pgfsys@defobject{currentmarker}{\pgfqpoint{-0.034028in}{-0.034028in}}{\pgfqpoint{0.034028in}{0.034028in}}{%
\pgfpathmoveto{\pgfqpoint{0.000000in}{-0.034028in}}%
\pgfpathcurveto{\pgfqpoint{0.009024in}{-0.034028in}}{\pgfqpoint{0.017680in}{-0.030442in}}{\pgfqpoint{0.024061in}{-0.024061in}}%
\pgfpathcurveto{\pgfqpoint{0.030442in}{-0.017680in}}{\pgfqpoint{0.034028in}{-0.009024in}}{\pgfqpoint{0.034028in}{0.000000in}}%
\pgfpathcurveto{\pgfqpoint{0.034028in}{0.009024in}}{\pgfqpoint{0.030442in}{0.017680in}}{\pgfqpoint{0.024061in}{0.024061in}}%
\pgfpathcurveto{\pgfqpoint{0.017680in}{0.030442in}}{\pgfqpoint{0.009024in}{0.034028in}}{\pgfqpoint{0.000000in}{0.034028in}}%
\pgfpathcurveto{\pgfqpoint{-0.009024in}{0.034028in}}{\pgfqpoint{-0.017680in}{0.030442in}}{\pgfqpoint{-0.024061in}{0.024061in}}%
\pgfpathcurveto{\pgfqpoint{-0.030442in}{0.017680in}}{\pgfqpoint{-0.034028in}{0.009024in}}{\pgfqpoint{-0.034028in}{0.000000in}}%
\pgfpathcurveto{\pgfqpoint{-0.034028in}{-0.009024in}}{\pgfqpoint{-0.030442in}{-0.017680in}}{\pgfqpoint{-0.024061in}{-0.024061in}}%
\pgfpathcurveto{\pgfqpoint{-0.017680in}{-0.030442in}}{\pgfqpoint{-0.009024in}{-0.034028in}}{\pgfqpoint{0.000000in}{-0.034028in}}%
\pgfpathclose%
\pgfusepath{stroke,fill}%
}%
\begin{pgfscope}%
\pgfsys@transformshift{3.135309in}{2.524750in}%
\pgfsys@useobject{currentmarker}{}%
\end{pgfscope}%
\end{pgfscope}%
\begin{pgfscope}%
\pgfpathrectangle{\pgfqpoint{0.100000in}{0.100000in}}{\pgfqpoint{5.307240in}{3.397500in}}%
\pgfusepath{clip}%
\pgfsetrectcap%
\pgfsetroundjoin%
\pgfsetlinewidth{1.505625pt}%
\definecolor{currentstroke}{rgb}{0.678431,1.000000,0.184314}%
\pgfsetstrokecolor{currentstroke}%
\pgfsetstrokeopacity{0.500000}%
\pgfsetdash{}{0pt}%
\pgfpathmoveto{\pgfqpoint{3.194935in}{2.620713in}}%
\pgfusepath{stroke}%
\end{pgfscope}%
\begin{pgfscope}%
\pgfpathrectangle{\pgfqpoint{0.100000in}{0.100000in}}{\pgfqpoint{5.307240in}{3.397500in}}%
\pgfusepath{clip}%
\pgfsetbuttcap%
\pgfsetroundjoin%
\definecolor{currentfill}{rgb}{0.678431,1.000000,0.184314}%
\pgfsetfillcolor{currentfill}%
\pgfsetfillopacity{0.500000}%
\pgfsetlinewidth{0.250937pt}%
\definecolor{currentstroke}{rgb}{0.000000,0.000000,0.000000}%
\pgfsetstrokecolor{currentstroke}%
\pgfsetstrokeopacity{0.500000}%
\pgfsetdash{}{0pt}%
\pgfsys@defobject{currentmarker}{\pgfqpoint{-0.045139in}{-0.045139in}}{\pgfqpoint{0.045139in}{0.045139in}}{%
\pgfpathmoveto{\pgfqpoint{0.000000in}{-0.045139in}}%
\pgfpathcurveto{\pgfqpoint{0.011971in}{-0.045139in}}{\pgfqpoint{0.023453in}{-0.040383in}}{\pgfqpoint{0.031918in}{-0.031918in}}%
\pgfpathcurveto{\pgfqpoint{0.040383in}{-0.023453in}}{\pgfqpoint{0.045139in}{-0.011971in}}{\pgfqpoint{0.045139in}{0.000000in}}%
\pgfpathcurveto{\pgfqpoint{0.045139in}{0.011971in}}{\pgfqpoint{0.040383in}{0.023453in}}{\pgfqpoint{0.031918in}{0.031918in}}%
\pgfpathcurveto{\pgfqpoint{0.023453in}{0.040383in}}{\pgfqpoint{0.011971in}{0.045139in}}{\pgfqpoint{0.000000in}{0.045139in}}%
\pgfpathcurveto{\pgfqpoint{-0.011971in}{0.045139in}}{\pgfqpoint{-0.023453in}{0.040383in}}{\pgfqpoint{-0.031918in}{0.031918in}}%
\pgfpathcurveto{\pgfqpoint{-0.040383in}{0.023453in}}{\pgfqpoint{-0.045139in}{0.011971in}}{\pgfqpoint{-0.045139in}{0.000000in}}%
\pgfpathcurveto{\pgfqpoint{-0.045139in}{-0.011971in}}{\pgfqpoint{-0.040383in}{-0.023453in}}{\pgfqpoint{-0.031918in}{-0.031918in}}%
\pgfpathcurveto{\pgfqpoint{-0.023453in}{-0.040383in}}{\pgfqpoint{-0.011971in}{-0.045139in}}{\pgfqpoint{0.000000in}{-0.045139in}}%
\pgfpathclose%
\pgfusepath{stroke,fill}%
}%
\begin{pgfscope}%
\pgfsys@transformshift{3.194935in}{2.620713in}%
\pgfsys@useobject{currentmarker}{}%
\end{pgfscope}%
\end{pgfscope}%
\begin{pgfscope}%
\pgfpathrectangle{\pgfqpoint{0.100000in}{0.100000in}}{\pgfqpoint{5.307240in}{3.397500in}}%
\pgfusepath{clip}%
\pgfsetrectcap%
\pgfsetroundjoin%
\pgfsetlinewidth{1.505625pt}%
\definecolor{currentstroke}{rgb}{0.678431,1.000000,0.184314}%
\pgfsetstrokecolor{currentstroke}%
\pgfsetstrokeopacity{0.500000}%
\pgfsetdash{}{0pt}%
\pgfpathmoveto{\pgfqpoint{3.264375in}{2.510905in}}%
\pgfusepath{stroke}%
\end{pgfscope}%
\begin{pgfscope}%
\pgfpathrectangle{\pgfqpoint{0.100000in}{0.100000in}}{\pgfqpoint{5.307240in}{3.397500in}}%
\pgfusepath{clip}%
\pgfsetbuttcap%
\pgfsetroundjoin%
\definecolor{currentfill}{rgb}{0.678431,1.000000,0.184314}%
\pgfsetfillcolor{currentfill}%
\pgfsetfillopacity{0.500000}%
\pgfsetlinewidth{0.250937pt}%
\definecolor{currentstroke}{rgb}{0.000000,0.000000,0.000000}%
\pgfsetstrokecolor{currentstroke}%
\pgfsetstrokeopacity{0.500000}%
\pgfsetdash{}{0pt}%
\pgfsys@defobject{currentmarker}{\pgfqpoint{-0.029861in}{-0.029861in}}{\pgfqpoint{0.029861in}{0.029861in}}{%
\pgfpathmoveto{\pgfqpoint{0.000000in}{-0.029861in}}%
\pgfpathcurveto{\pgfqpoint{0.007919in}{-0.029861in}}{\pgfqpoint{0.015515in}{-0.026715in}}{\pgfqpoint{0.021115in}{-0.021115in}}%
\pgfpathcurveto{\pgfqpoint{0.026715in}{-0.015515in}}{\pgfqpoint{0.029861in}{-0.007919in}}{\pgfqpoint{0.029861in}{0.000000in}}%
\pgfpathcurveto{\pgfqpoint{0.029861in}{0.007919in}}{\pgfqpoint{0.026715in}{0.015515in}}{\pgfqpoint{0.021115in}{0.021115in}}%
\pgfpathcurveto{\pgfqpoint{0.015515in}{0.026715in}}{\pgfqpoint{0.007919in}{0.029861in}}{\pgfqpoint{0.000000in}{0.029861in}}%
\pgfpathcurveto{\pgfqpoint{-0.007919in}{0.029861in}}{\pgfqpoint{-0.015515in}{0.026715in}}{\pgfqpoint{-0.021115in}{0.021115in}}%
\pgfpathcurveto{\pgfqpoint{-0.026715in}{0.015515in}}{\pgfqpoint{-0.029861in}{0.007919in}}{\pgfqpoint{-0.029861in}{0.000000in}}%
\pgfpathcurveto{\pgfqpoint{-0.029861in}{-0.007919in}}{\pgfqpoint{-0.026715in}{-0.015515in}}{\pgfqpoint{-0.021115in}{-0.021115in}}%
\pgfpathcurveto{\pgfqpoint{-0.015515in}{-0.026715in}}{\pgfqpoint{-0.007919in}{-0.029861in}}{\pgfqpoint{0.000000in}{-0.029861in}}%
\pgfpathclose%
\pgfusepath{stroke,fill}%
}%
\begin{pgfscope}%
\pgfsys@transformshift{3.264375in}{2.510905in}%
\pgfsys@useobject{currentmarker}{}%
\end{pgfscope}%
\end{pgfscope}%
\begin{pgfscope}%
\pgfpathrectangle{\pgfqpoint{0.100000in}{0.100000in}}{\pgfqpoint{5.307240in}{3.397500in}}%
\pgfusepath{clip}%
\pgfsetrectcap%
\pgfsetroundjoin%
\pgfsetlinewidth{1.505625pt}%
\definecolor{currentstroke}{rgb}{0.678431,1.000000,0.184314}%
\pgfsetstrokecolor{currentstroke}%
\pgfsetstrokeopacity{0.500000}%
\pgfsetdash{}{0pt}%
\pgfpathmoveto{\pgfqpoint{3.120004in}{2.687996in}}%
\pgfusepath{stroke}%
\end{pgfscope}%
\begin{pgfscope}%
\pgfpathrectangle{\pgfqpoint{0.100000in}{0.100000in}}{\pgfqpoint{5.307240in}{3.397500in}}%
\pgfusepath{clip}%
\pgfsetbuttcap%
\pgfsetroundjoin%
\definecolor{currentfill}{rgb}{0.678431,1.000000,0.184314}%
\pgfsetfillcolor{currentfill}%
\pgfsetfillopacity{0.500000}%
\pgfsetlinewidth{0.250937pt}%
\definecolor{currentstroke}{rgb}{0.000000,0.000000,0.000000}%
\pgfsetstrokecolor{currentstroke}%
\pgfsetstrokeopacity{0.500000}%
\pgfsetdash{}{0pt}%
\pgfsys@defobject{currentmarker}{\pgfqpoint{-0.034722in}{-0.034722in}}{\pgfqpoint{0.034722in}{0.034722in}}{%
\pgfpathmoveto{\pgfqpoint{0.000000in}{-0.034722in}}%
\pgfpathcurveto{\pgfqpoint{0.009208in}{-0.034722in}}{\pgfqpoint{0.018041in}{-0.031064in}}{\pgfqpoint{0.024552in}{-0.024552in}}%
\pgfpathcurveto{\pgfqpoint{0.031064in}{-0.018041in}}{\pgfqpoint{0.034722in}{-0.009208in}}{\pgfqpoint{0.034722in}{0.000000in}}%
\pgfpathcurveto{\pgfqpoint{0.034722in}{0.009208in}}{\pgfqpoint{0.031064in}{0.018041in}}{\pgfqpoint{0.024552in}{0.024552in}}%
\pgfpathcurveto{\pgfqpoint{0.018041in}{0.031064in}}{\pgfqpoint{0.009208in}{0.034722in}}{\pgfqpoint{0.000000in}{0.034722in}}%
\pgfpathcurveto{\pgfqpoint{-0.009208in}{0.034722in}}{\pgfqpoint{-0.018041in}{0.031064in}}{\pgfqpoint{-0.024552in}{0.024552in}}%
\pgfpathcurveto{\pgfqpoint{-0.031064in}{0.018041in}}{\pgfqpoint{-0.034722in}{0.009208in}}{\pgfqpoint{-0.034722in}{0.000000in}}%
\pgfpathcurveto{\pgfqpoint{-0.034722in}{-0.009208in}}{\pgfqpoint{-0.031064in}{-0.018041in}}{\pgfqpoint{-0.024552in}{-0.024552in}}%
\pgfpathcurveto{\pgfqpoint{-0.018041in}{-0.031064in}}{\pgfqpoint{-0.009208in}{-0.034722in}}{\pgfqpoint{0.000000in}{-0.034722in}}%
\pgfpathclose%
\pgfusepath{stroke,fill}%
}%
\begin{pgfscope}%
\pgfsys@transformshift{3.120004in}{2.687996in}%
\pgfsys@useobject{currentmarker}{}%
\end{pgfscope}%
\end{pgfscope}%
\begin{pgfscope}%
\pgfpathrectangle{\pgfqpoint{0.100000in}{0.100000in}}{\pgfqpoint{5.307240in}{3.397500in}}%
\pgfusepath{clip}%
\pgfsetrectcap%
\pgfsetroundjoin%
\pgfsetlinewidth{1.505625pt}%
\definecolor{currentstroke}{rgb}{0.678431,1.000000,0.184314}%
\pgfsetstrokecolor{currentstroke}%
\pgfsetstrokeopacity{0.500000}%
\pgfsetdash{}{0pt}%
\pgfpathmoveto{\pgfqpoint{3.650258in}{0.935362in}}%
\pgfusepath{stroke}%
\end{pgfscope}%
\begin{pgfscope}%
\pgfpathrectangle{\pgfqpoint{0.100000in}{0.100000in}}{\pgfqpoint{5.307240in}{3.397500in}}%
\pgfusepath{clip}%
\pgfsetbuttcap%
\pgfsetroundjoin%
\definecolor{currentfill}{rgb}{0.678431,1.000000,0.184314}%
\pgfsetfillcolor{currentfill}%
\pgfsetfillopacity{0.500000}%
\pgfsetlinewidth{0.250937pt}%
\definecolor{currentstroke}{rgb}{0.000000,0.000000,0.000000}%
\pgfsetstrokecolor{currentstroke}%
\pgfsetstrokeopacity{0.500000}%
\pgfsetdash{}{0pt}%
\pgfsys@defobject{currentmarker}{\pgfqpoint{-0.119444in}{-0.119444in}}{\pgfqpoint{0.119444in}{0.119444in}}{%
\pgfpathmoveto{\pgfqpoint{0.000000in}{-0.119444in}}%
\pgfpathcurveto{\pgfqpoint{0.031677in}{-0.119444in}}{\pgfqpoint{0.062061in}{-0.106859in}}{\pgfqpoint{0.084460in}{-0.084460in}}%
\pgfpathcurveto{\pgfqpoint{0.106859in}{-0.062061in}}{\pgfqpoint{0.119444in}{-0.031677in}}{\pgfqpoint{0.119444in}{0.000000in}}%
\pgfpathcurveto{\pgfqpoint{0.119444in}{0.031677in}}{\pgfqpoint{0.106859in}{0.062061in}}{\pgfqpoint{0.084460in}{0.084460in}}%
\pgfpathcurveto{\pgfqpoint{0.062061in}{0.106859in}}{\pgfqpoint{0.031677in}{0.119444in}}{\pgfqpoint{0.000000in}{0.119444in}}%
\pgfpathcurveto{\pgfqpoint{-0.031677in}{0.119444in}}{\pgfqpoint{-0.062061in}{0.106859in}}{\pgfqpoint{-0.084460in}{0.084460in}}%
\pgfpathcurveto{\pgfqpoint{-0.106859in}{0.062061in}}{\pgfqpoint{-0.119444in}{0.031677in}}{\pgfqpoint{-0.119444in}{0.000000in}}%
\pgfpathcurveto{\pgfqpoint{-0.119444in}{-0.031677in}}{\pgfqpoint{-0.106859in}{-0.062061in}}{\pgfqpoint{-0.084460in}{-0.084460in}}%
\pgfpathcurveto{\pgfqpoint{-0.062061in}{-0.106859in}}{\pgfqpoint{-0.031677in}{-0.119444in}}{\pgfqpoint{0.000000in}{-0.119444in}}%
\pgfpathclose%
\pgfusepath{stroke,fill}%
}%
\begin{pgfscope}%
\pgfsys@transformshift{3.650258in}{0.935362in}%
\pgfsys@useobject{currentmarker}{}%
\end{pgfscope}%
\end{pgfscope}%
\begin{pgfscope}%
\pgfpathrectangle{\pgfqpoint{0.100000in}{0.100000in}}{\pgfqpoint{5.307240in}{3.397500in}}%
\pgfusepath{clip}%
\pgfsetrectcap%
\pgfsetroundjoin%
\pgfsetlinewidth{1.505625pt}%
\definecolor{currentstroke}{rgb}{0.678431,1.000000,0.184314}%
\pgfsetstrokecolor{currentstroke}%
\pgfsetstrokeopacity{0.500000}%
\pgfsetdash{}{0pt}%
\pgfpathmoveto{\pgfqpoint{3.623189in}{1.046547in}}%
\pgfusepath{stroke}%
\end{pgfscope}%
\begin{pgfscope}%
\pgfpathrectangle{\pgfqpoint{0.100000in}{0.100000in}}{\pgfqpoint{5.307240in}{3.397500in}}%
\pgfusepath{clip}%
\pgfsetbuttcap%
\pgfsetroundjoin%
\definecolor{currentfill}{rgb}{0.678431,1.000000,0.184314}%
\pgfsetfillcolor{currentfill}%
\pgfsetfillopacity{0.500000}%
\pgfsetlinewidth{0.250937pt}%
\definecolor{currentstroke}{rgb}{0.000000,0.000000,0.000000}%
\pgfsetstrokecolor{currentstroke}%
\pgfsetstrokeopacity{0.500000}%
\pgfsetdash{}{0pt}%
\pgfsys@defobject{currentmarker}{\pgfqpoint{-0.050694in}{-0.050694in}}{\pgfqpoint{0.050694in}{0.050694in}}{%
\pgfpathmoveto{\pgfqpoint{0.000000in}{-0.050694in}}%
\pgfpathcurveto{\pgfqpoint{0.013444in}{-0.050694in}}{\pgfqpoint{0.026340in}{-0.045353in}}{\pgfqpoint{0.035846in}{-0.035846in}}%
\pgfpathcurveto{\pgfqpoint{0.045353in}{-0.026340in}}{\pgfqpoint{0.050694in}{-0.013444in}}{\pgfqpoint{0.050694in}{0.000000in}}%
\pgfpathcurveto{\pgfqpoint{0.050694in}{0.013444in}}{\pgfqpoint{0.045353in}{0.026340in}}{\pgfqpoint{0.035846in}{0.035846in}}%
\pgfpathcurveto{\pgfqpoint{0.026340in}{0.045353in}}{\pgfqpoint{0.013444in}{0.050694in}}{\pgfqpoint{0.000000in}{0.050694in}}%
\pgfpathcurveto{\pgfqpoint{-0.013444in}{0.050694in}}{\pgfqpoint{-0.026340in}{0.045353in}}{\pgfqpoint{-0.035846in}{0.035846in}}%
\pgfpathcurveto{\pgfqpoint{-0.045353in}{0.026340in}}{\pgfqpoint{-0.050694in}{0.013444in}}{\pgfqpoint{-0.050694in}{0.000000in}}%
\pgfpathcurveto{\pgfqpoint{-0.050694in}{-0.013444in}}{\pgfqpoint{-0.045353in}{-0.026340in}}{\pgfqpoint{-0.035846in}{-0.035846in}}%
\pgfpathcurveto{\pgfqpoint{-0.026340in}{-0.045353in}}{\pgfqpoint{-0.013444in}{-0.050694in}}{\pgfqpoint{0.000000in}{-0.050694in}}%
\pgfpathclose%
\pgfusepath{stroke,fill}%
}%
\begin{pgfscope}%
\pgfsys@transformshift{3.623189in}{1.046547in}%
\pgfsys@useobject{currentmarker}{}%
\end{pgfscope}%
\end{pgfscope}%
\begin{pgfscope}%
\pgfpathrectangle{\pgfqpoint{0.100000in}{0.100000in}}{\pgfqpoint{5.307240in}{3.397500in}}%
\pgfusepath{clip}%
\pgfsetrectcap%
\pgfsetroundjoin%
\pgfsetlinewidth{1.505625pt}%
\definecolor{currentstroke}{rgb}{0.678431,1.000000,0.184314}%
\pgfsetstrokecolor{currentstroke}%
\pgfsetstrokeopacity{0.500000}%
\pgfsetdash{}{0pt}%
\pgfpathmoveto{\pgfqpoint{3.529247in}{1.168212in}}%
\pgfusepath{stroke}%
\end{pgfscope}%
\begin{pgfscope}%
\pgfpathrectangle{\pgfqpoint{0.100000in}{0.100000in}}{\pgfqpoint{5.307240in}{3.397500in}}%
\pgfusepath{clip}%
\pgfsetbuttcap%
\pgfsetroundjoin%
\definecolor{currentfill}{rgb}{0.678431,1.000000,0.184314}%
\pgfsetfillcolor{currentfill}%
\pgfsetfillopacity{0.500000}%
\pgfsetlinewidth{0.250937pt}%
\definecolor{currentstroke}{rgb}{0.000000,0.000000,0.000000}%
\pgfsetstrokecolor{currentstroke}%
\pgfsetstrokeopacity{0.500000}%
\pgfsetdash{}{0pt}%
\pgfsys@defobject{currentmarker}{\pgfqpoint{-0.065972in}{-0.065972in}}{\pgfqpoint{0.065972in}{0.065972in}}{%
\pgfpathmoveto{\pgfqpoint{0.000000in}{-0.065972in}}%
\pgfpathcurveto{\pgfqpoint{0.017496in}{-0.065972in}}{\pgfqpoint{0.034278in}{-0.059021in}}{\pgfqpoint{0.046649in}{-0.046649in}}%
\pgfpathcurveto{\pgfqpoint{0.059021in}{-0.034278in}}{\pgfqpoint{0.065972in}{-0.017496in}}{\pgfqpoint{0.065972in}{0.000000in}}%
\pgfpathcurveto{\pgfqpoint{0.065972in}{0.017496in}}{\pgfqpoint{0.059021in}{0.034278in}}{\pgfqpoint{0.046649in}{0.046649in}}%
\pgfpathcurveto{\pgfqpoint{0.034278in}{0.059021in}}{\pgfqpoint{0.017496in}{0.065972in}}{\pgfqpoint{0.000000in}{0.065972in}}%
\pgfpathcurveto{\pgfqpoint{-0.017496in}{0.065972in}}{\pgfqpoint{-0.034278in}{0.059021in}}{\pgfqpoint{-0.046649in}{0.046649in}}%
\pgfpathcurveto{\pgfqpoint{-0.059021in}{0.034278in}}{\pgfqpoint{-0.065972in}{0.017496in}}{\pgfqpoint{-0.065972in}{0.000000in}}%
\pgfpathcurveto{\pgfqpoint{-0.065972in}{-0.017496in}}{\pgfqpoint{-0.059021in}{-0.034278in}}{\pgfqpoint{-0.046649in}{-0.046649in}}%
\pgfpathcurveto{\pgfqpoint{-0.034278in}{-0.059021in}}{\pgfqpoint{-0.017496in}{-0.065972in}}{\pgfqpoint{0.000000in}{-0.065972in}}%
\pgfpathclose%
\pgfusepath{stroke,fill}%
}%
\begin{pgfscope}%
\pgfsys@transformshift{3.529247in}{1.168212in}%
\pgfsys@useobject{currentmarker}{}%
\end{pgfscope}%
\end{pgfscope}%
\begin{pgfscope}%
\pgfpathrectangle{\pgfqpoint{0.100000in}{0.100000in}}{\pgfqpoint{5.307240in}{3.397500in}}%
\pgfusepath{clip}%
\pgfsetrectcap%
\pgfsetroundjoin%
\pgfsetlinewidth{1.505625pt}%
\definecolor{currentstroke}{rgb}{0.678431,1.000000,0.184314}%
\pgfsetstrokecolor{currentstroke}%
\pgfsetstrokeopacity{0.500000}%
\pgfsetdash{}{0pt}%
\pgfpathmoveto{\pgfqpoint{3.558348in}{1.741566in}}%
\pgfusepath{stroke}%
\end{pgfscope}%
\begin{pgfscope}%
\pgfpathrectangle{\pgfqpoint{0.100000in}{0.100000in}}{\pgfqpoint{5.307240in}{3.397500in}}%
\pgfusepath{clip}%
\pgfsetbuttcap%
\pgfsetroundjoin%
\definecolor{currentfill}{rgb}{0.678431,1.000000,0.184314}%
\pgfsetfillcolor{currentfill}%
\pgfsetfillopacity{0.500000}%
\pgfsetlinewidth{0.250937pt}%
\definecolor{currentstroke}{rgb}{0.000000,0.000000,0.000000}%
\pgfsetstrokecolor{currentstroke}%
\pgfsetstrokeopacity{0.500000}%
\pgfsetdash{}{0pt}%
\pgfsys@defobject{currentmarker}{\pgfqpoint{-0.050000in}{-0.050000in}}{\pgfqpoint{0.050000in}{0.050000in}}{%
\pgfpathmoveto{\pgfqpoint{0.000000in}{-0.050000in}}%
\pgfpathcurveto{\pgfqpoint{0.013260in}{-0.050000in}}{\pgfqpoint{0.025979in}{-0.044732in}}{\pgfqpoint{0.035355in}{-0.035355in}}%
\pgfpathcurveto{\pgfqpoint{0.044732in}{-0.025979in}}{\pgfqpoint{0.050000in}{-0.013260in}}{\pgfqpoint{0.050000in}{0.000000in}}%
\pgfpathcurveto{\pgfqpoint{0.050000in}{0.013260in}}{\pgfqpoint{0.044732in}{0.025979in}}{\pgfqpoint{0.035355in}{0.035355in}}%
\pgfpathcurveto{\pgfqpoint{0.025979in}{0.044732in}}{\pgfqpoint{0.013260in}{0.050000in}}{\pgfqpoint{0.000000in}{0.050000in}}%
\pgfpathcurveto{\pgfqpoint{-0.013260in}{0.050000in}}{\pgfqpoint{-0.025979in}{0.044732in}}{\pgfqpoint{-0.035355in}{0.035355in}}%
\pgfpathcurveto{\pgfqpoint{-0.044732in}{0.025979in}}{\pgfqpoint{-0.050000in}{0.013260in}}{\pgfqpoint{-0.050000in}{0.000000in}}%
\pgfpathcurveto{\pgfqpoint{-0.050000in}{-0.013260in}}{\pgfqpoint{-0.044732in}{-0.025979in}}{\pgfqpoint{-0.035355in}{-0.035355in}}%
\pgfpathcurveto{\pgfqpoint{-0.025979in}{-0.044732in}}{\pgfqpoint{-0.013260in}{-0.050000in}}{\pgfqpoint{0.000000in}{-0.050000in}}%
\pgfpathclose%
\pgfusepath{stroke,fill}%
}%
\begin{pgfscope}%
\pgfsys@transformshift{3.558348in}{1.741566in}%
\pgfsys@useobject{currentmarker}{}%
\end{pgfscope}%
\end{pgfscope}%
\begin{pgfscope}%
\pgfpathrectangle{\pgfqpoint{0.100000in}{0.100000in}}{\pgfqpoint{5.307240in}{3.397500in}}%
\pgfusepath{clip}%
\pgfsetrectcap%
\pgfsetroundjoin%
\pgfsetlinewidth{1.505625pt}%
\definecolor{currentstroke}{rgb}{0.678431,1.000000,0.184314}%
\pgfsetstrokecolor{currentstroke}%
\pgfsetstrokeopacity{0.500000}%
\pgfsetdash{}{0pt}%
\pgfpathmoveto{\pgfqpoint{3.292535in}{1.921191in}}%
\pgfusepath{stroke}%
\end{pgfscope}%
\begin{pgfscope}%
\pgfpathrectangle{\pgfqpoint{0.100000in}{0.100000in}}{\pgfqpoint{5.307240in}{3.397500in}}%
\pgfusepath{clip}%
\pgfsetbuttcap%
\pgfsetroundjoin%
\definecolor{currentfill}{rgb}{0.678431,1.000000,0.184314}%
\pgfsetfillcolor{currentfill}%
\pgfsetfillopacity{0.500000}%
\pgfsetlinewidth{0.250937pt}%
\definecolor{currentstroke}{rgb}{0.000000,0.000000,0.000000}%
\pgfsetstrokecolor{currentstroke}%
\pgfsetstrokeopacity{0.500000}%
\pgfsetdash{}{0pt}%
\pgfsys@defobject{currentmarker}{\pgfqpoint{-0.031944in}{-0.031944in}}{\pgfqpoint{0.031944in}{0.031944in}}{%
\pgfpathmoveto{\pgfqpoint{0.000000in}{-0.031944in}}%
\pgfpathcurveto{\pgfqpoint{0.008472in}{-0.031944in}}{\pgfqpoint{0.016598in}{-0.028579in}}{\pgfqpoint{0.022588in}{-0.022588in}}%
\pgfpathcurveto{\pgfqpoint{0.028579in}{-0.016598in}}{\pgfqpoint{0.031944in}{-0.008472in}}{\pgfqpoint{0.031944in}{0.000000in}}%
\pgfpathcurveto{\pgfqpoint{0.031944in}{0.008472in}}{\pgfqpoint{0.028579in}{0.016598in}}{\pgfqpoint{0.022588in}{0.022588in}}%
\pgfpathcurveto{\pgfqpoint{0.016598in}{0.028579in}}{\pgfqpoint{0.008472in}{0.031944in}}{\pgfqpoint{0.000000in}{0.031944in}}%
\pgfpathcurveto{\pgfqpoint{-0.008472in}{0.031944in}}{\pgfqpoint{-0.016598in}{0.028579in}}{\pgfqpoint{-0.022588in}{0.022588in}}%
\pgfpathcurveto{\pgfqpoint{-0.028579in}{0.016598in}}{\pgfqpoint{-0.031944in}{0.008472in}}{\pgfqpoint{-0.031944in}{0.000000in}}%
\pgfpathcurveto{\pgfqpoint{-0.031944in}{-0.008472in}}{\pgfqpoint{-0.028579in}{-0.016598in}}{\pgfqpoint{-0.022588in}{-0.022588in}}%
\pgfpathcurveto{\pgfqpoint{-0.016598in}{-0.028579in}}{\pgfqpoint{-0.008472in}{-0.031944in}}{\pgfqpoint{0.000000in}{-0.031944in}}%
\pgfpathclose%
\pgfusepath{stroke,fill}%
}%
\begin{pgfscope}%
\pgfsys@transformshift{3.292535in}{1.921191in}%
\pgfsys@useobject{currentmarker}{}%
\end{pgfscope}%
\end{pgfscope}%
\begin{pgfscope}%
\pgfpathrectangle{\pgfqpoint{0.100000in}{0.100000in}}{\pgfqpoint{5.307240in}{3.397500in}}%
\pgfusepath{clip}%
\pgfsetrectcap%
\pgfsetroundjoin%
\pgfsetlinewidth{1.505625pt}%
\definecolor{currentstroke}{rgb}{0.678431,1.000000,0.184314}%
\pgfsetstrokecolor{currentstroke}%
\pgfsetstrokeopacity{0.500000}%
\pgfsetdash{}{0pt}%
\pgfpathmoveto{\pgfqpoint{3.308466in}{1.878114in}}%
\pgfusepath{stroke}%
\end{pgfscope}%
\begin{pgfscope}%
\pgfpathrectangle{\pgfqpoint{0.100000in}{0.100000in}}{\pgfqpoint{5.307240in}{3.397500in}}%
\pgfusepath{clip}%
\pgfsetbuttcap%
\pgfsetroundjoin%
\definecolor{currentfill}{rgb}{0.678431,1.000000,0.184314}%
\pgfsetfillcolor{currentfill}%
\pgfsetfillopacity{0.500000}%
\pgfsetlinewidth{0.250937pt}%
\definecolor{currentstroke}{rgb}{0.000000,0.000000,0.000000}%
\pgfsetstrokecolor{currentstroke}%
\pgfsetstrokeopacity{0.500000}%
\pgfsetdash{}{0pt}%
\pgfsys@defobject{currentmarker}{\pgfqpoint{-0.030556in}{-0.030556in}}{\pgfqpoint{0.030556in}{0.030556in}}{%
\pgfpathmoveto{\pgfqpoint{0.000000in}{-0.030556in}}%
\pgfpathcurveto{\pgfqpoint{0.008103in}{-0.030556in}}{\pgfqpoint{0.015876in}{-0.027336in}}{\pgfqpoint{0.021606in}{-0.021606in}}%
\pgfpathcurveto{\pgfqpoint{0.027336in}{-0.015876in}}{\pgfqpoint{0.030556in}{-0.008103in}}{\pgfqpoint{0.030556in}{0.000000in}}%
\pgfpathcurveto{\pgfqpoint{0.030556in}{0.008103in}}{\pgfqpoint{0.027336in}{0.015876in}}{\pgfqpoint{0.021606in}{0.021606in}}%
\pgfpathcurveto{\pgfqpoint{0.015876in}{0.027336in}}{\pgfqpoint{0.008103in}{0.030556in}}{\pgfqpoint{0.000000in}{0.030556in}}%
\pgfpathcurveto{\pgfqpoint{-0.008103in}{0.030556in}}{\pgfqpoint{-0.015876in}{0.027336in}}{\pgfqpoint{-0.021606in}{0.021606in}}%
\pgfpathcurveto{\pgfqpoint{-0.027336in}{0.015876in}}{\pgfqpoint{-0.030556in}{0.008103in}}{\pgfqpoint{-0.030556in}{0.000000in}}%
\pgfpathcurveto{\pgfqpoint{-0.030556in}{-0.008103in}}{\pgfqpoint{-0.027336in}{-0.015876in}}{\pgfqpoint{-0.021606in}{-0.021606in}}%
\pgfpathcurveto{\pgfqpoint{-0.015876in}{-0.027336in}}{\pgfqpoint{-0.008103in}{-0.030556in}}{\pgfqpoint{0.000000in}{-0.030556in}}%
\pgfpathclose%
\pgfusepath{stroke,fill}%
}%
\begin{pgfscope}%
\pgfsys@transformshift{3.308466in}{1.878114in}%
\pgfsys@useobject{currentmarker}{}%
\end{pgfscope}%
\end{pgfscope}%
\begin{pgfscope}%
\pgfpathrectangle{\pgfqpoint{0.100000in}{0.100000in}}{\pgfqpoint{5.307240in}{3.397500in}}%
\pgfusepath{clip}%
\pgfsetrectcap%
\pgfsetroundjoin%
\pgfsetlinewidth{1.505625pt}%
\definecolor{currentstroke}{rgb}{0.678431,1.000000,0.184314}%
\pgfsetstrokecolor{currentstroke}%
\pgfsetstrokeopacity{0.500000}%
\pgfsetdash{}{0pt}%
\pgfpathmoveto{\pgfqpoint{3.096692in}{1.700615in}}%
\pgfusepath{stroke}%
\end{pgfscope}%
\begin{pgfscope}%
\pgfpathrectangle{\pgfqpoint{0.100000in}{0.100000in}}{\pgfqpoint{5.307240in}{3.397500in}}%
\pgfusepath{clip}%
\pgfsetbuttcap%
\pgfsetroundjoin%
\definecolor{currentfill}{rgb}{0.678431,1.000000,0.184314}%
\pgfsetfillcolor{currentfill}%
\pgfsetfillopacity{0.500000}%
\pgfsetlinewidth{0.250937pt}%
\definecolor{currentstroke}{rgb}{0.000000,0.000000,0.000000}%
\pgfsetstrokecolor{currentstroke}%
\pgfsetstrokeopacity{0.500000}%
\pgfsetdash{}{0pt}%
\pgfsys@defobject{currentmarker}{\pgfqpoint{-0.052778in}{-0.052778in}}{\pgfqpoint{0.052778in}{0.052778in}}{%
\pgfpathmoveto{\pgfqpoint{0.000000in}{-0.052778in}}%
\pgfpathcurveto{\pgfqpoint{0.013997in}{-0.052778in}}{\pgfqpoint{0.027422in}{-0.047217in}}{\pgfqpoint{0.037320in}{-0.037320in}}%
\pgfpathcurveto{\pgfqpoint{0.047217in}{-0.027422in}}{\pgfqpoint{0.052778in}{-0.013997in}}{\pgfqpoint{0.052778in}{0.000000in}}%
\pgfpathcurveto{\pgfqpoint{0.052778in}{0.013997in}}{\pgfqpoint{0.047217in}{0.027422in}}{\pgfqpoint{0.037320in}{0.037320in}}%
\pgfpathcurveto{\pgfqpoint{0.027422in}{0.047217in}}{\pgfqpoint{0.013997in}{0.052778in}}{\pgfqpoint{0.000000in}{0.052778in}}%
\pgfpathcurveto{\pgfqpoint{-0.013997in}{0.052778in}}{\pgfqpoint{-0.027422in}{0.047217in}}{\pgfqpoint{-0.037320in}{0.037320in}}%
\pgfpathcurveto{\pgfqpoint{-0.047217in}{0.027422in}}{\pgfqpoint{-0.052778in}{0.013997in}}{\pgfqpoint{-0.052778in}{0.000000in}}%
\pgfpathcurveto{\pgfqpoint{-0.052778in}{-0.013997in}}{\pgfqpoint{-0.047217in}{-0.027422in}}{\pgfqpoint{-0.037320in}{-0.037320in}}%
\pgfpathcurveto{\pgfqpoint{-0.027422in}{-0.047217in}}{\pgfqpoint{-0.013997in}{-0.052778in}}{\pgfqpoint{0.000000in}{-0.052778in}}%
\pgfpathclose%
\pgfusepath{stroke,fill}%
}%
\begin{pgfscope}%
\pgfsys@transformshift{3.096692in}{1.700615in}%
\pgfsys@useobject{currentmarker}{}%
\end{pgfscope}%
\end{pgfscope}%
\begin{pgfscope}%
\pgfpathrectangle{\pgfqpoint{0.100000in}{0.100000in}}{\pgfqpoint{5.307240in}{3.397500in}}%
\pgfusepath{clip}%
\pgfsetrectcap%
\pgfsetroundjoin%
\pgfsetlinewidth{1.505625pt}%
\definecolor{currentstroke}{rgb}{0.678431,1.000000,0.184314}%
\pgfsetstrokecolor{currentstroke}%
\pgfsetstrokeopacity{0.500000}%
\pgfsetdash{}{0pt}%
\pgfpathmoveto{\pgfqpoint{3.090957in}{1.933163in}}%
\pgfusepath{stroke}%
\end{pgfscope}%
\begin{pgfscope}%
\pgfpathrectangle{\pgfqpoint{0.100000in}{0.100000in}}{\pgfqpoint{5.307240in}{3.397500in}}%
\pgfusepath{clip}%
\pgfsetbuttcap%
\pgfsetroundjoin%
\definecolor{currentfill}{rgb}{0.678431,1.000000,0.184314}%
\pgfsetfillcolor{currentfill}%
\pgfsetfillopacity{0.500000}%
\pgfsetlinewidth{0.250937pt}%
\definecolor{currentstroke}{rgb}{0.000000,0.000000,0.000000}%
\pgfsetstrokecolor{currentstroke}%
\pgfsetstrokeopacity{0.500000}%
\pgfsetdash{}{0pt}%
\pgfsys@defobject{currentmarker}{\pgfqpoint{-0.058333in}{-0.058333in}}{\pgfqpoint{0.058333in}{0.058333in}}{%
\pgfpathmoveto{\pgfqpoint{0.000000in}{-0.058333in}}%
\pgfpathcurveto{\pgfqpoint{0.015470in}{-0.058333in}}{\pgfqpoint{0.030309in}{-0.052187in}}{\pgfqpoint{0.041248in}{-0.041248in}}%
\pgfpathcurveto{\pgfqpoint{0.052187in}{-0.030309in}}{\pgfqpoint{0.058333in}{-0.015470in}}{\pgfqpoint{0.058333in}{0.000000in}}%
\pgfpathcurveto{\pgfqpoint{0.058333in}{0.015470in}}{\pgfqpoint{0.052187in}{0.030309in}}{\pgfqpoint{0.041248in}{0.041248in}}%
\pgfpathcurveto{\pgfqpoint{0.030309in}{0.052187in}}{\pgfqpoint{0.015470in}{0.058333in}}{\pgfqpoint{0.000000in}{0.058333in}}%
\pgfpathcurveto{\pgfqpoint{-0.015470in}{0.058333in}}{\pgfqpoint{-0.030309in}{0.052187in}}{\pgfqpoint{-0.041248in}{0.041248in}}%
\pgfpathcurveto{\pgfqpoint{-0.052187in}{0.030309in}}{\pgfqpoint{-0.058333in}{0.015470in}}{\pgfqpoint{-0.058333in}{0.000000in}}%
\pgfpathcurveto{\pgfqpoint{-0.058333in}{-0.015470in}}{\pgfqpoint{-0.052187in}{-0.030309in}}{\pgfqpoint{-0.041248in}{-0.041248in}}%
\pgfpathcurveto{\pgfqpoint{-0.030309in}{-0.052187in}}{\pgfqpoint{-0.015470in}{-0.058333in}}{\pgfqpoint{0.000000in}{-0.058333in}}%
\pgfpathclose%
\pgfusepath{stroke,fill}%
}%
\begin{pgfscope}%
\pgfsys@transformshift{3.090957in}{1.933163in}%
\pgfsys@useobject{currentmarker}{}%
\end{pgfscope}%
\end{pgfscope}%
\begin{pgfscope}%
\pgfpathrectangle{\pgfqpoint{0.100000in}{0.100000in}}{\pgfqpoint{5.307240in}{3.397500in}}%
\pgfusepath{clip}%
\pgfsetrectcap%
\pgfsetroundjoin%
\pgfsetlinewidth{1.505625pt}%
\definecolor{currentstroke}{rgb}{0.678431,1.000000,0.184314}%
\pgfsetstrokecolor{currentstroke}%
\pgfsetstrokeopacity{0.500000}%
\pgfsetdash{}{0pt}%
\pgfpathmoveto{\pgfqpoint{3.065232in}{2.012612in}}%
\pgfusepath{stroke}%
\end{pgfscope}%
\begin{pgfscope}%
\pgfpathrectangle{\pgfqpoint{0.100000in}{0.100000in}}{\pgfqpoint{5.307240in}{3.397500in}}%
\pgfusepath{clip}%
\pgfsetbuttcap%
\pgfsetroundjoin%
\definecolor{currentfill}{rgb}{0.678431,1.000000,0.184314}%
\pgfsetfillcolor{currentfill}%
\pgfsetfillopacity{0.500000}%
\pgfsetlinewidth{0.250937pt}%
\definecolor{currentstroke}{rgb}{0.000000,0.000000,0.000000}%
\pgfsetstrokecolor{currentstroke}%
\pgfsetstrokeopacity{0.500000}%
\pgfsetdash{}{0pt}%
\pgfsys@defobject{currentmarker}{\pgfqpoint{-0.030556in}{-0.030556in}}{\pgfqpoint{0.030556in}{0.030556in}}{%
\pgfpathmoveto{\pgfqpoint{0.000000in}{-0.030556in}}%
\pgfpathcurveto{\pgfqpoint{0.008103in}{-0.030556in}}{\pgfqpoint{0.015876in}{-0.027336in}}{\pgfqpoint{0.021606in}{-0.021606in}}%
\pgfpathcurveto{\pgfqpoint{0.027336in}{-0.015876in}}{\pgfqpoint{0.030556in}{-0.008103in}}{\pgfqpoint{0.030556in}{0.000000in}}%
\pgfpathcurveto{\pgfqpoint{0.030556in}{0.008103in}}{\pgfqpoint{0.027336in}{0.015876in}}{\pgfqpoint{0.021606in}{0.021606in}}%
\pgfpathcurveto{\pgfqpoint{0.015876in}{0.027336in}}{\pgfqpoint{0.008103in}{0.030556in}}{\pgfqpoint{0.000000in}{0.030556in}}%
\pgfpathcurveto{\pgfqpoint{-0.008103in}{0.030556in}}{\pgfqpoint{-0.015876in}{0.027336in}}{\pgfqpoint{-0.021606in}{0.021606in}}%
\pgfpathcurveto{\pgfqpoint{-0.027336in}{0.015876in}}{\pgfqpoint{-0.030556in}{0.008103in}}{\pgfqpoint{-0.030556in}{0.000000in}}%
\pgfpathcurveto{\pgfqpoint{-0.030556in}{-0.008103in}}{\pgfqpoint{-0.027336in}{-0.015876in}}{\pgfqpoint{-0.021606in}{-0.021606in}}%
\pgfpathcurveto{\pgfqpoint{-0.015876in}{-0.027336in}}{\pgfqpoint{-0.008103in}{-0.030556in}}{\pgfqpoint{0.000000in}{-0.030556in}}%
\pgfpathclose%
\pgfusepath{stroke,fill}%
}%
\begin{pgfscope}%
\pgfsys@transformshift{3.065232in}{2.012612in}%
\pgfsys@useobject{currentmarker}{}%
\end{pgfscope}%
\end{pgfscope}%
\begin{pgfscope}%
\pgfpathrectangle{\pgfqpoint{0.100000in}{0.100000in}}{\pgfqpoint{5.307240in}{3.397500in}}%
\pgfusepath{clip}%
\pgfsetrectcap%
\pgfsetroundjoin%
\pgfsetlinewidth{1.505625pt}%
\definecolor{currentstroke}{rgb}{0.678431,1.000000,0.184314}%
\pgfsetstrokecolor{currentstroke}%
\pgfsetstrokeopacity{0.500000}%
\pgfsetdash{}{0pt}%
\pgfpathmoveto{\pgfqpoint{3.487451in}{1.891439in}}%
\pgfusepath{stroke}%
\end{pgfscope}%
\begin{pgfscope}%
\pgfpathrectangle{\pgfqpoint{0.100000in}{0.100000in}}{\pgfqpoint{5.307240in}{3.397500in}}%
\pgfusepath{clip}%
\pgfsetbuttcap%
\pgfsetroundjoin%
\definecolor{currentfill}{rgb}{0.678431,1.000000,0.184314}%
\pgfsetfillcolor{currentfill}%
\pgfsetfillopacity{0.500000}%
\pgfsetlinewidth{0.250937pt}%
\definecolor{currentstroke}{rgb}{0.000000,0.000000,0.000000}%
\pgfsetstrokecolor{currentstroke}%
\pgfsetstrokeopacity{0.500000}%
\pgfsetdash{}{0pt}%
\pgfsys@defobject{currentmarker}{\pgfqpoint{-0.056944in}{-0.056944in}}{\pgfqpoint{0.056944in}{0.056944in}}{%
\pgfpathmoveto{\pgfqpoint{0.000000in}{-0.056944in}}%
\pgfpathcurveto{\pgfqpoint{0.015102in}{-0.056944in}}{\pgfqpoint{0.029587in}{-0.050944in}}{\pgfqpoint{0.040266in}{-0.040266in}}%
\pgfpathcurveto{\pgfqpoint{0.050944in}{-0.029587in}}{\pgfqpoint{0.056944in}{-0.015102in}}{\pgfqpoint{0.056944in}{0.000000in}}%
\pgfpathcurveto{\pgfqpoint{0.056944in}{0.015102in}}{\pgfqpoint{0.050944in}{0.029587in}}{\pgfqpoint{0.040266in}{0.040266in}}%
\pgfpathcurveto{\pgfqpoint{0.029587in}{0.050944in}}{\pgfqpoint{0.015102in}{0.056944in}}{\pgfqpoint{0.000000in}{0.056944in}}%
\pgfpathcurveto{\pgfqpoint{-0.015102in}{0.056944in}}{\pgfqpoint{-0.029587in}{0.050944in}}{\pgfqpoint{-0.040266in}{0.040266in}}%
\pgfpathcurveto{\pgfqpoint{-0.050944in}{0.029587in}}{\pgfqpoint{-0.056944in}{0.015102in}}{\pgfqpoint{-0.056944in}{0.000000in}}%
\pgfpathcurveto{\pgfqpoint{-0.056944in}{-0.015102in}}{\pgfqpoint{-0.050944in}{-0.029587in}}{\pgfqpoint{-0.040266in}{-0.040266in}}%
\pgfpathcurveto{\pgfqpoint{-0.029587in}{-0.050944in}}{\pgfqpoint{-0.015102in}{-0.056944in}}{\pgfqpoint{0.000000in}{-0.056944in}}%
\pgfpathclose%
\pgfusepath{stroke,fill}%
}%
\begin{pgfscope}%
\pgfsys@transformshift{3.487451in}{1.891439in}%
\pgfsys@useobject{currentmarker}{}%
\end{pgfscope}%
\end{pgfscope}%
\begin{pgfscope}%
\pgfpathrectangle{\pgfqpoint{0.100000in}{0.100000in}}{\pgfqpoint{5.307240in}{3.397500in}}%
\pgfusepath{clip}%
\pgfsetrectcap%
\pgfsetroundjoin%
\pgfsetlinewidth{1.505625pt}%
\definecolor{currentstroke}{rgb}{0.678431,1.000000,0.184314}%
\pgfsetstrokecolor{currentstroke}%
\pgfsetstrokeopacity{0.500000}%
\pgfsetdash{}{0pt}%
\pgfpathmoveto{\pgfqpoint{3.209145in}{1.717219in}}%
\pgfusepath{stroke}%
\end{pgfscope}%
\begin{pgfscope}%
\pgfpathrectangle{\pgfqpoint{0.100000in}{0.100000in}}{\pgfqpoint{5.307240in}{3.397500in}}%
\pgfusepath{clip}%
\pgfsetbuttcap%
\pgfsetroundjoin%
\definecolor{currentfill}{rgb}{0.678431,1.000000,0.184314}%
\pgfsetfillcolor{currentfill}%
\pgfsetfillopacity{0.500000}%
\pgfsetlinewidth{0.250937pt}%
\definecolor{currentstroke}{rgb}{0.000000,0.000000,0.000000}%
\pgfsetstrokecolor{currentstroke}%
\pgfsetstrokeopacity{0.500000}%
\pgfsetdash{}{0pt}%
\pgfsys@defobject{currentmarker}{\pgfqpoint{-0.047222in}{-0.047222in}}{\pgfqpoint{0.047222in}{0.047222in}}{%
\pgfpathmoveto{\pgfqpoint{0.000000in}{-0.047222in}}%
\pgfpathcurveto{\pgfqpoint{0.012523in}{-0.047222in}}{\pgfqpoint{0.024536in}{-0.042247in}}{\pgfqpoint{0.033391in}{-0.033391in}}%
\pgfpathcurveto{\pgfqpoint{0.042247in}{-0.024536in}}{\pgfqpoint{0.047222in}{-0.012523in}}{\pgfqpoint{0.047222in}{0.000000in}}%
\pgfpathcurveto{\pgfqpoint{0.047222in}{0.012523in}}{\pgfqpoint{0.042247in}{0.024536in}}{\pgfqpoint{0.033391in}{0.033391in}}%
\pgfpathcurveto{\pgfqpoint{0.024536in}{0.042247in}}{\pgfqpoint{0.012523in}{0.047222in}}{\pgfqpoint{0.000000in}{0.047222in}}%
\pgfpathcurveto{\pgfqpoint{-0.012523in}{0.047222in}}{\pgfqpoint{-0.024536in}{0.042247in}}{\pgfqpoint{-0.033391in}{0.033391in}}%
\pgfpathcurveto{\pgfqpoint{-0.042247in}{0.024536in}}{\pgfqpoint{-0.047222in}{0.012523in}}{\pgfqpoint{-0.047222in}{0.000000in}}%
\pgfpathcurveto{\pgfqpoint{-0.047222in}{-0.012523in}}{\pgfqpoint{-0.042247in}{-0.024536in}}{\pgfqpoint{-0.033391in}{-0.033391in}}%
\pgfpathcurveto{\pgfqpoint{-0.024536in}{-0.042247in}}{\pgfqpoint{-0.012523in}{-0.047222in}}{\pgfqpoint{0.000000in}{-0.047222in}}%
\pgfpathclose%
\pgfusepath{stroke,fill}%
}%
\begin{pgfscope}%
\pgfsys@transformshift{3.209145in}{1.717219in}%
\pgfsys@useobject{currentmarker}{}%
\end{pgfscope}%
\end{pgfscope}%
\begin{pgfscope}%
\pgfpathrectangle{\pgfqpoint{0.100000in}{0.100000in}}{\pgfqpoint{5.307240in}{3.397500in}}%
\pgfusepath{clip}%
\pgfsetrectcap%
\pgfsetroundjoin%
\pgfsetlinewidth{1.505625pt}%
\definecolor{currentstroke}{rgb}{0.678431,1.000000,0.184314}%
\pgfsetstrokecolor{currentstroke}%
\pgfsetstrokeopacity{0.500000}%
\pgfsetdash{}{0pt}%
\pgfpathmoveto{\pgfqpoint{1.953749in}{2.795786in}}%
\pgfusepath{stroke}%
\end{pgfscope}%
\begin{pgfscope}%
\pgfpathrectangle{\pgfqpoint{0.100000in}{0.100000in}}{\pgfqpoint{5.307240in}{3.397500in}}%
\pgfusepath{clip}%
\pgfsetbuttcap%
\pgfsetroundjoin%
\definecolor{currentfill}{rgb}{0.678431,1.000000,0.184314}%
\pgfsetfillcolor{currentfill}%
\pgfsetfillopacity{0.500000}%
\pgfsetlinewidth{0.250937pt}%
\definecolor{currentstroke}{rgb}{0.000000,0.000000,0.000000}%
\pgfsetstrokecolor{currentstroke}%
\pgfsetstrokeopacity{0.500000}%
\pgfsetdash{}{0pt}%
\pgfsys@defobject{currentmarker}{\pgfqpoint{-0.056944in}{-0.056944in}}{\pgfqpoint{0.056944in}{0.056944in}}{%
\pgfpathmoveto{\pgfqpoint{0.000000in}{-0.056944in}}%
\pgfpathcurveto{\pgfqpoint{0.015102in}{-0.056944in}}{\pgfqpoint{0.029587in}{-0.050944in}}{\pgfqpoint{0.040266in}{-0.040266in}}%
\pgfpathcurveto{\pgfqpoint{0.050944in}{-0.029587in}}{\pgfqpoint{0.056944in}{-0.015102in}}{\pgfqpoint{0.056944in}{0.000000in}}%
\pgfpathcurveto{\pgfqpoint{0.056944in}{0.015102in}}{\pgfqpoint{0.050944in}{0.029587in}}{\pgfqpoint{0.040266in}{0.040266in}}%
\pgfpathcurveto{\pgfqpoint{0.029587in}{0.050944in}}{\pgfqpoint{0.015102in}{0.056944in}}{\pgfqpoint{0.000000in}{0.056944in}}%
\pgfpathcurveto{\pgfqpoint{-0.015102in}{0.056944in}}{\pgfqpoint{-0.029587in}{0.050944in}}{\pgfqpoint{-0.040266in}{0.040266in}}%
\pgfpathcurveto{\pgfqpoint{-0.050944in}{0.029587in}}{\pgfqpoint{-0.056944in}{0.015102in}}{\pgfqpoint{-0.056944in}{0.000000in}}%
\pgfpathcurveto{\pgfqpoint{-0.056944in}{-0.015102in}}{\pgfqpoint{-0.050944in}{-0.029587in}}{\pgfqpoint{-0.040266in}{-0.040266in}}%
\pgfpathcurveto{\pgfqpoint{-0.029587in}{-0.050944in}}{\pgfqpoint{-0.015102in}{-0.056944in}}{\pgfqpoint{0.000000in}{-0.056944in}}%
\pgfpathclose%
\pgfusepath{stroke,fill}%
}%
\begin{pgfscope}%
\pgfsys@transformshift{1.953749in}{2.795786in}%
\pgfsys@useobject{currentmarker}{}%
\end{pgfscope}%
\end{pgfscope}%
\begin{pgfscope}%
\pgfpathrectangle{\pgfqpoint{0.100000in}{0.100000in}}{\pgfqpoint{5.307240in}{3.397500in}}%
\pgfusepath{clip}%
\pgfsetrectcap%
\pgfsetroundjoin%
\pgfsetlinewidth{1.505625pt}%
\definecolor{currentstroke}{rgb}{0.678431,1.000000,0.184314}%
\pgfsetstrokecolor{currentstroke}%
\pgfsetstrokeopacity{0.500000}%
\pgfsetdash{}{0pt}%
\pgfpathmoveto{\pgfqpoint{1.764491in}{3.031271in}}%
\pgfusepath{stroke}%
\end{pgfscope}%
\begin{pgfscope}%
\pgfpathrectangle{\pgfqpoint{0.100000in}{0.100000in}}{\pgfqpoint{5.307240in}{3.397500in}}%
\pgfusepath{clip}%
\pgfsetbuttcap%
\pgfsetroundjoin%
\definecolor{currentfill}{rgb}{0.678431,1.000000,0.184314}%
\pgfsetfillcolor{currentfill}%
\pgfsetfillopacity{0.500000}%
\pgfsetlinewidth{0.250937pt}%
\definecolor{currentstroke}{rgb}{0.000000,0.000000,0.000000}%
\pgfsetstrokecolor{currentstroke}%
\pgfsetstrokeopacity{0.500000}%
\pgfsetdash{}{0pt}%
\pgfsys@defobject{currentmarker}{\pgfqpoint{-0.062500in}{-0.062500in}}{\pgfqpoint{0.062500in}{0.062500in}}{%
\pgfpathmoveto{\pgfqpoint{0.000000in}{-0.062500in}}%
\pgfpathcurveto{\pgfqpoint{0.016575in}{-0.062500in}}{\pgfqpoint{0.032474in}{-0.055915in}}{\pgfqpoint{0.044194in}{-0.044194in}}%
\pgfpathcurveto{\pgfqpoint{0.055915in}{-0.032474in}}{\pgfqpoint{0.062500in}{-0.016575in}}{\pgfqpoint{0.062500in}{0.000000in}}%
\pgfpathcurveto{\pgfqpoint{0.062500in}{0.016575in}}{\pgfqpoint{0.055915in}{0.032474in}}{\pgfqpoint{0.044194in}{0.044194in}}%
\pgfpathcurveto{\pgfqpoint{0.032474in}{0.055915in}}{\pgfqpoint{0.016575in}{0.062500in}}{\pgfqpoint{0.000000in}{0.062500in}}%
\pgfpathcurveto{\pgfqpoint{-0.016575in}{0.062500in}}{\pgfqpoint{-0.032474in}{0.055915in}}{\pgfqpoint{-0.044194in}{0.044194in}}%
\pgfpathcurveto{\pgfqpoint{-0.055915in}{0.032474in}}{\pgfqpoint{-0.062500in}{0.016575in}}{\pgfqpoint{-0.062500in}{0.000000in}}%
\pgfpathcurveto{\pgfqpoint{-0.062500in}{-0.016575in}}{\pgfqpoint{-0.055915in}{-0.032474in}}{\pgfqpoint{-0.044194in}{-0.044194in}}%
\pgfpathcurveto{\pgfqpoint{-0.032474in}{-0.055915in}}{\pgfqpoint{-0.016575in}{-0.062500in}}{\pgfqpoint{0.000000in}{-0.062500in}}%
\pgfpathclose%
\pgfusepath{stroke,fill}%
}%
\begin{pgfscope}%
\pgfsys@transformshift{1.764491in}{3.031271in}%
\pgfsys@useobject{currentmarker}{}%
\end{pgfscope}%
\end{pgfscope}%
\begin{pgfscope}%
\pgfpathrectangle{\pgfqpoint{0.100000in}{0.100000in}}{\pgfqpoint{5.307240in}{3.397500in}}%
\pgfusepath{clip}%
\pgfsetrectcap%
\pgfsetroundjoin%
\pgfsetlinewidth{1.505625pt}%
\definecolor{currentstroke}{rgb}{0.678431,1.000000,0.184314}%
\pgfsetstrokecolor{currentstroke}%
\pgfsetstrokeopacity{0.500000}%
\pgfsetdash{}{0pt}%
\pgfpathmoveto{\pgfqpoint{1.538347in}{2.999746in}}%
\pgfusepath{stroke}%
\end{pgfscope}%
\begin{pgfscope}%
\pgfpathrectangle{\pgfqpoint{0.100000in}{0.100000in}}{\pgfqpoint{5.307240in}{3.397500in}}%
\pgfusepath{clip}%
\pgfsetbuttcap%
\pgfsetroundjoin%
\definecolor{currentfill}{rgb}{0.678431,1.000000,0.184314}%
\pgfsetfillcolor{currentfill}%
\pgfsetfillopacity{0.500000}%
\pgfsetlinewidth{0.250937pt}%
\definecolor{currentstroke}{rgb}{0.000000,0.000000,0.000000}%
\pgfsetstrokecolor{currentstroke}%
\pgfsetstrokeopacity{0.500000}%
\pgfsetdash{}{0pt}%
\pgfsys@defobject{currentmarker}{\pgfqpoint{-0.069444in}{-0.069444in}}{\pgfqpoint{0.069444in}{0.069444in}}{%
\pgfpathmoveto{\pgfqpoint{0.000000in}{-0.069444in}}%
\pgfpathcurveto{\pgfqpoint{0.018417in}{-0.069444in}}{\pgfqpoint{0.036082in}{-0.062127in}}{\pgfqpoint{0.049105in}{-0.049105in}}%
\pgfpathcurveto{\pgfqpoint{0.062127in}{-0.036082in}}{\pgfqpoint{0.069444in}{-0.018417in}}{\pgfqpoint{0.069444in}{0.000000in}}%
\pgfpathcurveto{\pgfqpoint{0.069444in}{0.018417in}}{\pgfqpoint{0.062127in}{0.036082in}}{\pgfqpoint{0.049105in}{0.049105in}}%
\pgfpathcurveto{\pgfqpoint{0.036082in}{0.062127in}}{\pgfqpoint{0.018417in}{0.069444in}}{\pgfqpoint{0.000000in}{0.069444in}}%
\pgfpathcurveto{\pgfqpoint{-0.018417in}{0.069444in}}{\pgfqpoint{-0.036082in}{0.062127in}}{\pgfqpoint{-0.049105in}{0.049105in}}%
\pgfpathcurveto{\pgfqpoint{-0.062127in}{0.036082in}}{\pgfqpoint{-0.069444in}{0.018417in}}{\pgfqpoint{-0.069444in}{0.000000in}}%
\pgfpathcurveto{\pgfqpoint{-0.069444in}{-0.018417in}}{\pgfqpoint{-0.062127in}{-0.036082in}}{\pgfqpoint{-0.049105in}{-0.049105in}}%
\pgfpathcurveto{\pgfqpoint{-0.036082in}{-0.062127in}}{\pgfqpoint{-0.018417in}{-0.069444in}}{\pgfqpoint{0.000000in}{-0.069444in}}%
\pgfpathclose%
\pgfusepath{stroke,fill}%
}%
\begin{pgfscope}%
\pgfsys@transformshift{1.538347in}{2.999746in}%
\pgfsys@useobject{currentmarker}{}%
\end{pgfscope}%
\end{pgfscope}%
\begin{pgfscope}%
\pgfpathrectangle{\pgfqpoint{0.100000in}{0.100000in}}{\pgfqpoint{5.307240in}{3.397500in}}%
\pgfusepath{clip}%
\pgfsetrectcap%
\pgfsetroundjoin%
\pgfsetlinewidth{1.505625pt}%
\definecolor{currentstroke}{rgb}{0.678431,1.000000,0.184314}%
\pgfsetstrokecolor{currentstroke}%
\pgfsetstrokeopacity{0.500000}%
\pgfsetdash{}{0pt}%
\pgfpathmoveto{\pgfqpoint{2.758186in}{2.152371in}}%
\pgfusepath{stroke}%
\end{pgfscope}%
\begin{pgfscope}%
\pgfpathrectangle{\pgfqpoint{0.100000in}{0.100000in}}{\pgfqpoint{5.307240in}{3.397500in}}%
\pgfusepath{clip}%
\pgfsetbuttcap%
\pgfsetroundjoin%
\definecolor{currentfill}{rgb}{0.678431,1.000000,0.184314}%
\pgfsetfillcolor{currentfill}%
\pgfsetfillopacity{0.500000}%
\pgfsetlinewidth{0.250937pt}%
\definecolor{currentstroke}{rgb}{0.000000,0.000000,0.000000}%
\pgfsetstrokecolor{currentstroke}%
\pgfsetstrokeopacity{0.500000}%
\pgfsetdash{}{0pt}%
\pgfsys@defobject{currentmarker}{\pgfqpoint{-0.055556in}{-0.055556in}}{\pgfqpoint{0.055556in}{0.055556in}}{%
\pgfpathmoveto{\pgfqpoint{0.000000in}{-0.055556in}}%
\pgfpathcurveto{\pgfqpoint{0.014734in}{-0.055556in}}{\pgfqpoint{0.028866in}{-0.049702in}}{\pgfqpoint{0.039284in}{-0.039284in}}%
\pgfpathcurveto{\pgfqpoint{0.049702in}{-0.028866in}}{\pgfqpoint{0.055556in}{-0.014734in}}{\pgfqpoint{0.055556in}{0.000000in}}%
\pgfpathcurveto{\pgfqpoint{0.055556in}{0.014734in}}{\pgfqpoint{0.049702in}{0.028866in}}{\pgfqpoint{0.039284in}{0.039284in}}%
\pgfpathcurveto{\pgfqpoint{0.028866in}{0.049702in}}{\pgfqpoint{0.014734in}{0.055556in}}{\pgfqpoint{0.000000in}{0.055556in}}%
\pgfpathcurveto{\pgfqpoint{-0.014734in}{0.055556in}}{\pgfqpoint{-0.028866in}{0.049702in}}{\pgfqpoint{-0.039284in}{0.039284in}}%
\pgfpathcurveto{\pgfqpoint{-0.049702in}{0.028866in}}{\pgfqpoint{-0.055556in}{0.014734in}}{\pgfqpoint{-0.055556in}{0.000000in}}%
\pgfpathcurveto{\pgfqpoint{-0.055556in}{-0.014734in}}{\pgfqpoint{-0.049702in}{-0.028866in}}{\pgfqpoint{-0.039284in}{-0.039284in}}%
\pgfpathcurveto{\pgfqpoint{-0.028866in}{-0.049702in}}{\pgfqpoint{-0.014734in}{-0.055556in}}{\pgfqpoint{0.000000in}{-0.055556in}}%
\pgfpathclose%
\pgfusepath{stroke,fill}%
}%
\begin{pgfscope}%
\pgfsys@transformshift{2.758186in}{2.152371in}%
\pgfsys@useobject{currentmarker}{}%
\end{pgfscope}%
\end{pgfscope}%
\begin{pgfscope}%
\pgfpathrectangle{\pgfqpoint{0.100000in}{0.100000in}}{\pgfqpoint{5.307240in}{3.397500in}}%
\pgfusepath{clip}%
\pgfsetrectcap%
\pgfsetroundjoin%
\pgfsetlinewidth{1.505625pt}%
\definecolor{currentstroke}{rgb}{0.678431,1.000000,0.184314}%
\pgfsetstrokecolor{currentstroke}%
\pgfsetstrokeopacity{0.500000}%
\pgfsetdash{}{0pt}%
\pgfpathmoveto{\pgfqpoint{2.904700in}{2.133875in}}%
\pgfusepath{stroke}%
\end{pgfscope}%
\begin{pgfscope}%
\pgfpathrectangle{\pgfqpoint{0.100000in}{0.100000in}}{\pgfqpoint{5.307240in}{3.397500in}}%
\pgfusepath{clip}%
\pgfsetbuttcap%
\pgfsetroundjoin%
\definecolor{currentfill}{rgb}{0.678431,1.000000,0.184314}%
\pgfsetfillcolor{currentfill}%
\pgfsetfillopacity{0.500000}%
\pgfsetlinewidth{0.250937pt}%
\definecolor{currentstroke}{rgb}{0.000000,0.000000,0.000000}%
\pgfsetstrokecolor{currentstroke}%
\pgfsetstrokeopacity{0.500000}%
\pgfsetdash{}{0pt}%
\pgfsys@defobject{currentmarker}{\pgfqpoint{-0.045833in}{-0.045833in}}{\pgfqpoint{0.045833in}{0.045833in}}{%
\pgfpathmoveto{\pgfqpoint{0.000000in}{-0.045833in}}%
\pgfpathcurveto{\pgfqpoint{0.012155in}{-0.045833in}}{\pgfqpoint{0.023814in}{-0.041004in}}{\pgfqpoint{0.032409in}{-0.032409in}}%
\pgfpathcurveto{\pgfqpoint{0.041004in}{-0.023814in}}{\pgfqpoint{0.045833in}{-0.012155in}}{\pgfqpoint{0.045833in}{0.000000in}}%
\pgfpathcurveto{\pgfqpoint{0.045833in}{0.012155in}}{\pgfqpoint{0.041004in}{0.023814in}}{\pgfqpoint{0.032409in}{0.032409in}}%
\pgfpathcurveto{\pgfqpoint{0.023814in}{0.041004in}}{\pgfqpoint{0.012155in}{0.045833in}}{\pgfqpoint{0.000000in}{0.045833in}}%
\pgfpathcurveto{\pgfqpoint{-0.012155in}{0.045833in}}{\pgfqpoint{-0.023814in}{0.041004in}}{\pgfqpoint{-0.032409in}{0.032409in}}%
\pgfpathcurveto{\pgfqpoint{-0.041004in}{0.023814in}}{\pgfqpoint{-0.045833in}{0.012155in}}{\pgfqpoint{-0.045833in}{0.000000in}}%
\pgfpathcurveto{\pgfqpoint{-0.045833in}{-0.012155in}}{\pgfqpoint{-0.041004in}{-0.023814in}}{\pgfqpoint{-0.032409in}{-0.032409in}}%
\pgfpathcurveto{\pgfqpoint{-0.023814in}{-0.041004in}}{\pgfqpoint{-0.012155in}{-0.045833in}}{\pgfqpoint{0.000000in}{-0.045833in}}%
\pgfpathclose%
\pgfusepath{stroke,fill}%
}%
\begin{pgfscope}%
\pgfsys@transformshift{2.904700in}{2.133875in}%
\pgfsys@useobject{currentmarker}{}%
\end{pgfscope}%
\end{pgfscope}%
\begin{pgfscope}%
\pgfpathrectangle{\pgfqpoint{0.100000in}{0.100000in}}{\pgfqpoint{5.307240in}{3.397500in}}%
\pgfusepath{clip}%
\pgfsetrectcap%
\pgfsetroundjoin%
\pgfsetlinewidth{1.505625pt}%
\definecolor{currentstroke}{rgb}{0.678431,1.000000,0.184314}%
\pgfsetstrokecolor{currentstroke}%
\pgfsetstrokeopacity{0.500000}%
\pgfsetdash{}{0pt}%
\pgfpathmoveto{\pgfqpoint{2.969509in}{2.186310in}}%
\pgfusepath{stroke}%
\end{pgfscope}%
\begin{pgfscope}%
\pgfpathrectangle{\pgfqpoint{0.100000in}{0.100000in}}{\pgfqpoint{5.307240in}{3.397500in}}%
\pgfusepath{clip}%
\pgfsetbuttcap%
\pgfsetroundjoin%
\definecolor{currentfill}{rgb}{0.678431,1.000000,0.184314}%
\pgfsetfillcolor{currentfill}%
\pgfsetfillopacity{0.500000}%
\pgfsetlinewidth{0.250937pt}%
\definecolor{currentstroke}{rgb}{0.000000,0.000000,0.000000}%
\pgfsetstrokecolor{currentstroke}%
\pgfsetstrokeopacity{0.500000}%
\pgfsetdash{}{0pt}%
\pgfsys@defobject{currentmarker}{\pgfqpoint{-0.048611in}{-0.048611in}}{\pgfqpoint{0.048611in}{0.048611in}}{%
\pgfpathmoveto{\pgfqpoint{0.000000in}{-0.048611in}}%
\pgfpathcurveto{\pgfqpoint{0.012892in}{-0.048611in}}{\pgfqpoint{0.025257in}{-0.043489in}}{\pgfqpoint{0.034373in}{-0.034373in}}%
\pgfpathcurveto{\pgfqpoint{0.043489in}{-0.025257in}}{\pgfqpoint{0.048611in}{-0.012892in}}{\pgfqpoint{0.048611in}{0.000000in}}%
\pgfpathcurveto{\pgfqpoint{0.048611in}{0.012892in}}{\pgfqpoint{0.043489in}{0.025257in}}{\pgfqpoint{0.034373in}{0.034373in}}%
\pgfpathcurveto{\pgfqpoint{0.025257in}{0.043489in}}{\pgfqpoint{0.012892in}{0.048611in}}{\pgfqpoint{0.000000in}{0.048611in}}%
\pgfpathcurveto{\pgfqpoint{-0.012892in}{0.048611in}}{\pgfqpoint{-0.025257in}{0.043489in}}{\pgfqpoint{-0.034373in}{0.034373in}}%
\pgfpathcurveto{\pgfqpoint{-0.043489in}{0.025257in}}{\pgfqpoint{-0.048611in}{0.012892in}}{\pgfqpoint{-0.048611in}{0.000000in}}%
\pgfpathcurveto{\pgfqpoint{-0.048611in}{-0.012892in}}{\pgfqpoint{-0.043489in}{-0.025257in}}{\pgfqpoint{-0.034373in}{-0.034373in}}%
\pgfpathcurveto{\pgfqpoint{-0.025257in}{-0.043489in}}{\pgfqpoint{-0.012892in}{-0.048611in}}{\pgfqpoint{0.000000in}{-0.048611in}}%
\pgfpathclose%
\pgfusepath{stroke,fill}%
}%
\begin{pgfscope}%
\pgfsys@transformshift{2.969509in}{2.186310in}%
\pgfsys@useobject{currentmarker}{}%
\end{pgfscope}%
\end{pgfscope}%
\begin{pgfscope}%
\pgfpathrectangle{\pgfqpoint{0.100000in}{0.100000in}}{\pgfqpoint{5.307240in}{3.397500in}}%
\pgfusepath{clip}%
\pgfsetrectcap%
\pgfsetroundjoin%
\pgfsetlinewidth{1.505625pt}%
\definecolor{currentstroke}{rgb}{0.678431,1.000000,0.184314}%
\pgfsetstrokecolor{currentstroke}%
\pgfsetstrokeopacity{0.500000}%
\pgfsetdash{}{0pt}%
\pgfpathmoveto{\pgfqpoint{0.846688in}{2.244616in}}%
\pgfusepath{stroke}%
\end{pgfscope}%
\begin{pgfscope}%
\pgfpathrectangle{\pgfqpoint{0.100000in}{0.100000in}}{\pgfqpoint{5.307240in}{3.397500in}}%
\pgfusepath{clip}%
\pgfsetbuttcap%
\pgfsetroundjoin%
\definecolor{currentfill}{rgb}{0.678431,1.000000,0.184314}%
\pgfsetfillcolor{currentfill}%
\pgfsetfillopacity{0.500000}%
\pgfsetlinewidth{0.250937pt}%
\definecolor{currentstroke}{rgb}{0.000000,0.000000,0.000000}%
\pgfsetstrokecolor{currentstroke}%
\pgfsetstrokeopacity{0.500000}%
\pgfsetdash{}{0pt}%
\pgfsys@defobject{currentmarker}{\pgfqpoint{-0.120833in}{-0.120833in}}{\pgfqpoint{0.120833in}{0.120833in}}{%
\pgfpathmoveto{\pgfqpoint{0.000000in}{-0.120833in}}%
\pgfpathcurveto{\pgfqpoint{0.032045in}{-0.120833in}}{\pgfqpoint{0.062783in}{-0.108102in}}{\pgfqpoint{0.085442in}{-0.085442in}}%
\pgfpathcurveto{\pgfqpoint{0.108102in}{-0.062783in}}{\pgfqpoint{0.120833in}{-0.032045in}}{\pgfqpoint{0.120833in}{0.000000in}}%
\pgfpathcurveto{\pgfqpoint{0.120833in}{0.032045in}}{\pgfqpoint{0.108102in}{0.062783in}}{\pgfqpoint{0.085442in}{0.085442in}}%
\pgfpathcurveto{\pgfqpoint{0.062783in}{0.108102in}}{\pgfqpoint{0.032045in}{0.120833in}}{\pgfqpoint{0.000000in}{0.120833in}}%
\pgfpathcurveto{\pgfqpoint{-0.032045in}{0.120833in}}{\pgfqpoint{-0.062783in}{0.108102in}}{\pgfqpoint{-0.085442in}{0.085442in}}%
\pgfpathcurveto{\pgfqpoint{-0.108102in}{0.062783in}}{\pgfqpoint{-0.120833in}{0.032045in}}{\pgfqpoint{-0.120833in}{0.000000in}}%
\pgfpathcurveto{\pgfqpoint{-0.120833in}{-0.032045in}}{\pgfqpoint{-0.108102in}{-0.062783in}}{\pgfqpoint{-0.085442in}{-0.085442in}}%
\pgfpathcurveto{\pgfqpoint{-0.062783in}{-0.108102in}}{\pgfqpoint{-0.032045in}{-0.120833in}}{\pgfqpoint{0.000000in}{-0.120833in}}%
\pgfpathclose%
\pgfusepath{stroke,fill}%
}%
\begin{pgfscope}%
\pgfsys@transformshift{0.846688in}{2.244616in}%
\pgfsys@useobject{currentmarker}{}%
\end{pgfscope}%
\end{pgfscope}%
\begin{pgfscope}%
\pgfpathrectangle{\pgfqpoint{0.100000in}{0.100000in}}{\pgfqpoint{5.307240in}{3.397500in}}%
\pgfusepath{clip}%
\pgfsetrectcap%
\pgfsetroundjoin%
\pgfsetlinewidth{1.505625pt}%
\definecolor{currentstroke}{rgb}{0.678431,1.000000,0.184314}%
\pgfsetstrokecolor{currentstroke}%
\pgfsetstrokeopacity{0.500000}%
\pgfsetdash{}{0pt}%
\pgfpathmoveto{\pgfqpoint{1.173590in}{1.802731in}}%
\pgfusepath{stroke}%
\end{pgfscope}%
\begin{pgfscope}%
\pgfpathrectangle{\pgfqpoint{0.100000in}{0.100000in}}{\pgfqpoint{5.307240in}{3.397500in}}%
\pgfusepath{clip}%
\pgfsetbuttcap%
\pgfsetroundjoin%
\definecolor{currentfill}{rgb}{0.678431,1.000000,0.184314}%
\pgfsetfillcolor{currentfill}%
\pgfsetfillopacity{0.500000}%
\pgfsetlinewidth{0.250937pt}%
\definecolor{currentstroke}{rgb}{0.000000,0.000000,0.000000}%
\pgfsetstrokecolor{currentstroke}%
\pgfsetstrokeopacity{0.500000}%
\pgfsetdash{}{0pt}%
\pgfsys@defobject{currentmarker}{\pgfqpoint{-0.204861in}{-0.204861in}}{\pgfqpoint{0.204861in}{0.204861in}}{%
\pgfpathmoveto{\pgfqpoint{0.000000in}{-0.204861in}}%
\pgfpathcurveto{\pgfqpoint{0.054330in}{-0.204861in}}{\pgfqpoint{0.106442in}{-0.183276in}}{\pgfqpoint{0.144859in}{-0.144859in}}%
\pgfpathcurveto{\pgfqpoint{0.183276in}{-0.106442in}}{\pgfqpoint{0.204861in}{-0.054330in}}{\pgfqpoint{0.204861in}{0.000000in}}%
\pgfpathcurveto{\pgfqpoint{0.204861in}{0.054330in}}{\pgfqpoint{0.183276in}{0.106442in}}{\pgfqpoint{0.144859in}{0.144859in}}%
\pgfpathcurveto{\pgfqpoint{0.106442in}{0.183276in}}{\pgfqpoint{0.054330in}{0.204861in}}{\pgfqpoint{0.000000in}{0.204861in}}%
\pgfpathcurveto{\pgfqpoint{-0.054330in}{0.204861in}}{\pgfqpoint{-0.106442in}{0.183276in}}{\pgfqpoint{-0.144859in}{0.144859in}}%
\pgfpathcurveto{\pgfqpoint{-0.183276in}{0.106442in}}{\pgfqpoint{-0.204861in}{0.054330in}}{\pgfqpoint{-0.204861in}{0.000000in}}%
\pgfpathcurveto{\pgfqpoint{-0.204861in}{-0.054330in}}{\pgfqpoint{-0.183276in}{-0.106442in}}{\pgfqpoint{-0.144859in}{-0.144859in}}%
\pgfpathcurveto{\pgfqpoint{-0.106442in}{-0.183276in}}{\pgfqpoint{-0.054330in}{-0.204861in}}{\pgfqpoint{0.000000in}{-0.204861in}}%
\pgfpathclose%
\pgfusepath{stroke,fill}%
}%
\begin{pgfscope}%
\pgfsys@transformshift{1.173590in}{1.802731in}%
\pgfsys@useobject{currentmarker}{}%
\end{pgfscope}%
\end{pgfscope}%
\begin{pgfscope}%
\pgfpathrectangle{\pgfqpoint{0.100000in}{0.100000in}}{\pgfqpoint{5.307240in}{3.397500in}}%
\pgfusepath{clip}%
\pgfsetrectcap%
\pgfsetroundjoin%
\pgfsetlinewidth{1.505625pt}%
\definecolor{currentstroke}{rgb}{0.678431,1.000000,0.184314}%
\pgfsetstrokecolor{currentstroke}%
\pgfsetstrokeopacity{0.500000}%
\pgfsetdash{}{0pt}%
\pgfpathmoveto{\pgfqpoint{0.854129in}{2.286652in}}%
\pgfusepath{stroke}%
\end{pgfscope}%
\begin{pgfscope}%
\pgfpathrectangle{\pgfqpoint{0.100000in}{0.100000in}}{\pgfqpoint{5.307240in}{3.397500in}}%
\pgfusepath{clip}%
\pgfsetbuttcap%
\pgfsetroundjoin%
\definecolor{currentfill}{rgb}{0.678431,1.000000,0.184314}%
\pgfsetfillcolor{currentfill}%
\pgfsetfillopacity{0.500000}%
\pgfsetlinewidth{0.250937pt}%
\definecolor{currentstroke}{rgb}{0.000000,0.000000,0.000000}%
\pgfsetstrokecolor{currentstroke}%
\pgfsetstrokeopacity{0.500000}%
\pgfsetdash{}{0pt}%
\pgfsys@defobject{currentmarker}{\pgfqpoint{-0.113194in}{-0.113194in}}{\pgfqpoint{0.113194in}{0.113194in}}{%
\pgfpathmoveto{\pgfqpoint{0.000000in}{-0.113194in}}%
\pgfpathcurveto{\pgfqpoint{0.030020in}{-0.113194in}}{\pgfqpoint{0.058814in}{-0.101268in}}{\pgfqpoint{0.080041in}{-0.080041in}}%
\pgfpathcurveto{\pgfqpoint{0.101268in}{-0.058814in}}{\pgfqpoint{0.113194in}{-0.030020in}}{\pgfqpoint{0.113194in}{0.000000in}}%
\pgfpathcurveto{\pgfqpoint{0.113194in}{0.030020in}}{\pgfqpoint{0.101268in}{0.058814in}}{\pgfqpoint{0.080041in}{0.080041in}}%
\pgfpathcurveto{\pgfqpoint{0.058814in}{0.101268in}}{\pgfqpoint{0.030020in}{0.113194in}}{\pgfqpoint{0.000000in}{0.113194in}}%
\pgfpathcurveto{\pgfqpoint{-0.030020in}{0.113194in}}{\pgfqpoint{-0.058814in}{0.101268in}}{\pgfqpoint{-0.080041in}{0.080041in}}%
\pgfpathcurveto{\pgfqpoint{-0.101268in}{0.058814in}}{\pgfqpoint{-0.113194in}{0.030020in}}{\pgfqpoint{-0.113194in}{0.000000in}}%
\pgfpathcurveto{\pgfqpoint{-0.113194in}{-0.030020in}}{\pgfqpoint{-0.101268in}{-0.058814in}}{\pgfqpoint{-0.080041in}{-0.080041in}}%
\pgfpathcurveto{\pgfqpoint{-0.058814in}{-0.101268in}}{\pgfqpoint{-0.030020in}{-0.113194in}}{\pgfqpoint{0.000000in}{-0.113194in}}%
\pgfpathclose%
\pgfusepath{stroke,fill}%
}%
\begin{pgfscope}%
\pgfsys@transformshift{0.854129in}{2.286652in}%
\pgfsys@useobject{currentmarker}{}%
\end{pgfscope}%
\end{pgfscope}%
\begin{pgfscope}%
\pgfpathrectangle{\pgfqpoint{0.100000in}{0.100000in}}{\pgfqpoint{5.307240in}{3.397500in}}%
\pgfusepath{clip}%
\pgfsetrectcap%
\pgfsetroundjoin%
\pgfsetlinewidth{1.505625pt}%
\definecolor{currentstroke}{rgb}{0.678431,1.000000,0.184314}%
\pgfsetstrokecolor{currentstroke}%
\pgfsetstrokeopacity{0.500000}%
\pgfsetdash{}{0pt}%
\pgfpathmoveto{\pgfqpoint{5.077509in}{2.682190in}}%
\pgfusepath{stroke}%
\end{pgfscope}%
\begin{pgfscope}%
\pgfpathrectangle{\pgfqpoint{0.100000in}{0.100000in}}{\pgfqpoint{5.307240in}{3.397500in}}%
\pgfusepath{clip}%
\pgfsetbuttcap%
\pgfsetroundjoin%
\definecolor{currentfill}{rgb}{0.678431,1.000000,0.184314}%
\pgfsetfillcolor{currentfill}%
\pgfsetfillopacity{0.500000}%
\pgfsetlinewidth{0.250937pt}%
\definecolor{currentstroke}{rgb}{0.000000,0.000000,0.000000}%
\pgfsetstrokecolor{currentstroke}%
\pgfsetstrokeopacity{0.500000}%
\pgfsetdash{}{0pt}%
\pgfsys@defobject{currentmarker}{\pgfqpoint{-0.091667in}{-0.091667in}}{\pgfqpoint{0.091667in}{0.091667in}}{%
\pgfpathmoveto{\pgfqpoint{0.000000in}{-0.091667in}}%
\pgfpathcurveto{\pgfqpoint{0.024310in}{-0.091667in}}{\pgfqpoint{0.047628in}{-0.082008in}}{\pgfqpoint{0.064818in}{-0.064818in}}%
\pgfpathcurveto{\pgfqpoint{0.082008in}{-0.047628in}}{\pgfqpoint{0.091667in}{-0.024310in}}{\pgfqpoint{0.091667in}{0.000000in}}%
\pgfpathcurveto{\pgfqpoint{0.091667in}{0.024310in}}{\pgfqpoint{0.082008in}{0.047628in}}{\pgfqpoint{0.064818in}{0.064818in}}%
\pgfpathcurveto{\pgfqpoint{0.047628in}{0.082008in}}{\pgfqpoint{0.024310in}{0.091667in}}{\pgfqpoint{0.000000in}{0.091667in}}%
\pgfpathcurveto{\pgfqpoint{-0.024310in}{0.091667in}}{\pgfqpoint{-0.047628in}{0.082008in}}{\pgfqpoint{-0.064818in}{0.064818in}}%
\pgfpathcurveto{\pgfqpoint{-0.082008in}{0.047628in}}{\pgfqpoint{-0.091667in}{0.024310in}}{\pgfqpoint{-0.091667in}{0.000000in}}%
\pgfpathcurveto{\pgfqpoint{-0.091667in}{-0.024310in}}{\pgfqpoint{-0.082008in}{-0.047628in}}{\pgfqpoint{-0.064818in}{-0.064818in}}%
\pgfpathcurveto{\pgfqpoint{-0.047628in}{-0.082008in}}{\pgfqpoint{-0.024310in}{-0.091667in}}{\pgfqpoint{0.000000in}{-0.091667in}}%
\pgfpathclose%
\pgfusepath{stroke,fill}%
}%
\begin{pgfscope}%
\pgfsys@transformshift{5.077509in}{2.682190in}%
\pgfsys@useobject{currentmarker}{}%
\end{pgfscope}%
\end{pgfscope}%
\begin{pgfscope}%
\pgfpathrectangle{\pgfqpoint{0.100000in}{0.100000in}}{\pgfqpoint{5.307240in}{3.397500in}}%
\pgfusepath{clip}%
\pgfsetrectcap%
\pgfsetroundjoin%
\pgfsetlinewidth{1.505625pt}%
\definecolor{currentstroke}{rgb}{0.678431,1.000000,0.184314}%
\pgfsetstrokecolor{currentstroke}%
\pgfsetstrokeopacity{0.500000}%
\pgfsetdash{}{0pt}%
\pgfpathmoveto{\pgfqpoint{5.035859in}{2.646606in}}%
\pgfusepath{stroke}%
\end{pgfscope}%
\begin{pgfscope}%
\pgfpathrectangle{\pgfqpoint{0.100000in}{0.100000in}}{\pgfqpoint{5.307240in}{3.397500in}}%
\pgfusepath{clip}%
\pgfsetbuttcap%
\pgfsetroundjoin%
\definecolor{currentfill}{rgb}{0.678431,1.000000,0.184314}%
\pgfsetfillcolor{currentfill}%
\pgfsetfillopacity{0.500000}%
\pgfsetlinewidth{0.250937pt}%
\definecolor{currentstroke}{rgb}{0.000000,0.000000,0.000000}%
\pgfsetstrokecolor{currentstroke}%
\pgfsetstrokeopacity{0.500000}%
\pgfsetdash{}{0pt}%
\pgfsys@defobject{currentmarker}{\pgfqpoint{-0.104861in}{-0.104861in}}{\pgfqpoint{0.104861in}{0.104861in}}{%
\pgfpathmoveto{\pgfqpoint{0.000000in}{-0.104861in}}%
\pgfpathcurveto{\pgfqpoint{0.027809in}{-0.104861in}}{\pgfqpoint{0.054484in}{-0.093812in}}{\pgfqpoint{0.074148in}{-0.074148in}}%
\pgfpathcurveto{\pgfqpoint{0.093812in}{-0.054484in}}{\pgfqpoint{0.104861in}{-0.027809in}}{\pgfqpoint{0.104861in}{0.000000in}}%
\pgfpathcurveto{\pgfqpoint{0.104861in}{0.027809in}}{\pgfqpoint{0.093812in}{0.054484in}}{\pgfqpoint{0.074148in}{0.074148in}}%
\pgfpathcurveto{\pgfqpoint{0.054484in}{0.093812in}}{\pgfqpoint{0.027809in}{0.104861in}}{\pgfqpoint{0.000000in}{0.104861in}}%
\pgfpathcurveto{\pgfqpoint{-0.027809in}{0.104861in}}{\pgfqpoint{-0.054484in}{0.093812in}}{\pgfqpoint{-0.074148in}{0.074148in}}%
\pgfpathcurveto{\pgfqpoint{-0.093812in}{0.054484in}}{\pgfqpoint{-0.104861in}{0.027809in}}{\pgfqpoint{-0.104861in}{0.000000in}}%
\pgfpathcurveto{\pgfqpoint{-0.104861in}{-0.027809in}}{\pgfqpoint{-0.093812in}{-0.054484in}}{\pgfqpoint{-0.074148in}{-0.074148in}}%
\pgfpathcurveto{\pgfqpoint{-0.054484in}{-0.093812in}}{\pgfqpoint{-0.027809in}{-0.104861in}}{\pgfqpoint{0.000000in}{-0.104861in}}%
\pgfpathclose%
\pgfusepath{stroke,fill}%
}%
\begin{pgfscope}%
\pgfsys@transformshift{5.035859in}{2.646606in}%
\pgfsys@useobject{currentmarker}{}%
\end{pgfscope}%
\end{pgfscope}%
\begin{pgfscope}%
\pgfpathrectangle{\pgfqpoint{0.100000in}{0.100000in}}{\pgfqpoint{5.307240in}{3.397500in}}%
\pgfusepath{clip}%
\pgfsetrectcap%
\pgfsetroundjoin%
\pgfsetlinewidth{1.505625pt}%
\definecolor{currentstroke}{rgb}{0.678431,1.000000,0.184314}%
\pgfsetstrokecolor{currentstroke}%
\pgfsetstrokeopacity{0.500000}%
\pgfsetdash{}{0pt}%
\pgfpathmoveto{\pgfqpoint{5.090492in}{2.670495in}}%
\pgfusepath{stroke}%
\end{pgfscope}%
\begin{pgfscope}%
\pgfpathrectangle{\pgfqpoint{0.100000in}{0.100000in}}{\pgfqpoint{5.307240in}{3.397500in}}%
\pgfusepath{clip}%
\pgfsetbuttcap%
\pgfsetroundjoin%
\definecolor{currentfill}{rgb}{0.678431,1.000000,0.184314}%
\pgfsetfillcolor{currentfill}%
\pgfsetfillopacity{0.500000}%
\pgfsetlinewidth{0.250937pt}%
\definecolor{currentstroke}{rgb}{0.000000,0.000000,0.000000}%
\pgfsetstrokecolor{currentstroke}%
\pgfsetstrokeopacity{0.500000}%
\pgfsetdash{}{0pt}%
\pgfsys@defobject{currentmarker}{\pgfqpoint{-0.088889in}{-0.088889in}}{\pgfqpoint{0.088889in}{0.088889in}}{%
\pgfpathmoveto{\pgfqpoint{0.000000in}{-0.088889in}}%
\pgfpathcurveto{\pgfqpoint{0.023574in}{-0.088889in}}{\pgfqpoint{0.046185in}{-0.079523in}}{\pgfqpoint{0.062854in}{-0.062854in}}%
\pgfpathcurveto{\pgfqpoint{0.079523in}{-0.046185in}}{\pgfqpoint{0.088889in}{-0.023574in}}{\pgfqpoint{0.088889in}{0.000000in}}%
\pgfpathcurveto{\pgfqpoint{0.088889in}{0.023574in}}{\pgfqpoint{0.079523in}{0.046185in}}{\pgfqpoint{0.062854in}{0.062854in}}%
\pgfpathcurveto{\pgfqpoint{0.046185in}{0.079523in}}{\pgfqpoint{0.023574in}{0.088889in}}{\pgfqpoint{0.000000in}{0.088889in}}%
\pgfpathcurveto{\pgfqpoint{-0.023574in}{0.088889in}}{\pgfqpoint{-0.046185in}{0.079523in}}{\pgfqpoint{-0.062854in}{0.062854in}}%
\pgfpathcurveto{\pgfqpoint{-0.079523in}{0.046185in}}{\pgfqpoint{-0.088889in}{0.023574in}}{\pgfqpoint{-0.088889in}{0.000000in}}%
\pgfpathcurveto{\pgfqpoint{-0.088889in}{-0.023574in}}{\pgfqpoint{-0.079523in}{-0.046185in}}{\pgfqpoint{-0.062854in}{-0.062854in}}%
\pgfpathcurveto{\pgfqpoint{-0.046185in}{-0.079523in}}{\pgfqpoint{-0.023574in}{-0.088889in}}{\pgfqpoint{0.000000in}{-0.088889in}}%
\pgfpathclose%
\pgfusepath{stroke,fill}%
}%
\begin{pgfscope}%
\pgfsys@transformshift{5.090492in}{2.670495in}%
\pgfsys@useobject{currentmarker}{}%
\end{pgfscope}%
\end{pgfscope}%
\begin{pgfscope}%
\pgfpathrectangle{\pgfqpoint{0.100000in}{0.100000in}}{\pgfqpoint{5.307240in}{3.397500in}}%
\pgfusepath{clip}%
\pgfsetrectcap%
\pgfsetroundjoin%
\pgfsetlinewidth{1.505625pt}%
\definecolor{currentstroke}{rgb}{0.678431,1.000000,0.184314}%
\pgfsetstrokecolor{currentstroke}%
\pgfsetstrokeopacity{0.500000}%
\pgfsetdash{}{0pt}%
\pgfpathmoveto{\pgfqpoint{4.885170in}{2.174083in}}%
\pgfusepath{stroke}%
\end{pgfscope}%
\begin{pgfscope}%
\pgfpathrectangle{\pgfqpoint{0.100000in}{0.100000in}}{\pgfqpoint{5.307240in}{3.397500in}}%
\pgfusepath{clip}%
\pgfsetbuttcap%
\pgfsetroundjoin%
\definecolor{currentfill}{rgb}{0.678431,1.000000,0.184314}%
\pgfsetfillcolor{currentfill}%
\pgfsetfillopacity{0.500000}%
\pgfsetlinewidth{0.250937pt}%
\definecolor{currentstroke}{rgb}{0.000000,0.000000,0.000000}%
\pgfsetstrokecolor{currentstroke}%
\pgfsetstrokeopacity{0.500000}%
\pgfsetdash{}{0pt}%
\pgfsys@defobject{currentmarker}{\pgfqpoint{-0.202083in}{-0.202083in}}{\pgfqpoint{0.202083in}{0.202083in}}{%
\pgfpathmoveto{\pgfqpoint{0.000000in}{-0.202083in}}%
\pgfpathcurveto{\pgfqpoint{0.053593in}{-0.202083in}}{\pgfqpoint{0.104998in}{-0.180791in}}{\pgfqpoint{0.142894in}{-0.142894in}}%
\pgfpathcurveto{\pgfqpoint{0.180791in}{-0.104998in}}{\pgfqpoint{0.202083in}{-0.053593in}}{\pgfqpoint{0.202083in}{0.000000in}}%
\pgfpathcurveto{\pgfqpoint{0.202083in}{0.053593in}}{\pgfqpoint{0.180791in}{0.104998in}}{\pgfqpoint{0.142894in}{0.142894in}}%
\pgfpathcurveto{\pgfqpoint{0.104998in}{0.180791in}}{\pgfqpoint{0.053593in}{0.202083in}}{\pgfqpoint{0.000000in}{0.202083in}}%
\pgfpathcurveto{\pgfqpoint{-0.053593in}{0.202083in}}{\pgfqpoint{-0.104998in}{0.180791in}}{\pgfqpoint{-0.142894in}{0.142894in}}%
\pgfpathcurveto{\pgfqpoint{-0.180791in}{0.104998in}}{\pgfqpoint{-0.202083in}{0.053593in}}{\pgfqpoint{-0.202083in}{0.000000in}}%
\pgfpathcurveto{\pgfqpoint{-0.202083in}{-0.053593in}}{\pgfqpoint{-0.180791in}{-0.104998in}}{\pgfqpoint{-0.142894in}{-0.142894in}}%
\pgfpathcurveto{\pgfqpoint{-0.104998in}{-0.180791in}}{\pgfqpoint{-0.053593in}{-0.202083in}}{\pgfqpoint{0.000000in}{-0.202083in}}%
\pgfpathclose%
\pgfusepath{stroke,fill}%
}%
\begin{pgfscope}%
\pgfsys@transformshift{4.885170in}{2.174083in}%
\pgfsys@useobject{currentmarker}{}%
\end{pgfscope}%
\end{pgfscope}%
\begin{pgfscope}%
\pgfpathrectangle{\pgfqpoint{0.100000in}{0.100000in}}{\pgfqpoint{5.307240in}{3.397500in}}%
\pgfusepath{clip}%
\pgfsetrectcap%
\pgfsetroundjoin%
\pgfsetlinewidth{1.505625pt}%
\definecolor{currentstroke}{rgb}{0.678431,1.000000,0.184314}%
\pgfsetstrokecolor{currentstroke}%
\pgfsetstrokeopacity{0.500000}%
\pgfsetdash{}{0pt}%
\pgfpathmoveto{\pgfqpoint{4.874129in}{2.161212in}}%
\pgfusepath{stroke}%
\end{pgfscope}%
\begin{pgfscope}%
\pgfpathrectangle{\pgfqpoint{0.100000in}{0.100000in}}{\pgfqpoint{5.307240in}{3.397500in}}%
\pgfusepath{clip}%
\pgfsetbuttcap%
\pgfsetroundjoin%
\definecolor{currentfill}{rgb}{0.678431,1.000000,0.184314}%
\pgfsetfillcolor{currentfill}%
\pgfsetfillopacity{0.500000}%
\pgfsetlinewidth{0.250937pt}%
\definecolor{currentstroke}{rgb}{0.000000,0.000000,0.000000}%
\pgfsetstrokecolor{currentstroke}%
\pgfsetstrokeopacity{0.500000}%
\pgfsetdash{}{0pt}%
\pgfsys@defobject{currentmarker}{\pgfqpoint{-0.133333in}{-0.133333in}}{\pgfqpoint{0.133333in}{0.133333in}}{%
\pgfpathmoveto{\pgfqpoint{0.000000in}{-0.133333in}}%
\pgfpathcurveto{\pgfqpoint{0.035360in}{-0.133333in}}{\pgfqpoint{0.069277in}{-0.119284in}}{\pgfqpoint{0.094281in}{-0.094281in}}%
\pgfpathcurveto{\pgfqpoint{0.119284in}{-0.069277in}}{\pgfqpoint{0.133333in}{-0.035360in}}{\pgfqpoint{0.133333in}{0.000000in}}%
\pgfpathcurveto{\pgfqpoint{0.133333in}{0.035360in}}{\pgfqpoint{0.119284in}{0.069277in}}{\pgfqpoint{0.094281in}{0.094281in}}%
\pgfpathcurveto{\pgfqpoint{0.069277in}{0.119284in}}{\pgfqpoint{0.035360in}{0.133333in}}{\pgfqpoint{0.000000in}{0.133333in}}%
\pgfpathcurveto{\pgfqpoint{-0.035360in}{0.133333in}}{\pgfqpoint{-0.069277in}{0.119284in}}{\pgfqpoint{-0.094281in}{0.094281in}}%
\pgfpathcurveto{\pgfqpoint{-0.119284in}{0.069277in}}{\pgfqpoint{-0.133333in}{0.035360in}}{\pgfqpoint{-0.133333in}{0.000000in}}%
\pgfpathcurveto{\pgfqpoint{-0.133333in}{-0.035360in}}{\pgfqpoint{-0.119284in}{-0.069277in}}{\pgfqpoint{-0.094281in}{-0.094281in}}%
\pgfpathcurveto{\pgfqpoint{-0.069277in}{-0.119284in}}{\pgfqpoint{-0.035360in}{-0.133333in}}{\pgfqpoint{0.000000in}{-0.133333in}}%
\pgfpathclose%
\pgfusepath{stroke,fill}%
}%
\begin{pgfscope}%
\pgfsys@transformshift{4.874129in}{2.161212in}%
\pgfsys@useobject{currentmarker}{}%
\end{pgfscope}%
\end{pgfscope}%
\begin{pgfscope}%
\pgfpathrectangle{\pgfqpoint{0.100000in}{0.100000in}}{\pgfqpoint{5.307240in}{3.397500in}}%
\pgfusepath{clip}%
\pgfsetrectcap%
\pgfsetroundjoin%
\pgfsetlinewidth{1.505625pt}%
\definecolor{currentstroke}{rgb}{0.678431,1.000000,0.184314}%
\pgfsetstrokecolor{currentstroke}%
\pgfsetstrokeopacity{0.500000}%
\pgfsetdash{}{0pt}%
\pgfpathmoveto{\pgfqpoint{4.835214in}{2.264365in}}%
\pgfusepath{stroke}%
\end{pgfscope}%
\begin{pgfscope}%
\pgfpathrectangle{\pgfqpoint{0.100000in}{0.100000in}}{\pgfqpoint{5.307240in}{3.397500in}}%
\pgfusepath{clip}%
\pgfsetbuttcap%
\pgfsetroundjoin%
\definecolor{currentfill}{rgb}{0.678431,1.000000,0.184314}%
\pgfsetfillcolor{currentfill}%
\pgfsetfillopacity{0.500000}%
\pgfsetlinewidth{0.250937pt}%
\definecolor{currentstroke}{rgb}{0.000000,0.000000,0.000000}%
\pgfsetstrokecolor{currentstroke}%
\pgfsetstrokeopacity{0.500000}%
\pgfsetdash{}{0pt}%
\pgfsys@defobject{currentmarker}{\pgfqpoint{-0.056250in}{-0.056250in}}{\pgfqpoint{0.056250in}{0.056250in}}{%
\pgfpathmoveto{\pgfqpoint{0.000000in}{-0.056250in}}%
\pgfpathcurveto{\pgfqpoint{0.014918in}{-0.056250in}}{\pgfqpoint{0.029226in}{-0.050323in}}{\pgfqpoint{0.039775in}{-0.039775in}}%
\pgfpathcurveto{\pgfqpoint{0.050323in}{-0.029226in}}{\pgfqpoint{0.056250in}{-0.014918in}}{\pgfqpoint{0.056250in}{0.000000in}}%
\pgfpathcurveto{\pgfqpoint{0.056250in}{0.014918in}}{\pgfqpoint{0.050323in}{0.029226in}}{\pgfqpoint{0.039775in}{0.039775in}}%
\pgfpathcurveto{\pgfqpoint{0.029226in}{0.050323in}}{\pgfqpoint{0.014918in}{0.056250in}}{\pgfqpoint{0.000000in}{0.056250in}}%
\pgfpathcurveto{\pgfqpoint{-0.014918in}{0.056250in}}{\pgfqpoint{-0.029226in}{0.050323in}}{\pgfqpoint{-0.039775in}{0.039775in}}%
\pgfpathcurveto{\pgfqpoint{-0.050323in}{0.029226in}}{\pgfqpoint{-0.056250in}{0.014918in}}{\pgfqpoint{-0.056250in}{0.000000in}}%
\pgfpathcurveto{\pgfqpoint{-0.056250in}{-0.014918in}}{\pgfqpoint{-0.050323in}{-0.029226in}}{\pgfqpoint{-0.039775in}{-0.039775in}}%
\pgfpathcurveto{\pgfqpoint{-0.029226in}{-0.050323in}}{\pgfqpoint{-0.014918in}{-0.056250in}}{\pgfqpoint{0.000000in}{-0.056250in}}%
\pgfpathclose%
\pgfusepath{stroke,fill}%
}%
\begin{pgfscope}%
\pgfsys@transformshift{4.835214in}{2.264365in}%
\pgfsys@useobject{currentmarker}{}%
\end{pgfscope}%
\end{pgfscope}%
\begin{pgfscope}%
\pgfpathrectangle{\pgfqpoint{0.100000in}{0.100000in}}{\pgfqpoint{5.307240in}{3.397500in}}%
\pgfusepath{clip}%
\pgfsetrectcap%
\pgfsetroundjoin%
\pgfsetlinewidth{1.505625pt}%
\definecolor{currentstroke}{rgb}{0.678431,1.000000,0.184314}%
\pgfsetstrokecolor{currentstroke}%
\pgfsetstrokeopacity{0.500000}%
\pgfsetdash{}{0pt}%
\pgfpathmoveto{\pgfqpoint{4.831652in}{2.174911in}}%
\pgfusepath{stroke}%
\end{pgfscope}%
\begin{pgfscope}%
\pgfpathrectangle{\pgfqpoint{0.100000in}{0.100000in}}{\pgfqpoint{5.307240in}{3.397500in}}%
\pgfusepath{clip}%
\pgfsetbuttcap%
\pgfsetroundjoin%
\definecolor{currentfill}{rgb}{0.678431,1.000000,0.184314}%
\pgfsetfillcolor{currentfill}%
\pgfsetfillopacity{0.500000}%
\pgfsetlinewidth{0.250937pt}%
\definecolor{currentstroke}{rgb}{0.000000,0.000000,0.000000}%
\pgfsetstrokecolor{currentstroke}%
\pgfsetstrokeopacity{0.500000}%
\pgfsetdash{}{0pt}%
\pgfsys@defobject{currentmarker}{\pgfqpoint{-0.081250in}{-0.081250in}}{\pgfqpoint{0.081250in}{0.081250in}}{%
\pgfpathmoveto{\pgfqpoint{0.000000in}{-0.081250in}}%
\pgfpathcurveto{\pgfqpoint{0.021548in}{-0.081250in}}{\pgfqpoint{0.042216in}{-0.072689in}}{\pgfqpoint{0.057452in}{-0.057452in}}%
\pgfpathcurveto{\pgfqpoint{0.072689in}{-0.042216in}}{\pgfqpoint{0.081250in}{-0.021548in}}{\pgfqpoint{0.081250in}{0.000000in}}%
\pgfpathcurveto{\pgfqpoint{0.081250in}{0.021548in}}{\pgfqpoint{0.072689in}{0.042216in}}{\pgfqpoint{0.057452in}{0.057452in}}%
\pgfpathcurveto{\pgfqpoint{0.042216in}{0.072689in}}{\pgfqpoint{0.021548in}{0.081250in}}{\pgfqpoint{0.000000in}{0.081250in}}%
\pgfpathcurveto{\pgfqpoint{-0.021548in}{0.081250in}}{\pgfqpoint{-0.042216in}{0.072689in}}{\pgfqpoint{-0.057452in}{0.057452in}}%
\pgfpathcurveto{\pgfqpoint{-0.072689in}{0.042216in}}{\pgfqpoint{-0.081250in}{0.021548in}}{\pgfqpoint{-0.081250in}{0.000000in}}%
\pgfpathcurveto{\pgfqpoint{-0.081250in}{-0.021548in}}{\pgfqpoint{-0.072689in}{-0.042216in}}{\pgfqpoint{-0.057452in}{-0.057452in}}%
\pgfpathcurveto{\pgfqpoint{-0.042216in}{-0.072689in}}{\pgfqpoint{-0.021548in}{-0.081250in}}{\pgfqpoint{0.000000in}{-0.081250in}}%
\pgfpathclose%
\pgfusepath{stroke,fill}%
}%
\begin{pgfscope}%
\pgfsys@transformshift{4.831652in}{2.174911in}%
\pgfsys@useobject{currentmarker}{}%
\end{pgfscope}%
\end{pgfscope}%
\begin{pgfscope}%
\pgfpathrectangle{\pgfqpoint{0.100000in}{0.100000in}}{\pgfqpoint{5.307240in}{3.397500in}}%
\pgfusepath{clip}%
\pgfsetrectcap%
\pgfsetroundjoin%
\pgfsetlinewidth{1.505625pt}%
\definecolor{currentstroke}{rgb}{0.678431,1.000000,0.184314}%
\pgfsetstrokecolor{currentstroke}%
\pgfsetstrokeopacity{0.500000}%
\pgfsetdash{}{0pt}%
\pgfpathmoveto{\pgfqpoint{1.943573in}{1.538775in}}%
\pgfusepath{stroke}%
\end{pgfscope}%
\begin{pgfscope}%
\pgfpathrectangle{\pgfqpoint{0.100000in}{0.100000in}}{\pgfqpoint{5.307240in}{3.397500in}}%
\pgfusepath{clip}%
\pgfsetbuttcap%
\pgfsetroundjoin%
\definecolor{currentfill}{rgb}{0.678431,1.000000,0.184314}%
\pgfsetfillcolor{currentfill}%
\pgfsetfillopacity{0.500000}%
\pgfsetlinewidth{0.250937pt}%
\definecolor{currentstroke}{rgb}{0.000000,0.000000,0.000000}%
\pgfsetstrokecolor{currentstroke}%
\pgfsetstrokeopacity{0.500000}%
\pgfsetdash{}{0pt}%
\pgfsys@defobject{currentmarker}{\pgfqpoint{-0.057639in}{-0.057639in}}{\pgfqpoint{0.057639in}{0.057639in}}{%
\pgfpathmoveto{\pgfqpoint{0.000000in}{-0.057639in}}%
\pgfpathcurveto{\pgfqpoint{0.015286in}{-0.057639in}}{\pgfqpoint{0.029948in}{-0.051566in}}{\pgfqpoint{0.040757in}{-0.040757in}}%
\pgfpathcurveto{\pgfqpoint{0.051566in}{-0.029948in}}{\pgfqpoint{0.057639in}{-0.015286in}}{\pgfqpoint{0.057639in}{0.000000in}}%
\pgfpathcurveto{\pgfqpoint{0.057639in}{0.015286in}}{\pgfqpoint{0.051566in}{0.029948in}}{\pgfqpoint{0.040757in}{0.040757in}}%
\pgfpathcurveto{\pgfqpoint{0.029948in}{0.051566in}}{\pgfqpoint{0.015286in}{0.057639in}}{\pgfqpoint{0.000000in}{0.057639in}}%
\pgfpathcurveto{\pgfqpoint{-0.015286in}{0.057639in}}{\pgfqpoint{-0.029948in}{0.051566in}}{\pgfqpoint{-0.040757in}{0.040757in}}%
\pgfpathcurveto{\pgfqpoint{-0.051566in}{0.029948in}}{\pgfqpoint{-0.057639in}{0.015286in}}{\pgfqpoint{-0.057639in}{0.000000in}}%
\pgfpathcurveto{\pgfqpoint{-0.057639in}{-0.015286in}}{\pgfqpoint{-0.051566in}{-0.029948in}}{\pgfqpoint{-0.040757in}{-0.040757in}}%
\pgfpathcurveto{\pgfqpoint{-0.029948in}{-0.051566in}}{\pgfqpoint{-0.015286in}{-0.057639in}}{\pgfqpoint{0.000000in}{-0.057639in}}%
\pgfpathclose%
\pgfusepath{stroke,fill}%
}%
\begin{pgfscope}%
\pgfsys@transformshift{1.943573in}{1.538775in}%
\pgfsys@useobject{currentmarker}{}%
\end{pgfscope}%
\end{pgfscope}%
\begin{pgfscope}%
\pgfpathrectangle{\pgfqpoint{0.100000in}{0.100000in}}{\pgfqpoint{5.307240in}{3.397500in}}%
\pgfusepath{clip}%
\pgfsetrectcap%
\pgfsetroundjoin%
\pgfsetlinewidth{1.505625pt}%
\definecolor{currentstroke}{rgb}{0.678431,1.000000,0.184314}%
\pgfsetstrokecolor{currentstroke}%
\pgfsetstrokeopacity{0.500000}%
\pgfsetdash{}{0pt}%
\pgfpathmoveto{\pgfqpoint{1.824166in}{1.748691in}}%
\pgfusepath{stroke}%
\end{pgfscope}%
\begin{pgfscope}%
\pgfpathrectangle{\pgfqpoint{0.100000in}{0.100000in}}{\pgfqpoint{5.307240in}{3.397500in}}%
\pgfusepath{clip}%
\pgfsetbuttcap%
\pgfsetroundjoin%
\definecolor{currentfill}{rgb}{0.678431,1.000000,0.184314}%
\pgfsetfillcolor{currentfill}%
\pgfsetfillopacity{0.500000}%
\pgfsetlinewidth{0.250937pt}%
\definecolor{currentstroke}{rgb}{0.000000,0.000000,0.000000}%
\pgfsetstrokecolor{currentstroke}%
\pgfsetstrokeopacity{0.500000}%
\pgfsetdash{}{0pt}%
\pgfsys@defobject{currentmarker}{\pgfqpoint{-0.058333in}{-0.058333in}}{\pgfqpoint{0.058333in}{0.058333in}}{%
\pgfpathmoveto{\pgfqpoint{0.000000in}{-0.058333in}}%
\pgfpathcurveto{\pgfqpoint{0.015470in}{-0.058333in}}{\pgfqpoint{0.030309in}{-0.052187in}}{\pgfqpoint{0.041248in}{-0.041248in}}%
\pgfpathcurveto{\pgfqpoint{0.052187in}{-0.030309in}}{\pgfqpoint{0.058333in}{-0.015470in}}{\pgfqpoint{0.058333in}{0.000000in}}%
\pgfpathcurveto{\pgfqpoint{0.058333in}{0.015470in}}{\pgfqpoint{0.052187in}{0.030309in}}{\pgfqpoint{0.041248in}{0.041248in}}%
\pgfpathcurveto{\pgfqpoint{0.030309in}{0.052187in}}{\pgfqpoint{0.015470in}{0.058333in}}{\pgfqpoint{0.000000in}{0.058333in}}%
\pgfpathcurveto{\pgfqpoint{-0.015470in}{0.058333in}}{\pgfqpoint{-0.030309in}{0.052187in}}{\pgfqpoint{-0.041248in}{0.041248in}}%
\pgfpathcurveto{\pgfqpoint{-0.052187in}{0.030309in}}{\pgfqpoint{-0.058333in}{0.015470in}}{\pgfqpoint{-0.058333in}{0.000000in}}%
\pgfpathcurveto{\pgfqpoint{-0.058333in}{-0.015470in}}{\pgfqpoint{-0.052187in}{-0.030309in}}{\pgfqpoint{-0.041248in}{-0.041248in}}%
\pgfpathcurveto{\pgfqpoint{-0.030309in}{-0.052187in}}{\pgfqpoint{-0.015470in}{-0.058333in}}{\pgfqpoint{0.000000in}{-0.058333in}}%
\pgfpathclose%
\pgfusepath{stroke,fill}%
}%
\begin{pgfscope}%
\pgfsys@transformshift{1.824166in}{1.748691in}%
\pgfsys@useobject{currentmarker}{}%
\end{pgfscope}%
\end{pgfscope}%
\begin{pgfscope}%
\pgfpathrectangle{\pgfqpoint{0.100000in}{0.100000in}}{\pgfqpoint{5.307240in}{3.397500in}}%
\pgfusepath{clip}%
\pgfsetrectcap%
\pgfsetroundjoin%
\pgfsetlinewidth{1.505625pt}%
\definecolor{currentstroke}{rgb}{0.678431,1.000000,0.184314}%
\pgfsetstrokecolor{currentstroke}%
\pgfsetstrokeopacity{0.500000}%
\pgfsetdash{}{0pt}%
\pgfpathmoveto{\pgfqpoint{1.889582in}{1.219625in}}%
\pgfusepath{stroke}%
\end{pgfscope}%
\begin{pgfscope}%
\pgfpathrectangle{\pgfqpoint{0.100000in}{0.100000in}}{\pgfqpoint{5.307240in}{3.397500in}}%
\pgfusepath{clip}%
\pgfsetbuttcap%
\pgfsetroundjoin%
\definecolor{currentfill}{rgb}{0.678431,1.000000,0.184314}%
\pgfsetfillcolor{currentfill}%
\pgfsetfillopacity{0.500000}%
\pgfsetlinewidth{0.250937pt}%
\definecolor{currentstroke}{rgb}{0.000000,0.000000,0.000000}%
\pgfsetstrokecolor{currentstroke}%
\pgfsetstrokeopacity{0.500000}%
\pgfsetdash{}{0pt}%
\pgfsys@defobject{currentmarker}{\pgfqpoint{-0.042361in}{-0.042361in}}{\pgfqpoint{0.042361in}{0.042361in}}{%
\pgfpathmoveto{\pgfqpoint{0.000000in}{-0.042361in}}%
\pgfpathcurveto{\pgfqpoint{0.011234in}{-0.042361in}}{\pgfqpoint{0.022010in}{-0.037898in}}{\pgfqpoint{0.029954in}{-0.029954in}}%
\pgfpathcurveto{\pgfqpoint{0.037898in}{-0.022010in}}{\pgfqpoint{0.042361in}{-0.011234in}}{\pgfqpoint{0.042361in}{0.000000in}}%
\pgfpathcurveto{\pgfqpoint{0.042361in}{0.011234in}}{\pgfqpoint{0.037898in}{0.022010in}}{\pgfqpoint{0.029954in}{0.029954in}}%
\pgfpathcurveto{\pgfqpoint{0.022010in}{0.037898in}}{\pgfqpoint{0.011234in}{0.042361in}}{\pgfqpoint{0.000000in}{0.042361in}}%
\pgfpathcurveto{\pgfqpoint{-0.011234in}{0.042361in}}{\pgfqpoint{-0.022010in}{0.037898in}}{\pgfqpoint{-0.029954in}{0.029954in}}%
\pgfpathcurveto{\pgfqpoint{-0.037898in}{0.022010in}}{\pgfqpoint{-0.042361in}{0.011234in}}{\pgfqpoint{-0.042361in}{0.000000in}}%
\pgfpathcurveto{\pgfqpoint{-0.042361in}{-0.011234in}}{\pgfqpoint{-0.037898in}{-0.022010in}}{\pgfqpoint{-0.029954in}{-0.029954in}}%
\pgfpathcurveto{\pgfqpoint{-0.022010in}{-0.037898in}}{\pgfqpoint{-0.011234in}{-0.042361in}}{\pgfqpoint{0.000000in}{-0.042361in}}%
\pgfpathclose%
\pgfusepath{stroke,fill}%
}%
\begin{pgfscope}%
\pgfsys@transformshift{1.889582in}{1.219625in}%
\pgfsys@useobject{currentmarker}{}%
\end{pgfscope}%
\end{pgfscope}%
\begin{pgfscope}%
\pgfpathrectangle{\pgfqpoint{0.100000in}{0.100000in}}{\pgfqpoint{5.307240in}{3.397500in}}%
\pgfusepath{clip}%
\pgfsetrectcap%
\pgfsetroundjoin%
\pgfsetlinewidth{1.505625pt}%
\definecolor{currentstroke}{rgb}{0.678431,1.000000,0.184314}%
\pgfsetstrokecolor{currentstroke}%
\pgfsetstrokeopacity{0.500000}%
\pgfsetdash{}{0pt}%
\pgfpathmoveto{\pgfqpoint{2.019488in}{1.600067in}}%
\pgfusepath{stroke}%
\end{pgfscope}%
\begin{pgfscope}%
\pgfpathrectangle{\pgfqpoint{0.100000in}{0.100000in}}{\pgfqpoint{5.307240in}{3.397500in}}%
\pgfusepath{clip}%
\pgfsetbuttcap%
\pgfsetroundjoin%
\definecolor{currentfill}{rgb}{0.678431,1.000000,0.184314}%
\pgfsetfillcolor{currentfill}%
\pgfsetfillopacity{0.500000}%
\pgfsetlinewidth{0.250937pt}%
\definecolor{currentstroke}{rgb}{0.000000,0.000000,0.000000}%
\pgfsetstrokecolor{currentstroke}%
\pgfsetstrokeopacity{0.500000}%
\pgfsetdash{}{0pt}%
\pgfsys@defobject{currentmarker}{\pgfqpoint{-0.059722in}{-0.059722in}}{\pgfqpoint{0.059722in}{0.059722in}}{%
\pgfpathmoveto{\pgfqpoint{0.000000in}{-0.059722in}}%
\pgfpathcurveto{\pgfqpoint{0.015839in}{-0.059722in}}{\pgfqpoint{0.031030in}{-0.053430in}}{\pgfqpoint{0.042230in}{-0.042230in}}%
\pgfpathcurveto{\pgfqpoint{0.053430in}{-0.031030in}}{\pgfqpoint{0.059722in}{-0.015839in}}{\pgfqpoint{0.059722in}{0.000000in}}%
\pgfpathcurveto{\pgfqpoint{0.059722in}{0.015839in}}{\pgfqpoint{0.053430in}{0.031030in}}{\pgfqpoint{0.042230in}{0.042230in}}%
\pgfpathcurveto{\pgfqpoint{0.031030in}{0.053430in}}{\pgfqpoint{0.015839in}{0.059722in}}{\pgfqpoint{0.000000in}{0.059722in}}%
\pgfpathcurveto{\pgfqpoint{-0.015839in}{0.059722in}}{\pgfqpoint{-0.031030in}{0.053430in}}{\pgfqpoint{-0.042230in}{0.042230in}}%
\pgfpathcurveto{\pgfqpoint{-0.053430in}{0.031030in}}{\pgfqpoint{-0.059722in}{0.015839in}}{\pgfqpoint{-0.059722in}{0.000000in}}%
\pgfpathcurveto{\pgfqpoint{-0.059722in}{-0.015839in}}{\pgfqpoint{-0.053430in}{-0.031030in}}{\pgfqpoint{-0.042230in}{-0.042230in}}%
\pgfpathcurveto{\pgfqpoint{-0.031030in}{-0.053430in}}{\pgfqpoint{-0.015839in}{-0.059722in}}{\pgfqpoint{0.000000in}{-0.059722in}}%
\pgfpathclose%
\pgfusepath{stroke,fill}%
}%
\begin{pgfscope}%
\pgfsys@transformshift{2.019488in}{1.600067in}%
\pgfsys@useobject{currentmarker}{}%
\end{pgfscope}%
\end{pgfscope}%
\begin{pgfscope}%
\pgfpathrectangle{\pgfqpoint{0.100000in}{0.100000in}}{\pgfqpoint{5.307240in}{3.397500in}}%
\pgfusepath{clip}%
\pgfsetrectcap%
\pgfsetroundjoin%
\pgfsetlinewidth{1.505625pt}%
\definecolor{currentstroke}{rgb}{0.678431,1.000000,0.184314}%
\pgfsetstrokecolor{currentstroke}%
\pgfsetstrokeopacity{0.500000}%
\pgfsetdash{}{0pt}%
\pgfpathmoveto{\pgfqpoint{4.855044in}{2.559744in}}%
\pgfusepath{stroke}%
\end{pgfscope}%
\begin{pgfscope}%
\pgfpathrectangle{\pgfqpoint{0.100000in}{0.100000in}}{\pgfqpoint{5.307240in}{3.397500in}}%
\pgfusepath{clip}%
\pgfsetbuttcap%
\pgfsetroundjoin%
\definecolor{currentfill}{rgb}{0.678431,1.000000,0.184314}%
\pgfsetfillcolor{currentfill}%
\pgfsetfillopacity{0.500000}%
\pgfsetlinewidth{0.250937pt}%
\definecolor{currentstroke}{rgb}{0.000000,0.000000,0.000000}%
\pgfsetstrokecolor{currentstroke}%
\pgfsetstrokeopacity{0.500000}%
\pgfsetdash{}{0pt}%
\pgfsys@defobject{currentmarker}{\pgfqpoint{-0.063889in}{-0.063889in}}{\pgfqpoint{0.063889in}{0.063889in}}{%
\pgfpathmoveto{\pgfqpoint{0.000000in}{-0.063889in}}%
\pgfpathcurveto{\pgfqpoint{0.016944in}{-0.063889in}}{\pgfqpoint{0.033195in}{-0.057157in}}{\pgfqpoint{0.045176in}{-0.045176in}}%
\pgfpathcurveto{\pgfqpoint{0.057157in}{-0.033195in}}{\pgfqpoint{0.063889in}{-0.016944in}}{\pgfqpoint{0.063889in}{0.000000in}}%
\pgfpathcurveto{\pgfqpoint{0.063889in}{0.016944in}}{\pgfqpoint{0.057157in}{0.033195in}}{\pgfqpoint{0.045176in}{0.045176in}}%
\pgfpathcurveto{\pgfqpoint{0.033195in}{0.057157in}}{\pgfqpoint{0.016944in}{0.063889in}}{\pgfqpoint{0.000000in}{0.063889in}}%
\pgfpathcurveto{\pgfqpoint{-0.016944in}{0.063889in}}{\pgfqpoint{-0.033195in}{0.057157in}}{\pgfqpoint{-0.045176in}{0.045176in}}%
\pgfpathcurveto{\pgfqpoint{-0.057157in}{0.033195in}}{\pgfqpoint{-0.063889in}{0.016944in}}{\pgfqpoint{-0.063889in}{0.000000in}}%
\pgfpathcurveto{\pgfqpoint{-0.063889in}{-0.016944in}}{\pgfqpoint{-0.057157in}{-0.033195in}}{\pgfqpoint{-0.045176in}{-0.045176in}}%
\pgfpathcurveto{\pgfqpoint{-0.033195in}{-0.057157in}}{\pgfqpoint{-0.016944in}{-0.063889in}}{\pgfqpoint{0.000000in}{-0.063889in}}%
\pgfpathclose%
\pgfusepath{stroke,fill}%
}%
\begin{pgfscope}%
\pgfsys@transformshift{4.855044in}{2.559744in}%
\pgfsys@useobject{currentmarker}{}%
\end{pgfscope}%
\end{pgfscope}%
\begin{pgfscope}%
\pgfpathrectangle{\pgfqpoint{0.100000in}{0.100000in}}{\pgfqpoint{5.307240in}{3.397500in}}%
\pgfusepath{clip}%
\pgfsetrectcap%
\pgfsetroundjoin%
\pgfsetlinewidth{1.505625pt}%
\definecolor{currentstroke}{rgb}{0.678431,1.000000,0.184314}%
\pgfsetstrokecolor{currentstroke}%
\pgfsetstrokeopacity{0.500000}%
\pgfsetdash{}{0pt}%
\pgfpathmoveto{\pgfqpoint{4.688046in}{2.455916in}}%
\pgfusepath{stroke}%
\end{pgfscope}%
\begin{pgfscope}%
\pgfpathrectangle{\pgfqpoint{0.100000in}{0.100000in}}{\pgfqpoint{5.307240in}{3.397500in}}%
\pgfusepath{clip}%
\pgfsetbuttcap%
\pgfsetroundjoin%
\definecolor{currentfill}{rgb}{0.678431,1.000000,0.184314}%
\pgfsetfillcolor{currentfill}%
\pgfsetfillopacity{0.500000}%
\pgfsetlinewidth{0.250937pt}%
\definecolor{currentstroke}{rgb}{0.000000,0.000000,0.000000}%
\pgfsetstrokecolor{currentstroke}%
\pgfsetstrokeopacity{0.500000}%
\pgfsetdash{}{0pt}%
\pgfsys@defobject{currentmarker}{\pgfqpoint{-0.075000in}{-0.075000in}}{\pgfqpoint{0.075000in}{0.075000in}}{%
\pgfpathmoveto{\pgfqpoint{0.000000in}{-0.075000in}}%
\pgfpathcurveto{\pgfqpoint{0.019890in}{-0.075000in}}{\pgfqpoint{0.038968in}{-0.067098in}}{\pgfqpoint{0.053033in}{-0.053033in}}%
\pgfpathcurveto{\pgfqpoint{0.067098in}{-0.038968in}}{\pgfqpoint{0.075000in}{-0.019890in}}{\pgfqpoint{0.075000in}{0.000000in}}%
\pgfpathcurveto{\pgfqpoint{0.075000in}{0.019890in}}{\pgfqpoint{0.067098in}{0.038968in}}{\pgfqpoint{0.053033in}{0.053033in}}%
\pgfpathcurveto{\pgfqpoint{0.038968in}{0.067098in}}{\pgfqpoint{0.019890in}{0.075000in}}{\pgfqpoint{0.000000in}{0.075000in}}%
\pgfpathcurveto{\pgfqpoint{-0.019890in}{0.075000in}}{\pgfqpoint{-0.038968in}{0.067098in}}{\pgfqpoint{-0.053033in}{0.053033in}}%
\pgfpathcurveto{\pgfqpoint{-0.067098in}{0.038968in}}{\pgfqpoint{-0.075000in}{0.019890in}}{\pgfqpoint{-0.075000in}{0.000000in}}%
\pgfpathcurveto{\pgfqpoint{-0.075000in}{-0.019890in}}{\pgfqpoint{-0.067098in}{-0.038968in}}{\pgfqpoint{-0.053033in}{-0.053033in}}%
\pgfpathcurveto{\pgfqpoint{-0.038968in}{-0.067098in}}{\pgfqpoint{-0.019890in}{-0.075000in}}{\pgfqpoint{0.000000in}{-0.075000in}}%
\pgfpathclose%
\pgfusepath{stroke,fill}%
}%
\begin{pgfscope}%
\pgfsys@transformshift{4.688046in}{2.455916in}%
\pgfsys@useobject{currentmarker}{}%
\end{pgfscope}%
\end{pgfscope}%
\begin{pgfscope}%
\pgfpathrectangle{\pgfqpoint{0.100000in}{0.100000in}}{\pgfqpoint{5.307240in}{3.397500in}}%
\pgfusepath{clip}%
\pgfsetrectcap%
\pgfsetroundjoin%
\pgfsetlinewidth{1.505625pt}%
\definecolor{currentstroke}{rgb}{0.678431,1.000000,0.184314}%
\pgfsetstrokecolor{currentstroke}%
\pgfsetstrokeopacity{0.500000}%
\pgfsetdash{}{0pt}%
\pgfpathmoveto{\pgfqpoint{4.420581in}{2.497162in}}%
\pgfusepath{stroke}%
\end{pgfscope}%
\begin{pgfscope}%
\pgfpathrectangle{\pgfqpoint{0.100000in}{0.100000in}}{\pgfqpoint{5.307240in}{3.397500in}}%
\pgfusepath{clip}%
\pgfsetbuttcap%
\pgfsetroundjoin%
\definecolor{currentfill}{rgb}{0.678431,1.000000,0.184314}%
\pgfsetfillcolor{currentfill}%
\pgfsetfillopacity{0.500000}%
\pgfsetlinewidth{0.250937pt}%
\definecolor{currentstroke}{rgb}{0.000000,0.000000,0.000000}%
\pgfsetstrokecolor{currentstroke}%
\pgfsetstrokeopacity{0.500000}%
\pgfsetdash{}{0pt}%
\pgfsys@defobject{currentmarker}{\pgfqpoint{-0.106250in}{-0.106250in}}{\pgfqpoint{0.106250in}{0.106250in}}{%
\pgfpathmoveto{\pgfqpoint{0.000000in}{-0.106250in}}%
\pgfpathcurveto{\pgfqpoint{0.028178in}{-0.106250in}}{\pgfqpoint{0.055205in}{-0.095055in}}{\pgfqpoint{0.075130in}{-0.075130in}}%
\pgfpathcurveto{\pgfqpoint{0.095055in}{-0.055205in}}{\pgfqpoint{0.106250in}{-0.028178in}}{\pgfqpoint{0.106250in}{0.000000in}}%
\pgfpathcurveto{\pgfqpoint{0.106250in}{0.028178in}}{\pgfqpoint{0.095055in}{0.055205in}}{\pgfqpoint{0.075130in}{0.075130in}}%
\pgfpathcurveto{\pgfqpoint{0.055205in}{0.095055in}}{\pgfqpoint{0.028178in}{0.106250in}}{\pgfqpoint{0.000000in}{0.106250in}}%
\pgfpathcurveto{\pgfqpoint{-0.028178in}{0.106250in}}{\pgfqpoint{-0.055205in}{0.095055in}}{\pgfqpoint{-0.075130in}{0.075130in}}%
\pgfpathcurveto{\pgfqpoint{-0.095055in}{0.055205in}}{\pgfqpoint{-0.106250in}{0.028178in}}{\pgfqpoint{-0.106250in}{0.000000in}}%
\pgfpathcurveto{\pgfqpoint{-0.106250in}{-0.028178in}}{\pgfqpoint{-0.095055in}{-0.055205in}}{\pgfqpoint{-0.075130in}{-0.075130in}}%
\pgfpathcurveto{\pgfqpoint{-0.055205in}{-0.095055in}}{\pgfqpoint{-0.028178in}{-0.106250in}}{\pgfqpoint{0.000000in}{-0.106250in}}%
\pgfpathclose%
\pgfusepath{stroke,fill}%
}%
\begin{pgfscope}%
\pgfsys@transformshift{4.420581in}{2.497162in}%
\pgfsys@useobject{currentmarker}{}%
\end{pgfscope}%
\end{pgfscope}%
\begin{pgfscope}%
\pgfpathrectangle{\pgfqpoint{0.100000in}{0.100000in}}{\pgfqpoint{5.307240in}{3.397500in}}%
\pgfusepath{clip}%
\pgfsetrectcap%
\pgfsetroundjoin%
\pgfsetlinewidth{1.505625pt}%
\definecolor{currentstroke}{rgb}{0.678431,1.000000,0.184314}%
\pgfsetstrokecolor{currentstroke}%
\pgfsetstrokeopacity{0.500000}%
\pgfsetdash{}{0pt}%
\pgfpathmoveto{\pgfqpoint{4.612658in}{2.439381in}}%
\pgfusepath{stroke}%
\end{pgfscope}%
\begin{pgfscope}%
\pgfpathrectangle{\pgfqpoint{0.100000in}{0.100000in}}{\pgfqpoint{5.307240in}{3.397500in}}%
\pgfusepath{clip}%
\pgfsetbuttcap%
\pgfsetroundjoin%
\definecolor{currentfill}{rgb}{0.678431,1.000000,0.184314}%
\pgfsetfillcolor{currentfill}%
\pgfsetfillopacity{0.500000}%
\pgfsetlinewidth{0.250937pt}%
\definecolor{currentstroke}{rgb}{0.000000,0.000000,0.000000}%
\pgfsetstrokecolor{currentstroke}%
\pgfsetstrokeopacity{0.500000}%
\pgfsetdash{}{0pt}%
\pgfsys@defobject{currentmarker}{\pgfqpoint{-0.084028in}{-0.084028in}}{\pgfqpoint{0.084028in}{0.084028in}}{%
\pgfpathmoveto{\pgfqpoint{0.000000in}{-0.084028in}}%
\pgfpathcurveto{\pgfqpoint{0.022284in}{-0.084028in}}{\pgfqpoint{0.043659in}{-0.075174in}}{\pgfqpoint{0.059417in}{-0.059417in}}%
\pgfpathcurveto{\pgfqpoint{0.075174in}{-0.043659in}}{\pgfqpoint{0.084028in}{-0.022284in}}{\pgfqpoint{0.084028in}{0.000000in}}%
\pgfpathcurveto{\pgfqpoint{0.084028in}{0.022284in}}{\pgfqpoint{0.075174in}{0.043659in}}{\pgfqpoint{0.059417in}{0.059417in}}%
\pgfpathcurveto{\pgfqpoint{0.043659in}{0.075174in}}{\pgfqpoint{0.022284in}{0.084028in}}{\pgfqpoint{0.000000in}{0.084028in}}%
\pgfpathcurveto{\pgfqpoint{-0.022284in}{0.084028in}}{\pgfqpoint{-0.043659in}{0.075174in}}{\pgfqpoint{-0.059417in}{0.059417in}}%
\pgfpathcurveto{\pgfqpoint{-0.075174in}{0.043659in}}{\pgfqpoint{-0.084028in}{0.022284in}}{\pgfqpoint{-0.084028in}{0.000000in}}%
\pgfpathcurveto{\pgfqpoint{-0.084028in}{-0.022284in}}{\pgfqpoint{-0.075174in}{-0.043659in}}{\pgfqpoint{-0.059417in}{-0.059417in}}%
\pgfpathcurveto{\pgfqpoint{-0.043659in}{-0.075174in}}{\pgfqpoint{-0.022284in}{-0.084028in}}{\pgfqpoint{0.000000in}{-0.084028in}}%
\pgfpathclose%
\pgfusepath{stroke,fill}%
}%
\begin{pgfscope}%
\pgfsys@transformshift{4.612658in}{2.439381in}%
\pgfsys@useobject{currentmarker}{}%
\end{pgfscope}%
\end{pgfscope}%
\begin{pgfscope}%
\pgfpathrectangle{\pgfqpoint{0.100000in}{0.100000in}}{\pgfqpoint{5.307240in}{3.397500in}}%
\pgfusepath{clip}%
\pgfsetrectcap%
\pgfsetroundjoin%
\pgfsetlinewidth{1.505625pt}%
\definecolor{currentstroke}{rgb}{0.678431,1.000000,0.184314}%
\pgfsetstrokecolor{currentstroke}%
\pgfsetstrokeopacity{0.500000}%
\pgfsetdash{}{0pt}%
\pgfpathmoveto{\pgfqpoint{4.846380in}{2.636597in}}%
\pgfusepath{stroke}%
\end{pgfscope}%
\begin{pgfscope}%
\pgfpathrectangle{\pgfqpoint{0.100000in}{0.100000in}}{\pgfqpoint{5.307240in}{3.397500in}}%
\pgfusepath{clip}%
\pgfsetbuttcap%
\pgfsetroundjoin%
\definecolor{currentfill}{rgb}{0.678431,1.000000,0.184314}%
\pgfsetfillcolor{currentfill}%
\pgfsetfillopacity{0.500000}%
\pgfsetlinewidth{0.250937pt}%
\definecolor{currentstroke}{rgb}{0.000000,0.000000,0.000000}%
\pgfsetstrokecolor{currentstroke}%
\pgfsetstrokeopacity{0.500000}%
\pgfsetdash{}{0pt}%
\pgfsys@defobject{currentmarker}{\pgfqpoint{-0.078472in}{-0.078472in}}{\pgfqpoint{0.078472in}{0.078472in}}{%
\pgfpathmoveto{\pgfqpoint{0.000000in}{-0.078472in}}%
\pgfpathcurveto{\pgfqpoint{0.020811in}{-0.078472in}}{\pgfqpoint{0.040773in}{-0.070204in}}{\pgfqpoint{0.055488in}{-0.055488in}}%
\pgfpathcurveto{\pgfqpoint{0.070204in}{-0.040773in}}{\pgfqpoint{0.078472in}{-0.020811in}}{\pgfqpoint{0.078472in}{0.000000in}}%
\pgfpathcurveto{\pgfqpoint{0.078472in}{0.020811in}}{\pgfqpoint{0.070204in}{0.040773in}}{\pgfqpoint{0.055488in}{0.055488in}}%
\pgfpathcurveto{\pgfqpoint{0.040773in}{0.070204in}}{\pgfqpoint{0.020811in}{0.078472in}}{\pgfqpoint{0.000000in}{0.078472in}}%
\pgfpathcurveto{\pgfqpoint{-0.020811in}{0.078472in}}{\pgfqpoint{-0.040773in}{0.070204in}}{\pgfqpoint{-0.055488in}{0.055488in}}%
\pgfpathcurveto{\pgfqpoint{-0.070204in}{0.040773in}}{\pgfqpoint{-0.078472in}{0.020811in}}{\pgfqpoint{-0.078472in}{0.000000in}}%
\pgfpathcurveto{\pgfqpoint{-0.078472in}{-0.020811in}}{\pgfqpoint{-0.070204in}{-0.040773in}}{\pgfqpoint{-0.055488in}{-0.055488in}}%
\pgfpathcurveto{\pgfqpoint{-0.040773in}{-0.070204in}}{\pgfqpoint{-0.020811in}{-0.078472in}}{\pgfqpoint{0.000000in}{-0.078472in}}%
\pgfpathclose%
\pgfusepath{stroke,fill}%
}%
\begin{pgfscope}%
\pgfsys@transformshift{4.846380in}{2.636597in}%
\pgfsys@useobject{currentmarker}{}%
\end{pgfscope}%
\end{pgfscope}%
\begin{pgfscope}%
\pgfpathrectangle{\pgfqpoint{0.100000in}{0.100000in}}{\pgfqpoint{5.307240in}{3.397500in}}%
\pgfusepath{clip}%
\pgfsetrectcap%
\pgfsetroundjoin%
\pgfsetlinewidth{1.505625pt}%
\definecolor{currentstroke}{rgb}{0.678431,1.000000,0.184314}%
\pgfsetstrokecolor{currentstroke}%
\pgfsetstrokeopacity{0.500000}%
\pgfsetdash{}{0pt}%
\pgfpathmoveto{\pgfqpoint{4.630738in}{2.484654in}}%
\pgfusepath{stroke}%
\end{pgfscope}%
\begin{pgfscope}%
\pgfpathrectangle{\pgfqpoint{0.100000in}{0.100000in}}{\pgfqpoint{5.307240in}{3.397500in}}%
\pgfusepath{clip}%
\pgfsetbuttcap%
\pgfsetroundjoin%
\definecolor{currentfill}{rgb}{0.678431,1.000000,0.184314}%
\pgfsetfillcolor{currentfill}%
\pgfsetfillopacity{0.500000}%
\pgfsetlinewidth{0.250937pt}%
\definecolor{currentstroke}{rgb}{0.000000,0.000000,0.000000}%
\pgfsetstrokecolor{currentstroke}%
\pgfsetstrokeopacity{0.500000}%
\pgfsetdash{}{0pt}%
\pgfsys@defobject{currentmarker}{\pgfqpoint{-0.047917in}{-0.047917in}}{\pgfqpoint{0.047917in}{0.047917in}}{%
\pgfpathmoveto{\pgfqpoint{0.000000in}{-0.047917in}}%
\pgfpathcurveto{\pgfqpoint{0.012708in}{-0.047917in}}{\pgfqpoint{0.024897in}{-0.042868in}}{\pgfqpoint{0.033882in}{-0.033882in}}%
\pgfpathcurveto{\pgfqpoint{0.042868in}{-0.024897in}}{\pgfqpoint{0.047917in}{-0.012708in}}{\pgfqpoint{0.047917in}{0.000000in}}%
\pgfpathcurveto{\pgfqpoint{0.047917in}{0.012708in}}{\pgfqpoint{0.042868in}{0.024897in}}{\pgfqpoint{0.033882in}{0.033882in}}%
\pgfpathcurveto{\pgfqpoint{0.024897in}{0.042868in}}{\pgfqpoint{0.012708in}{0.047917in}}{\pgfqpoint{0.000000in}{0.047917in}}%
\pgfpathcurveto{\pgfqpoint{-0.012708in}{0.047917in}}{\pgfqpoint{-0.024897in}{0.042868in}}{\pgfqpoint{-0.033882in}{0.033882in}}%
\pgfpathcurveto{\pgfqpoint{-0.042868in}{0.024897in}}{\pgfqpoint{-0.047917in}{0.012708in}}{\pgfqpoint{-0.047917in}{0.000000in}}%
\pgfpathcurveto{\pgfqpoint{-0.047917in}{-0.012708in}}{\pgfqpoint{-0.042868in}{-0.024897in}}{\pgfqpoint{-0.033882in}{-0.033882in}}%
\pgfpathcurveto{\pgfqpoint{-0.024897in}{-0.042868in}}{\pgfqpoint{-0.012708in}{-0.047917in}}{\pgfqpoint{0.000000in}{-0.047917in}}%
\pgfpathclose%
\pgfusepath{stroke,fill}%
}%
\begin{pgfscope}%
\pgfsys@transformshift{4.630738in}{2.484654in}%
\pgfsys@useobject{currentmarker}{}%
\end{pgfscope}%
\end{pgfscope}%
\begin{pgfscope}%
\pgfpathrectangle{\pgfqpoint{0.100000in}{0.100000in}}{\pgfqpoint{5.307240in}{3.397500in}}%
\pgfusepath{clip}%
\pgfsetrectcap%
\pgfsetroundjoin%
\pgfsetlinewidth{1.505625pt}%
\definecolor{currentstroke}{rgb}{0.678431,1.000000,0.184314}%
\pgfsetstrokecolor{currentstroke}%
\pgfsetstrokeopacity{0.500000}%
\pgfsetdash{}{0pt}%
\pgfpathmoveto{\pgfqpoint{4.853681in}{2.472986in}}%
\pgfusepath{stroke}%
\end{pgfscope}%
\begin{pgfscope}%
\pgfpathrectangle{\pgfqpoint{0.100000in}{0.100000in}}{\pgfqpoint{5.307240in}{3.397500in}}%
\pgfusepath{clip}%
\pgfsetbuttcap%
\pgfsetroundjoin%
\definecolor{currentfill}{rgb}{0.678431,1.000000,0.184314}%
\pgfsetfillcolor{currentfill}%
\pgfsetfillopacity{0.500000}%
\pgfsetlinewidth{0.250937pt}%
\definecolor{currentstroke}{rgb}{0.000000,0.000000,0.000000}%
\pgfsetstrokecolor{currentstroke}%
\pgfsetstrokeopacity{0.500000}%
\pgfsetdash{}{0pt}%
\pgfsys@defobject{currentmarker}{\pgfqpoint{-0.078472in}{-0.078472in}}{\pgfqpoint{0.078472in}{0.078472in}}{%
\pgfpathmoveto{\pgfqpoint{0.000000in}{-0.078472in}}%
\pgfpathcurveto{\pgfqpoint{0.020811in}{-0.078472in}}{\pgfqpoint{0.040773in}{-0.070204in}}{\pgfqpoint{0.055488in}{-0.055488in}}%
\pgfpathcurveto{\pgfqpoint{0.070204in}{-0.040773in}}{\pgfqpoint{0.078472in}{-0.020811in}}{\pgfqpoint{0.078472in}{0.000000in}}%
\pgfpathcurveto{\pgfqpoint{0.078472in}{0.020811in}}{\pgfqpoint{0.070204in}{0.040773in}}{\pgfqpoint{0.055488in}{0.055488in}}%
\pgfpathcurveto{\pgfqpoint{0.040773in}{0.070204in}}{\pgfqpoint{0.020811in}{0.078472in}}{\pgfqpoint{0.000000in}{0.078472in}}%
\pgfpathcurveto{\pgfqpoint{-0.020811in}{0.078472in}}{\pgfqpoint{-0.040773in}{0.070204in}}{\pgfqpoint{-0.055488in}{0.055488in}}%
\pgfpathcurveto{\pgfqpoint{-0.070204in}{0.040773in}}{\pgfqpoint{-0.078472in}{0.020811in}}{\pgfqpoint{-0.078472in}{0.000000in}}%
\pgfpathcurveto{\pgfqpoint{-0.078472in}{-0.020811in}}{\pgfqpoint{-0.070204in}{-0.040773in}}{\pgfqpoint{-0.055488in}{-0.055488in}}%
\pgfpathcurveto{\pgfqpoint{-0.040773in}{-0.070204in}}{\pgfqpoint{-0.020811in}{-0.078472in}}{\pgfqpoint{0.000000in}{-0.078472in}}%
\pgfpathclose%
\pgfusepath{stroke,fill}%
}%
\begin{pgfscope}%
\pgfsys@transformshift{4.853681in}{2.472986in}%
\pgfsys@useobject{currentmarker}{}%
\end{pgfscope}%
\end{pgfscope}%
\begin{pgfscope}%
\pgfpathrectangle{\pgfqpoint{0.100000in}{0.100000in}}{\pgfqpoint{5.307240in}{3.397500in}}%
\pgfusepath{clip}%
\pgfsetrectcap%
\pgfsetroundjoin%
\pgfsetlinewidth{1.505625pt}%
\definecolor{currentstroke}{rgb}{0.678431,1.000000,0.184314}%
\pgfsetstrokecolor{currentstroke}%
\pgfsetstrokeopacity{0.500000}%
\pgfsetdash{}{0pt}%
\pgfpathmoveto{\pgfqpoint{4.886959in}{2.337611in}}%
\pgfusepath{stroke}%
\end{pgfscope}%
\begin{pgfscope}%
\pgfpathrectangle{\pgfqpoint{0.100000in}{0.100000in}}{\pgfqpoint{5.307240in}{3.397500in}}%
\pgfusepath{clip}%
\pgfsetbuttcap%
\pgfsetroundjoin%
\definecolor{currentfill}{rgb}{0.678431,1.000000,0.184314}%
\pgfsetfillcolor{currentfill}%
\pgfsetfillopacity{0.500000}%
\pgfsetlinewidth{0.250937pt}%
\definecolor{currentstroke}{rgb}{0.000000,0.000000,0.000000}%
\pgfsetstrokecolor{currentstroke}%
\pgfsetstrokeopacity{0.500000}%
\pgfsetdash{}{0pt}%
\pgfsys@defobject{currentmarker}{\pgfqpoint{-0.082639in}{-0.082639in}}{\pgfqpoint{0.082639in}{0.082639in}}{%
\pgfpathmoveto{\pgfqpoint{0.000000in}{-0.082639in}}%
\pgfpathcurveto{\pgfqpoint{0.021916in}{-0.082639in}}{\pgfqpoint{0.042938in}{-0.073932in}}{\pgfqpoint{0.058435in}{-0.058435in}}%
\pgfpathcurveto{\pgfqpoint{0.073932in}{-0.042938in}}{\pgfqpoint{0.082639in}{-0.021916in}}{\pgfqpoint{0.082639in}{0.000000in}}%
\pgfpathcurveto{\pgfqpoint{0.082639in}{0.021916in}}{\pgfqpoint{0.073932in}{0.042938in}}{\pgfqpoint{0.058435in}{0.058435in}}%
\pgfpathcurveto{\pgfqpoint{0.042938in}{0.073932in}}{\pgfqpoint{0.021916in}{0.082639in}}{\pgfqpoint{0.000000in}{0.082639in}}%
\pgfpathcurveto{\pgfqpoint{-0.021916in}{0.082639in}}{\pgfqpoint{-0.042938in}{0.073932in}}{\pgfqpoint{-0.058435in}{0.058435in}}%
\pgfpathcurveto{\pgfqpoint{-0.073932in}{0.042938in}}{\pgfqpoint{-0.082639in}{0.021916in}}{\pgfqpoint{-0.082639in}{0.000000in}}%
\pgfpathcurveto{\pgfqpoint{-0.082639in}{-0.021916in}}{\pgfqpoint{-0.073932in}{-0.042938in}}{\pgfqpoint{-0.058435in}{-0.058435in}}%
\pgfpathcurveto{\pgfqpoint{-0.042938in}{-0.073932in}}{\pgfqpoint{-0.021916in}{-0.082639in}}{\pgfqpoint{0.000000in}{-0.082639in}}%
\pgfpathclose%
\pgfusepath{stroke,fill}%
}%
\begin{pgfscope}%
\pgfsys@transformshift{4.886959in}{2.337611in}%
\pgfsys@useobject{currentmarker}{}%
\end{pgfscope}%
\end{pgfscope}%
\begin{pgfscope}%
\pgfpathrectangle{\pgfqpoint{0.100000in}{0.100000in}}{\pgfqpoint{5.307240in}{3.397500in}}%
\pgfusepath{clip}%
\pgfsetrectcap%
\pgfsetroundjoin%
\pgfsetlinewidth{1.505625pt}%
\definecolor{currentstroke}{rgb}{0.678431,1.000000,0.184314}%
\pgfsetstrokecolor{currentstroke}%
\pgfsetstrokeopacity{0.500000}%
\pgfsetdash{}{0pt}%
\pgfpathmoveto{\pgfqpoint{4.520592in}{2.547883in}}%
\pgfusepath{stroke}%
\end{pgfscope}%
\begin{pgfscope}%
\pgfpathrectangle{\pgfqpoint{0.100000in}{0.100000in}}{\pgfqpoint{5.307240in}{3.397500in}}%
\pgfusepath{clip}%
\pgfsetbuttcap%
\pgfsetroundjoin%
\definecolor{currentfill}{rgb}{0.678431,1.000000,0.184314}%
\pgfsetfillcolor{currentfill}%
\pgfsetfillopacity{0.500000}%
\pgfsetlinewidth{0.250937pt}%
\definecolor{currentstroke}{rgb}{0.000000,0.000000,0.000000}%
\pgfsetstrokecolor{currentstroke}%
\pgfsetstrokeopacity{0.500000}%
\pgfsetdash{}{0pt}%
\pgfsys@defobject{currentmarker}{\pgfqpoint{-0.077778in}{-0.077778in}}{\pgfqpoint{0.077778in}{0.077778in}}{%
\pgfpathmoveto{\pgfqpoint{0.000000in}{-0.077778in}}%
\pgfpathcurveto{\pgfqpoint{0.020627in}{-0.077778in}}{\pgfqpoint{0.040412in}{-0.069583in}}{\pgfqpoint{0.054997in}{-0.054997in}}%
\pgfpathcurveto{\pgfqpoint{0.069583in}{-0.040412in}}{\pgfqpoint{0.077778in}{-0.020627in}}{\pgfqpoint{0.077778in}{0.000000in}}%
\pgfpathcurveto{\pgfqpoint{0.077778in}{0.020627in}}{\pgfqpoint{0.069583in}{0.040412in}}{\pgfqpoint{0.054997in}{0.054997in}}%
\pgfpathcurveto{\pgfqpoint{0.040412in}{0.069583in}}{\pgfqpoint{0.020627in}{0.077778in}}{\pgfqpoint{0.000000in}{0.077778in}}%
\pgfpathcurveto{\pgfqpoint{-0.020627in}{0.077778in}}{\pgfqpoint{-0.040412in}{0.069583in}}{\pgfqpoint{-0.054997in}{0.054997in}}%
\pgfpathcurveto{\pgfqpoint{-0.069583in}{0.040412in}}{\pgfqpoint{-0.077778in}{0.020627in}}{\pgfqpoint{-0.077778in}{0.000000in}}%
\pgfpathcurveto{\pgfqpoint{-0.077778in}{-0.020627in}}{\pgfqpoint{-0.069583in}{-0.040412in}}{\pgfqpoint{-0.054997in}{-0.054997in}}%
\pgfpathcurveto{\pgfqpoint{-0.040412in}{-0.069583in}}{\pgfqpoint{-0.020627in}{-0.077778in}}{\pgfqpoint{0.000000in}{-0.077778in}}%
\pgfpathclose%
\pgfusepath{stroke,fill}%
}%
\begin{pgfscope}%
\pgfsys@transformshift{4.520592in}{2.547883in}%
\pgfsys@useobject{currentmarker}{}%
\end{pgfscope}%
\end{pgfscope}%
\begin{pgfscope}%
\pgfpathrectangle{\pgfqpoint{0.100000in}{0.100000in}}{\pgfqpoint{5.307240in}{3.397500in}}%
\pgfusepath{clip}%
\pgfsetrectcap%
\pgfsetroundjoin%
\pgfsetlinewidth{1.505625pt}%
\definecolor{currentstroke}{rgb}{0.678431,1.000000,0.184314}%
\pgfsetstrokecolor{currentstroke}%
\pgfsetstrokeopacity{0.500000}%
\pgfsetdash{}{0pt}%
\pgfpathmoveto{\pgfqpoint{4.645532in}{2.560150in}}%
\pgfusepath{stroke}%
\end{pgfscope}%
\begin{pgfscope}%
\pgfpathrectangle{\pgfqpoint{0.100000in}{0.100000in}}{\pgfqpoint{5.307240in}{3.397500in}}%
\pgfusepath{clip}%
\pgfsetbuttcap%
\pgfsetroundjoin%
\definecolor{currentfill}{rgb}{0.678431,1.000000,0.184314}%
\pgfsetfillcolor{currentfill}%
\pgfsetfillopacity{0.500000}%
\pgfsetlinewidth{0.250937pt}%
\definecolor{currentstroke}{rgb}{0.000000,0.000000,0.000000}%
\pgfsetstrokecolor{currentstroke}%
\pgfsetstrokeopacity{0.500000}%
\pgfsetdash{}{0pt}%
\pgfsys@defobject{currentmarker}{\pgfqpoint{-0.086806in}{-0.086806in}}{\pgfqpoint{0.086806in}{0.086806in}}{%
\pgfpathmoveto{\pgfqpoint{0.000000in}{-0.086806in}}%
\pgfpathcurveto{\pgfqpoint{0.023021in}{-0.086806in}}{\pgfqpoint{0.045102in}{-0.077659in}}{\pgfqpoint{0.061381in}{-0.061381in}}%
\pgfpathcurveto{\pgfqpoint{0.077659in}{-0.045102in}}{\pgfqpoint{0.086806in}{-0.023021in}}{\pgfqpoint{0.086806in}{0.000000in}}%
\pgfpathcurveto{\pgfqpoint{0.086806in}{0.023021in}}{\pgfqpoint{0.077659in}{0.045102in}}{\pgfqpoint{0.061381in}{0.061381in}}%
\pgfpathcurveto{\pgfqpoint{0.045102in}{0.077659in}}{\pgfqpoint{0.023021in}{0.086806in}}{\pgfqpoint{0.000000in}{0.086806in}}%
\pgfpathcurveto{\pgfqpoint{-0.023021in}{0.086806in}}{\pgfqpoint{-0.045102in}{0.077659in}}{\pgfqpoint{-0.061381in}{0.061381in}}%
\pgfpathcurveto{\pgfqpoint{-0.077659in}{0.045102in}}{\pgfqpoint{-0.086806in}{0.023021in}}{\pgfqpoint{-0.086806in}{0.000000in}}%
\pgfpathcurveto{\pgfqpoint{-0.086806in}{-0.023021in}}{\pgfqpoint{-0.077659in}{-0.045102in}}{\pgfqpoint{-0.061381in}{-0.061381in}}%
\pgfpathcurveto{\pgfqpoint{-0.045102in}{-0.077659in}}{\pgfqpoint{-0.023021in}{-0.086806in}}{\pgfqpoint{0.000000in}{-0.086806in}}%
\pgfpathclose%
\pgfusepath{stroke,fill}%
}%
\begin{pgfscope}%
\pgfsys@transformshift{4.645532in}{2.560150in}%
\pgfsys@useobject{currentmarker}{}%
\end{pgfscope}%
\end{pgfscope}%
\begin{pgfscope}%
\pgfpathrectangle{\pgfqpoint{0.100000in}{0.100000in}}{\pgfqpoint{5.307240in}{3.397500in}}%
\pgfusepath{clip}%
\pgfsetrectcap%
\pgfsetroundjoin%
\pgfsetlinewidth{1.505625pt}%
\definecolor{currentstroke}{rgb}{0.678431,1.000000,0.184314}%
\pgfsetstrokecolor{currentstroke}%
\pgfsetstrokeopacity{0.500000}%
\pgfsetdash{}{0pt}%
\pgfpathmoveto{\pgfqpoint{4.720353in}{2.582584in}}%
\pgfusepath{stroke}%
\end{pgfscope}%
\begin{pgfscope}%
\pgfpathrectangle{\pgfqpoint{0.100000in}{0.100000in}}{\pgfqpoint{5.307240in}{3.397500in}}%
\pgfusepath{clip}%
\pgfsetbuttcap%
\pgfsetroundjoin%
\definecolor{currentfill}{rgb}{0.678431,1.000000,0.184314}%
\pgfsetfillcolor{currentfill}%
\pgfsetfillopacity{0.500000}%
\pgfsetlinewidth{0.250937pt}%
\definecolor{currentstroke}{rgb}{0.000000,0.000000,0.000000}%
\pgfsetstrokecolor{currentstroke}%
\pgfsetstrokeopacity{0.500000}%
\pgfsetdash{}{0pt}%
\pgfsys@defobject{currentmarker}{\pgfqpoint{-0.076389in}{-0.076389in}}{\pgfqpoint{0.076389in}{0.076389in}}{%
\pgfpathmoveto{\pgfqpoint{0.000000in}{-0.076389in}}%
\pgfpathcurveto{\pgfqpoint{0.020259in}{-0.076389in}}{\pgfqpoint{0.039690in}{-0.068340in}}{\pgfqpoint{0.054015in}{-0.054015in}}%
\pgfpathcurveto{\pgfqpoint{0.068340in}{-0.039690in}}{\pgfqpoint{0.076389in}{-0.020259in}}{\pgfqpoint{0.076389in}{0.000000in}}%
\pgfpathcurveto{\pgfqpoint{0.076389in}{0.020259in}}{\pgfqpoint{0.068340in}{0.039690in}}{\pgfqpoint{0.054015in}{0.054015in}}%
\pgfpathcurveto{\pgfqpoint{0.039690in}{0.068340in}}{\pgfqpoint{0.020259in}{0.076389in}}{\pgfqpoint{0.000000in}{0.076389in}}%
\pgfpathcurveto{\pgfqpoint{-0.020259in}{0.076389in}}{\pgfqpoint{-0.039690in}{0.068340in}}{\pgfqpoint{-0.054015in}{0.054015in}}%
\pgfpathcurveto{\pgfqpoint{-0.068340in}{0.039690in}}{\pgfqpoint{-0.076389in}{0.020259in}}{\pgfqpoint{-0.076389in}{0.000000in}}%
\pgfpathcurveto{\pgfqpoint{-0.076389in}{-0.020259in}}{\pgfqpoint{-0.068340in}{-0.039690in}}{\pgfqpoint{-0.054015in}{-0.054015in}}%
\pgfpathcurveto{\pgfqpoint{-0.039690in}{-0.068340in}}{\pgfqpoint{-0.020259in}{-0.076389in}}{\pgfqpoint{0.000000in}{-0.076389in}}%
\pgfpathclose%
\pgfusepath{stroke,fill}%
}%
\begin{pgfscope}%
\pgfsys@transformshift{4.720353in}{2.582584in}%
\pgfsys@useobject{currentmarker}{}%
\end{pgfscope}%
\end{pgfscope}%
\begin{pgfscope}%
\pgfpathrectangle{\pgfqpoint{0.100000in}{0.100000in}}{\pgfqpoint{5.307240in}{3.397500in}}%
\pgfusepath{clip}%
\pgfsetrectcap%
\pgfsetroundjoin%
\pgfsetlinewidth{1.505625pt}%
\definecolor{currentstroke}{rgb}{0.678431,1.000000,0.184314}%
\pgfsetstrokecolor{currentstroke}%
\pgfsetstrokeopacity{0.500000}%
\pgfsetdash{}{0pt}%
\pgfpathmoveto{\pgfqpoint{4.642718in}{2.670006in}}%
\pgfusepath{stroke}%
\end{pgfscope}%
\begin{pgfscope}%
\pgfpathrectangle{\pgfqpoint{0.100000in}{0.100000in}}{\pgfqpoint{5.307240in}{3.397500in}}%
\pgfusepath{clip}%
\pgfsetbuttcap%
\pgfsetroundjoin%
\definecolor{currentfill}{rgb}{0.678431,1.000000,0.184314}%
\pgfsetfillcolor{currentfill}%
\pgfsetfillopacity{0.500000}%
\pgfsetlinewidth{0.250937pt}%
\definecolor{currentstroke}{rgb}{0.000000,0.000000,0.000000}%
\pgfsetstrokecolor{currentstroke}%
\pgfsetstrokeopacity{0.500000}%
\pgfsetdash{}{0pt}%
\pgfsys@defobject{currentmarker}{\pgfqpoint{-0.079861in}{-0.079861in}}{\pgfqpoint{0.079861in}{0.079861in}}{%
\pgfpathmoveto{\pgfqpoint{0.000000in}{-0.079861in}}%
\pgfpathcurveto{\pgfqpoint{0.021179in}{-0.079861in}}{\pgfqpoint{0.041494in}{-0.071446in}}{\pgfqpoint{0.056470in}{-0.056470in}}%
\pgfpathcurveto{\pgfqpoint{0.071446in}{-0.041494in}}{\pgfqpoint{0.079861in}{-0.021179in}}{\pgfqpoint{0.079861in}{0.000000in}}%
\pgfpathcurveto{\pgfqpoint{0.079861in}{0.021179in}}{\pgfqpoint{0.071446in}{0.041494in}}{\pgfqpoint{0.056470in}{0.056470in}}%
\pgfpathcurveto{\pgfqpoint{0.041494in}{0.071446in}}{\pgfqpoint{0.021179in}{0.079861in}}{\pgfqpoint{0.000000in}{0.079861in}}%
\pgfpathcurveto{\pgfqpoint{-0.021179in}{0.079861in}}{\pgfqpoint{-0.041494in}{0.071446in}}{\pgfqpoint{-0.056470in}{0.056470in}}%
\pgfpathcurveto{\pgfqpoint{-0.071446in}{0.041494in}}{\pgfqpoint{-0.079861in}{0.021179in}}{\pgfqpoint{-0.079861in}{0.000000in}}%
\pgfpathcurveto{\pgfqpoint{-0.079861in}{-0.021179in}}{\pgfqpoint{-0.071446in}{-0.041494in}}{\pgfqpoint{-0.056470in}{-0.056470in}}%
\pgfpathcurveto{\pgfqpoint{-0.041494in}{-0.071446in}}{\pgfqpoint{-0.021179in}{-0.079861in}}{\pgfqpoint{0.000000in}{-0.079861in}}%
\pgfpathclose%
\pgfusepath{stroke,fill}%
}%
\begin{pgfscope}%
\pgfsys@transformshift{4.642718in}{2.670006in}%
\pgfsys@useobject{currentmarker}{}%
\end{pgfscope}%
\end{pgfscope}%
\begin{pgfscope}%
\pgfpathrectangle{\pgfqpoint{0.100000in}{0.100000in}}{\pgfqpoint{5.307240in}{3.397500in}}%
\pgfusepath{clip}%
\pgfsetrectcap%
\pgfsetroundjoin%
\pgfsetlinewidth{1.505625pt}%
\definecolor{currentstroke}{rgb}{0.678431,1.000000,0.184314}%
\pgfsetstrokecolor{currentstroke}%
\pgfsetstrokeopacity{0.500000}%
\pgfsetdash{}{0pt}%
\pgfpathmoveto{\pgfqpoint{4.226404in}{1.608442in}}%
\pgfusepath{stroke}%
\end{pgfscope}%
\begin{pgfscope}%
\pgfpathrectangle{\pgfqpoint{0.100000in}{0.100000in}}{\pgfqpoint{5.307240in}{3.397500in}}%
\pgfusepath{clip}%
\pgfsetbuttcap%
\pgfsetroundjoin%
\definecolor{currentfill}{rgb}{0.678431,1.000000,0.184314}%
\pgfsetfillcolor{currentfill}%
\pgfsetfillopacity{0.500000}%
\pgfsetlinewidth{0.250937pt}%
\definecolor{currentstroke}{rgb}{0.000000,0.000000,0.000000}%
\pgfsetstrokecolor{currentstroke}%
\pgfsetstrokeopacity{0.500000}%
\pgfsetdash{}{0pt}%
\pgfsys@defobject{currentmarker}{\pgfqpoint{-0.090972in}{-0.090972in}}{\pgfqpoint{0.090972in}{0.090972in}}{%
\pgfpathmoveto{\pgfqpoint{0.000000in}{-0.090972in}}%
\pgfpathcurveto{\pgfqpoint{0.024126in}{-0.090972in}}{\pgfqpoint{0.047267in}{-0.081387in}}{\pgfqpoint{0.064327in}{-0.064327in}}%
\pgfpathcurveto{\pgfqpoint{0.081387in}{-0.047267in}}{\pgfqpoint{0.090972in}{-0.024126in}}{\pgfqpoint{0.090972in}{0.000000in}}%
\pgfpathcurveto{\pgfqpoint{0.090972in}{0.024126in}}{\pgfqpoint{0.081387in}{0.047267in}}{\pgfqpoint{0.064327in}{0.064327in}}%
\pgfpathcurveto{\pgfqpoint{0.047267in}{0.081387in}}{\pgfqpoint{0.024126in}{0.090972in}}{\pgfqpoint{0.000000in}{0.090972in}}%
\pgfpathcurveto{\pgfqpoint{-0.024126in}{0.090972in}}{\pgfqpoint{-0.047267in}{0.081387in}}{\pgfqpoint{-0.064327in}{0.064327in}}%
\pgfpathcurveto{\pgfqpoint{-0.081387in}{0.047267in}}{\pgfqpoint{-0.090972in}{0.024126in}}{\pgfqpoint{-0.090972in}{0.000000in}}%
\pgfpathcurveto{\pgfqpoint{-0.090972in}{-0.024126in}}{\pgfqpoint{-0.081387in}{-0.047267in}}{\pgfqpoint{-0.064327in}{-0.064327in}}%
\pgfpathcurveto{\pgfqpoint{-0.047267in}{-0.081387in}}{\pgfqpoint{-0.024126in}{-0.090972in}}{\pgfqpoint{0.000000in}{-0.090972in}}%
\pgfpathclose%
\pgfusepath{stroke,fill}%
}%
\begin{pgfscope}%
\pgfsys@transformshift{4.226404in}{1.608442in}%
\pgfsys@useobject{currentmarker}{}%
\end{pgfscope}%
\end{pgfscope}%
\begin{pgfscope}%
\pgfpathrectangle{\pgfqpoint{0.100000in}{0.100000in}}{\pgfqpoint{5.307240in}{3.397500in}}%
\pgfusepath{clip}%
\pgfsetrectcap%
\pgfsetroundjoin%
\pgfsetlinewidth{1.505625pt}%
\definecolor{currentstroke}{rgb}{0.678431,1.000000,0.184314}%
\pgfsetstrokecolor{currentstroke}%
\pgfsetstrokeopacity{0.500000}%
\pgfsetdash{}{0pt}%
\pgfpathmoveto{\pgfqpoint{4.508163in}{1.710495in}}%
\pgfusepath{stroke}%
\end{pgfscope}%
\begin{pgfscope}%
\pgfpathrectangle{\pgfqpoint{0.100000in}{0.100000in}}{\pgfqpoint{5.307240in}{3.397500in}}%
\pgfusepath{clip}%
\pgfsetbuttcap%
\pgfsetroundjoin%
\definecolor{currentfill}{rgb}{0.678431,1.000000,0.184314}%
\pgfsetfillcolor{currentfill}%
\pgfsetfillopacity{0.500000}%
\pgfsetlinewidth{0.250937pt}%
\definecolor{currentstroke}{rgb}{0.000000,0.000000,0.000000}%
\pgfsetstrokecolor{currentstroke}%
\pgfsetstrokeopacity{0.500000}%
\pgfsetdash{}{0pt}%
\pgfsys@defobject{currentmarker}{\pgfqpoint{-0.063889in}{-0.063889in}}{\pgfqpoint{0.063889in}{0.063889in}}{%
\pgfpathmoveto{\pgfqpoint{0.000000in}{-0.063889in}}%
\pgfpathcurveto{\pgfqpoint{0.016944in}{-0.063889in}}{\pgfqpoint{0.033195in}{-0.057157in}}{\pgfqpoint{0.045176in}{-0.045176in}}%
\pgfpathcurveto{\pgfqpoint{0.057157in}{-0.033195in}}{\pgfqpoint{0.063889in}{-0.016944in}}{\pgfqpoint{0.063889in}{0.000000in}}%
\pgfpathcurveto{\pgfqpoint{0.063889in}{0.016944in}}{\pgfqpoint{0.057157in}{0.033195in}}{\pgfqpoint{0.045176in}{0.045176in}}%
\pgfpathcurveto{\pgfqpoint{0.033195in}{0.057157in}}{\pgfqpoint{0.016944in}{0.063889in}}{\pgfqpoint{0.000000in}{0.063889in}}%
\pgfpathcurveto{\pgfqpoint{-0.016944in}{0.063889in}}{\pgfqpoint{-0.033195in}{0.057157in}}{\pgfqpoint{-0.045176in}{0.045176in}}%
\pgfpathcurveto{\pgfqpoint{-0.057157in}{0.033195in}}{\pgfqpoint{-0.063889in}{0.016944in}}{\pgfqpoint{-0.063889in}{0.000000in}}%
\pgfpathcurveto{\pgfqpoint{-0.063889in}{-0.016944in}}{\pgfqpoint{-0.057157in}{-0.033195in}}{\pgfqpoint{-0.045176in}{-0.045176in}}%
\pgfpathcurveto{\pgfqpoint{-0.033195in}{-0.057157in}}{\pgfqpoint{-0.016944in}{-0.063889in}}{\pgfqpoint{0.000000in}{-0.063889in}}%
\pgfpathclose%
\pgfusepath{stroke,fill}%
}%
\begin{pgfscope}%
\pgfsys@transformshift{4.508163in}{1.710495in}%
\pgfsys@useobject{currentmarker}{}%
\end{pgfscope}%
\end{pgfscope}%
\begin{pgfscope}%
\pgfpathrectangle{\pgfqpoint{0.100000in}{0.100000in}}{\pgfqpoint{5.307240in}{3.397500in}}%
\pgfusepath{clip}%
\pgfsetrectcap%
\pgfsetroundjoin%
\pgfsetlinewidth{1.505625pt}%
\definecolor{currentstroke}{rgb}{0.678431,1.000000,0.184314}%
\pgfsetstrokecolor{currentstroke}%
\pgfsetstrokeopacity{0.500000}%
\pgfsetdash{}{0pt}%
\pgfpathmoveto{\pgfqpoint{4.393436in}{1.589048in}}%
\pgfusepath{stroke}%
\end{pgfscope}%
\begin{pgfscope}%
\pgfpathrectangle{\pgfqpoint{0.100000in}{0.100000in}}{\pgfqpoint{5.307240in}{3.397500in}}%
\pgfusepath{clip}%
\pgfsetbuttcap%
\pgfsetroundjoin%
\definecolor{currentfill}{rgb}{0.678431,1.000000,0.184314}%
\pgfsetfillcolor{currentfill}%
\pgfsetfillopacity{0.500000}%
\pgfsetlinewidth{0.250937pt}%
\definecolor{currentstroke}{rgb}{0.000000,0.000000,0.000000}%
\pgfsetstrokecolor{currentstroke}%
\pgfsetstrokeopacity{0.500000}%
\pgfsetdash{}{0pt}%
\pgfsys@defobject{currentmarker}{\pgfqpoint{-0.065278in}{-0.065278in}}{\pgfqpoint{0.065278in}{0.065278in}}{%
\pgfpathmoveto{\pgfqpoint{0.000000in}{-0.065278in}}%
\pgfpathcurveto{\pgfqpoint{0.017312in}{-0.065278in}}{\pgfqpoint{0.033917in}{-0.058400in}}{\pgfqpoint{0.046158in}{-0.046158in}}%
\pgfpathcurveto{\pgfqpoint{0.058400in}{-0.033917in}}{\pgfqpoint{0.065278in}{-0.017312in}}{\pgfqpoint{0.065278in}{0.000000in}}%
\pgfpathcurveto{\pgfqpoint{0.065278in}{0.017312in}}{\pgfqpoint{0.058400in}{0.033917in}}{\pgfqpoint{0.046158in}{0.046158in}}%
\pgfpathcurveto{\pgfqpoint{0.033917in}{0.058400in}}{\pgfqpoint{0.017312in}{0.065278in}}{\pgfqpoint{0.000000in}{0.065278in}}%
\pgfpathcurveto{\pgfqpoint{-0.017312in}{0.065278in}}{\pgfqpoint{-0.033917in}{0.058400in}}{\pgfqpoint{-0.046158in}{0.046158in}}%
\pgfpathcurveto{\pgfqpoint{-0.058400in}{0.033917in}}{\pgfqpoint{-0.065278in}{0.017312in}}{\pgfqpoint{-0.065278in}{0.000000in}}%
\pgfpathcurveto{\pgfqpoint{-0.065278in}{-0.017312in}}{\pgfqpoint{-0.058400in}{-0.033917in}}{\pgfqpoint{-0.046158in}{-0.046158in}}%
\pgfpathcurveto{\pgfqpoint{-0.033917in}{-0.058400in}}{\pgfqpoint{-0.017312in}{-0.065278in}}{\pgfqpoint{0.000000in}{-0.065278in}}%
\pgfpathclose%
\pgfusepath{stroke,fill}%
}%
\begin{pgfscope}%
\pgfsys@transformshift{4.393436in}{1.589048in}%
\pgfsys@useobject{currentmarker}{}%
\end{pgfscope}%
\end{pgfscope}%
\begin{pgfscope}%
\pgfpathrectangle{\pgfqpoint{0.100000in}{0.100000in}}{\pgfqpoint{5.307240in}{3.397500in}}%
\pgfusepath{clip}%
\pgfsetrectcap%
\pgfsetroundjoin%
\pgfsetlinewidth{1.505625pt}%
\definecolor{currentstroke}{rgb}{0.678431,1.000000,0.184314}%
\pgfsetstrokecolor{currentstroke}%
\pgfsetstrokeopacity{0.500000}%
\pgfsetdash{}{0pt}%
\pgfpathmoveto{\pgfqpoint{4.560196in}{1.707636in}}%
\pgfusepath{stroke}%
\end{pgfscope}%
\begin{pgfscope}%
\pgfpathrectangle{\pgfqpoint{0.100000in}{0.100000in}}{\pgfqpoint{5.307240in}{3.397500in}}%
\pgfusepath{clip}%
\pgfsetbuttcap%
\pgfsetroundjoin%
\definecolor{currentfill}{rgb}{0.678431,1.000000,0.184314}%
\pgfsetfillcolor{currentfill}%
\pgfsetfillopacity{0.500000}%
\pgfsetlinewidth{0.250937pt}%
\definecolor{currentstroke}{rgb}{0.000000,0.000000,0.000000}%
\pgfsetstrokecolor{currentstroke}%
\pgfsetstrokeopacity{0.500000}%
\pgfsetdash{}{0pt}%
\pgfsys@defobject{currentmarker}{\pgfqpoint{-0.043750in}{-0.043750in}}{\pgfqpoint{0.043750in}{0.043750in}}{%
\pgfpathmoveto{\pgfqpoint{0.000000in}{-0.043750in}}%
\pgfpathcurveto{\pgfqpoint{0.011603in}{-0.043750in}}{\pgfqpoint{0.022732in}{-0.039140in}}{\pgfqpoint{0.030936in}{-0.030936in}}%
\pgfpathcurveto{\pgfqpoint{0.039140in}{-0.022732in}}{\pgfqpoint{0.043750in}{-0.011603in}}{\pgfqpoint{0.043750in}{0.000000in}}%
\pgfpathcurveto{\pgfqpoint{0.043750in}{0.011603in}}{\pgfqpoint{0.039140in}{0.022732in}}{\pgfqpoint{0.030936in}{0.030936in}}%
\pgfpathcurveto{\pgfqpoint{0.022732in}{0.039140in}}{\pgfqpoint{0.011603in}{0.043750in}}{\pgfqpoint{0.000000in}{0.043750in}}%
\pgfpathcurveto{\pgfqpoint{-0.011603in}{0.043750in}}{\pgfqpoint{-0.022732in}{0.039140in}}{\pgfqpoint{-0.030936in}{0.030936in}}%
\pgfpathcurveto{\pgfqpoint{-0.039140in}{0.022732in}}{\pgfqpoint{-0.043750in}{0.011603in}}{\pgfqpoint{-0.043750in}{0.000000in}}%
\pgfpathcurveto{\pgfqpoint{-0.043750in}{-0.011603in}}{\pgfqpoint{-0.039140in}{-0.022732in}}{\pgfqpoint{-0.030936in}{-0.030936in}}%
\pgfpathcurveto{\pgfqpoint{-0.022732in}{-0.039140in}}{\pgfqpoint{-0.011603in}{-0.043750in}}{\pgfqpoint{0.000000in}{-0.043750in}}%
\pgfpathclose%
\pgfusepath{stroke,fill}%
}%
\begin{pgfscope}%
\pgfsys@transformshift{4.560196in}{1.707636in}%
\pgfsys@useobject{currentmarker}{}%
\end{pgfscope}%
\end{pgfscope}%
\begin{pgfscope}%
\pgfpathrectangle{\pgfqpoint{0.100000in}{0.100000in}}{\pgfqpoint{5.307240in}{3.397500in}}%
\pgfusepath{clip}%
\pgfsetrectcap%
\pgfsetroundjoin%
\pgfsetlinewidth{1.505625pt}%
\definecolor{currentstroke}{rgb}{0.678431,1.000000,0.184314}%
\pgfsetstrokecolor{currentstroke}%
\pgfsetstrokeopacity{0.500000}%
\pgfsetdash{}{0pt}%
\pgfpathmoveto{\pgfqpoint{4.581384in}{1.600082in}}%
\pgfusepath{stroke}%
\end{pgfscope}%
\begin{pgfscope}%
\pgfpathrectangle{\pgfqpoint{0.100000in}{0.100000in}}{\pgfqpoint{5.307240in}{3.397500in}}%
\pgfusepath{clip}%
\pgfsetbuttcap%
\pgfsetroundjoin%
\definecolor{currentfill}{rgb}{0.678431,1.000000,0.184314}%
\pgfsetfillcolor{currentfill}%
\pgfsetfillopacity{0.500000}%
\pgfsetlinewidth{0.250937pt}%
\definecolor{currentstroke}{rgb}{0.000000,0.000000,0.000000}%
\pgfsetstrokecolor{currentstroke}%
\pgfsetstrokeopacity{0.500000}%
\pgfsetdash{}{0pt}%
\pgfsys@defobject{currentmarker}{\pgfqpoint{-0.069444in}{-0.069444in}}{\pgfqpoint{0.069444in}{0.069444in}}{%
\pgfpathmoveto{\pgfqpoint{0.000000in}{-0.069444in}}%
\pgfpathcurveto{\pgfqpoint{0.018417in}{-0.069444in}}{\pgfqpoint{0.036082in}{-0.062127in}}{\pgfqpoint{0.049105in}{-0.049105in}}%
\pgfpathcurveto{\pgfqpoint{0.062127in}{-0.036082in}}{\pgfqpoint{0.069444in}{-0.018417in}}{\pgfqpoint{0.069444in}{0.000000in}}%
\pgfpathcurveto{\pgfqpoint{0.069444in}{0.018417in}}{\pgfqpoint{0.062127in}{0.036082in}}{\pgfqpoint{0.049105in}{0.049105in}}%
\pgfpathcurveto{\pgfqpoint{0.036082in}{0.062127in}}{\pgfqpoint{0.018417in}{0.069444in}}{\pgfqpoint{0.000000in}{0.069444in}}%
\pgfpathcurveto{\pgfqpoint{-0.018417in}{0.069444in}}{\pgfqpoint{-0.036082in}{0.062127in}}{\pgfqpoint{-0.049105in}{0.049105in}}%
\pgfpathcurveto{\pgfqpoint{-0.062127in}{0.036082in}}{\pgfqpoint{-0.069444in}{0.018417in}}{\pgfqpoint{-0.069444in}{0.000000in}}%
\pgfpathcurveto{\pgfqpoint{-0.069444in}{-0.018417in}}{\pgfqpoint{-0.062127in}{-0.036082in}}{\pgfqpoint{-0.049105in}{-0.049105in}}%
\pgfpathcurveto{\pgfqpoint{-0.036082in}{-0.062127in}}{\pgfqpoint{-0.018417in}{-0.069444in}}{\pgfqpoint{0.000000in}{-0.069444in}}%
\pgfpathclose%
\pgfusepath{stroke,fill}%
}%
\begin{pgfscope}%
\pgfsys@transformshift{4.581384in}{1.600082in}%
\pgfsys@useobject{currentmarker}{}%
\end{pgfscope}%
\end{pgfscope}%
\begin{pgfscope}%
\pgfpathrectangle{\pgfqpoint{0.100000in}{0.100000in}}{\pgfqpoint{5.307240in}{3.397500in}}%
\pgfusepath{clip}%
\pgfsetrectcap%
\pgfsetroundjoin%
\pgfsetlinewidth{1.505625pt}%
\definecolor{currentstroke}{rgb}{0.678431,1.000000,0.184314}%
\pgfsetstrokecolor{currentstroke}%
\pgfsetstrokeopacity{0.500000}%
\pgfsetdash{}{0pt}%
\pgfpathmoveto{\pgfqpoint{4.657305in}{1.653395in}}%
\pgfusepath{stroke}%
\end{pgfscope}%
\begin{pgfscope}%
\pgfpathrectangle{\pgfqpoint{0.100000in}{0.100000in}}{\pgfqpoint{5.307240in}{3.397500in}}%
\pgfusepath{clip}%
\pgfsetbuttcap%
\pgfsetroundjoin%
\definecolor{currentfill}{rgb}{0.678431,1.000000,0.184314}%
\pgfsetfillcolor{currentfill}%
\pgfsetfillopacity{0.500000}%
\pgfsetlinewidth{0.250937pt}%
\definecolor{currentstroke}{rgb}{0.000000,0.000000,0.000000}%
\pgfsetstrokecolor{currentstroke}%
\pgfsetstrokeopacity{0.500000}%
\pgfsetdash{}{0pt}%
\pgfsys@defobject{currentmarker}{\pgfqpoint{-0.040278in}{-0.040278in}}{\pgfqpoint{0.040278in}{0.040278in}}{%
\pgfpathmoveto{\pgfqpoint{0.000000in}{-0.040278in}}%
\pgfpathcurveto{\pgfqpoint{0.010682in}{-0.040278in}}{\pgfqpoint{0.020928in}{-0.036034in}}{\pgfqpoint{0.028481in}{-0.028481in}}%
\pgfpathcurveto{\pgfqpoint{0.036034in}{-0.020928in}}{\pgfqpoint{0.040278in}{-0.010682in}}{\pgfqpoint{0.040278in}{0.000000in}}%
\pgfpathcurveto{\pgfqpoint{0.040278in}{0.010682in}}{\pgfqpoint{0.036034in}{0.020928in}}{\pgfqpoint{0.028481in}{0.028481in}}%
\pgfpathcurveto{\pgfqpoint{0.020928in}{0.036034in}}{\pgfqpoint{0.010682in}{0.040278in}}{\pgfqpoint{0.000000in}{0.040278in}}%
\pgfpathcurveto{\pgfqpoint{-0.010682in}{0.040278in}}{\pgfqpoint{-0.020928in}{0.036034in}}{\pgfqpoint{-0.028481in}{0.028481in}}%
\pgfpathcurveto{\pgfqpoint{-0.036034in}{0.020928in}}{\pgfqpoint{-0.040278in}{0.010682in}}{\pgfqpoint{-0.040278in}{0.000000in}}%
\pgfpathcurveto{\pgfqpoint{-0.040278in}{-0.010682in}}{\pgfqpoint{-0.036034in}{-0.020928in}}{\pgfqpoint{-0.028481in}{-0.028481in}}%
\pgfpathcurveto{\pgfqpoint{-0.020928in}{-0.036034in}}{\pgfqpoint{-0.010682in}{-0.040278in}}{\pgfqpoint{0.000000in}{-0.040278in}}%
\pgfpathclose%
\pgfusepath{stroke,fill}%
}%
\begin{pgfscope}%
\pgfsys@transformshift{4.657305in}{1.653395in}%
\pgfsys@useobject{currentmarker}{}%
\end{pgfscope}%
\end{pgfscope}%
\begin{pgfscope}%
\pgfpathrectangle{\pgfqpoint{0.100000in}{0.100000in}}{\pgfqpoint{5.307240in}{3.397500in}}%
\pgfusepath{clip}%
\pgfsetrectcap%
\pgfsetroundjoin%
\pgfsetlinewidth{1.505625pt}%
\definecolor{currentstroke}{rgb}{0.678431,1.000000,0.184314}%
\pgfsetstrokecolor{currentstroke}%
\pgfsetstrokeopacity{0.500000}%
\pgfsetdash{}{0pt}%
\pgfpathmoveto{\pgfqpoint{4.475772in}{1.702235in}}%
\pgfusepath{stroke}%
\end{pgfscope}%
\begin{pgfscope}%
\pgfpathrectangle{\pgfqpoint{0.100000in}{0.100000in}}{\pgfqpoint{5.307240in}{3.397500in}}%
\pgfusepath{clip}%
\pgfsetbuttcap%
\pgfsetroundjoin%
\definecolor{currentfill}{rgb}{0.678431,1.000000,0.184314}%
\pgfsetfillcolor{currentfill}%
\pgfsetfillopacity{0.500000}%
\pgfsetlinewidth{0.250937pt}%
\definecolor{currentstroke}{rgb}{0.000000,0.000000,0.000000}%
\pgfsetstrokecolor{currentstroke}%
\pgfsetstrokeopacity{0.500000}%
\pgfsetdash{}{0pt}%
\pgfsys@defobject{currentmarker}{\pgfqpoint{-0.078472in}{-0.078472in}}{\pgfqpoint{0.078472in}{0.078472in}}{%
\pgfpathmoveto{\pgfqpoint{0.000000in}{-0.078472in}}%
\pgfpathcurveto{\pgfqpoint{0.020811in}{-0.078472in}}{\pgfqpoint{0.040773in}{-0.070204in}}{\pgfqpoint{0.055488in}{-0.055488in}}%
\pgfpathcurveto{\pgfqpoint{0.070204in}{-0.040773in}}{\pgfqpoint{0.078472in}{-0.020811in}}{\pgfqpoint{0.078472in}{0.000000in}}%
\pgfpathcurveto{\pgfqpoint{0.078472in}{0.020811in}}{\pgfqpoint{0.070204in}{0.040773in}}{\pgfqpoint{0.055488in}{0.055488in}}%
\pgfpathcurveto{\pgfqpoint{0.040773in}{0.070204in}}{\pgfqpoint{0.020811in}{0.078472in}}{\pgfqpoint{0.000000in}{0.078472in}}%
\pgfpathcurveto{\pgfqpoint{-0.020811in}{0.078472in}}{\pgfqpoint{-0.040773in}{0.070204in}}{\pgfqpoint{-0.055488in}{0.055488in}}%
\pgfpathcurveto{\pgfqpoint{-0.070204in}{0.040773in}}{\pgfqpoint{-0.078472in}{0.020811in}}{\pgfqpoint{-0.078472in}{0.000000in}}%
\pgfpathcurveto{\pgfqpoint{-0.078472in}{-0.020811in}}{\pgfqpoint{-0.070204in}{-0.040773in}}{\pgfqpoint{-0.055488in}{-0.055488in}}%
\pgfpathcurveto{\pgfqpoint{-0.040773in}{-0.070204in}}{\pgfqpoint{-0.020811in}{-0.078472in}}{\pgfqpoint{0.000000in}{-0.078472in}}%
\pgfpathclose%
\pgfusepath{stroke,fill}%
}%
\begin{pgfscope}%
\pgfsys@transformshift{4.475772in}{1.702235in}%
\pgfsys@useobject{currentmarker}{}%
\end{pgfscope}%
\end{pgfscope}%
\begin{pgfscope}%
\pgfpathrectangle{\pgfqpoint{0.100000in}{0.100000in}}{\pgfqpoint{5.307240in}{3.397500in}}%
\pgfusepath{clip}%
\pgfsetrectcap%
\pgfsetroundjoin%
\pgfsetlinewidth{1.505625pt}%
\definecolor{currentstroke}{rgb}{0.678431,1.000000,0.184314}%
\pgfsetstrokecolor{currentstroke}%
\pgfsetstrokeopacity{0.500000}%
\pgfsetdash{}{0pt}%
\pgfpathmoveto{\pgfqpoint{4.710602in}{1.690697in}}%
\pgfusepath{stroke}%
\end{pgfscope}%
\begin{pgfscope}%
\pgfpathrectangle{\pgfqpoint{0.100000in}{0.100000in}}{\pgfqpoint{5.307240in}{3.397500in}}%
\pgfusepath{clip}%
\pgfsetbuttcap%
\pgfsetroundjoin%
\definecolor{currentfill}{rgb}{0.678431,1.000000,0.184314}%
\pgfsetfillcolor{currentfill}%
\pgfsetfillopacity{0.500000}%
\pgfsetlinewidth{0.250937pt}%
\definecolor{currentstroke}{rgb}{0.000000,0.000000,0.000000}%
\pgfsetstrokecolor{currentstroke}%
\pgfsetstrokeopacity{0.500000}%
\pgfsetdash{}{0pt}%
\pgfsys@defobject{currentmarker}{\pgfqpoint{-0.040278in}{-0.040278in}}{\pgfqpoint{0.040278in}{0.040278in}}{%
\pgfpathmoveto{\pgfqpoint{0.000000in}{-0.040278in}}%
\pgfpathcurveto{\pgfqpoint{0.010682in}{-0.040278in}}{\pgfqpoint{0.020928in}{-0.036034in}}{\pgfqpoint{0.028481in}{-0.028481in}}%
\pgfpathcurveto{\pgfqpoint{0.036034in}{-0.020928in}}{\pgfqpoint{0.040278in}{-0.010682in}}{\pgfqpoint{0.040278in}{0.000000in}}%
\pgfpathcurveto{\pgfqpoint{0.040278in}{0.010682in}}{\pgfqpoint{0.036034in}{0.020928in}}{\pgfqpoint{0.028481in}{0.028481in}}%
\pgfpathcurveto{\pgfqpoint{0.020928in}{0.036034in}}{\pgfqpoint{0.010682in}{0.040278in}}{\pgfqpoint{0.000000in}{0.040278in}}%
\pgfpathcurveto{\pgfqpoint{-0.010682in}{0.040278in}}{\pgfqpoint{-0.020928in}{0.036034in}}{\pgfqpoint{-0.028481in}{0.028481in}}%
\pgfpathcurveto{\pgfqpoint{-0.036034in}{0.020928in}}{\pgfqpoint{-0.040278in}{0.010682in}}{\pgfqpoint{-0.040278in}{0.000000in}}%
\pgfpathcurveto{\pgfqpoint{-0.040278in}{-0.010682in}}{\pgfqpoint{-0.036034in}{-0.020928in}}{\pgfqpoint{-0.028481in}{-0.028481in}}%
\pgfpathcurveto{\pgfqpoint{-0.020928in}{-0.036034in}}{\pgfqpoint{-0.010682in}{-0.040278in}}{\pgfqpoint{0.000000in}{-0.040278in}}%
\pgfpathclose%
\pgfusepath{stroke,fill}%
}%
\begin{pgfscope}%
\pgfsys@transformshift{4.710602in}{1.690697in}%
\pgfsys@useobject{currentmarker}{}%
\end{pgfscope}%
\end{pgfscope}%
\begin{pgfscope}%
\pgfpathrectangle{\pgfqpoint{0.100000in}{0.100000in}}{\pgfqpoint{5.307240in}{3.397500in}}%
\pgfusepath{clip}%
\pgfsetrectcap%
\pgfsetroundjoin%
\pgfsetlinewidth{1.505625pt}%
\definecolor{currentstroke}{rgb}{0.678431,1.000000,0.184314}%
\pgfsetstrokecolor{currentstroke}%
\pgfsetstrokeopacity{0.500000}%
\pgfsetdash{}{0pt}%
\pgfpathmoveto{\pgfqpoint{4.337761in}{1.640106in}}%
\pgfusepath{stroke}%
\end{pgfscope}%
\begin{pgfscope}%
\pgfpathrectangle{\pgfqpoint{0.100000in}{0.100000in}}{\pgfqpoint{5.307240in}{3.397500in}}%
\pgfusepath{clip}%
\pgfsetbuttcap%
\pgfsetroundjoin%
\definecolor{currentfill}{rgb}{0.678431,1.000000,0.184314}%
\pgfsetfillcolor{currentfill}%
\pgfsetfillopacity{0.500000}%
\pgfsetlinewidth{0.250937pt}%
\definecolor{currentstroke}{rgb}{0.000000,0.000000,0.000000}%
\pgfsetstrokecolor{currentstroke}%
\pgfsetstrokeopacity{0.500000}%
\pgfsetdash{}{0pt}%
\pgfsys@defobject{currentmarker}{\pgfqpoint{-0.099306in}{-0.099306in}}{\pgfqpoint{0.099306in}{0.099306in}}{%
\pgfpathmoveto{\pgfqpoint{0.000000in}{-0.099306in}}%
\pgfpathcurveto{\pgfqpoint{0.026336in}{-0.099306in}}{\pgfqpoint{0.051597in}{-0.088842in}}{\pgfqpoint{0.070220in}{-0.070220in}}%
\pgfpathcurveto{\pgfqpoint{0.088842in}{-0.051597in}}{\pgfqpoint{0.099306in}{-0.026336in}}{\pgfqpoint{0.099306in}{0.000000in}}%
\pgfpathcurveto{\pgfqpoint{0.099306in}{0.026336in}}{\pgfqpoint{0.088842in}{0.051597in}}{\pgfqpoint{0.070220in}{0.070220in}}%
\pgfpathcurveto{\pgfqpoint{0.051597in}{0.088842in}}{\pgfqpoint{0.026336in}{0.099306in}}{\pgfqpoint{0.000000in}{0.099306in}}%
\pgfpathcurveto{\pgfqpoint{-0.026336in}{0.099306in}}{\pgfqpoint{-0.051597in}{0.088842in}}{\pgfqpoint{-0.070220in}{0.070220in}}%
\pgfpathcurveto{\pgfqpoint{-0.088842in}{0.051597in}}{\pgfqpoint{-0.099306in}{0.026336in}}{\pgfqpoint{-0.099306in}{0.000000in}}%
\pgfpathcurveto{\pgfqpoint{-0.099306in}{-0.026336in}}{\pgfqpoint{-0.088842in}{-0.051597in}}{\pgfqpoint{-0.070220in}{-0.070220in}}%
\pgfpathcurveto{\pgfqpoint{-0.051597in}{-0.088842in}}{\pgfqpoint{-0.026336in}{-0.099306in}}{\pgfqpoint{0.000000in}{-0.099306in}}%
\pgfpathclose%
\pgfusepath{stroke,fill}%
}%
\begin{pgfscope}%
\pgfsys@transformshift{4.337761in}{1.640106in}%
\pgfsys@useobject{currentmarker}{}%
\end{pgfscope}%
\end{pgfscope}%
\begin{pgfscope}%
\pgfpathrectangle{\pgfqpoint{0.100000in}{0.100000in}}{\pgfqpoint{5.307240in}{3.397500in}}%
\pgfusepath{clip}%
\pgfsetrectcap%
\pgfsetroundjoin%
\pgfsetlinewidth{1.505625pt}%
\definecolor{currentstroke}{rgb}{0.678431,1.000000,0.184314}%
\pgfsetstrokecolor{currentstroke}%
\pgfsetstrokeopacity{0.500000}%
\pgfsetdash{}{0pt}%
\pgfpathmoveto{\pgfqpoint{4.723888in}{1.591355in}}%
\pgfusepath{stroke}%
\end{pgfscope}%
\begin{pgfscope}%
\pgfpathrectangle{\pgfqpoint{0.100000in}{0.100000in}}{\pgfqpoint{5.307240in}{3.397500in}}%
\pgfusepath{clip}%
\pgfsetbuttcap%
\pgfsetroundjoin%
\definecolor{currentfill}{rgb}{0.678431,1.000000,0.184314}%
\pgfsetfillcolor{currentfill}%
\pgfsetfillopacity{0.500000}%
\pgfsetlinewidth{0.250937pt}%
\definecolor{currentstroke}{rgb}{0.000000,0.000000,0.000000}%
\pgfsetstrokecolor{currentstroke}%
\pgfsetstrokeopacity{0.500000}%
\pgfsetdash{}{0pt}%
\pgfsys@defobject{currentmarker}{\pgfqpoint{-0.052083in}{-0.052083in}}{\pgfqpoint{0.052083in}{0.052083in}}{%
\pgfpathmoveto{\pgfqpoint{0.000000in}{-0.052083in}}%
\pgfpathcurveto{\pgfqpoint{0.013813in}{-0.052083in}}{\pgfqpoint{0.027061in}{-0.046596in}}{\pgfqpoint{0.036828in}{-0.036828in}}%
\pgfpathcurveto{\pgfqpoint{0.046596in}{-0.027061in}}{\pgfqpoint{0.052083in}{-0.013813in}}{\pgfqpoint{0.052083in}{0.000000in}}%
\pgfpathcurveto{\pgfqpoint{0.052083in}{0.013813in}}{\pgfqpoint{0.046596in}{0.027061in}}{\pgfqpoint{0.036828in}{0.036828in}}%
\pgfpathcurveto{\pgfqpoint{0.027061in}{0.046596in}}{\pgfqpoint{0.013813in}{0.052083in}}{\pgfqpoint{0.000000in}{0.052083in}}%
\pgfpathcurveto{\pgfqpoint{-0.013813in}{0.052083in}}{\pgfqpoint{-0.027061in}{0.046596in}}{\pgfqpoint{-0.036828in}{0.036828in}}%
\pgfpathcurveto{\pgfqpoint{-0.046596in}{0.027061in}}{\pgfqpoint{-0.052083in}{0.013813in}}{\pgfqpoint{-0.052083in}{0.000000in}}%
\pgfpathcurveto{\pgfqpoint{-0.052083in}{-0.013813in}}{\pgfqpoint{-0.046596in}{-0.027061in}}{\pgfqpoint{-0.036828in}{-0.036828in}}%
\pgfpathcurveto{\pgfqpoint{-0.027061in}{-0.046596in}}{\pgfqpoint{-0.013813in}{-0.052083in}}{\pgfqpoint{0.000000in}{-0.052083in}}%
\pgfpathclose%
\pgfusepath{stroke,fill}%
}%
\begin{pgfscope}%
\pgfsys@transformshift{4.723888in}{1.591355in}%
\pgfsys@useobject{currentmarker}{}%
\end{pgfscope}%
\end{pgfscope}%
\begin{pgfscope}%
\pgfpathrectangle{\pgfqpoint{0.100000in}{0.100000in}}{\pgfqpoint{5.307240in}{3.397500in}}%
\pgfusepath{clip}%
\pgfsetrectcap%
\pgfsetroundjoin%
\pgfsetlinewidth{1.505625pt}%
\definecolor{currentstroke}{rgb}{0.678431,1.000000,0.184314}%
\pgfsetstrokecolor{currentstroke}%
\pgfsetstrokeopacity{0.500000}%
\pgfsetdash{}{0pt}%
\pgfpathmoveto{\pgfqpoint{4.752058in}{1.639045in}}%
\pgfusepath{stroke}%
\end{pgfscope}%
\begin{pgfscope}%
\pgfpathrectangle{\pgfqpoint{0.100000in}{0.100000in}}{\pgfqpoint{5.307240in}{3.397500in}}%
\pgfusepath{clip}%
\pgfsetbuttcap%
\pgfsetroundjoin%
\definecolor{currentfill}{rgb}{0.678431,1.000000,0.184314}%
\pgfsetfillcolor{currentfill}%
\pgfsetfillopacity{0.500000}%
\pgfsetlinewidth{0.250937pt}%
\definecolor{currentstroke}{rgb}{0.000000,0.000000,0.000000}%
\pgfsetstrokecolor{currentstroke}%
\pgfsetstrokeopacity{0.500000}%
\pgfsetdash{}{0pt}%
\pgfsys@defobject{currentmarker}{\pgfqpoint{-0.044444in}{-0.044444in}}{\pgfqpoint{0.044444in}{0.044444in}}{%
\pgfpathmoveto{\pgfqpoint{0.000000in}{-0.044444in}}%
\pgfpathcurveto{\pgfqpoint{0.011787in}{-0.044444in}}{\pgfqpoint{0.023092in}{-0.039761in}}{\pgfqpoint{0.031427in}{-0.031427in}}%
\pgfpathcurveto{\pgfqpoint{0.039761in}{-0.023092in}}{\pgfqpoint{0.044444in}{-0.011787in}}{\pgfqpoint{0.044444in}{0.000000in}}%
\pgfpathcurveto{\pgfqpoint{0.044444in}{0.011787in}}{\pgfqpoint{0.039761in}{0.023092in}}{\pgfqpoint{0.031427in}{0.031427in}}%
\pgfpathcurveto{\pgfqpoint{0.023092in}{0.039761in}}{\pgfqpoint{0.011787in}{0.044444in}}{\pgfqpoint{0.000000in}{0.044444in}}%
\pgfpathcurveto{\pgfqpoint{-0.011787in}{0.044444in}}{\pgfqpoint{-0.023092in}{0.039761in}}{\pgfqpoint{-0.031427in}{0.031427in}}%
\pgfpathcurveto{\pgfqpoint{-0.039761in}{0.023092in}}{\pgfqpoint{-0.044444in}{0.011787in}}{\pgfqpoint{-0.044444in}{0.000000in}}%
\pgfpathcurveto{\pgfqpoint{-0.044444in}{-0.011787in}}{\pgfqpoint{-0.039761in}{-0.023092in}}{\pgfqpoint{-0.031427in}{-0.031427in}}%
\pgfpathcurveto{\pgfqpoint{-0.023092in}{-0.039761in}}{\pgfqpoint{-0.011787in}{-0.044444in}}{\pgfqpoint{0.000000in}{-0.044444in}}%
\pgfpathclose%
\pgfusepath{stroke,fill}%
}%
\begin{pgfscope}%
\pgfsys@transformshift{4.752058in}{1.639045in}%
\pgfsys@useobject{currentmarker}{}%
\end{pgfscope}%
\end{pgfscope}%
\begin{pgfscope}%
\pgfpathrectangle{\pgfqpoint{0.100000in}{0.100000in}}{\pgfqpoint{5.307240in}{3.397500in}}%
\pgfusepath{clip}%
\pgfsetrectcap%
\pgfsetroundjoin%
\pgfsetlinewidth{1.505625pt}%
\definecolor{currentstroke}{rgb}{0.678431,1.000000,0.184314}%
\pgfsetstrokecolor{currentstroke}%
\pgfsetstrokeopacity{0.500000}%
\pgfsetdash{}{0pt}%
\pgfpathmoveto{\pgfqpoint{4.588705in}{1.687508in}}%
\pgfusepath{stroke}%
\end{pgfscope}%
\begin{pgfscope}%
\pgfpathrectangle{\pgfqpoint{0.100000in}{0.100000in}}{\pgfqpoint{5.307240in}{3.397500in}}%
\pgfusepath{clip}%
\pgfsetbuttcap%
\pgfsetroundjoin%
\definecolor{currentfill}{rgb}{0.678431,1.000000,0.184314}%
\pgfsetfillcolor{currentfill}%
\pgfsetfillopacity{0.500000}%
\pgfsetlinewidth{0.250937pt}%
\definecolor{currentstroke}{rgb}{0.000000,0.000000,0.000000}%
\pgfsetstrokecolor{currentstroke}%
\pgfsetstrokeopacity{0.500000}%
\pgfsetdash{}{0pt}%
\pgfsys@defobject{currentmarker}{\pgfqpoint{-0.054861in}{-0.054861in}}{\pgfqpoint{0.054861in}{0.054861in}}{%
\pgfpathmoveto{\pgfqpoint{0.000000in}{-0.054861in}}%
\pgfpathcurveto{\pgfqpoint{0.014549in}{-0.054861in}}{\pgfqpoint{0.028505in}{-0.049081in}}{\pgfqpoint{0.038793in}{-0.038793in}}%
\pgfpathcurveto{\pgfqpoint{0.049081in}{-0.028505in}}{\pgfqpoint{0.054861in}{-0.014549in}}{\pgfqpoint{0.054861in}{0.000000in}}%
\pgfpathcurveto{\pgfqpoint{0.054861in}{0.014549in}}{\pgfqpoint{0.049081in}{0.028505in}}{\pgfqpoint{0.038793in}{0.038793in}}%
\pgfpathcurveto{\pgfqpoint{0.028505in}{0.049081in}}{\pgfqpoint{0.014549in}{0.054861in}}{\pgfqpoint{0.000000in}{0.054861in}}%
\pgfpathcurveto{\pgfqpoint{-0.014549in}{0.054861in}}{\pgfqpoint{-0.028505in}{0.049081in}}{\pgfqpoint{-0.038793in}{0.038793in}}%
\pgfpathcurveto{\pgfqpoint{-0.049081in}{0.028505in}}{\pgfqpoint{-0.054861in}{0.014549in}}{\pgfqpoint{-0.054861in}{0.000000in}}%
\pgfpathcurveto{\pgfqpoint{-0.054861in}{-0.014549in}}{\pgfqpoint{-0.049081in}{-0.028505in}}{\pgfqpoint{-0.038793in}{-0.038793in}}%
\pgfpathcurveto{\pgfqpoint{-0.028505in}{-0.049081in}}{\pgfqpoint{-0.014549in}{-0.054861in}}{\pgfqpoint{0.000000in}{-0.054861in}}%
\pgfpathclose%
\pgfusepath{stroke,fill}%
}%
\begin{pgfscope}%
\pgfsys@transformshift{4.588705in}{1.687508in}%
\pgfsys@useobject{currentmarker}{}%
\end{pgfscope}%
\end{pgfscope}%
\begin{pgfscope}%
\pgfpathrectangle{\pgfqpoint{0.100000in}{0.100000in}}{\pgfqpoint{5.307240in}{3.397500in}}%
\pgfusepath{clip}%
\pgfsetrectcap%
\pgfsetroundjoin%
\pgfsetlinewidth{1.505625pt}%
\definecolor{currentstroke}{rgb}{0.678431,1.000000,0.184314}%
\pgfsetstrokecolor{currentstroke}%
\pgfsetstrokeopacity{0.500000}%
\pgfsetdash{}{0pt}%
\pgfpathmoveto{\pgfqpoint{4.663472in}{1.720661in}}%
\pgfusepath{stroke}%
\end{pgfscope}%
\begin{pgfscope}%
\pgfpathrectangle{\pgfqpoint{0.100000in}{0.100000in}}{\pgfqpoint{5.307240in}{3.397500in}}%
\pgfusepath{clip}%
\pgfsetbuttcap%
\pgfsetroundjoin%
\definecolor{currentfill}{rgb}{0.678431,1.000000,0.184314}%
\pgfsetfillcolor{currentfill}%
\pgfsetfillopacity{0.500000}%
\pgfsetlinewidth{0.250937pt}%
\definecolor{currentstroke}{rgb}{0.000000,0.000000,0.000000}%
\pgfsetstrokecolor{currentstroke}%
\pgfsetstrokeopacity{0.500000}%
\pgfsetdash{}{0pt}%
\pgfsys@defobject{currentmarker}{\pgfqpoint{-0.053472in}{-0.053472in}}{\pgfqpoint{0.053472in}{0.053472in}}{%
\pgfpathmoveto{\pgfqpoint{0.000000in}{-0.053472in}}%
\pgfpathcurveto{\pgfqpoint{0.014181in}{-0.053472in}}{\pgfqpoint{0.027783in}{-0.047838in}}{\pgfqpoint{0.037811in}{-0.037811in}}%
\pgfpathcurveto{\pgfqpoint{0.047838in}{-0.027783in}}{\pgfqpoint{0.053472in}{-0.014181in}}{\pgfqpoint{0.053472in}{0.000000in}}%
\pgfpathcurveto{\pgfqpoint{0.053472in}{0.014181in}}{\pgfqpoint{0.047838in}{0.027783in}}{\pgfqpoint{0.037811in}{0.037811in}}%
\pgfpathcurveto{\pgfqpoint{0.027783in}{0.047838in}}{\pgfqpoint{0.014181in}{0.053472in}}{\pgfqpoint{0.000000in}{0.053472in}}%
\pgfpathcurveto{\pgfqpoint{-0.014181in}{0.053472in}}{\pgfqpoint{-0.027783in}{0.047838in}}{\pgfqpoint{-0.037811in}{0.037811in}}%
\pgfpathcurveto{\pgfqpoint{-0.047838in}{0.027783in}}{\pgfqpoint{-0.053472in}{0.014181in}}{\pgfqpoint{-0.053472in}{0.000000in}}%
\pgfpathcurveto{\pgfqpoint{-0.053472in}{-0.014181in}}{\pgfqpoint{-0.047838in}{-0.027783in}}{\pgfqpoint{-0.037811in}{-0.037811in}}%
\pgfpathcurveto{\pgfqpoint{-0.027783in}{-0.047838in}}{\pgfqpoint{-0.014181in}{-0.053472in}}{\pgfqpoint{0.000000in}{-0.053472in}}%
\pgfpathclose%
\pgfusepath{stroke,fill}%
}%
\begin{pgfscope}%
\pgfsys@transformshift{4.663472in}{1.720661in}%
\pgfsys@useobject{currentmarker}{}%
\end{pgfscope}%
\end{pgfscope}%
\begin{pgfscope}%
\pgfpathrectangle{\pgfqpoint{0.100000in}{0.100000in}}{\pgfqpoint{5.307240in}{3.397500in}}%
\pgfusepath{clip}%
\pgfsetrectcap%
\pgfsetroundjoin%
\pgfsetlinewidth{1.505625pt}%
\definecolor{currentstroke}{rgb}{0.678431,1.000000,0.184314}%
\pgfsetstrokecolor{currentstroke}%
\pgfsetstrokeopacity{0.500000}%
\pgfsetdash{}{0pt}%
\pgfpathmoveto{\pgfqpoint{4.686999in}{1.521495in}}%
\pgfusepath{stroke}%
\end{pgfscope}%
\begin{pgfscope}%
\pgfpathrectangle{\pgfqpoint{0.100000in}{0.100000in}}{\pgfqpoint{5.307240in}{3.397500in}}%
\pgfusepath{clip}%
\pgfsetbuttcap%
\pgfsetroundjoin%
\definecolor{currentfill}{rgb}{0.678431,1.000000,0.184314}%
\pgfsetfillcolor{currentfill}%
\pgfsetfillopacity{0.500000}%
\pgfsetlinewidth{0.250937pt}%
\definecolor{currentstroke}{rgb}{0.000000,0.000000,0.000000}%
\pgfsetstrokecolor{currentstroke}%
\pgfsetstrokeopacity{0.500000}%
\pgfsetdash{}{0pt}%
\pgfsys@defobject{currentmarker}{\pgfqpoint{-0.077778in}{-0.077778in}}{\pgfqpoint{0.077778in}{0.077778in}}{%
\pgfpathmoveto{\pgfqpoint{0.000000in}{-0.077778in}}%
\pgfpathcurveto{\pgfqpoint{0.020627in}{-0.077778in}}{\pgfqpoint{0.040412in}{-0.069583in}}{\pgfqpoint{0.054997in}{-0.054997in}}%
\pgfpathcurveto{\pgfqpoint{0.069583in}{-0.040412in}}{\pgfqpoint{0.077778in}{-0.020627in}}{\pgfqpoint{0.077778in}{0.000000in}}%
\pgfpathcurveto{\pgfqpoint{0.077778in}{0.020627in}}{\pgfqpoint{0.069583in}{0.040412in}}{\pgfqpoint{0.054997in}{0.054997in}}%
\pgfpathcurveto{\pgfqpoint{0.040412in}{0.069583in}}{\pgfqpoint{0.020627in}{0.077778in}}{\pgfqpoint{0.000000in}{0.077778in}}%
\pgfpathcurveto{\pgfqpoint{-0.020627in}{0.077778in}}{\pgfqpoint{-0.040412in}{0.069583in}}{\pgfqpoint{-0.054997in}{0.054997in}}%
\pgfpathcurveto{\pgfqpoint{-0.069583in}{0.040412in}}{\pgfqpoint{-0.077778in}{0.020627in}}{\pgfqpoint{-0.077778in}{0.000000in}}%
\pgfpathcurveto{\pgfqpoint{-0.077778in}{-0.020627in}}{\pgfqpoint{-0.069583in}{-0.040412in}}{\pgfqpoint{-0.054997in}{-0.054997in}}%
\pgfpathcurveto{\pgfqpoint{-0.040412in}{-0.069583in}}{\pgfqpoint{-0.020627in}{-0.077778in}}{\pgfqpoint{0.000000in}{-0.077778in}}%
\pgfpathclose%
\pgfusepath{stroke,fill}%
}%
\begin{pgfscope}%
\pgfsys@transformshift{4.686999in}{1.521495in}%
\pgfsys@useobject{currentmarker}{}%
\end{pgfscope}%
\end{pgfscope}%
\begin{pgfscope}%
\pgfpathrectangle{\pgfqpoint{0.100000in}{0.100000in}}{\pgfqpoint{5.307240in}{3.397500in}}%
\pgfusepath{clip}%
\pgfsetrectcap%
\pgfsetroundjoin%
\pgfsetlinewidth{1.505625pt}%
\definecolor{currentstroke}{rgb}{0.678431,1.000000,0.184314}%
\pgfsetstrokecolor{currentstroke}%
\pgfsetstrokeopacity{0.500000}%
\pgfsetdash{}{0pt}%
\pgfpathmoveto{\pgfqpoint{4.433302in}{1.698376in}}%
\pgfusepath{stroke}%
\end{pgfscope}%
\begin{pgfscope}%
\pgfpathrectangle{\pgfqpoint{0.100000in}{0.100000in}}{\pgfqpoint{5.307240in}{3.397500in}}%
\pgfusepath{clip}%
\pgfsetbuttcap%
\pgfsetroundjoin%
\definecolor{currentfill}{rgb}{0.678431,1.000000,0.184314}%
\pgfsetfillcolor{currentfill}%
\pgfsetfillopacity{0.500000}%
\pgfsetlinewidth{0.250937pt}%
\definecolor{currentstroke}{rgb}{0.000000,0.000000,0.000000}%
\pgfsetstrokecolor{currentstroke}%
\pgfsetstrokeopacity{0.500000}%
\pgfsetdash{}{0pt}%
\pgfsys@defobject{currentmarker}{\pgfqpoint{-0.065278in}{-0.065278in}}{\pgfqpoint{0.065278in}{0.065278in}}{%
\pgfpathmoveto{\pgfqpoint{0.000000in}{-0.065278in}}%
\pgfpathcurveto{\pgfqpoint{0.017312in}{-0.065278in}}{\pgfqpoint{0.033917in}{-0.058400in}}{\pgfqpoint{0.046158in}{-0.046158in}}%
\pgfpathcurveto{\pgfqpoint{0.058400in}{-0.033917in}}{\pgfqpoint{0.065278in}{-0.017312in}}{\pgfqpoint{0.065278in}{0.000000in}}%
\pgfpathcurveto{\pgfqpoint{0.065278in}{0.017312in}}{\pgfqpoint{0.058400in}{0.033917in}}{\pgfqpoint{0.046158in}{0.046158in}}%
\pgfpathcurveto{\pgfqpoint{0.033917in}{0.058400in}}{\pgfqpoint{0.017312in}{0.065278in}}{\pgfqpoint{0.000000in}{0.065278in}}%
\pgfpathcurveto{\pgfqpoint{-0.017312in}{0.065278in}}{\pgfqpoint{-0.033917in}{0.058400in}}{\pgfqpoint{-0.046158in}{0.046158in}}%
\pgfpathcurveto{\pgfqpoint{-0.058400in}{0.033917in}}{\pgfqpoint{-0.065278in}{0.017312in}}{\pgfqpoint{-0.065278in}{0.000000in}}%
\pgfpathcurveto{\pgfqpoint{-0.065278in}{-0.017312in}}{\pgfqpoint{-0.058400in}{-0.033917in}}{\pgfqpoint{-0.046158in}{-0.046158in}}%
\pgfpathcurveto{\pgfqpoint{-0.033917in}{-0.058400in}}{\pgfqpoint{-0.017312in}{-0.065278in}}{\pgfqpoint{0.000000in}{-0.065278in}}%
\pgfpathclose%
\pgfusepath{stroke,fill}%
}%
\begin{pgfscope}%
\pgfsys@transformshift{4.433302in}{1.698376in}%
\pgfsys@useobject{currentmarker}{}%
\end{pgfscope}%
\end{pgfscope}%
\begin{pgfscope}%
\pgfpathrectangle{\pgfqpoint{0.100000in}{0.100000in}}{\pgfqpoint{5.307240in}{3.397500in}}%
\pgfusepath{clip}%
\pgfsetrectcap%
\pgfsetroundjoin%
\pgfsetlinewidth{1.505625pt}%
\definecolor{currentstroke}{rgb}{0.678431,1.000000,0.184314}%
\pgfsetstrokecolor{currentstroke}%
\pgfsetstrokeopacity{0.500000}%
\pgfsetdash{}{0pt}%
\pgfpathmoveto{\pgfqpoint{2.587270in}{2.848554in}}%
\pgfusepath{stroke}%
\end{pgfscope}%
\begin{pgfscope}%
\pgfpathrectangle{\pgfqpoint{0.100000in}{0.100000in}}{\pgfqpoint{5.307240in}{3.397500in}}%
\pgfusepath{clip}%
\pgfsetbuttcap%
\pgfsetroundjoin%
\definecolor{currentfill}{rgb}{0.678431,1.000000,0.184314}%
\pgfsetfillcolor{currentfill}%
\pgfsetfillopacity{0.500000}%
\pgfsetlinewidth{0.250937pt}%
\definecolor{currentstroke}{rgb}{0.000000,0.000000,0.000000}%
\pgfsetstrokecolor{currentstroke}%
\pgfsetstrokeopacity{0.500000}%
\pgfsetdash{}{0pt}%
\pgfsys@defobject{currentmarker}{\pgfqpoint{-0.047917in}{-0.047917in}}{\pgfqpoint{0.047917in}{0.047917in}}{%
\pgfpathmoveto{\pgfqpoint{0.000000in}{-0.047917in}}%
\pgfpathcurveto{\pgfqpoint{0.012708in}{-0.047917in}}{\pgfqpoint{0.024897in}{-0.042868in}}{\pgfqpoint{0.033882in}{-0.033882in}}%
\pgfpathcurveto{\pgfqpoint{0.042868in}{-0.024897in}}{\pgfqpoint{0.047917in}{-0.012708in}}{\pgfqpoint{0.047917in}{0.000000in}}%
\pgfpathcurveto{\pgfqpoint{0.047917in}{0.012708in}}{\pgfqpoint{0.042868in}{0.024897in}}{\pgfqpoint{0.033882in}{0.033882in}}%
\pgfpathcurveto{\pgfqpoint{0.024897in}{0.042868in}}{\pgfqpoint{0.012708in}{0.047917in}}{\pgfqpoint{0.000000in}{0.047917in}}%
\pgfpathcurveto{\pgfqpoint{-0.012708in}{0.047917in}}{\pgfqpoint{-0.024897in}{0.042868in}}{\pgfqpoint{-0.033882in}{0.033882in}}%
\pgfpathcurveto{\pgfqpoint{-0.042868in}{0.024897in}}{\pgfqpoint{-0.047917in}{0.012708in}}{\pgfqpoint{-0.047917in}{0.000000in}}%
\pgfpathcurveto{\pgfqpoint{-0.047917in}{-0.012708in}}{\pgfqpoint{-0.042868in}{-0.024897in}}{\pgfqpoint{-0.033882in}{-0.033882in}}%
\pgfpathcurveto{\pgfqpoint{-0.024897in}{-0.042868in}}{\pgfqpoint{-0.012708in}{-0.047917in}}{\pgfqpoint{0.000000in}{-0.047917in}}%
\pgfpathclose%
\pgfusepath{stroke,fill}%
}%
\begin{pgfscope}%
\pgfsys@transformshift{2.587270in}{2.848554in}%
\pgfsys@useobject{currentmarker}{}%
\end{pgfscope}%
\end{pgfscope}%
\begin{pgfscope}%
\pgfpathrectangle{\pgfqpoint{0.100000in}{0.100000in}}{\pgfqpoint{5.307240in}{3.397500in}}%
\pgfusepath{clip}%
\pgfsetrectcap%
\pgfsetroundjoin%
\pgfsetlinewidth{1.505625pt}%
\definecolor{currentstroke}{rgb}{0.678431,1.000000,0.184314}%
\pgfsetstrokecolor{currentstroke}%
\pgfsetstrokeopacity{0.500000}%
\pgfsetdash{}{0pt}%
\pgfpathmoveto{\pgfqpoint{2.907931in}{2.843276in}}%
\pgfusepath{stroke}%
\end{pgfscope}%
\begin{pgfscope}%
\pgfpathrectangle{\pgfqpoint{0.100000in}{0.100000in}}{\pgfqpoint{5.307240in}{3.397500in}}%
\pgfusepath{clip}%
\pgfsetbuttcap%
\pgfsetroundjoin%
\definecolor{currentfill}{rgb}{0.678431,1.000000,0.184314}%
\pgfsetfillcolor{currentfill}%
\pgfsetfillopacity{0.500000}%
\pgfsetlinewidth{0.250937pt}%
\definecolor{currentstroke}{rgb}{0.000000,0.000000,0.000000}%
\pgfsetstrokecolor{currentstroke}%
\pgfsetstrokeopacity{0.500000}%
\pgfsetdash{}{0pt}%
\pgfsys@defobject{currentmarker}{\pgfqpoint{-0.040972in}{-0.040972in}}{\pgfqpoint{0.040972in}{0.040972in}}{%
\pgfpathmoveto{\pgfqpoint{0.000000in}{-0.040972in}}%
\pgfpathcurveto{\pgfqpoint{0.010866in}{-0.040972in}}{\pgfqpoint{0.021288in}{-0.036655in}}{\pgfqpoint{0.028972in}{-0.028972in}}%
\pgfpathcurveto{\pgfqpoint{0.036655in}{-0.021288in}}{\pgfqpoint{0.040972in}{-0.010866in}}{\pgfqpoint{0.040972in}{0.000000in}}%
\pgfpathcurveto{\pgfqpoint{0.040972in}{0.010866in}}{\pgfqpoint{0.036655in}{0.021288in}}{\pgfqpoint{0.028972in}{0.028972in}}%
\pgfpathcurveto{\pgfqpoint{0.021288in}{0.036655in}}{\pgfqpoint{0.010866in}{0.040972in}}{\pgfqpoint{0.000000in}{0.040972in}}%
\pgfpathcurveto{\pgfqpoint{-0.010866in}{0.040972in}}{\pgfqpoint{-0.021288in}{0.036655in}}{\pgfqpoint{-0.028972in}{0.028972in}}%
\pgfpathcurveto{\pgfqpoint{-0.036655in}{0.021288in}}{\pgfqpoint{-0.040972in}{0.010866in}}{\pgfqpoint{-0.040972in}{0.000000in}}%
\pgfpathcurveto{\pgfqpoint{-0.040972in}{-0.010866in}}{\pgfqpoint{-0.036655in}{-0.021288in}}{\pgfqpoint{-0.028972in}{-0.028972in}}%
\pgfpathcurveto{\pgfqpoint{-0.021288in}{-0.036655in}}{\pgfqpoint{-0.010866in}{-0.040972in}}{\pgfqpoint{0.000000in}{-0.040972in}}%
\pgfpathclose%
\pgfusepath{stroke,fill}%
}%
\begin{pgfscope}%
\pgfsys@transformshift{2.907931in}{2.843276in}%
\pgfsys@useobject{currentmarker}{}%
\end{pgfscope}%
\end{pgfscope}%
\begin{pgfscope}%
\pgfpathrectangle{\pgfqpoint{0.100000in}{0.100000in}}{\pgfqpoint{5.307240in}{3.397500in}}%
\pgfusepath{clip}%
\pgfsetrectcap%
\pgfsetroundjoin%
\pgfsetlinewidth{1.505625pt}%
\definecolor{currentstroke}{rgb}{0.678431,1.000000,0.184314}%
\pgfsetstrokecolor{currentstroke}%
\pgfsetstrokeopacity{0.500000}%
\pgfsetdash{}{0pt}%
\pgfpathmoveto{\pgfqpoint{2.889066in}{2.964837in}}%
\pgfusepath{stroke}%
\end{pgfscope}%
\begin{pgfscope}%
\pgfpathrectangle{\pgfqpoint{0.100000in}{0.100000in}}{\pgfqpoint{5.307240in}{3.397500in}}%
\pgfusepath{clip}%
\pgfsetbuttcap%
\pgfsetroundjoin%
\definecolor{currentfill}{rgb}{0.678431,1.000000,0.184314}%
\pgfsetfillcolor{currentfill}%
\pgfsetfillopacity{0.500000}%
\pgfsetlinewidth{0.250937pt}%
\definecolor{currentstroke}{rgb}{0.000000,0.000000,0.000000}%
\pgfsetstrokecolor{currentstroke}%
\pgfsetstrokeopacity{0.500000}%
\pgfsetdash{}{0pt}%
\pgfsys@defobject{currentmarker}{\pgfqpoint{-0.040278in}{-0.040278in}}{\pgfqpoint{0.040278in}{0.040278in}}{%
\pgfpathmoveto{\pgfqpoint{0.000000in}{-0.040278in}}%
\pgfpathcurveto{\pgfqpoint{0.010682in}{-0.040278in}}{\pgfqpoint{0.020928in}{-0.036034in}}{\pgfqpoint{0.028481in}{-0.028481in}}%
\pgfpathcurveto{\pgfqpoint{0.036034in}{-0.020928in}}{\pgfqpoint{0.040278in}{-0.010682in}}{\pgfqpoint{0.040278in}{0.000000in}}%
\pgfpathcurveto{\pgfqpoint{0.040278in}{0.010682in}}{\pgfqpoint{0.036034in}{0.020928in}}{\pgfqpoint{0.028481in}{0.028481in}}%
\pgfpathcurveto{\pgfqpoint{0.020928in}{0.036034in}}{\pgfqpoint{0.010682in}{0.040278in}}{\pgfqpoint{0.000000in}{0.040278in}}%
\pgfpathcurveto{\pgfqpoint{-0.010682in}{0.040278in}}{\pgfqpoint{-0.020928in}{0.036034in}}{\pgfqpoint{-0.028481in}{0.028481in}}%
\pgfpathcurveto{\pgfqpoint{-0.036034in}{0.020928in}}{\pgfqpoint{-0.040278in}{0.010682in}}{\pgfqpoint{-0.040278in}{0.000000in}}%
\pgfpathcurveto{\pgfqpoint{-0.040278in}{-0.010682in}}{\pgfqpoint{-0.036034in}{-0.020928in}}{\pgfqpoint{-0.028481in}{-0.028481in}}%
\pgfpathcurveto{\pgfqpoint{-0.020928in}{-0.036034in}}{\pgfqpoint{-0.010682in}{-0.040278in}}{\pgfqpoint{0.000000in}{-0.040278in}}%
\pgfpathclose%
\pgfusepath{stroke,fill}%
}%
\begin{pgfscope}%
\pgfsys@transformshift{2.889066in}{2.964837in}%
\pgfsys@useobject{currentmarker}{}%
\end{pgfscope}%
\end{pgfscope}%
\begin{pgfscope}%
\pgfpathrectangle{\pgfqpoint{0.100000in}{0.100000in}}{\pgfqpoint{5.307240in}{3.397500in}}%
\pgfusepath{clip}%
\pgfsetrectcap%
\pgfsetroundjoin%
\pgfsetlinewidth{1.505625pt}%
\definecolor{currentstroke}{rgb}{0.678431,1.000000,0.184314}%
\pgfsetstrokecolor{currentstroke}%
\pgfsetstrokeopacity{0.500000}%
\pgfsetdash{}{0pt}%
\pgfpathmoveto{\pgfqpoint{4.229226in}{2.252968in}}%
\pgfusepath{stroke}%
\end{pgfscope}%
\begin{pgfscope}%
\pgfpathrectangle{\pgfqpoint{0.100000in}{0.100000in}}{\pgfqpoint{5.307240in}{3.397500in}}%
\pgfusepath{clip}%
\pgfsetbuttcap%
\pgfsetroundjoin%
\definecolor{currentfill}{rgb}{0.678431,1.000000,0.184314}%
\pgfsetfillcolor{currentfill}%
\pgfsetfillopacity{0.500000}%
\pgfsetlinewidth{0.250937pt}%
\definecolor{currentstroke}{rgb}{0.000000,0.000000,0.000000}%
\pgfsetstrokecolor{currentstroke}%
\pgfsetstrokeopacity{0.500000}%
\pgfsetdash{}{0pt}%
\pgfsys@defobject{currentmarker}{\pgfqpoint{-0.085417in}{-0.085417in}}{\pgfqpoint{0.085417in}{0.085417in}}{%
\pgfpathmoveto{\pgfqpoint{0.000000in}{-0.085417in}}%
\pgfpathcurveto{\pgfqpoint{0.022653in}{-0.085417in}}{\pgfqpoint{0.044381in}{-0.076417in}}{\pgfqpoint{0.060399in}{-0.060399in}}%
\pgfpathcurveto{\pgfqpoint{0.076417in}{-0.044381in}}{\pgfqpoint{0.085417in}{-0.022653in}}{\pgfqpoint{0.085417in}{0.000000in}}%
\pgfpathcurveto{\pgfqpoint{0.085417in}{0.022653in}}{\pgfqpoint{0.076417in}{0.044381in}}{\pgfqpoint{0.060399in}{0.060399in}}%
\pgfpathcurveto{\pgfqpoint{0.044381in}{0.076417in}}{\pgfqpoint{0.022653in}{0.085417in}}{\pgfqpoint{0.000000in}{0.085417in}}%
\pgfpathcurveto{\pgfqpoint{-0.022653in}{0.085417in}}{\pgfqpoint{-0.044381in}{0.076417in}}{\pgfqpoint{-0.060399in}{0.060399in}}%
\pgfpathcurveto{\pgfqpoint{-0.076417in}{0.044381in}}{\pgfqpoint{-0.085417in}{0.022653in}}{\pgfqpoint{-0.085417in}{0.000000in}}%
\pgfpathcurveto{\pgfqpoint{-0.085417in}{-0.022653in}}{\pgfqpoint{-0.076417in}{-0.044381in}}{\pgfqpoint{-0.060399in}{-0.060399in}}%
\pgfpathcurveto{\pgfqpoint{-0.044381in}{-0.076417in}}{\pgfqpoint{-0.022653in}{-0.085417in}}{\pgfqpoint{0.000000in}{-0.085417in}}%
\pgfpathclose%
\pgfusepath{stroke,fill}%
}%
\begin{pgfscope}%
\pgfsys@transformshift{4.229226in}{2.252968in}%
\pgfsys@useobject{currentmarker}{}%
\end{pgfscope}%
\end{pgfscope}%
\begin{pgfscope}%
\pgfpathrectangle{\pgfqpoint{0.100000in}{0.100000in}}{\pgfqpoint{5.307240in}{3.397500in}}%
\pgfusepath{clip}%
\pgfsetrectcap%
\pgfsetroundjoin%
\pgfsetlinewidth{1.505625pt}%
\definecolor{currentstroke}{rgb}{0.678431,1.000000,0.184314}%
\pgfsetstrokecolor{currentstroke}%
\pgfsetstrokeopacity{0.500000}%
\pgfsetdash{}{0pt}%
\pgfpathmoveto{\pgfqpoint{4.246303in}{2.222117in}}%
\pgfusepath{stroke}%
\end{pgfscope}%
\begin{pgfscope}%
\pgfpathrectangle{\pgfqpoint{0.100000in}{0.100000in}}{\pgfqpoint{5.307240in}{3.397500in}}%
\pgfusepath{clip}%
\pgfsetbuttcap%
\pgfsetroundjoin%
\definecolor{currentfill}{rgb}{0.678431,1.000000,0.184314}%
\pgfsetfillcolor{currentfill}%
\pgfsetfillopacity{0.500000}%
\pgfsetlinewidth{0.250937pt}%
\definecolor{currentstroke}{rgb}{0.000000,0.000000,0.000000}%
\pgfsetstrokecolor{currentstroke}%
\pgfsetstrokeopacity{0.500000}%
\pgfsetdash{}{0pt}%
\pgfsys@defobject{currentmarker}{\pgfqpoint{-0.085417in}{-0.085417in}}{\pgfqpoint{0.085417in}{0.085417in}}{%
\pgfpathmoveto{\pgfqpoint{0.000000in}{-0.085417in}}%
\pgfpathcurveto{\pgfqpoint{0.022653in}{-0.085417in}}{\pgfqpoint{0.044381in}{-0.076417in}}{\pgfqpoint{0.060399in}{-0.060399in}}%
\pgfpathcurveto{\pgfqpoint{0.076417in}{-0.044381in}}{\pgfqpoint{0.085417in}{-0.022653in}}{\pgfqpoint{0.085417in}{0.000000in}}%
\pgfpathcurveto{\pgfqpoint{0.085417in}{0.022653in}}{\pgfqpoint{0.076417in}{0.044381in}}{\pgfqpoint{0.060399in}{0.060399in}}%
\pgfpathcurveto{\pgfqpoint{0.044381in}{0.076417in}}{\pgfqpoint{0.022653in}{0.085417in}}{\pgfqpoint{0.000000in}{0.085417in}}%
\pgfpathcurveto{\pgfqpoint{-0.022653in}{0.085417in}}{\pgfqpoint{-0.044381in}{0.076417in}}{\pgfqpoint{-0.060399in}{0.060399in}}%
\pgfpathcurveto{\pgfqpoint{-0.076417in}{0.044381in}}{\pgfqpoint{-0.085417in}{0.022653in}}{\pgfqpoint{-0.085417in}{0.000000in}}%
\pgfpathcurveto{\pgfqpoint{-0.085417in}{-0.022653in}}{\pgfqpoint{-0.076417in}{-0.044381in}}{\pgfqpoint{-0.060399in}{-0.060399in}}%
\pgfpathcurveto{\pgfqpoint{-0.044381in}{-0.076417in}}{\pgfqpoint{-0.022653in}{-0.085417in}}{\pgfqpoint{0.000000in}{-0.085417in}}%
\pgfpathclose%
\pgfusepath{stroke,fill}%
}%
\begin{pgfscope}%
\pgfsys@transformshift{4.246303in}{2.222117in}%
\pgfsys@useobject{currentmarker}{}%
\end{pgfscope}%
\end{pgfscope}%
\begin{pgfscope}%
\pgfpathrectangle{\pgfqpoint{0.100000in}{0.100000in}}{\pgfqpoint{5.307240in}{3.397500in}}%
\pgfusepath{clip}%
\pgfsetrectcap%
\pgfsetroundjoin%
\pgfsetlinewidth{1.505625pt}%
\definecolor{currentstroke}{rgb}{0.678431,1.000000,0.184314}%
\pgfsetstrokecolor{currentstroke}%
\pgfsetstrokeopacity{0.500000}%
\pgfsetdash{}{0pt}%
\pgfpathmoveto{\pgfqpoint{3.995543in}{1.989575in}}%
\pgfusepath{stroke}%
\end{pgfscope}%
\begin{pgfscope}%
\pgfpathrectangle{\pgfqpoint{0.100000in}{0.100000in}}{\pgfqpoint{5.307240in}{3.397500in}}%
\pgfusepath{clip}%
\pgfsetbuttcap%
\pgfsetroundjoin%
\definecolor{currentfill}{rgb}{0.678431,1.000000,0.184314}%
\pgfsetfillcolor{currentfill}%
\pgfsetfillopacity{0.500000}%
\pgfsetlinewidth{0.250937pt}%
\definecolor{currentstroke}{rgb}{0.000000,0.000000,0.000000}%
\pgfsetstrokecolor{currentstroke}%
\pgfsetstrokeopacity{0.500000}%
\pgfsetdash{}{0pt}%
\pgfsys@defobject{currentmarker}{\pgfqpoint{-0.076389in}{-0.076389in}}{\pgfqpoint{0.076389in}{0.076389in}}{%
\pgfpathmoveto{\pgfqpoint{0.000000in}{-0.076389in}}%
\pgfpathcurveto{\pgfqpoint{0.020259in}{-0.076389in}}{\pgfqpoint{0.039690in}{-0.068340in}}{\pgfqpoint{0.054015in}{-0.054015in}}%
\pgfpathcurveto{\pgfqpoint{0.068340in}{-0.039690in}}{\pgfqpoint{0.076389in}{-0.020259in}}{\pgfqpoint{0.076389in}{0.000000in}}%
\pgfpathcurveto{\pgfqpoint{0.076389in}{0.020259in}}{\pgfqpoint{0.068340in}{0.039690in}}{\pgfqpoint{0.054015in}{0.054015in}}%
\pgfpathcurveto{\pgfqpoint{0.039690in}{0.068340in}}{\pgfqpoint{0.020259in}{0.076389in}}{\pgfqpoint{0.000000in}{0.076389in}}%
\pgfpathcurveto{\pgfqpoint{-0.020259in}{0.076389in}}{\pgfqpoint{-0.039690in}{0.068340in}}{\pgfqpoint{-0.054015in}{0.054015in}}%
\pgfpathcurveto{\pgfqpoint{-0.068340in}{0.039690in}}{\pgfqpoint{-0.076389in}{0.020259in}}{\pgfqpoint{-0.076389in}{0.000000in}}%
\pgfpathcurveto{\pgfqpoint{-0.076389in}{-0.020259in}}{\pgfqpoint{-0.068340in}{-0.039690in}}{\pgfqpoint{-0.054015in}{-0.054015in}}%
\pgfpathcurveto{\pgfqpoint{-0.039690in}{-0.068340in}}{\pgfqpoint{-0.020259in}{-0.076389in}}{\pgfqpoint{0.000000in}{-0.076389in}}%
\pgfpathclose%
\pgfusepath{stroke,fill}%
}%
\begin{pgfscope}%
\pgfsys@transformshift{3.995543in}{1.989575in}%
\pgfsys@useobject{currentmarker}{}%
\end{pgfscope}%
\end{pgfscope}%
\begin{pgfscope}%
\pgfpathrectangle{\pgfqpoint{0.100000in}{0.100000in}}{\pgfqpoint{5.307240in}{3.397500in}}%
\pgfusepath{clip}%
\pgfsetrectcap%
\pgfsetroundjoin%
\pgfsetlinewidth{1.505625pt}%
\definecolor{currentstroke}{rgb}{0.678431,1.000000,0.184314}%
\pgfsetstrokecolor{currentstroke}%
\pgfsetstrokeopacity{0.500000}%
\pgfsetdash{}{0pt}%
\pgfpathmoveto{\pgfqpoint{4.206888in}{2.299295in}}%
\pgfusepath{stroke}%
\end{pgfscope}%
\begin{pgfscope}%
\pgfpathrectangle{\pgfqpoint{0.100000in}{0.100000in}}{\pgfqpoint{5.307240in}{3.397500in}}%
\pgfusepath{clip}%
\pgfsetbuttcap%
\pgfsetroundjoin%
\definecolor{currentfill}{rgb}{0.678431,1.000000,0.184314}%
\pgfsetfillcolor{currentfill}%
\pgfsetfillopacity{0.500000}%
\pgfsetlinewidth{0.250937pt}%
\definecolor{currentstroke}{rgb}{0.000000,0.000000,0.000000}%
\pgfsetstrokecolor{currentstroke}%
\pgfsetstrokeopacity{0.500000}%
\pgfsetdash{}{0pt}%
\pgfsys@defobject{currentmarker}{\pgfqpoint{-0.130556in}{-0.130556in}}{\pgfqpoint{0.130556in}{0.130556in}}{%
\pgfpathmoveto{\pgfqpoint{0.000000in}{-0.130556in}}%
\pgfpathcurveto{\pgfqpoint{0.034624in}{-0.130556in}}{\pgfqpoint{0.067834in}{-0.116799in}}{\pgfqpoint{0.092317in}{-0.092317in}}%
\pgfpathcurveto{\pgfqpoint{0.116799in}{-0.067834in}}{\pgfqpoint{0.130556in}{-0.034624in}}{\pgfqpoint{0.130556in}{0.000000in}}%
\pgfpathcurveto{\pgfqpoint{0.130556in}{0.034624in}}{\pgfqpoint{0.116799in}{0.067834in}}{\pgfqpoint{0.092317in}{0.092317in}}%
\pgfpathcurveto{\pgfqpoint{0.067834in}{0.116799in}}{\pgfqpoint{0.034624in}{0.130556in}}{\pgfqpoint{0.000000in}{0.130556in}}%
\pgfpathcurveto{\pgfqpoint{-0.034624in}{0.130556in}}{\pgfqpoint{-0.067834in}{0.116799in}}{\pgfqpoint{-0.092317in}{0.092317in}}%
\pgfpathcurveto{\pgfqpoint{-0.116799in}{0.067834in}}{\pgfqpoint{-0.130556in}{0.034624in}}{\pgfqpoint{-0.130556in}{0.000000in}}%
\pgfpathcurveto{\pgfqpoint{-0.130556in}{-0.034624in}}{\pgfqpoint{-0.116799in}{-0.067834in}}{\pgfqpoint{-0.092317in}{-0.092317in}}%
\pgfpathcurveto{\pgfqpoint{-0.067834in}{-0.116799in}}{\pgfqpoint{-0.034624in}{-0.130556in}}{\pgfqpoint{0.000000in}{-0.130556in}}%
\pgfpathclose%
\pgfusepath{stroke,fill}%
}%
\begin{pgfscope}%
\pgfsys@transformshift{4.206888in}{2.299295in}%
\pgfsys@useobject{currentmarker}{}%
\end{pgfscope}%
\end{pgfscope}%
\begin{pgfscope}%
\pgfpathrectangle{\pgfqpoint{0.100000in}{0.100000in}}{\pgfqpoint{5.307240in}{3.397500in}}%
\pgfusepath{clip}%
\pgfsetrectcap%
\pgfsetroundjoin%
\pgfsetlinewidth{1.505625pt}%
\definecolor{currentstroke}{rgb}{0.678431,1.000000,0.184314}%
\pgfsetstrokecolor{currentstroke}%
\pgfsetstrokeopacity{0.500000}%
\pgfsetdash{}{0pt}%
\pgfpathmoveto{\pgfqpoint{4.117704in}{2.105606in}}%
\pgfusepath{stroke}%
\end{pgfscope}%
\begin{pgfscope}%
\pgfpathrectangle{\pgfqpoint{0.100000in}{0.100000in}}{\pgfqpoint{5.307240in}{3.397500in}}%
\pgfusepath{clip}%
\pgfsetbuttcap%
\pgfsetroundjoin%
\definecolor{currentfill}{rgb}{0.678431,1.000000,0.184314}%
\pgfsetfillcolor{currentfill}%
\pgfsetfillopacity{0.500000}%
\pgfsetlinewidth{0.250937pt}%
\definecolor{currentstroke}{rgb}{0.000000,0.000000,0.000000}%
\pgfsetstrokecolor{currentstroke}%
\pgfsetstrokeopacity{0.500000}%
\pgfsetdash{}{0pt}%
\pgfsys@defobject{currentmarker}{\pgfqpoint{-0.074306in}{-0.074306in}}{\pgfqpoint{0.074306in}{0.074306in}}{%
\pgfpathmoveto{\pgfqpoint{0.000000in}{-0.074306in}}%
\pgfpathcurveto{\pgfqpoint{0.019706in}{-0.074306in}}{\pgfqpoint{0.038608in}{-0.066476in}}{\pgfqpoint{0.052542in}{-0.052542in}}%
\pgfpathcurveto{\pgfqpoint{0.066476in}{-0.038608in}}{\pgfqpoint{0.074306in}{-0.019706in}}{\pgfqpoint{0.074306in}{0.000000in}}%
\pgfpathcurveto{\pgfqpoint{0.074306in}{0.019706in}}{\pgfqpoint{0.066476in}{0.038608in}}{\pgfqpoint{0.052542in}{0.052542in}}%
\pgfpathcurveto{\pgfqpoint{0.038608in}{0.066476in}}{\pgfqpoint{0.019706in}{0.074306in}}{\pgfqpoint{0.000000in}{0.074306in}}%
\pgfpathcurveto{\pgfqpoint{-0.019706in}{0.074306in}}{\pgfqpoint{-0.038608in}{0.066476in}}{\pgfqpoint{-0.052542in}{0.052542in}}%
\pgfpathcurveto{\pgfqpoint{-0.066476in}{0.038608in}}{\pgfqpoint{-0.074306in}{0.019706in}}{\pgfqpoint{-0.074306in}{0.000000in}}%
\pgfpathcurveto{\pgfqpoint{-0.074306in}{-0.019706in}}{\pgfqpoint{-0.066476in}{-0.038608in}}{\pgfqpoint{-0.052542in}{-0.052542in}}%
\pgfpathcurveto{\pgfqpoint{-0.038608in}{-0.066476in}}{\pgfqpoint{-0.019706in}{-0.074306in}}{\pgfqpoint{0.000000in}{-0.074306in}}%
\pgfpathclose%
\pgfusepath{stroke,fill}%
}%
\begin{pgfscope}%
\pgfsys@transformshift{4.117704in}{2.105606in}%
\pgfsys@useobject{currentmarker}{}%
\end{pgfscope}%
\end{pgfscope}%
\begin{pgfscope}%
\pgfpathrectangle{\pgfqpoint{0.100000in}{0.100000in}}{\pgfqpoint{5.307240in}{3.397500in}}%
\pgfusepath{clip}%
\pgfsetrectcap%
\pgfsetroundjoin%
\pgfsetlinewidth{1.505625pt}%
\definecolor{currentstroke}{rgb}{0.678431,1.000000,0.184314}%
\pgfsetstrokecolor{currentstroke}%
\pgfsetstrokeopacity{0.500000}%
\pgfsetdash{}{0pt}%
\pgfpathmoveto{\pgfqpoint{4.015221in}{2.068854in}}%
\pgfusepath{stroke}%
\end{pgfscope}%
\begin{pgfscope}%
\pgfpathrectangle{\pgfqpoint{0.100000in}{0.100000in}}{\pgfqpoint{5.307240in}{3.397500in}}%
\pgfusepath{clip}%
\pgfsetbuttcap%
\pgfsetroundjoin%
\definecolor{currentfill}{rgb}{0.678431,1.000000,0.184314}%
\pgfsetfillcolor{currentfill}%
\pgfsetfillopacity{0.500000}%
\pgfsetlinewidth{0.250937pt}%
\definecolor{currentstroke}{rgb}{0.000000,0.000000,0.000000}%
\pgfsetstrokecolor{currentstroke}%
\pgfsetstrokeopacity{0.500000}%
\pgfsetdash{}{0pt}%
\pgfsys@defobject{currentmarker}{\pgfqpoint{-0.086806in}{-0.086806in}}{\pgfqpoint{0.086806in}{0.086806in}}{%
\pgfpathmoveto{\pgfqpoint{0.000000in}{-0.086806in}}%
\pgfpathcurveto{\pgfqpoint{0.023021in}{-0.086806in}}{\pgfqpoint{0.045102in}{-0.077659in}}{\pgfqpoint{0.061381in}{-0.061381in}}%
\pgfpathcurveto{\pgfqpoint{0.077659in}{-0.045102in}}{\pgfqpoint{0.086806in}{-0.023021in}}{\pgfqpoint{0.086806in}{0.000000in}}%
\pgfpathcurveto{\pgfqpoint{0.086806in}{0.023021in}}{\pgfqpoint{0.077659in}{0.045102in}}{\pgfqpoint{0.061381in}{0.061381in}}%
\pgfpathcurveto{\pgfqpoint{0.045102in}{0.077659in}}{\pgfqpoint{0.023021in}{0.086806in}}{\pgfqpoint{0.000000in}{0.086806in}}%
\pgfpathcurveto{\pgfqpoint{-0.023021in}{0.086806in}}{\pgfqpoint{-0.045102in}{0.077659in}}{\pgfqpoint{-0.061381in}{0.061381in}}%
\pgfpathcurveto{\pgfqpoint{-0.077659in}{0.045102in}}{\pgfqpoint{-0.086806in}{0.023021in}}{\pgfqpoint{-0.086806in}{0.000000in}}%
\pgfpathcurveto{\pgfqpoint{-0.086806in}{-0.023021in}}{\pgfqpoint{-0.077659in}{-0.045102in}}{\pgfqpoint{-0.061381in}{-0.061381in}}%
\pgfpathcurveto{\pgfqpoint{-0.045102in}{-0.077659in}}{\pgfqpoint{-0.023021in}{-0.086806in}}{\pgfqpoint{0.000000in}{-0.086806in}}%
\pgfpathclose%
\pgfusepath{stroke,fill}%
}%
\begin{pgfscope}%
\pgfsys@transformshift{4.015221in}{2.068854in}%
\pgfsys@useobject{currentmarker}{}%
\end{pgfscope}%
\end{pgfscope}%
\begin{pgfscope}%
\pgfpathrectangle{\pgfqpoint{0.100000in}{0.100000in}}{\pgfqpoint{5.307240in}{3.397500in}}%
\pgfusepath{clip}%
\pgfsetrectcap%
\pgfsetroundjoin%
\pgfsetlinewidth{1.505625pt}%
\definecolor{currentstroke}{rgb}{0.678431,1.000000,0.184314}%
\pgfsetstrokecolor{currentstroke}%
\pgfsetstrokeopacity{0.500000}%
\pgfsetdash{}{0pt}%
\pgfpathmoveto{\pgfqpoint{4.009207in}{2.183321in}}%
\pgfusepath{stroke}%
\end{pgfscope}%
\begin{pgfscope}%
\pgfpathrectangle{\pgfqpoint{0.100000in}{0.100000in}}{\pgfqpoint{5.307240in}{3.397500in}}%
\pgfusepath{clip}%
\pgfsetbuttcap%
\pgfsetroundjoin%
\definecolor{currentfill}{rgb}{0.678431,1.000000,0.184314}%
\pgfsetfillcolor{currentfill}%
\pgfsetfillopacity{0.500000}%
\pgfsetlinewidth{0.250937pt}%
\definecolor{currentstroke}{rgb}{0.000000,0.000000,0.000000}%
\pgfsetstrokecolor{currentstroke}%
\pgfsetstrokeopacity{0.500000}%
\pgfsetdash{}{0pt}%
\pgfsys@defobject{currentmarker}{\pgfqpoint{-0.120139in}{-0.120139in}}{\pgfqpoint{0.120139in}{0.120139in}}{%
\pgfpathmoveto{\pgfqpoint{0.000000in}{-0.120139in}}%
\pgfpathcurveto{\pgfqpoint{0.031861in}{-0.120139in}}{\pgfqpoint{0.062422in}{-0.107480in}}{\pgfqpoint{0.084951in}{-0.084951in}}%
\pgfpathcurveto{\pgfqpoint{0.107480in}{-0.062422in}}{\pgfqpoint{0.120139in}{-0.031861in}}{\pgfqpoint{0.120139in}{0.000000in}}%
\pgfpathcurveto{\pgfqpoint{0.120139in}{0.031861in}}{\pgfqpoint{0.107480in}{0.062422in}}{\pgfqpoint{0.084951in}{0.084951in}}%
\pgfpathcurveto{\pgfqpoint{0.062422in}{0.107480in}}{\pgfqpoint{0.031861in}{0.120139in}}{\pgfqpoint{0.000000in}{0.120139in}}%
\pgfpathcurveto{\pgfqpoint{-0.031861in}{0.120139in}}{\pgfqpoint{-0.062422in}{0.107480in}}{\pgfqpoint{-0.084951in}{0.084951in}}%
\pgfpathcurveto{\pgfqpoint{-0.107480in}{0.062422in}}{\pgfqpoint{-0.120139in}{0.031861in}}{\pgfqpoint{-0.120139in}{0.000000in}}%
\pgfpathcurveto{\pgfqpoint{-0.120139in}{-0.031861in}}{\pgfqpoint{-0.107480in}{-0.062422in}}{\pgfqpoint{-0.084951in}{-0.084951in}}%
\pgfpathcurveto{\pgfqpoint{-0.062422in}{-0.107480in}}{\pgfqpoint{-0.031861in}{-0.120139in}}{\pgfqpoint{0.000000in}{-0.120139in}}%
\pgfpathclose%
\pgfusepath{stroke,fill}%
}%
\begin{pgfscope}%
\pgfsys@transformshift{4.009207in}{2.183321in}%
\pgfsys@useobject{currentmarker}{}%
\end{pgfscope}%
\end{pgfscope}%
\begin{pgfscope}%
\pgfpathrectangle{\pgfqpoint{0.100000in}{0.100000in}}{\pgfqpoint{5.307240in}{3.397500in}}%
\pgfusepath{clip}%
\pgfsetrectcap%
\pgfsetroundjoin%
\pgfsetlinewidth{1.505625pt}%
\definecolor{currentstroke}{rgb}{0.678431,1.000000,0.184314}%
\pgfsetstrokecolor{currentstroke}%
\pgfsetstrokeopacity{0.500000}%
\pgfsetdash{}{0pt}%
\pgfpathmoveto{\pgfqpoint{4.147877in}{2.203115in}}%
\pgfusepath{stroke}%
\end{pgfscope}%
\begin{pgfscope}%
\pgfpathrectangle{\pgfqpoint{0.100000in}{0.100000in}}{\pgfqpoint{5.307240in}{3.397500in}}%
\pgfusepath{clip}%
\pgfsetbuttcap%
\pgfsetroundjoin%
\definecolor{currentfill}{rgb}{0.678431,1.000000,0.184314}%
\pgfsetfillcolor{currentfill}%
\pgfsetfillopacity{0.500000}%
\pgfsetlinewidth{0.250937pt}%
\definecolor{currentstroke}{rgb}{0.000000,0.000000,0.000000}%
\pgfsetstrokecolor{currentstroke}%
\pgfsetstrokeopacity{0.500000}%
\pgfsetdash{}{0pt}%
\pgfsys@defobject{currentmarker}{\pgfqpoint{-0.114583in}{-0.114583in}}{\pgfqpoint{0.114583in}{0.114583in}}{%
\pgfpathmoveto{\pgfqpoint{0.000000in}{-0.114583in}}%
\pgfpathcurveto{\pgfqpoint{0.030388in}{-0.114583in}}{\pgfqpoint{0.059535in}{-0.102510in}}{\pgfqpoint{0.081023in}{-0.081023in}}%
\pgfpathcurveto{\pgfqpoint{0.102510in}{-0.059535in}}{\pgfqpoint{0.114583in}{-0.030388in}}{\pgfqpoint{0.114583in}{0.000000in}}%
\pgfpathcurveto{\pgfqpoint{0.114583in}{0.030388in}}{\pgfqpoint{0.102510in}{0.059535in}}{\pgfqpoint{0.081023in}{0.081023in}}%
\pgfpathcurveto{\pgfqpoint{0.059535in}{0.102510in}}{\pgfqpoint{0.030388in}{0.114583in}}{\pgfqpoint{0.000000in}{0.114583in}}%
\pgfpathcurveto{\pgfqpoint{-0.030388in}{0.114583in}}{\pgfqpoint{-0.059535in}{0.102510in}}{\pgfqpoint{-0.081023in}{0.081023in}}%
\pgfpathcurveto{\pgfqpoint{-0.102510in}{0.059535in}}{\pgfqpoint{-0.114583in}{0.030388in}}{\pgfqpoint{-0.114583in}{0.000000in}}%
\pgfpathcurveto{\pgfqpoint{-0.114583in}{-0.030388in}}{\pgfqpoint{-0.102510in}{-0.059535in}}{\pgfqpoint{-0.081023in}{-0.081023in}}%
\pgfpathcurveto{\pgfqpoint{-0.059535in}{-0.102510in}}{\pgfqpoint{-0.030388in}{-0.114583in}}{\pgfqpoint{0.000000in}{-0.114583in}}%
\pgfpathclose%
\pgfusepath{stroke,fill}%
}%
\begin{pgfscope}%
\pgfsys@transformshift{4.147877in}{2.203115in}%
\pgfsys@useobject{currentmarker}{}%
\end{pgfscope}%
\end{pgfscope}%
\begin{pgfscope}%
\pgfpathrectangle{\pgfqpoint{0.100000in}{0.100000in}}{\pgfqpoint{5.307240in}{3.397500in}}%
\pgfusepath{clip}%
\pgfsetrectcap%
\pgfsetroundjoin%
\pgfsetlinewidth{1.505625pt}%
\definecolor{currentstroke}{rgb}{0.678431,1.000000,0.184314}%
\pgfsetstrokecolor{currentstroke}%
\pgfsetstrokeopacity{0.500000}%
\pgfsetdash{}{0pt}%
\pgfpathmoveto{\pgfqpoint{4.046821in}{2.092054in}}%
\pgfusepath{stroke}%
\end{pgfscope}%
\begin{pgfscope}%
\pgfpathrectangle{\pgfqpoint{0.100000in}{0.100000in}}{\pgfqpoint{5.307240in}{3.397500in}}%
\pgfusepath{clip}%
\pgfsetbuttcap%
\pgfsetroundjoin%
\definecolor{currentfill}{rgb}{0.678431,1.000000,0.184314}%
\pgfsetfillcolor{currentfill}%
\pgfsetfillopacity{0.500000}%
\pgfsetlinewidth{0.250937pt}%
\definecolor{currentstroke}{rgb}{0.000000,0.000000,0.000000}%
\pgfsetstrokecolor{currentstroke}%
\pgfsetstrokeopacity{0.500000}%
\pgfsetdash{}{0pt}%
\pgfsys@defobject{currentmarker}{\pgfqpoint{-0.096528in}{-0.096528in}}{\pgfqpoint{0.096528in}{0.096528in}}{%
\pgfpathmoveto{\pgfqpoint{0.000000in}{-0.096528in}}%
\pgfpathcurveto{\pgfqpoint{0.025599in}{-0.096528in}}{\pgfqpoint{0.050154in}{-0.086357in}}{\pgfqpoint{0.068255in}{-0.068255in}}%
\pgfpathcurveto{\pgfqpoint{0.086357in}{-0.050154in}}{\pgfqpoint{0.096528in}{-0.025599in}}{\pgfqpoint{0.096528in}{0.000000in}}%
\pgfpathcurveto{\pgfqpoint{0.096528in}{0.025599in}}{\pgfqpoint{0.086357in}{0.050154in}}{\pgfqpoint{0.068255in}{0.068255in}}%
\pgfpathcurveto{\pgfqpoint{0.050154in}{0.086357in}}{\pgfqpoint{0.025599in}{0.096528in}}{\pgfqpoint{0.000000in}{0.096528in}}%
\pgfpathcurveto{\pgfqpoint{-0.025599in}{0.096528in}}{\pgfqpoint{-0.050154in}{0.086357in}}{\pgfqpoint{-0.068255in}{0.068255in}}%
\pgfpathcurveto{\pgfqpoint{-0.086357in}{0.050154in}}{\pgfqpoint{-0.096528in}{0.025599in}}{\pgfqpoint{-0.096528in}{0.000000in}}%
\pgfpathcurveto{\pgfqpoint{-0.096528in}{-0.025599in}}{\pgfqpoint{-0.086357in}{-0.050154in}}{\pgfqpoint{-0.068255in}{-0.068255in}}%
\pgfpathcurveto{\pgfqpoint{-0.050154in}{-0.086357in}}{\pgfqpoint{-0.025599in}{-0.096528in}}{\pgfqpoint{0.000000in}{-0.096528in}}%
\pgfpathclose%
\pgfusepath{stroke,fill}%
}%
\begin{pgfscope}%
\pgfsys@transformshift{4.046821in}{2.092054in}%
\pgfsys@useobject{currentmarker}{}%
\end{pgfscope}%
\end{pgfscope}%
\begin{pgfscope}%
\pgfpathrectangle{\pgfqpoint{0.100000in}{0.100000in}}{\pgfqpoint{5.307240in}{3.397500in}}%
\pgfusepath{clip}%
\pgfsetrectcap%
\pgfsetroundjoin%
\pgfsetlinewidth{1.505625pt}%
\definecolor{currentstroke}{rgb}{0.678431,1.000000,0.184314}%
\pgfsetstrokecolor{currentstroke}%
\pgfsetstrokeopacity{0.500000}%
\pgfsetdash{}{0pt}%
\pgfpathmoveto{\pgfqpoint{4.047290in}{2.297766in}}%
\pgfusepath{stroke}%
\end{pgfscope}%
\begin{pgfscope}%
\pgfpathrectangle{\pgfqpoint{0.100000in}{0.100000in}}{\pgfqpoint{5.307240in}{3.397500in}}%
\pgfusepath{clip}%
\pgfsetbuttcap%
\pgfsetroundjoin%
\definecolor{currentfill}{rgb}{0.678431,1.000000,0.184314}%
\pgfsetfillcolor{currentfill}%
\pgfsetfillopacity{0.500000}%
\pgfsetlinewidth{0.250937pt}%
\definecolor{currentstroke}{rgb}{0.000000,0.000000,0.000000}%
\pgfsetstrokecolor{currentstroke}%
\pgfsetstrokeopacity{0.500000}%
\pgfsetdash{}{0pt}%
\pgfsys@defobject{currentmarker}{\pgfqpoint{-0.127083in}{-0.127083in}}{\pgfqpoint{0.127083in}{0.127083in}}{%
\pgfpathmoveto{\pgfqpoint{0.000000in}{-0.127083in}}%
\pgfpathcurveto{\pgfqpoint{0.033703in}{-0.127083in}}{\pgfqpoint{0.066030in}{-0.113693in}}{\pgfqpoint{0.089861in}{-0.089861in}}%
\pgfpathcurveto{\pgfqpoint{0.113693in}{-0.066030in}}{\pgfqpoint{0.127083in}{-0.033703in}}{\pgfqpoint{0.127083in}{0.000000in}}%
\pgfpathcurveto{\pgfqpoint{0.127083in}{0.033703in}}{\pgfqpoint{0.113693in}{0.066030in}}{\pgfqpoint{0.089861in}{0.089861in}}%
\pgfpathcurveto{\pgfqpoint{0.066030in}{0.113693in}}{\pgfqpoint{0.033703in}{0.127083in}}{\pgfqpoint{0.000000in}{0.127083in}}%
\pgfpathcurveto{\pgfqpoint{-0.033703in}{0.127083in}}{\pgfqpoint{-0.066030in}{0.113693in}}{\pgfqpoint{-0.089861in}{0.089861in}}%
\pgfpathcurveto{\pgfqpoint{-0.113693in}{0.066030in}}{\pgfqpoint{-0.127083in}{0.033703in}}{\pgfqpoint{-0.127083in}{0.000000in}}%
\pgfpathcurveto{\pgfqpoint{-0.127083in}{-0.033703in}}{\pgfqpoint{-0.113693in}{-0.066030in}}{\pgfqpoint{-0.089861in}{-0.089861in}}%
\pgfpathcurveto{\pgfqpoint{-0.066030in}{-0.113693in}}{\pgfqpoint{-0.033703in}{-0.127083in}}{\pgfqpoint{0.000000in}{-0.127083in}}%
\pgfpathclose%
\pgfusepath{stroke,fill}%
}%
\begin{pgfscope}%
\pgfsys@transformshift{4.047290in}{2.297766in}%
\pgfsys@useobject{currentmarker}{}%
\end{pgfscope}%
\end{pgfscope}%
\begin{pgfscope}%
\pgfpathrectangle{\pgfqpoint{0.100000in}{0.100000in}}{\pgfqpoint{5.307240in}{3.397500in}}%
\pgfusepath{clip}%
\pgfsetrectcap%
\pgfsetroundjoin%
\pgfsetlinewidth{1.505625pt}%
\definecolor{currentstroke}{rgb}{0.678431,1.000000,0.184314}%
\pgfsetstrokecolor{currentstroke}%
\pgfsetstrokeopacity{0.500000}%
\pgfsetdash{}{0pt}%
\pgfpathmoveto{\pgfqpoint{4.321915in}{2.189151in}}%
\pgfusepath{stroke}%
\end{pgfscope}%
\begin{pgfscope}%
\pgfpathrectangle{\pgfqpoint{0.100000in}{0.100000in}}{\pgfqpoint{5.307240in}{3.397500in}}%
\pgfusepath{clip}%
\pgfsetbuttcap%
\pgfsetroundjoin%
\definecolor{currentfill}{rgb}{0.678431,1.000000,0.184314}%
\pgfsetfillcolor{currentfill}%
\pgfsetfillopacity{0.500000}%
\pgfsetlinewidth{0.250937pt}%
\definecolor{currentstroke}{rgb}{0.000000,0.000000,0.000000}%
\pgfsetstrokecolor{currentstroke}%
\pgfsetstrokeopacity{0.500000}%
\pgfsetdash{}{0pt}%
\pgfsys@defobject{currentmarker}{\pgfqpoint{-0.093750in}{-0.093750in}}{\pgfqpoint{0.093750in}{0.093750in}}{%
\pgfpathmoveto{\pgfqpoint{0.000000in}{-0.093750in}}%
\pgfpathcurveto{\pgfqpoint{0.024863in}{-0.093750in}}{\pgfqpoint{0.048711in}{-0.083872in}}{\pgfqpoint{0.066291in}{-0.066291in}}%
\pgfpathcurveto{\pgfqpoint{0.083872in}{-0.048711in}}{\pgfqpoint{0.093750in}{-0.024863in}}{\pgfqpoint{0.093750in}{0.000000in}}%
\pgfpathcurveto{\pgfqpoint{0.093750in}{0.024863in}}{\pgfqpoint{0.083872in}{0.048711in}}{\pgfqpoint{0.066291in}{0.066291in}}%
\pgfpathcurveto{\pgfqpoint{0.048711in}{0.083872in}}{\pgfqpoint{0.024863in}{0.093750in}}{\pgfqpoint{0.000000in}{0.093750in}}%
\pgfpathcurveto{\pgfqpoint{-0.024863in}{0.093750in}}{\pgfqpoint{-0.048711in}{0.083872in}}{\pgfqpoint{-0.066291in}{0.066291in}}%
\pgfpathcurveto{\pgfqpoint{-0.083872in}{0.048711in}}{\pgfqpoint{-0.093750in}{0.024863in}}{\pgfqpoint{-0.093750in}{0.000000in}}%
\pgfpathcurveto{\pgfqpoint{-0.093750in}{-0.024863in}}{\pgfqpoint{-0.083872in}{-0.048711in}}{\pgfqpoint{-0.066291in}{-0.066291in}}%
\pgfpathcurveto{\pgfqpoint{-0.048711in}{-0.083872in}}{\pgfqpoint{-0.024863in}{-0.093750in}}{\pgfqpoint{0.000000in}{-0.093750in}}%
\pgfpathclose%
\pgfusepath{stroke,fill}%
}%
\begin{pgfscope}%
\pgfsys@transformshift{4.321915in}{2.189151in}%
\pgfsys@useobject{currentmarker}{}%
\end{pgfscope}%
\end{pgfscope}%
\begin{pgfscope}%
\pgfpathrectangle{\pgfqpoint{0.100000in}{0.100000in}}{\pgfqpoint{5.307240in}{3.397500in}}%
\pgfusepath{clip}%
\pgfsetrectcap%
\pgfsetroundjoin%
\pgfsetlinewidth{1.505625pt}%
\definecolor{currentstroke}{rgb}{0.678431,1.000000,0.184314}%
\pgfsetstrokecolor{currentstroke}%
\pgfsetstrokeopacity{0.500000}%
\pgfsetdash{}{0pt}%
\pgfpathmoveto{\pgfqpoint{4.304218in}{2.266509in}}%
\pgfusepath{stroke}%
\end{pgfscope}%
\begin{pgfscope}%
\pgfpathrectangle{\pgfqpoint{0.100000in}{0.100000in}}{\pgfqpoint{5.307240in}{3.397500in}}%
\pgfusepath{clip}%
\pgfsetbuttcap%
\pgfsetroundjoin%
\definecolor{currentfill}{rgb}{0.678431,1.000000,0.184314}%
\pgfsetfillcolor{currentfill}%
\pgfsetfillopacity{0.500000}%
\pgfsetlinewidth{0.250937pt}%
\definecolor{currentstroke}{rgb}{0.000000,0.000000,0.000000}%
\pgfsetstrokecolor{currentstroke}%
\pgfsetstrokeopacity{0.500000}%
\pgfsetdash{}{0pt}%
\pgfsys@defobject{currentmarker}{\pgfqpoint{-0.101389in}{-0.101389in}}{\pgfqpoint{0.101389in}{0.101389in}}{%
\pgfpathmoveto{\pgfqpoint{0.000000in}{-0.101389in}}%
\pgfpathcurveto{\pgfqpoint{0.026889in}{-0.101389in}}{\pgfqpoint{0.052680in}{-0.090706in}}{\pgfqpoint{0.071693in}{-0.071693in}}%
\pgfpathcurveto{\pgfqpoint{0.090706in}{-0.052680in}}{\pgfqpoint{0.101389in}{-0.026889in}}{\pgfqpoint{0.101389in}{0.000000in}}%
\pgfpathcurveto{\pgfqpoint{0.101389in}{0.026889in}}{\pgfqpoint{0.090706in}{0.052680in}}{\pgfqpoint{0.071693in}{0.071693in}}%
\pgfpathcurveto{\pgfqpoint{0.052680in}{0.090706in}}{\pgfqpoint{0.026889in}{0.101389in}}{\pgfqpoint{0.000000in}{0.101389in}}%
\pgfpathcurveto{\pgfqpoint{-0.026889in}{0.101389in}}{\pgfqpoint{-0.052680in}{0.090706in}}{\pgfqpoint{-0.071693in}{0.071693in}}%
\pgfpathcurveto{\pgfqpoint{-0.090706in}{0.052680in}}{\pgfqpoint{-0.101389in}{0.026889in}}{\pgfqpoint{-0.101389in}{0.000000in}}%
\pgfpathcurveto{\pgfqpoint{-0.101389in}{-0.026889in}}{\pgfqpoint{-0.090706in}{-0.052680in}}{\pgfqpoint{-0.071693in}{-0.071693in}}%
\pgfpathcurveto{\pgfqpoint{-0.052680in}{-0.090706in}}{\pgfqpoint{-0.026889in}{-0.101389in}}{\pgfqpoint{0.000000in}{-0.101389in}}%
\pgfpathclose%
\pgfusepath{stroke,fill}%
}%
\begin{pgfscope}%
\pgfsys@transformshift{4.304218in}{2.266509in}%
\pgfsys@useobject{currentmarker}{}%
\end{pgfscope}%
\end{pgfscope}%
\begin{pgfscope}%
\pgfpathrectangle{\pgfqpoint{0.100000in}{0.100000in}}{\pgfqpoint{5.307240in}{3.397500in}}%
\pgfusepath{clip}%
\pgfsetrectcap%
\pgfsetroundjoin%
\pgfsetlinewidth{1.505625pt}%
\definecolor{currentstroke}{rgb}{0.678431,1.000000,0.184314}%
\pgfsetstrokecolor{currentstroke}%
\pgfsetstrokeopacity{0.500000}%
\pgfsetdash{}{0pt}%
\pgfpathmoveto{\pgfqpoint{2.726246in}{1.418306in}}%
\pgfusepath{stroke}%
\end{pgfscope}%
\begin{pgfscope}%
\pgfpathrectangle{\pgfqpoint{0.100000in}{0.100000in}}{\pgfqpoint{5.307240in}{3.397500in}}%
\pgfusepath{clip}%
\pgfsetbuttcap%
\pgfsetroundjoin%
\definecolor{currentfill}{rgb}{0.678431,1.000000,0.184314}%
\pgfsetfillcolor{currentfill}%
\pgfsetfillopacity{0.500000}%
\pgfsetlinewidth{0.250937pt}%
\definecolor{currentstroke}{rgb}{0.000000,0.000000,0.000000}%
\pgfsetstrokecolor{currentstroke}%
\pgfsetstrokeopacity{0.500000}%
\pgfsetdash{}{0pt}%
\pgfsys@defobject{currentmarker}{\pgfqpoint{-0.111111in}{-0.111111in}}{\pgfqpoint{0.111111in}{0.111111in}}{%
\pgfpathmoveto{\pgfqpoint{0.000000in}{-0.111111in}}%
\pgfpathcurveto{\pgfqpoint{0.029467in}{-0.111111in}}{\pgfqpoint{0.057731in}{-0.099404in}}{\pgfqpoint{0.078567in}{-0.078567in}}%
\pgfpathcurveto{\pgfqpoint{0.099404in}{-0.057731in}}{\pgfqpoint{0.111111in}{-0.029467in}}{\pgfqpoint{0.111111in}{0.000000in}}%
\pgfpathcurveto{\pgfqpoint{0.111111in}{0.029467in}}{\pgfqpoint{0.099404in}{0.057731in}}{\pgfqpoint{0.078567in}{0.078567in}}%
\pgfpathcurveto{\pgfqpoint{0.057731in}{0.099404in}}{\pgfqpoint{0.029467in}{0.111111in}}{\pgfqpoint{0.000000in}{0.111111in}}%
\pgfpathcurveto{\pgfqpoint{-0.029467in}{0.111111in}}{\pgfqpoint{-0.057731in}{0.099404in}}{\pgfqpoint{-0.078567in}{0.078567in}}%
\pgfpathcurveto{\pgfqpoint{-0.099404in}{0.057731in}}{\pgfqpoint{-0.111111in}{0.029467in}}{\pgfqpoint{-0.111111in}{0.000000in}}%
\pgfpathcurveto{\pgfqpoint{-0.111111in}{-0.029467in}}{\pgfqpoint{-0.099404in}{-0.057731in}}{\pgfqpoint{-0.078567in}{-0.078567in}}%
\pgfpathcurveto{\pgfqpoint{-0.057731in}{-0.099404in}}{\pgfqpoint{-0.029467in}{-0.111111in}}{\pgfqpoint{0.000000in}{-0.111111in}}%
\pgfpathclose%
\pgfusepath{stroke,fill}%
}%
\begin{pgfscope}%
\pgfsys@transformshift{2.726246in}{1.418306in}%
\pgfsys@useobject{currentmarker}{}%
\end{pgfscope}%
\end{pgfscope}%
\begin{pgfscope}%
\pgfpathrectangle{\pgfqpoint{0.100000in}{0.100000in}}{\pgfqpoint{5.307240in}{3.397500in}}%
\pgfusepath{clip}%
\pgfsetrectcap%
\pgfsetroundjoin%
\pgfsetlinewidth{1.505625pt}%
\definecolor{currentstroke}{rgb}{0.678431,1.000000,0.184314}%
\pgfsetstrokecolor{currentstroke}%
\pgfsetstrokeopacity{0.500000}%
\pgfsetdash{}{0pt}%
\pgfpathmoveto{\pgfqpoint{2.812796in}{1.516267in}}%
\pgfusepath{stroke}%
\end{pgfscope}%
\begin{pgfscope}%
\pgfpathrectangle{\pgfqpoint{0.100000in}{0.100000in}}{\pgfqpoint{5.307240in}{3.397500in}}%
\pgfusepath{clip}%
\pgfsetbuttcap%
\pgfsetroundjoin%
\definecolor{currentfill}{rgb}{0.678431,1.000000,0.184314}%
\pgfsetfillcolor{currentfill}%
\pgfsetfillopacity{0.500000}%
\pgfsetlinewidth{0.250937pt}%
\definecolor{currentstroke}{rgb}{0.000000,0.000000,0.000000}%
\pgfsetstrokecolor{currentstroke}%
\pgfsetstrokeopacity{0.500000}%
\pgfsetdash{}{0pt}%
\pgfsys@defobject{currentmarker}{\pgfqpoint{-0.084722in}{-0.084722in}}{\pgfqpoint{0.084722in}{0.084722in}}{%
\pgfpathmoveto{\pgfqpoint{0.000000in}{-0.084722in}}%
\pgfpathcurveto{\pgfqpoint{0.022469in}{-0.084722in}}{\pgfqpoint{0.044020in}{-0.075795in}}{\pgfqpoint{0.059908in}{-0.059908in}}%
\pgfpathcurveto{\pgfqpoint{0.075795in}{-0.044020in}}{\pgfqpoint{0.084722in}{-0.022469in}}{\pgfqpoint{0.084722in}{0.000000in}}%
\pgfpathcurveto{\pgfqpoint{0.084722in}{0.022469in}}{\pgfqpoint{0.075795in}{0.044020in}}{\pgfqpoint{0.059908in}{0.059908in}}%
\pgfpathcurveto{\pgfqpoint{0.044020in}{0.075795in}}{\pgfqpoint{0.022469in}{0.084722in}}{\pgfqpoint{0.000000in}{0.084722in}}%
\pgfpathcurveto{\pgfqpoint{-0.022469in}{0.084722in}}{\pgfqpoint{-0.044020in}{0.075795in}}{\pgfqpoint{-0.059908in}{0.059908in}}%
\pgfpathcurveto{\pgfqpoint{-0.075795in}{0.044020in}}{\pgfqpoint{-0.084722in}{0.022469in}}{\pgfqpoint{-0.084722in}{0.000000in}}%
\pgfpathcurveto{\pgfqpoint{-0.084722in}{-0.022469in}}{\pgfqpoint{-0.075795in}{-0.044020in}}{\pgfqpoint{-0.059908in}{-0.059908in}}%
\pgfpathcurveto{\pgfqpoint{-0.044020in}{-0.075795in}}{\pgfqpoint{-0.022469in}{-0.084722in}}{\pgfqpoint{0.000000in}{-0.084722in}}%
\pgfpathclose%
\pgfusepath{stroke,fill}%
}%
\begin{pgfscope}%
\pgfsys@transformshift{2.812796in}{1.516267in}%
\pgfsys@useobject{currentmarker}{}%
\end{pgfscope}%
\end{pgfscope}%
\begin{pgfscope}%
\pgfpathrectangle{\pgfqpoint{0.100000in}{0.100000in}}{\pgfqpoint{5.307240in}{3.397500in}}%
\pgfusepath{clip}%
\pgfsetrectcap%
\pgfsetroundjoin%
\pgfsetlinewidth{1.505625pt}%
\definecolor{currentstroke}{rgb}{0.678431,1.000000,0.184314}%
\pgfsetstrokecolor{currentstroke}%
\pgfsetstrokeopacity{0.500000}%
\pgfsetdash{}{0pt}%
\pgfpathmoveto{\pgfqpoint{2.958212in}{1.592961in}}%
\pgfusepath{stroke}%
\end{pgfscope}%
\begin{pgfscope}%
\pgfpathrectangle{\pgfqpoint{0.100000in}{0.100000in}}{\pgfqpoint{5.307240in}{3.397500in}}%
\pgfusepath{clip}%
\pgfsetbuttcap%
\pgfsetroundjoin%
\definecolor{currentfill}{rgb}{0.678431,1.000000,0.184314}%
\pgfsetfillcolor{currentfill}%
\pgfsetfillopacity{0.500000}%
\pgfsetlinewidth{0.250937pt}%
\definecolor{currentstroke}{rgb}{0.000000,0.000000,0.000000}%
\pgfsetstrokecolor{currentstroke}%
\pgfsetstrokeopacity{0.500000}%
\pgfsetdash{}{0pt}%
\pgfsys@defobject{currentmarker}{\pgfqpoint{-0.084722in}{-0.084722in}}{\pgfqpoint{0.084722in}{0.084722in}}{%
\pgfpathmoveto{\pgfqpoint{0.000000in}{-0.084722in}}%
\pgfpathcurveto{\pgfqpoint{0.022469in}{-0.084722in}}{\pgfqpoint{0.044020in}{-0.075795in}}{\pgfqpoint{0.059908in}{-0.059908in}}%
\pgfpathcurveto{\pgfqpoint{0.075795in}{-0.044020in}}{\pgfqpoint{0.084722in}{-0.022469in}}{\pgfqpoint{0.084722in}{0.000000in}}%
\pgfpathcurveto{\pgfqpoint{0.084722in}{0.022469in}}{\pgfqpoint{0.075795in}{0.044020in}}{\pgfqpoint{0.059908in}{0.059908in}}%
\pgfpathcurveto{\pgfqpoint{0.044020in}{0.075795in}}{\pgfqpoint{0.022469in}{0.084722in}}{\pgfqpoint{0.000000in}{0.084722in}}%
\pgfpathcurveto{\pgfqpoint{-0.022469in}{0.084722in}}{\pgfqpoint{-0.044020in}{0.075795in}}{\pgfqpoint{-0.059908in}{0.059908in}}%
\pgfpathcurveto{\pgfqpoint{-0.075795in}{0.044020in}}{\pgfqpoint{-0.084722in}{0.022469in}}{\pgfqpoint{-0.084722in}{0.000000in}}%
\pgfpathcurveto{\pgfqpoint{-0.084722in}{-0.022469in}}{\pgfqpoint{-0.075795in}{-0.044020in}}{\pgfqpoint{-0.059908in}{-0.059908in}}%
\pgfpathcurveto{\pgfqpoint{-0.044020in}{-0.075795in}}{\pgfqpoint{-0.022469in}{-0.084722in}}{\pgfqpoint{0.000000in}{-0.084722in}}%
\pgfpathclose%
\pgfusepath{stroke,fill}%
}%
\begin{pgfscope}%
\pgfsys@transformshift{2.958212in}{1.592961in}%
\pgfsys@useobject{currentmarker}{}%
\end{pgfscope}%
\end{pgfscope}%
\begin{pgfscope}%
\pgfpathrectangle{\pgfqpoint{0.100000in}{0.100000in}}{\pgfqpoint{5.307240in}{3.397500in}}%
\pgfusepath{clip}%
\pgfsetrectcap%
\pgfsetroundjoin%
\pgfsetlinewidth{1.505625pt}%
\definecolor{currentstroke}{rgb}{0.678431,1.000000,0.184314}%
\pgfsetstrokecolor{currentstroke}%
\pgfsetstrokeopacity{0.500000}%
\pgfsetdash{}{0pt}%
\pgfpathmoveto{\pgfqpoint{0.752388in}{2.937918in}}%
\pgfusepath{stroke}%
\end{pgfscope}%
\begin{pgfscope}%
\pgfpathrectangle{\pgfqpoint{0.100000in}{0.100000in}}{\pgfqpoint{5.307240in}{3.397500in}}%
\pgfusepath{clip}%
\pgfsetbuttcap%
\pgfsetroundjoin%
\definecolor{currentfill}{rgb}{0.678431,1.000000,0.184314}%
\pgfsetfillcolor{currentfill}%
\pgfsetfillopacity{0.500000}%
\pgfsetlinewidth{0.250937pt}%
\definecolor{currentstroke}{rgb}{0.000000,0.000000,0.000000}%
\pgfsetstrokecolor{currentstroke}%
\pgfsetstrokeopacity{0.500000}%
\pgfsetdash{}{0pt}%
\pgfsys@defobject{currentmarker}{\pgfqpoint{-0.077778in}{-0.077778in}}{\pgfqpoint{0.077778in}{0.077778in}}{%
\pgfpathmoveto{\pgfqpoint{0.000000in}{-0.077778in}}%
\pgfpathcurveto{\pgfqpoint{0.020627in}{-0.077778in}}{\pgfqpoint{0.040412in}{-0.069583in}}{\pgfqpoint{0.054997in}{-0.054997in}}%
\pgfpathcurveto{\pgfqpoint{0.069583in}{-0.040412in}}{\pgfqpoint{0.077778in}{-0.020627in}}{\pgfqpoint{0.077778in}{0.000000in}}%
\pgfpathcurveto{\pgfqpoint{0.077778in}{0.020627in}}{\pgfqpoint{0.069583in}{0.040412in}}{\pgfqpoint{0.054997in}{0.054997in}}%
\pgfpathcurveto{\pgfqpoint{0.040412in}{0.069583in}}{\pgfqpoint{0.020627in}{0.077778in}}{\pgfqpoint{0.000000in}{0.077778in}}%
\pgfpathcurveto{\pgfqpoint{-0.020627in}{0.077778in}}{\pgfqpoint{-0.040412in}{0.069583in}}{\pgfqpoint{-0.054997in}{0.054997in}}%
\pgfpathcurveto{\pgfqpoint{-0.069583in}{0.040412in}}{\pgfqpoint{-0.077778in}{0.020627in}}{\pgfqpoint{-0.077778in}{0.000000in}}%
\pgfpathcurveto{\pgfqpoint{-0.077778in}{-0.020627in}}{\pgfqpoint{-0.069583in}{-0.040412in}}{\pgfqpoint{-0.054997in}{-0.054997in}}%
\pgfpathcurveto{\pgfqpoint{-0.040412in}{-0.069583in}}{\pgfqpoint{-0.020627in}{-0.077778in}}{\pgfqpoint{0.000000in}{-0.077778in}}%
\pgfpathclose%
\pgfusepath{stroke,fill}%
}%
\begin{pgfscope}%
\pgfsys@transformshift{0.752388in}{2.937918in}%
\pgfsys@useobject{currentmarker}{}%
\end{pgfscope}%
\end{pgfscope}%
\begin{pgfscope}%
\pgfpathrectangle{\pgfqpoint{0.100000in}{0.100000in}}{\pgfqpoint{5.307240in}{3.397500in}}%
\pgfusepath{clip}%
\pgfsetrectcap%
\pgfsetroundjoin%
\pgfsetlinewidth{1.505625pt}%
\definecolor{currentstroke}{rgb}{0.678431,1.000000,0.184314}%
\pgfsetstrokecolor{currentstroke}%
\pgfsetstrokeopacity{0.500000}%
\pgfsetdash{}{0pt}%
\pgfpathmoveto{\pgfqpoint{0.875322in}{2.829304in}}%
\pgfusepath{stroke}%
\end{pgfscope}%
\begin{pgfscope}%
\pgfpathrectangle{\pgfqpoint{0.100000in}{0.100000in}}{\pgfqpoint{5.307240in}{3.397500in}}%
\pgfusepath{clip}%
\pgfsetbuttcap%
\pgfsetroundjoin%
\definecolor{currentfill}{rgb}{0.678431,1.000000,0.184314}%
\pgfsetfillcolor{currentfill}%
\pgfsetfillopacity{0.500000}%
\pgfsetlinewidth{0.250937pt}%
\definecolor{currentstroke}{rgb}{0.000000,0.000000,0.000000}%
\pgfsetstrokecolor{currentstroke}%
\pgfsetstrokeopacity{0.500000}%
\pgfsetdash{}{0pt}%
\pgfsys@defobject{currentmarker}{\pgfqpoint{-0.099306in}{-0.099306in}}{\pgfqpoint{0.099306in}{0.099306in}}{%
\pgfpathmoveto{\pgfqpoint{0.000000in}{-0.099306in}}%
\pgfpathcurveto{\pgfqpoint{0.026336in}{-0.099306in}}{\pgfqpoint{0.051597in}{-0.088842in}}{\pgfqpoint{0.070220in}{-0.070220in}}%
\pgfpathcurveto{\pgfqpoint{0.088842in}{-0.051597in}}{\pgfqpoint{0.099306in}{-0.026336in}}{\pgfqpoint{0.099306in}{0.000000in}}%
\pgfpathcurveto{\pgfqpoint{0.099306in}{0.026336in}}{\pgfqpoint{0.088842in}{0.051597in}}{\pgfqpoint{0.070220in}{0.070220in}}%
\pgfpathcurveto{\pgfqpoint{0.051597in}{0.088842in}}{\pgfqpoint{0.026336in}{0.099306in}}{\pgfqpoint{0.000000in}{0.099306in}}%
\pgfpathcurveto{\pgfqpoint{-0.026336in}{0.099306in}}{\pgfqpoint{-0.051597in}{0.088842in}}{\pgfqpoint{-0.070220in}{0.070220in}}%
\pgfpathcurveto{\pgfqpoint{-0.088842in}{0.051597in}}{\pgfqpoint{-0.099306in}{0.026336in}}{\pgfqpoint{-0.099306in}{0.000000in}}%
\pgfpathcurveto{\pgfqpoint{-0.099306in}{-0.026336in}}{\pgfqpoint{-0.088842in}{-0.051597in}}{\pgfqpoint{-0.070220in}{-0.070220in}}%
\pgfpathcurveto{\pgfqpoint{-0.051597in}{-0.088842in}}{\pgfqpoint{-0.026336in}{-0.099306in}}{\pgfqpoint{0.000000in}{-0.099306in}}%
\pgfpathclose%
\pgfusepath{stroke,fill}%
}%
\begin{pgfscope}%
\pgfsys@transformshift{0.875322in}{2.829304in}%
\pgfsys@useobject{currentmarker}{}%
\end{pgfscope}%
\end{pgfscope}%
\begin{pgfscope}%
\pgfpathrectangle{\pgfqpoint{0.100000in}{0.100000in}}{\pgfqpoint{5.307240in}{3.397500in}}%
\pgfusepath{clip}%
\pgfsetrectcap%
\pgfsetroundjoin%
\pgfsetlinewidth{1.505625pt}%
\definecolor{currentstroke}{rgb}{0.678431,1.000000,0.184314}%
\pgfsetstrokecolor{currentstroke}%
\pgfsetstrokeopacity{0.500000}%
\pgfsetdash{}{0pt}%
\pgfpathmoveto{\pgfqpoint{0.737460in}{2.933878in}}%
\pgfusepath{stroke}%
\end{pgfscope}%
\begin{pgfscope}%
\pgfpathrectangle{\pgfqpoint{0.100000in}{0.100000in}}{\pgfqpoint{5.307240in}{3.397500in}}%
\pgfusepath{clip}%
\pgfsetbuttcap%
\pgfsetroundjoin%
\definecolor{currentfill}{rgb}{0.678431,1.000000,0.184314}%
\pgfsetfillcolor{currentfill}%
\pgfsetfillopacity{0.500000}%
\pgfsetlinewidth{0.250937pt}%
\definecolor{currentstroke}{rgb}{0.000000,0.000000,0.000000}%
\pgfsetstrokecolor{currentstroke}%
\pgfsetstrokeopacity{0.500000}%
\pgfsetdash{}{0pt}%
\pgfsys@defobject{currentmarker}{\pgfqpoint{-0.051389in}{-0.051389in}}{\pgfqpoint{0.051389in}{0.051389in}}{%
\pgfpathmoveto{\pgfqpoint{0.000000in}{-0.051389in}}%
\pgfpathcurveto{\pgfqpoint{0.013628in}{-0.051389in}}{\pgfqpoint{0.026701in}{-0.045974in}}{\pgfqpoint{0.036337in}{-0.036337in}}%
\pgfpathcurveto{\pgfqpoint{0.045974in}{-0.026701in}}{\pgfqpoint{0.051389in}{-0.013628in}}{\pgfqpoint{0.051389in}{0.000000in}}%
\pgfpathcurveto{\pgfqpoint{0.051389in}{0.013628in}}{\pgfqpoint{0.045974in}{0.026701in}}{\pgfqpoint{0.036337in}{0.036337in}}%
\pgfpathcurveto{\pgfqpoint{0.026701in}{0.045974in}}{\pgfqpoint{0.013628in}{0.051389in}}{\pgfqpoint{0.000000in}{0.051389in}}%
\pgfpathcurveto{\pgfqpoint{-0.013628in}{0.051389in}}{\pgfqpoint{-0.026701in}{0.045974in}}{\pgfqpoint{-0.036337in}{0.036337in}}%
\pgfpathcurveto{\pgfqpoint{-0.045974in}{0.026701in}}{\pgfqpoint{-0.051389in}{0.013628in}}{\pgfqpoint{-0.051389in}{0.000000in}}%
\pgfpathcurveto{\pgfqpoint{-0.051389in}{-0.013628in}}{\pgfqpoint{-0.045974in}{-0.026701in}}{\pgfqpoint{-0.036337in}{-0.036337in}}%
\pgfpathcurveto{\pgfqpoint{-0.026701in}{-0.045974in}}{\pgfqpoint{-0.013628in}{-0.051389in}}{\pgfqpoint{0.000000in}{-0.051389in}}%
\pgfpathclose%
\pgfusepath{stroke,fill}%
}%
\begin{pgfscope}%
\pgfsys@transformshift{0.737460in}{2.933878in}%
\pgfsys@useobject{currentmarker}{}%
\end{pgfscope}%
\end{pgfscope}%
\begin{pgfscope}%
\pgfpathrectangle{\pgfqpoint{0.100000in}{0.100000in}}{\pgfqpoint{5.307240in}{3.397500in}}%
\pgfusepath{clip}%
\pgfsetrectcap%
\pgfsetroundjoin%
\pgfsetlinewidth{1.505625pt}%
\definecolor{currentstroke}{rgb}{0.678431,1.000000,0.184314}%
\pgfsetstrokecolor{currentstroke}%
\pgfsetstrokeopacity{0.500000}%
\pgfsetdash{}{0pt}%
\pgfpathmoveto{\pgfqpoint{0.729656in}{2.879565in}}%
\pgfusepath{stroke}%
\end{pgfscope}%
\begin{pgfscope}%
\pgfpathrectangle{\pgfqpoint{0.100000in}{0.100000in}}{\pgfqpoint{5.307240in}{3.397500in}}%
\pgfusepath{clip}%
\pgfsetbuttcap%
\pgfsetroundjoin%
\definecolor{currentfill}{rgb}{0.678431,1.000000,0.184314}%
\pgfsetfillcolor{currentfill}%
\pgfsetfillopacity{0.500000}%
\pgfsetlinewidth{0.250937pt}%
\definecolor{currentstroke}{rgb}{0.000000,0.000000,0.000000}%
\pgfsetstrokecolor{currentstroke}%
\pgfsetstrokeopacity{0.500000}%
\pgfsetdash{}{0pt}%
\pgfsys@defobject{currentmarker}{\pgfqpoint{-0.082639in}{-0.082639in}}{\pgfqpoint{0.082639in}{0.082639in}}{%
\pgfpathmoveto{\pgfqpoint{0.000000in}{-0.082639in}}%
\pgfpathcurveto{\pgfqpoint{0.021916in}{-0.082639in}}{\pgfqpoint{0.042938in}{-0.073932in}}{\pgfqpoint{0.058435in}{-0.058435in}}%
\pgfpathcurveto{\pgfqpoint{0.073932in}{-0.042938in}}{\pgfqpoint{0.082639in}{-0.021916in}}{\pgfqpoint{0.082639in}{0.000000in}}%
\pgfpathcurveto{\pgfqpoint{0.082639in}{0.021916in}}{\pgfqpoint{0.073932in}{0.042938in}}{\pgfqpoint{0.058435in}{0.058435in}}%
\pgfpathcurveto{\pgfqpoint{0.042938in}{0.073932in}}{\pgfqpoint{0.021916in}{0.082639in}}{\pgfqpoint{0.000000in}{0.082639in}}%
\pgfpathcurveto{\pgfqpoint{-0.021916in}{0.082639in}}{\pgfqpoint{-0.042938in}{0.073932in}}{\pgfqpoint{-0.058435in}{0.058435in}}%
\pgfpathcurveto{\pgfqpoint{-0.073932in}{0.042938in}}{\pgfqpoint{-0.082639in}{0.021916in}}{\pgfqpoint{-0.082639in}{0.000000in}}%
\pgfpathcurveto{\pgfqpoint{-0.082639in}{-0.021916in}}{\pgfqpoint{-0.073932in}{-0.042938in}}{\pgfqpoint{-0.058435in}{-0.058435in}}%
\pgfpathcurveto{\pgfqpoint{-0.042938in}{-0.073932in}}{\pgfqpoint{-0.021916in}{-0.082639in}}{\pgfqpoint{0.000000in}{-0.082639in}}%
\pgfpathclose%
\pgfusepath{stroke,fill}%
}%
\begin{pgfscope}%
\pgfsys@transformshift{0.729656in}{2.879565in}%
\pgfsys@useobject{currentmarker}{}%
\end{pgfscope}%
\end{pgfscope}%
\begin{pgfscope}%
\pgfpathrectangle{\pgfqpoint{0.100000in}{0.100000in}}{\pgfqpoint{5.307240in}{3.397500in}}%
\pgfusepath{clip}%
\pgfsetrectcap%
\pgfsetroundjoin%
\pgfsetlinewidth{1.505625pt}%
\definecolor{currentstroke}{rgb}{0.678431,1.000000,0.184314}%
\pgfsetstrokecolor{currentstroke}%
\pgfsetstrokeopacity{0.500000}%
\pgfsetdash{}{0pt}%
\pgfpathmoveto{\pgfqpoint{0.656291in}{2.699662in}}%
\pgfusepath{stroke}%
\end{pgfscope}%
\begin{pgfscope}%
\pgfpathrectangle{\pgfqpoint{0.100000in}{0.100000in}}{\pgfqpoint{5.307240in}{3.397500in}}%
\pgfusepath{clip}%
\pgfsetbuttcap%
\pgfsetroundjoin%
\definecolor{currentfill}{rgb}{0.678431,1.000000,0.184314}%
\pgfsetfillcolor{currentfill}%
\pgfsetfillopacity{0.500000}%
\pgfsetlinewidth{0.250937pt}%
\definecolor{currentstroke}{rgb}{0.000000,0.000000,0.000000}%
\pgfsetstrokecolor{currentstroke}%
\pgfsetstrokeopacity{0.500000}%
\pgfsetdash{}{0pt}%
\pgfsys@defobject{currentmarker}{\pgfqpoint{-0.072222in}{-0.072222in}}{\pgfqpoint{0.072222in}{0.072222in}}{%
\pgfpathmoveto{\pgfqpoint{0.000000in}{-0.072222in}}%
\pgfpathcurveto{\pgfqpoint{0.019154in}{-0.072222in}}{\pgfqpoint{0.037525in}{-0.064612in}}{\pgfqpoint{0.051069in}{-0.051069in}}%
\pgfpathcurveto{\pgfqpoint{0.064612in}{-0.037525in}}{\pgfqpoint{0.072222in}{-0.019154in}}{\pgfqpoint{0.072222in}{0.000000in}}%
\pgfpathcurveto{\pgfqpoint{0.072222in}{0.019154in}}{\pgfqpoint{0.064612in}{0.037525in}}{\pgfqpoint{0.051069in}{0.051069in}}%
\pgfpathcurveto{\pgfqpoint{0.037525in}{0.064612in}}{\pgfqpoint{0.019154in}{0.072222in}}{\pgfqpoint{0.000000in}{0.072222in}}%
\pgfpathcurveto{\pgfqpoint{-0.019154in}{0.072222in}}{\pgfqpoint{-0.037525in}{0.064612in}}{\pgfqpoint{-0.051069in}{0.051069in}}%
\pgfpathcurveto{\pgfqpoint{-0.064612in}{0.037525in}}{\pgfqpoint{-0.072222in}{0.019154in}}{\pgfqpoint{-0.072222in}{0.000000in}}%
\pgfpathcurveto{\pgfqpoint{-0.072222in}{-0.019154in}}{\pgfqpoint{-0.064612in}{-0.037525in}}{\pgfqpoint{-0.051069in}{-0.051069in}}%
\pgfpathcurveto{\pgfqpoint{-0.037525in}{-0.064612in}}{\pgfqpoint{-0.019154in}{-0.072222in}}{\pgfqpoint{0.000000in}{-0.072222in}}%
\pgfpathclose%
\pgfusepath{stroke,fill}%
}%
\begin{pgfscope}%
\pgfsys@transformshift{0.656291in}{2.699662in}%
\pgfsys@useobject{currentmarker}{}%
\end{pgfscope}%
\end{pgfscope}%
\begin{pgfscope}%
\pgfpathrectangle{\pgfqpoint{0.100000in}{0.100000in}}{\pgfqpoint{5.307240in}{3.397500in}}%
\pgfusepath{clip}%
\pgfsetrectcap%
\pgfsetroundjoin%
\pgfsetlinewidth{1.505625pt}%
\definecolor{currentstroke}{rgb}{0.678431,1.000000,0.184314}%
\pgfsetstrokecolor{currentstroke}%
\pgfsetstrokeopacity{0.500000}%
\pgfsetdash{}{0pt}%
\pgfpathmoveto{\pgfqpoint{0.689601in}{2.675214in}}%
\pgfusepath{stroke}%
\end{pgfscope}%
\begin{pgfscope}%
\pgfpathrectangle{\pgfqpoint{0.100000in}{0.100000in}}{\pgfqpoint{5.307240in}{3.397500in}}%
\pgfusepath{clip}%
\pgfsetbuttcap%
\pgfsetroundjoin%
\definecolor{currentfill}{rgb}{0.678431,1.000000,0.184314}%
\pgfsetfillcolor{currentfill}%
\pgfsetfillopacity{0.500000}%
\pgfsetlinewidth{0.250937pt}%
\definecolor{currentstroke}{rgb}{0.000000,0.000000,0.000000}%
\pgfsetstrokecolor{currentstroke}%
\pgfsetstrokeopacity{0.500000}%
\pgfsetdash{}{0pt}%
\pgfsys@defobject{currentmarker}{\pgfqpoint{-0.079167in}{-0.079167in}}{\pgfqpoint{0.079167in}{0.079167in}}{%
\pgfpathmoveto{\pgfqpoint{0.000000in}{-0.079167in}}%
\pgfpathcurveto{\pgfqpoint{0.020995in}{-0.079167in}}{\pgfqpoint{0.041133in}{-0.070825in}}{\pgfqpoint{0.055979in}{-0.055979in}}%
\pgfpathcurveto{\pgfqpoint{0.070825in}{-0.041133in}}{\pgfqpoint{0.079167in}{-0.020995in}}{\pgfqpoint{0.079167in}{0.000000in}}%
\pgfpathcurveto{\pgfqpoint{0.079167in}{0.020995in}}{\pgfqpoint{0.070825in}{0.041133in}}{\pgfqpoint{0.055979in}{0.055979in}}%
\pgfpathcurveto{\pgfqpoint{0.041133in}{0.070825in}}{\pgfqpoint{0.020995in}{0.079167in}}{\pgfqpoint{0.000000in}{0.079167in}}%
\pgfpathcurveto{\pgfqpoint{-0.020995in}{0.079167in}}{\pgfqpoint{-0.041133in}{0.070825in}}{\pgfqpoint{-0.055979in}{0.055979in}}%
\pgfpathcurveto{\pgfqpoint{-0.070825in}{0.041133in}}{\pgfqpoint{-0.079167in}{0.020995in}}{\pgfqpoint{-0.079167in}{0.000000in}}%
\pgfpathcurveto{\pgfqpoint{-0.079167in}{-0.020995in}}{\pgfqpoint{-0.070825in}{-0.041133in}}{\pgfqpoint{-0.055979in}{-0.055979in}}%
\pgfpathcurveto{\pgfqpoint{-0.041133in}{-0.070825in}}{\pgfqpoint{-0.020995in}{-0.079167in}}{\pgfqpoint{0.000000in}{-0.079167in}}%
\pgfpathclose%
\pgfusepath{stroke,fill}%
}%
\begin{pgfscope}%
\pgfsys@transformshift{0.689601in}{2.675214in}%
\pgfsys@useobject{currentmarker}{}%
\end{pgfscope}%
\end{pgfscope}%
\begin{pgfscope}%
\pgfpathrectangle{\pgfqpoint{0.100000in}{0.100000in}}{\pgfqpoint{5.307240in}{3.397500in}}%
\pgfusepath{clip}%
\pgfsetrectcap%
\pgfsetroundjoin%
\pgfsetlinewidth{1.505625pt}%
\definecolor{currentstroke}{rgb}{0.678431,1.000000,0.184314}%
\pgfsetstrokecolor{currentstroke}%
\pgfsetstrokeopacity{0.500000}%
\pgfsetdash{}{0pt}%
\pgfpathmoveto{\pgfqpoint{0.817576in}{3.025640in}}%
\pgfusepath{stroke}%
\end{pgfscope}%
\begin{pgfscope}%
\pgfpathrectangle{\pgfqpoint{0.100000in}{0.100000in}}{\pgfqpoint{5.307240in}{3.397500in}}%
\pgfusepath{clip}%
\pgfsetbuttcap%
\pgfsetroundjoin%
\definecolor{currentfill}{rgb}{0.678431,1.000000,0.184314}%
\pgfsetfillcolor{currentfill}%
\pgfsetfillopacity{0.500000}%
\pgfsetlinewidth{0.250937pt}%
\definecolor{currentstroke}{rgb}{0.000000,0.000000,0.000000}%
\pgfsetstrokecolor{currentstroke}%
\pgfsetstrokeopacity{0.500000}%
\pgfsetdash{}{0pt}%
\pgfsys@defobject{currentmarker}{\pgfqpoint{-0.072222in}{-0.072222in}}{\pgfqpoint{0.072222in}{0.072222in}}{%
\pgfpathmoveto{\pgfqpoint{0.000000in}{-0.072222in}}%
\pgfpathcurveto{\pgfqpoint{0.019154in}{-0.072222in}}{\pgfqpoint{0.037525in}{-0.064612in}}{\pgfqpoint{0.051069in}{-0.051069in}}%
\pgfpathcurveto{\pgfqpoint{0.064612in}{-0.037525in}}{\pgfqpoint{0.072222in}{-0.019154in}}{\pgfqpoint{0.072222in}{0.000000in}}%
\pgfpathcurveto{\pgfqpoint{0.072222in}{0.019154in}}{\pgfqpoint{0.064612in}{0.037525in}}{\pgfqpoint{0.051069in}{0.051069in}}%
\pgfpathcurveto{\pgfqpoint{0.037525in}{0.064612in}}{\pgfqpoint{0.019154in}{0.072222in}}{\pgfqpoint{0.000000in}{0.072222in}}%
\pgfpathcurveto{\pgfqpoint{-0.019154in}{0.072222in}}{\pgfqpoint{-0.037525in}{0.064612in}}{\pgfqpoint{-0.051069in}{0.051069in}}%
\pgfpathcurveto{\pgfqpoint{-0.064612in}{0.037525in}}{\pgfqpoint{-0.072222in}{0.019154in}}{\pgfqpoint{-0.072222in}{0.000000in}}%
\pgfpathcurveto{\pgfqpoint{-0.072222in}{-0.019154in}}{\pgfqpoint{-0.064612in}{-0.037525in}}{\pgfqpoint{-0.051069in}{-0.051069in}}%
\pgfpathcurveto{\pgfqpoint{-0.037525in}{-0.064612in}}{\pgfqpoint{-0.019154in}{-0.072222in}}{\pgfqpoint{0.000000in}{-0.072222in}}%
\pgfpathclose%
\pgfusepath{stroke,fill}%
}%
\begin{pgfscope}%
\pgfsys@transformshift{0.817576in}{3.025640in}%
\pgfsys@useobject{currentmarker}{}%
\end{pgfscope}%
\end{pgfscope}%
\begin{pgfscope}%
\pgfpathrectangle{\pgfqpoint{0.100000in}{0.100000in}}{\pgfqpoint{5.307240in}{3.397500in}}%
\pgfusepath{clip}%
\pgfsetrectcap%
\pgfsetroundjoin%
\pgfsetlinewidth{1.505625pt}%
\definecolor{currentstroke}{rgb}{0.678431,1.000000,0.184314}%
\pgfsetstrokecolor{currentstroke}%
\pgfsetstrokeopacity{0.500000}%
\pgfsetdash{}{0pt}%
\pgfpathmoveto{\pgfqpoint{0.768890in}{2.969772in}}%
\pgfusepath{stroke}%
\end{pgfscope}%
\begin{pgfscope}%
\pgfpathrectangle{\pgfqpoint{0.100000in}{0.100000in}}{\pgfqpoint{5.307240in}{3.397500in}}%
\pgfusepath{clip}%
\pgfsetbuttcap%
\pgfsetroundjoin%
\definecolor{currentfill}{rgb}{0.678431,1.000000,0.184314}%
\pgfsetfillcolor{currentfill}%
\pgfsetfillopacity{0.500000}%
\pgfsetlinewidth{0.250937pt}%
\definecolor{currentstroke}{rgb}{0.000000,0.000000,0.000000}%
\pgfsetstrokecolor{currentstroke}%
\pgfsetstrokeopacity{0.500000}%
\pgfsetdash{}{0pt}%
\pgfsys@defobject{currentmarker}{\pgfqpoint{-0.063889in}{-0.063889in}}{\pgfqpoint{0.063889in}{0.063889in}}{%
\pgfpathmoveto{\pgfqpoint{0.000000in}{-0.063889in}}%
\pgfpathcurveto{\pgfqpoint{0.016944in}{-0.063889in}}{\pgfqpoint{0.033195in}{-0.057157in}}{\pgfqpoint{0.045176in}{-0.045176in}}%
\pgfpathcurveto{\pgfqpoint{0.057157in}{-0.033195in}}{\pgfqpoint{0.063889in}{-0.016944in}}{\pgfqpoint{0.063889in}{0.000000in}}%
\pgfpathcurveto{\pgfqpoint{0.063889in}{0.016944in}}{\pgfqpoint{0.057157in}{0.033195in}}{\pgfqpoint{0.045176in}{0.045176in}}%
\pgfpathcurveto{\pgfqpoint{0.033195in}{0.057157in}}{\pgfqpoint{0.016944in}{0.063889in}}{\pgfqpoint{0.000000in}{0.063889in}}%
\pgfpathcurveto{\pgfqpoint{-0.016944in}{0.063889in}}{\pgfqpoint{-0.033195in}{0.057157in}}{\pgfqpoint{-0.045176in}{0.045176in}}%
\pgfpathcurveto{\pgfqpoint{-0.057157in}{0.033195in}}{\pgfqpoint{-0.063889in}{0.016944in}}{\pgfqpoint{-0.063889in}{0.000000in}}%
\pgfpathcurveto{\pgfqpoint{-0.063889in}{-0.016944in}}{\pgfqpoint{-0.057157in}{-0.033195in}}{\pgfqpoint{-0.045176in}{-0.045176in}}%
\pgfpathcurveto{\pgfqpoint{-0.033195in}{-0.057157in}}{\pgfqpoint{-0.016944in}{-0.063889in}}{\pgfqpoint{0.000000in}{-0.063889in}}%
\pgfpathclose%
\pgfusepath{stroke,fill}%
}%
\begin{pgfscope}%
\pgfsys@transformshift{0.768890in}{2.969772in}%
\pgfsys@useobject{currentmarker}{}%
\end{pgfscope}%
\end{pgfscope}%
\begin{pgfscope}%
\pgfpathrectangle{\pgfqpoint{0.100000in}{0.100000in}}{\pgfqpoint{5.307240in}{3.397500in}}%
\pgfusepath{clip}%
\pgfsetrectcap%
\pgfsetroundjoin%
\pgfsetlinewidth{1.505625pt}%
\definecolor{currentstroke}{rgb}{0.678431,1.000000,0.184314}%
\pgfsetstrokecolor{currentstroke}%
\pgfsetstrokeopacity{0.500000}%
\pgfsetdash{}{0pt}%
\pgfpathmoveto{\pgfqpoint{4.762541in}{2.294078in}}%
\pgfusepath{stroke}%
\end{pgfscope}%
\begin{pgfscope}%
\pgfpathrectangle{\pgfqpoint{0.100000in}{0.100000in}}{\pgfqpoint{5.307240in}{3.397500in}}%
\pgfusepath{clip}%
\pgfsetbuttcap%
\pgfsetroundjoin%
\definecolor{currentfill}{rgb}{0.678431,1.000000,0.184314}%
\pgfsetfillcolor{currentfill}%
\pgfsetfillopacity{0.500000}%
\pgfsetlinewidth{0.250937pt}%
\definecolor{currentstroke}{rgb}{0.000000,0.000000,0.000000}%
\pgfsetstrokecolor{currentstroke}%
\pgfsetstrokeopacity{0.500000}%
\pgfsetdash{}{0pt}%
\pgfsys@defobject{currentmarker}{\pgfqpoint{-0.085417in}{-0.085417in}}{\pgfqpoint{0.085417in}{0.085417in}}{%
\pgfpathmoveto{\pgfqpoint{0.000000in}{-0.085417in}}%
\pgfpathcurveto{\pgfqpoint{0.022653in}{-0.085417in}}{\pgfqpoint{0.044381in}{-0.076417in}}{\pgfqpoint{0.060399in}{-0.060399in}}%
\pgfpathcurveto{\pgfqpoint{0.076417in}{-0.044381in}}{\pgfqpoint{0.085417in}{-0.022653in}}{\pgfqpoint{0.085417in}{0.000000in}}%
\pgfpathcurveto{\pgfqpoint{0.085417in}{0.022653in}}{\pgfqpoint{0.076417in}{0.044381in}}{\pgfqpoint{0.060399in}{0.060399in}}%
\pgfpathcurveto{\pgfqpoint{0.044381in}{0.076417in}}{\pgfqpoint{0.022653in}{0.085417in}}{\pgfqpoint{0.000000in}{0.085417in}}%
\pgfpathcurveto{\pgfqpoint{-0.022653in}{0.085417in}}{\pgfqpoint{-0.044381in}{0.076417in}}{\pgfqpoint{-0.060399in}{0.060399in}}%
\pgfpathcurveto{\pgfqpoint{-0.076417in}{0.044381in}}{\pgfqpoint{-0.085417in}{0.022653in}}{\pgfqpoint{-0.085417in}{0.000000in}}%
\pgfpathcurveto{\pgfqpoint{-0.085417in}{-0.022653in}}{\pgfqpoint{-0.076417in}{-0.044381in}}{\pgfqpoint{-0.060399in}{-0.060399in}}%
\pgfpathcurveto{\pgfqpoint{-0.044381in}{-0.076417in}}{\pgfqpoint{-0.022653in}{-0.085417in}}{\pgfqpoint{0.000000in}{-0.085417in}}%
\pgfpathclose%
\pgfusepath{stroke,fill}%
}%
\begin{pgfscope}%
\pgfsys@transformshift{4.762541in}{2.294078in}%
\pgfsys@useobject{currentmarker}{}%
\end{pgfscope}%
\end{pgfscope}%
\begin{pgfscope}%
\pgfpathrectangle{\pgfqpoint{0.100000in}{0.100000in}}{\pgfqpoint{5.307240in}{3.397500in}}%
\pgfusepath{clip}%
\pgfsetrectcap%
\pgfsetroundjoin%
\pgfsetlinewidth{1.505625pt}%
\definecolor{currentstroke}{rgb}{0.678431,1.000000,0.184314}%
\pgfsetstrokecolor{currentstroke}%
\pgfsetstrokeopacity{0.500000}%
\pgfsetdash{}{0pt}%
\pgfpathmoveto{\pgfqpoint{4.511280in}{2.233550in}}%
\pgfusepath{stroke}%
\end{pgfscope}%
\begin{pgfscope}%
\pgfpathrectangle{\pgfqpoint{0.100000in}{0.100000in}}{\pgfqpoint{5.307240in}{3.397500in}}%
\pgfusepath{clip}%
\pgfsetbuttcap%
\pgfsetroundjoin%
\definecolor{currentfill}{rgb}{0.678431,1.000000,0.184314}%
\pgfsetfillcolor{currentfill}%
\pgfsetfillopacity{0.500000}%
\pgfsetlinewidth{0.250937pt}%
\definecolor{currentstroke}{rgb}{0.000000,0.000000,0.000000}%
\pgfsetstrokecolor{currentstroke}%
\pgfsetstrokeopacity{0.500000}%
\pgfsetdash{}{0pt}%
\pgfsys@defobject{currentmarker}{\pgfqpoint{-0.095833in}{-0.095833in}}{\pgfqpoint{0.095833in}{0.095833in}}{%
\pgfpathmoveto{\pgfqpoint{0.000000in}{-0.095833in}}%
\pgfpathcurveto{\pgfqpoint{0.025415in}{-0.095833in}}{\pgfqpoint{0.049793in}{-0.085736in}}{\pgfqpoint{0.067764in}{-0.067764in}}%
\pgfpathcurveto{\pgfqpoint{0.085736in}{-0.049793in}}{\pgfqpoint{0.095833in}{-0.025415in}}{\pgfqpoint{0.095833in}{0.000000in}}%
\pgfpathcurveto{\pgfqpoint{0.095833in}{0.025415in}}{\pgfqpoint{0.085736in}{0.049793in}}{\pgfqpoint{0.067764in}{0.067764in}}%
\pgfpathcurveto{\pgfqpoint{0.049793in}{0.085736in}}{\pgfqpoint{0.025415in}{0.095833in}}{\pgfqpoint{0.000000in}{0.095833in}}%
\pgfpathcurveto{\pgfqpoint{-0.025415in}{0.095833in}}{\pgfqpoint{-0.049793in}{0.085736in}}{\pgfqpoint{-0.067764in}{0.067764in}}%
\pgfpathcurveto{\pgfqpoint{-0.085736in}{0.049793in}}{\pgfqpoint{-0.095833in}{0.025415in}}{\pgfqpoint{-0.095833in}{0.000000in}}%
\pgfpathcurveto{\pgfqpoint{-0.095833in}{-0.025415in}}{\pgfqpoint{-0.085736in}{-0.049793in}}{\pgfqpoint{-0.067764in}{-0.067764in}}%
\pgfpathcurveto{\pgfqpoint{-0.049793in}{-0.085736in}}{\pgfqpoint{-0.025415in}{-0.095833in}}{\pgfqpoint{0.000000in}{-0.095833in}}%
\pgfpathclose%
\pgfusepath{stroke,fill}%
}%
\begin{pgfscope}%
\pgfsys@transformshift{4.511280in}{2.233550in}%
\pgfsys@useobject{currentmarker}{}%
\end{pgfscope}%
\end{pgfscope}%
\begin{pgfscope}%
\pgfpathrectangle{\pgfqpoint{0.100000in}{0.100000in}}{\pgfqpoint{5.307240in}{3.397500in}}%
\pgfusepath{clip}%
\pgfsetrectcap%
\pgfsetroundjoin%
\pgfsetlinewidth{1.505625pt}%
\definecolor{currentstroke}{rgb}{0.678431,1.000000,0.184314}%
\pgfsetstrokecolor{currentstroke}%
\pgfsetstrokeopacity{0.500000}%
\pgfsetdash{}{0pt}%
\pgfpathmoveto{\pgfqpoint{4.668213in}{2.321811in}}%
\pgfusepath{stroke}%
\end{pgfscope}%
\begin{pgfscope}%
\pgfpathrectangle{\pgfqpoint{0.100000in}{0.100000in}}{\pgfqpoint{5.307240in}{3.397500in}}%
\pgfusepath{clip}%
\pgfsetbuttcap%
\pgfsetroundjoin%
\definecolor{currentfill}{rgb}{0.678431,1.000000,0.184314}%
\pgfsetfillcolor{currentfill}%
\pgfsetfillopacity{0.500000}%
\pgfsetlinewidth{0.250937pt}%
\definecolor{currentstroke}{rgb}{0.000000,0.000000,0.000000}%
\pgfsetstrokecolor{currentstroke}%
\pgfsetstrokeopacity{0.500000}%
\pgfsetdash{}{0pt}%
\pgfsys@defobject{currentmarker}{\pgfqpoint{-0.073611in}{-0.073611in}}{\pgfqpoint{0.073611in}{0.073611in}}{%
\pgfpathmoveto{\pgfqpoint{0.000000in}{-0.073611in}}%
\pgfpathcurveto{\pgfqpoint{0.019522in}{-0.073611in}}{\pgfqpoint{0.038247in}{-0.065855in}}{\pgfqpoint{0.052051in}{-0.052051in}}%
\pgfpathcurveto{\pgfqpoint{0.065855in}{-0.038247in}}{\pgfqpoint{0.073611in}{-0.019522in}}{\pgfqpoint{0.073611in}{0.000000in}}%
\pgfpathcurveto{\pgfqpoint{0.073611in}{0.019522in}}{\pgfqpoint{0.065855in}{0.038247in}}{\pgfqpoint{0.052051in}{0.052051in}}%
\pgfpathcurveto{\pgfqpoint{0.038247in}{0.065855in}}{\pgfqpoint{0.019522in}{0.073611in}}{\pgfqpoint{0.000000in}{0.073611in}}%
\pgfpathcurveto{\pgfqpoint{-0.019522in}{0.073611in}}{\pgfqpoint{-0.038247in}{0.065855in}}{\pgfqpoint{-0.052051in}{0.052051in}}%
\pgfpathcurveto{\pgfqpoint{-0.065855in}{0.038247in}}{\pgfqpoint{-0.073611in}{0.019522in}}{\pgfqpoint{-0.073611in}{0.000000in}}%
\pgfpathcurveto{\pgfqpoint{-0.073611in}{-0.019522in}}{\pgfqpoint{-0.065855in}{-0.038247in}}{\pgfqpoint{-0.052051in}{-0.052051in}}%
\pgfpathcurveto{\pgfqpoint{-0.038247in}{-0.065855in}}{\pgfqpoint{-0.019522in}{-0.073611in}}{\pgfqpoint{0.000000in}{-0.073611in}}%
\pgfpathclose%
\pgfusepath{stroke,fill}%
}%
\begin{pgfscope}%
\pgfsys@transformshift{4.668213in}{2.321811in}%
\pgfsys@useobject{currentmarker}{}%
\end{pgfscope}%
\end{pgfscope}%
\begin{pgfscope}%
\pgfpathrectangle{\pgfqpoint{0.100000in}{0.100000in}}{\pgfqpoint{5.307240in}{3.397500in}}%
\pgfusepath{clip}%
\pgfsetrectcap%
\pgfsetroundjoin%
\pgfsetlinewidth{1.505625pt}%
\definecolor{currentstroke}{rgb}{0.678431,1.000000,0.184314}%
\pgfsetstrokecolor{currentstroke}%
\pgfsetstrokeopacity{0.500000}%
\pgfsetdash{}{0pt}%
\pgfpathmoveto{\pgfqpoint{4.587861in}{2.179616in}}%
\pgfusepath{stroke}%
\end{pgfscope}%
\begin{pgfscope}%
\pgfpathrectangle{\pgfqpoint{0.100000in}{0.100000in}}{\pgfqpoint{5.307240in}{3.397500in}}%
\pgfusepath{clip}%
\pgfsetbuttcap%
\pgfsetroundjoin%
\definecolor{currentfill}{rgb}{0.678431,1.000000,0.184314}%
\pgfsetfillcolor{currentfill}%
\pgfsetfillopacity{0.500000}%
\pgfsetlinewidth{0.250937pt}%
\definecolor{currentstroke}{rgb}{0.000000,0.000000,0.000000}%
\pgfsetstrokecolor{currentstroke}%
\pgfsetstrokeopacity{0.500000}%
\pgfsetdash{}{0pt}%
\pgfsys@defobject{currentmarker}{\pgfqpoint{-0.068750in}{-0.068750in}}{\pgfqpoint{0.068750in}{0.068750in}}{%
\pgfpathmoveto{\pgfqpoint{0.000000in}{-0.068750in}}%
\pgfpathcurveto{\pgfqpoint{0.018233in}{-0.068750in}}{\pgfqpoint{0.035721in}{-0.061506in}}{\pgfqpoint{0.048614in}{-0.048614in}}%
\pgfpathcurveto{\pgfqpoint{0.061506in}{-0.035721in}}{\pgfqpoint{0.068750in}{-0.018233in}}{\pgfqpoint{0.068750in}{0.000000in}}%
\pgfpathcurveto{\pgfqpoint{0.068750in}{0.018233in}}{\pgfqpoint{0.061506in}{0.035721in}}{\pgfqpoint{0.048614in}{0.048614in}}%
\pgfpathcurveto{\pgfqpoint{0.035721in}{0.061506in}}{\pgfqpoint{0.018233in}{0.068750in}}{\pgfqpoint{0.000000in}{0.068750in}}%
\pgfpathcurveto{\pgfqpoint{-0.018233in}{0.068750in}}{\pgfqpoint{-0.035721in}{0.061506in}}{\pgfqpoint{-0.048614in}{0.048614in}}%
\pgfpathcurveto{\pgfqpoint{-0.061506in}{0.035721in}}{\pgfqpoint{-0.068750in}{0.018233in}}{\pgfqpoint{-0.068750in}{0.000000in}}%
\pgfpathcurveto{\pgfqpoint{-0.068750in}{-0.018233in}}{\pgfqpoint{-0.061506in}{-0.035721in}}{\pgfqpoint{-0.048614in}{-0.048614in}}%
\pgfpathcurveto{\pgfqpoint{-0.035721in}{-0.061506in}}{\pgfqpoint{-0.018233in}{-0.068750in}}{\pgfqpoint{0.000000in}{-0.068750in}}%
\pgfpathclose%
\pgfusepath{stroke,fill}%
}%
\begin{pgfscope}%
\pgfsys@transformshift{4.587861in}{2.179616in}%
\pgfsys@useobject{currentmarker}{}%
\end{pgfscope}%
\end{pgfscope}%
\begin{pgfscope}%
\pgfpathrectangle{\pgfqpoint{0.100000in}{0.100000in}}{\pgfqpoint{5.307240in}{3.397500in}}%
\pgfusepath{clip}%
\pgfsetrectcap%
\pgfsetroundjoin%
\pgfsetlinewidth{1.505625pt}%
\definecolor{currentstroke}{rgb}{0.678431,1.000000,0.184314}%
\pgfsetstrokecolor{currentstroke}%
\pgfsetstrokeopacity{0.500000}%
\pgfsetdash{}{0pt}%
\pgfpathmoveto{\pgfqpoint{4.777549in}{2.344694in}}%
\pgfusepath{stroke}%
\end{pgfscope}%
\begin{pgfscope}%
\pgfpathrectangle{\pgfqpoint{0.100000in}{0.100000in}}{\pgfqpoint{5.307240in}{3.397500in}}%
\pgfusepath{clip}%
\pgfsetbuttcap%
\pgfsetroundjoin%
\definecolor{currentfill}{rgb}{0.678431,1.000000,0.184314}%
\pgfsetfillcolor{currentfill}%
\pgfsetfillopacity{0.500000}%
\pgfsetlinewidth{0.250937pt}%
\definecolor{currentstroke}{rgb}{0.000000,0.000000,0.000000}%
\pgfsetstrokecolor{currentstroke}%
\pgfsetstrokeopacity{0.500000}%
\pgfsetdash{}{0pt}%
\pgfsys@defobject{currentmarker}{\pgfqpoint{-0.113194in}{-0.113194in}}{\pgfqpoint{0.113194in}{0.113194in}}{%
\pgfpathmoveto{\pgfqpoint{0.000000in}{-0.113194in}}%
\pgfpathcurveto{\pgfqpoint{0.030020in}{-0.113194in}}{\pgfqpoint{0.058814in}{-0.101268in}}{\pgfqpoint{0.080041in}{-0.080041in}}%
\pgfpathcurveto{\pgfqpoint{0.101268in}{-0.058814in}}{\pgfqpoint{0.113194in}{-0.030020in}}{\pgfqpoint{0.113194in}{0.000000in}}%
\pgfpathcurveto{\pgfqpoint{0.113194in}{0.030020in}}{\pgfqpoint{0.101268in}{0.058814in}}{\pgfqpoint{0.080041in}{0.080041in}}%
\pgfpathcurveto{\pgfqpoint{0.058814in}{0.101268in}}{\pgfqpoint{0.030020in}{0.113194in}}{\pgfqpoint{0.000000in}{0.113194in}}%
\pgfpathcurveto{\pgfqpoint{-0.030020in}{0.113194in}}{\pgfqpoint{-0.058814in}{0.101268in}}{\pgfqpoint{-0.080041in}{0.080041in}}%
\pgfpathcurveto{\pgfqpoint{-0.101268in}{0.058814in}}{\pgfqpoint{-0.113194in}{0.030020in}}{\pgfqpoint{-0.113194in}{0.000000in}}%
\pgfpathcurveto{\pgfqpoint{-0.113194in}{-0.030020in}}{\pgfqpoint{-0.101268in}{-0.058814in}}{\pgfqpoint{-0.080041in}{-0.080041in}}%
\pgfpathcurveto{\pgfqpoint{-0.058814in}{-0.101268in}}{\pgfqpoint{-0.030020in}{-0.113194in}}{\pgfqpoint{0.000000in}{-0.113194in}}%
\pgfpathclose%
\pgfusepath{stroke,fill}%
}%
\begin{pgfscope}%
\pgfsys@transformshift{4.777549in}{2.344694in}%
\pgfsys@useobject{currentmarker}{}%
\end{pgfscope}%
\end{pgfscope}%
\begin{pgfscope}%
\pgfpathrectangle{\pgfqpoint{0.100000in}{0.100000in}}{\pgfqpoint{5.307240in}{3.397500in}}%
\pgfusepath{clip}%
\pgfsetrectcap%
\pgfsetroundjoin%
\pgfsetlinewidth{1.505625pt}%
\definecolor{currentstroke}{rgb}{0.678431,1.000000,0.184314}%
\pgfsetstrokecolor{currentstroke}%
\pgfsetstrokeopacity{0.500000}%
\pgfsetdash{}{0pt}%
\pgfpathmoveto{\pgfqpoint{4.333471in}{2.392603in}}%
\pgfusepath{stroke}%
\end{pgfscope}%
\begin{pgfscope}%
\pgfpathrectangle{\pgfqpoint{0.100000in}{0.100000in}}{\pgfqpoint{5.307240in}{3.397500in}}%
\pgfusepath{clip}%
\pgfsetbuttcap%
\pgfsetroundjoin%
\definecolor{currentfill}{rgb}{0.678431,1.000000,0.184314}%
\pgfsetfillcolor{currentfill}%
\pgfsetfillopacity{0.500000}%
\pgfsetlinewidth{0.250937pt}%
\definecolor{currentstroke}{rgb}{0.000000,0.000000,0.000000}%
\pgfsetstrokecolor{currentstroke}%
\pgfsetstrokeopacity{0.500000}%
\pgfsetdash{}{0pt}%
\pgfsys@defobject{currentmarker}{\pgfqpoint{-0.090278in}{-0.090278in}}{\pgfqpoint{0.090278in}{0.090278in}}{%
\pgfpathmoveto{\pgfqpoint{0.000000in}{-0.090278in}}%
\pgfpathcurveto{\pgfqpoint{0.023942in}{-0.090278in}}{\pgfqpoint{0.046907in}{-0.080766in}}{\pgfqpoint{0.063836in}{-0.063836in}}%
\pgfpathcurveto{\pgfqpoint{0.080766in}{-0.046907in}}{\pgfqpoint{0.090278in}{-0.023942in}}{\pgfqpoint{0.090278in}{0.000000in}}%
\pgfpathcurveto{\pgfqpoint{0.090278in}{0.023942in}}{\pgfqpoint{0.080766in}{0.046907in}}{\pgfqpoint{0.063836in}{0.063836in}}%
\pgfpathcurveto{\pgfqpoint{0.046907in}{0.080766in}}{\pgfqpoint{0.023942in}{0.090278in}}{\pgfqpoint{0.000000in}{0.090278in}}%
\pgfpathcurveto{\pgfqpoint{-0.023942in}{0.090278in}}{\pgfqpoint{-0.046907in}{0.080766in}}{\pgfqpoint{-0.063836in}{0.063836in}}%
\pgfpathcurveto{\pgfqpoint{-0.080766in}{0.046907in}}{\pgfqpoint{-0.090278in}{0.023942in}}{\pgfqpoint{-0.090278in}{0.000000in}}%
\pgfpathcurveto{\pgfqpoint{-0.090278in}{-0.023942in}}{\pgfqpoint{-0.080766in}{-0.046907in}}{\pgfqpoint{-0.063836in}{-0.063836in}}%
\pgfpathcurveto{\pgfqpoint{-0.046907in}{-0.080766in}}{\pgfqpoint{-0.023942in}{-0.090278in}}{\pgfqpoint{0.000000in}{-0.090278in}}%
\pgfpathclose%
\pgfusepath{stroke,fill}%
}%
\begin{pgfscope}%
\pgfsys@transformshift{4.333471in}{2.392603in}%
\pgfsys@useobject{currentmarker}{}%
\end{pgfscope}%
\end{pgfscope}%
\begin{pgfscope}%
\pgfpathrectangle{\pgfqpoint{0.100000in}{0.100000in}}{\pgfqpoint{5.307240in}{3.397500in}}%
\pgfusepath{clip}%
\pgfsetrectcap%
\pgfsetroundjoin%
\pgfsetlinewidth{1.505625pt}%
\definecolor{currentstroke}{rgb}{0.678431,1.000000,0.184314}%
\pgfsetstrokecolor{currentstroke}%
\pgfsetstrokeopacity{0.500000}%
\pgfsetdash{}{0pt}%
\pgfpathmoveto{\pgfqpoint{4.627847in}{2.174448in}}%
\pgfusepath{stroke}%
\end{pgfscope}%
\begin{pgfscope}%
\pgfpathrectangle{\pgfqpoint{0.100000in}{0.100000in}}{\pgfqpoint{5.307240in}{3.397500in}}%
\pgfusepath{clip}%
\pgfsetbuttcap%
\pgfsetroundjoin%
\definecolor{currentfill}{rgb}{0.678431,1.000000,0.184314}%
\pgfsetfillcolor{currentfill}%
\pgfsetfillopacity{0.500000}%
\pgfsetlinewidth{0.250937pt}%
\definecolor{currentstroke}{rgb}{0.000000,0.000000,0.000000}%
\pgfsetstrokecolor{currentstroke}%
\pgfsetstrokeopacity{0.500000}%
\pgfsetdash{}{0pt}%
\pgfsys@defobject{currentmarker}{\pgfqpoint{-0.084722in}{-0.084722in}}{\pgfqpoint{0.084722in}{0.084722in}}{%
\pgfpathmoveto{\pgfqpoint{0.000000in}{-0.084722in}}%
\pgfpathcurveto{\pgfqpoint{0.022469in}{-0.084722in}}{\pgfqpoint{0.044020in}{-0.075795in}}{\pgfqpoint{0.059908in}{-0.059908in}}%
\pgfpathcurveto{\pgfqpoint{0.075795in}{-0.044020in}}{\pgfqpoint{0.084722in}{-0.022469in}}{\pgfqpoint{0.084722in}{0.000000in}}%
\pgfpathcurveto{\pgfqpoint{0.084722in}{0.022469in}}{\pgfqpoint{0.075795in}{0.044020in}}{\pgfqpoint{0.059908in}{0.059908in}}%
\pgfpathcurveto{\pgfqpoint{0.044020in}{0.075795in}}{\pgfqpoint{0.022469in}{0.084722in}}{\pgfqpoint{0.000000in}{0.084722in}}%
\pgfpathcurveto{\pgfqpoint{-0.022469in}{0.084722in}}{\pgfqpoint{-0.044020in}{0.075795in}}{\pgfqpoint{-0.059908in}{0.059908in}}%
\pgfpathcurveto{\pgfqpoint{-0.075795in}{0.044020in}}{\pgfqpoint{-0.084722in}{0.022469in}}{\pgfqpoint{-0.084722in}{0.000000in}}%
\pgfpathcurveto{\pgfqpoint{-0.084722in}{-0.022469in}}{\pgfqpoint{-0.075795in}{-0.044020in}}{\pgfqpoint{-0.059908in}{-0.059908in}}%
\pgfpathcurveto{\pgfqpoint{-0.044020in}{-0.075795in}}{\pgfqpoint{-0.022469in}{-0.084722in}}{\pgfqpoint{0.000000in}{-0.084722in}}%
\pgfpathclose%
\pgfusepath{stroke,fill}%
}%
\begin{pgfscope}%
\pgfsys@transformshift{4.627847in}{2.174448in}%
\pgfsys@useobject{currentmarker}{}%
\end{pgfscope}%
\end{pgfscope}%
\begin{pgfscope}%
\pgfpathrectangle{\pgfqpoint{0.100000in}{0.100000in}}{\pgfqpoint{5.307240in}{3.397500in}}%
\pgfusepath{clip}%
\pgfsetrectcap%
\pgfsetroundjoin%
\pgfsetlinewidth{1.505625pt}%
\definecolor{currentstroke}{rgb}{0.678431,1.000000,0.184314}%
\pgfsetstrokecolor{currentstroke}%
\pgfsetstrokeopacity{0.500000}%
\pgfsetdash{}{0pt}%
\pgfpathmoveto{\pgfqpoint{4.648026in}{2.230106in}}%
\pgfusepath{stroke}%
\end{pgfscope}%
\begin{pgfscope}%
\pgfpathrectangle{\pgfqpoint{0.100000in}{0.100000in}}{\pgfqpoint{5.307240in}{3.397500in}}%
\pgfusepath{clip}%
\pgfsetbuttcap%
\pgfsetroundjoin%
\definecolor{currentfill}{rgb}{0.678431,1.000000,0.184314}%
\pgfsetfillcolor{currentfill}%
\pgfsetfillopacity{0.500000}%
\pgfsetlinewidth{0.250937pt}%
\definecolor{currentstroke}{rgb}{0.000000,0.000000,0.000000}%
\pgfsetstrokecolor{currentstroke}%
\pgfsetstrokeopacity{0.500000}%
\pgfsetdash{}{0pt}%
\pgfsys@defobject{currentmarker}{\pgfqpoint{-0.072222in}{-0.072222in}}{\pgfqpoint{0.072222in}{0.072222in}}{%
\pgfpathmoveto{\pgfqpoint{0.000000in}{-0.072222in}}%
\pgfpathcurveto{\pgfqpoint{0.019154in}{-0.072222in}}{\pgfqpoint{0.037525in}{-0.064612in}}{\pgfqpoint{0.051069in}{-0.051069in}}%
\pgfpathcurveto{\pgfqpoint{0.064612in}{-0.037525in}}{\pgfqpoint{0.072222in}{-0.019154in}}{\pgfqpoint{0.072222in}{0.000000in}}%
\pgfpathcurveto{\pgfqpoint{0.072222in}{0.019154in}}{\pgfqpoint{0.064612in}{0.037525in}}{\pgfqpoint{0.051069in}{0.051069in}}%
\pgfpathcurveto{\pgfqpoint{0.037525in}{0.064612in}}{\pgfqpoint{0.019154in}{0.072222in}}{\pgfqpoint{0.000000in}{0.072222in}}%
\pgfpathcurveto{\pgfqpoint{-0.019154in}{0.072222in}}{\pgfqpoint{-0.037525in}{0.064612in}}{\pgfqpoint{-0.051069in}{0.051069in}}%
\pgfpathcurveto{\pgfqpoint{-0.064612in}{0.037525in}}{\pgfqpoint{-0.072222in}{0.019154in}}{\pgfqpoint{-0.072222in}{0.000000in}}%
\pgfpathcurveto{\pgfqpoint{-0.072222in}{-0.019154in}}{\pgfqpoint{-0.064612in}{-0.037525in}}{\pgfqpoint{-0.051069in}{-0.051069in}}%
\pgfpathcurveto{\pgfqpoint{-0.037525in}{-0.064612in}}{\pgfqpoint{-0.019154in}{-0.072222in}}{\pgfqpoint{0.000000in}{-0.072222in}}%
\pgfpathclose%
\pgfusepath{stroke,fill}%
}%
\begin{pgfscope}%
\pgfsys@transformshift{4.648026in}{2.230106in}%
\pgfsys@useobject{currentmarker}{}%
\end{pgfscope}%
\end{pgfscope}%
\begin{pgfscope}%
\pgfpathrectangle{\pgfqpoint{0.100000in}{0.100000in}}{\pgfqpoint{5.307240in}{3.397500in}}%
\pgfusepath{clip}%
\pgfsetrectcap%
\pgfsetroundjoin%
\pgfsetlinewidth{1.505625pt}%
\definecolor{currentstroke}{rgb}{0.678431,1.000000,0.184314}%
\pgfsetstrokecolor{currentstroke}%
\pgfsetstrokeopacity{0.500000}%
\pgfsetdash{}{0pt}%
\pgfpathmoveto{\pgfqpoint{4.469352in}{2.203243in}}%
\pgfusepath{stroke}%
\end{pgfscope}%
\begin{pgfscope}%
\pgfpathrectangle{\pgfqpoint{0.100000in}{0.100000in}}{\pgfqpoint{5.307240in}{3.397500in}}%
\pgfusepath{clip}%
\pgfsetbuttcap%
\pgfsetroundjoin%
\definecolor{currentfill}{rgb}{0.678431,1.000000,0.184314}%
\pgfsetfillcolor{currentfill}%
\pgfsetfillopacity{0.500000}%
\pgfsetlinewidth{0.250937pt}%
\definecolor{currentstroke}{rgb}{0.000000,0.000000,0.000000}%
\pgfsetstrokecolor{currentstroke}%
\pgfsetstrokeopacity{0.500000}%
\pgfsetdash{}{0pt}%
\pgfsys@defobject{currentmarker}{\pgfqpoint{-0.092361in}{-0.092361in}}{\pgfqpoint{0.092361in}{0.092361in}}{%
\pgfpathmoveto{\pgfqpoint{0.000000in}{-0.092361in}}%
\pgfpathcurveto{\pgfqpoint{0.024494in}{-0.092361in}}{\pgfqpoint{0.047989in}{-0.082629in}}{\pgfqpoint{0.065309in}{-0.065309in}}%
\pgfpathcurveto{\pgfqpoint{0.082629in}{-0.047989in}}{\pgfqpoint{0.092361in}{-0.024494in}}{\pgfqpoint{0.092361in}{0.000000in}}%
\pgfpathcurveto{\pgfqpoint{0.092361in}{0.024494in}}{\pgfqpoint{0.082629in}{0.047989in}}{\pgfqpoint{0.065309in}{0.065309in}}%
\pgfpathcurveto{\pgfqpoint{0.047989in}{0.082629in}}{\pgfqpoint{0.024494in}{0.092361in}}{\pgfqpoint{0.000000in}{0.092361in}}%
\pgfpathcurveto{\pgfqpoint{-0.024494in}{0.092361in}}{\pgfqpoint{-0.047989in}{0.082629in}}{\pgfqpoint{-0.065309in}{0.065309in}}%
\pgfpathcurveto{\pgfqpoint{-0.082629in}{0.047989in}}{\pgfqpoint{-0.092361in}{0.024494in}}{\pgfqpoint{-0.092361in}{0.000000in}}%
\pgfpathcurveto{\pgfqpoint{-0.092361in}{-0.024494in}}{\pgfqpoint{-0.082629in}{-0.047989in}}{\pgfqpoint{-0.065309in}{-0.065309in}}%
\pgfpathcurveto{\pgfqpoint{-0.047989in}{-0.082629in}}{\pgfqpoint{-0.024494in}{-0.092361in}}{\pgfqpoint{0.000000in}{-0.092361in}}%
\pgfpathclose%
\pgfusepath{stroke,fill}%
}%
\begin{pgfscope}%
\pgfsys@transformshift{4.469352in}{2.203243in}%
\pgfsys@useobject{currentmarker}{}%
\end{pgfscope}%
\end{pgfscope}%
\begin{pgfscope}%
\pgfpathrectangle{\pgfqpoint{0.100000in}{0.100000in}}{\pgfqpoint{5.307240in}{3.397500in}}%
\pgfusepath{clip}%
\pgfsetrectcap%
\pgfsetroundjoin%
\pgfsetlinewidth{1.505625pt}%
\definecolor{currentstroke}{rgb}{0.678431,1.000000,0.184314}%
\pgfsetstrokecolor{currentstroke}%
\pgfsetstrokeopacity{0.500000}%
\pgfsetdash{}{0pt}%
\pgfpathmoveto{\pgfqpoint{4.703891in}{2.214493in}}%
\pgfusepath{stroke}%
\end{pgfscope}%
\begin{pgfscope}%
\pgfpathrectangle{\pgfqpoint{0.100000in}{0.100000in}}{\pgfqpoint{5.307240in}{3.397500in}}%
\pgfusepath{clip}%
\pgfsetbuttcap%
\pgfsetroundjoin%
\definecolor{currentfill}{rgb}{0.678431,1.000000,0.184314}%
\pgfsetfillcolor{currentfill}%
\pgfsetfillopacity{0.500000}%
\pgfsetlinewidth{0.250937pt}%
\definecolor{currentstroke}{rgb}{0.000000,0.000000,0.000000}%
\pgfsetstrokecolor{currentstroke}%
\pgfsetstrokeopacity{0.500000}%
\pgfsetdash{}{0pt}%
\pgfsys@defobject{currentmarker}{\pgfqpoint{-0.083333in}{-0.083333in}}{\pgfqpoint{0.083333in}{0.083333in}}{%
\pgfpathmoveto{\pgfqpoint{0.000000in}{-0.083333in}}%
\pgfpathcurveto{\pgfqpoint{0.022100in}{-0.083333in}}{\pgfqpoint{0.043298in}{-0.074553in}}{\pgfqpoint{0.058926in}{-0.058926in}}%
\pgfpathcurveto{\pgfqpoint{0.074553in}{-0.043298in}}{\pgfqpoint{0.083333in}{-0.022100in}}{\pgfqpoint{0.083333in}{0.000000in}}%
\pgfpathcurveto{\pgfqpoint{0.083333in}{0.022100in}}{\pgfqpoint{0.074553in}{0.043298in}}{\pgfqpoint{0.058926in}{0.058926in}}%
\pgfpathcurveto{\pgfqpoint{0.043298in}{0.074553in}}{\pgfqpoint{0.022100in}{0.083333in}}{\pgfqpoint{0.000000in}{0.083333in}}%
\pgfpathcurveto{\pgfqpoint{-0.022100in}{0.083333in}}{\pgfqpoint{-0.043298in}{0.074553in}}{\pgfqpoint{-0.058926in}{0.058926in}}%
\pgfpathcurveto{\pgfqpoint{-0.074553in}{0.043298in}}{\pgfqpoint{-0.083333in}{0.022100in}}{\pgfqpoint{-0.083333in}{0.000000in}}%
\pgfpathcurveto{\pgfqpoint{-0.083333in}{-0.022100in}}{\pgfqpoint{-0.074553in}{-0.043298in}}{\pgfqpoint{-0.058926in}{-0.058926in}}%
\pgfpathcurveto{\pgfqpoint{-0.043298in}{-0.074553in}}{\pgfqpoint{-0.022100in}{-0.083333in}}{\pgfqpoint{0.000000in}{-0.083333in}}%
\pgfpathclose%
\pgfusepath{stroke,fill}%
}%
\begin{pgfscope}%
\pgfsys@transformshift{4.703891in}{2.214493in}%
\pgfsys@useobject{currentmarker}{}%
\end{pgfscope}%
\end{pgfscope}%
\begin{pgfscope}%
\pgfpathrectangle{\pgfqpoint{0.100000in}{0.100000in}}{\pgfqpoint{5.307240in}{3.397500in}}%
\pgfusepath{clip}%
\pgfsetrectcap%
\pgfsetroundjoin%
\pgfsetlinewidth{1.505625pt}%
\definecolor{currentstroke}{rgb}{0.678431,1.000000,0.184314}%
\pgfsetstrokecolor{currentstroke}%
\pgfsetstrokeopacity{0.500000}%
\pgfsetdash{}{0pt}%
\pgfpathmoveto{\pgfqpoint{4.685919in}{2.247383in}}%
\pgfusepath{stroke}%
\end{pgfscope}%
\begin{pgfscope}%
\pgfpathrectangle{\pgfqpoint{0.100000in}{0.100000in}}{\pgfqpoint{5.307240in}{3.397500in}}%
\pgfusepath{clip}%
\pgfsetbuttcap%
\pgfsetroundjoin%
\definecolor{currentfill}{rgb}{0.678431,1.000000,0.184314}%
\pgfsetfillcolor{currentfill}%
\pgfsetfillopacity{0.500000}%
\pgfsetlinewidth{0.250937pt}%
\definecolor{currentstroke}{rgb}{0.000000,0.000000,0.000000}%
\pgfsetstrokecolor{currentstroke}%
\pgfsetstrokeopacity{0.500000}%
\pgfsetdash{}{0pt}%
\pgfsys@defobject{currentmarker}{\pgfqpoint{-0.074306in}{-0.074306in}}{\pgfqpoint{0.074306in}{0.074306in}}{%
\pgfpathmoveto{\pgfqpoint{0.000000in}{-0.074306in}}%
\pgfpathcurveto{\pgfqpoint{0.019706in}{-0.074306in}}{\pgfqpoint{0.038608in}{-0.066476in}}{\pgfqpoint{0.052542in}{-0.052542in}}%
\pgfpathcurveto{\pgfqpoint{0.066476in}{-0.038608in}}{\pgfqpoint{0.074306in}{-0.019706in}}{\pgfqpoint{0.074306in}{0.000000in}}%
\pgfpathcurveto{\pgfqpoint{0.074306in}{0.019706in}}{\pgfqpoint{0.066476in}{0.038608in}}{\pgfqpoint{0.052542in}{0.052542in}}%
\pgfpathcurveto{\pgfqpoint{0.038608in}{0.066476in}}{\pgfqpoint{0.019706in}{0.074306in}}{\pgfqpoint{0.000000in}{0.074306in}}%
\pgfpathcurveto{\pgfqpoint{-0.019706in}{0.074306in}}{\pgfqpoint{-0.038608in}{0.066476in}}{\pgfqpoint{-0.052542in}{0.052542in}}%
\pgfpathcurveto{\pgfqpoint{-0.066476in}{0.038608in}}{\pgfqpoint{-0.074306in}{0.019706in}}{\pgfqpoint{-0.074306in}{0.000000in}}%
\pgfpathcurveto{\pgfqpoint{-0.074306in}{-0.019706in}}{\pgfqpoint{-0.066476in}{-0.038608in}}{\pgfqpoint{-0.052542in}{-0.052542in}}%
\pgfpathcurveto{\pgfqpoint{-0.038608in}{-0.066476in}}{\pgfqpoint{-0.019706in}{-0.074306in}}{\pgfqpoint{0.000000in}{-0.074306in}}%
\pgfpathclose%
\pgfusepath{stroke,fill}%
}%
\begin{pgfscope}%
\pgfsys@transformshift{4.685919in}{2.247383in}%
\pgfsys@useobject{currentmarker}{}%
\end{pgfscope}%
\end{pgfscope}%
\begin{pgfscope}%
\pgfpathrectangle{\pgfqpoint{0.100000in}{0.100000in}}{\pgfqpoint{5.307240in}{3.397500in}}%
\pgfusepath{clip}%
\pgfsetrectcap%
\pgfsetroundjoin%
\pgfsetlinewidth{1.505625pt}%
\definecolor{currentstroke}{rgb}{0.678431,1.000000,0.184314}%
\pgfsetstrokecolor{currentstroke}%
\pgfsetstrokeopacity{0.500000}%
\pgfsetdash{}{0pt}%
\pgfpathmoveto{\pgfqpoint{4.805435in}{2.226152in}}%
\pgfusepath{stroke}%
\end{pgfscope}%
\begin{pgfscope}%
\pgfpathrectangle{\pgfqpoint{0.100000in}{0.100000in}}{\pgfqpoint{5.307240in}{3.397500in}}%
\pgfusepath{clip}%
\pgfsetbuttcap%
\pgfsetroundjoin%
\definecolor{currentfill}{rgb}{0.678431,1.000000,0.184314}%
\pgfsetfillcolor{currentfill}%
\pgfsetfillopacity{0.500000}%
\pgfsetlinewidth{0.250937pt}%
\definecolor{currentstroke}{rgb}{0.000000,0.000000,0.000000}%
\pgfsetstrokecolor{currentstroke}%
\pgfsetstrokeopacity{0.500000}%
\pgfsetdash{}{0pt}%
\pgfsys@defobject{currentmarker}{\pgfqpoint{-0.077083in}{-0.077083in}}{\pgfqpoint{0.077083in}{0.077083in}}{%
\pgfpathmoveto{\pgfqpoint{0.000000in}{-0.077083in}}%
\pgfpathcurveto{\pgfqpoint{0.020443in}{-0.077083in}}{\pgfqpoint{0.040051in}{-0.068961in}}{\pgfqpoint{0.054506in}{-0.054506in}}%
\pgfpathcurveto{\pgfqpoint{0.068961in}{-0.040051in}}{\pgfqpoint{0.077083in}{-0.020443in}}{\pgfqpoint{0.077083in}{0.000000in}}%
\pgfpathcurveto{\pgfqpoint{0.077083in}{0.020443in}}{\pgfqpoint{0.068961in}{0.040051in}}{\pgfqpoint{0.054506in}{0.054506in}}%
\pgfpathcurveto{\pgfqpoint{0.040051in}{0.068961in}}{\pgfqpoint{0.020443in}{0.077083in}}{\pgfqpoint{0.000000in}{0.077083in}}%
\pgfpathcurveto{\pgfqpoint{-0.020443in}{0.077083in}}{\pgfqpoint{-0.040051in}{0.068961in}}{\pgfqpoint{-0.054506in}{0.054506in}}%
\pgfpathcurveto{\pgfqpoint{-0.068961in}{0.040051in}}{\pgfqpoint{-0.077083in}{0.020443in}}{\pgfqpoint{-0.077083in}{0.000000in}}%
\pgfpathcurveto{\pgfqpoint{-0.077083in}{-0.020443in}}{\pgfqpoint{-0.068961in}{-0.040051in}}{\pgfqpoint{-0.054506in}{-0.054506in}}%
\pgfpathcurveto{\pgfqpoint{-0.040051in}{-0.068961in}}{\pgfqpoint{-0.020443in}{-0.077083in}}{\pgfqpoint{0.000000in}{-0.077083in}}%
\pgfpathclose%
\pgfusepath{stroke,fill}%
}%
\begin{pgfscope}%
\pgfsys@transformshift{4.805435in}{2.226152in}%
\pgfsys@useobject{currentmarker}{}%
\end{pgfscope}%
\end{pgfscope}%
\begin{pgfscope}%
\pgfpathrectangle{\pgfqpoint{0.100000in}{0.100000in}}{\pgfqpoint{5.307240in}{3.397500in}}%
\pgfusepath{clip}%
\pgfsetrectcap%
\pgfsetroundjoin%
\pgfsetlinewidth{1.505625pt}%
\definecolor{currentstroke}{rgb}{0.678431,1.000000,0.184314}%
\pgfsetstrokecolor{currentstroke}%
\pgfsetstrokeopacity{0.500000}%
\pgfsetdash{}{0pt}%
\pgfpathmoveto{\pgfqpoint{4.373852in}{2.200312in}}%
\pgfusepath{stroke}%
\end{pgfscope}%
\begin{pgfscope}%
\pgfpathrectangle{\pgfqpoint{0.100000in}{0.100000in}}{\pgfqpoint{5.307240in}{3.397500in}}%
\pgfusepath{clip}%
\pgfsetbuttcap%
\pgfsetroundjoin%
\definecolor{currentfill}{rgb}{0.678431,1.000000,0.184314}%
\pgfsetfillcolor{currentfill}%
\pgfsetfillopacity{0.500000}%
\pgfsetlinewidth{0.250937pt}%
\definecolor{currentstroke}{rgb}{0.000000,0.000000,0.000000}%
\pgfsetstrokecolor{currentstroke}%
\pgfsetstrokeopacity{0.500000}%
\pgfsetdash{}{0pt}%
\pgfsys@defobject{currentmarker}{\pgfqpoint{-0.088889in}{-0.088889in}}{\pgfqpoint{0.088889in}{0.088889in}}{%
\pgfpathmoveto{\pgfqpoint{0.000000in}{-0.088889in}}%
\pgfpathcurveto{\pgfqpoint{0.023574in}{-0.088889in}}{\pgfqpoint{0.046185in}{-0.079523in}}{\pgfqpoint{0.062854in}{-0.062854in}}%
\pgfpathcurveto{\pgfqpoint{0.079523in}{-0.046185in}}{\pgfqpoint{0.088889in}{-0.023574in}}{\pgfqpoint{0.088889in}{0.000000in}}%
\pgfpathcurveto{\pgfqpoint{0.088889in}{0.023574in}}{\pgfqpoint{0.079523in}{0.046185in}}{\pgfqpoint{0.062854in}{0.062854in}}%
\pgfpathcurveto{\pgfqpoint{0.046185in}{0.079523in}}{\pgfqpoint{0.023574in}{0.088889in}}{\pgfqpoint{0.000000in}{0.088889in}}%
\pgfpathcurveto{\pgfqpoint{-0.023574in}{0.088889in}}{\pgfqpoint{-0.046185in}{0.079523in}}{\pgfqpoint{-0.062854in}{0.062854in}}%
\pgfpathcurveto{\pgfqpoint{-0.079523in}{0.046185in}}{\pgfqpoint{-0.088889in}{0.023574in}}{\pgfqpoint{-0.088889in}{0.000000in}}%
\pgfpathcurveto{\pgfqpoint{-0.088889in}{-0.023574in}}{\pgfqpoint{-0.079523in}{-0.046185in}}{\pgfqpoint{-0.062854in}{-0.062854in}}%
\pgfpathcurveto{\pgfqpoint{-0.046185in}{-0.079523in}}{\pgfqpoint{-0.023574in}{-0.088889in}}{\pgfqpoint{0.000000in}{-0.088889in}}%
\pgfpathclose%
\pgfusepath{stroke,fill}%
}%
\begin{pgfscope}%
\pgfsys@transformshift{4.373852in}{2.200312in}%
\pgfsys@useobject{currentmarker}{}%
\end{pgfscope}%
\end{pgfscope}%
\begin{pgfscope}%
\pgfpathrectangle{\pgfqpoint{0.100000in}{0.100000in}}{\pgfqpoint{5.307240in}{3.397500in}}%
\pgfusepath{clip}%
\pgfsetrectcap%
\pgfsetroundjoin%
\pgfsetlinewidth{1.505625pt}%
\definecolor{currentstroke}{rgb}{0.678431,1.000000,0.184314}%
\pgfsetstrokecolor{currentstroke}%
\pgfsetstrokeopacity{0.500000}%
\pgfsetdash{}{0pt}%
\pgfpathmoveto{\pgfqpoint{4.729591in}{2.255228in}}%
\pgfusepath{stroke}%
\end{pgfscope}%
\begin{pgfscope}%
\pgfpathrectangle{\pgfqpoint{0.100000in}{0.100000in}}{\pgfqpoint{5.307240in}{3.397500in}}%
\pgfusepath{clip}%
\pgfsetbuttcap%
\pgfsetroundjoin%
\definecolor{currentfill}{rgb}{0.678431,1.000000,0.184314}%
\pgfsetfillcolor{currentfill}%
\pgfsetfillopacity{0.500000}%
\pgfsetlinewidth{0.250937pt}%
\definecolor{currentstroke}{rgb}{0.000000,0.000000,0.000000}%
\pgfsetstrokecolor{currentstroke}%
\pgfsetstrokeopacity{0.500000}%
\pgfsetdash{}{0pt}%
\pgfsys@defobject{currentmarker}{\pgfqpoint{-0.094444in}{-0.094444in}}{\pgfqpoint{0.094444in}{0.094444in}}{%
\pgfpathmoveto{\pgfqpoint{0.000000in}{-0.094444in}}%
\pgfpathcurveto{\pgfqpoint{0.025047in}{-0.094444in}}{\pgfqpoint{0.049071in}{-0.084493in}}{\pgfqpoint{0.066782in}{-0.066782in}}%
\pgfpathcurveto{\pgfqpoint{0.084493in}{-0.049071in}}{\pgfqpoint{0.094444in}{-0.025047in}}{\pgfqpoint{0.094444in}{0.000000in}}%
\pgfpathcurveto{\pgfqpoint{0.094444in}{0.025047in}}{\pgfqpoint{0.084493in}{0.049071in}}{\pgfqpoint{0.066782in}{0.066782in}}%
\pgfpathcurveto{\pgfqpoint{0.049071in}{0.084493in}}{\pgfqpoint{0.025047in}{0.094444in}}{\pgfqpoint{0.000000in}{0.094444in}}%
\pgfpathcurveto{\pgfqpoint{-0.025047in}{0.094444in}}{\pgfqpoint{-0.049071in}{0.084493in}}{\pgfqpoint{-0.066782in}{0.066782in}}%
\pgfpathcurveto{\pgfqpoint{-0.084493in}{0.049071in}}{\pgfqpoint{-0.094444in}{0.025047in}}{\pgfqpoint{-0.094444in}{0.000000in}}%
\pgfpathcurveto{\pgfqpoint{-0.094444in}{-0.025047in}}{\pgfqpoint{-0.084493in}{-0.049071in}}{\pgfqpoint{-0.066782in}{-0.066782in}}%
\pgfpathcurveto{\pgfqpoint{-0.049071in}{-0.084493in}}{\pgfqpoint{-0.025047in}{-0.094444in}}{\pgfqpoint{0.000000in}{-0.094444in}}%
\pgfpathclose%
\pgfusepath{stroke,fill}%
}%
\begin{pgfscope}%
\pgfsys@transformshift{4.729591in}{2.255228in}%
\pgfsys@useobject{currentmarker}{}%
\end{pgfscope}%
\end{pgfscope}%
\begin{pgfscope}%
\pgfpathrectangle{\pgfqpoint{0.100000in}{0.100000in}}{\pgfqpoint{5.307240in}{3.397500in}}%
\pgfusepath{clip}%
\pgfsetrectcap%
\pgfsetroundjoin%
\pgfsetlinewidth{1.505625pt}%
\definecolor{currentstroke}{rgb}{0.678431,1.000000,0.184314}%
\pgfsetstrokecolor{currentstroke}%
\pgfsetstrokeopacity{0.500000}%
\pgfsetdash{}{0pt}%
\pgfpathmoveto{\pgfqpoint{4.726263in}{2.382193in}}%
\pgfusepath{stroke}%
\end{pgfscope}%
\begin{pgfscope}%
\pgfpathrectangle{\pgfqpoint{0.100000in}{0.100000in}}{\pgfqpoint{5.307240in}{3.397500in}}%
\pgfusepath{clip}%
\pgfsetbuttcap%
\pgfsetroundjoin%
\definecolor{currentfill}{rgb}{0.678431,1.000000,0.184314}%
\pgfsetfillcolor{currentfill}%
\pgfsetfillopacity{0.500000}%
\pgfsetlinewidth{0.250937pt}%
\definecolor{currentstroke}{rgb}{0.000000,0.000000,0.000000}%
\pgfsetstrokecolor{currentstroke}%
\pgfsetstrokeopacity{0.500000}%
\pgfsetdash{}{0pt}%
\pgfsys@defobject{currentmarker}{\pgfqpoint{-0.093056in}{-0.093056in}}{\pgfqpoint{0.093056in}{0.093056in}}{%
\pgfpathmoveto{\pgfqpoint{0.000000in}{-0.093056in}}%
\pgfpathcurveto{\pgfqpoint{0.024679in}{-0.093056in}}{\pgfqpoint{0.048350in}{-0.083251in}}{\pgfqpoint{0.065800in}{-0.065800in}}%
\pgfpathcurveto{\pgfqpoint{0.083251in}{-0.048350in}}{\pgfqpoint{0.093056in}{-0.024679in}}{\pgfqpoint{0.093056in}{0.000000in}}%
\pgfpathcurveto{\pgfqpoint{0.093056in}{0.024679in}}{\pgfqpoint{0.083251in}{0.048350in}}{\pgfqpoint{0.065800in}{0.065800in}}%
\pgfpathcurveto{\pgfqpoint{0.048350in}{0.083251in}}{\pgfqpoint{0.024679in}{0.093056in}}{\pgfqpoint{0.000000in}{0.093056in}}%
\pgfpathcurveto{\pgfqpoint{-0.024679in}{0.093056in}}{\pgfqpoint{-0.048350in}{0.083251in}}{\pgfqpoint{-0.065800in}{0.065800in}}%
\pgfpathcurveto{\pgfqpoint{-0.083251in}{0.048350in}}{\pgfqpoint{-0.093056in}{0.024679in}}{\pgfqpoint{-0.093056in}{0.000000in}}%
\pgfpathcurveto{\pgfqpoint{-0.093056in}{-0.024679in}}{\pgfqpoint{-0.083251in}{-0.048350in}}{\pgfqpoint{-0.065800in}{-0.065800in}}%
\pgfpathcurveto{\pgfqpoint{-0.048350in}{-0.083251in}}{\pgfqpoint{-0.024679in}{-0.093056in}}{\pgfqpoint{0.000000in}{-0.093056in}}%
\pgfpathclose%
\pgfusepath{stroke,fill}%
}%
\begin{pgfscope}%
\pgfsys@transformshift{4.726263in}{2.382193in}%
\pgfsys@useobject{currentmarker}{}%
\end{pgfscope}%
\end{pgfscope}%
\begin{pgfscope}%
\pgfpathrectangle{\pgfqpoint{0.100000in}{0.100000in}}{\pgfqpoint{5.307240in}{3.397500in}}%
\pgfusepath{clip}%
\pgfsetrectcap%
\pgfsetroundjoin%
\pgfsetlinewidth{1.505625pt}%
\definecolor{currentstroke}{rgb}{0.678431,1.000000,0.184314}%
\pgfsetstrokecolor{currentstroke}%
\pgfsetstrokeopacity{0.500000}%
\pgfsetdash{}{0pt}%
\pgfpathmoveto{\pgfqpoint{4.551674in}{2.273717in}}%
\pgfusepath{stroke}%
\end{pgfscope}%
\begin{pgfscope}%
\pgfpathrectangle{\pgfqpoint{0.100000in}{0.100000in}}{\pgfqpoint{5.307240in}{3.397500in}}%
\pgfusepath{clip}%
\pgfsetbuttcap%
\pgfsetroundjoin%
\definecolor{currentfill}{rgb}{0.678431,1.000000,0.184314}%
\pgfsetfillcolor{currentfill}%
\pgfsetfillopacity{0.500000}%
\pgfsetlinewidth{0.250937pt}%
\definecolor{currentstroke}{rgb}{0.000000,0.000000,0.000000}%
\pgfsetstrokecolor{currentstroke}%
\pgfsetstrokeopacity{0.500000}%
\pgfsetdash{}{0pt}%
\pgfsys@defobject{currentmarker}{\pgfqpoint{-0.049306in}{-0.049306in}}{\pgfqpoint{0.049306in}{0.049306in}}{%
\pgfpathmoveto{\pgfqpoint{0.000000in}{-0.049306in}}%
\pgfpathcurveto{\pgfqpoint{0.013076in}{-0.049306in}}{\pgfqpoint{0.025618in}{-0.044110in}}{\pgfqpoint{0.034864in}{-0.034864in}}%
\pgfpathcurveto{\pgfqpoint{0.044110in}{-0.025618in}}{\pgfqpoint{0.049306in}{-0.013076in}}{\pgfqpoint{0.049306in}{0.000000in}}%
\pgfpathcurveto{\pgfqpoint{0.049306in}{0.013076in}}{\pgfqpoint{0.044110in}{0.025618in}}{\pgfqpoint{0.034864in}{0.034864in}}%
\pgfpathcurveto{\pgfqpoint{0.025618in}{0.044110in}}{\pgfqpoint{0.013076in}{0.049306in}}{\pgfqpoint{0.000000in}{0.049306in}}%
\pgfpathcurveto{\pgfqpoint{-0.013076in}{0.049306in}}{\pgfqpoint{-0.025618in}{0.044110in}}{\pgfqpoint{-0.034864in}{0.034864in}}%
\pgfpathcurveto{\pgfqpoint{-0.044110in}{0.025618in}}{\pgfqpoint{-0.049306in}{0.013076in}}{\pgfqpoint{-0.049306in}{0.000000in}}%
\pgfpathcurveto{\pgfqpoint{-0.049306in}{-0.013076in}}{\pgfqpoint{-0.044110in}{-0.025618in}}{\pgfqpoint{-0.034864in}{-0.034864in}}%
\pgfpathcurveto{\pgfqpoint{-0.025618in}{-0.044110in}}{\pgfqpoint{-0.013076in}{-0.049306in}}{\pgfqpoint{0.000000in}{-0.049306in}}%
\pgfpathclose%
\pgfusepath{stroke,fill}%
}%
\begin{pgfscope}%
\pgfsys@transformshift{4.551674in}{2.273717in}%
\pgfsys@useobject{currentmarker}{}%
\end{pgfscope}%
\end{pgfscope}%
\begin{pgfscope}%
\pgfpathrectangle{\pgfqpoint{0.100000in}{0.100000in}}{\pgfqpoint{5.307240in}{3.397500in}}%
\pgfusepath{clip}%
\pgfsetrectcap%
\pgfsetroundjoin%
\pgfsetlinewidth{1.505625pt}%
\definecolor{currentstroke}{rgb}{0.678431,1.000000,0.184314}%
\pgfsetstrokecolor{currentstroke}%
\pgfsetstrokeopacity{0.500000}%
\pgfsetdash{}{0pt}%
\pgfpathmoveto{\pgfqpoint{4.615386in}{2.340154in}}%
\pgfusepath{stroke}%
\end{pgfscope}%
\begin{pgfscope}%
\pgfpathrectangle{\pgfqpoint{0.100000in}{0.100000in}}{\pgfqpoint{5.307240in}{3.397500in}}%
\pgfusepath{clip}%
\pgfsetbuttcap%
\pgfsetroundjoin%
\definecolor{currentfill}{rgb}{0.678431,1.000000,0.184314}%
\pgfsetfillcolor{currentfill}%
\pgfsetfillopacity{0.500000}%
\pgfsetlinewidth{0.250937pt}%
\definecolor{currentstroke}{rgb}{0.000000,0.000000,0.000000}%
\pgfsetstrokecolor{currentstroke}%
\pgfsetstrokeopacity{0.500000}%
\pgfsetdash{}{0pt}%
\pgfsys@defobject{currentmarker}{\pgfqpoint{-0.090972in}{-0.090972in}}{\pgfqpoint{0.090972in}{0.090972in}}{%
\pgfpathmoveto{\pgfqpoint{0.000000in}{-0.090972in}}%
\pgfpathcurveto{\pgfqpoint{0.024126in}{-0.090972in}}{\pgfqpoint{0.047267in}{-0.081387in}}{\pgfqpoint{0.064327in}{-0.064327in}}%
\pgfpathcurveto{\pgfqpoint{0.081387in}{-0.047267in}}{\pgfqpoint{0.090972in}{-0.024126in}}{\pgfqpoint{0.090972in}{0.000000in}}%
\pgfpathcurveto{\pgfqpoint{0.090972in}{0.024126in}}{\pgfqpoint{0.081387in}{0.047267in}}{\pgfqpoint{0.064327in}{0.064327in}}%
\pgfpathcurveto{\pgfqpoint{0.047267in}{0.081387in}}{\pgfqpoint{0.024126in}{0.090972in}}{\pgfqpoint{0.000000in}{0.090972in}}%
\pgfpathcurveto{\pgfqpoint{-0.024126in}{0.090972in}}{\pgfqpoint{-0.047267in}{0.081387in}}{\pgfqpoint{-0.064327in}{0.064327in}}%
\pgfpathcurveto{\pgfqpoint{-0.081387in}{0.047267in}}{\pgfqpoint{-0.090972in}{0.024126in}}{\pgfqpoint{-0.090972in}{0.000000in}}%
\pgfpathcurveto{\pgfqpoint{-0.090972in}{-0.024126in}}{\pgfqpoint{-0.081387in}{-0.047267in}}{\pgfqpoint{-0.064327in}{-0.064327in}}%
\pgfpathcurveto{\pgfqpoint{-0.047267in}{-0.081387in}}{\pgfqpoint{-0.024126in}{-0.090972in}}{\pgfqpoint{0.000000in}{-0.090972in}}%
\pgfpathclose%
\pgfusepath{stroke,fill}%
}%
\begin{pgfscope}%
\pgfsys@transformshift{4.615386in}{2.340154in}%
\pgfsys@useobject{currentmarker}{}%
\end{pgfscope}%
\end{pgfscope}%
\begin{pgfscope}%
\pgfpathrectangle{\pgfqpoint{0.100000in}{0.100000in}}{\pgfqpoint{5.307240in}{3.397500in}}%
\pgfusepath{clip}%
\pgfsetrectcap%
\pgfsetroundjoin%
\pgfsetlinewidth{1.505625pt}%
\definecolor{currentstroke}{rgb}{0.678431,1.000000,0.184314}%
\pgfsetstrokecolor{currentstroke}%
\pgfsetstrokeopacity{0.500000}%
\pgfsetdash{}{0pt}%
\pgfpathmoveto{\pgfqpoint{4.668884in}{2.198282in}}%
\pgfusepath{stroke}%
\end{pgfscope}%
\begin{pgfscope}%
\pgfpathrectangle{\pgfqpoint{0.100000in}{0.100000in}}{\pgfqpoint{5.307240in}{3.397500in}}%
\pgfusepath{clip}%
\pgfsetbuttcap%
\pgfsetroundjoin%
\definecolor{currentfill}{rgb}{0.678431,1.000000,0.184314}%
\pgfsetfillcolor{currentfill}%
\pgfsetfillopacity{0.500000}%
\pgfsetlinewidth{0.250937pt}%
\definecolor{currentstroke}{rgb}{0.000000,0.000000,0.000000}%
\pgfsetstrokecolor{currentstroke}%
\pgfsetstrokeopacity{0.500000}%
\pgfsetdash{}{0pt}%
\pgfsys@defobject{currentmarker}{\pgfqpoint{-0.081944in}{-0.081944in}}{\pgfqpoint{0.081944in}{0.081944in}}{%
\pgfpathmoveto{\pgfqpoint{0.000000in}{-0.081944in}}%
\pgfpathcurveto{\pgfqpoint{0.021732in}{-0.081944in}}{\pgfqpoint{0.042577in}{-0.073310in}}{\pgfqpoint{0.057943in}{-0.057943in}}%
\pgfpathcurveto{\pgfqpoint{0.073310in}{-0.042577in}}{\pgfqpoint{0.081944in}{-0.021732in}}{\pgfqpoint{0.081944in}{0.000000in}}%
\pgfpathcurveto{\pgfqpoint{0.081944in}{0.021732in}}{\pgfqpoint{0.073310in}{0.042577in}}{\pgfqpoint{0.057943in}{0.057943in}}%
\pgfpathcurveto{\pgfqpoint{0.042577in}{0.073310in}}{\pgfqpoint{0.021732in}{0.081944in}}{\pgfqpoint{0.000000in}{0.081944in}}%
\pgfpathcurveto{\pgfqpoint{-0.021732in}{0.081944in}}{\pgfqpoint{-0.042577in}{0.073310in}}{\pgfqpoint{-0.057943in}{0.057943in}}%
\pgfpathcurveto{\pgfqpoint{-0.073310in}{0.042577in}}{\pgfqpoint{-0.081944in}{0.021732in}}{\pgfqpoint{-0.081944in}{0.000000in}}%
\pgfpathcurveto{\pgfqpoint{-0.081944in}{-0.021732in}}{\pgfqpoint{-0.073310in}{-0.042577in}}{\pgfqpoint{-0.057943in}{-0.057943in}}%
\pgfpathcurveto{\pgfqpoint{-0.042577in}{-0.073310in}}{\pgfqpoint{-0.021732in}{-0.081944in}}{\pgfqpoint{0.000000in}{-0.081944in}}%
\pgfpathclose%
\pgfusepath{stroke,fill}%
}%
\begin{pgfscope}%
\pgfsys@transformshift{4.668884in}{2.198282in}%
\pgfsys@useobject{currentmarker}{}%
\end{pgfscope}%
\end{pgfscope}%
\begin{pgfscope}%
\pgfpathrectangle{\pgfqpoint{0.100000in}{0.100000in}}{\pgfqpoint{5.307240in}{3.397500in}}%
\pgfusepath{clip}%
\pgfsetrectcap%
\pgfsetroundjoin%
\pgfsetlinewidth{1.505625pt}%
\definecolor{currentstroke}{rgb}{0.678431,1.000000,0.184314}%
\pgfsetstrokecolor{currentstroke}%
\pgfsetstrokeopacity{0.500000}%
\pgfsetdash{}{0pt}%
\pgfpathmoveto{\pgfqpoint{5.074332in}{2.515653in}}%
\pgfusepath{stroke}%
\end{pgfscope}%
\begin{pgfscope}%
\pgfpathrectangle{\pgfqpoint{0.100000in}{0.100000in}}{\pgfqpoint{5.307240in}{3.397500in}}%
\pgfusepath{clip}%
\pgfsetbuttcap%
\pgfsetroundjoin%
\definecolor{currentfill}{rgb}{0.678431,1.000000,0.184314}%
\pgfsetfillcolor{currentfill}%
\pgfsetfillopacity{0.500000}%
\pgfsetlinewidth{0.250937pt}%
\definecolor{currentstroke}{rgb}{0.000000,0.000000,0.000000}%
\pgfsetstrokecolor{currentstroke}%
\pgfsetstrokeopacity{0.500000}%
\pgfsetdash{}{0pt}%
\pgfsys@defobject{currentmarker}{\pgfqpoint{-0.103472in}{-0.103472in}}{\pgfqpoint{0.103472in}{0.103472in}}{%
\pgfpathmoveto{\pgfqpoint{0.000000in}{-0.103472in}}%
\pgfpathcurveto{\pgfqpoint{0.027441in}{-0.103472in}}{\pgfqpoint{0.053762in}{-0.092570in}}{\pgfqpoint{0.073166in}{-0.073166in}}%
\pgfpathcurveto{\pgfqpoint{0.092570in}{-0.053762in}}{\pgfqpoint{0.103472in}{-0.027441in}}{\pgfqpoint{0.103472in}{0.000000in}}%
\pgfpathcurveto{\pgfqpoint{0.103472in}{0.027441in}}{\pgfqpoint{0.092570in}{0.053762in}}{\pgfqpoint{0.073166in}{0.073166in}}%
\pgfpathcurveto{\pgfqpoint{0.053762in}{0.092570in}}{\pgfqpoint{0.027441in}{0.103472in}}{\pgfqpoint{0.000000in}{0.103472in}}%
\pgfpathcurveto{\pgfqpoint{-0.027441in}{0.103472in}}{\pgfqpoint{-0.053762in}{0.092570in}}{\pgfqpoint{-0.073166in}{0.073166in}}%
\pgfpathcurveto{\pgfqpoint{-0.092570in}{0.053762in}}{\pgfqpoint{-0.103472in}{0.027441in}}{\pgfqpoint{-0.103472in}{0.000000in}}%
\pgfpathcurveto{\pgfqpoint{-0.103472in}{-0.027441in}}{\pgfqpoint{-0.092570in}{-0.053762in}}{\pgfqpoint{-0.073166in}{-0.073166in}}%
\pgfpathcurveto{\pgfqpoint{-0.053762in}{-0.092570in}}{\pgfqpoint{-0.027441in}{-0.103472in}}{\pgfqpoint{0.000000in}{-0.103472in}}%
\pgfpathclose%
\pgfusepath{stroke,fill}%
}%
\begin{pgfscope}%
\pgfsys@transformshift{5.074332in}{2.515653in}%
\pgfsys@useobject{currentmarker}{}%
\end{pgfscope}%
\end{pgfscope}%
\begin{pgfscope}%
\pgfpathrectangle{\pgfqpoint{0.100000in}{0.100000in}}{\pgfqpoint{5.307240in}{3.397500in}}%
\pgfusepath{clip}%
\pgfsetrectcap%
\pgfsetroundjoin%
\pgfsetlinewidth{1.505625pt}%
\definecolor{currentstroke}{rgb}{0.678431,1.000000,0.184314}%
\pgfsetstrokecolor{currentstroke}%
\pgfsetstrokeopacity{0.500000}%
\pgfsetdash{}{0pt}%
\pgfpathmoveto{\pgfqpoint{4.525219in}{1.322000in}}%
\pgfusepath{stroke}%
\end{pgfscope}%
\begin{pgfscope}%
\pgfpathrectangle{\pgfqpoint{0.100000in}{0.100000in}}{\pgfqpoint{5.307240in}{3.397500in}}%
\pgfusepath{clip}%
\pgfsetbuttcap%
\pgfsetroundjoin%
\definecolor{currentfill}{rgb}{0.678431,1.000000,0.184314}%
\pgfsetfillcolor{currentfill}%
\pgfsetfillopacity{0.500000}%
\pgfsetlinewidth{0.250937pt}%
\definecolor{currentstroke}{rgb}{0.000000,0.000000,0.000000}%
\pgfsetstrokecolor{currentstroke}%
\pgfsetstrokeopacity{0.500000}%
\pgfsetdash{}{0pt}%
\pgfsys@defobject{currentmarker}{\pgfqpoint{-0.068750in}{-0.068750in}}{\pgfqpoint{0.068750in}{0.068750in}}{%
\pgfpathmoveto{\pgfqpoint{0.000000in}{-0.068750in}}%
\pgfpathcurveto{\pgfqpoint{0.018233in}{-0.068750in}}{\pgfqpoint{0.035721in}{-0.061506in}}{\pgfqpoint{0.048614in}{-0.048614in}}%
\pgfpathcurveto{\pgfqpoint{0.061506in}{-0.035721in}}{\pgfqpoint{0.068750in}{-0.018233in}}{\pgfqpoint{0.068750in}{0.000000in}}%
\pgfpathcurveto{\pgfqpoint{0.068750in}{0.018233in}}{\pgfqpoint{0.061506in}{0.035721in}}{\pgfqpoint{0.048614in}{0.048614in}}%
\pgfpathcurveto{\pgfqpoint{0.035721in}{0.061506in}}{\pgfqpoint{0.018233in}{0.068750in}}{\pgfqpoint{0.000000in}{0.068750in}}%
\pgfpathcurveto{\pgfqpoint{-0.018233in}{0.068750in}}{\pgfqpoint{-0.035721in}{0.061506in}}{\pgfqpoint{-0.048614in}{0.048614in}}%
\pgfpathcurveto{\pgfqpoint{-0.061506in}{0.035721in}}{\pgfqpoint{-0.068750in}{0.018233in}}{\pgfqpoint{-0.068750in}{0.000000in}}%
\pgfpathcurveto{\pgfqpoint{-0.068750in}{-0.018233in}}{\pgfqpoint{-0.061506in}{-0.035721in}}{\pgfqpoint{-0.048614in}{-0.048614in}}%
\pgfpathcurveto{\pgfqpoint{-0.035721in}{-0.061506in}}{\pgfqpoint{-0.018233in}{-0.068750in}}{\pgfqpoint{0.000000in}{-0.068750in}}%
\pgfpathclose%
\pgfusepath{stroke,fill}%
}%
\begin{pgfscope}%
\pgfsys@transformshift{4.525219in}{1.322000in}%
\pgfsys@useobject{currentmarker}{}%
\end{pgfscope}%
\end{pgfscope}%
\begin{pgfscope}%
\pgfpathrectangle{\pgfqpoint{0.100000in}{0.100000in}}{\pgfqpoint{5.307240in}{3.397500in}}%
\pgfusepath{clip}%
\pgfsetrectcap%
\pgfsetroundjoin%
\pgfsetlinewidth{1.505625pt}%
\definecolor{currentstroke}{rgb}{0.678431,1.000000,0.184314}%
\pgfsetstrokecolor{currentstroke}%
\pgfsetstrokeopacity{0.500000}%
\pgfsetdash{}{0pt}%
\pgfpathmoveto{\pgfqpoint{4.397344in}{1.444939in}}%
\pgfusepath{stroke}%
\end{pgfscope}%
\begin{pgfscope}%
\pgfpathrectangle{\pgfqpoint{0.100000in}{0.100000in}}{\pgfqpoint{5.307240in}{3.397500in}}%
\pgfusepath{clip}%
\pgfsetbuttcap%
\pgfsetroundjoin%
\definecolor{currentfill}{rgb}{0.678431,1.000000,0.184314}%
\pgfsetfillcolor{currentfill}%
\pgfsetfillopacity{0.500000}%
\pgfsetlinewidth{0.250937pt}%
\definecolor{currentstroke}{rgb}{0.000000,0.000000,0.000000}%
\pgfsetstrokecolor{currentstroke}%
\pgfsetstrokeopacity{0.500000}%
\pgfsetdash{}{0pt}%
\pgfsys@defobject{currentmarker}{\pgfqpoint{-0.043056in}{-0.043056in}}{\pgfqpoint{0.043056in}{0.043056in}}{%
\pgfpathmoveto{\pgfqpoint{0.000000in}{-0.043056in}}%
\pgfpathcurveto{\pgfqpoint{0.011418in}{-0.043056in}}{\pgfqpoint{0.022371in}{-0.038519in}}{\pgfqpoint{0.030445in}{-0.030445in}}%
\pgfpathcurveto{\pgfqpoint{0.038519in}{-0.022371in}}{\pgfqpoint{0.043056in}{-0.011418in}}{\pgfqpoint{0.043056in}{0.000000in}}%
\pgfpathcurveto{\pgfqpoint{0.043056in}{0.011418in}}{\pgfqpoint{0.038519in}{0.022371in}}{\pgfqpoint{0.030445in}{0.030445in}}%
\pgfpathcurveto{\pgfqpoint{0.022371in}{0.038519in}}{\pgfqpoint{0.011418in}{0.043056in}}{\pgfqpoint{0.000000in}{0.043056in}}%
\pgfpathcurveto{\pgfqpoint{-0.011418in}{0.043056in}}{\pgfqpoint{-0.022371in}{0.038519in}}{\pgfqpoint{-0.030445in}{0.030445in}}%
\pgfpathcurveto{\pgfqpoint{-0.038519in}{0.022371in}}{\pgfqpoint{-0.043056in}{0.011418in}}{\pgfqpoint{-0.043056in}{0.000000in}}%
\pgfpathcurveto{\pgfqpoint{-0.043056in}{-0.011418in}}{\pgfqpoint{-0.038519in}{-0.022371in}}{\pgfqpoint{-0.030445in}{-0.030445in}}%
\pgfpathcurveto{\pgfqpoint{-0.022371in}{-0.038519in}}{\pgfqpoint{-0.011418in}{-0.043056in}}{\pgfqpoint{0.000000in}{-0.043056in}}%
\pgfpathclose%
\pgfusepath{stroke,fill}%
}%
\begin{pgfscope}%
\pgfsys@transformshift{4.397344in}{1.444939in}%
\pgfsys@useobject{currentmarker}{}%
\end{pgfscope}%
\end{pgfscope}%
\begin{pgfscope}%
\pgfpathrectangle{\pgfqpoint{0.100000in}{0.100000in}}{\pgfqpoint{5.307240in}{3.397500in}}%
\pgfusepath{clip}%
\pgfsetrectcap%
\pgfsetroundjoin%
\pgfsetlinewidth{1.505625pt}%
\definecolor{currentstroke}{rgb}{0.678431,1.000000,0.184314}%
\pgfsetstrokecolor{currentstroke}%
\pgfsetstrokeopacity{0.500000}%
\pgfsetdash{}{0pt}%
\pgfpathmoveto{\pgfqpoint{4.514510in}{1.487292in}}%
\pgfusepath{stroke}%
\end{pgfscope}%
\begin{pgfscope}%
\pgfpathrectangle{\pgfqpoint{0.100000in}{0.100000in}}{\pgfqpoint{5.307240in}{3.397500in}}%
\pgfusepath{clip}%
\pgfsetbuttcap%
\pgfsetroundjoin%
\definecolor{currentfill}{rgb}{0.678431,1.000000,0.184314}%
\pgfsetfillcolor{currentfill}%
\pgfsetfillopacity{0.500000}%
\pgfsetlinewidth{0.250937pt}%
\definecolor{currentstroke}{rgb}{0.000000,0.000000,0.000000}%
\pgfsetstrokecolor{currentstroke}%
\pgfsetstrokeopacity{0.500000}%
\pgfsetdash{}{0pt}%
\pgfsys@defobject{currentmarker}{\pgfqpoint{-0.043750in}{-0.043750in}}{\pgfqpoint{0.043750in}{0.043750in}}{%
\pgfpathmoveto{\pgfqpoint{0.000000in}{-0.043750in}}%
\pgfpathcurveto{\pgfqpoint{0.011603in}{-0.043750in}}{\pgfqpoint{0.022732in}{-0.039140in}}{\pgfqpoint{0.030936in}{-0.030936in}}%
\pgfpathcurveto{\pgfqpoint{0.039140in}{-0.022732in}}{\pgfqpoint{0.043750in}{-0.011603in}}{\pgfqpoint{0.043750in}{0.000000in}}%
\pgfpathcurveto{\pgfqpoint{0.043750in}{0.011603in}}{\pgfqpoint{0.039140in}{0.022732in}}{\pgfqpoint{0.030936in}{0.030936in}}%
\pgfpathcurveto{\pgfqpoint{0.022732in}{0.039140in}}{\pgfqpoint{0.011603in}{0.043750in}}{\pgfqpoint{0.000000in}{0.043750in}}%
\pgfpathcurveto{\pgfqpoint{-0.011603in}{0.043750in}}{\pgfqpoint{-0.022732in}{0.039140in}}{\pgfqpoint{-0.030936in}{0.030936in}}%
\pgfpathcurveto{\pgfqpoint{-0.039140in}{0.022732in}}{\pgfqpoint{-0.043750in}{0.011603in}}{\pgfqpoint{-0.043750in}{0.000000in}}%
\pgfpathcurveto{\pgfqpoint{-0.043750in}{-0.011603in}}{\pgfqpoint{-0.039140in}{-0.022732in}}{\pgfqpoint{-0.030936in}{-0.030936in}}%
\pgfpathcurveto{\pgfqpoint{-0.022732in}{-0.039140in}}{\pgfqpoint{-0.011603in}{-0.043750in}}{\pgfqpoint{0.000000in}{-0.043750in}}%
\pgfpathclose%
\pgfusepath{stroke,fill}%
}%
\begin{pgfscope}%
\pgfsys@transformshift{4.514510in}{1.487292in}%
\pgfsys@useobject{currentmarker}{}%
\end{pgfscope}%
\end{pgfscope}%
\begin{pgfscope}%
\pgfpathrectangle{\pgfqpoint{0.100000in}{0.100000in}}{\pgfqpoint{5.307240in}{3.397500in}}%
\pgfusepath{clip}%
\pgfsetrectcap%
\pgfsetroundjoin%
\pgfsetlinewidth{1.505625pt}%
\definecolor{currentstroke}{rgb}{0.678431,1.000000,0.184314}%
\pgfsetstrokecolor{currentstroke}%
\pgfsetstrokeopacity{0.500000}%
\pgfsetdash{}{0pt}%
\pgfpathmoveto{\pgfqpoint{4.234613in}{1.480874in}}%
\pgfusepath{stroke}%
\end{pgfscope}%
\begin{pgfscope}%
\pgfpathrectangle{\pgfqpoint{0.100000in}{0.100000in}}{\pgfqpoint{5.307240in}{3.397500in}}%
\pgfusepath{clip}%
\pgfsetbuttcap%
\pgfsetroundjoin%
\definecolor{currentfill}{rgb}{0.678431,1.000000,0.184314}%
\pgfsetfillcolor{currentfill}%
\pgfsetfillopacity{0.500000}%
\pgfsetlinewidth{0.250937pt}%
\definecolor{currentstroke}{rgb}{0.000000,0.000000,0.000000}%
\pgfsetstrokecolor{currentstroke}%
\pgfsetstrokeopacity{0.500000}%
\pgfsetdash{}{0pt}%
\pgfsys@defobject{currentmarker}{\pgfqpoint{-0.069444in}{-0.069444in}}{\pgfqpoint{0.069444in}{0.069444in}}{%
\pgfpathmoveto{\pgfqpoint{0.000000in}{-0.069444in}}%
\pgfpathcurveto{\pgfqpoint{0.018417in}{-0.069444in}}{\pgfqpoint{0.036082in}{-0.062127in}}{\pgfqpoint{0.049105in}{-0.049105in}}%
\pgfpathcurveto{\pgfqpoint{0.062127in}{-0.036082in}}{\pgfqpoint{0.069444in}{-0.018417in}}{\pgfqpoint{0.069444in}{0.000000in}}%
\pgfpathcurveto{\pgfqpoint{0.069444in}{0.018417in}}{\pgfqpoint{0.062127in}{0.036082in}}{\pgfqpoint{0.049105in}{0.049105in}}%
\pgfpathcurveto{\pgfqpoint{0.036082in}{0.062127in}}{\pgfqpoint{0.018417in}{0.069444in}}{\pgfqpoint{0.000000in}{0.069444in}}%
\pgfpathcurveto{\pgfqpoint{-0.018417in}{0.069444in}}{\pgfqpoint{-0.036082in}{0.062127in}}{\pgfqpoint{-0.049105in}{0.049105in}}%
\pgfpathcurveto{\pgfqpoint{-0.062127in}{0.036082in}}{\pgfqpoint{-0.069444in}{0.018417in}}{\pgfqpoint{-0.069444in}{0.000000in}}%
\pgfpathcurveto{\pgfqpoint{-0.069444in}{-0.018417in}}{\pgfqpoint{-0.062127in}{-0.036082in}}{\pgfqpoint{-0.049105in}{-0.049105in}}%
\pgfpathcurveto{\pgfqpoint{-0.036082in}{-0.062127in}}{\pgfqpoint{-0.018417in}{-0.069444in}}{\pgfqpoint{0.000000in}{-0.069444in}}%
\pgfpathclose%
\pgfusepath{stroke,fill}%
}%
\begin{pgfscope}%
\pgfsys@transformshift{4.234613in}{1.480874in}%
\pgfsys@useobject{currentmarker}{}%
\end{pgfscope}%
\end{pgfscope}%
\begin{pgfscope}%
\pgfpathrectangle{\pgfqpoint{0.100000in}{0.100000in}}{\pgfqpoint{5.307240in}{3.397500in}}%
\pgfusepath{clip}%
\pgfsetrectcap%
\pgfsetroundjoin%
\pgfsetlinewidth{1.505625pt}%
\definecolor{currentstroke}{rgb}{0.678431,1.000000,0.184314}%
\pgfsetstrokecolor{currentstroke}%
\pgfsetstrokeopacity{0.500000}%
\pgfsetdash{}{0pt}%
\pgfpathmoveto{\pgfqpoint{4.253063in}{1.523968in}}%
\pgfusepath{stroke}%
\end{pgfscope}%
\begin{pgfscope}%
\pgfpathrectangle{\pgfqpoint{0.100000in}{0.100000in}}{\pgfqpoint{5.307240in}{3.397500in}}%
\pgfusepath{clip}%
\pgfsetbuttcap%
\pgfsetroundjoin%
\definecolor{currentfill}{rgb}{0.678431,1.000000,0.184314}%
\pgfsetfillcolor{currentfill}%
\pgfsetfillopacity{0.500000}%
\pgfsetlinewidth{0.250937pt}%
\definecolor{currentstroke}{rgb}{0.000000,0.000000,0.000000}%
\pgfsetstrokecolor{currentstroke}%
\pgfsetstrokeopacity{0.500000}%
\pgfsetdash{}{0pt}%
\pgfsys@defobject{currentmarker}{\pgfqpoint{-0.069444in}{-0.069444in}}{\pgfqpoint{0.069444in}{0.069444in}}{%
\pgfpathmoveto{\pgfqpoint{0.000000in}{-0.069444in}}%
\pgfpathcurveto{\pgfqpoint{0.018417in}{-0.069444in}}{\pgfqpoint{0.036082in}{-0.062127in}}{\pgfqpoint{0.049105in}{-0.049105in}}%
\pgfpathcurveto{\pgfqpoint{0.062127in}{-0.036082in}}{\pgfqpoint{0.069444in}{-0.018417in}}{\pgfqpoint{0.069444in}{0.000000in}}%
\pgfpathcurveto{\pgfqpoint{0.069444in}{0.018417in}}{\pgfqpoint{0.062127in}{0.036082in}}{\pgfqpoint{0.049105in}{0.049105in}}%
\pgfpathcurveto{\pgfqpoint{0.036082in}{0.062127in}}{\pgfqpoint{0.018417in}{0.069444in}}{\pgfqpoint{0.000000in}{0.069444in}}%
\pgfpathcurveto{\pgfqpoint{-0.018417in}{0.069444in}}{\pgfqpoint{-0.036082in}{0.062127in}}{\pgfqpoint{-0.049105in}{0.049105in}}%
\pgfpathcurveto{\pgfqpoint{-0.062127in}{0.036082in}}{\pgfqpoint{-0.069444in}{0.018417in}}{\pgfqpoint{-0.069444in}{0.000000in}}%
\pgfpathcurveto{\pgfqpoint{-0.069444in}{-0.018417in}}{\pgfqpoint{-0.062127in}{-0.036082in}}{\pgfqpoint{-0.049105in}{-0.049105in}}%
\pgfpathcurveto{\pgfqpoint{-0.036082in}{-0.062127in}}{\pgfqpoint{-0.018417in}{-0.069444in}}{\pgfqpoint{0.000000in}{-0.069444in}}%
\pgfpathclose%
\pgfusepath{stroke,fill}%
}%
\begin{pgfscope}%
\pgfsys@transformshift{4.253063in}{1.523968in}%
\pgfsys@useobject{currentmarker}{}%
\end{pgfscope}%
\end{pgfscope}%
\begin{pgfscope}%
\pgfpathrectangle{\pgfqpoint{0.100000in}{0.100000in}}{\pgfqpoint{5.307240in}{3.397500in}}%
\pgfusepath{clip}%
\pgfsetrectcap%
\pgfsetroundjoin%
\pgfsetlinewidth{1.505625pt}%
\definecolor{currentstroke}{rgb}{0.678431,1.000000,0.184314}%
\pgfsetstrokecolor{currentstroke}%
\pgfsetstrokeopacity{0.500000}%
\pgfsetdash{}{0pt}%
\pgfpathmoveto{\pgfqpoint{4.456834in}{1.243159in}}%
\pgfusepath{stroke}%
\end{pgfscope}%
\begin{pgfscope}%
\pgfpathrectangle{\pgfqpoint{0.100000in}{0.100000in}}{\pgfqpoint{5.307240in}{3.397500in}}%
\pgfusepath{clip}%
\pgfsetbuttcap%
\pgfsetroundjoin%
\definecolor{currentfill}{rgb}{0.678431,1.000000,0.184314}%
\pgfsetfillcolor{currentfill}%
\pgfsetfillopacity{0.500000}%
\pgfsetlinewidth{0.250937pt}%
\definecolor{currentstroke}{rgb}{0.000000,0.000000,0.000000}%
\pgfsetstrokecolor{currentstroke}%
\pgfsetstrokeopacity{0.500000}%
\pgfsetdash{}{0pt}%
\pgfsys@defobject{currentmarker}{\pgfqpoint{-0.063194in}{-0.063194in}}{\pgfqpoint{0.063194in}{0.063194in}}{%
\pgfpathmoveto{\pgfqpoint{0.000000in}{-0.063194in}}%
\pgfpathcurveto{\pgfqpoint{0.016759in}{-0.063194in}}{\pgfqpoint{0.032835in}{-0.056536in}}{\pgfqpoint{0.044685in}{-0.044685in}}%
\pgfpathcurveto{\pgfqpoint{0.056536in}{-0.032835in}}{\pgfqpoint{0.063194in}{-0.016759in}}{\pgfqpoint{0.063194in}{0.000000in}}%
\pgfpathcurveto{\pgfqpoint{0.063194in}{0.016759in}}{\pgfqpoint{0.056536in}{0.032835in}}{\pgfqpoint{0.044685in}{0.044685in}}%
\pgfpathcurveto{\pgfqpoint{0.032835in}{0.056536in}}{\pgfqpoint{0.016759in}{0.063194in}}{\pgfqpoint{0.000000in}{0.063194in}}%
\pgfpathcurveto{\pgfqpoint{-0.016759in}{0.063194in}}{\pgfqpoint{-0.032835in}{0.056536in}}{\pgfqpoint{-0.044685in}{0.044685in}}%
\pgfpathcurveto{\pgfqpoint{-0.056536in}{0.032835in}}{\pgfqpoint{-0.063194in}{0.016759in}}{\pgfqpoint{-0.063194in}{0.000000in}}%
\pgfpathcurveto{\pgfqpoint{-0.063194in}{-0.016759in}}{\pgfqpoint{-0.056536in}{-0.032835in}}{\pgfqpoint{-0.044685in}{-0.044685in}}%
\pgfpathcurveto{\pgfqpoint{-0.032835in}{-0.056536in}}{\pgfqpoint{-0.016759in}{-0.063194in}}{\pgfqpoint{0.000000in}{-0.063194in}}%
\pgfpathclose%
\pgfusepath{stroke,fill}%
}%
\begin{pgfscope}%
\pgfsys@transformshift{4.456834in}{1.243159in}%
\pgfsys@useobject{currentmarker}{}%
\end{pgfscope}%
\end{pgfscope}%
\begin{pgfscope}%
\pgfpathrectangle{\pgfqpoint{0.100000in}{0.100000in}}{\pgfqpoint{5.307240in}{3.397500in}}%
\pgfusepath{clip}%
\pgfsetrectcap%
\pgfsetroundjoin%
\pgfsetlinewidth{1.505625pt}%
\definecolor{currentstroke}{rgb}{0.678431,1.000000,0.184314}%
\pgfsetstrokecolor{currentstroke}%
\pgfsetstrokeopacity{0.500000}%
\pgfsetdash{}{0pt}%
\pgfpathmoveto{\pgfqpoint{4.608238in}{1.444054in}}%
\pgfusepath{stroke}%
\end{pgfscope}%
\begin{pgfscope}%
\pgfpathrectangle{\pgfqpoint{0.100000in}{0.100000in}}{\pgfqpoint{5.307240in}{3.397500in}}%
\pgfusepath{clip}%
\pgfsetbuttcap%
\pgfsetroundjoin%
\definecolor{currentfill}{rgb}{0.678431,1.000000,0.184314}%
\pgfsetfillcolor{currentfill}%
\pgfsetfillopacity{0.500000}%
\pgfsetlinewidth{0.250937pt}%
\definecolor{currentstroke}{rgb}{0.000000,0.000000,0.000000}%
\pgfsetstrokecolor{currentstroke}%
\pgfsetstrokeopacity{0.500000}%
\pgfsetdash{}{0pt}%
\pgfsys@defobject{currentmarker}{\pgfqpoint{-0.123611in}{-0.123611in}}{\pgfqpoint{0.123611in}{0.123611in}}{%
\pgfpathmoveto{\pgfqpoint{0.000000in}{-0.123611in}}%
\pgfpathcurveto{\pgfqpoint{0.032782in}{-0.123611in}}{\pgfqpoint{0.064226in}{-0.110587in}}{\pgfqpoint{0.087406in}{-0.087406in}}%
\pgfpathcurveto{\pgfqpoint{0.110587in}{-0.064226in}}{\pgfqpoint{0.123611in}{-0.032782in}}{\pgfqpoint{0.123611in}{0.000000in}}%
\pgfpathcurveto{\pgfqpoint{0.123611in}{0.032782in}}{\pgfqpoint{0.110587in}{0.064226in}}{\pgfqpoint{0.087406in}{0.087406in}}%
\pgfpathcurveto{\pgfqpoint{0.064226in}{0.110587in}}{\pgfqpoint{0.032782in}{0.123611in}}{\pgfqpoint{0.000000in}{0.123611in}}%
\pgfpathcurveto{\pgfqpoint{-0.032782in}{0.123611in}}{\pgfqpoint{-0.064226in}{0.110587in}}{\pgfqpoint{-0.087406in}{0.087406in}}%
\pgfpathcurveto{\pgfqpoint{-0.110587in}{0.064226in}}{\pgfqpoint{-0.123611in}{0.032782in}}{\pgfqpoint{-0.123611in}{0.000000in}}%
\pgfpathcurveto{\pgfqpoint{-0.123611in}{-0.032782in}}{\pgfqpoint{-0.110587in}{-0.064226in}}{\pgfqpoint{-0.087406in}{-0.087406in}}%
\pgfpathcurveto{\pgfqpoint{-0.064226in}{-0.110587in}}{\pgfqpoint{-0.032782in}{-0.123611in}}{\pgfqpoint{0.000000in}{-0.123611in}}%
\pgfpathclose%
\pgfusepath{stroke,fill}%
}%
\begin{pgfscope}%
\pgfsys@transformshift{4.608238in}{1.444054in}%
\pgfsys@useobject{currentmarker}{}%
\end{pgfscope}%
\end{pgfscope}%
\begin{pgfscope}%
\pgfpathrectangle{\pgfqpoint{0.100000in}{0.100000in}}{\pgfqpoint{5.307240in}{3.397500in}}%
\pgfusepath{clip}%
\pgfsetrectcap%
\pgfsetroundjoin%
\pgfsetlinewidth{1.505625pt}%
\definecolor{currentstroke}{rgb}{0.678431,1.000000,0.184314}%
\pgfsetstrokecolor{currentstroke}%
\pgfsetstrokeopacity{0.500000}%
\pgfsetdash{}{0pt}%
\pgfpathmoveto{\pgfqpoint{4.295595in}{1.541605in}}%
\pgfusepath{stroke}%
\end{pgfscope}%
\begin{pgfscope}%
\pgfpathrectangle{\pgfqpoint{0.100000in}{0.100000in}}{\pgfqpoint{5.307240in}{3.397500in}}%
\pgfusepath{clip}%
\pgfsetbuttcap%
\pgfsetroundjoin%
\definecolor{currentfill}{rgb}{0.678431,1.000000,0.184314}%
\pgfsetfillcolor{currentfill}%
\pgfsetfillopacity{0.500000}%
\pgfsetlinewidth{0.250937pt}%
\definecolor{currentstroke}{rgb}{0.000000,0.000000,0.000000}%
\pgfsetstrokecolor{currentstroke}%
\pgfsetstrokeopacity{0.500000}%
\pgfsetdash{}{0pt}%
\pgfsys@defobject{currentmarker}{\pgfqpoint{-0.089583in}{-0.089583in}}{\pgfqpoint{0.089583in}{0.089583in}}{%
\pgfpathmoveto{\pgfqpoint{0.000000in}{-0.089583in}}%
\pgfpathcurveto{\pgfqpoint{0.023758in}{-0.089583in}}{\pgfqpoint{0.046546in}{-0.080144in}}{\pgfqpoint{0.063345in}{-0.063345in}}%
\pgfpathcurveto{\pgfqpoint{0.080144in}{-0.046546in}}{\pgfqpoint{0.089583in}{-0.023758in}}{\pgfqpoint{0.089583in}{0.000000in}}%
\pgfpathcurveto{\pgfqpoint{0.089583in}{0.023758in}}{\pgfqpoint{0.080144in}{0.046546in}}{\pgfqpoint{0.063345in}{0.063345in}}%
\pgfpathcurveto{\pgfqpoint{0.046546in}{0.080144in}}{\pgfqpoint{0.023758in}{0.089583in}}{\pgfqpoint{0.000000in}{0.089583in}}%
\pgfpathcurveto{\pgfqpoint{-0.023758in}{0.089583in}}{\pgfqpoint{-0.046546in}{0.080144in}}{\pgfqpoint{-0.063345in}{0.063345in}}%
\pgfpathcurveto{\pgfqpoint{-0.080144in}{0.046546in}}{\pgfqpoint{-0.089583in}{0.023758in}}{\pgfqpoint{-0.089583in}{0.000000in}}%
\pgfpathcurveto{\pgfqpoint{-0.089583in}{-0.023758in}}{\pgfqpoint{-0.080144in}{-0.046546in}}{\pgfqpoint{-0.063345in}{-0.063345in}}%
\pgfpathcurveto{\pgfqpoint{-0.046546in}{-0.080144in}}{\pgfqpoint{-0.023758in}{-0.089583in}}{\pgfqpoint{0.000000in}{-0.089583in}}%
\pgfpathclose%
\pgfusepath{stroke,fill}%
}%
\begin{pgfscope}%
\pgfsys@transformshift{4.295595in}{1.541605in}%
\pgfsys@useobject{currentmarker}{}%
\end{pgfscope}%
\end{pgfscope}%
\begin{pgfscope}%
\pgfpathrectangle{\pgfqpoint{0.100000in}{0.100000in}}{\pgfqpoint{5.307240in}{3.397500in}}%
\pgfusepath{clip}%
\pgfsetrectcap%
\pgfsetroundjoin%
\pgfsetlinewidth{1.505625pt}%
\definecolor{currentstroke}{rgb}{0.678431,1.000000,0.184314}%
\pgfsetstrokecolor{currentstroke}%
\pgfsetstrokeopacity{0.500000}%
\pgfsetdash{}{0pt}%
\pgfpathmoveto{\pgfqpoint{4.465052in}{1.446200in}}%
\pgfusepath{stroke}%
\end{pgfscope}%
\begin{pgfscope}%
\pgfpathrectangle{\pgfqpoint{0.100000in}{0.100000in}}{\pgfqpoint{5.307240in}{3.397500in}}%
\pgfusepath{clip}%
\pgfsetbuttcap%
\pgfsetroundjoin%
\definecolor{currentfill}{rgb}{0.678431,1.000000,0.184314}%
\pgfsetfillcolor{currentfill}%
\pgfsetfillopacity{0.500000}%
\pgfsetlinewidth{0.250937pt}%
\definecolor{currentstroke}{rgb}{0.000000,0.000000,0.000000}%
\pgfsetstrokecolor{currentstroke}%
\pgfsetstrokeopacity{0.500000}%
\pgfsetdash{}{0pt}%
\pgfsys@defobject{currentmarker}{\pgfqpoint{-0.056250in}{-0.056250in}}{\pgfqpoint{0.056250in}{0.056250in}}{%
\pgfpathmoveto{\pgfqpoint{0.000000in}{-0.056250in}}%
\pgfpathcurveto{\pgfqpoint{0.014918in}{-0.056250in}}{\pgfqpoint{0.029226in}{-0.050323in}}{\pgfqpoint{0.039775in}{-0.039775in}}%
\pgfpathcurveto{\pgfqpoint{0.050323in}{-0.029226in}}{\pgfqpoint{0.056250in}{-0.014918in}}{\pgfqpoint{0.056250in}{0.000000in}}%
\pgfpathcurveto{\pgfqpoint{0.056250in}{0.014918in}}{\pgfqpoint{0.050323in}{0.029226in}}{\pgfqpoint{0.039775in}{0.039775in}}%
\pgfpathcurveto{\pgfqpoint{0.029226in}{0.050323in}}{\pgfqpoint{0.014918in}{0.056250in}}{\pgfqpoint{0.000000in}{0.056250in}}%
\pgfpathcurveto{\pgfqpoint{-0.014918in}{0.056250in}}{\pgfqpoint{-0.029226in}{0.050323in}}{\pgfqpoint{-0.039775in}{0.039775in}}%
\pgfpathcurveto{\pgfqpoint{-0.050323in}{0.029226in}}{\pgfqpoint{-0.056250in}{0.014918in}}{\pgfqpoint{-0.056250in}{0.000000in}}%
\pgfpathcurveto{\pgfqpoint{-0.056250in}{-0.014918in}}{\pgfqpoint{-0.050323in}{-0.029226in}}{\pgfqpoint{-0.039775in}{-0.039775in}}%
\pgfpathcurveto{\pgfqpoint{-0.029226in}{-0.050323in}}{\pgfqpoint{-0.014918in}{-0.056250in}}{\pgfqpoint{0.000000in}{-0.056250in}}%
\pgfpathclose%
\pgfusepath{stroke,fill}%
}%
\begin{pgfscope}%
\pgfsys@transformshift{4.465052in}{1.446200in}%
\pgfsys@useobject{currentmarker}{}%
\end{pgfscope}%
\end{pgfscope}%
\begin{pgfscope}%
\pgfpathrectangle{\pgfqpoint{0.100000in}{0.100000in}}{\pgfqpoint{5.307240in}{3.397500in}}%
\pgfusepath{clip}%
\pgfsetrectcap%
\pgfsetroundjoin%
\pgfsetlinewidth{1.505625pt}%
\definecolor{currentstroke}{rgb}{0.678431,1.000000,0.184314}%
\pgfsetstrokecolor{currentstroke}%
\pgfsetstrokeopacity{0.500000}%
\pgfsetdash{}{0pt}%
\pgfpathmoveto{\pgfqpoint{2.362675in}{2.545890in}}%
\pgfusepath{stroke}%
\end{pgfscope}%
\begin{pgfscope}%
\pgfpathrectangle{\pgfqpoint{0.100000in}{0.100000in}}{\pgfqpoint{5.307240in}{3.397500in}}%
\pgfusepath{clip}%
\pgfsetbuttcap%
\pgfsetroundjoin%
\definecolor{currentfill}{rgb}{0.678431,1.000000,0.184314}%
\pgfsetfillcolor{currentfill}%
\pgfsetfillopacity{0.500000}%
\pgfsetlinewidth{0.250937pt}%
\definecolor{currentstroke}{rgb}{0.000000,0.000000,0.000000}%
\pgfsetstrokecolor{currentstroke}%
\pgfsetstrokeopacity{0.500000}%
\pgfsetdash{}{0pt}%
\pgfsys@defobject{currentmarker}{\pgfqpoint{-0.071528in}{-0.071528in}}{\pgfqpoint{0.071528in}{0.071528in}}{%
\pgfpathmoveto{\pgfqpoint{0.000000in}{-0.071528in}}%
\pgfpathcurveto{\pgfqpoint{0.018969in}{-0.071528in}}{\pgfqpoint{0.037164in}{-0.063991in}}{\pgfqpoint{0.050578in}{-0.050578in}}%
\pgfpathcurveto{\pgfqpoint{0.063991in}{-0.037164in}}{\pgfqpoint{0.071528in}{-0.018969in}}{\pgfqpoint{0.071528in}{0.000000in}}%
\pgfpathcurveto{\pgfqpoint{0.071528in}{0.018969in}}{\pgfqpoint{0.063991in}{0.037164in}}{\pgfqpoint{0.050578in}{0.050578in}}%
\pgfpathcurveto{\pgfqpoint{0.037164in}{0.063991in}}{\pgfqpoint{0.018969in}{0.071528in}}{\pgfqpoint{0.000000in}{0.071528in}}%
\pgfpathcurveto{\pgfqpoint{-0.018969in}{0.071528in}}{\pgfqpoint{-0.037164in}{0.063991in}}{\pgfqpoint{-0.050578in}{0.050578in}}%
\pgfpathcurveto{\pgfqpoint{-0.063991in}{0.037164in}}{\pgfqpoint{-0.071528in}{0.018969in}}{\pgfqpoint{-0.071528in}{0.000000in}}%
\pgfpathcurveto{\pgfqpoint{-0.071528in}{-0.018969in}}{\pgfqpoint{-0.063991in}{-0.037164in}}{\pgfqpoint{-0.050578in}{-0.050578in}}%
\pgfpathcurveto{\pgfqpoint{-0.037164in}{-0.063991in}}{\pgfqpoint{-0.018969in}{-0.071528in}}{\pgfqpoint{0.000000in}{-0.071528in}}%
\pgfpathclose%
\pgfusepath{stroke,fill}%
}%
\begin{pgfscope}%
\pgfsys@transformshift{2.362675in}{2.545890in}%
\pgfsys@useobject{currentmarker}{}%
\end{pgfscope}%
\end{pgfscope}%
\begin{pgfscope}%
\pgfpathrectangle{\pgfqpoint{0.100000in}{0.100000in}}{\pgfqpoint{5.307240in}{3.397500in}}%
\pgfusepath{clip}%
\pgfsetrectcap%
\pgfsetroundjoin%
\pgfsetlinewidth{1.505625pt}%
\definecolor{currentstroke}{rgb}{0.678431,1.000000,0.184314}%
\pgfsetstrokecolor{currentstroke}%
\pgfsetstrokeopacity{0.500000}%
\pgfsetdash{}{0pt}%
\pgfpathmoveto{\pgfqpoint{2.907829in}{2.454061in}}%
\pgfusepath{stroke}%
\end{pgfscope}%
\begin{pgfscope}%
\pgfpathrectangle{\pgfqpoint{0.100000in}{0.100000in}}{\pgfqpoint{5.307240in}{3.397500in}}%
\pgfusepath{clip}%
\pgfsetbuttcap%
\pgfsetroundjoin%
\definecolor{currentfill}{rgb}{0.678431,1.000000,0.184314}%
\pgfsetfillcolor{currentfill}%
\pgfsetfillopacity{0.500000}%
\pgfsetlinewidth{0.250937pt}%
\definecolor{currentstroke}{rgb}{0.000000,0.000000,0.000000}%
\pgfsetstrokecolor{currentstroke}%
\pgfsetstrokeopacity{0.500000}%
\pgfsetdash{}{0pt}%
\pgfsys@defobject{currentmarker}{\pgfqpoint{-0.054167in}{-0.054167in}}{\pgfqpoint{0.054167in}{0.054167in}}{%
\pgfpathmoveto{\pgfqpoint{0.000000in}{-0.054167in}}%
\pgfpathcurveto{\pgfqpoint{0.014365in}{-0.054167in}}{\pgfqpoint{0.028144in}{-0.048459in}}{\pgfqpoint{0.038302in}{-0.038302in}}%
\pgfpathcurveto{\pgfqpoint{0.048459in}{-0.028144in}}{\pgfqpoint{0.054167in}{-0.014365in}}{\pgfqpoint{0.054167in}{0.000000in}}%
\pgfpathcurveto{\pgfqpoint{0.054167in}{0.014365in}}{\pgfqpoint{0.048459in}{0.028144in}}{\pgfqpoint{0.038302in}{0.038302in}}%
\pgfpathcurveto{\pgfqpoint{0.028144in}{0.048459in}}{\pgfqpoint{0.014365in}{0.054167in}}{\pgfqpoint{0.000000in}{0.054167in}}%
\pgfpathcurveto{\pgfqpoint{-0.014365in}{0.054167in}}{\pgfqpoint{-0.028144in}{0.048459in}}{\pgfqpoint{-0.038302in}{0.038302in}}%
\pgfpathcurveto{\pgfqpoint{-0.048459in}{0.028144in}}{\pgfqpoint{-0.054167in}{0.014365in}}{\pgfqpoint{-0.054167in}{0.000000in}}%
\pgfpathcurveto{\pgfqpoint{-0.054167in}{-0.014365in}}{\pgfqpoint{-0.048459in}{-0.028144in}}{\pgfqpoint{-0.038302in}{-0.038302in}}%
\pgfpathcurveto{\pgfqpoint{-0.028144in}{-0.048459in}}{\pgfqpoint{-0.014365in}{-0.054167in}}{\pgfqpoint{0.000000in}{-0.054167in}}%
\pgfpathclose%
\pgfusepath{stroke,fill}%
}%
\begin{pgfscope}%
\pgfsys@transformshift{2.907829in}{2.454061in}%
\pgfsys@useobject{currentmarker}{}%
\end{pgfscope}%
\end{pgfscope}%
\begin{pgfscope}%
\pgfpathrectangle{\pgfqpoint{0.100000in}{0.100000in}}{\pgfqpoint{5.307240in}{3.397500in}}%
\pgfusepath{clip}%
\pgfsetrectcap%
\pgfsetroundjoin%
\pgfsetlinewidth{1.505625pt}%
\definecolor{currentstroke}{rgb}{0.678431,1.000000,0.184314}%
\pgfsetstrokecolor{currentstroke}%
\pgfsetstrokeopacity{0.500000}%
\pgfsetdash{}{0pt}%
\pgfpathmoveto{\pgfqpoint{3.974288in}{1.512415in}}%
\pgfusepath{stroke}%
\end{pgfscope}%
\begin{pgfscope}%
\pgfpathrectangle{\pgfqpoint{0.100000in}{0.100000in}}{\pgfqpoint{5.307240in}{3.397500in}}%
\pgfusepath{clip}%
\pgfsetbuttcap%
\pgfsetroundjoin%
\definecolor{currentfill}{rgb}{0.678431,1.000000,0.184314}%
\pgfsetfillcolor{currentfill}%
\pgfsetfillopacity{0.500000}%
\pgfsetlinewidth{0.250937pt}%
\definecolor{currentstroke}{rgb}{0.000000,0.000000,0.000000}%
\pgfsetstrokecolor{currentstroke}%
\pgfsetstrokeopacity{0.500000}%
\pgfsetdash{}{0pt}%
\pgfsys@defobject{currentmarker}{\pgfqpoint{-0.072222in}{-0.072222in}}{\pgfqpoint{0.072222in}{0.072222in}}{%
\pgfpathmoveto{\pgfqpoint{0.000000in}{-0.072222in}}%
\pgfpathcurveto{\pgfqpoint{0.019154in}{-0.072222in}}{\pgfqpoint{0.037525in}{-0.064612in}}{\pgfqpoint{0.051069in}{-0.051069in}}%
\pgfpathcurveto{\pgfqpoint{0.064612in}{-0.037525in}}{\pgfqpoint{0.072222in}{-0.019154in}}{\pgfqpoint{0.072222in}{0.000000in}}%
\pgfpathcurveto{\pgfqpoint{0.072222in}{0.019154in}}{\pgfqpoint{0.064612in}{0.037525in}}{\pgfqpoint{0.051069in}{0.051069in}}%
\pgfpathcurveto{\pgfqpoint{0.037525in}{0.064612in}}{\pgfqpoint{0.019154in}{0.072222in}}{\pgfqpoint{0.000000in}{0.072222in}}%
\pgfpathcurveto{\pgfqpoint{-0.019154in}{0.072222in}}{\pgfqpoint{-0.037525in}{0.064612in}}{\pgfqpoint{-0.051069in}{0.051069in}}%
\pgfpathcurveto{\pgfqpoint{-0.064612in}{0.037525in}}{\pgfqpoint{-0.072222in}{0.019154in}}{\pgfqpoint{-0.072222in}{0.000000in}}%
\pgfpathcurveto{\pgfqpoint{-0.072222in}{-0.019154in}}{\pgfqpoint{-0.064612in}{-0.037525in}}{\pgfqpoint{-0.051069in}{-0.051069in}}%
\pgfpathcurveto{\pgfqpoint{-0.037525in}{-0.064612in}}{\pgfqpoint{-0.019154in}{-0.072222in}}{\pgfqpoint{0.000000in}{-0.072222in}}%
\pgfpathclose%
\pgfusepath{stroke,fill}%
}%
\begin{pgfscope}%
\pgfsys@transformshift{3.974288in}{1.512415in}%
\pgfsys@useobject{currentmarker}{}%
\end{pgfscope}%
\end{pgfscope}%
\begin{pgfscope}%
\pgfpathrectangle{\pgfqpoint{0.100000in}{0.100000in}}{\pgfqpoint{5.307240in}{3.397500in}}%
\pgfusepath{clip}%
\pgfsetrectcap%
\pgfsetroundjoin%
\pgfsetlinewidth{1.505625pt}%
\definecolor{currentstroke}{rgb}{0.678431,1.000000,0.184314}%
\pgfsetstrokecolor{currentstroke}%
\pgfsetstrokeopacity{0.500000}%
\pgfsetdash{}{0pt}%
\pgfpathmoveto{\pgfqpoint{3.765181in}{1.665775in}}%
\pgfusepath{stroke}%
\end{pgfscope}%
\begin{pgfscope}%
\pgfpathrectangle{\pgfqpoint{0.100000in}{0.100000in}}{\pgfqpoint{5.307240in}{3.397500in}}%
\pgfusepath{clip}%
\pgfsetbuttcap%
\pgfsetroundjoin%
\definecolor{currentfill}{rgb}{0.678431,1.000000,0.184314}%
\pgfsetfillcolor{currentfill}%
\pgfsetfillopacity{0.500000}%
\pgfsetlinewidth{0.250937pt}%
\definecolor{currentstroke}{rgb}{0.000000,0.000000,0.000000}%
\pgfsetstrokecolor{currentstroke}%
\pgfsetstrokeopacity{0.500000}%
\pgfsetdash{}{0pt}%
\pgfsys@defobject{currentmarker}{\pgfqpoint{-0.085417in}{-0.085417in}}{\pgfqpoint{0.085417in}{0.085417in}}{%
\pgfpathmoveto{\pgfqpoint{0.000000in}{-0.085417in}}%
\pgfpathcurveto{\pgfqpoint{0.022653in}{-0.085417in}}{\pgfqpoint{0.044381in}{-0.076417in}}{\pgfqpoint{0.060399in}{-0.060399in}}%
\pgfpathcurveto{\pgfqpoint{0.076417in}{-0.044381in}}{\pgfqpoint{0.085417in}{-0.022653in}}{\pgfqpoint{0.085417in}{0.000000in}}%
\pgfpathcurveto{\pgfqpoint{0.085417in}{0.022653in}}{\pgfqpoint{0.076417in}{0.044381in}}{\pgfqpoint{0.060399in}{0.060399in}}%
\pgfpathcurveto{\pgfqpoint{0.044381in}{0.076417in}}{\pgfqpoint{0.022653in}{0.085417in}}{\pgfqpoint{0.000000in}{0.085417in}}%
\pgfpathcurveto{\pgfqpoint{-0.022653in}{0.085417in}}{\pgfqpoint{-0.044381in}{0.076417in}}{\pgfqpoint{-0.060399in}{0.060399in}}%
\pgfpathcurveto{\pgfqpoint{-0.076417in}{0.044381in}}{\pgfqpoint{-0.085417in}{0.022653in}}{\pgfqpoint{-0.085417in}{0.000000in}}%
\pgfpathcurveto{\pgfqpoint{-0.085417in}{-0.022653in}}{\pgfqpoint{-0.076417in}{-0.044381in}}{\pgfqpoint{-0.060399in}{-0.060399in}}%
\pgfpathcurveto{\pgfqpoint{-0.044381in}{-0.076417in}}{\pgfqpoint{-0.022653in}{-0.085417in}}{\pgfqpoint{0.000000in}{-0.085417in}}%
\pgfpathclose%
\pgfusepath{stroke,fill}%
}%
\begin{pgfscope}%
\pgfsys@transformshift{3.765181in}{1.665775in}%
\pgfsys@useobject{currentmarker}{}%
\end{pgfscope}%
\end{pgfscope}%
\begin{pgfscope}%
\pgfpathrectangle{\pgfqpoint{0.100000in}{0.100000in}}{\pgfqpoint{5.307240in}{3.397500in}}%
\pgfusepath{clip}%
\pgfsetrectcap%
\pgfsetroundjoin%
\pgfsetlinewidth{1.505625pt}%
\definecolor{currentstroke}{rgb}{0.678431,1.000000,0.184314}%
\pgfsetstrokecolor{currentstroke}%
\pgfsetstrokeopacity{0.500000}%
\pgfsetdash{}{0pt}%
\pgfpathmoveto{\pgfqpoint{4.013884in}{1.530099in}}%
\pgfusepath{stroke}%
\end{pgfscope}%
\begin{pgfscope}%
\pgfpathrectangle{\pgfqpoint{0.100000in}{0.100000in}}{\pgfqpoint{5.307240in}{3.397500in}}%
\pgfusepath{clip}%
\pgfsetbuttcap%
\pgfsetroundjoin%
\definecolor{currentfill}{rgb}{0.678431,1.000000,0.184314}%
\pgfsetfillcolor{currentfill}%
\pgfsetfillopacity{0.500000}%
\pgfsetlinewidth{0.250937pt}%
\definecolor{currentstroke}{rgb}{0.000000,0.000000,0.000000}%
\pgfsetstrokecolor{currentstroke}%
\pgfsetstrokeopacity{0.500000}%
\pgfsetdash{}{0pt}%
\pgfsys@defobject{currentmarker}{\pgfqpoint{-0.072917in}{-0.072917in}}{\pgfqpoint{0.072917in}{0.072917in}}{%
\pgfpathmoveto{\pgfqpoint{0.000000in}{-0.072917in}}%
\pgfpathcurveto{\pgfqpoint{0.019338in}{-0.072917in}}{\pgfqpoint{0.037886in}{-0.065234in}}{\pgfqpoint{0.051560in}{-0.051560in}}%
\pgfpathcurveto{\pgfqpoint{0.065234in}{-0.037886in}}{\pgfqpoint{0.072917in}{-0.019338in}}{\pgfqpoint{0.072917in}{0.000000in}}%
\pgfpathcurveto{\pgfqpoint{0.072917in}{0.019338in}}{\pgfqpoint{0.065234in}{0.037886in}}{\pgfqpoint{0.051560in}{0.051560in}}%
\pgfpathcurveto{\pgfqpoint{0.037886in}{0.065234in}}{\pgfqpoint{0.019338in}{0.072917in}}{\pgfqpoint{0.000000in}{0.072917in}}%
\pgfpathcurveto{\pgfqpoint{-0.019338in}{0.072917in}}{\pgfqpoint{-0.037886in}{0.065234in}}{\pgfqpoint{-0.051560in}{0.051560in}}%
\pgfpathcurveto{\pgfqpoint{-0.065234in}{0.037886in}}{\pgfqpoint{-0.072917in}{0.019338in}}{\pgfqpoint{-0.072917in}{0.000000in}}%
\pgfpathcurveto{\pgfqpoint{-0.072917in}{-0.019338in}}{\pgfqpoint{-0.065234in}{-0.037886in}}{\pgfqpoint{-0.051560in}{-0.051560in}}%
\pgfpathcurveto{\pgfqpoint{-0.037886in}{-0.065234in}}{\pgfqpoint{-0.019338in}{-0.072917in}}{\pgfqpoint{0.000000in}{-0.072917in}}%
\pgfpathclose%
\pgfusepath{stroke,fill}%
}%
\begin{pgfscope}%
\pgfsys@transformshift{4.013884in}{1.530099in}%
\pgfsys@useobject{currentmarker}{}%
\end{pgfscope}%
\end{pgfscope}%
\begin{pgfscope}%
\pgfpathrectangle{\pgfqpoint{0.100000in}{0.100000in}}{\pgfqpoint{5.307240in}{3.397500in}}%
\pgfusepath{clip}%
\pgfsetrectcap%
\pgfsetroundjoin%
\pgfsetlinewidth{1.505625pt}%
\definecolor{currentstroke}{rgb}{0.678431,1.000000,0.184314}%
\pgfsetstrokecolor{currentstroke}%
\pgfsetstrokeopacity{0.500000}%
\pgfsetdash{}{0pt}%
\pgfpathmoveto{\pgfqpoint{3.636390in}{1.549337in}}%
\pgfusepath{stroke}%
\end{pgfscope}%
\begin{pgfscope}%
\pgfpathrectangle{\pgfqpoint{0.100000in}{0.100000in}}{\pgfqpoint{5.307240in}{3.397500in}}%
\pgfusepath{clip}%
\pgfsetbuttcap%
\pgfsetroundjoin%
\definecolor{currentfill}{rgb}{0.678431,1.000000,0.184314}%
\pgfsetfillcolor{currentfill}%
\pgfsetfillopacity{0.500000}%
\pgfsetlinewidth{0.250937pt}%
\definecolor{currentstroke}{rgb}{0.000000,0.000000,0.000000}%
\pgfsetstrokecolor{currentstroke}%
\pgfsetstrokeopacity{0.500000}%
\pgfsetdash{}{0pt}%
\pgfsys@defobject{currentmarker}{\pgfqpoint{-0.072917in}{-0.072917in}}{\pgfqpoint{0.072917in}{0.072917in}}{%
\pgfpathmoveto{\pgfqpoint{0.000000in}{-0.072917in}}%
\pgfpathcurveto{\pgfqpoint{0.019338in}{-0.072917in}}{\pgfqpoint{0.037886in}{-0.065234in}}{\pgfqpoint{0.051560in}{-0.051560in}}%
\pgfpathcurveto{\pgfqpoint{0.065234in}{-0.037886in}}{\pgfqpoint{0.072917in}{-0.019338in}}{\pgfqpoint{0.072917in}{0.000000in}}%
\pgfpathcurveto{\pgfqpoint{0.072917in}{0.019338in}}{\pgfqpoint{0.065234in}{0.037886in}}{\pgfqpoint{0.051560in}{0.051560in}}%
\pgfpathcurveto{\pgfqpoint{0.037886in}{0.065234in}}{\pgfqpoint{0.019338in}{0.072917in}}{\pgfqpoint{0.000000in}{0.072917in}}%
\pgfpathcurveto{\pgfqpoint{-0.019338in}{0.072917in}}{\pgfqpoint{-0.037886in}{0.065234in}}{\pgfqpoint{-0.051560in}{0.051560in}}%
\pgfpathcurveto{\pgfqpoint{-0.065234in}{0.037886in}}{\pgfqpoint{-0.072917in}{0.019338in}}{\pgfqpoint{-0.072917in}{0.000000in}}%
\pgfpathcurveto{\pgfqpoint{-0.072917in}{-0.019338in}}{\pgfqpoint{-0.065234in}{-0.037886in}}{\pgfqpoint{-0.051560in}{-0.051560in}}%
\pgfpathcurveto{\pgfqpoint{-0.037886in}{-0.065234in}}{\pgfqpoint{-0.019338in}{-0.072917in}}{\pgfqpoint{0.000000in}{-0.072917in}}%
\pgfpathclose%
\pgfusepath{stroke,fill}%
}%
\begin{pgfscope}%
\pgfsys@transformshift{3.636390in}{1.549337in}%
\pgfsys@useobject{currentmarker}{}%
\end{pgfscope}%
\end{pgfscope}%
\begin{pgfscope}%
\pgfpathrectangle{\pgfqpoint{0.100000in}{0.100000in}}{\pgfqpoint{5.307240in}{3.397500in}}%
\pgfusepath{clip}%
\pgfsetrectcap%
\pgfsetroundjoin%
\pgfsetlinewidth{1.505625pt}%
\definecolor{currentstroke}{rgb}{0.678431,1.000000,0.184314}%
\pgfsetstrokecolor{currentstroke}%
\pgfsetstrokeopacity{0.500000}%
\pgfsetdash{}{0pt}%
\pgfpathmoveto{\pgfqpoint{4.233707in}{1.693203in}}%
\pgfusepath{stroke}%
\end{pgfscope}%
\begin{pgfscope}%
\pgfpathrectangle{\pgfqpoint{0.100000in}{0.100000in}}{\pgfqpoint{5.307240in}{3.397500in}}%
\pgfusepath{clip}%
\pgfsetbuttcap%
\pgfsetroundjoin%
\definecolor{currentfill}{rgb}{0.678431,1.000000,0.184314}%
\pgfsetfillcolor{currentfill}%
\pgfsetfillopacity{0.500000}%
\pgfsetlinewidth{0.250937pt}%
\definecolor{currentstroke}{rgb}{0.000000,0.000000,0.000000}%
\pgfsetstrokecolor{currentstroke}%
\pgfsetstrokeopacity{0.500000}%
\pgfsetdash{}{0pt}%
\pgfsys@defobject{currentmarker}{\pgfqpoint{-0.069444in}{-0.069444in}}{\pgfqpoint{0.069444in}{0.069444in}}{%
\pgfpathmoveto{\pgfqpoint{0.000000in}{-0.069444in}}%
\pgfpathcurveto{\pgfqpoint{0.018417in}{-0.069444in}}{\pgfqpoint{0.036082in}{-0.062127in}}{\pgfqpoint{0.049105in}{-0.049105in}}%
\pgfpathcurveto{\pgfqpoint{0.062127in}{-0.036082in}}{\pgfqpoint{0.069444in}{-0.018417in}}{\pgfqpoint{0.069444in}{0.000000in}}%
\pgfpathcurveto{\pgfqpoint{0.069444in}{0.018417in}}{\pgfqpoint{0.062127in}{0.036082in}}{\pgfqpoint{0.049105in}{0.049105in}}%
\pgfpathcurveto{\pgfqpoint{0.036082in}{0.062127in}}{\pgfqpoint{0.018417in}{0.069444in}}{\pgfqpoint{0.000000in}{0.069444in}}%
\pgfpathcurveto{\pgfqpoint{-0.018417in}{0.069444in}}{\pgfqpoint{-0.036082in}{0.062127in}}{\pgfqpoint{-0.049105in}{0.049105in}}%
\pgfpathcurveto{\pgfqpoint{-0.062127in}{0.036082in}}{\pgfqpoint{-0.069444in}{0.018417in}}{\pgfqpoint{-0.069444in}{0.000000in}}%
\pgfpathcurveto{\pgfqpoint{-0.069444in}{-0.018417in}}{\pgfqpoint{-0.062127in}{-0.036082in}}{\pgfqpoint{-0.049105in}{-0.049105in}}%
\pgfpathcurveto{\pgfqpoint{-0.036082in}{-0.062127in}}{\pgfqpoint{-0.018417in}{-0.069444in}}{\pgfqpoint{0.000000in}{-0.069444in}}%
\pgfpathclose%
\pgfusepath{stroke,fill}%
}%
\begin{pgfscope}%
\pgfsys@transformshift{4.233707in}{1.693203in}%
\pgfsys@useobject{currentmarker}{}%
\end{pgfscope}%
\end{pgfscope}%
\begin{pgfscope}%
\pgfpathrectangle{\pgfqpoint{0.100000in}{0.100000in}}{\pgfqpoint{5.307240in}{3.397500in}}%
\pgfusepath{clip}%
\pgfsetrectcap%
\pgfsetroundjoin%
\pgfsetlinewidth{1.505625pt}%
\definecolor{currentstroke}{rgb}{0.678431,1.000000,0.184314}%
\pgfsetstrokecolor{currentstroke}%
\pgfsetstrokeopacity{0.500000}%
\pgfsetdash{}{0pt}%
\pgfpathmoveto{\pgfqpoint{4.244407in}{1.727914in}}%
\pgfusepath{stroke}%
\end{pgfscope}%
\begin{pgfscope}%
\pgfpathrectangle{\pgfqpoint{0.100000in}{0.100000in}}{\pgfqpoint{5.307240in}{3.397500in}}%
\pgfusepath{clip}%
\pgfsetbuttcap%
\pgfsetroundjoin%
\definecolor{currentfill}{rgb}{0.678431,1.000000,0.184314}%
\pgfsetfillcolor{currentfill}%
\pgfsetfillopacity{0.500000}%
\pgfsetlinewidth{0.250937pt}%
\definecolor{currentstroke}{rgb}{0.000000,0.000000,0.000000}%
\pgfsetstrokecolor{currentstroke}%
\pgfsetstrokeopacity{0.500000}%
\pgfsetdash{}{0pt}%
\pgfsys@defobject{currentmarker}{\pgfqpoint{-0.073611in}{-0.073611in}}{\pgfqpoint{0.073611in}{0.073611in}}{%
\pgfpathmoveto{\pgfqpoint{0.000000in}{-0.073611in}}%
\pgfpathcurveto{\pgfqpoint{0.019522in}{-0.073611in}}{\pgfqpoint{0.038247in}{-0.065855in}}{\pgfqpoint{0.052051in}{-0.052051in}}%
\pgfpathcurveto{\pgfqpoint{0.065855in}{-0.038247in}}{\pgfqpoint{0.073611in}{-0.019522in}}{\pgfqpoint{0.073611in}{0.000000in}}%
\pgfpathcurveto{\pgfqpoint{0.073611in}{0.019522in}}{\pgfqpoint{0.065855in}{0.038247in}}{\pgfqpoint{0.052051in}{0.052051in}}%
\pgfpathcurveto{\pgfqpoint{0.038247in}{0.065855in}}{\pgfqpoint{0.019522in}{0.073611in}}{\pgfqpoint{0.000000in}{0.073611in}}%
\pgfpathcurveto{\pgfqpoint{-0.019522in}{0.073611in}}{\pgfqpoint{-0.038247in}{0.065855in}}{\pgfqpoint{-0.052051in}{0.052051in}}%
\pgfpathcurveto{\pgfqpoint{-0.065855in}{0.038247in}}{\pgfqpoint{-0.073611in}{0.019522in}}{\pgfqpoint{-0.073611in}{0.000000in}}%
\pgfpathcurveto{\pgfqpoint{-0.073611in}{-0.019522in}}{\pgfqpoint{-0.065855in}{-0.038247in}}{\pgfqpoint{-0.052051in}{-0.052051in}}%
\pgfpathcurveto{\pgfqpoint{-0.038247in}{-0.065855in}}{\pgfqpoint{-0.019522in}{-0.073611in}}{\pgfqpoint{0.000000in}{-0.073611in}}%
\pgfpathclose%
\pgfusepath{stroke,fill}%
}%
\begin{pgfscope}%
\pgfsys@transformshift{4.244407in}{1.727914in}%
\pgfsys@useobject{currentmarker}{}%
\end{pgfscope}%
\end{pgfscope}%
\begin{pgfscope}%
\pgfpathrectangle{\pgfqpoint{0.100000in}{0.100000in}}{\pgfqpoint{5.307240in}{3.397500in}}%
\pgfusepath{clip}%
\pgfsetrectcap%
\pgfsetroundjoin%
\pgfsetlinewidth{1.505625pt}%
\definecolor{currentstroke}{rgb}{0.678431,1.000000,0.184314}%
\pgfsetstrokecolor{currentstroke}%
\pgfsetstrokeopacity{0.500000}%
\pgfsetdash{}{0pt}%
\pgfpathmoveto{\pgfqpoint{4.210815in}{1.717844in}}%
\pgfusepath{stroke}%
\end{pgfscope}%
\begin{pgfscope}%
\pgfpathrectangle{\pgfqpoint{0.100000in}{0.100000in}}{\pgfqpoint{5.307240in}{3.397500in}}%
\pgfusepath{clip}%
\pgfsetbuttcap%
\pgfsetroundjoin%
\definecolor{currentfill}{rgb}{0.678431,1.000000,0.184314}%
\pgfsetfillcolor{currentfill}%
\pgfsetfillopacity{0.500000}%
\pgfsetlinewidth{0.250937pt}%
\definecolor{currentstroke}{rgb}{0.000000,0.000000,0.000000}%
\pgfsetstrokecolor{currentstroke}%
\pgfsetstrokeopacity{0.500000}%
\pgfsetdash{}{0pt}%
\pgfsys@defobject{currentmarker}{\pgfqpoint{-0.073611in}{-0.073611in}}{\pgfqpoint{0.073611in}{0.073611in}}{%
\pgfpathmoveto{\pgfqpoint{0.000000in}{-0.073611in}}%
\pgfpathcurveto{\pgfqpoint{0.019522in}{-0.073611in}}{\pgfqpoint{0.038247in}{-0.065855in}}{\pgfqpoint{0.052051in}{-0.052051in}}%
\pgfpathcurveto{\pgfqpoint{0.065855in}{-0.038247in}}{\pgfqpoint{0.073611in}{-0.019522in}}{\pgfqpoint{0.073611in}{0.000000in}}%
\pgfpathcurveto{\pgfqpoint{0.073611in}{0.019522in}}{\pgfqpoint{0.065855in}{0.038247in}}{\pgfqpoint{0.052051in}{0.052051in}}%
\pgfpathcurveto{\pgfqpoint{0.038247in}{0.065855in}}{\pgfqpoint{0.019522in}{0.073611in}}{\pgfqpoint{0.000000in}{0.073611in}}%
\pgfpathcurveto{\pgfqpoint{-0.019522in}{0.073611in}}{\pgfqpoint{-0.038247in}{0.065855in}}{\pgfqpoint{-0.052051in}{0.052051in}}%
\pgfpathcurveto{\pgfqpoint{-0.065855in}{0.038247in}}{\pgfqpoint{-0.073611in}{0.019522in}}{\pgfqpoint{-0.073611in}{0.000000in}}%
\pgfpathcurveto{\pgfqpoint{-0.073611in}{-0.019522in}}{\pgfqpoint{-0.065855in}{-0.038247in}}{\pgfqpoint{-0.052051in}{-0.052051in}}%
\pgfpathcurveto{\pgfqpoint{-0.038247in}{-0.065855in}}{\pgfqpoint{-0.019522in}{-0.073611in}}{\pgfqpoint{0.000000in}{-0.073611in}}%
\pgfpathclose%
\pgfusepath{stroke,fill}%
}%
\begin{pgfscope}%
\pgfsys@transformshift{4.210815in}{1.717844in}%
\pgfsys@useobject{currentmarker}{}%
\end{pgfscope}%
\end{pgfscope}%
\begin{pgfscope}%
\pgfpathrectangle{\pgfqpoint{0.100000in}{0.100000in}}{\pgfqpoint{5.307240in}{3.397500in}}%
\pgfusepath{clip}%
\pgfsetrectcap%
\pgfsetroundjoin%
\pgfsetlinewidth{1.505625pt}%
\definecolor{currentstroke}{rgb}{0.678431,1.000000,0.184314}%
\pgfsetstrokecolor{currentstroke}%
\pgfsetstrokeopacity{0.500000}%
\pgfsetdash{}{0pt}%
\pgfpathmoveto{\pgfqpoint{4.092950in}{1.633277in}}%
\pgfusepath{stroke}%
\end{pgfscope}%
\begin{pgfscope}%
\pgfpathrectangle{\pgfqpoint{0.100000in}{0.100000in}}{\pgfqpoint{5.307240in}{3.397500in}}%
\pgfusepath{clip}%
\pgfsetbuttcap%
\pgfsetroundjoin%
\definecolor{currentfill}{rgb}{0.678431,1.000000,0.184314}%
\pgfsetfillcolor{currentfill}%
\pgfsetfillopacity{0.500000}%
\pgfsetlinewidth{0.250937pt}%
\definecolor{currentstroke}{rgb}{0.000000,0.000000,0.000000}%
\pgfsetstrokecolor{currentstroke}%
\pgfsetstrokeopacity{0.500000}%
\pgfsetdash{}{0pt}%
\pgfsys@defobject{currentmarker}{\pgfqpoint{-0.075000in}{-0.075000in}}{\pgfqpoint{0.075000in}{0.075000in}}{%
\pgfpathmoveto{\pgfqpoint{0.000000in}{-0.075000in}}%
\pgfpathcurveto{\pgfqpoint{0.019890in}{-0.075000in}}{\pgfqpoint{0.038968in}{-0.067098in}}{\pgfqpoint{0.053033in}{-0.053033in}}%
\pgfpathcurveto{\pgfqpoint{0.067098in}{-0.038968in}}{\pgfqpoint{0.075000in}{-0.019890in}}{\pgfqpoint{0.075000in}{0.000000in}}%
\pgfpathcurveto{\pgfqpoint{0.075000in}{0.019890in}}{\pgfqpoint{0.067098in}{0.038968in}}{\pgfqpoint{0.053033in}{0.053033in}}%
\pgfpathcurveto{\pgfqpoint{0.038968in}{0.067098in}}{\pgfqpoint{0.019890in}{0.075000in}}{\pgfqpoint{0.000000in}{0.075000in}}%
\pgfpathcurveto{\pgfqpoint{-0.019890in}{0.075000in}}{\pgfqpoint{-0.038968in}{0.067098in}}{\pgfqpoint{-0.053033in}{0.053033in}}%
\pgfpathcurveto{\pgfqpoint{-0.067098in}{0.038968in}}{\pgfqpoint{-0.075000in}{0.019890in}}{\pgfqpoint{-0.075000in}{0.000000in}}%
\pgfpathcurveto{\pgfqpoint{-0.075000in}{-0.019890in}}{\pgfqpoint{-0.067098in}{-0.038968in}}{\pgfqpoint{-0.053033in}{-0.053033in}}%
\pgfpathcurveto{\pgfqpoint{-0.038968in}{-0.067098in}}{\pgfqpoint{-0.019890in}{-0.075000in}}{\pgfqpoint{0.000000in}{-0.075000in}}%
\pgfpathclose%
\pgfusepath{stroke,fill}%
}%
\begin{pgfscope}%
\pgfsys@transformshift{4.092950in}{1.633277in}%
\pgfsys@useobject{currentmarker}{}%
\end{pgfscope}%
\end{pgfscope}%
\begin{pgfscope}%
\pgfpathrectangle{\pgfqpoint{0.100000in}{0.100000in}}{\pgfqpoint{5.307240in}{3.397500in}}%
\pgfusepath{clip}%
\pgfsetrectcap%
\pgfsetroundjoin%
\pgfsetlinewidth{1.505625pt}%
\definecolor{currentstroke}{rgb}{0.678431,1.000000,0.184314}%
\pgfsetstrokecolor{currentstroke}%
\pgfsetstrokeopacity{0.500000}%
\pgfsetdash{}{0pt}%
\pgfpathmoveto{\pgfqpoint{3.522741in}{1.488051in}}%
\pgfusepath{stroke}%
\end{pgfscope}%
\begin{pgfscope}%
\pgfpathrectangle{\pgfqpoint{0.100000in}{0.100000in}}{\pgfqpoint{5.307240in}{3.397500in}}%
\pgfusepath{clip}%
\pgfsetbuttcap%
\pgfsetroundjoin%
\definecolor{currentfill}{rgb}{0.678431,1.000000,0.184314}%
\pgfsetfillcolor{currentfill}%
\pgfsetfillopacity{0.500000}%
\pgfsetlinewidth{0.250937pt}%
\definecolor{currentstroke}{rgb}{0.000000,0.000000,0.000000}%
\pgfsetstrokecolor{currentstroke}%
\pgfsetstrokeopacity{0.500000}%
\pgfsetdash{}{0pt}%
\pgfsys@defobject{currentmarker}{\pgfqpoint{-0.062500in}{-0.062500in}}{\pgfqpoint{0.062500in}{0.062500in}}{%
\pgfpathmoveto{\pgfqpoint{0.000000in}{-0.062500in}}%
\pgfpathcurveto{\pgfqpoint{0.016575in}{-0.062500in}}{\pgfqpoint{0.032474in}{-0.055915in}}{\pgfqpoint{0.044194in}{-0.044194in}}%
\pgfpathcurveto{\pgfqpoint{0.055915in}{-0.032474in}}{\pgfqpoint{0.062500in}{-0.016575in}}{\pgfqpoint{0.062500in}{0.000000in}}%
\pgfpathcurveto{\pgfqpoint{0.062500in}{0.016575in}}{\pgfqpoint{0.055915in}{0.032474in}}{\pgfqpoint{0.044194in}{0.044194in}}%
\pgfpathcurveto{\pgfqpoint{0.032474in}{0.055915in}}{\pgfqpoint{0.016575in}{0.062500in}}{\pgfqpoint{0.000000in}{0.062500in}}%
\pgfpathcurveto{\pgfqpoint{-0.016575in}{0.062500in}}{\pgfqpoint{-0.032474in}{0.055915in}}{\pgfqpoint{-0.044194in}{0.044194in}}%
\pgfpathcurveto{\pgfqpoint{-0.055915in}{0.032474in}}{\pgfqpoint{-0.062500in}{0.016575in}}{\pgfqpoint{-0.062500in}{0.000000in}}%
\pgfpathcurveto{\pgfqpoint{-0.062500in}{-0.016575in}}{\pgfqpoint{-0.055915in}{-0.032474in}}{\pgfqpoint{-0.044194in}{-0.044194in}}%
\pgfpathcurveto{\pgfqpoint{-0.032474in}{-0.055915in}}{\pgfqpoint{-0.016575in}{-0.062500in}}{\pgfqpoint{0.000000in}{-0.062500in}}%
\pgfpathclose%
\pgfusepath{stroke,fill}%
}%
\begin{pgfscope}%
\pgfsys@transformshift{3.522741in}{1.488051in}%
\pgfsys@useobject{currentmarker}{}%
\end{pgfscope}%
\end{pgfscope}%
\begin{pgfscope}%
\pgfpathrectangle{\pgfqpoint{0.100000in}{0.100000in}}{\pgfqpoint{5.307240in}{3.397500in}}%
\pgfusepath{clip}%
\pgfsetrectcap%
\pgfsetroundjoin%
\pgfsetlinewidth{1.505625pt}%
\definecolor{currentstroke}{rgb}{0.678431,1.000000,0.184314}%
\pgfsetstrokecolor{currentstroke}%
\pgfsetstrokeopacity{0.500000}%
\pgfsetdash{}{0pt}%
\pgfpathmoveto{\pgfqpoint{4.147698in}{1.669929in}}%
\pgfusepath{stroke}%
\end{pgfscope}%
\begin{pgfscope}%
\pgfpathrectangle{\pgfqpoint{0.100000in}{0.100000in}}{\pgfqpoint{5.307240in}{3.397500in}}%
\pgfusepath{clip}%
\pgfsetbuttcap%
\pgfsetroundjoin%
\definecolor{currentfill}{rgb}{0.678431,1.000000,0.184314}%
\pgfsetfillcolor{currentfill}%
\pgfsetfillopacity{0.500000}%
\pgfsetlinewidth{0.250937pt}%
\definecolor{currentstroke}{rgb}{0.000000,0.000000,0.000000}%
\pgfsetstrokecolor{currentstroke}%
\pgfsetstrokeopacity{0.500000}%
\pgfsetdash{}{0pt}%
\pgfsys@defobject{currentmarker}{\pgfqpoint{-0.090278in}{-0.090278in}}{\pgfqpoint{0.090278in}{0.090278in}}{%
\pgfpathmoveto{\pgfqpoint{0.000000in}{-0.090278in}}%
\pgfpathcurveto{\pgfqpoint{0.023942in}{-0.090278in}}{\pgfqpoint{0.046907in}{-0.080766in}}{\pgfqpoint{0.063836in}{-0.063836in}}%
\pgfpathcurveto{\pgfqpoint{0.080766in}{-0.046907in}}{\pgfqpoint{0.090278in}{-0.023942in}}{\pgfqpoint{0.090278in}{0.000000in}}%
\pgfpathcurveto{\pgfqpoint{0.090278in}{0.023942in}}{\pgfqpoint{0.080766in}{0.046907in}}{\pgfqpoint{0.063836in}{0.063836in}}%
\pgfpathcurveto{\pgfqpoint{0.046907in}{0.080766in}}{\pgfqpoint{0.023942in}{0.090278in}}{\pgfqpoint{0.000000in}{0.090278in}}%
\pgfpathcurveto{\pgfqpoint{-0.023942in}{0.090278in}}{\pgfqpoint{-0.046907in}{0.080766in}}{\pgfqpoint{-0.063836in}{0.063836in}}%
\pgfpathcurveto{\pgfqpoint{-0.080766in}{0.046907in}}{\pgfqpoint{-0.090278in}{0.023942in}}{\pgfqpoint{-0.090278in}{0.000000in}}%
\pgfpathcurveto{\pgfqpoint{-0.090278in}{-0.023942in}}{\pgfqpoint{-0.080766in}{-0.046907in}}{\pgfqpoint{-0.063836in}{-0.063836in}}%
\pgfpathcurveto{\pgfqpoint{-0.046907in}{-0.080766in}}{\pgfqpoint{-0.023942in}{-0.090278in}}{\pgfqpoint{0.000000in}{-0.090278in}}%
\pgfpathclose%
\pgfusepath{stroke,fill}%
}%
\begin{pgfscope}%
\pgfsys@transformshift{4.147698in}{1.669929in}%
\pgfsys@useobject{currentmarker}{}%
\end{pgfscope}%
\end{pgfscope}%
\begin{pgfscope}%
\pgfpathrectangle{\pgfqpoint{0.100000in}{0.100000in}}{\pgfqpoint{5.307240in}{3.397500in}}%
\pgfusepath{clip}%
\pgfsetrectcap%
\pgfsetroundjoin%
\pgfsetlinewidth{1.505625pt}%
\definecolor{currentstroke}{rgb}{0.678431,1.000000,0.184314}%
\pgfsetstrokecolor{currentstroke}%
\pgfsetstrokeopacity{0.500000}%
\pgfsetdash{}{0pt}%
\pgfpathmoveto{\pgfqpoint{3.823475in}{1.628173in}}%
\pgfusepath{stroke}%
\end{pgfscope}%
\begin{pgfscope}%
\pgfpathrectangle{\pgfqpoint{0.100000in}{0.100000in}}{\pgfqpoint{5.307240in}{3.397500in}}%
\pgfusepath{clip}%
\pgfsetbuttcap%
\pgfsetroundjoin%
\definecolor{currentfill}{rgb}{0.678431,1.000000,0.184314}%
\pgfsetfillcolor{currentfill}%
\pgfsetfillopacity{0.500000}%
\pgfsetlinewidth{0.250937pt}%
\definecolor{currentstroke}{rgb}{0.000000,0.000000,0.000000}%
\pgfsetstrokecolor{currentstroke}%
\pgfsetstrokeopacity{0.500000}%
\pgfsetdash{}{0pt}%
\pgfsys@defobject{currentmarker}{\pgfqpoint{-0.089583in}{-0.089583in}}{\pgfqpoint{0.089583in}{0.089583in}}{%
\pgfpathmoveto{\pgfqpoint{0.000000in}{-0.089583in}}%
\pgfpathcurveto{\pgfqpoint{0.023758in}{-0.089583in}}{\pgfqpoint{0.046546in}{-0.080144in}}{\pgfqpoint{0.063345in}{-0.063345in}}%
\pgfpathcurveto{\pgfqpoint{0.080144in}{-0.046546in}}{\pgfqpoint{0.089583in}{-0.023758in}}{\pgfqpoint{0.089583in}{0.000000in}}%
\pgfpathcurveto{\pgfqpoint{0.089583in}{0.023758in}}{\pgfqpoint{0.080144in}{0.046546in}}{\pgfqpoint{0.063345in}{0.063345in}}%
\pgfpathcurveto{\pgfqpoint{0.046546in}{0.080144in}}{\pgfqpoint{0.023758in}{0.089583in}}{\pgfqpoint{0.000000in}{0.089583in}}%
\pgfpathcurveto{\pgfqpoint{-0.023758in}{0.089583in}}{\pgfqpoint{-0.046546in}{0.080144in}}{\pgfqpoint{-0.063345in}{0.063345in}}%
\pgfpathcurveto{\pgfqpoint{-0.080144in}{0.046546in}}{\pgfqpoint{-0.089583in}{0.023758in}}{\pgfqpoint{-0.089583in}{0.000000in}}%
\pgfpathcurveto{\pgfqpoint{-0.089583in}{-0.023758in}}{\pgfqpoint{-0.080144in}{-0.046546in}}{\pgfqpoint{-0.063345in}{-0.063345in}}%
\pgfpathcurveto{\pgfqpoint{-0.046546in}{-0.080144in}}{\pgfqpoint{-0.023758in}{-0.089583in}}{\pgfqpoint{0.000000in}{-0.089583in}}%
\pgfpathclose%
\pgfusepath{stroke,fill}%
}%
\begin{pgfscope}%
\pgfsys@transformshift{3.823475in}{1.628173in}%
\pgfsys@useobject{currentmarker}{}%
\end{pgfscope}%
\end{pgfscope}%
\begin{pgfscope}%
\pgfpathrectangle{\pgfqpoint{0.100000in}{0.100000in}}{\pgfqpoint{5.307240in}{3.397500in}}%
\pgfusepath{clip}%
\pgfsetrectcap%
\pgfsetroundjoin%
\pgfsetlinewidth{1.505625pt}%
\definecolor{currentstroke}{rgb}{0.678431,1.000000,0.184314}%
\pgfsetstrokecolor{currentstroke}%
\pgfsetstrokeopacity{0.500000}%
\pgfsetdash{}{0pt}%
\pgfpathmoveto{\pgfqpoint{2.584344in}{1.171877in}}%
\pgfusepath{stroke}%
\end{pgfscope}%
\begin{pgfscope}%
\pgfpathrectangle{\pgfqpoint{0.100000in}{0.100000in}}{\pgfqpoint{5.307240in}{3.397500in}}%
\pgfusepath{clip}%
\pgfsetbuttcap%
\pgfsetroundjoin%
\definecolor{currentfill}{rgb}{0.678431,1.000000,0.184314}%
\pgfsetfillcolor{currentfill}%
\pgfsetfillopacity{0.500000}%
\pgfsetlinewidth{0.250937pt}%
\definecolor{currentstroke}{rgb}{0.000000,0.000000,0.000000}%
\pgfsetstrokecolor{currentstroke}%
\pgfsetstrokeopacity{0.500000}%
\pgfsetdash{}{0pt}%
\pgfsys@defobject{currentmarker}{\pgfqpoint{-0.044444in}{-0.044444in}}{\pgfqpoint{0.044444in}{0.044444in}}{%
\pgfpathmoveto{\pgfqpoint{0.000000in}{-0.044444in}}%
\pgfpathcurveto{\pgfqpoint{0.011787in}{-0.044444in}}{\pgfqpoint{0.023092in}{-0.039761in}}{\pgfqpoint{0.031427in}{-0.031427in}}%
\pgfpathcurveto{\pgfqpoint{0.039761in}{-0.023092in}}{\pgfqpoint{0.044444in}{-0.011787in}}{\pgfqpoint{0.044444in}{0.000000in}}%
\pgfpathcurveto{\pgfqpoint{0.044444in}{0.011787in}}{\pgfqpoint{0.039761in}{0.023092in}}{\pgfqpoint{0.031427in}{0.031427in}}%
\pgfpathcurveto{\pgfqpoint{0.023092in}{0.039761in}}{\pgfqpoint{0.011787in}{0.044444in}}{\pgfqpoint{0.000000in}{0.044444in}}%
\pgfpathcurveto{\pgfqpoint{-0.011787in}{0.044444in}}{\pgfqpoint{-0.023092in}{0.039761in}}{\pgfqpoint{-0.031427in}{0.031427in}}%
\pgfpathcurveto{\pgfqpoint{-0.039761in}{0.023092in}}{\pgfqpoint{-0.044444in}{0.011787in}}{\pgfqpoint{-0.044444in}{0.000000in}}%
\pgfpathcurveto{\pgfqpoint{-0.044444in}{-0.011787in}}{\pgfqpoint{-0.039761in}{-0.023092in}}{\pgfqpoint{-0.031427in}{-0.031427in}}%
\pgfpathcurveto{\pgfqpoint{-0.023092in}{-0.039761in}}{\pgfqpoint{-0.011787in}{-0.044444in}}{\pgfqpoint{0.000000in}{-0.044444in}}%
\pgfpathclose%
\pgfusepath{stroke,fill}%
}%
\begin{pgfscope}%
\pgfsys@transformshift{2.584344in}{1.171877in}%
\pgfsys@useobject{currentmarker}{}%
\end{pgfscope}%
\end{pgfscope}%
\begin{pgfscope}%
\pgfpathrectangle{\pgfqpoint{0.100000in}{0.100000in}}{\pgfqpoint{5.307240in}{3.397500in}}%
\pgfusepath{clip}%
\pgfsetrectcap%
\pgfsetroundjoin%
\pgfsetlinewidth{1.505625pt}%
\definecolor{currentstroke}{rgb}{0.678431,1.000000,0.184314}%
\pgfsetstrokecolor{currentstroke}%
\pgfsetstrokeopacity{0.500000}%
\pgfsetdash{}{0pt}%
\pgfpathmoveto{\pgfqpoint{2.401571in}{1.506449in}}%
\pgfusepath{stroke}%
\end{pgfscope}%
\begin{pgfscope}%
\pgfpathrectangle{\pgfqpoint{0.100000in}{0.100000in}}{\pgfqpoint{5.307240in}{3.397500in}}%
\pgfusepath{clip}%
\pgfsetbuttcap%
\pgfsetroundjoin%
\definecolor{currentfill}{rgb}{0.678431,1.000000,0.184314}%
\pgfsetfillcolor{currentfill}%
\pgfsetfillopacity{0.500000}%
\pgfsetlinewidth{0.250937pt}%
\definecolor{currentstroke}{rgb}{0.000000,0.000000,0.000000}%
\pgfsetstrokecolor{currentstroke}%
\pgfsetstrokeopacity{0.500000}%
\pgfsetdash{}{0pt}%
\pgfsys@defobject{currentmarker}{\pgfqpoint{-0.046528in}{-0.046528in}}{\pgfqpoint{0.046528in}{0.046528in}}{%
\pgfpathmoveto{\pgfqpoint{0.000000in}{-0.046528in}}%
\pgfpathcurveto{\pgfqpoint{0.012339in}{-0.046528in}}{\pgfqpoint{0.024175in}{-0.041625in}}{\pgfqpoint{0.032900in}{-0.032900in}}%
\pgfpathcurveto{\pgfqpoint{0.041625in}{-0.024175in}}{\pgfqpoint{0.046528in}{-0.012339in}}{\pgfqpoint{0.046528in}{0.000000in}}%
\pgfpathcurveto{\pgfqpoint{0.046528in}{0.012339in}}{\pgfqpoint{0.041625in}{0.024175in}}{\pgfqpoint{0.032900in}{0.032900in}}%
\pgfpathcurveto{\pgfqpoint{0.024175in}{0.041625in}}{\pgfqpoint{0.012339in}{0.046528in}}{\pgfqpoint{0.000000in}{0.046528in}}%
\pgfpathcurveto{\pgfqpoint{-0.012339in}{0.046528in}}{\pgfqpoint{-0.024175in}{0.041625in}}{\pgfqpoint{-0.032900in}{0.032900in}}%
\pgfpathcurveto{\pgfqpoint{-0.041625in}{0.024175in}}{\pgfqpoint{-0.046528in}{0.012339in}}{\pgfqpoint{-0.046528in}{0.000000in}}%
\pgfpathcurveto{\pgfqpoint{-0.046528in}{-0.012339in}}{\pgfqpoint{-0.041625in}{-0.024175in}}{\pgfqpoint{-0.032900in}{-0.032900in}}%
\pgfpathcurveto{\pgfqpoint{-0.024175in}{-0.041625in}}{\pgfqpoint{-0.012339in}{-0.046528in}}{\pgfqpoint{0.000000in}{-0.046528in}}%
\pgfpathclose%
\pgfusepath{stroke,fill}%
}%
\begin{pgfscope}%
\pgfsys@transformshift{2.401571in}{1.506449in}%
\pgfsys@useobject{currentmarker}{}%
\end{pgfscope}%
\end{pgfscope}%
\begin{pgfscope}%
\pgfpathrectangle{\pgfqpoint{0.100000in}{0.100000in}}{\pgfqpoint{5.307240in}{3.397500in}}%
\pgfusepath{clip}%
\pgfsetrectcap%
\pgfsetroundjoin%
\pgfsetlinewidth{1.505625pt}%
\definecolor{currentstroke}{rgb}{0.678431,1.000000,0.184314}%
\pgfsetstrokecolor{currentstroke}%
\pgfsetstrokeopacity{0.500000}%
\pgfsetdash{}{0pt}%
\pgfpathmoveto{\pgfqpoint{2.772940in}{0.908781in}}%
\pgfusepath{stroke}%
\end{pgfscope}%
\begin{pgfscope}%
\pgfpathrectangle{\pgfqpoint{0.100000in}{0.100000in}}{\pgfqpoint{5.307240in}{3.397500in}}%
\pgfusepath{clip}%
\pgfsetbuttcap%
\pgfsetroundjoin%
\definecolor{currentfill}{rgb}{0.678431,1.000000,0.184314}%
\pgfsetfillcolor{currentfill}%
\pgfsetfillopacity{0.500000}%
\pgfsetlinewidth{0.250937pt}%
\definecolor{currentstroke}{rgb}{0.000000,0.000000,0.000000}%
\pgfsetstrokecolor{currentstroke}%
\pgfsetstrokeopacity{0.500000}%
\pgfsetdash{}{0pt}%
\pgfsys@defobject{currentmarker}{\pgfqpoint{-0.068750in}{-0.068750in}}{\pgfqpoint{0.068750in}{0.068750in}}{%
\pgfpathmoveto{\pgfqpoint{0.000000in}{-0.068750in}}%
\pgfpathcurveto{\pgfqpoint{0.018233in}{-0.068750in}}{\pgfqpoint{0.035721in}{-0.061506in}}{\pgfqpoint{0.048614in}{-0.048614in}}%
\pgfpathcurveto{\pgfqpoint{0.061506in}{-0.035721in}}{\pgfqpoint{0.068750in}{-0.018233in}}{\pgfqpoint{0.068750in}{0.000000in}}%
\pgfpathcurveto{\pgfqpoint{0.068750in}{0.018233in}}{\pgfqpoint{0.061506in}{0.035721in}}{\pgfqpoint{0.048614in}{0.048614in}}%
\pgfpathcurveto{\pgfqpoint{0.035721in}{0.061506in}}{\pgfqpoint{0.018233in}{0.068750in}}{\pgfqpoint{0.000000in}{0.068750in}}%
\pgfpathcurveto{\pgfqpoint{-0.018233in}{0.068750in}}{\pgfqpoint{-0.035721in}{0.061506in}}{\pgfqpoint{-0.048614in}{0.048614in}}%
\pgfpathcurveto{\pgfqpoint{-0.061506in}{0.035721in}}{\pgfqpoint{-0.068750in}{0.018233in}}{\pgfqpoint{-0.068750in}{0.000000in}}%
\pgfpathcurveto{\pgfqpoint{-0.068750in}{-0.018233in}}{\pgfqpoint{-0.061506in}{-0.035721in}}{\pgfqpoint{-0.048614in}{-0.048614in}}%
\pgfpathcurveto{\pgfqpoint{-0.035721in}{-0.061506in}}{\pgfqpoint{-0.018233in}{-0.068750in}}{\pgfqpoint{0.000000in}{-0.068750in}}%
\pgfpathclose%
\pgfusepath{stroke,fill}%
}%
\begin{pgfscope}%
\pgfsys@transformshift{2.772940in}{0.908781in}%
\pgfsys@useobject{currentmarker}{}%
\end{pgfscope}%
\end{pgfscope}%
\begin{pgfscope}%
\pgfpathrectangle{\pgfqpoint{0.100000in}{0.100000in}}{\pgfqpoint{5.307240in}{3.397500in}}%
\pgfusepath{clip}%
\pgfsetrectcap%
\pgfsetroundjoin%
\pgfsetlinewidth{1.505625pt}%
\definecolor{currentstroke}{rgb}{0.678431,1.000000,0.184314}%
\pgfsetstrokecolor{currentstroke}%
\pgfsetstrokeopacity{0.500000}%
\pgfsetdash{}{0pt}%
\pgfpathmoveto{\pgfqpoint{3.142951in}{0.883258in}}%
\pgfusepath{stroke}%
\end{pgfscope}%
\begin{pgfscope}%
\pgfpathrectangle{\pgfqpoint{0.100000in}{0.100000in}}{\pgfqpoint{5.307240in}{3.397500in}}%
\pgfusepath{clip}%
\pgfsetbuttcap%
\pgfsetroundjoin%
\definecolor{currentfill}{rgb}{0.678431,1.000000,0.184314}%
\pgfsetfillcolor{currentfill}%
\pgfsetfillopacity{0.500000}%
\pgfsetlinewidth{0.250937pt}%
\definecolor{currentstroke}{rgb}{0.000000,0.000000,0.000000}%
\pgfsetstrokecolor{currentstroke}%
\pgfsetstrokeopacity{0.500000}%
\pgfsetdash{}{0pt}%
\pgfsys@defobject{currentmarker}{\pgfqpoint{-0.093750in}{-0.093750in}}{\pgfqpoint{0.093750in}{0.093750in}}{%
\pgfpathmoveto{\pgfqpoint{0.000000in}{-0.093750in}}%
\pgfpathcurveto{\pgfqpoint{0.024863in}{-0.093750in}}{\pgfqpoint{0.048711in}{-0.083872in}}{\pgfqpoint{0.066291in}{-0.066291in}}%
\pgfpathcurveto{\pgfqpoint{0.083872in}{-0.048711in}}{\pgfqpoint{0.093750in}{-0.024863in}}{\pgfqpoint{0.093750in}{0.000000in}}%
\pgfpathcurveto{\pgfqpoint{0.093750in}{0.024863in}}{\pgfqpoint{0.083872in}{0.048711in}}{\pgfqpoint{0.066291in}{0.066291in}}%
\pgfpathcurveto{\pgfqpoint{0.048711in}{0.083872in}}{\pgfqpoint{0.024863in}{0.093750in}}{\pgfqpoint{0.000000in}{0.093750in}}%
\pgfpathcurveto{\pgfqpoint{-0.024863in}{0.093750in}}{\pgfqpoint{-0.048711in}{0.083872in}}{\pgfqpoint{-0.066291in}{0.066291in}}%
\pgfpathcurveto{\pgfqpoint{-0.083872in}{0.048711in}}{\pgfqpoint{-0.093750in}{0.024863in}}{\pgfqpoint{-0.093750in}{0.000000in}}%
\pgfpathcurveto{\pgfqpoint{-0.093750in}{-0.024863in}}{\pgfqpoint{-0.083872in}{-0.048711in}}{\pgfqpoint{-0.066291in}{-0.066291in}}%
\pgfpathcurveto{\pgfqpoint{-0.048711in}{-0.083872in}}{\pgfqpoint{-0.024863in}{-0.093750in}}{\pgfqpoint{0.000000in}{-0.093750in}}%
\pgfpathclose%
\pgfusepath{stroke,fill}%
}%
\begin{pgfscope}%
\pgfsys@transformshift{3.142951in}{0.883258in}%
\pgfsys@useobject{currentmarker}{}%
\end{pgfscope}%
\end{pgfscope}%
\begin{pgfscope}%
\pgfpathrectangle{\pgfqpoint{0.100000in}{0.100000in}}{\pgfqpoint{5.307240in}{3.397500in}}%
\pgfusepath{clip}%
\pgfsetrectcap%
\pgfsetroundjoin%
\pgfsetlinewidth{1.505625pt}%
\definecolor{currentstroke}{rgb}{0.678431,1.000000,0.184314}%
\pgfsetstrokecolor{currentstroke}%
\pgfsetstrokeopacity{0.500000}%
\pgfsetdash{}{0pt}%
\pgfpathmoveto{\pgfqpoint{2.784665in}{0.392585in}}%
\pgfusepath{stroke}%
\end{pgfscope}%
\begin{pgfscope}%
\pgfpathrectangle{\pgfqpoint{0.100000in}{0.100000in}}{\pgfqpoint{5.307240in}{3.397500in}}%
\pgfusepath{clip}%
\pgfsetbuttcap%
\pgfsetroundjoin%
\definecolor{currentfill}{rgb}{0.678431,1.000000,0.184314}%
\pgfsetfillcolor{currentfill}%
\pgfsetfillopacity{0.500000}%
\pgfsetlinewidth{0.250937pt}%
\definecolor{currentstroke}{rgb}{0.000000,0.000000,0.000000}%
\pgfsetstrokecolor{currentstroke}%
\pgfsetstrokeopacity{0.500000}%
\pgfsetdash{}{0pt}%
\pgfsys@defobject{currentmarker}{\pgfqpoint{-0.086111in}{-0.086111in}}{\pgfqpoint{0.086111in}{0.086111in}}{%
\pgfpathmoveto{\pgfqpoint{0.000000in}{-0.086111in}}%
\pgfpathcurveto{\pgfqpoint{0.022837in}{-0.086111in}}{\pgfqpoint{0.044742in}{-0.077038in}}{\pgfqpoint{0.060890in}{-0.060890in}}%
\pgfpathcurveto{\pgfqpoint{0.077038in}{-0.044742in}}{\pgfqpoint{0.086111in}{-0.022837in}}{\pgfqpoint{0.086111in}{0.000000in}}%
\pgfpathcurveto{\pgfqpoint{0.086111in}{0.022837in}}{\pgfqpoint{0.077038in}{0.044742in}}{\pgfqpoint{0.060890in}{0.060890in}}%
\pgfpathcurveto{\pgfqpoint{0.044742in}{0.077038in}}{\pgfqpoint{0.022837in}{0.086111in}}{\pgfqpoint{0.000000in}{0.086111in}}%
\pgfpathcurveto{\pgfqpoint{-0.022837in}{0.086111in}}{\pgfqpoint{-0.044742in}{0.077038in}}{\pgfqpoint{-0.060890in}{0.060890in}}%
\pgfpathcurveto{\pgfqpoint{-0.077038in}{0.044742in}}{\pgfqpoint{-0.086111in}{0.022837in}}{\pgfqpoint{-0.086111in}{0.000000in}}%
\pgfpathcurveto{\pgfqpoint{-0.086111in}{-0.022837in}}{\pgfqpoint{-0.077038in}{-0.044742in}}{\pgfqpoint{-0.060890in}{-0.060890in}}%
\pgfpathcurveto{\pgfqpoint{-0.044742in}{-0.077038in}}{\pgfqpoint{-0.022837in}{-0.086111in}}{\pgfqpoint{0.000000in}{-0.086111in}}%
\pgfpathclose%
\pgfusepath{stroke,fill}%
}%
\begin{pgfscope}%
\pgfsys@transformshift{2.784665in}{0.392585in}%
\pgfsys@useobject{currentmarker}{}%
\end{pgfscope}%
\end{pgfscope}%
\begin{pgfscope}%
\pgfpathrectangle{\pgfqpoint{0.100000in}{0.100000in}}{\pgfqpoint{5.307240in}{3.397500in}}%
\pgfusepath{clip}%
\pgfsetrectcap%
\pgfsetroundjoin%
\pgfsetlinewidth{1.505625pt}%
\definecolor{currentstroke}{rgb}{0.678431,1.000000,0.184314}%
\pgfsetstrokecolor{currentstroke}%
\pgfsetstrokeopacity{0.500000}%
\pgfsetdash{}{0pt}%
\pgfpathmoveto{\pgfqpoint{2.917364in}{0.947187in}}%
\pgfusepath{stroke}%
\end{pgfscope}%
\begin{pgfscope}%
\pgfpathrectangle{\pgfqpoint{0.100000in}{0.100000in}}{\pgfqpoint{5.307240in}{3.397500in}}%
\pgfusepath{clip}%
\pgfsetbuttcap%
\pgfsetroundjoin%
\definecolor{currentfill}{rgb}{0.678431,1.000000,0.184314}%
\pgfsetfillcolor{currentfill}%
\pgfsetfillopacity{0.500000}%
\pgfsetlinewidth{0.250937pt}%
\definecolor{currentstroke}{rgb}{0.000000,0.000000,0.000000}%
\pgfsetstrokecolor{currentstroke}%
\pgfsetstrokeopacity{0.500000}%
\pgfsetdash{}{0pt}%
\pgfsys@defobject{currentmarker}{\pgfqpoint{-0.046528in}{-0.046528in}}{\pgfqpoint{0.046528in}{0.046528in}}{%
\pgfpathmoveto{\pgfqpoint{0.000000in}{-0.046528in}}%
\pgfpathcurveto{\pgfqpoint{0.012339in}{-0.046528in}}{\pgfqpoint{0.024175in}{-0.041625in}}{\pgfqpoint{0.032900in}{-0.032900in}}%
\pgfpathcurveto{\pgfqpoint{0.041625in}{-0.024175in}}{\pgfqpoint{0.046528in}{-0.012339in}}{\pgfqpoint{0.046528in}{0.000000in}}%
\pgfpathcurveto{\pgfqpoint{0.046528in}{0.012339in}}{\pgfqpoint{0.041625in}{0.024175in}}{\pgfqpoint{0.032900in}{0.032900in}}%
\pgfpathcurveto{\pgfqpoint{0.024175in}{0.041625in}}{\pgfqpoint{0.012339in}{0.046528in}}{\pgfqpoint{0.000000in}{0.046528in}}%
\pgfpathcurveto{\pgfqpoint{-0.012339in}{0.046528in}}{\pgfqpoint{-0.024175in}{0.041625in}}{\pgfqpoint{-0.032900in}{0.032900in}}%
\pgfpathcurveto{\pgfqpoint{-0.041625in}{0.024175in}}{\pgfqpoint{-0.046528in}{0.012339in}}{\pgfqpoint{-0.046528in}{0.000000in}}%
\pgfpathcurveto{\pgfqpoint{-0.046528in}{-0.012339in}}{\pgfqpoint{-0.041625in}{-0.024175in}}{\pgfqpoint{-0.032900in}{-0.032900in}}%
\pgfpathcurveto{\pgfqpoint{-0.024175in}{-0.041625in}}{\pgfqpoint{-0.012339in}{-0.046528in}}{\pgfqpoint{0.000000in}{-0.046528in}}%
\pgfpathclose%
\pgfusepath{stroke,fill}%
}%
\begin{pgfscope}%
\pgfsys@transformshift{2.917364in}{0.947187in}%
\pgfsys@useobject{currentmarker}{}%
\end{pgfscope}%
\end{pgfscope}%
\begin{pgfscope}%
\pgfpathrectangle{\pgfqpoint{0.100000in}{0.100000in}}{\pgfqpoint{5.307240in}{3.397500in}}%
\pgfusepath{clip}%
\pgfsetrectcap%
\pgfsetroundjoin%
\pgfsetlinewidth{1.505625pt}%
\definecolor{currentstroke}{rgb}{0.678431,1.000000,0.184314}%
\pgfsetstrokecolor{currentstroke}%
\pgfsetstrokeopacity{0.500000}%
\pgfsetdash{}{0pt}%
\pgfpathmoveto{\pgfqpoint{2.800595in}{0.616343in}}%
\pgfusepath{stroke}%
\end{pgfscope}%
\begin{pgfscope}%
\pgfpathrectangle{\pgfqpoint{0.100000in}{0.100000in}}{\pgfqpoint{5.307240in}{3.397500in}}%
\pgfusepath{clip}%
\pgfsetbuttcap%
\pgfsetroundjoin%
\definecolor{currentfill}{rgb}{0.678431,1.000000,0.184314}%
\pgfsetfillcolor{currentfill}%
\pgfsetfillopacity{0.500000}%
\pgfsetlinewidth{0.250937pt}%
\definecolor{currentstroke}{rgb}{0.000000,0.000000,0.000000}%
\pgfsetstrokecolor{currentstroke}%
\pgfsetstrokeopacity{0.500000}%
\pgfsetdash{}{0pt}%
\pgfsys@defobject{currentmarker}{\pgfqpoint{-0.085417in}{-0.085417in}}{\pgfqpoint{0.085417in}{0.085417in}}{%
\pgfpathmoveto{\pgfqpoint{0.000000in}{-0.085417in}}%
\pgfpathcurveto{\pgfqpoint{0.022653in}{-0.085417in}}{\pgfqpoint{0.044381in}{-0.076417in}}{\pgfqpoint{0.060399in}{-0.060399in}}%
\pgfpathcurveto{\pgfqpoint{0.076417in}{-0.044381in}}{\pgfqpoint{0.085417in}{-0.022653in}}{\pgfqpoint{0.085417in}{0.000000in}}%
\pgfpathcurveto{\pgfqpoint{0.085417in}{0.022653in}}{\pgfqpoint{0.076417in}{0.044381in}}{\pgfqpoint{0.060399in}{0.060399in}}%
\pgfpathcurveto{\pgfqpoint{0.044381in}{0.076417in}}{\pgfqpoint{0.022653in}{0.085417in}}{\pgfqpoint{0.000000in}{0.085417in}}%
\pgfpathcurveto{\pgfqpoint{-0.022653in}{0.085417in}}{\pgfqpoint{-0.044381in}{0.076417in}}{\pgfqpoint{-0.060399in}{0.060399in}}%
\pgfpathcurveto{\pgfqpoint{-0.076417in}{0.044381in}}{\pgfqpoint{-0.085417in}{0.022653in}}{\pgfqpoint{-0.085417in}{0.000000in}}%
\pgfpathcurveto{\pgfqpoint{-0.085417in}{-0.022653in}}{\pgfqpoint{-0.076417in}{-0.044381in}}{\pgfqpoint{-0.060399in}{-0.060399in}}%
\pgfpathcurveto{\pgfqpoint{-0.044381in}{-0.076417in}}{\pgfqpoint{-0.022653in}{-0.085417in}}{\pgfqpoint{0.000000in}{-0.085417in}}%
\pgfpathclose%
\pgfusepath{stroke,fill}%
}%
\begin{pgfscope}%
\pgfsys@transformshift{2.800595in}{0.616343in}%
\pgfsys@useobject{currentmarker}{}%
\end{pgfscope}%
\end{pgfscope}%
\begin{pgfscope}%
\pgfpathrectangle{\pgfqpoint{0.100000in}{0.100000in}}{\pgfqpoint{5.307240in}{3.397500in}}%
\pgfusepath{clip}%
\pgfsetrectcap%
\pgfsetroundjoin%
\pgfsetlinewidth{1.505625pt}%
\definecolor{currentstroke}{rgb}{0.678431,1.000000,0.184314}%
\pgfsetstrokecolor{currentstroke}%
\pgfsetstrokeopacity{0.500000}%
\pgfsetdash{}{0pt}%
\pgfpathmoveto{\pgfqpoint{2.874871in}{1.200035in}}%
\pgfusepath{stroke}%
\end{pgfscope}%
\begin{pgfscope}%
\pgfpathrectangle{\pgfqpoint{0.100000in}{0.100000in}}{\pgfqpoint{5.307240in}{3.397500in}}%
\pgfusepath{clip}%
\pgfsetbuttcap%
\pgfsetroundjoin%
\definecolor{currentfill}{rgb}{0.678431,1.000000,0.184314}%
\pgfsetfillcolor{currentfill}%
\pgfsetfillopacity{0.500000}%
\pgfsetlinewidth{0.250937pt}%
\definecolor{currentstroke}{rgb}{0.000000,0.000000,0.000000}%
\pgfsetstrokecolor{currentstroke}%
\pgfsetstrokeopacity{0.500000}%
\pgfsetdash{}{0pt}%
\pgfsys@defobject{currentmarker}{\pgfqpoint{-0.069444in}{-0.069444in}}{\pgfqpoint{0.069444in}{0.069444in}}{%
\pgfpathmoveto{\pgfqpoint{0.000000in}{-0.069444in}}%
\pgfpathcurveto{\pgfqpoint{0.018417in}{-0.069444in}}{\pgfqpoint{0.036082in}{-0.062127in}}{\pgfqpoint{0.049105in}{-0.049105in}}%
\pgfpathcurveto{\pgfqpoint{0.062127in}{-0.036082in}}{\pgfqpoint{0.069444in}{-0.018417in}}{\pgfqpoint{0.069444in}{0.000000in}}%
\pgfpathcurveto{\pgfqpoint{0.069444in}{0.018417in}}{\pgfqpoint{0.062127in}{0.036082in}}{\pgfqpoint{0.049105in}{0.049105in}}%
\pgfpathcurveto{\pgfqpoint{0.036082in}{0.062127in}}{\pgfqpoint{0.018417in}{0.069444in}}{\pgfqpoint{0.000000in}{0.069444in}}%
\pgfpathcurveto{\pgfqpoint{-0.018417in}{0.069444in}}{\pgfqpoint{-0.036082in}{0.062127in}}{\pgfqpoint{-0.049105in}{0.049105in}}%
\pgfpathcurveto{\pgfqpoint{-0.062127in}{0.036082in}}{\pgfqpoint{-0.069444in}{0.018417in}}{\pgfqpoint{-0.069444in}{0.000000in}}%
\pgfpathcurveto{\pgfqpoint{-0.069444in}{-0.018417in}}{\pgfqpoint{-0.062127in}{-0.036082in}}{\pgfqpoint{-0.049105in}{-0.049105in}}%
\pgfpathcurveto{\pgfqpoint{-0.036082in}{-0.062127in}}{\pgfqpoint{-0.018417in}{-0.069444in}}{\pgfqpoint{0.000000in}{-0.069444in}}%
\pgfpathclose%
\pgfusepath{stroke,fill}%
}%
\begin{pgfscope}%
\pgfsys@transformshift{2.874871in}{1.200035in}%
\pgfsys@useobject{currentmarker}{}%
\end{pgfscope}%
\end{pgfscope}%
\begin{pgfscope}%
\pgfpathrectangle{\pgfqpoint{0.100000in}{0.100000in}}{\pgfqpoint{5.307240in}{3.397500in}}%
\pgfusepath{clip}%
\pgfsetrectcap%
\pgfsetroundjoin%
\pgfsetlinewidth{1.505625pt}%
\definecolor{currentstroke}{rgb}{0.678431,1.000000,0.184314}%
\pgfsetstrokecolor{currentstroke}%
\pgfsetstrokeopacity{0.500000}%
\pgfsetdash{}{0pt}%
\pgfpathmoveto{\pgfqpoint{1.910101in}{1.151571in}}%
\pgfusepath{stroke}%
\end{pgfscope}%
\begin{pgfscope}%
\pgfpathrectangle{\pgfqpoint{0.100000in}{0.100000in}}{\pgfqpoint{5.307240in}{3.397500in}}%
\pgfusepath{clip}%
\pgfsetbuttcap%
\pgfsetroundjoin%
\definecolor{currentfill}{rgb}{0.678431,1.000000,0.184314}%
\pgfsetfillcolor{currentfill}%
\pgfsetfillopacity{0.500000}%
\pgfsetlinewidth{0.250937pt}%
\definecolor{currentstroke}{rgb}{0.000000,0.000000,0.000000}%
\pgfsetstrokecolor{currentstroke}%
\pgfsetstrokeopacity{0.500000}%
\pgfsetdash{}{0pt}%
\pgfsys@defobject{currentmarker}{\pgfqpoint{-0.079861in}{-0.079861in}}{\pgfqpoint{0.079861in}{0.079861in}}{%
\pgfpathmoveto{\pgfqpoint{0.000000in}{-0.079861in}}%
\pgfpathcurveto{\pgfqpoint{0.021179in}{-0.079861in}}{\pgfqpoint{0.041494in}{-0.071446in}}{\pgfqpoint{0.056470in}{-0.056470in}}%
\pgfpathcurveto{\pgfqpoint{0.071446in}{-0.041494in}}{\pgfqpoint{0.079861in}{-0.021179in}}{\pgfqpoint{0.079861in}{0.000000in}}%
\pgfpathcurveto{\pgfqpoint{0.079861in}{0.021179in}}{\pgfqpoint{0.071446in}{0.041494in}}{\pgfqpoint{0.056470in}{0.056470in}}%
\pgfpathcurveto{\pgfqpoint{0.041494in}{0.071446in}}{\pgfqpoint{0.021179in}{0.079861in}}{\pgfqpoint{0.000000in}{0.079861in}}%
\pgfpathcurveto{\pgfqpoint{-0.021179in}{0.079861in}}{\pgfqpoint{-0.041494in}{0.071446in}}{\pgfqpoint{-0.056470in}{0.056470in}}%
\pgfpathcurveto{\pgfqpoint{-0.071446in}{0.041494in}}{\pgfqpoint{-0.079861in}{0.021179in}}{\pgfqpoint{-0.079861in}{0.000000in}}%
\pgfpathcurveto{\pgfqpoint{-0.079861in}{-0.021179in}}{\pgfqpoint{-0.071446in}{-0.041494in}}{\pgfqpoint{-0.056470in}{-0.056470in}}%
\pgfpathcurveto{\pgfqpoint{-0.041494in}{-0.071446in}}{\pgfqpoint{-0.021179in}{-0.079861in}}{\pgfqpoint{0.000000in}{-0.079861in}}%
\pgfpathclose%
\pgfusepath{stroke,fill}%
}%
\begin{pgfscope}%
\pgfsys@transformshift{1.910101in}{1.151571in}%
\pgfsys@useobject{currentmarker}{}%
\end{pgfscope}%
\end{pgfscope}%
\begin{pgfscope}%
\pgfpathrectangle{\pgfqpoint{0.100000in}{0.100000in}}{\pgfqpoint{5.307240in}{3.397500in}}%
\pgfusepath{clip}%
\pgfsetrectcap%
\pgfsetroundjoin%
\pgfsetlinewidth{1.505625pt}%
\definecolor{currentstroke}{rgb}{0.678431,1.000000,0.184314}%
\pgfsetstrokecolor{currentstroke}%
\pgfsetstrokeopacity{0.500000}%
\pgfsetdash{}{0pt}%
\pgfpathmoveto{\pgfqpoint{3.013943in}{0.844385in}}%
\pgfusepath{stroke}%
\end{pgfscope}%
\begin{pgfscope}%
\pgfpathrectangle{\pgfqpoint{0.100000in}{0.100000in}}{\pgfqpoint{5.307240in}{3.397500in}}%
\pgfusepath{clip}%
\pgfsetbuttcap%
\pgfsetroundjoin%
\definecolor{currentfill}{rgb}{0.678431,1.000000,0.184314}%
\pgfsetfillcolor{currentfill}%
\pgfsetfillopacity{0.500000}%
\pgfsetlinewidth{0.250937pt}%
\definecolor{currentstroke}{rgb}{0.000000,0.000000,0.000000}%
\pgfsetstrokecolor{currentstroke}%
\pgfsetstrokeopacity{0.500000}%
\pgfsetdash{}{0pt}%
\pgfsys@defobject{currentmarker}{\pgfqpoint{-0.075694in}{-0.075694in}}{\pgfqpoint{0.075694in}{0.075694in}}{%
\pgfpathmoveto{\pgfqpoint{0.000000in}{-0.075694in}}%
\pgfpathcurveto{\pgfqpoint{0.020074in}{-0.075694in}}{\pgfqpoint{0.039329in}{-0.067719in}}{\pgfqpoint{0.053524in}{-0.053524in}}%
\pgfpathcurveto{\pgfqpoint{0.067719in}{-0.039329in}}{\pgfqpoint{0.075694in}{-0.020074in}}{\pgfqpoint{0.075694in}{0.000000in}}%
\pgfpathcurveto{\pgfqpoint{0.075694in}{0.020074in}}{\pgfqpoint{0.067719in}{0.039329in}}{\pgfqpoint{0.053524in}{0.053524in}}%
\pgfpathcurveto{\pgfqpoint{0.039329in}{0.067719in}}{\pgfqpoint{0.020074in}{0.075694in}}{\pgfqpoint{0.000000in}{0.075694in}}%
\pgfpathcurveto{\pgfqpoint{-0.020074in}{0.075694in}}{\pgfqpoint{-0.039329in}{0.067719in}}{\pgfqpoint{-0.053524in}{0.053524in}}%
\pgfpathcurveto{\pgfqpoint{-0.067719in}{0.039329in}}{\pgfqpoint{-0.075694in}{0.020074in}}{\pgfqpoint{-0.075694in}{0.000000in}}%
\pgfpathcurveto{\pgfqpoint{-0.075694in}{-0.020074in}}{\pgfqpoint{-0.067719in}{-0.039329in}}{\pgfqpoint{-0.053524in}{-0.053524in}}%
\pgfpathcurveto{\pgfqpoint{-0.039329in}{-0.067719in}}{\pgfqpoint{-0.020074in}{-0.075694in}}{\pgfqpoint{0.000000in}{-0.075694in}}%
\pgfpathclose%
\pgfusepath{stroke,fill}%
}%
\begin{pgfscope}%
\pgfsys@transformshift{3.013943in}{0.844385in}%
\pgfsys@useobject{currentmarker}{}%
\end{pgfscope}%
\end{pgfscope}%
\begin{pgfscope}%
\pgfpathrectangle{\pgfqpoint{0.100000in}{0.100000in}}{\pgfqpoint{5.307240in}{3.397500in}}%
\pgfusepath{clip}%
\pgfsetrectcap%
\pgfsetroundjoin%
\pgfsetlinewidth{1.505625pt}%
\definecolor{currentstroke}{rgb}{0.678431,1.000000,0.184314}%
\pgfsetstrokecolor{currentstroke}%
\pgfsetstrokeopacity{0.500000}%
\pgfsetdash{}{0pt}%
\pgfpathmoveto{\pgfqpoint{2.777537in}{1.008062in}}%
\pgfusepath{stroke}%
\end{pgfscope}%
\begin{pgfscope}%
\pgfpathrectangle{\pgfqpoint{0.100000in}{0.100000in}}{\pgfqpoint{5.307240in}{3.397500in}}%
\pgfusepath{clip}%
\pgfsetbuttcap%
\pgfsetroundjoin%
\definecolor{currentfill}{rgb}{0.678431,1.000000,0.184314}%
\pgfsetfillcolor{currentfill}%
\pgfsetfillopacity{0.500000}%
\pgfsetlinewidth{0.250937pt}%
\definecolor{currentstroke}{rgb}{0.000000,0.000000,0.000000}%
\pgfsetstrokecolor{currentstroke}%
\pgfsetstrokeopacity{0.500000}%
\pgfsetdash{}{0pt}%
\pgfsys@defobject{currentmarker}{\pgfqpoint{-0.056250in}{-0.056250in}}{\pgfqpoint{0.056250in}{0.056250in}}{%
\pgfpathmoveto{\pgfqpoint{0.000000in}{-0.056250in}}%
\pgfpathcurveto{\pgfqpoint{0.014918in}{-0.056250in}}{\pgfqpoint{0.029226in}{-0.050323in}}{\pgfqpoint{0.039775in}{-0.039775in}}%
\pgfpathcurveto{\pgfqpoint{0.050323in}{-0.029226in}}{\pgfqpoint{0.056250in}{-0.014918in}}{\pgfqpoint{0.056250in}{0.000000in}}%
\pgfpathcurveto{\pgfqpoint{0.056250in}{0.014918in}}{\pgfqpoint{0.050323in}{0.029226in}}{\pgfqpoint{0.039775in}{0.039775in}}%
\pgfpathcurveto{\pgfqpoint{0.029226in}{0.050323in}}{\pgfqpoint{0.014918in}{0.056250in}}{\pgfqpoint{0.000000in}{0.056250in}}%
\pgfpathcurveto{\pgfqpoint{-0.014918in}{0.056250in}}{\pgfqpoint{-0.029226in}{0.050323in}}{\pgfqpoint{-0.039775in}{0.039775in}}%
\pgfpathcurveto{\pgfqpoint{-0.050323in}{0.029226in}}{\pgfqpoint{-0.056250in}{0.014918in}}{\pgfqpoint{-0.056250in}{0.000000in}}%
\pgfpathcurveto{\pgfqpoint{-0.056250in}{-0.014918in}}{\pgfqpoint{-0.050323in}{-0.029226in}}{\pgfqpoint{-0.039775in}{-0.039775in}}%
\pgfpathcurveto{\pgfqpoint{-0.029226in}{-0.050323in}}{\pgfqpoint{-0.014918in}{-0.056250in}}{\pgfqpoint{0.000000in}{-0.056250in}}%
\pgfpathclose%
\pgfusepath{stroke,fill}%
}%
\begin{pgfscope}%
\pgfsys@transformshift{2.777537in}{1.008062in}%
\pgfsys@useobject{currentmarker}{}%
\end{pgfscope}%
\end{pgfscope}%
\begin{pgfscope}%
\pgfpathrectangle{\pgfqpoint{0.100000in}{0.100000in}}{\pgfqpoint{5.307240in}{3.397500in}}%
\pgfusepath{clip}%
\pgfsetrectcap%
\pgfsetroundjoin%
\pgfsetlinewidth{1.505625pt}%
\definecolor{currentstroke}{rgb}{0.678431,1.000000,0.184314}%
\pgfsetstrokecolor{currentstroke}%
\pgfsetstrokeopacity{0.500000}%
\pgfsetdash{}{0pt}%
\pgfpathmoveto{\pgfqpoint{2.577638in}{0.589941in}}%
\pgfusepath{stroke}%
\end{pgfscope}%
\begin{pgfscope}%
\pgfpathrectangle{\pgfqpoint{0.100000in}{0.100000in}}{\pgfqpoint{5.307240in}{3.397500in}}%
\pgfusepath{clip}%
\pgfsetbuttcap%
\pgfsetroundjoin%
\definecolor{currentfill}{rgb}{0.678431,1.000000,0.184314}%
\pgfsetfillcolor{currentfill}%
\pgfsetfillopacity{0.500000}%
\pgfsetlinewidth{0.250937pt}%
\definecolor{currentstroke}{rgb}{0.000000,0.000000,0.000000}%
\pgfsetstrokecolor{currentstroke}%
\pgfsetstrokeopacity{0.500000}%
\pgfsetdash{}{0pt}%
\pgfsys@defobject{currentmarker}{\pgfqpoint{-0.072222in}{-0.072222in}}{\pgfqpoint{0.072222in}{0.072222in}}{%
\pgfpathmoveto{\pgfqpoint{0.000000in}{-0.072222in}}%
\pgfpathcurveto{\pgfqpoint{0.019154in}{-0.072222in}}{\pgfqpoint{0.037525in}{-0.064612in}}{\pgfqpoint{0.051069in}{-0.051069in}}%
\pgfpathcurveto{\pgfqpoint{0.064612in}{-0.037525in}}{\pgfqpoint{0.072222in}{-0.019154in}}{\pgfqpoint{0.072222in}{0.000000in}}%
\pgfpathcurveto{\pgfqpoint{0.072222in}{0.019154in}}{\pgfqpoint{0.064612in}{0.037525in}}{\pgfqpoint{0.051069in}{0.051069in}}%
\pgfpathcurveto{\pgfqpoint{0.037525in}{0.064612in}}{\pgfqpoint{0.019154in}{0.072222in}}{\pgfqpoint{0.000000in}{0.072222in}}%
\pgfpathcurveto{\pgfqpoint{-0.019154in}{0.072222in}}{\pgfqpoint{-0.037525in}{0.064612in}}{\pgfqpoint{-0.051069in}{0.051069in}}%
\pgfpathcurveto{\pgfqpoint{-0.064612in}{0.037525in}}{\pgfqpoint{-0.072222in}{0.019154in}}{\pgfqpoint{-0.072222in}{0.000000in}}%
\pgfpathcurveto{\pgfqpoint{-0.072222in}{-0.019154in}}{\pgfqpoint{-0.064612in}{-0.037525in}}{\pgfqpoint{-0.051069in}{-0.051069in}}%
\pgfpathcurveto{\pgfqpoint{-0.037525in}{-0.064612in}}{\pgfqpoint{-0.019154in}{-0.072222in}}{\pgfqpoint{0.000000in}{-0.072222in}}%
\pgfpathclose%
\pgfusepath{stroke,fill}%
}%
\begin{pgfscope}%
\pgfsys@transformshift{2.577638in}{0.589941in}%
\pgfsys@useobject{currentmarker}{}%
\end{pgfscope}%
\end{pgfscope}%
\begin{pgfscope}%
\pgfpathrectangle{\pgfqpoint{0.100000in}{0.100000in}}{\pgfqpoint{5.307240in}{3.397500in}}%
\pgfusepath{clip}%
\pgfsetrectcap%
\pgfsetroundjoin%
\pgfsetlinewidth{1.505625pt}%
\definecolor{currentstroke}{rgb}{0.678431,1.000000,0.184314}%
\pgfsetstrokecolor{currentstroke}%
\pgfsetstrokeopacity{0.500000}%
\pgfsetdash{}{0pt}%
\pgfpathmoveto{\pgfqpoint{3.077134in}{1.166092in}}%
\pgfusepath{stroke}%
\end{pgfscope}%
\begin{pgfscope}%
\pgfpathrectangle{\pgfqpoint{0.100000in}{0.100000in}}{\pgfqpoint{5.307240in}{3.397500in}}%
\pgfusepath{clip}%
\pgfsetbuttcap%
\pgfsetroundjoin%
\definecolor{currentfill}{rgb}{0.678431,1.000000,0.184314}%
\pgfsetfillcolor{currentfill}%
\pgfsetfillopacity{0.500000}%
\pgfsetlinewidth{0.250937pt}%
\definecolor{currentstroke}{rgb}{0.000000,0.000000,0.000000}%
\pgfsetstrokecolor{currentstroke}%
\pgfsetstrokeopacity{0.500000}%
\pgfsetdash{}{0pt}%
\pgfsys@defobject{currentmarker}{\pgfqpoint{-0.063889in}{-0.063889in}}{\pgfqpoint{0.063889in}{0.063889in}}{%
\pgfpathmoveto{\pgfqpoint{0.000000in}{-0.063889in}}%
\pgfpathcurveto{\pgfqpoint{0.016944in}{-0.063889in}}{\pgfqpoint{0.033195in}{-0.057157in}}{\pgfqpoint{0.045176in}{-0.045176in}}%
\pgfpathcurveto{\pgfqpoint{0.057157in}{-0.033195in}}{\pgfqpoint{0.063889in}{-0.016944in}}{\pgfqpoint{0.063889in}{0.000000in}}%
\pgfpathcurveto{\pgfqpoint{0.063889in}{0.016944in}}{\pgfqpoint{0.057157in}{0.033195in}}{\pgfqpoint{0.045176in}{0.045176in}}%
\pgfpathcurveto{\pgfqpoint{0.033195in}{0.057157in}}{\pgfqpoint{0.016944in}{0.063889in}}{\pgfqpoint{0.000000in}{0.063889in}}%
\pgfpathcurveto{\pgfqpoint{-0.016944in}{0.063889in}}{\pgfqpoint{-0.033195in}{0.057157in}}{\pgfqpoint{-0.045176in}{0.045176in}}%
\pgfpathcurveto{\pgfqpoint{-0.057157in}{0.033195in}}{\pgfqpoint{-0.063889in}{0.016944in}}{\pgfqpoint{-0.063889in}{0.000000in}}%
\pgfpathcurveto{\pgfqpoint{-0.063889in}{-0.016944in}}{\pgfqpoint{-0.057157in}{-0.033195in}}{\pgfqpoint{-0.045176in}{-0.045176in}}%
\pgfpathcurveto{\pgfqpoint{-0.033195in}{-0.057157in}}{\pgfqpoint{-0.016944in}{-0.063889in}}{\pgfqpoint{0.000000in}{-0.063889in}}%
\pgfpathclose%
\pgfusepath{stroke,fill}%
}%
\begin{pgfscope}%
\pgfsys@transformshift{3.077134in}{1.166092in}%
\pgfsys@useobject{currentmarker}{}%
\end{pgfscope}%
\end{pgfscope}%
\begin{pgfscope}%
\pgfpathrectangle{\pgfqpoint{0.100000in}{0.100000in}}{\pgfqpoint{5.307240in}{3.397500in}}%
\pgfusepath{clip}%
\pgfsetrectcap%
\pgfsetroundjoin%
\pgfsetlinewidth{1.505625pt}%
\definecolor{currentstroke}{rgb}{0.678431,1.000000,0.184314}%
\pgfsetstrokecolor{currentstroke}%
\pgfsetstrokeopacity{0.500000}%
\pgfsetdash{}{0pt}%
\pgfpathmoveto{\pgfqpoint{2.385223in}{1.317101in}}%
\pgfusepath{stroke}%
\end{pgfscope}%
\begin{pgfscope}%
\pgfpathrectangle{\pgfqpoint{0.100000in}{0.100000in}}{\pgfqpoint{5.307240in}{3.397500in}}%
\pgfusepath{clip}%
\pgfsetbuttcap%
\pgfsetroundjoin%
\definecolor{currentfill}{rgb}{0.678431,1.000000,0.184314}%
\pgfsetfillcolor{currentfill}%
\pgfsetfillopacity{0.500000}%
\pgfsetlinewidth{0.250937pt}%
\definecolor{currentstroke}{rgb}{0.000000,0.000000,0.000000}%
\pgfsetstrokecolor{currentstroke}%
\pgfsetstrokeopacity{0.500000}%
\pgfsetdash{}{0pt}%
\pgfsys@defobject{currentmarker}{\pgfqpoint{-0.052778in}{-0.052778in}}{\pgfqpoint{0.052778in}{0.052778in}}{%
\pgfpathmoveto{\pgfqpoint{0.000000in}{-0.052778in}}%
\pgfpathcurveto{\pgfqpoint{0.013997in}{-0.052778in}}{\pgfqpoint{0.027422in}{-0.047217in}}{\pgfqpoint{0.037320in}{-0.037320in}}%
\pgfpathcurveto{\pgfqpoint{0.047217in}{-0.027422in}}{\pgfqpoint{0.052778in}{-0.013997in}}{\pgfqpoint{0.052778in}{0.000000in}}%
\pgfpathcurveto{\pgfqpoint{0.052778in}{0.013997in}}{\pgfqpoint{0.047217in}{0.027422in}}{\pgfqpoint{0.037320in}{0.037320in}}%
\pgfpathcurveto{\pgfqpoint{0.027422in}{0.047217in}}{\pgfqpoint{0.013997in}{0.052778in}}{\pgfqpoint{0.000000in}{0.052778in}}%
\pgfpathcurveto{\pgfqpoint{-0.013997in}{0.052778in}}{\pgfqpoint{-0.027422in}{0.047217in}}{\pgfqpoint{-0.037320in}{0.037320in}}%
\pgfpathcurveto{\pgfqpoint{-0.047217in}{0.027422in}}{\pgfqpoint{-0.052778in}{0.013997in}}{\pgfqpoint{-0.052778in}{0.000000in}}%
\pgfpathcurveto{\pgfqpoint{-0.052778in}{-0.013997in}}{\pgfqpoint{-0.047217in}{-0.027422in}}{\pgfqpoint{-0.037320in}{-0.037320in}}%
\pgfpathcurveto{\pgfqpoint{-0.027422in}{-0.047217in}}{\pgfqpoint{-0.013997in}{-0.052778in}}{\pgfqpoint{0.000000in}{-0.052778in}}%
\pgfpathclose%
\pgfusepath{stroke,fill}%
}%
\begin{pgfscope}%
\pgfsys@transformshift{2.385223in}{1.317101in}%
\pgfsys@useobject{currentmarker}{}%
\end{pgfscope}%
\end{pgfscope}%
\begin{pgfscope}%
\pgfpathrectangle{\pgfqpoint{0.100000in}{0.100000in}}{\pgfqpoint{5.307240in}{3.397500in}}%
\pgfusepath{clip}%
\pgfsetrectcap%
\pgfsetroundjoin%
\pgfsetlinewidth{1.505625pt}%
\definecolor{currentstroke}{rgb}{0.678431,1.000000,0.184314}%
\pgfsetstrokecolor{currentstroke}%
\pgfsetstrokeopacity{0.500000}%
\pgfsetdash{}{0pt}%
\pgfpathmoveto{\pgfqpoint{2.706479in}{0.429657in}}%
\pgfusepath{stroke}%
\end{pgfscope}%
\begin{pgfscope}%
\pgfpathrectangle{\pgfqpoint{0.100000in}{0.100000in}}{\pgfqpoint{5.307240in}{3.397500in}}%
\pgfusepath{clip}%
\pgfsetbuttcap%
\pgfsetroundjoin%
\definecolor{currentfill}{rgb}{0.678431,1.000000,0.184314}%
\pgfsetfillcolor{currentfill}%
\pgfsetfillopacity{0.500000}%
\pgfsetlinewidth{0.250937pt}%
\definecolor{currentstroke}{rgb}{0.000000,0.000000,0.000000}%
\pgfsetstrokecolor{currentstroke}%
\pgfsetstrokeopacity{0.500000}%
\pgfsetdash{}{0pt}%
\pgfsys@defobject{currentmarker}{\pgfqpoint{-0.091667in}{-0.091667in}}{\pgfqpoint{0.091667in}{0.091667in}}{%
\pgfpathmoveto{\pgfqpoint{0.000000in}{-0.091667in}}%
\pgfpathcurveto{\pgfqpoint{0.024310in}{-0.091667in}}{\pgfqpoint{0.047628in}{-0.082008in}}{\pgfqpoint{0.064818in}{-0.064818in}}%
\pgfpathcurveto{\pgfqpoint{0.082008in}{-0.047628in}}{\pgfqpoint{0.091667in}{-0.024310in}}{\pgfqpoint{0.091667in}{0.000000in}}%
\pgfpathcurveto{\pgfqpoint{0.091667in}{0.024310in}}{\pgfqpoint{0.082008in}{0.047628in}}{\pgfqpoint{0.064818in}{0.064818in}}%
\pgfpathcurveto{\pgfqpoint{0.047628in}{0.082008in}}{\pgfqpoint{0.024310in}{0.091667in}}{\pgfqpoint{0.000000in}{0.091667in}}%
\pgfpathcurveto{\pgfqpoint{-0.024310in}{0.091667in}}{\pgfqpoint{-0.047628in}{0.082008in}}{\pgfqpoint{-0.064818in}{0.064818in}}%
\pgfpathcurveto{\pgfqpoint{-0.082008in}{0.047628in}}{\pgfqpoint{-0.091667in}{0.024310in}}{\pgfqpoint{-0.091667in}{0.000000in}}%
\pgfpathcurveto{\pgfqpoint{-0.091667in}{-0.024310in}}{\pgfqpoint{-0.082008in}{-0.047628in}}{\pgfqpoint{-0.064818in}{-0.064818in}}%
\pgfpathcurveto{\pgfqpoint{-0.047628in}{-0.082008in}}{\pgfqpoint{-0.024310in}{-0.091667in}}{\pgfqpoint{0.000000in}{-0.091667in}}%
\pgfpathclose%
\pgfusepath{stroke,fill}%
}%
\begin{pgfscope}%
\pgfsys@transformshift{2.706479in}{0.429657in}%
\pgfsys@useobject{currentmarker}{}%
\end{pgfscope}%
\end{pgfscope}%
\begin{pgfscope}%
\pgfpathrectangle{\pgfqpoint{0.100000in}{0.100000in}}{\pgfqpoint{5.307240in}{3.397500in}}%
\pgfusepath{clip}%
\pgfsetrectcap%
\pgfsetroundjoin%
\pgfsetlinewidth{1.505625pt}%
\definecolor{currentstroke}{rgb}{0.678431,1.000000,0.184314}%
\pgfsetstrokecolor{currentstroke}%
\pgfsetstrokeopacity{0.500000}%
\pgfsetdash{}{0pt}%
\pgfpathmoveto{\pgfqpoint{2.349232in}{1.134499in}}%
\pgfusepath{stroke}%
\end{pgfscope}%
\begin{pgfscope}%
\pgfpathrectangle{\pgfqpoint{0.100000in}{0.100000in}}{\pgfqpoint{5.307240in}{3.397500in}}%
\pgfusepath{clip}%
\pgfsetbuttcap%
\pgfsetroundjoin%
\definecolor{currentfill}{rgb}{0.678431,1.000000,0.184314}%
\pgfsetfillcolor{currentfill}%
\pgfsetfillopacity{0.500000}%
\pgfsetlinewidth{0.250937pt}%
\definecolor{currentstroke}{rgb}{0.000000,0.000000,0.000000}%
\pgfsetstrokecolor{currentstroke}%
\pgfsetstrokeopacity{0.500000}%
\pgfsetdash{}{0pt}%
\pgfsys@defobject{currentmarker}{\pgfqpoint{-0.059028in}{-0.059028in}}{\pgfqpoint{0.059028in}{0.059028in}}{%
\pgfpathmoveto{\pgfqpoint{0.000000in}{-0.059028in}}%
\pgfpathcurveto{\pgfqpoint{0.015654in}{-0.059028in}}{\pgfqpoint{0.030670in}{-0.052808in}}{\pgfqpoint{0.041739in}{-0.041739in}}%
\pgfpathcurveto{\pgfqpoint{0.052808in}{-0.030670in}}{\pgfqpoint{0.059028in}{-0.015654in}}{\pgfqpoint{0.059028in}{0.000000in}}%
\pgfpathcurveto{\pgfqpoint{0.059028in}{0.015654in}}{\pgfqpoint{0.052808in}{0.030670in}}{\pgfqpoint{0.041739in}{0.041739in}}%
\pgfpathcurveto{\pgfqpoint{0.030670in}{0.052808in}}{\pgfqpoint{0.015654in}{0.059028in}}{\pgfqpoint{0.000000in}{0.059028in}}%
\pgfpathcurveto{\pgfqpoint{-0.015654in}{0.059028in}}{\pgfqpoint{-0.030670in}{0.052808in}}{\pgfqpoint{-0.041739in}{0.041739in}}%
\pgfpathcurveto{\pgfqpoint{-0.052808in}{0.030670in}}{\pgfqpoint{-0.059028in}{0.015654in}}{\pgfqpoint{-0.059028in}{0.000000in}}%
\pgfpathcurveto{\pgfqpoint{-0.059028in}{-0.015654in}}{\pgfqpoint{-0.052808in}{-0.030670in}}{\pgfqpoint{-0.041739in}{-0.041739in}}%
\pgfpathcurveto{\pgfqpoint{-0.030670in}{-0.052808in}}{\pgfqpoint{-0.015654in}{-0.059028in}}{\pgfqpoint{0.000000in}{-0.059028in}}%
\pgfpathclose%
\pgfusepath{stroke,fill}%
}%
\begin{pgfscope}%
\pgfsys@transformshift{2.349232in}{1.134499in}%
\pgfsys@useobject{currentmarker}{}%
\end{pgfscope}%
\end{pgfscope}%
\begin{pgfscope}%
\pgfpathrectangle{\pgfqpoint{0.100000in}{0.100000in}}{\pgfqpoint{5.307240in}{3.397500in}}%
\pgfusepath{clip}%
\pgfsetrectcap%
\pgfsetroundjoin%
\pgfsetlinewidth{1.505625pt}%
\definecolor{currentstroke}{rgb}{0.678431,1.000000,0.184314}%
\pgfsetstrokecolor{currentstroke}%
\pgfsetstrokeopacity{0.500000}%
\pgfsetdash{}{0pt}%
\pgfpathmoveto{\pgfqpoint{2.319106in}{1.119079in}}%
\pgfusepath{stroke}%
\end{pgfscope}%
\begin{pgfscope}%
\pgfpathrectangle{\pgfqpoint{0.100000in}{0.100000in}}{\pgfqpoint{5.307240in}{3.397500in}}%
\pgfusepath{clip}%
\pgfsetbuttcap%
\pgfsetroundjoin%
\definecolor{currentfill}{rgb}{0.678431,1.000000,0.184314}%
\pgfsetfillcolor{currentfill}%
\pgfsetfillopacity{0.500000}%
\pgfsetlinewidth{0.250937pt}%
\definecolor{currentstroke}{rgb}{0.000000,0.000000,0.000000}%
\pgfsetstrokecolor{currentstroke}%
\pgfsetstrokeopacity{0.500000}%
\pgfsetdash{}{0pt}%
\pgfsys@defobject{currentmarker}{\pgfqpoint{-0.079861in}{-0.079861in}}{\pgfqpoint{0.079861in}{0.079861in}}{%
\pgfpathmoveto{\pgfqpoint{0.000000in}{-0.079861in}}%
\pgfpathcurveto{\pgfqpoint{0.021179in}{-0.079861in}}{\pgfqpoint{0.041494in}{-0.071446in}}{\pgfqpoint{0.056470in}{-0.056470in}}%
\pgfpathcurveto{\pgfqpoint{0.071446in}{-0.041494in}}{\pgfqpoint{0.079861in}{-0.021179in}}{\pgfqpoint{0.079861in}{0.000000in}}%
\pgfpathcurveto{\pgfqpoint{0.079861in}{0.021179in}}{\pgfqpoint{0.071446in}{0.041494in}}{\pgfqpoint{0.056470in}{0.056470in}}%
\pgfpathcurveto{\pgfqpoint{0.041494in}{0.071446in}}{\pgfqpoint{0.021179in}{0.079861in}}{\pgfqpoint{0.000000in}{0.079861in}}%
\pgfpathcurveto{\pgfqpoint{-0.021179in}{0.079861in}}{\pgfqpoint{-0.041494in}{0.071446in}}{\pgfqpoint{-0.056470in}{0.056470in}}%
\pgfpathcurveto{\pgfqpoint{-0.071446in}{0.041494in}}{\pgfqpoint{-0.079861in}{0.021179in}}{\pgfqpoint{-0.079861in}{0.000000in}}%
\pgfpathcurveto{\pgfqpoint{-0.079861in}{-0.021179in}}{\pgfqpoint{-0.071446in}{-0.041494in}}{\pgfqpoint{-0.056470in}{-0.056470in}}%
\pgfpathcurveto{\pgfqpoint{-0.041494in}{-0.071446in}}{\pgfqpoint{-0.021179in}{-0.079861in}}{\pgfqpoint{0.000000in}{-0.079861in}}%
\pgfpathclose%
\pgfusepath{stroke,fill}%
}%
\begin{pgfscope}%
\pgfsys@transformshift{2.319106in}{1.119079in}%
\pgfsys@useobject{currentmarker}{}%
\end{pgfscope}%
\end{pgfscope}%
\begin{pgfscope}%
\pgfpathrectangle{\pgfqpoint{0.100000in}{0.100000in}}{\pgfqpoint{5.307240in}{3.397500in}}%
\pgfusepath{clip}%
\pgfsetrectcap%
\pgfsetroundjoin%
\pgfsetlinewidth{1.505625pt}%
\definecolor{currentstroke}{rgb}{0.678431,1.000000,0.184314}%
\pgfsetstrokecolor{currentstroke}%
\pgfsetstrokeopacity{0.500000}%
\pgfsetdash{}{0pt}%
\pgfpathmoveto{\pgfqpoint{2.507813in}{1.061011in}}%
\pgfusepath{stroke}%
\end{pgfscope}%
\begin{pgfscope}%
\pgfpathrectangle{\pgfqpoint{0.100000in}{0.100000in}}{\pgfqpoint{5.307240in}{3.397500in}}%
\pgfusepath{clip}%
\pgfsetbuttcap%
\pgfsetroundjoin%
\definecolor{currentfill}{rgb}{0.678431,1.000000,0.184314}%
\pgfsetfillcolor{currentfill}%
\pgfsetfillopacity{0.500000}%
\pgfsetlinewidth{0.250937pt}%
\definecolor{currentstroke}{rgb}{0.000000,0.000000,0.000000}%
\pgfsetstrokecolor{currentstroke}%
\pgfsetstrokeopacity{0.500000}%
\pgfsetdash{}{0pt}%
\pgfsys@defobject{currentmarker}{\pgfqpoint{-0.056250in}{-0.056250in}}{\pgfqpoint{0.056250in}{0.056250in}}{%
\pgfpathmoveto{\pgfqpoint{0.000000in}{-0.056250in}}%
\pgfpathcurveto{\pgfqpoint{0.014918in}{-0.056250in}}{\pgfqpoint{0.029226in}{-0.050323in}}{\pgfqpoint{0.039775in}{-0.039775in}}%
\pgfpathcurveto{\pgfqpoint{0.050323in}{-0.029226in}}{\pgfqpoint{0.056250in}{-0.014918in}}{\pgfqpoint{0.056250in}{0.000000in}}%
\pgfpathcurveto{\pgfqpoint{0.056250in}{0.014918in}}{\pgfqpoint{0.050323in}{0.029226in}}{\pgfqpoint{0.039775in}{0.039775in}}%
\pgfpathcurveto{\pgfqpoint{0.029226in}{0.050323in}}{\pgfqpoint{0.014918in}{0.056250in}}{\pgfqpoint{0.000000in}{0.056250in}}%
\pgfpathcurveto{\pgfqpoint{-0.014918in}{0.056250in}}{\pgfqpoint{-0.029226in}{0.050323in}}{\pgfqpoint{-0.039775in}{0.039775in}}%
\pgfpathcurveto{\pgfqpoint{-0.050323in}{0.029226in}}{\pgfqpoint{-0.056250in}{0.014918in}}{\pgfqpoint{-0.056250in}{0.000000in}}%
\pgfpathcurveto{\pgfqpoint{-0.056250in}{-0.014918in}}{\pgfqpoint{-0.050323in}{-0.029226in}}{\pgfqpoint{-0.039775in}{-0.039775in}}%
\pgfpathcurveto{\pgfqpoint{-0.029226in}{-0.050323in}}{\pgfqpoint{-0.014918in}{-0.056250in}}{\pgfqpoint{0.000000in}{-0.056250in}}%
\pgfpathclose%
\pgfusepath{stroke,fill}%
}%
\begin{pgfscope}%
\pgfsys@transformshift{2.507813in}{1.061011in}%
\pgfsys@useobject{currentmarker}{}%
\end{pgfscope}%
\end{pgfscope}%
\begin{pgfscope}%
\pgfpathrectangle{\pgfqpoint{0.100000in}{0.100000in}}{\pgfqpoint{5.307240in}{3.397500in}}%
\pgfusepath{clip}%
\pgfsetrectcap%
\pgfsetroundjoin%
\pgfsetlinewidth{1.505625pt}%
\definecolor{currentstroke}{rgb}{0.678431,1.000000,0.184314}%
\pgfsetstrokecolor{currentstroke}%
\pgfsetstrokeopacity{0.500000}%
\pgfsetdash{}{0pt}%
\pgfpathmoveto{\pgfqpoint{2.692843in}{0.811823in}}%
\pgfusepath{stroke}%
\end{pgfscope}%
\begin{pgfscope}%
\pgfpathrectangle{\pgfqpoint{0.100000in}{0.100000in}}{\pgfqpoint{5.307240in}{3.397500in}}%
\pgfusepath{clip}%
\pgfsetbuttcap%
\pgfsetroundjoin%
\definecolor{currentfill}{rgb}{0.678431,1.000000,0.184314}%
\pgfsetfillcolor{currentfill}%
\pgfsetfillopacity{0.500000}%
\pgfsetlinewidth{0.250937pt}%
\definecolor{currentstroke}{rgb}{0.000000,0.000000,0.000000}%
\pgfsetstrokecolor{currentstroke}%
\pgfsetstrokeopacity{0.500000}%
\pgfsetdash{}{0pt}%
\pgfsys@defobject{currentmarker}{\pgfqpoint{-0.073611in}{-0.073611in}}{\pgfqpoint{0.073611in}{0.073611in}}{%
\pgfpathmoveto{\pgfqpoint{0.000000in}{-0.073611in}}%
\pgfpathcurveto{\pgfqpoint{0.019522in}{-0.073611in}}{\pgfqpoint{0.038247in}{-0.065855in}}{\pgfqpoint{0.052051in}{-0.052051in}}%
\pgfpathcurveto{\pgfqpoint{0.065855in}{-0.038247in}}{\pgfqpoint{0.073611in}{-0.019522in}}{\pgfqpoint{0.073611in}{0.000000in}}%
\pgfpathcurveto{\pgfqpoint{0.073611in}{0.019522in}}{\pgfqpoint{0.065855in}{0.038247in}}{\pgfqpoint{0.052051in}{0.052051in}}%
\pgfpathcurveto{\pgfqpoint{0.038247in}{0.065855in}}{\pgfqpoint{0.019522in}{0.073611in}}{\pgfqpoint{0.000000in}{0.073611in}}%
\pgfpathcurveto{\pgfqpoint{-0.019522in}{0.073611in}}{\pgfqpoint{-0.038247in}{0.065855in}}{\pgfqpoint{-0.052051in}{0.052051in}}%
\pgfpathcurveto{\pgfqpoint{-0.065855in}{0.038247in}}{\pgfqpoint{-0.073611in}{0.019522in}}{\pgfqpoint{-0.073611in}{0.000000in}}%
\pgfpathcurveto{\pgfqpoint{-0.073611in}{-0.019522in}}{\pgfqpoint{-0.065855in}{-0.038247in}}{\pgfqpoint{-0.052051in}{-0.052051in}}%
\pgfpathcurveto{\pgfqpoint{-0.038247in}{-0.065855in}}{\pgfqpoint{-0.019522in}{-0.073611in}}{\pgfqpoint{0.000000in}{-0.073611in}}%
\pgfpathclose%
\pgfusepath{stroke,fill}%
}%
\begin{pgfscope}%
\pgfsys@transformshift{2.692843in}{0.811823in}%
\pgfsys@useobject{currentmarker}{}%
\end{pgfscope}%
\end{pgfscope}%
\begin{pgfscope}%
\pgfpathrectangle{\pgfqpoint{0.100000in}{0.100000in}}{\pgfqpoint{5.307240in}{3.397500in}}%
\pgfusepath{clip}%
\pgfsetrectcap%
\pgfsetroundjoin%
\pgfsetlinewidth{1.505625pt}%
\definecolor{currentstroke}{rgb}{0.678431,1.000000,0.184314}%
\pgfsetstrokecolor{currentstroke}%
\pgfsetstrokeopacity{0.500000}%
\pgfsetdash{}{0pt}%
\pgfpathmoveto{\pgfqpoint{2.895040in}{1.300134in}}%
\pgfusepath{stroke}%
\end{pgfscope}%
\begin{pgfscope}%
\pgfpathrectangle{\pgfqpoint{0.100000in}{0.100000in}}{\pgfqpoint{5.307240in}{3.397500in}}%
\pgfusepath{clip}%
\pgfsetbuttcap%
\pgfsetroundjoin%
\definecolor{currentfill}{rgb}{0.678431,1.000000,0.184314}%
\pgfsetfillcolor{currentfill}%
\pgfsetfillopacity{0.500000}%
\pgfsetlinewidth{0.250937pt}%
\definecolor{currentstroke}{rgb}{0.000000,0.000000,0.000000}%
\pgfsetstrokecolor{currentstroke}%
\pgfsetstrokeopacity{0.500000}%
\pgfsetdash{}{0pt}%
\pgfsys@defobject{currentmarker}{\pgfqpoint{-0.055556in}{-0.055556in}}{\pgfqpoint{0.055556in}{0.055556in}}{%
\pgfpathmoveto{\pgfqpoint{0.000000in}{-0.055556in}}%
\pgfpathcurveto{\pgfqpoint{0.014734in}{-0.055556in}}{\pgfqpoint{0.028866in}{-0.049702in}}{\pgfqpoint{0.039284in}{-0.039284in}}%
\pgfpathcurveto{\pgfqpoint{0.049702in}{-0.028866in}}{\pgfqpoint{0.055556in}{-0.014734in}}{\pgfqpoint{0.055556in}{0.000000in}}%
\pgfpathcurveto{\pgfqpoint{0.055556in}{0.014734in}}{\pgfqpoint{0.049702in}{0.028866in}}{\pgfqpoint{0.039284in}{0.039284in}}%
\pgfpathcurveto{\pgfqpoint{0.028866in}{0.049702in}}{\pgfqpoint{0.014734in}{0.055556in}}{\pgfqpoint{0.000000in}{0.055556in}}%
\pgfpathcurveto{\pgfqpoint{-0.014734in}{0.055556in}}{\pgfqpoint{-0.028866in}{0.049702in}}{\pgfqpoint{-0.039284in}{0.039284in}}%
\pgfpathcurveto{\pgfqpoint{-0.049702in}{0.028866in}}{\pgfqpoint{-0.055556in}{0.014734in}}{\pgfqpoint{-0.055556in}{0.000000in}}%
\pgfpathcurveto{\pgfqpoint{-0.055556in}{-0.014734in}}{\pgfqpoint{-0.049702in}{-0.028866in}}{\pgfqpoint{-0.039284in}{-0.039284in}}%
\pgfpathcurveto{\pgfqpoint{-0.028866in}{-0.049702in}}{\pgfqpoint{-0.014734in}{-0.055556in}}{\pgfqpoint{0.000000in}{-0.055556in}}%
\pgfpathclose%
\pgfusepath{stroke,fill}%
}%
\begin{pgfscope}%
\pgfsys@transformshift{2.895040in}{1.300134in}%
\pgfsys@useobject{currentmarker}{}%
\end{pgfscope}%
\end{pgfscope}%
\begin{pgfscope}%
\pgfpathrectangle{\pgfqpoint{0.100000in}{0.100000in}}{\pgfqpoint{5.307240in}{3.397500in}}%
\pgfusepath{clip}%
\pgfsetrectcap%
\pgfsetroundjoin%
\pgfsetlinewidth{1.505625pt}%
\definecolor{currentstroke}{rgb}{0.678431,1.000000,0.184314}%
\pgfsetstrokecolor{currentstroke}%
\pgfsetstrokeopacity{0.500000}%
\pgfsetdash{}{0pt}%
\pgfpathmoveto{\pgfqpoint{3.144363in}{1.274606in}}%
\pgfusepath{stroke}%
\end{pgfscope}%
\begin{pgfscope}%
\pgfpathrectangle{\pgfqpoint{0.100000in}{0.100000in}}{\pgfqpoint{5.307240in}{3.397500in}}%
\pgfusepath{clip}%
\pgfsetbuttcap%
\pgfsetroundjoin%
\definecolor{currentfill}{rgb}{0.678431,1.000000,0.184314}%
\pgfsetfillcolor{currentfill}%
\pgfsetfillopacity{0.500000}%
\pgfsetlinewidth{0.250937pt}%
\definecolor{currentstroke}{rgb}{0.000000,0.000000,0.000000}%
\pgfsetstrokecolor{currentstroke}%
\pgfsetstrokeopacity{0.500000}%
\pgfsetdash{}{0pt}%
\pgfsys@defobject{currentmarker}{\pgfqpoint{-0.070139in}{-0.070139in}}{\pgfqpoint{0.070139in}{0.070139in}}{%
\pgfpathmoveto{\pgfqpoint{0.000000in}{-0.070139in}}%
\pgfpathcurveto{\pgfqpoint{0.018601in}{-0.070139in}}{\pgfqpoint{0.036443in}{-0.062749in}}{\pgfqpoint{0.049596in}{-0.049596in}}%
\pgfpathcurveto{\pgfqpoint{0.062749in}{-0.036443in}}{\pgfqpoint{0.070139in}{-0.018601in}}{\pgfqpoint{0.070139in}{0.000000in}}%
\pgfpathcurveto{\pgfqpoint{0.070139in}{0.018601in}}{\pgfqpoint{0.062749in}{0.036443in}}{\pgfqpoint{0.049596in}{0.049596in}}%
\pgfpathcurveto{\pgfqpoint{0.036443in}{0.062749in}}{\pgfqpoint{0.018601in}{0.070139in}}{\pgfqpoint{0.000000in}{0.070139in}}%
\pgfpathcurveto{\pgfqpoint{-0.018601in}{0.070139in}}{\pgfqpoint{-0.036443in}{0.062749in}}{\pgfqpoint{-0.049596in}{0.049596in}}%
\pgfpathcurveto{\pgfqpoint{-0.062749in}{0.036443in}}{\pgfqpoint{-0.070139in}{0.018601in}}{\pgfqpoint{-0.070139in}{0.000000in}}%
\pgfpathcurveto{\pgfqpoint{-0.070139in}{-0.018601in}}{\pgfqpoint{-0.062749in}{-0.036443in}}{\pgfqpoint{-0.049596in}{-0.049596in}}%
\pgfpathcurveto{\pgfqpoint{-0.036443in}{-0.062749in}}{\pgfqpoint{-0.018601in}{-0.070139in}}{\pgfqpoint{0.000000in}{-0.070139in}}%
\pgfpathclose%
\pgfusepath{stroke,fill}%
}%
\begin{pgfscope}%
\pgfsys@transformshift{3.144363in}{1.274606in}%
\pgfsys@useobject{currentmarker}{}%
\end{pgfscope}%
\end{pgfscope}%
\begin{pgfscope}%
\pgfpathrectangle{\pgfqpoint{0.100000in}{0.100000in}}{\pgfqpoint{5.307240in}{3.397500in}}%
\pgfusepath{clip}%
\pgfsetrectcap%
\pgfsetroundjoin%
\pgfsetlinewidth{1.505625pt}%
\definecolor{currentstroke}{rgb}{0.678431,1.000000,0.184314}%
\pgfsetstrokecolor{currentstroke}%
\pgfsetstrokeopacity{0.500000}%
\pgfsetdash{}{0pt}%
\pgfpathmoveto{\pgfqpoint{3.021762in}{1.148617in}}%
\pgfusepath{stroke}%
\end{pgfscope}%
\begin{pgfscope}%
\pgfpathrectangle{\pgfqpoint{0.100000in}{0.100000in}}{\pgfqpoint{5.307240in}{3.397500in}}%
\pgfusepath{clip}%
\pgfsetbuttcap%
\pgfsetroundjoin%
\definecolor{currentfill}{rgb}{0.678431,1.000000,0.184314}%
\pgfsetfillcolor{currentfill}%
\pgfsetfillopacity{0.500000}%
\pgfsetlinewidth{0.250937pt}%
\definecolor{currentstroke}{rgb}{0.000000,0.000000,0.000000}%
\pgfsetstrokecolor{currentstroke}%
\pgfsetstrokeopacity{0.500000}%
\pgfsetdash{}{0pt}%
\pgfsys@defobject{currentmarker}{\pgfqpoint{-0.065278in}{-0.065278in}}{\pgfqpoint{0.065278in}{0.065278in}}{%
\pgfpathmoveto{\pgfqpoint{0.000000in}{-0.065278in}}%
\pgfpathcurveto{\pgfqpoint{0.017312in}{-0.065278in}}{\pgfqpoint{0.033917in}{-0.058400in}}{\pgfqpoint{0.046158in}{-0.046158in}}%
\pgfpathcurveto{\pgfqpoint{0.058400in}{-0.033917in}}{\pgfqpoint{0.065278in}{-0.017312in}}{\pgfqpoint{0.065278in}{0.000000in}}%
\pgfpathcurveto{\pgfqpoint{0.065278in}{0.017312in}}{\pgfqpoint{0.058400in}{0.033917in}}{\pgfqpoint{0.046158in}{0.046158in}}%
\pgfpathcurveto{\pgfqpoint{0.033917in}{0.058400in}}{\pgfqpoint{0.017312in}{0.065278in}}{\pgfqpoint{0.000000in}{0.065278in}}%
\pgfpathcurveto{\pgfqpoint{-0.017312in}{0.065278in}}{\pgfqpoint{-0.033917in}{0.058400in}}{\pgfqpoint{-0.046158in}{0.046158in}}%
\pgfpathcurveto{\pgfqpoint{-0.058400in}{0.033917in}}{\pgfqpoint{-0.065278in}{0.017312in}}{\pgfqpoint{-0.065278in}{0.000000in}}%
\pgfpathcurveto{\pgfqpoint{-0.065278in}{-0.017312in}}{\pgfqpoint{-0.058400in}{-0.033917in}}{\pgfqpoint{-0.046158in}{-0.046158in}}%
\pgfpathcurveto{\pgfqpoint{-0.033917in}{-0.058400in}}{\pgfqpoint{-0.017312in}{-0.065278in}}{\pgfqpoint{0.000000in}{-0.065278in}}%
\pgfpathclose%
\pgfusepath{stroke,fill}%
}%
\begin{pgfscope}%
\pgfsys@transformshift{3.021762in}{1.148617in}%
\pgfsys@useobject{currentmarker}{}%
\end{pgfscope}%
\end{pgfscope}%
\begin{pgfscope}%
\pgfpathrectangle{\pgfqpoint{0.100000in}{0.100000in}}{\pgfqpoint{5.307240in}{3.397500in}}%
\pgfusepath{clip}%
\pgfsetrectcap%
\pgfsetroundjoin%
\pgfsetlinewidth{1.505625pt}%
\definecolor{currentstroke}{rgb}{0.678431,1.000000,0.184314}%
\pgfsetstrokecolor{currentstroke}%
\pgfsetstrokeopacity{0.500000}%
\pgfsetdash{}{0pt}%
\pgfpathmoveto{\pgfqpoint{2.844266in}{0.734156in}}%
\pgfusepath{stroke}%
\end{pgfscope}%
\begin{pgfscope}%
\pgfpathrectangle{\pgfqpoint{0.100000in}{0.100000in}}{\pgfqpoint{5.307240in}{3.397500in}}%
\pgfusepath{clip}%
\pgfsetbuttcap%
\pgfsetroundjoin%
\definecolor{currentfill}{rgb}{0.678431,1.000000,0.184314}%
\pgfsetfillcolor{currentfill}%
\pgfsetfillopacity{0.500000}%
\pgfsetlinewidth{0.250937pt}%
\definecolor{currentstroke}{rgb}{0.000000,0.000000,0.000000}%
\pgfsetstrokecolor{currentstroke}%
\pgfsetstrokeopacity{0.500000}%
\pgfsetdash{}{0pt}%
\pgfsys@defobject{currentmarker}{\pgfqpoint{-0.078472in}{-0.078472in}}{\pgfqpoint{0.078472in}{0.078472in}}{%
\pgfpathmoveto{\pgfqpoint{0.000000in}{-0.078472in}}%
\pgfpathcurveto{\pgfqpoint{0.020811in}{-0.078472in}}{\pgfqpoint{0.040773in}{-0.070204in}}{\pgfqpoint{0.055488in}{-0.055488in}}%
\pgfpathcurveto{\pgfqpoint{0.070204in}{-0.040773in}}{\pgfqpoint{0.078472in}{-0.020811in}}{\pgfqpoint{0.078472in}{0.000000in}}%
\pgfpathcurveto{\pgfqpoint{0.078472in}{0.020811in}}{\pgfqpoint{0.070204in}{0.040773in}}{\pgfqpoint{0.055488in}{0.055488in}}%
\pgfpathcurveto{\pgfqpoint{0.040773in}{0.070204in}}{\pgfqpoint{0.020811in}{0.078472in}}{\pgfqpoint{0.000000in}{0.078472in}}%
\pgfpathcurveto{\pgfqpoint{-0.020811in}{0.078472in}}{\pgfqpoint{-0.040773in}{0.070204in}}{\pgfqpoint{-0.055488in}{0.055488in}}%
\pgfpathcurveto{\pgfqpoint{-0.070204in}{0.040773in}}{\pgfqpoint{-0.078472in}{0.020811in}}{\pgfqpoint{-0.078472in}{0.000000in}}%
\pgfpathcurveto{\pgfqpoint{-0.078472in}{-0.020811in}}{\pgfqpoint{-0.070204in}{-0.040773in}}{\pgfqpoint{-0.055488in}{-0.055488in}}%
\pgfpathcurveto{\pgfqpoint{-0.040773in}{-0.070204in}}{\pgfqpoint{-0.020811in}{-0.078472in}}{\pgfqpoint{0.000000in}{-0.078472in}}%
\pgfpathclose%
\pgfusepath{stroke,fill}%
}%
\begin{pgfscope}%
\pgfsys@transformshift{2.844266in}{0.734156in}%
\pgfsys@useobject{currentmarker}{}%
\end{pgfscope}%
\end{pgfscope}%
\begin{pgfscope}%
\pgfpathrectangle{\pgfqpoint{0.100000in}{0.100000in}}{\pgfqpoint{5.307240in}{3.397500in}}%
\pgfusepath{clip}%
\pgfsetrectcap%
\pgfsetroundjoin%
\pgfsetlinewidth{1.505625pt}%
\definecolor{currentstroke}{rgb}{0.678431,1.000000,0.184314}%
\pgfsetstrokecolor{currentstroke}%
\pgfsetstrokeopacity{0.500000}%
\pgfsetdash{}{0pt}%
\pgfpathmoveto{\pgfqpoint{2.837092in}{1.057172in}}%
\pgfusepath{stroke}%
\end{pgfscope}%
\begin{pgfscope}%
\pgfpathrectangle{\pgfqpoint{0.100000in}{0.100000in}}{\pgfqpoint{5.307240in}{3.397500in}}%
\pgfusepath{clip}%
\pgfsetbuttcap%
\pgfsetroundjoin%
\definecolor{currentfill}{rgb}{0.678431,1.000000,0.184314}%
\pgfsetfillcolor{currentfill}%
\pgfsetfillopacity{0.500000}%
\pgfsetlinewidth{0.250937pt}%
\definecolor{currentstroke}{rgb}{0.000000,0.000000,0.000000}%
\pgfsetstrokecolor{currentstroke}%
\pgfsetstrokeopacity{0.500000}%
\pgfsetdash{}{0pt}%
\pgfsys@defobject{currentmarker}{\pgfqpoint{-0.054167in}{-0.054167in}}{\pgfqpoint{0.054167in}{0.054167in}}{%
\pgfpathmoveto{\pgfqpoint{0.000000in}{-0.054167in}}%
\pgfpathcurveto{\pgfqpoint{0.014365in}{-0.054167in}}{\pgfqpoint{0.028144in}{-0.048459in}}{\pgfqpoint{0.038302in}{-0.038302in}}%
\pgfpathcurveto{\pgfqpoint{0.048459in}{-0.028144in}}{\pgfqpoint{0.054167in}{-0.014365in}}{\pgfqpoint{0.054167in}{0.000000in}}%
\pgfpathcurveto{\pgfqpoint{0.054167in}{0.014365in}}{\pgfqpoint{0.048459in}{0.028144in}}{\pgfqpoint{0.038302in}{0.038302in}}%
\pgfpathcurveto{\pgfqpoint{0.028144in}{0.048459in}}{\pgfqpoint{0.014365in}{0.054167in}}{\pgfqpoint{0.000000in}{0.054167in}}%
\pgfpathcurveto{\pgfqpoint{-0.014365in}{0.054167in}}{\pgfqpoint{-0.028144in}{0.048459in}}{\pgfqpoint{-0.038302in}{0.038302in}}%
\pgfpathcurveto{\pgfqpoint{-0.048459in}{0.028144in}}{\pgfqpoint{-0.054167in}{0.014365in}}{\pgfqpoint{-0.054167in}{0.000000in}}%
\pgfpathcurveto{\pgfqpoint{-0.054167in}{-0.014365in}}{\pgfqpoint{-0.048459in}{-0.028144in}}{\pgfqpoint{-0.038302in}{-0.038302in}}%
\pgfpathcurveto{\pgfqpoint{-0.028144in}{-0.048459in}}{\pgfqpoint{-0.014365in}{-0.054167in}}{\pgfqpoint{0.000000in}{-0.054167in}}%
\pgfpathclose%
\pgfusepath{stroke,fill}%
}%
\begin{pgfscope}%
\pgfsys@transformshift{2.837092in}{1.057172in}%
\pgfsys@useobject{currentmarker}{}%
\end{pgfscope}%
\end{pgfscope}%
\begin{pgfscope}%
\pgfpathrectangle{\pgfqpoint{0.100000in}{0.100000in}}{\pgfqpoint{5.307240in}{3.397500in}}%
\pgfusepath{clip}%
\pgfsetrectcap%
\pgfsetroundjoin%
\pgfsetlinewidth{1.505625pt}%
\definecolor{currentstroke}{rgb}{0.678431,1.000000,0.184314}%
\pgfsetstrokecolor{currentstroke}%
\pgfsetstrokeopacity{0.500000}%
\pgfsetdash{}{0pt}%
\pgfpathmoveto{\pgfqpoint{2.713247in}{1.337669in}}%
\pgfusepath{stroke}%
\end{pgfscope}%
\begin{pgfscope}%
\pgfpathrectangle{\pgfqpoint{0.100000in}{0.100000in}}{\pgfqpoint{5.307240in}{3.397500in}}%
\pgfusepath{clip}%
\pgfsetbuttcap%
\pgfsetroundjoin%
\definecolor{currentfill}{rgb}{0.678431,1.000000,0.184314}%
\pgfsetfillcolor{currentfill}%
\pgfsetfillopacity{0.500000}%
\pgfsetlinewidth{0.250937pt}%
\definecolor{currentstroke}{rgb}{0.000000,0.000000,0.000000}%
\pgfsetstrokecolor{currentstroke}%
\pgfsetstrokeopacity{0.500000}%
\pgfsetdash{}{0pt}%
\pgfsys@defobject{currentmarker}{\pgfqpoint{-0.059722in}{-0.059722in}}{\pgfqpoint{0.059722in}{0.059722in}}{%
\pgfpathmoveto{\pgfqpoint{0.000000in}{-0.059722in}}%
\pgfpathcurveto{\pgfqpoint{0.015839in}{-0.059722in}}{\pgfqpoint{0.031030in}{-0.053430in}}{\pgfqpoint{0.042230in}{-0.042230in}}%
\pgfpathcurveto{\pgfqpoint{0.053430in}{-0.031030in}}{\pgfqpoint{0.059722in}{-0.015839in}}{\pgfqpoint{0.059722in}{0.000000in}}%
\pgfpathcurveto{\pgfqpoint{0.059722in}{0.015839in}}{\pgfqpoint{0.053430in}{0.031030in}}{\pgfqpoint{0.042230in}{0.042230in}}%
\pgfpathcurveto{\pgfqpoint{0.031030in}{0.053430in}}{\pgfqpoint{0.015839in}{0.059722in}}{\pgfqpoint{0.000000in}{0.059722in}}%
\pgfpathcurveto{\pgfqpoint{-0.015839in}{0.059722in}}{\pgfqpoint{-0.031030in}{0.053430in}}{\pgfqpoint{-0.042230in}{0.042230in}}%
\pgfpathcurveto{\pgfqpoint{-0.053430in}{0.031030in}}{\pgfqpoint{-0.059722in}{0.015839in}}{\pgfqpoint{-0.059722in}{0.000000in}}%
\pgfpathcurveto{\pgfqpoint{-0.059722in}{-0.015839in}}{\pgfqpoint{-0.053430in}{-0.031030in}}{\pgfqpoint{-0.042230in}{-0.042230in}}%
\pgfpathcurveto{\pgfqpoint{-0.031030in}{-0.053430in}}{\pgfqpoint{-0.015839in}{-0.059722in}}{\pgfqpoint{0.000000in}{-0.059722in}}%
\pgfpathclose%
\pgfusepath{stroke,fill}%
}%
\begin{pgfscope}%
\pgfsys@transformshift{2.713247in}{1.337669in}%
\pgfsys@useobject{currentmarker}{}%
\end{pgfscope}%
\end{pgfscope}%
\begin{pgfscope}%
\pgfpathrectangle{\pgfqpoint{0.100000in}{0.100000in}}{\pgfqpoint{5.307240in}{3.397500in}}%
\pgfusepath{clip}%
\pgfsetrectcap%
\pgfsetroundjoin%
\pgfsetlinewidth{1.505625pt}%
\definecolor{currentstroke}{rgb}{0.678431,1.000000,0.184314}%
\pgfsetstrokecolor{currentstroke}%
\pgfsetstrokeopacity{0.500000}%
\pgfsetdash{}{0pt}%
\pgfpathmoveto{\pgfqpoint{1.597804in}{2.375914in}}%
\pgfusepath{stroke}%
\end{pgfscope}%
\begin{pgfscope}%
\pgfpathrectangle{\pgfqpoint{0.100000in}{0.100000in}}{\pgfqpoint{5.307240in}{3.397500in}}%
\pgfusepath{clip}%
\pgfsetbuttcap%
\pgfsetroundjoin%
\definecolor{currentfill}{rgb}{0.678431,1.000000,0.184314}%
\pgfsetfillcolor{currentfill}%
\pgfsetfillopacity{0.500000}%
\pgfsetlinewidth{0.250937pt}%
\definecolor{currentstroke}{rgb}{0.000000,0.000000,0.000000}%
\pgfsetstrokecolor{currentstroke}%
\pgfsetstrokeopacity{0.500000}%
\pgfsetdash{}{0pt}%
\pgfsys@defobject{currentmarker}{\pgfqpoint{-0.027778in}{-0.027778in}}{\pgfqpoint{0.027778in}{0.027778in}}{%
\pgfpathmoveto{\pgfqpoint{0.000000in}{-0.027778in}}%
\pgfpathcurveto{\pgfqpoint{0.007367in}{-0.027778in}}{\pgfqpoint{0.014433in}{-0.024851in}}{\pgfqpoint{0.019642in}{-0.019642in}}%
\pgfpathcurveto{\pgfqpoint{0.024851in}{-0.014433in}}{\pgfqpoint{0.027778in}{-0.007367in}}{\pgfqpoint{0.027778in}{0.000000in}}%
\pgfpathcurveto{\pgfqpoint{0.027778in}{0.007367in}}{\pgfqpoint{0.024851in}{0.014433in}}{\pgfqpoint{0.019642in}{0.019642in}}%
\pgfpathcurveto{\pgfqpoint{0.014433in}{0.024851in}}{\pgfqpoint{0.007367in}{0.027778in}}{\pgfqpoint{0.000000in}{0.027778in}}%
\pgfpathcurveto{\pgfqpoint{-0.007367in}{0.027778in}}{\pgfqpoint{-0.014433in}{0.024851in}}{\pgfqpoint{-0.019642in}{0.019642in}}%
\pgfpathcurveto{\pgfqpoint{-0.024851in}{0.014433in}}{\pgfqpoint{-0.027778in}{0.007367in}}{\pgfqpoint{-0.027778in}{0.000000in}}%
\pgfpathcurveto{\pgfqpoint{-0.027778in}{-0.007367in}}{\pgfqpoint{-0.024851in}{-0.014433in}}{\pgfqpoint{-0.019642in}{-0.019642in}}%
\pgfpathcurveto{\pgfqpoint{-0.014433in}{-0.024851in}}{\pgfqpoint{-0.007367in}{-0.027778in}}{\pgfqpoint{0.000000in}{-0.027778in}}%
\pgfpathclose%
\pgfusepath{stroke,fill}%
}%
\begin{pgfscope}%
\pgfsys@transformshift{1.597804in}{2.375914in}%
\pgfsys@useobject{currentmarker}{}%
\end{pgfscope}%
\end{pgfscope}%
\begin{pgfscope}%
\pgfpathrectangle{\pgfqpoint{0.100000in}{0.100000in}}{\pgfqpoint{5.307240in}{3.397500in}}%
\pgfusepath{clip}%
\pgfsetrectcap%
\pgfsetroundjoin%
\pgfsetlinewidth{1.505625pt}%
\definecolor{currentstroke}{rgb}{0.678431,1.000000,0.184314}%
\pgfsetstrokecolor{currentstroke}%
\pgfsetstrokeopacity{0.500000}%
\pgfsetdash{}{0pt}%
\pgfpathmoveto{\pgfqpoint{1.574959in}{2.320028in}}%
\pgfusepath{stroke}%
\end{pgfscope}%
\begin{pgfscope}%
\pgfpathrectangle{\pgfqpoint{0.100000in}{0.100000in}}{\pgfqpoint{5.307240in}{3.397500in}}%
\pgfusepath{clip}%
\pgfsetbuttcap%
\pgfsetroundjoin%
\definecolor{currentfill}{rgb}{0.678431,1.000000,0.184314}%
\pgfsetfillcolor{currentfill}%
\pgfsetfillopacity{0.500000}%
\pgfsetlinewidth{0.250937pt}%
\definecolor{currentstroke}{rgb}{0.000000,0.000000,0.000000}%
\pgfsetstrokecolor{currentstroke}%
\pgfsetstrokeopacity{0.500000}%
\pgfsetdash{}{0pt}%
\pgfsys@defobject{currentmarker}{\pgfqpoint{-0.049306in}{-0.049306in}}{\pgfqpoint{0.049306in}{0.049306in}}{%
\pgfpathmoveto{\pgfqpoint{0.000000in}{-0.049306in}}%
\pgfpathcurveto{\pgfqpoint{0.013076in}{-0.049306in}}{\pgfqpoint{0.025618in}{-0.044110in}}{\pgfqpoint{0.034864in}{-0.034864in}}%
\pgfpathcurveto{\pgfqpoint{0.044110in}{-0.025618in}}{\pgfqpoint{0.049306in}{-0.013076in}}{\pgfqpoint{0.049306in}{0.000000in}}%
\pgfpathcurveto{\pgfqpoint{0.049306in}{0.013076in}}{\pgfqpoint{0.044110in}{0.025618in}}{\pgfqpoint{0.034864in}{0.034864in}}%
\pgfpathcurveto{\pgfqpoint{0.025618in}{0.044110in}}{\pgfqpoint{0.013076in}{0.049306in}}{\pgfqpoint{0.000000in}{0.049306in}}%
\pgfpathcurveto{\pgfqpoint{-0.013076in}{0.049306in}}{\pgfqpoint{-0.025618in}{0.044110in}}{\pgfqpoint{-0.034864in}{0.034864in}}%
\pgfpathcurveto{\pgfqpoint{-0.044110in}{0.025618in}}{\pgfqpoint{-0.049306in}{0.013076in}}{\pgfqpoint{-0.049306in}{0.000000in}}%
\pgfpathcurveto{\pgfqpoint{-0.049306in}{-0.013076in}}{\pgfqpoint{-0.044110in}{-0.025618in}}{\pgfqpoint{-0.034864in}{-0.034864in}}%
\pgfpathcurveto{\pgfqpoint{-0.025618in}{-0.044110in}}{\pgfqpoint{-0.013076in}{-0.049306in}}{\pgfqpoint{0.000000in}{-0.049306in}}%
\pgfpathclose%
\pgfusepath{stroke,fill}%
}%
\begin{pgfscope}%
\pgfsys@transformshift{1.574959in}{2.320028in}%
\pgfsys@useobject{currentmarker}{}%
\end{pgfscope}%
\end{pgfscope}%
\begin{pgfscope}%
\pgfpathrectangle{\pgfqpoint{0.100000in}{0.100000in}}{\pgfqpoint{5.307240in}{3.397500in}}%
\pgfusepath{clip}%
\pgfsetrectcap%
\pgfsetroundjoin%
\pgfsetlinewidth{1.505625pt}%
\definecolor{currentstroke}{rgb}{0.678431,1.000000,0.184314}%
\pgfsetstrokecolor{currentstroke}%
\pgfsetstrokeopacity{0.500000}%
\pgfsetdash{}{0pt}%
\pgfpathmoveto{\pgfqpoint{1.581107in}{2.201841in}}%
\pgfusepath{stroke}%
\end{pgfscope}%
\begin{pgfscope}%
\pgfpathrectangle{\pgfqpoint{0.100000in}{0.100000in}}{\pgfqpoint{5.307240in}{3.397500in}}%
\pgfusepath{clip}%
\pgfsetbuttcap%
\pgfsetroundjoin%
\definecolor{currentfill}{rgb}{0.678431,1.000000,0.184314}%
\pgfsetfillcolor{currentfill}%
\pgfsetfillopacity{0.500000}%
\pgfsetlinewidth{0.250937pt}%
\definecolor{currentstroke}{rgb}{0.000000,0.000000,0.000000}%
\pgfsetstrokecolor{currentstroke}%
\pgfsetstrokeopacity{0.500000}%
\pgfsetdash{}{0pt}%
\pgfsys@defobject{currentmarker}{\pgfqpoint{-0.037500in}{-0.037500in}}{\pgfqpoint{0.037500in}{0.037500in}}{%
\pgfpathmoveto{\pgfqpoint{0.000000in}{-0.037500in}}%
\pgfpathcurveto{\pgfqpoint{0.009945in}{-0.037500in}}{\pgfqpoint{0.019484in}{-0.033549in}}{\pgfqpoint{0.026517in}{-0.026517in}}%
\pgfpathcurveto{\pgfqpoint{0.033549in}{-0.019484in}}{\pgfqpoint{0.037500in}{-0.009945in}}{\pgfqpoint{0.037500in}{0.000000in}}%
\pgfpathcurveto{\pgfqpoint{0.037500in}{0.009945in}}{\pgfqpoint{0.033549in}{0.019484in}}{\pgfqpoint{0.026517in}{0.026517in}}%
\pgfpathcurveto{\pgfqpoint{0.019484in}{0.033549in}}{\pgfqpoint{0.009945in}{0.037500in}}{\pgfqpoint{0.000000in}{0.037500in}}%
\pgfpathcurveto{\pgfqpoint{-0.009945in}{0.037500in}}{\pgfqpoint{-0.019484in}{0.033549in}}{\pgfqpoint{-0.026517in}{0.026517in}}%
\pgfpathcurveto{\pgfqpoint{-0.033549in}{0.019484in}}{\pgfqpoint{-0.037500in}{0.009945in}}{\pgfqpoint{-0.037500in}{0.000000in}}%
\pgfpathcurveto{\pgfqpoint{-0.037500in}{-0.009945in}}{\pgfqpoint{-0.033549in}{-0.019484in}}{\pgfqpoint{-0.026517in}{-0.026517in}}%
\pgfpathcurveto{\pgfqpoint{-0.019484in}{-0.033549in}}{\pgfqpoint{-0.009945in}{-0.037500in}}{\pgfqpoint{0.000000in}{-0.037500in}}%
\pgfpathclose%
\pgfusepath{stroke,fill}%
}%
\begin{pgfscope}%
\pgfsys@transformshift{1.581107in}{2.201841in}%
\pgfsys@useobject{currentmarker}{}%
\end{pgfscope}%
\end{pgfscope}%
\begin{pgfscope}%
\pgfpathrectangle{\pgfqpoint{0.100000in}{0.100000in}}{\pgfqpoint{5.307240in}{3.397500in}}%
\pgfusepath{clip}%
\pgfsetrectcap%
\pgfsetroundjoin%
\pgfsetlinewidth{1.505625pt}%
\definecolor{currentstroke}{rgb}{0.678431,1.000000,0.184314}%
\pgfsetstrokecolor{currentstroke}%
\pgfsetstrokeopacity{0.500000}%
\pgfsetdash{}{0pt}%
\pgfpathmoveto{\pgfqpoint{1.339493in}{1.878478in}}%
\pgfusepath{stroke}%
\end{pgfscope}%
\begin{pgfscope}%
\pgfpathrectangle{\pgfqpoint{0.100000in}{0.100000in}}{\pgfqpoint{5.307240in}{3.397500in}}%
\pgfusepath{clip}%
\pgfsetbuttcap%
\pgfsetroundjoin%
\definecolor{currentfill}{rgb}{0.678431,1.000000,0.184314}%
\pgfsetfillcolor{currentfill}%
\pgfsetfillopacity{0.500000}%
\pgfsetlinewidth{0.250937pt}%
\definecolor{currentstroke}{rgb}{0.000000,0.000000,0.000000}%
\pgfsetstrokecolor{currentstroke}%
\pgfsetstrokeopacity{0.500000}%
\pgfsetdash{}{0pt}%
\pgfsys@defobject{currentmarker}{\pgfqpoint{-0.065278in}{-0.065278in}}{\pgfqpoint{0.065278in}{0.065278in}}{%
\pgfpathmoveto{\pgfqpoint{0.000000in}{-0.065278in}}%
\pgfpathcurveto{\pgfqpoint{0.017312in}{-0.065278in}}{\pgfqpoint{0.033917in}{-0.058400in}}{\pgfqpoint{0.046158in}{-0.046158in}}%
\pgfpathcurveto{\pgfqpoint{0.058400in}{-0.033917in}}{\pgfqpoint{0.065278in}{-0.017312in}}{\pgfqpoint{0.065278in}{0.000000in}}%
\pgfpathcurveto{\pgfqpoint{0.065278in}{0.017312in}}{\pgfqpoint{0.058400in}{0.033917in}}{\pgfqpoint{0.046158in}{0.046158in}}%
\pgfpathcurveto{\pgfqpoint{0.033917in}{0.058400in}}{\pgfqpoint{0.017312in}{0.065278in}}{\pgfqpoint{0.000000in}{0.065278in}}%
\pgfpathcurveto{\pgfqpoint{-0.017312in}{0.065278in}}{\pgfqpoint{-0.033917in}{0.058400in}}{\pgfqpoint{-0.046158in}{0.046158in}}%
\pgfpathcurveto{\pgfqpoint{-0.058400in}{0.033917in}}{\pgfqpoint{-0.065278in}{0.017312in}}{\pgfqpoint{-0.065278in}{0.000000in}}%
\pgfpathcurveto{\pgfqpoint{-0.065278in}{-0.017312in}}{\pgfqpoint{-0.058400in}{-0.033917in}}{\pgfqpoint{-0.046158in}{-0.046158in}}%
\pgfpathcurveto{\pgfqpoint{-0.033917in}{-0.058400in}}{\pgfqpoint{-0.017312in}{-0.065278in}}{\pgfqpoint{0.000000in}{-0.065278in}}%
\pgfpathclose%
\pgfusepath{stroke,fill}%
}%
\begin{pgfscope}%
\pgfsys@transformshift{1.339493in}{1.878478in}%
\pgfsys@useobject{currentmarker}{}%
\end{pgfscope}%
\end{pgfscope}%
\begin{pgfscope}%
\pgfpathrectangle{\pgfqpoint{0.100000in}{0.100000in}}{\pgfqpoint{5.307240in}{3.397500in}}%
\pgfusepath{clip}%
\pgfsetrectcap%
\pgfsetroundjoin%
\pgfsetlinewidth{1.505625pt}%
\definecolor{currentstroke}{rgb}{0.678431,1.000000,0.184314}%
\pgfsetstrokecolor{currentstroke}%
\pgfsetstrokeopacity{0.500000}%
\pgfsetdash{}{0pt}%
\pgfpathmoveto{\pgfqpoint{1.572329in}{2.266551in}}%
\pgfusepath{stroke}%
\end{pgfscope}%
\begin{pgfscope}%
\pgfpathrectangle{\pgfqpoint{0.100000in}{0.100000in}}{\pgfqpoint{5.307240in}{3.397500in}}%
\pgfusepath{clip}%
\pgfsetbuttcap%
\pgfsetroundjoin%
\definecolor{currentfill}{rgb}{0.678431,1.000000,0.184314}%
\pgfsetfillcolor{currentfill}%
\pgfsetfillopacity{0.500000}%
\pgfsetlinewidth{0.250937pt}%
\definecolor{currentstroke}{rgb}{0.000000,0.000000,0.000000}%
\pgfsetstrokecolor{currentstroke}%
\pgfsetstrokeopacity{0.500000}%
\pgfsetdash{}{0pt}%
\pgfsys@defobject{currentmarker}{\pgfqpoint{-0.059722in}{-0.059722in}}{\pgfqpoint{0.059722in}{0.059722in}}{%
\pgfpathmoveto{\pgfqpoint{0.000000in}{-0.059722in}}%
\pgfpathcurveto{\pgfqpoint{0.015839in}{-0.059722in}}{\pgfqpoint{0.031030in}{-0.053430in}}{\pgfqpoint{0.042230in}{-0.042230in}}%
\pgfpathcurveto{\pgfqpoint{0.053430in}{-0.031030in}}{\pgfqpoint{0.059722in}{-0.015839in}}{\pgfqpoint{0.059722in}{0.000000in}}%
\pgfpathcurveto{\pgfqpoint{0.059722in}{0.015839in}}{\pgfqpoint{0.053430in}{0.031030in}}{\pgfqpoint{0.042230in}{0.042230in}}%
\pgfpathcurveto{\pgfqpoint{0.031030in}{0.053430in}}{\pgfqpoint{0.015839in}{0.059722in}}{\pgfqpoint{0.000000in}{0.059722in}}%
\pgfpathcurveto{\pgfqpoint{-0.015839in}{0.059722in}}{\pgfqpoint{-0.031030in}{0.053430in}}{\pgfqpoint{-0.042230in}{0.042230in}}%
\pgfpathcurveto{\pgfqpoint{-0.053430in}{0.031030in}}{\pgfqpoint{-0.059722in}{0.015839in}}{\pgfqpoint{-0.059722in}{0.000000in}}%
\pgfpathcurveto{\pgfqpoint{-0.059722in}{-0.015839in}}{\pgfqpoint{-0.053430in}{-0.031030in}}{\pgfqpoint{-0.042230in}{-0.042230in}}%
\pgfpathcurveto{\pgfqpoint{-0.031030in}{-0.053430in}}{\pgfqpoint{-0.015839in}{-0.059722in}}{\pgfqpoint{0.000000in}{-0.059722in}}%
\pgfpathclose%
\pgfusepath{stroke,fill}%
}%
\begin{pgfscope}%
\pgfsys@transformshift{1.572329in}{2.266551in}%
\pgfsys@useobject{currentmarker}{}%
\end{pgfscope}%
\end{pgfscope}%
\begin{pgfscope}%
\pgfpathrectangle{\pgfqpoint{0.100000in}{0.100000in}}{\pgfqpoint{5.307240in}{3.397500in}}%
\pgfusepath{clip}%
\pgfsetrectcap%
\pgfsetroundjoin%
\pgfsetlinewidth{1.505625pt}%
\definecolor{currentstroke}{rgb}{0.678431,1.000000,0.184314}%
\pgfsetstrokecolor{currentstroke}%
\pgfsetstrokeopacity{0.500000}%
\pgfsetdash{}{0pt}%
\pgfpathmoveto{\pgfqpoint{4.849868in}{2.776979in}}%
\pgfusepath{stroke}%
\end{pgfscope}%
\begin{pgfscope}%
\pgfpathrectangle{\pgfqpoint{0.100000in}{0.100000in}}{\pgfqpoint{5.307240in}{3.397500in}}%
\pgfusepath{clip}%
\pgfsetbuttcap%
\pgfsetroundjoin%
\definecolor{currentfill}{rgb}{0.678431,1.000000,0.184314}%
\pgfsetfillcolor{currentfill}%
\pgfsetfillopacity{0.500000}%
\pgfsetlinewidth{0.250937pt}%
\definecolor{currentstroke}{rgb}{0.000000,0.000000,0.000000}%
\pgfsetstrokecolor{currentstroke}%
\pgfsetstrokeopacity{0.500000}%
\pgfsetdash{}{0pt}%
\pgfsys@defobject{currentmarker}{\pgfqpoint{-0.088889in}{-0.088889in}}{\pgfqpoint{0.088889in}{0.088889in}}{%
\pgfpathmoveto{\pgfqpoint{0.000000in}{-0.088889in}}%
\pgfpathcurveto{\pgfqpoint{0.023574in}{-0.088889in}}{\pgfqpoint{0.046185in}{-0.079523in}}{\pgfqpoint{0.062854in}{-0.062854in}}%
\pgfpathcurveto{\pgfqpoint{0.079523in}{-0.046185in}}{\pgfqpoint{0.088889in}{-0.023574in}}{\pgfqpoint{0.088889in}{0.000000in}}%
\pgfpathcurveto{\pgfqpoint{0.088889in}{0.023574in}}{\pgfqpoint{0.079523in}{0.046185in}}{\pgfqpoint{0.062854in}{0.062854in}}%
\pgfpathcurveto{\pgfqpoint{0.046185in}{0.079523in}}{\pgfqpoint{0.023574in}{0.088889in}}{\pgfqpoint{0.000000in}{0.088889in}}%
\pgfpathcurveto{\pgfqpoint{-0.023574in}{0.088889in}}{\pgfqpoint{-0.046185in}{0.079523in}}{\pgfqpoint{-0.062854in}{0.062854in}}%
\pgfpathcurveto{\pgfqpoint{-0.079523in}{0.046185in}}{\pgfqpoint{-0.088889in}{0.023574in}}{\pgfqpoint{-0.088889in}{0.000000in}}%
\pgfpathcurveto{\pgfqpoint{-0.088889in}{-0.023574in}}{\pgfqpoint{-0.079523in}{-0.046185in}}{\pgfqpoint{-0.062854in}{-0.062854in}}%
\pgfpathcurveto{\pgfqpoint{-0.046185in}{-0.079523in}}{\pgfqpoint{-0.023574in}{-0.088889in}}{\pgfqpoint{0.000000in}{-0.088889in}}%
\pgfpathclose%
\pgfusepath{stroke,fill}%
}%
\begin{pgfscope}%
\pgfsys@transformshift{4.849868in}{2.776979in}%
\pgfsys@useobject{currentmarker}{}%
\end{pgfscope}%
\end{pgfscope}%
\begin{pgfscope}%
\pgfpathrectangle{\pgfqpoint{0.100000in}{0.100000in}}{\pgfqpoint{5.307240in}{3.397500in}}%
\pgfusepath{clip}%
\pgfsetrectcap%
\pgfsetroundjoin%
\pgfsetlinewidth{1.505625pt}%
\definecolor{currentstroke}{rgb}{0.678431,1.000000,0.184314}%
\pgfsetstrokecolor{currentstroke}%
\pgfsetstrokeopacity{0.500000}%
\pgfsetdash{}{0pt}%
\pgfpathmoveto{\pgfqpoint{4.396545in}{1.825584in}}%
\pgfusepath{stroke}%
\end{pgfscope}%
\begin{pgfscope}%
\pgfpathrectangle{\pgfqpoint{0.100000in}{0.100000in}}{\pgfqpoint{5.307240in}{3.397500in}}%
\pgfusepath{clip}%
\pgfsetbuttcap%
\pgfsetroundjoin%
\definecolor{currentfill}{rgb}{0.678431,1.000000,0.184314}%
\pgfsetfillcolor{currentfill}%
\pgfsetfillopacity{0.500000}%
\pgfsetlinewidth{0.250937pt}%
\definecolor{currentstroke}{rgb}{0.000000,0.000000,0.000000}%
\pgfsetstrokecolor{currentstroke}%
\pgfsetstrokeopacity{0.500000}%
\pgfsetdash{}{0pt}%
\pgfsys@defobject{currentmarker}{\pgfqpoint{-0.063194in}{-0.063194in}}{\pgfqpoint{0.063194in}{0.063194in}}{%
\pgfpathmoveto{\pgfqpoint{0.000000in}{-0.063194in}}%
\pgfpathcurveto{\pgfqpoint{0.016759in}{-0.063194in}}{\pgfqpoint{0.032835in}{-0.056536in}}{\pgfqpoint{0.044685in}{-0.044685in}}%
\pgfpathcurveto{\pgfqpoint{0.056536in}{-0.032835in}}{\pgfqpoint{0.063194in}{-0.016759in}}{\pgfqpoint{0.063194in}{0.000000in}}%
\pgfpathcurveto{\pgfqpoint{0.063194in}{0.016759in}}{\pgfqpoint{0.056536in}{0.032835in}}{\pgfqpoint{0.044685in}{0.044685in}}%
\pgfpathcurveto{\pgfqpoint{0.032835in}{0.056536in}}{\pgfqpoint{0.016759in}{0.063194in}}{\pgfqpoint{0.000000in}{0.063194in}}%
\pgfpathcurveto{\pgfqpoint{-0.016759in}{0.063194in}}{\pgfqpoint{-0.032835in}{0.056536in}}{\pgfqpoint{-0.044685in}{0.044685in}}%
\pgfpathcurveto{\pgfqpoint{-0.056536in}{0.032835in}}{\pgfqpoint{-0.063194in}{0.016759in}}{\pgfqpoint{-0.063194in}{0.000000in}}%
\pgfpathcurveto{\pgfqpoint{-0.063194in}{-0.016759in}}{\pgfqpoint{-0.056536in}{-0.032835in}}{\pgfqpoint{-0.044685in}{-0.044685in}}%
\pgfpathcurveto{\pgfqpoint{-0.032835in}{-0.056536in}}{\pgfqpoint{-0.016759in}{-0.063194in}}{\pgfqpoint{0.000000in}{-0.063194in}}%
\pgfpathclose%
\pgfusepath{stroke,fill}%
}%
\begin{pgfscope}%
\pgfsys@transformshift{4.396545in}{1.825584in}%
\pgfsys@useobject{currentmarker}{}%
\end{pgfscope}%
\end{pgfscope}%
\begin{pgfscope}%
\pgfpathrectangle{\pgfqpoint{0.100000in}{0.100000in}}{\pgfqpoint{5.307240in}{3.397500in}}%
\pgfusepath{clip}%
\pgfsetrectcap%
\pgfsetroundjoin%
\pgfsetlinewidth{1.505625pt}%
\definecolor{currentstroke}{rgb}{0.678431,1.000000,0.184314}%
\pgfsetstrokecolor{currentstroke}%
\pgfsetstrokeopacity{0.500000}%
\pgfsetdash{}{0pt}%
\pgfpathmoveto{\pgfqpoint{4.556488in}{1.947594in}}%
\pgfusepath{stroke}%
\end{pgfscope}%
\begin{pgfscope}%
\pgfpathrectangle{\pgfqpoint{0.100000in}{0.100000in}}{\pgfqpoint{5.307240in}{3.397500in}}%
\pgfusepath{clip}%
\pgfsetbuttcap%
\pgfsetroundjoin%
\definecolor{currentfill}{rgb}{0.678431,1.000000,0.184314}%
\pgfsetfillcolor{currentfill}%
\pgfsetfillopacity{0.500000}%
\pgfsetlinewidth{0.250937pt}%
\definecolor{currentstroke}{rgb}{0.000000,0.000000,0.000000}%
\pgfsetstrokecolor{currentstroke}%
\pgfsetstrokeopacity{0.500000}%
\pgfsetdash{}{0pt}%
\pgfsys@defobject{currentmarker}{\pgfqpoint{-0.052083in}{-0.052083in}}{\pgfqpoint{0.052083in}{0.052083in}}{%
\pgfpathmoveto{\pgfqpoint{0.000000in}{-0.052083in}}%
\pgfpathcurveto{\pgfqpoint{0.013813in}{-0.052083in}}{\pgfqpoint{0.027061in}{-0.046596in}}{\pgfqpoint{0.036828in}{-0.036828in}}%
\pgfpathcurveto{\pgfqpoint{0.046596in}{-0.027061in}}{\pgfqpoint{0.052083in}{-0.013813in}}{\pgfqpoint{0.052083in}{0.000000in}}%
\pgfpathcurveto{\pgfqpoint{0.052083in}{0.013813in}}{\pgfqpoint{0.046596in}{0.027061in}}{\pgfqpoint{0.036828in}{0.036828in}}%
\pgfpathcurveto{\pgfqpoint{0.027061in}{0.046596in}}{\pgfqpoint{0.013813in}{0.052083in}}{\pgfqpoint{0.000000in}{0.052083in}}%
\pgfpathcurveto{\pgfqpoint{-0.013813in}{0.052083in}}{\pgfqpoint{-0.027061in}{0.046596in}}{\pgfqpoint{-0.036828in}{0.036828in}}%
\pgfpathcurveto{\pgfqpoint{-0.046596in}{0.027061in}}{\pgfqpoint{-0.052083in}{0.013813in}}{\pgfqpoint{-0.052083in}{0.000000in}}%
\pgfpathcurveto{\pgfqpoint{-0.052083in}{-0.013813in}}{\pgfqpoint{-0.046596in}{-0.027061in}}{\pgfqpoint{-0.036828in}{-0.036828in}}%
\pgfpathcurveto{\pgfqpoint{-0.027061in}{-0.046596in}}{\pgfqpoint{-0.013813in}{-0.052083in}}{\pgfqpoint{0.000000in}{-0.052083in}}%
\pgfpathclose%
\pgfusepath{stroke,fill}%
}%
\begin{pgfscope}%
\pgfsys@transformshift{4.556488in}{1.947594in}%
\pgfsys@useobject{currentmarker}{}%
\end{pgfscope}%
\end{pgfscope}%
\begin{pgfscope}%
\pgfpathrectangle{\pgfqpoint{0.100000in}{0.100000in}}{\pgfqpoint{5.307240in}{3.397500in}}%
\pgfusepath{clip}%
\pgfsetrectcap%
\pgfsetroundjoin%
\pgfsetlinewidth{1.505625pt}%
\definecolor{currentstroke}{rgb}{0.678431,1.000000,0.184314}%
\pgfsetstrokecolor{currentstroke}%
\pgfsetstrokeopacity{0.500000}%
\pgfsetdash{}{0pt}%
\pgfpathmoveto{\pgfqpoint{4.512523in}{1.989241in}}%
\pgfusepath{stroke}%
\end{pgfscope}%
\begin{pgfscope}%
\pgfpathrectangle{\pgfqpoint{0.100000in}{0.100000in}}{\pgfqpoint{5.307240in}{3.397500in}}%
\pgfusepath{clip}%
\pgfsetbuttcap%
\pgfsetroundjoin%
\definecolor{currentfill}{rgb}{0.678431,1.000000,0.184314}%
\pgfsetfillcolor{currentfill}%
\pgfsetfillopacity{0.500000}%
\pgfsetlinewidth{0.250937pt}%
\definecolor{currentstroke}{rgb}{0.000000,0.000000,0.000000}%
\pgfsetstrokecolor{currentstroke}%
\pgfsetstrokeopacity{0.500000}%
\pgfsetdash{}{0pt}%
\pgfsys@defobject{currentmarker}{\pgfqpoint{-0.054861in}{-0.054861in}}{\pgfqpoint{0.054861in}{0.054861in}}{%
\pgfpathmoveto{\pgfqpoint{0.000000in}{-0.054861in}}%
\pgfpathcurveto{\pgfqpoint{0.014549in}{-0.054861in}}{\pgfqpoint{0.028505in}{-0.049081in}}{\pgfqpoint{0.038793in}{-0.038793in}}%
\pgfpathcurveto{\pgfqpoint{0.049081in}{-0.028505in}}{\pgfqpoint{0.054861in}{-0.014549in}}{\pgfqpoint{0.054861in}{0.000000in}}%
\pgfpathcurveto{\pgfqpoint{0.054861in}{0.014549in}}{\pgfqpoint{0.049081in}{0.028505in}}{\pgfqpoint{0.038793in}{0.038793in}}%
\pgfpathcurveto{\pgfqpoint{0.028505in}{0.049081in}}{\pgfqpoint{0.014549in}{0.054861in}}{\pgfqpoint{0.000000in}{0.054861in}}%
\pgfpathcurveto{\pgfqpoint{-0.014549in}{0.054861in}}{\pgfqpoint{-0.028505in}{0.049081in}}{\pgfqpoint{-0.038793in}{0.038793in}}%
\pgfpathcurveto{\pgfqpoint{-0.049081in}{0.028505in}}{\pgfqpoint{-0.054861in}{0.014549in}}{\pgfqpoint{-0.054861in}{0.000000in}}%
\pgfpathcurveto{\pgfqpoint{-0.054861in}{-0.014549in}}{\pgfqpoint{-0.049081in}{-0.028505in}}{\pgfqpoint{-0.038793in}{-0.038793in}}%
\pgfpathcurveto{\pgfqpoint{-0.028505in}{-0.049081in}}{\pgfqpoint{-0.014549in}{-0.054861in}}{\pgfqpoint{0.000000in}{-0.054861in}}%
\pgfpathclose%
\pgfusepath{stroke,fill}%
}%
\begin{pgfscope}%
\pgfsys@transformshift{4.512523in}{1.989241in}%
\pgfsys@useobject{currentmarker}{}%
\end{pgfscope}%
\end{pgfscope}%
\begin{pgfscope}%
\pgfpathrectangle{\pgfqpoint{0.100000in}{0.100000in}}{\pgfqpoint{5.307240in}{3.397500in}}%
\pgfusepath{clip}%
\pgfsetrectcap%
\pgfsetroundjoin%
\pgfsetlinewidth{1.505625pt}%
\definecolor{currentstroke}{rgb}{0.678431,1.000000,0.184314}%
\pgfsetstrokecolor{currentstroke}%
\pgfsetstrokeopacity{0.500000}%
\pgfsetdash{}{0pt}%
\pgfpathmoveto{\pgfqpoint{4.505340in}{1.864149in}}%
\pgfusepath{stroke}%
\end{pgfscope}%
\begin{pgfscope}%
\pgfpathrectangle{\pgfqpoint{0.100000in}{0.100000in}}{\pgfqpoint{5.307240in}{3.397500in}}%
\pgfusepath{clip}%
\pgfsetbuttcap%
\pgfsetroundjoin%
\definecolor{currentfill}{rgb}{0.678431,1.000000,0.184314}%
\pgfsetfillcolor{currentfill}%
\pgfsetfillopacity{0.500000}%
\pgfsetlinewidth{0.250937pt}%
\definecolor{currentstroke}{rgb}{0.000000,0.000000,0.000000}%
\pgfsetstrokecolor{currentstroke}%
\pgfsetstrokeopacity{0.500000}%
\pgfsetdash{}{0pt}%
\pgfsys@defobject{currentmarker}{\pgfqpoint{-0.054861in}{-0.054861in}}{\pgfqpoint{0.054861in}{0.054861in}}{%
\pgfpathmoveto{\pgfqpoint{0.000000in}{-0.054861in}}%
\pgfpathcurveto{\pgfqpoint{0.014549in}{-0.054861in}}{\pgfqpoint{0.028505in}{-0.049081in}}{\pgfqpoint{0.038793in}{-0.038793in}}%
\pgfpathcurveto{\pgfqpoint{0.049081in}{-0.028505in}}{\pgfqpoint{0.054861in}{-0.014549in}}{\pgfqpoint{0.054861in}{0.000000in}}%
\pgfpathcurveto{\pgfqpoint{0.054861in}{0.014549in}}{\pgfqpoint{0.049081in}{0.028505in}}{\pgfqpoint{0.038793in}{0.038793in}}%
\pgfpathcurveto{\pgfqpoint{0.028505in}{0.049081in}}{\pgfqpoint{0.014549in}{0.054861in}}{\pgfqpoint{0.000000in}{0.054861in}}%
\pgfpathcurveto{\pgfqpoint{-0.014549in}{0.054861in}}{\pgfqpoint{-0.028505in}{0.049081in}}{\pgfqpoint{-0.038793in}{0.038793in}}%
\pgfpathcurveto{\pgfqpoint{-0.049081in}{0.028505in}}{\pgfqpoint{-0.054861in}{0.014549in}}{\pgfqpoint{-0.054861in}{0.000000in}}%
\pgfpathcurveto{\pgfqpoint{-0.054861in}{-0.014549in}}{\pgfqpoint{-0.049081in}{-0.028505in}}{\pgfqpoint{-0.038793in}{-0.038793in}}%
\pgfpathcurveto{\pgfqpoint{-0.028505in}{-0.049081in}}{\pgfqpoint{-0.014549in}{-0.054861in}}{\pgfqpoint{0.000000in}{-0.054861in}}%
\pgfpathclose%
\pgfusepath{stroke,fill}%
}%
\begin{pgfscope}%
\pgfsys@transformshift{4.505340in}{1.864149in}%
\pgfsys@useobject{currentmarker}{}%
\end{pgfscope}%
\end{pgfscope}%
\begin{pgfscope}%
\pgfpathrectangle{\pgfqpoint{0.100000in}{0.100000in}}{\pgfqpoint{5.307240in}{3.397500in}}%
\pgfusepath{clip}%
\pgfsetrectcap%
\pgfsetroundjoin%
\pgfsetlinewidth{1.505625pt}%
\definecolor{currentstroke}{rgb}{0.678431,1.000000,0.184314}%
\pgfsetstrokecolor{currentstroke}%
\pgfsetstrokeopacity{0.500000}%
\pgfsetdash{}{0pt}%
\pgfpathmoveto{\pgfqpoint{4.661244in}{1.909388in}}%
\pgfusepath{stroke}%
\end{pgfscope}%
\begin{pgfscope}%
\pgfpathrectangle{\pgfqpoint{0.100000in}{0.100000in}}{\pgfqpoint{5.307240in}{3.397500in}}%
\pgfusepath{clip}%
\pgfsetbuttcap%
\pgfsetroundjoin%
\definecolor{currentfill}{rgb}{0.678431,1.000000,0.184314}%
\pgfsetfillcolor{currentfill}%
\pgfsetfillopacity{0.500000}%
\pgfsetlinewidth{0.250937pt}%
\definecolor{currentstroke}{rgb}{0.000000,0.000000,0.000000}%
\pgfsetstrokecolor{currentstroke}%
\pgfsetstrokeopacity{0.500000}%
\pgfsetdash{}{0pt}%
\pgfsys@defobject{currentmarker}{\pgfqpoint{-0.060417in}{-0.060417in}}{\pgfqpoint{0.060417in}{0.060417in}}{%
\pgfpathmoveto{\pgfqpoint{0.000000in}{-0.060417in}}%
\pgfpathcurveto{\pgfqpoint{0.016023in}{-0.060417in}}{\pgfqpoint{0.031391in}{-0.054051in}}{\pgfqpoint{0.042721in}{-0.042721in}}%
\pgfpathcurveto{\pgfqpoint{0.054051in}{-0.031391in}}{\pgfqpoint{0.060417in}{-0.016023in}}{\pgfqpoint{0.060417in}{0.000000in}}%
\pgfpathcurveto{\pgfqpoint{0.060417in}{0.016023in}}{\pgfqpoint{0.054051in}{0.031391in}}{\pgfqpoint{0.042721in}{0.042721in}}%
\pgfpathcurveto{\pgfqpoint{0.031391in}{0.054051in}}{\pgfqpoint{0.016023in}{0.060417in}}{\pgfqpoint{0.000000in}{0.060417in}}%
\pgfpathcurveto{\pgfqpoint{-0.016023in}{0.060417in}}{\pgfqpoint{-0.031391in}{0.054051in}}{\pgfqpoint{-0.042721in}{0.042721in}}%
\pgfpathcurveto{\pgfqpoint{-0.054051in}{0.031391in}}{\pgfqpoint{-0.060417in}{0.016023in}}{\pgfqpoint{-0.060417in}{0.000000in}}%
\pgfpathcurveto{\pgfqpoint{-0.060417in}{-0.016023in}}{\pgfqpoint{-0.054051in}{-0.031391in}}{\pgfqpoint{-0.042721in}{-0.042721in}}%
\pgfpathcurveto{\pgfqpoint{-0.031391in}{-0.054051in}}{\pgfqpoint{-0.016023in}{-0.060417in}}{\pgfqpoint{0.000000in}{-0.060417in}}%
\pgfpathclose%
\pgfusepath{stroke,fill}%
}%
\begin{pgfscope}%
\pgfsys@transformshift{4.661244in}{1.909388in}%
\pgfsys@useobject{currentmarker}{}%
\end{pgfscope}%
\end{pgfscope}%
\begin{pgfscope}%
\pgfpathrectangle{\pgfqpoint{0.100000in}{0.100000in}}{\pgfqpoint{5.307240in}{3.397500in}}%
\pgfusepath{clip}%
\pgfsetrectcap%
\pgfsetroundjoin%
\pgfsetlinewidth{1.505625pt}%
\definecolor{currentstroke}{rgb}{0.678431,1.000000,0.184314}%
\pgfsetstrokecolor{currentstroke}%
\pgfsetstrokeopacity{0.500000}%
\pgfsetdash{}{0pt}%
\pgfpathmoveto{\pgfqpoint{4.438906in}{1.837425in}}%
\pgfusepath{stroke}%
\end{pgfscope}%
\begin{pgfscope}%
\pgfpathrectangle{\pgfqpoint{0.100000in}{0.100000in}}{\pgfqpoint{5.307240in}{3.397500in}}%
\pgfusepath{clip}%
\pgfsetbuttcap%
\pgfsetroundjoin%
\definecolor{currentfill}{rgb}{0.678431,1.000000,0.184314}%
\pgfsetfillcolor{currentfill}%
\pgfsetfillopacity{0.500000}%
\pgfsetlinewidth{0.250937pt}%
\definecolor{currentstroke}{rgb}{0.000000,0.000000,0.000000}%
\pgfsetstrokecolor{currentstroke}%
\pgfsetstrokeopacity{0.500000}%
\pgfsetdash{}{0pt}%
\pgfsys@defobject{currentmarker}{\pgfqpoint{-0.057639in}{-0.057639in}}{\pgfqpoint{0.057639in}{0.057639in}}{%
\pgfpathmoveto{\pgfqpoint{0.000000in}{-0.057639in}}%
\pgfpathcurveto{\pgfqpoint{0.015286in}{-0.057639in}}{\pgfqpoint{0.029948in}{-0.051566in}}{\pgfqpoint{0.040757in}{-0.040757in}}%
\pgfpathcurveto{\pgfqpoint{0.051566in}{-0.029948in}}{\pgfqpoint{0.057639in}{-0.015286in}}{\pgfqpoint{0.057639in}{0.000000in}}%
\pgfpathcurveto{\pgfqpoint{0.057639in}{0.015286in}}{\pgfqpoint{0.051566in}{0.029948in}}{\pgfqpoint{0.040757in}{0.040757in}}%
\pgfpathcurveto{\pgfqpoint{0.029948in}{0.051566in}}{\pgfqpoint{0.015286in}{0.057639in}}{\pgfqpoint{0.000000in}{0.057639in}}%
\pgfpathcurveto{\pgfqpoint{-0.015286in}{0.057639in}}{\pgfqpoint{-0.029948in}{0.051566in}}{\pgfqpoint{-0.040757in}{0.040757in}}%
\pgfpathcurveto{\pgfqpoint{-0.051566in}{0.029948in}}{\pgfqpoint{-0.057639in}{0.015286in}}{\pgfqpoint{-0.057639in}{0.000000in}}%
\pgfpathcurveto{\pgfqpoint{-0.057639in}{-0.015286in}}{\pgfqpoint{-0.051566in}{-0.029948in}}{\pgfqpoint{-0.040757in}{-0.040757in}}%
\pgfpathcurveto{\pgfqpoint{-0.029948in}{-0.051566in}}{\pgfqpoint{-0.015286in}{-0.057639in}}{\pgfqpoint{0.000000in}{-0.057639in}}%
\pgfpathclose%
\pgfusepath{stroke,fill}%
}%
\begin{pgfscope}%
\pgfsys@transformshift{4.438906in}{1.837425in}%
\pgfsys@useobject{currentmarker}{}%
\end{pgfscope}%
\end{pgfscope}%
\begin{pgfscope}%
\pgfpathrectangle{\pgfqpoint{0.100000in}{0.100000in}}{\pgfqpoint{5.307240in}{3.397500in}}%
\pgfusepath{clip}%
\pgfsetrectcap%
\pgfsetroundjoin%
\pgfsetlinewidth{1.505625pt}%
\definecolor{currentstroke}{rgb}{0.678431,1.000000,0.184314}%
\pgfsetstrokecolor{currentstroke}%
\pgfsetstrokeopacity{0.500000}%
\pgfsetdash{}{0pt}%
\pgfpathmoveto{\pgfqpoint{4.500344in}{1.951674in}}%
\pgfusepath{stroke}%
\end{pgfscope}%
\begin{pgfscope}%
\pgfpathrectangle{\pgfqpoint{0.100000in}{0.100000in}}{\pgfqpoint{5.307240in}{3.397500in}}%
\pgfusepath{clip}%
\pgfsetbuttcap%
\pgfsetroundjoin%
\definecolor{currentfill}{rgb}{0.678431,1.000000,0.184314}%
\pgfsetfillcolor{currentfill}%
\pgfsetfillopacity{0.500000}%
\pgfsetlinewidth{0.250937pt}%
\definecolor{currentstroke}{rgb}{0.000000,0.000000,0.000000}%
\pgfsetstrokecolor{currentstroke}%
\pgfsetstrokeopacity{0.500000}%
\pgfsetdash{}{0pt}%
\pgfsys@defobject{currentmarker}{\pgfqpoint{-0.050694in}{-0.050694in}}{\pgfqpoint{0.050694in}{0.050694in}}{%
\pgfpathmoveto{\pgfqpoint{0.000000in}{-0.050694in}}%
\pgfpathcurveto{\pgfqpoint{0.013444in}{-0.050694in}}{\pgfqpoint{0.026340in}{-0.045353in}}{\pgfqpoint{0.035846in}{-0.035846in}}%
\pgfpathcurveto{\pgfqpoint{0.045353in}{-0.026340in}}{\pgfqpoint{0.050694in}{-0.013444in}}{\pgfqpoint{0.050694in}{0.000000in}}%
\pgfpathcurveto{\pgfqpoint{0.050694in}{0.013444in}}{\pgfqpoint{0.045353in}{0.026340in}}{\pgfqpoint{0.035846in}{0.035846in}}%
\pgfpathcurveto{\pgfqpoint{0.026340in}{0.045353in}}{\pgfqpoint{0.013444in}{0.050694in}}{\pgfqpoint{0.000000in}{0.050694in}}%
\pgfpathcurveto{\pgfqpoint{-0.013444in}{0.050694in}}{\pgfqpoint{-0.026340in}{0.045353in}}{\pgfqpoint{-0.035846in}{0.035846in}}%
\pgfpathcurveto{\pgfqpoint{-0.045353in}{0.026340in}}{\pgfqpoint{-0.050694in}{0.013444in}}{\pgfqpoint{-0.050694in}{0.000000in}}%
\pgfpathcurveto{\pgfqpoint{-0.050694in}{-0.013444in}}{\pgfqpoint{-0.045353in}{-0.026340in}}{\pgfqpoint{-0.035846in}{-0.035846in}}%
\pgfpathcurveto{\pgfqpoint{-0.026340in}{-0.045353in}}{\pgfqpoint{-0.013444in}{-0.050694in}}{\pgfqpoint{0.000000in}{-0.050694in}}%
\pgfpathclose%
\pgfusepath{stroke,fill}%
}%
\begin{pgfscope}%
\pgfsys@transformshift{4.500344in}{1.951674in}%
\pgfsys@useobject{currentmarker}{}%
\end{pgfscope}%
\end{pgfscope}%
\begin{pgfscope}%
\pgfpathrectangle{\pgfqpoint{0.100000in}{0.100000in}}{\pgfqpoint{5.307240in}{3.397500in}}%
\pgfusepath{clip}%
\pgfsetrectcap%
\pgfsetroundjoin%
\pgfsetlinewidth{1.505625pt}%
\definecolor{currentstroke}{rgb}{0.678431,1.000000,0.184314}%
\pgfsetstrokecolor{currentstroke}%
\pgfsetstrokeopacity{0.500000}%
\pgfsetdash{}{0pt}%
\pgfpathmoveto{\pgfqpoint{4.809439in}{1.858254in}}%
\pgfusepath{stroke}%
\end{pgfscope}%
\begin{pgfscope}%
\pgfpathrectangle{\pgfqpoint{0.100000in}{0.100000in}}{\pgfqpoint{5.307240in}{3.397500in}}%
\pgfusepath{clip}%
\pgfsetbuttcap%
\pgfsetroundjoin%
\definecolor{currentfill}{rgb}{0.678431,1.000000,0.184314}%
\pgfsetfillcolor{currentfill}%
\pgfsetfillopacity{0.500000}%
\pgfsetlinewidth{0.250937pt}%
\definecolor{currentstroke}{rgb}{0.000000,0.000000,0.000000}%
\pgfsetstrokecolor{currentstroke}%
\pgfsetstrokeopacity{0.500000}%
\pgfsetdash{}{0pt}%
\pgfsys@defobject{currentmarker}{\pgfqpoint{-0.065278in}{-0.065278in}}{\pgfqpoint{0.065278in}{0.065278in}}{%
\pgfpathmoveto{\pgfqpoint{0.000000in}{-0.065278in}}%
\pgfpathcurveto{\pgfqpoint{0.017312in}{-0.065278in}}{\pgfqpoint{0.033917in}{-0.058400in}}{\pgfqpoint{0.046158in}{-0.046158in}}%
\pgfpathcurveto{\pgfqpoint{0.058400in}{-0.033917in}}{\pgfqpoint{0.065278in}{-0.017312in}}{\pgfqpoint{0.065278in}{0.000000in}}%
\pgfpathcurveto{\pgfqpoint{0.065278in}{0.017312in}}{\pgfqpoint{0.058400in}{0.033917in}}{\pgfqpoint{0.046158in}{0.046158in}}%
\pgfpathcurveto{\pgfqpoint{0.033917in}{0.058400in}}{\pgfqpoint{0.017312in}{0.065278in}}{\pgfqpoint{0.000000in}{0.065278in}}%
\pgfpathcurveto{\pgfqpoint{-0.017312in}{0.065278in}}{\pgfqpoint{-0.033917in}{0.058400in}}{\pgfqpoint{-0.046158in}{0.046158in}}%
\pgfpathcurveto{\pgfqpoint{-0.058400in}{0.033917in}}{\pgfqpoint{-0.065278in}{0.017312in}}{\pgfqpoint{-0.065278in}{0.000000in}}%
\pgfpathcurveto{\pgfqpoint{-0.065278in}{-0.017312in}}{\pgfqpoint{-0.058400in}{-0.033917in}}{\pgfqpoint{-0.046158in}{-0.046158in}}%
\pgfpathcurveto{\pgfqpoint{-0.033917in}{-0.058400in}}{\pgfqpoint{-0.017312in}{-0.065278in}}{\pgfqpoint{0.000000in}{-0.065278in}}%
\pgfpathclose%
\pgfusepath{stroke,fill}%
}%
\begin{pgfscope}%
\pgfsys@transformshift{4.809439in}{1.858254in}%
\pgfsys@useobject{currentmarker}{}%
\end{pgfscope}%
\end{pgfscope}%
\begin{pgfscope}%
\pgfpathrectangle{\pgfqpoint{0.100000in}{0.100000in}}{\pgfqpoint{5.307240in}{3.397500in}}%
\pgfusepath{clip}%
\pgfsetrectcap%
\pgfsetroundjoin%
\pgfsetlinewidth{1.505625pt}%
\definecolor{currentstroke}{rgb}{0.678431,1.000000,0.184314}%
\pgfsetstrokecolor{currentstroke}%
\pgfsetstrokeopacity{0.500000}%
\pgfsetdash{}{0pt}%
\pgfpathmoveto{\pgfqpoint{4.559780in}{2.084882in}}%
\pgfusepath{stroke}%
\end{pgfscope}%
\begin{pgfscope}%
\pgfpathrectangle{\pgfqpoint{0.100000in}{0.100000in}}{\pgfqpoint{5.307240in}{3.397500in}}%
\pgfusepath{clip}%
\pgfsetbuttcap%
\pgfsetroundjoin%
\definecolor{currentfill}{rgb}{0.678431,1.000000,0.184314}%
\pgfsetfillcolor{currentfill}%
\pgfsetfillopacity{0.500000}%
\pgfsetlinewidth{0.250937pt}%
\definecolor{currentstroke}{rgb}{0.000000,0.000000,0.000000}%
\pgfsetstrokecolor{currentstroke}%
\pgfsetstrokeopacity{0.500000}%
\pgfsetdash{}{0pt}%
\pgfsys@defobject{currentmarker}{\pgfqpoint{-0.052083in}{-0.052083in}}{\pgfqpoint{0.052083in}{0.052083in}}{%
\pgfpathmoveto{\pgfqpoint{0.000000in}{-0.052083in}}%
\pgfpathcurveto{\pgfqpoint{0.013813in}{-0.052083in}}{\pgfqpoint{0.027061in}{-0.046596in}}{\pgfqpoint{0.036828in}{-0.036828in}}%
\pgfpathcurveto{\pgfqpoint{0.046596in}{-0.027061in}}{\pgfqpoint{0.052083in}{-0.013813in}}{\pgfqpoint{0.052083in}{0.000000in}}%
\pgfpathcurveto{\pgfqpoint{0.052083in}{0.013813in}}{\pgfqpoint{0.046596in}{0.027061in}}{\pgfqpoint{0.036828in}{0.036828in}}%
\pgfpathcurveto{\pgfqpoint{0.027061in}{0.046596in}}{\pgfqpoint{0.013813in}{0.052083in}}{\pgfqpoint{0.000000in}{0.052083in}}%
\pgfpathcurveto{\pgfqpoint{-0.013813in}{0.052083in}}{\pgfqpoint{-0.027061in}{0.046596in}}{\pgfqpoint{-0.036828in}{0.036828in}}%
\pgfpathcurveto{\pgfqpoint{-0.046596in}{0.027061in}}{\pgfqpoint{-0.052083in}{0.013813in}}{\pgfqpoint{-0.052083in}{0.000000in}}%
\pgfpathcurveto{\pgfqpoint{-0.052083in}{-0.013813in}}{\pgfqpoint{-0.046596in}{-0.027061in}}{\pgfqpoint{-0.036828in}{-0.036828in}}%
\pgfpathcurveto{\pgfqpoint{-0.027061in}{-0.046596in}}{\pgfqpoint{-0.013813in}{-0.052083in}}{\pgfqpoint{0.000000in}{-0.052083in}}%
\pgfpathclose%
\pgfusepath{stroke,fill}%
}%
\begin{pgfscope}%
\pgfsys@transformshift{4.559780in}{2.084882in}%
\pgfsys@useobject{currentmarker}{}%
\end{pgfscope}%
\end{pgfscope}%
\begin{pgfscope}%
\pgfpathrectangle{\pgfqpoint{0.100000in}{0.100000in}}{\pgfqpoint{5.307240in}{3.397500in}}%
\pgfusepath{clip}%
\pgfsetrectcap%
\pgfsetroundjoin%
\pgfsetlinewidth{1.505625pt}%
\definecolor{currentstroke}{rgb}{0.678431,1.000000,0.184314}%
\pgfsetstrokecolor{currentstroke}%
\pgfsetstrokeopacity{0.500000}%
\pgfsetdash{}{0pt}%
\pgfpathmoveto{\pgfqpoint{0.945998in}{3.383549in}}%
\pgfusepath{stroke}%
\end{pgfscope}%
\begin{pgfscope}%
\pgfpathrectangle{\pgfqpoint{0.100000in}{0.100000in}}{\pgfqpoint{5.307240in}{3.397500in}}%
\pgfusepath{clip}%
\pgfsetbuttcap%
\pgfsetroundjoin%
\definecolor{currentfill}{rgb}{0.678431,1.000000,0.184314}%
\pgfsetfillcolor{currentfill}%
\pgfsetfillopacity{0.500000}%
\pgfsetlinewidth{0.250937pt}%
\definecolor{currentstroke}{rgb}{0.000000,0.000000,0.000000}%
\pgfsetstrokecolor{currentstroke}%
\pgfsetstrokeopacity{0.500000}%
\pgfsetdash{}{0pt}%
\pgfsys@defobject{currentmarker}{\pgfqpoint{-0.085417in}{-0.085417in}}{\pgfqpoint{0.085417in}{0.085417in}}{%
\pgfpathmoveto{\pgfqpoint{0.000000in}{-0.085417in}}%
\pgfpathcurveto{\pgfqpoint{0.022653in}{-0.085417in}}{\pgfqpoint{0.044381in}{-0.076417in}}{\pgfqpoint{0.060399in}{-0.060399in}}%
\pgfpathcurveto{\pgfqpoint{0.076417in}{-0.044381in}}{\pgfqpoint{0.085417in}{-0.022653in}}{\pgfqpoint{0.085417in}{0.000000in}}%
\pgfpathcurveto{\pgfqpoint{0.085417in}{0.022653in}}{\pgfqpoint{0.076417in}{0.044381in}}{\pgfqpoint{0.060399in}{0.060399in}}%
\pgfpathcurveto{\pgfqpoint{0.044381in}{0.076417in}}{\pgfqpoint{0.022653in}{0.085417in}}{\pgfqpoint{0.000000in}{0.085417in}}%
\pgfpathcurveto{\pgfqpoint{-0.022653in}{0.085417in}}{\pgfqpoint{-0.044381in}{0.076417in}}{\pgfqpoint{-0.060399in}{0.060399in}}%
\pgfpathcurveto{\pgfqpoint{-0.076417in}{0.044381in}}{\pgfqpoint{-0.085417in}{0.022653in}}{\pgfqpoint{-0.085417in}{0.000000in}}%
\pgfpathcurveto{\pgfqpoint{-0.085417in}{-0.022653in}}{\pgfqpoint{-0.076417in}{-0.044381in}}{\pgfqpoint{-0.060399in}{-0.060399in}}%
\pgfpathcurveto{\pgfqpoint{-0.044381in}{-0.076417in}}{\pgfqpoint{-0.022653in}{-0.085417in}}{\pgfqpoint{0.000000in}{-0.085417in}}%
\pgfpathclose%
\pgfusepath{stroke,fill}%
}%
\begin{pgfscope}%
\pgfsys@transformshift{0.945998in}{3.383549in}%
\pgfsys@useobject{currentmarker}{}%
\end{pgfscope}%
\end{pgfscope}%
\begin{pgfscope}%
\pgfpathrectangle{\pgfqpoint{0.100000in}{0.100000in}}{\pgfqpoint{5.307240in}{3.397500in}}%
\pgfusepath{clip}%
\pgfsetrectcap%
\pgfsetroundjoin%
\pgfsetlinewidth{1.505625pt}%
\definecolor{currentstroke}{rgb}{0.678431,1.000000,0.184314}%
\pgfsetstrokecolor{currentstroke}%
\pgfsetstrokeopacity{0.500000}%
\pgfsetdash{}{0pt}%
\pgfpathmoveto{\pgfqpoint{0.882294in}{3.252722in}}%
\pgfusepath{stroke}%
\end{pgfscope}%
\begin{pgfscope}%
\pgfpathrectangle{\pgfqpoint{0.100000in}{0.100000in}}{\pgfqpoint{5.307240in}{3.397500in}}%
\pgfusepath{clip}%
\pgfsetbuttcap%
\pgfsetroundjoin%
\definecolor{currentfill}{rgb}{0.678431,1.000000,0.184314}%
\pgfsetfillcolor{currentfill}%
\pgfsetfillopacity{0.500000}%
\pgfsetlinewidth{0.250937pt}%
\definecolor{currentstroke}{rgb}{0.000000,0.000000,0.000000}%
\pgfsetstrokecolor{currentstroke}%
\pgfsetstrokeopacity{0.500000}%
\pgfsetdash{}{0pt}%
\pgfsys@defobject{currentmarker}{\pgfqpoint{-0.064583in}{-0.064583in}}{\pgfqpoint{0.064583in}{0.064583in}}{%
\pgfpathmoveto{\pgfqpoint{0.000000in}{-0.064583in}}%
\pgfpathcurveto{\pgfqpoint{0.017128in}{-0.064583in}}{\pgfqpoint{0.033556in}{-0.057778in}}{\pgfqpoint{0.045667in}{-0.045667in}}%
\pgfpathcurveto{\pgfqpoint{0.057778in}{-0.033556in}}{\pgfqpoint{0.064583in}{-0.017128in}}{\pgfqpoint{0.064583in}{0.000000in}}%
\pgfpathcurveto{\pgfqpoint{0.064583in}{0.017128in}}{\pgfqpoint{0.057778in}{0.033556in}}{\pgfqpoint{0.045667in}{0.045667in}}%
\pgfpathcurveto{\pgfqpoint{0.033556in}{0.057778in}}{\pgfqpoint{0.017128in}{0.064583in}}{\pgfqpoint{0.000000in}{0.064583in}}%
\pgfpathcurveto{\pgfqpoint{-0.017128in}{0.064583in}}{\pgfqpoint{-0.033556in}{0.057778in}}{\pgfqpoint{-0.045667in}{0.045667in}}%
\pgfpathcurveto{\pgfqpoint{-0.057778in}{0.033556in}}{\pgfqpoint{-0.064583in}{0.017128in}}{\pgfqpoint{-0.064583in}{0.000000in}}%
\pgfpathcurveto{\pgfqpoint{-0.064583in}{-0.017128in}}{\pgfqpoint{-0.057778in}{-0.033556in}}{\pgfqpoint{-0.045667in}{-0.045667in}}%
\pgfpathcurveto{\pgfqpoint{-0.033556in}{-0.057778in}}{\pgfqpoint{-0.017128in}{-0.064583in}}{\pgfqpoint{0.000000in}{-0.064583in}}%
\pgfpathclose%
\pgfusepath{stroke,fill}%
}%
\begin{pgfscope}%
\pgfsys@transformshift{0.882294in}{3.252722in}%
\pgfsys@useobject{currentmarker}{}%
\end{pgfscope}%
\end{pgfscope}%
\begin{pgfscope}%
\pgfpathrectangle{\pgfqpoint{0.100000in}{0.100000in}}{\pgfqpoint{5.307240in}{3.397500in}}%
\pgfusepath{clip}%
\pgfsetrectcap%
\pgfsetroundjoin%
\pgfsetlinewidth{1.505625pt}%
\definecolor{currentstroke}{rgb}{0.678431,1.000000,0.184314}%
\pgfsetstrokecolor{currentstroke}%
\pgfsetstrokeopacity{0.500000}%
\pgfsetdash{}{0pt}%
\pgfpathmoveto{\pgfqpoint{1.117072in}{3.022176in}}%
\pgfusepath{stroke}%
\end{pgfscope}%
\begin{pgfscope}%
\pgfpathrectangle{\pgfqpoint{0.100000in}{0.100000in}}{\pgfqpoint{5.307240in}{3.397500in}}%
\pgfusepath{clip}%
\pgfsetbuttcap%
\pgfsetroundjoin%
\definecolor{currentfill}{rgb}{0.678431,1.000000,0.184314}%
\pgfsetfillcolor{currentfill}%
\pgfsetfillopacity{0.500000}%
\pgfsetlinewidth{0.250937pt}%
\definecolor{currentstroke}{rgb}{0.000000,0.000000,0.000000}%
\pgfsetstrokecolor{currentstroke}%
\pgfsetstrokeopacity{0.500000}%
\pgfsetdash{}{0pt}%
\pgfsys@defobject{currentmarker}{\pgfqpoint{-0.049306in}{-0.049306in}}{\pgfqpoint{0.049306in}{0.049306in}}{%
\pgfpathmoveto{\pgfqpoint{0.000000in}{-0.049306in}}%
\pgfpathcurveto{\pgfqpoint{0.013076in}{-0.049306in}}{\pgfqpoint{0.025618in}{-0.044110in}}{\pgfqpoint{0.034864in}{-0.034864in}}%
\pgfpathcurveto{\pgfqpoint{0.044110in}{-0.025618in}}{\pgfqpoint{0.049306in}{-0.013076in}}{\pgfqpoint{0.049306in}{0.000000in}}%
\pgfpathcurveto{\pgfqpoint{0.049306in}{0.013076in}}{\pgfqpoint{0.044110in}{0.025618in}}{\pgfqpoint{0.034864in}{0.034864in}}%
\pgfpathcurveto{\pgfqpoint{0.025618in}{0.044110in}}{\pgfqpoint{0.013076in}{0.049306in}}{\pgfqpoint{0.000000in}{0.049306in}}%
\pgfpathcurveto{\pgfqpoint{-0.013076in}{0.049306in}}{\pgfqpoint{-0.025618in}{0.044110in}}{\pgfqpoint{-0.034864in}{0.034864in}}%
\pgfpathcurveto{\pgfqpoint{-0.044110in}{0.025618in}}{\pgfqpoint{-0.049306in}{0.013076in}}{\pgfqpoint{-0.049306in}{0.000000in}}%
\pgfpathcurveto{\pgfqpoint{-0.049306in}{-0.013076in}}{\pgfqpoint{-0.044110in}{-0.025618in}}{\pgfqpoint{-0.034864in}{-0.034864in}}%
\pgfpathcurveto{\pgfqpoint{-0.025618in}{-0.044110in}}{\pgfqpoint{-0.013076in}{-0.049306in}}{\pgfqpoint{0.000000in}{-0.049306in}}%
\pgfpathclose%
\pgfusepath{stroke,fill}%
}%
\begin{pgfscope}%
\pgfsys@transformshift{1.117072in}{3.022176in}%
\pgfsys@useobject{currentmarker}{}%
\end{pgfscope}%
\end{pgfscope}%
\begin{pgfscope}%
\pgfpathrectangle{\pgfqpoint{0.100000in}{0.100000in}}{\pgfqpoint{5.307240in}{3.397500in}}%
\pgfusepath{clip}%
\pgfsetrectcap%
\pgfsetroundjoin%
\pgfsetlinewidth{1.505625pt}%
\definecolor{currentstroke}{rgb}{0.678431,1.000000,0.184314}%
\pgfsetstrokecolor{currentstroke}%
\pgfsetstrokeopacity{0.500000}%
\pgfsetdash{}{0pt}%
\pgfpathmoveto{\pgfqpoint{0.817644in}{3.102057in}}%
\pgfusepath{stroke}%
\end{pgfscope}%
\begin{pgfscope}%
\pgfpathrectangle{\pgfqpoint{0.100000in}{0.100000in}}{\pgfqpoint{5.307240in}{3.397500in}}%
\pgfusepath{clip}%
\pgfsetbuttcap%
\pgfsetroundjoin%
\definecolor{currentfill}{rgb}{0.678431,1.000000,0.184314}%
\pgfsetfillcolor{currentfill}%
\pgfsetfillopacity{0.500000}%
\pgfsetlinewidth{0.250937pt}%
\definecolor{currentstroke}{rgb}{0.000000,0.000000,0.000000}%
\pgfsetstrokecolor{currentstroke}%
\pgfsetstrokeopacity{0.500000}%
\pgfsetdash{}{0pt}%
\pgfsys@defobject{currentmarker}{\pgfqpoint{-0.065972in}{-0.065972in}}{\pgfqpoint{0.065972in}{0.065972in}}{%
\pgfpathmoveto{\pgfqpoint{0.000000in}{-0.065972in}}%
\pgfpathcurveto{\pgfqpoint{0.017496in}{-0.065972in}}{\pgfqpoint{0.034278in}{-0.059021in}}{\pgfqpoint{0.046649in}{-0.046649in}}%
\pgfpathcurveto{\pgfqpoint{0.059021in}{-0.034278in}}{\pgfqpoint{0.065972in}{-0.017496in}}{\pgfqpoint{0.065972in}{0.000000in}}%
\pgfpathcurveto{\pgfqpoint{0.065972in}{0.017496in}}{\pgfqpoint{0.059021in}{0.034278in}}{\pgfqpoint{0.046649in}{0.046649in}}%
\pgfpathcurveto{\pgfqpoint{0.034278in}{0.059021in}}{\pgfqpoint{0.017496in}{0.065972in}}{\pgfqpoint{0.000000in}{0.065972in}}%
\pgfpathcurveto{\pgfqpoint{-0.017496in}{0.065972in}}{\pgfqpoint{-0.034278in}{0.059021in}}{\pgfqpoint{-0.046649in}{0.046649in}}%
\pgfpathcurveto{\pgfqpoint{-0.059021in}{0.034278in}}{\pgfqpoint{-0.065972in}{0.017496in}}{\pgfqpoint{-0.065972in}{0.000000in}}%
\pgfpathcurveto{\pgfqpoint{-0.065972in}{-0.017496in}}{\pgfqpoint{-0.059021in}{-0.034278in}}{\pgfqpoint{-0.046649in}{-0.046649in}}%
\pgfpathcurveto{\pgfqpoint{-0.034278in}{-0.059021in}}{\pgfqpoint{-0.017496in}{-0.065972in}}{\pgfqpoint{0.000000in}{-0.065972in}}%
\pgfpathclose%
\pgfusepath{stroke,fill}%
}%
\begin{pgfscope}%
\pgfsys@transformshift{0.817644in}{3.102057in}%
\pgfsys@useobject{currentmarker}{}%
\end{pgfscope}%
\end{pgfscope}%
\begin{pgfscope}%
\pgfpathrectangle{\pgfqpoint{0.100000in}{0.100000in}}{\pgfqpoint{5.307240in}{3.397500in}}%
\pgfusepath{clip}%
\pgfsetrectcap%
\pgfsetroundjoin%
\pgfsetlinewidth{1.505625pt}%
\definecolor{currentstroke}{rgb}{0.678431,1.000000,0.184314}%
\pgfsetstrokecolor{currentstroke}%
\pgfsetstrokeopacity{0.500000}%
\pgfsetdash{}{0pt}%
\pgfpathmoveto{\pgfqpoint{0.945643in}{3.342359in}}%
\pgfusepath{stroke}%
\end{pgfscope}%
\begin{pgfscope}%
\pgfpathrectangle{\pgfqpoint{0.100000in}{0.100000in}}{\pgfqpoint{5.307240in}{3.397500in}}%
\pgfusepath{clip}%
\pgfsetbuttcap%
\pgfsetroundjoin%
\definecolor{currentfill}{rgb}{0.678431,1.000000,0.184314}%
\pgfsetfillcolor{currentfill}%
\pgfsetfillopacity{0.500000}%
\pgfsetlinewidth{0.250937pt}%
\definecolor{currentstroke}{rgb}{0.000000,0.000000,0.000000}%
\pgfsetstrokecolor{currentstroke}%
\pgfsetstrokeopacity{0.500000}%
\pgfsetdash{}{0pt}%
\pgfsys@defobject{currentmarker}{\pgfqpoint{-0.095139in}{-0.095139in}}{\pgfqpoint{0.095139in}{0.095139in}}{%
\pgfpathmoveto{\pgfqpoint{0.000000in}{-0.095139in}}%
\pgfpathcurveto{\pgfqpoint{0.025231in}{-0.095139in}}{\pgfqpoint{0.049432in}{-0.085114in}}{\pgfqpoint{0.067273in}{-0.067273in}}%
\pgfpathcurveto{\pgfqpoint{0.085114in}{-0.049432in}}{\pgfqpoint{0.095139in}{-0.025231in}}{\pgfqpoint{0.095139in}{0.000000in}}%
\pgfpathcurveto{\pgfqpoint{0.095139in}{0.025231in}}{\pgfqpoint{0.085114in}{0.049432in}}{\pgfqpoint{0.067273in}{0.067273in}}%
\pgfpathcurveto{\pgfqpoint{0.049432in}{0.085114in}}{\pgfqpoint{0.025231in}{0.095139in}}{\pgfqpoint{0.000000in}{0.095139in}}%
\pgfpathcurveto{\pgfqpoint{-0.025231in}{0.095139in}}{\pgfqpoint{-0.049432in}{0.085114in}}{\pgfqpoint{-0.067273in}{0.067273in}}%
\pgfpathcurveto{\pgfqpoint{-0.085114in}{0.049432in}}{\pgfqpoint{-0.095139in}{0.025231in}}{\pgfqpoint{-0.095139in}{0.000000in}}%
\pgfpathcurveto{\pgfqpoint{-0.095139in}{-0.025231in}}{\pgfqpoint{-0.085114in}{-0.049432in}}{\pgfqpoint{-0.067273in}{-0.067273in}}%
\pgfpathcurveto{\pgfqpoint{-0.049432in}{-0.085114in}}{\pgfqpoint{-0.025231in}{-0.095139in}}{\pgfqpoint{0.000000in}{-0.095139in}}%
\pgfpathclose%
\pgfusepath{stroke,fill}%
}%
\begin{pgfscope}%
\pgfsys@transformshift{0.945643in}{3.342359in}%
\pgfsys@useobject{currentmarker}{}%
\end{pgfscope}%
\end{pgfscope}%
\begin{pgfscope}%
\pgfpathrectangle{\pgfqpoint{0.100000in}{0.100000in}}{\pgfqpoint{5.307240in}{3.397500in}}%
\pgfusepath{clip}%
\pgfsetrectcap%
\pgfsetroundjoin%
\pgfsetlinewidth{1.505625pt}%
\definecolor{currentstroke}{rgb}{0.678431,1.000000,0.184314}%
\pgfsetstrokecolor{currentstroke}%
\pgfsetstrokeopacity{0.500000}%
\pgfsetdash{}{0pt}%
\pgfpathmoveto{\pgfqpoint{0.854325in}{3.201453in}}%
\pgfusepath{stroke}%
\end{pgfscope}%
\begin{pgfscope}%
\pgfpathrectangle{\pgfqpoint{0.100000in}{0.100000in}}{\pgfqpoint{5.307240in}{3.397500in}}%
\pgfusepath{clip}%
\pgfsetbuttcap%
\pgfsetroundjoin%
\definecolor{currentfill}{rgb}{0.678431,1.000000,0.184314}%
\pgfsetfillcolor{currentfill}%
\pgfsetfillopacity{0.500000}%
\pgfsetlinewidth{0.250937pt}%
\definecolor{currentstroke}{rgb}{0.000000,0.000000,0.000000}%
\pgfsetstrokecolor{currentstroke}%
\pgfsetstrokeopacity{0.500000}%
\pgfsetdash{}{0pt}%
\pgfsys@defobject{currentmarker}{\pgfqpoint{-0.068750in}{-0.068750in}}{\pgfqpoint{0.068750in}{0.068750in}}{%
\pgfpathmoveto{\pgfqpoint{0.000000in}{-0.068750in}}%
\pgfpathcurveto{\pgfqpoint{0.018233in}{-0.068750in}}{\pgfqpoint{0.035721in}{-0.061506in}}{\pgfqpoint{0.048614in}{-0.048614in}}%
\pgfpathcurveto{\pgfqpoint{0.061506in}{-0.035721in}}{\pgfqpoint{0.068750in}{-0.018233in}}{\pgfqpoint{0.068750in}{0.000000in}}%
\pgfpathcurveto{\pgfqpoint{0.068750in}{0.018233in}}{\pgfqpoint{0.061506in}{0.035721in}}{\pgfqpoint{0.048614in}{0.048614in}}%
\pgfpathcurveto{\pgfqpoint{0.035721in}{0.061506in}}{\pgfqpoint{0.018233in}{0.068750in}}{\pgfqpoint{0.000000in}{0.068750in}}%
\pgfpathcurveto{\pgfqpoint{-0.018233in}{0.068750in}}{\pgfqpoint{-0.035721in}{0.061506in}}{\pgfqpoint{-0.048614in}{0.048614in}}%
\pgfpathcurveto{\pgfqpoint{-0.061506in}{0.035721in}}{\pgfqpoint{-0.068750in}{0.018233in}}{\pgfqpoint{-0.068750in}{0.000000in}}%
\pgfpathcurveto{\pgfqpoint{-0.068750in}{-0.018233in}}{\pgfqpoint{-0.061506in}{-0.035721in}}{\pgfqpoint{-0.048614in}{-0.048614in}}%
\pgfpathcurveto{\pgfqpoint{-0.035721in}{-0.061506in}}{\pgfqpoint{-0.018233in}{-0.068750in}}{\pgfqpoint{0.000000in}{-0.068750in}}%
\pgfpathclose%
\pgfusepath{stroke,fill}%
}%
\begin{pgfscope}%
\pgfsys@transformshift{0.854325in}{3.201453in}%
\pgfsys@useobject{currentmarker}{}%
\end{pgfscope}%
\end{pgfscope}%
\begin{pgfscope}%
\pgfpathrectangle{\pgfqpoint{0.100000in}{0.100000in}}{\pgfqpoint{5.307240in}{3.397500in}}%
\pgfusepath{clip}%
\pgfsetrectcap%
\pgfsetroundjoin%
\pgfsetlinewidth{1.505625pt}%
\definecolor{currentstroke}{rgb}{0.678431,1.000000,0.184314}%
\pgfsetstrokecolor{currentstroke}%
\pgfsetstrokeopacity{0.500000}%
\pgfsetdash{}{0pt}%
\pgfpathmoveto{\pgfqpoint{0.916902in}{3.250652in}}%
\pgfusepath{stroke}%
\end{pgfscope}%
\begin{pgfscope}%
\pgfpathrectangle{\pgfqpoint{0.100000in}{0.100000in}}{\pgfqpoint{5.307240in}{3.397500in}}%
\pgfusepath{clip}%
\pgfsetbuttcap%
\pgfsetroundjoin%
\definecolor{currentfill}{rgb}{0.678431,1.000000,0.184314}%
\pgfsetfillcolor{currentfill}%
\pgfsetfillopacity{0.500000}%
\pgfsetlinewidth{0.250937pt}%
\definecolor{currentstroke}{rgb}{0.000000,0.000000,0.000000}%
\pgfsetstrokecolor{currentstroke}%
\pgfsetstrokeopacity{0.500000}%
\pgfsetdash{}{0pt}%
\pgfsys@defobject{currentmarker}{\pgfqpoint{-0.095139in}{-0.095139in}}{\pgfqpoint{0.095139in}{0.095139in}}{%
\pgfpathmoveto{\pgfqpoint{0.000000in}{-0.095139in}}%
\pgfpathcurveto{\pgfqpoint{0.025231in}{-0.095139in}}{\pgfqpoint{0.049432in}{-0.085114in}}{\pgfqpoint{0.067273in}{-0.067273in}}%
\pgfpathcurveto{\pgfqpoint{0.085114in}{-0.049432in}}{\pgfqpoint{0.095139in}{-0.025231in}}{\pgfqpoint{0.095139in}{0.000000in}}%
\pgfpathcurveto{\pgfqpoint{0.095139in}{0.025231in}}{\pgfqpoint{0.085114in}{0.049432in}}{\pgfqpoint{0.067273in}{0.067273in}}%
\pgfpathcurveto{\pgfqpoint{0.049432in}{0.085114in}}{\pgfqpoint{0.025231in}{0.095139in}}{\pgfqpoint{0.000000in}{0.095139in}}%
\pgfpathcurveto{\pgfqpoint{-0.025231in}{0.095139in}}{\pgfqpoint{-0.049432in}{0.085114in}}{\pgfqpoint{-0.067273in}{0.067273in}}%
\pgfpathcurveto{\pgfqpoint{-0.085114in}{0.049432in}}{\pgfqpoint{-0.095139in}{0.025231in}}{\pgfqpoint{-0.095139in}{0.000000in}}%
\pgfpathcurveto{\pgfqpoint{-0.095139in}{-0.025231in}}{\pgfqpoint{-0.085114in}{-0.049432in}}{\pgfqpoint{-0.067273in}{-0.067273in}}%
\pgfpathcurveto{\pgfqpoint{-0.049432in}{-0.085114in}}{\pgfqpoint{-0.025231in}{-0.095139in}}{\pgfqpoint{0.000000in}{-0.095139in}}%
\pgfpathclose%
\pgfusepath{stroke,fill}%
}%
\begin{pgfscope}%
\pgfsys@transformshift{0.916902in}{3.250652in}%
\pgfsys@useobject{currentmarker}{}%
\end{pgfscope}%
\end{pgfscope}%
\begin{pgfscope}%
\pgfpathrectangle{\pgfqpoint{0.100000in}{0.100000in}}{\pgfqpoint{5.307240in}{3.397500in}}%
\pgfusepath{clip}%
\pgfsetrectcap%
\pgfsetroundjoin%
\pgfsetlinewidth{1.505625pt}%
\definecolor{currentstroke}{rgb}{0.678431,1.000000,0.184314}%
\pgfsetstrokecolor{currentstroke}%
\pgfsetstrokeopacity{0.500000}%
\pgfsetdash{}{0pt}%
\pgfpathmoveto{\pgfqpoint{1.292925in}{3.151786in}}%
\pgfusepath{stroke}%
\end{pgfscope}%
\begin{pgfscope}%
\pgfpathrectangle{\pgfqpoint{0.100000in}{0.100000in}}{\pgfqpoint{5.307240in}{3.397500in}}%
\pgfusepath{clip}%
\pgfsetbuttcap%
\pgfsetroundjoin%
\definecolor{currentfill}{rgb}{0.678431,1.000000,0.184314}%
\pgfsetfillcolor{currentfill}%
\pgfsetfillopacity{0.500000}%
\pgfsetlinewidth{0.250937pt}%
\definecolor{currentstroke}{rgb}{0.000000,0.000000,0.000000}%
\pgfsetstrokecolor{currentstroke}%
\pgfsetstrokeopacity{0.500000}%
\pgfsetdash{}{0pt}%
\pgfsys@defobject{currentmarker}{\pgfqpoint{-0.070833in}{-0.070833in}}{\pgfqpoint{0.070833in}{0.070833in}}{%
\pgfpathmoveto{\pgfqpoint{0.000000in}{-0.070833in}}%
\pgfpathcurveto{\pgfqpoint{0.018785in}{-0.070833in}}{\pgfqpoint{0.036804in}{-0.063370in}}{\pgfqpoint{0.050087in}{-0.050087in}}%
\pgfpathcurveto{\pgfqpoint{0.063370in}{-0.036804in}}{\pgfqpoint{0.070833in}{-0.018785in}}{\pgfqpoint{0.070833in}{0.000000in}}%
\pgfpathcurveto{\pgfqpoint{0.070833in}{0.018785in}}{\pgfqpoint{0.063370in}{0.036804in}}{\pgfqpoint{0.050087in}{0.050087in}}%
\pgfpathcurveto{\pgfqpoint{0.036804in}{0.063370in}}{\pgfqpoint{0.018785in}{0.070833in}}{\pgfqpoint{0.000000in}{0.070833in}}%
\pgfpathcurveto{\pgfqpoint{-0.018785in}{0.070833in}}{\pgfqpoint{-0.036804in}{0.063370in}}{\pgfqpoint{-0.050087in}{0.050087in}}%
\pgfpathcurveto{\pgfqpoint{-0.063370in}{0.036804in}}{\pgfqpoint{-0.070833in}{0.018785in}}{\pgfqpoint{-0.070833in}{0.000000in}}%
\pgfpathcurveto{\pgfqpoint{-0.070833in}{-0.018785in}}{\pgfqpoint{-0.063370in}{-0.036804in}}{\pgfqpoint{-0.050087in}{-0.050087in}}%
\pgfpathcurveto{\pgfqpoint{-0.036804in}{-0.063370in}}{\pgfqpoint{-0.018785in}{-0.070833in}}{\pgfqpoint{0.000000in}{-0.070833in}}%
\pgfpathclose%
\pgfusepath{stroke,fill}%
}%
\begin{pgfscope}%
\pgfsys@transformshift{1.292925in}{3.151786in}%
\pgfsys@useobject{currentmarker}{}%
\end{pgfscope}%
\end{pgfscope}%
\begin{pgfscope}%
\pgfpathrectangle{\pgfqpoint{0.100000in}{0.100000in}}{\pgfqpoint{5.307240in}{3.397500in}}%
\pgfusepath{clip}%
\pgfsetrectcap%
\pgfsetroundjoin%
\pgfsetlinewidth{1.505625pt}%
\definecolor{currentstroke}{rgb}{0.678431,1.000000,0.184314}%
\pgfsetstrokecolor{currentstroke}%
\pgfsetstrokeopacity{0.500000}%
\pgfsetdash{}{0pt}%
\pgfpathmoveto{\pgfqpoint{1.175941in}{2.988705in}}%
\pgfusepath{stroke}%
\end{pgfscope}%
\begin{pgfscope}%
\pgfpathrectangle{\pgfqpoint{0.100000in}{0.100000in}}{\pgfqpoint{5.307240in}{3.397500in}}%
\pgfusepath{clip}%
\pgfsetbuttcap%
\pgfsetroundjoin%
\definecolor{currentfill}{rgb}{0.678431,1.000000,0.184314}%
\pgfsetfillcolor{currentfill}%
\pgfsetfillopacity{0.500000}%
\pgfsetlinewidth{0.250937pt}%
\definecolor{currentstroke}{rgb}{0.000000,0.000000,0.000000}%
\pgfsetstrokecolor{currentstroke}%
\pgfsetstrokeopacity{0.500000}%
\pgfsetdash{}{0pt}%
\pgfsys@defobject{currentmarker}{\pgfqpoint{-0.042361in}{-0.042361in}}{\pgfqpoint{0.042361in}{0.042361in}}{%
\pgfpathmoveto{\pgfqpoint{0.000000in}{-0.042361in}}%
\pgfpathcurveto{\pgfqpoint{0.011234in}{-0.042361in}}{\pgfqpoint{0.022010in}{-0.037898in}}{\pgfqpoint{0.029954in}{-0.029954in}}%
\pgfpathcurveto{\pgfqpoint{0.037898in}{-0.022010in}}{\pgfqpoint{0.042361in}{-0.011234in}}{\pgfqpoint{0.042361in}{0.000000in}}%
\pgfpathcurveto{\pgfqpoint{0.042361in}{0.011234in}}{\pgfqpoint{0.037898in}{0.022010in}}{\pgfqpoint{0.029954in}{0.029954in}}%
\pgfpathcurveto{\pgfqpoint{0.022010in}{0.037898in}}{\pgfqpoint{0.011234in}{0.042361in}}{\pgfqpoint{0.000000in}{0.042361in}}%
\pgfpathcurveto{\pgfqpoint{-0.011234in}{0.042361in}}{\pgfqpoint{-0.022010in}{0.037898in}}{\pgfqpoint{-0.029954in}{0.029954in}}%
\pgfpathcurveto{\pgfqpoint{-0.037898in}{0.022010in}}{\pgfqpoint{-0.042361in}{0.011234in}}{\pgfqpoint{-0.042361in}{0.000000in}}%
\pgfpathcurveto{\pgfqpoint{-0.042361in}{-0.011234in}}{\pgfqpoint{-0.037898in}{-0.022010in}}{\pgfqpoint{-0.029954in}{-0.029954in}}%
\pgfpathcurveto{\pgfqpoint{-0.022010in}{-0.037898in}}{\pgfqpoint{-0.011234in}{-0.042361in}}{\pgfqpoint{0.000000in}{-0.042361in}}%
\pgfpathclose%
\pgfusepath{stroke,fill}%
}%
\begin{pgfscope}%
\pgfsys@transformshift{1.175941in}{2.988705in}%
\pgfsys@useobject{currentmarker}{}%
\end{pgfscope}%
\end{pgfscope}%
\begin{pgfscope}%
\pgfpathrectangle{\pgfqpoint{0.100000in}{0.100000in}}{\pgfqpoint{5.307240in}{3.397500in}}%
\pgfusepath{clip}%
\pgfsetrectcap%
\pgfsetroundjoin%
\pgfsetlinewidth{1.505625pt}%
\definecolor{currentstroke}{rgb}{0.678431,1.000000,0.184314}%
\pgfsetstrokecolor{currentstroke}%
\pgfsetstrokeopacity{0.500000}%
\pgfsetdash{}{0pt}%
\pgfpathmoveto{\pgfqpoint{1.064552in}{3.184524in}}%
\pgfusepath{stroke}%
\end{pgfscope}%
\begin{pgfscope}%
\pgfpathrectangle{\pgfqpoint{0.100000in}{0.100000in}}{\pgfqpoint{5.307240in}{3.397500in}}%
\pgfusepath{clip}%
\pgfsetbuttcap%
\pgfsetroundjoin%
\definecolor{currentfill}{rgb}{0.678431,1.000000,0.184314}%
\pgfsetfillcolor{currentfill}%
\pgfsetfillopacity{0.500000}%
\pgfsetlinewidth{0.250937pt}%
\definecolor{currentstroke}{rgb}{0.000000,0.000000,0.000000}%
\pgfsetstrokecolor{currentstroke}%
\pgfsetstrokeopacity{0.500000}%
\pgfsetdash{}{0pt}%
\pgfsys@defobject{currentmarker}{\pgfqpoint{-0.067361in}{-0.067361in}}{\pgfqpoint{0.067361in}{0.067361in}}{%
\pgfpathmoveto{\pgfqpoint{0.000000in}{-0.067361in}}%
\pgfpathcurveto{\pgfqpoint{0.017864in}{-0.067361in}}{\pgfqpoint{0.034999in}{-0.060264in}}{\pgfqpoint{0.047631in}{-0.047631in}}%
\pgfpathcurveto{\pgfqpoint{0.060264in}{-0.034999in}}{\pgfqpoint{0.067361in}{-0.017864in}}{\pgfqpoint{0.067361in}{0.000000in}}%
\pgfpathcurveto{\pgfqpoint{0.067361in}{0.017864in}}{\pgfqpoint{0.060264in}{0.034999in}}{\pgfqpoint{0.047631in}{0.047631in}}%
\pgfpathcurveto{\pgfqpoint{0.034999in}{0.060264in}}{\pgfqpoint{0.017864in}{0.067361in}}{\pgfqpoint{0.000000in}{0.067361in}}%
\pgfpathcurveto{\pgfqpoint{-0.017864in}{0.067361in}}{\pgfqpoint{-0.034999in}{0.060264in}}{\pgfqpoint{-0.047631in}{0.047631in}}%
\pgfpathcurveto{\pgfqpoint{-0.060264in}{0.034999in}}{\pgfqpoint{-0.067361in}{0.017864in}}{\pgfqpoint{-0.067361in}{0.000000in}}%
\pgfpathcurveto{\pgfqpoint{-0.067361in}{-0.017864in}}{\pgfqpoint{-0.060264in}{-0.034999in}}{\pgfqpoint{-0.047631in}{-0.047631in}}%
\pgfpathcurveto{\pgfqpoint{-0.034999in}{-0.060264in}}{\pgfqpoint{-0.017864in}{-0.067361in}}{\pgfqpoint{0.000000in}{-0.067361in}}%
\pgfpathclose%
\pgfusepath{stroke,fill}%
}%
\begin{pgfscope}%
\pgfsys@transformshift{1.064552in}{3.184524in}%
\pgfsys@useobject{currentmarker}{}%
\end{pgfscope}%
\end{pgfscope}%
\begin{pgfscope}%
\pgfpathrectangle{\pgfqpoint{0.100000in}{0.100000in}}{\pgfqpoint{5.307240in}{3.397500in}}%
\pgfusepath{clip}%
\pgfsetrectcap%
\pgfsetroundjoin%
\pgfsetlinewidth{1.505625pt}%
\definecolor{currentstroke}{rgb}{0.678431,1.000000,0.184314}%
\pgfsetstrokecolor{currentstroke}%
\pgfsetstrokeopacity{0.500000}%
\pgfsetdash{}{0pt}%
\pgfpathmoveto{\pgfqpoint{1.022489in}{3.096196in}}%
\pgfusepath{stroke}%
\end{pgfscope}%
\begin{pgfscope}%
\pgfpathrectangle{\pgfqpoint{0.100000in}{0.100000in}}{\pgfqpoint{5.307240in}{3.397500in}}%
\pgfusepath{clip}%
\pgfsetbuttcap%
\pgfsetroundjoin%
\definecolor{currentfill}{rgb}{0.678431,1.000000,0.184314}%
\pgfsetfillcolor{currentfill}%
\pgfsetfillopacity{0.500000}%
\pgfsetlinewidth{0.250937pt}%
\definecolor{currentstroke}{rgb}{0.000000,0.000000,0.000000}%
\pgfsetstrokecolor{currentstroke}%
\pgfsetstrokeopacity{0.500000}%
\pgfsetdash{}{0pt}%
\pgfsys@defobject{currentmarker}{\pgfqpoint{-0.045833in}{-0.045833in}}{\pgfqpoint{0.045833in}{0.045833in}}{%
\pgfpathmoveto{\pgfqpoint{0.000000in}{-0.045833in}}%
\pgfpathcurveto{\pgfqpoint{0.012155in}{-0.045833in}}{\pgfqpoint{0.023814in}{-0.041004in}}{\pgfqpoint{0.032409in}{-0.032409in}}%
\pgfpathcurveto{\pgfqpoint{0.041004in}{-0.023814in}}{\pgfqpoint{0.045833in}{-0.012155in}}{\pgfqpoint{0.045833in}{0.000000in}}%
\pgfpathcurveto{\pgfqpoint{0.045833in}{0.012155in}}{\pgfqpoint{0.041004in}{0.023814in}}{\pgfqpoint{0.032409in}{0.032409in}}%
\pgfpathcurveto{\pgfqpoint{0.023814in}{0.041004in}}{\pgfqpoint{0.012155in}{0.045833in}}{\pgfqpoint{0.000000in}{0.045833in}}%
\pgfpathcurveto{\pgfqpoint{-0.012155in}{0.045833in}}{\pgfqpoint{-0.023814in}{0.041004in}}{\pgfqpoint{-0.032409in}{0.032409in}}%
\pgfpathcurveto{\pgfqpoint{-0.041004in}{0.023814in}}{\pgfqpoint{-0.045833in}{0.012155in}}{\pgfqpoint{-0.045833in}{0.000000in}}%
\pgfpathcurveto{\pgfqpoint{-0.045833in}{-0.012155in}}{\pgfqpoint{-0.041004in}{-0.023814in}}{\pgfqpoint{-0.032409in}{-0.032409in}}%
\pgfpathcurveto{\pgfqpoint{-0.023814in}{-0.041004in}}{\pgfqpoint{-0.012155in}{-0.045833in}}{\pgfqpoint{0.000000in}{-0.045833in}}%
\pgfpathclose%
\pgfusepath{stroke,fill}%
}%
\begin{pgfscope}%
\pgfsys@transformshift{1.022489in}{3.096196in}%
\pgfsys@useobject{currentmarker}{}%
\end{pgfscope}%
\end{pgfscope}%
\begin{pgfscope}%
\pgfpathrectangle{\pgfqpoint{0.100000in}{0.100000in}}{\pgfqpoint{5.307240in}{3.397500in}}%
\pgfusepath{clip}%
\pgfsetrectcap%
\pgfsetroundjoin%
\pgfsetlinewidth{1.505625pt}%
\definecolor{currentstroke}{rgb}{0.678431,1.000000,0.184314}%
\pgfsetstrokecolor{currentstroke}%
\pgfsetstrokeopacity{0.500000}%
\pgfsetdash{}{0pt}%
\pgfpathmoveto{\pgfqpoint{4.316039in}{1.877477in}}%
\pgfusepath{stroke}%
\end{pgfscope}%
\begin{pgfscope}%
\pgfpathrectangle{\pgfqpoint{0.100000in}{0.100000in}}{\pgfqpoint{5.307240in}{3.397500in}}%
\pgfusepath{clip}%
\pgfsetbuttcap%
\pgfsetroundjoin%
\definecolor{currentfill}{rgb}{0.678431,1.000000,0.184314}%
\pgfsetfillcolor{currentfill}%
\pgfsetfillopacity{0.500000}%
\pgfsetlinewidth{0.250937pt}%
\definecolor{currentstroke}{rgb}{0.000000,0.000000,0.000000}%
\pgfsetstrokecolor{currentstroke}%
\pgfsetstrokeopacity{0.500000}%
\pgfsetdash{}{0pt}%
\pgfsys@defobject{currentmarker}{\pgfqpoint{-0.096528in}{-0.096528in}}{\pgfqpoint{0.096528in}{0.096528in}}{%
\pgfpathmoveto{\pgfqpoint{0.000000in}{-0.096528in}}%
\pgfpathcurveto{\pgfqpoint{0.025599in}{-0.096528in}}{\pgfqpoint{0.050154in}{-0.086357in}}{\pgfqpoint{0.068255in}{-0.068255in}}%
\pgfpathcurveto{\pgfqpoint{0.086357in}{-0.050154in}}{\pgfqpoint{0.096528in}{-0.025599in}}{\pgfqpoint{0.096528in}{0.000000in}}%
\pgfpathcurveto{\pgfqpoint{0.096528in}{0.025599in}}{\pgfqpoint{0.086357in}{0.050154in}}{\pgfqpoint{0.068255in}{0.068255in}}%
\pgfpathcurveto{\pgfqpoint{0.050154in}{0.086357in}}{\pgfqpoint{0.025599in}{0.096528in}}{\pgfqpoint{0.000000in}{0.096528in}}%
\pgfpathcurveto{\pgfqpoint{-0.025599in}{0.096528in}}{\pgfqpoint{-0.050154in}{0.086357in}}{\pgfqpoint{-0.068255in}{0.068255in}}%
\pgfpathcurveto{\pgfqpoint{-0.086357in}{0.050154in}}{\pgfqpoint{-0.096528in}{0.025599in}}{\pgfqpoint{-0.096528in}{0.000000in}}%
\pgfpathcurveto{\pgfqpoint{-0.096528in}{-0.025599in}}{\pgfqpoint{-0.086357in}{-0.050154in}}{\pgfqpoint{-0.068255in}{-0.068255in}}%
\pgfpathcurveto{\pgfqpoint{-0.050154in}{-0.086357in}}{\pgfqpoint{-0.025599in}{-0.096528in}}{\pgfqpoint{0.000000in}{-0.096528in}}%
\pgfpathclose%
\pgfusepath{stroke,fill}%
}%
\begin{pgfscope}%
\pgfsys@transformshift{4.316039in}{1.877477in}%
\pgfsys@useobject{currentmarker}{}%
\end{pgfscope}%
\end{pgfscope}%
\begin{pgfscope}%
\pgfpathrectangle{\pgfqpoint{0.100000in}{0.100000in}}{\pgfqpoint{5.307240in}{3.397500in}}%
\pgfusepath{clip}%
\pgfsetrectcap%
\pgfsetroundjoin%
\pgfsetlinewidth{1.505625pt}%
\definecolor{currentstroke}{rgb}{0.678431,1.000000,0.184314}%
\pgfsetstrokecolor{currentstroke}%
\pgfsetstrokeopacity{0.500000}%
\pgfsetdash{}{0pt}%
\pgfpathmoveto{\pgfqpoint{4.265831in}{1.937219in}}%
\pgfusepath{stroke}%
\end{pgfscope}%
\begin{pgfscope}%
\pgfpathrectangle{\pgfqpoint{0.100000in}{0.100000in}}{\pgfqpoint{5.307240in}{3.397500in}}%
\pgfusepath{clip}%
\pgfsetbuttcap%
\pgfsetroundjoin%
\definecolor{currentfill}{rgb}{0.678431,1.000000,0.184314}%
\pgfsetfillcolor{currentfill}%
\pgfsetfillopacity{0.500000}%
\pgfsetlinewidth{0.250937pt}%
\definecolor{currentstroke}{rgb}{0.000000,0.000000,0.000000}%
\pgfsetstrokecolor{currentstroke}%
\pgfsetstrokeopacity{0.500000}%
\pgfsetdash{}{0pt}%
\pgfsys@defobject{currentmarker}{\pgfqpoint{-0.094444in}{-0.094444in}}{\pgfqpoint{0.094444in}{0.094444in}}{%
\pgfpathmoveto{\pgfqpoint{0.000000in}{-0.094444in}}%
\pgfpathcurveto{\pgfqpoint{0.025047in}{-0.094444in}}{\pgfqpoint{0.049071in}{-0.084493in}}{\pgfqpoint{0.066782in}{-0.066782in}}%
\pgfpathcurveto{\pgfqpoint{0.084493in}{-0.049071in}}{\pgfqpoint{0.094444in}{-0.025047in}}{\pgfqpoint{0.094444in}{0.000000in}}%
\pgfpathcurveto{\pgfqpoint{0.094444in}{0.025047in}}{\pgfqpoint{0.084493in}{0.049071in}}{\pgfqpoint{0.066782in}{0.066782in}}%
\pgfpathcurveto{\pgfqpoint{0.049071in}{0.084493in}}{\pgfqpoint{0.025047in}{0.094444in}}{\pgfqpoint{0.000000in}{0.094444in}}%
\pgfpathcurveto{\pgfqpoint{-0.025047in}{0.094444in}}{\pgfqpoint{-0.049071in}{0.084493in}}{\pgfqpoint{-0.066782in}{0.066782in}}%
\pgfpathcurveto{\pgfqpoint{-0.084493in}{0.049071in}}{\pgfqpoint{-0.094444in}{0.025047in}}{\pgfqpoint{-0.094444in}{0.000000in}}%
\pgfpathcurveto{\pgfqpoint{-0.094444in}{-0.025047in}}{\pgfqpoint{-0.084493in}{-0.049071in}}{\pgfqpoint{-0.066782in}{-0.066782in}}%
\pgfpathcurveto{\pgfqpoint{-0.049071in}{-0.084493in}}{\pgfqpoint{-0.025047in}{-0.094444in}}{\pgfqpoint{0.000000in}{-0.094444in}}%
\pgfpathclose%
\pgfusepath{stroke,fill}%
}%
\begin{pgfscope}%
\pgfsys@transformshift{4.265831in}{1.937219in}%
\pgfsys@useobject{currentmarker}{}%
\end{pgfscope}%
\end{pgfscope}%
\begin{pgfscope}%
\pgfpathrectangle{\pgfqpoint{0.100000in}{0.100000in}}{\pgfqpoint{5.307240in}{3.397500in}}%
\pgfusepath{clip}%
\pgfsetrectcap%
\pgfsetroundjoin%
\pgfsetlinewidth{1.505625pt}%
\definecolor{currentstroke}{rgb}{0.678431,1.000000,0.184314}%
\pgfsetstrokecolor{currentstroke}%
\pgfsetstrokeopacity{0.500000}%
\pgfsetdash{}{0pt}%
\pgfpathmoveto{\pgfqpoint{4.191463in}{1.934603in}}%
\pgfusepath{stroke}%
\end{pgfscope}%
\begin{pgfscope}%
\pgfpathrectangle{\pgfqpoint{0.100000in}{0.100000in}}{\pgfqpoint{5.307240in}{3.397500in}}%
\pgfusepath{clip}%
\pgfsetbuttcap%
\pgfsetroundjoin%
\definecolor{currentfill}{rgb}{0.678431,1.000000,0.184314}%
\pgfsetfillcolor{currentfill}%
\pgfsetfillopacity{0.500000}%
\pgfsetlinewidth{0.250937pt}%
\definecolor{currentstroke}{rgb}{0.000000,0.000000,0.000000}%
\pgfsetstrokecolor{currentstroke}%
\pgfsetstrokeopacity{0.500000}%
\pgfsetdash{}{0pt}%
\pgfsys@defobject{currentmarker}{\pgfqpoint{-0.084722in}{-0.084722in}}{\pgfqpoint{0.084722in}{0.084722in}}{%
\pgfpathmoveto{\pgfqpoint{0.000000in}{-0.084722in}}%
\pgfpathcurveto{\pgfqpoint{0.022469in}{-0.084722in}}{\pgfqpoint{0.044020in}{-0.075795in}}{\pgfqpoint{0.059908in}{-0.059908in}}%
\pgfpathcurveto{\pgfqpoint{0.075795in}{-0.044020in}}{\pgfqpoint{0.084722in}{-0.022469in}}{\pgfqpoint{0.084722in}{0.000000in}}%
\pgfpathcurveto{\pgfqpoint{0.084722in}{0.022469in}}{\pgfqpoint{0.075795in}{0.044020in}}{\pgfqpoint{0.059908in}{0.059908in}}%
\pgfpathcurveto{\pgfqpoint{0.044020in}{0.075795in}}{\pgfqpoint{0.022469in}{0.084722in}}{\pgfqpoint{0.000000in}{0.084722in}}%
\pgfpathcurveto{\pgfqpoint{-0.022469in}{0.084722in}}{\pgfqpoint{-0.044020in}{0.075795in}}{\pgfqpoint{-0.059908in}{0.059908in}}%
\pgfpathcurveto{\pgfqpoint{-0.075795in}{0.044020in}}{\pgfqpoint{-0.084722in}{0.022469in}}{\pgfqpoint{-0.084722in}{0.000000in}}%
\pgfpathcurveto{\pgfqpoint{-0.084722in}{-0.022469in}}{\pgfqpoint{-0.075795in}{-0.044020in}}{\pgfqpoint{-0.059908in}{-0.059908in}}%
\pgfpathcurveto{\pgfqpoint{-0.044020in}{-0.075795in}}{\pgfqpoint{-0.022469in}{-0.084722in}}{\pgfqpoint{0.000000in}{-0.084722in}}%
\pgfpathclose%
\pgfusepath{stroke,fill}%
}%
\begin{pgfscope}%
\pgfsys@transformshift{4.191463in}{1.934603in}%
\pgfsys@useobject{currentmarker}{}%
\end{pgfscope}%
\end{pgfscope}%
\begin{pgfscope}%
\pgfpathrectangle{\pgfqpoint{0.100000in}{0.100000in}}{\pgfqpoint{5.307240in}{3.397500in}}%
\pgfusepath{clip}%
\pgfsetrectcap%
\pgfsetroundjoin%
\pgfsetlinewidth{1.505625pt}%
\definecolor{currentstroke}{rgb}{0.678431,1.000000,0.184314}%
\pgfsetstrokecolor{currentstroke}%
\pgfsetstrokeopacity{0.500000}%
\pgfsetdash{}{0pt}%
\pgfpathmoveto{\pgfqpoint{4.392390in}{2.107690in}}%
\pgfusepath{stroke}%
\end{pgfscope}%
\begin{pgfscope}%
\pgfpathrectangle{\pgfqpoint{0.100000in}{0.100000in}}{\pgfqpoint{5.307240in}{3.397500in}}%
\pgfusepath{clip}%
\pgfsetbuttcap%
\pgfsetroundjoin%
\definecolor{currentfill}{rgb}{0.678431,1.000000,0.184314}%
\pgfsetfillcolor{currentfill}%
\pgfsetfillopacity{0.500000}%
\pgfsetlinewidth{0.250937pt}%
\definecolor{currentstroke}{rgb}{0.000000,0.000000,0.000000}%
\pgfsetstrokecolor{currentstroke}%
\pgfsetstrokeopacity{0.500000}%
\pgfsetdash{}{0pt}%
\pgfsys@defobject{currentmarker}{\pgfqpoint{-0.061806in}{-0.061806in}}{\pgfqpoint{0.061806in}{0.061806in}}{%
\pgfpathmoveto{\pgfqpoint{0.000000in}{-0.061806in}}%
\pgfpathcurveto{\pgfqpoint{0.016391in}{-0.061806in}}{\pgfqpoint{0.032113in}{-0.055293in}}{\pgfqpoint{0.043703in}{-0.043703in}}%
\pgfpathcurveto{\pgfqpoint{0.055293in}{-0.032113in}}{\pgfqpoint{0.061806in}{-0.016391in}}{\pgfqpoint{0.061806in}{0.000000in}}%
\pgfpathcurveto{\pgfqpoint{0.061806in}{0.016391in}}{\pgfqpoint{0.055293in}{0.032113in}}{\pgfqpoint{0.043703in}{0.043703in}}%
\pgfpathcurveto{\pgfqpoint{0.032113in}{0.055293in}}{\pgfqpoint{0.016391in}{0.061806in}}{\pgfqpoint{0.000000in}{0.061806in}}%
\pgfpathcurveto{\pgfqpoint{-0.016391in}{0.061806in}}{\pgfqpoint{-0.032113in}{0.055293in}}{\pgfqpoint{-0.043703in}{0.043703in}}%
\pgfpathcurveto{\pgfqpoint{-0.055293in}{0.032113in}}{\pgfqpoint{-0.061806in}{0.016391in}}{\pgfqpoint{-0.061806in}{0.000000in}}%
\pgfpathcurveto{\pgfqpoint{-0.061806in}{-0.016391in}}{\pgfqpoint{-0.055293in}{-0.032113in}}{\pgfqpoint{-0.043703in}{-0.043703in}}%
\pgfpathcurveto{\pgfqpoint{-0.032113in}{-0.055293in}}{\pgfqpoint{-0.016391in}{-0.061806in}}{\pgfqpoint{0.000000in}{-0.061806in}}%
\pgfpathclose%
\pgfusepath{stroke,fill}%
}%
\begin{pgfscope}%
\pgfsys@transformshift{4.392390in}{2.107690in}%
\pgfsys@useobject{currentmarker}{}%
\end{pgfscope}%
\end{pgfscope}%
\begin{pgfscope}%
\pgfpathrectangle{\pgfqpoint{0.100000in}{0.100000in}}{\pgfqpoint{5.307240in}{3.397500in}}%
\pgfusepath{clip}%
\pgfsetrectcap%
\pgfsetroundjoin%
\pgfsetlinewidth{1.505625pt}%
\definecolor{currentstroke}{rgb}{0.678431,1.000000,0.184314}%
\pgfsetstrokecolor{currentstroke}%
\pgfsetstrokeopacity{0.500000}%
\pgfsetdash{}{0pt}%
\pgfpathmoveto{\pgfqpoint{4.257520in}{2.044035in}}%
\pgfusepath{stroke}%
\end{pgfscope}%
\begin{pgfscope}%
\pgfpathrectangle{\pgfqpoint{0.100000in}{0.100000in}}{\pgfqpoint{5.307240in}{3.397500in}}%
\pgfusepath{clip}%
\pgfsetbuttcap%
\pgfsetroundjoin%
\definecolor{currentfill}{rgb}{0.678431,1.000000,0.184314}%
\pgfsetfillcolor{currentfill}%
\pgfsetfillopacity{0.500000}%
\pgfsetlinewidth{0.250937pt}%
\definecolor{currentstroke}{rgb}{0.000000,0.000000,0.000000}%
\pgfsetstrokecolor{currentstroke}%
\pgfsetstrokeopacity{0.500000}%
\pgfsetdash{}{0pt}%
\pgfsys@defobject{currentmarker}{\pgfqpoint{-0.084028in}{-0.084028in}}{\pgfqpoint{0.084028in}{0.084028in}}{%
\pgfpathmoveto{\pgfqpoint{0.000000in}{-0.084028in}}%
\pgfpathcurveto{\pgfqpoint{0.022284in}{-0.084028in}}{\pgfqpoint{0.043659in}{-0.075174in}}{\pgfqpoint{0.059417in}{-0.059417in}}%
\pgfpathcurveto{\pgfqpoint{0.075174in}{-0.043659in}}{\pgfqpoint{0.084028in}{-0.022284in}}{\pgfqpoint{0.084028in}{0.000000in}}%
\pgfpathcurveto{\pgfqpoint{0.084028in}{0.022284in}}{\pgfqpoint{0.075174in}{0.043659in}}{\pgfqpoint{0.059417in}{0.059417in}}%
\pgfpathcurveto{\pgfqpoint{0.043659in}{0.075174in}}{\pgfqpoint{0.022284in}{0.084028in}}{\pgfqpoint{0.000000in}{0.084028in}}%
\pgfpathcurveto{\pgfqpoint{-0.022284in}{0.084028in}}{\pgfqpoint{-0.043659in}{0.075174in}}{\pgfqpoint{-0.059417in}{0.059417in}}%
\pgfpathcurveto{\pgfqpoint{-0.075174in}{0.043659in}}{\pgfqpoint{-0.084028in}{0.022284in}}{\pgfqpoint{-0.084028in}{0.000000in}}%
\pgfpathcurveto{\pgfqpoint{-0.084028in}{-0.022284in}}{\pgfqpoint{-0.075174in}{-0.043659in}}{\pgfqpoint{-0.059417in}{-0.059417in}}%
\pgfpathcurveto{\pgfqpoint{-0.043659in}{-0.075174in}}{\pgfqpoint{-0.022284in}{-0.084028in}}{\pgfqpoint{0.000000in}{-0.084028in}}%
\pgfpathclose%
\pgfusepath{stroke,fill}%
}%
\begin{pgfscope}%
\pgfsys@transformshift{4.257520in}{2.044035in}%
\pgfsys@useobject{currentmarker}{}%
\end{pgfscope}%
\end{pgfscope}%
\begin{pgfscope}%
\pgfpathrectangle{\pgfqpoint{0.100000in}{0.100000in}}{\pgfqpoint{5.307240in}{3.397500in}}%
\pgfusepath{clip}%
\pgfsetrectcap%
\pgfsetroundjoin%
\pgfsetlinewidth{1.505625pt}%
\definecolor{currentstroke}{rgb}{0.678431,1.000000,0.184314}%
\pgfsetstrokecolor{currentstroke}%
\pgfsetstrokeopacity{0.500000}%
\pgfsetdash{}{0pt}%
\pgfpathmoveto{\pgfqpoint{4.316886in}{2.146555in}}%
\pgfusepath{stroke}%
\end{pgfscope}%
\begin{pgfscope}%
\pgfpathrectangle{\pgfqpoint{0.100000in}{0.100000in}}{\pgfqpoint{5.307240in}{3.397500in}}%
\pgfusepath{clip}%
\pgfsetbuttcap%
\pgfsetroundjoin%
\definecolor{currentfill}{rgb}{0.678431,1.000000,0.184314}%
\pgfsetfillcolor{currentfill}%
\pgfsetfillopacity{0.500000}%
\pgfsetlinewidth{0.250937pt}%
\definecolor{currentstroke}{rgb}{0.000000,0.000000,0.000000}%
\pgfsetstrokecolor{currentstroke}%
\pgfsetstrokeopacity{0.500000}%
\pgfsetdash{}{0pt}%
\pgfsys@defobject{currentmarker}{\pgfqpoint{-0.086806in}{-0.086806in}}{\pgfqpoint{0.086806in}{0.086806in}}{%
\pgfpathmoveto{\pgfqpoint{0.000000in}{-0.086806in}}%
\pgfpathcurveto{\pgfqpoint{0.023021in}{-0.086806in}}{\pgfqpoint{0.045102in}{-0.077659in}}{\pgfqpoint{0.061381in}{-0.061381in}}%
\pgfpathcurveto{\pgfqpoint{0.077659in}{-0.045102in}}{\pgfqpoint{0.086806in}{-0.023021in}}{\pgfqpoint{0.086806in}{0.000000in}}%
\pgfpathcurveto{\pgfqpoint{0.086806in}{0.023021in}}{\pgfqpoint{0.077659in}{0.045102in}}{\pgfqpoint{0.061381in}{0.061381in}}%
\pgfpathcurveto{\pgfqpoint{0.045102in}{0.077659in}}{\pgfqpoint{0.023021in}{0.086806in}}{\pgfqpoint{0.000000in}{0.086806in}}%
\pgfpathcurveto{\pgfqpoint{-0.023021in}{0.086806in}}{\pgfqpoint{-0.045102in}{0.077659in}}{\pgfqpoint{-0.061381in}{0.061381in}}%
\pgfpathcurveto{\pgfqpoint{-0.077659in}{0.045102in}}{\pgfqpoint{-0.086806in}{0.023021in}}{\pgfqpoint{-0.086806in}{0.000000in}}%
\pgfpathcurveto{\pgfqpoint{-0.086806in}{-0.023021in}}{\pgfqpoint{-0.077659in}{-0.045102in}}{\pgfqpoint{-0.061381in}{-0.061381in}}%
\pgfpathcurveto{\pgfqpoint{-0.045102in}{-0.077659in}}{\pgfqpoint{-0.023021in}{-0.086806in}}{\pgfqpoint{0.000000in}{-0.086806in}}%
\pgfpathclose%
\pgfusepath{stroke,fill}%
}%
\begin{pgfscope}%
\pgfsys@transformshift{4.316886in}{2.146555in}%
\pgfsys@useobject{currentmarker}{}%
\end{pgfscope}%
\end{pgfscope}%
\begin{pgfscope}%
\pgfpathrectangle{\pgfqpoint{0.100000in}{0.100000in}}{\pgfqpoint{5.307240in}{3.397500in}}%
\pgfusepath{clip}%
\pgfsetrectcap%
\pgfsetroundjoin%
\pgfsetlinewidth{1.505625pt}%
\definecolor{currentstroke}{rgb}{0.678431,1.000000,0.184314}%
\pgfsetstrokecolor{currentstroke}%
\pgfsetstrokeopacity{0.500000}%
\pgfsetdash{}{0pt}%
\pgfpathmoveto{\pgfqpoint{3.602263in}{2.555676in}}%
\pgfusepath{stroke}%
\end{pgfscope}%
\begin{pgfscope}%
\pgfpathrectangle{\pgfqpoint{0.100000in}{0.100000in}}{\pgfqpoint{5.307240in}{3.397500in}}%
\pgfusepath{clip}%
\pgfsetbuttcap%
\pgfsetroundjoin%
\definecolor{currentfill}{rgb}{0.678431,1.000000,0.184314}%
\pgfsetfillcolor{currentfill}%
\pgfsetfillopacity{0.500000}%
\pgfsetlinewidth{0.250937pt}%
\definecolor{currentstroke}{rgb}{0.000000,0.000000,0.000000}%
\pgfsetstrokecolor{currentstroke}%
\pgfsetstrokeopacity{0.500000}%
\pgfsetdash{}{0pt}%
\pgfsys@defobject{currentmarker}{\pgfqpoint{-0.074306in}{-0.074306in}}{\pgfqpoint{0.074306in}{0.074306in}}{%
\pgfpathmoveto{\pgfqpoint{0.000000in}{-0.074306in}}%
\pgfpathcurveto{\pgfqpoint{0.019706in}{-0.074306in}}{\pgfqpoint{0.038608in}{-0.066476in}}{\pgfqpoint{0.052542in}{-0.052542in}}%
\pgfpathcurveto{\pgfqpoint{0.066476in}{-0.038608in}}{\pgfqpoint{0.074306in}{-0.019706in}}{\pgfqpoint{0.074306in}{0.000000in}}%
\pgfpathcurveto{\pgfqpoint{0.074306in}{0.019706in}}{\pgfqpoint{0.066476in}{0.038608in}}{\pgfqpoint{0.052542in}{0.052542in}}%
\pgfpathcurveto{\pgfqpoint{0.038608in}{0.066476in}}{\pgfqpoint{0.019706in}{0.074306in}}{\pgfqpoint{0.000000in}{0.074306in}}%
\pgfpathcurveto{\pgfqpoint{-0.019706in}{0.074306in}}{\pgfqpoint{-0.038608in}{0.066476in}}{\pgfqpoint{-0.052542in}{0.052542in}}%
\pgfpathcurveto{\pgfqpoint{-0.066476in}{0.038608in}}{\pgfqpoint{-0.074306in}{0.019706in}}{\pgfqpoint{-0.074306in}{0.000000in}}%
\pgfpathcurveto{\pgfqpoint{-0.074306in}{-0.019706in}}{\pgfqpoint{-0.066476in}{-0.038608in}}{\pgfqpoint{-0.052542in}{-0.052542in}}%
\pgfpathcurveto{\pgfqpoint{-0.038608in}{-0.066476in}}{\pgfqpoint{-0.019706in}{-0.074306in}}{\pgfqpoint{0.000000in}{-0.074306in}}%
\pgfpathclose%
\pgfusepath{stroke,fill}%
}%
\begin{pgfscope}%
\pgfsys@transformshift{3.602263in}{2.555676in}%
\pgfsys@useobject{currentmarker}{}%
\end{pgfscope}%
\end{pgfscope}%
\begin{pgfscope}%
\pgfpathrectangle{\pgfqpoint{0.100000in}{0.100000in}}{\pgfqpoint{5.307240in}{3.397500in}}%
\pgfusepath{clip}%
\pgfsetrectcap%
\pgfsetroundjoin%
\pgfsetlinewidth{1.505625pt}%
\definecolor{currentstroke}{rgb}{0.678431,1.000000,0.184314}%
\pgfsetstrokecolor{currentstroke}%
\pgfsetstrokeopacity{0.500000}%
\pgfsetdash{}{0pt}%
\pgfpathmoveto{\pgfqpoint{3.341718in}{2.605542in}}%
\pgfusepath{stroke}%
\end{pgfscope}%
\begin{pgfscope}%
\pgfpathrectangle{\pgfqpoint{0.100000in}{0.100000in}}{\pgfqpoint{5.307240in}{3.397500in}}%
\pgfusepath{clip}%
\pgfsetbuttcap%
\pgfsetroundjoin%
\definecolor{currentfill}{rgb}{0.678431,1.000000,0.184314}%
\pgfsetfillcolor{currentfill}%
\pgfsetfillopacity{0.500000}%
\pgfsetlinewidth{0.250937pt}%
\definecolor{currentstroke}{rgb}{0.000000,0.000000,0.000000}%
\pgfsetstrokecolor{currentstroke}%
\pgfsetstrokeopacity{0.500000}%
\pgfsetdash{}{0pt}%
\pgfsys@defobject{currentmarker}{\pgfqpoint{-0.070139in}{-0.070139in}}{\pgfqpoint{0.070139in}{0.070139in}}{%
\pgfpathmoveto{\pgfqpoint{0.000000in}{-0.070139in}}%
\pgfpathcurveto{\pgfqpoint{0.018601in}{-0.070139in}}{\pgfqpoint{0.036443in}{-0.062749in}}{\pgfqpoint{0.049596in}{-0.049596in}}%
\pgfpathcurveto{\pgfqpoint{0.062749in}{-0.036443in}}{\pgfqpoint{0.070139in}{-0.018601in}}{\pgfqpoint{0.070139in}{0.000000in}}%
\pgfpathcurveto{\pgfqpoint{0.070139in}{0.018601in}}{\pgfqpoint{0.062749in}{0.036443in}}{\pgfqpoint{0.049596in}{0.049596in}}%
\pgfpathcurveto{\pgfqpoint{0.036443in}{0.062749in}}{\pgfqpoint{0.018601in}{0.070139in}}{\pgfqpoint{0.000000in}{0.070139in}}%
\pgfpathcurveto{\pgfqpoint{-0.018601in}{0.070139in}}{\pgfqpoint{-0.036443in}{0.062749in}}{\pgfqpoint{-0.049596in}{0.049596in}}%
\pgfpathcurveto{\pgfqpoint{-0.062749in}{0.036443in}}{\pgfqpoint{-0.070139in}{0.018601in}}{\pgfqpoint{-0.070139in}{0.000000in}}%
\pgfpathcurveto{\pgfqpoint{-0.070139in}{-0.018601in}}{\pgfqpoint{-0.062749in}{-0.036443in}}{\pgfqpoint{-0.049596in}{-0.049596in}}%
\pgfpathcurveto{\pgfqpoint{-0.036443in}{-0.062749in}}{\pgfqpoint{-0.018601in}{-0.070139in}}{\pgfqpoint{0.000000in}{-0.070139in}}%
\pgfpathclose%
\pgfusepath{stroke,fill}%
}%
\begin{pgfscope}%
\pgfsys@transformshift{3.341718in}{2.605542in}%
\pgfsys@useobject{currentmarker}{}%
\end{pgfscope}%
\end{pgfscope}%
\begin{pgfscope}%
\pgfpathrectangle{\pgfqpoint{0.100000in}{0.100000in}}{\pgfqpoint{5.307240in}{3.397500in}}%
\pgfusepath{clip}%
\pgfsetrectcap%
\pgfsetroundjoin%
\pgfsetlinewidth{1.505625pt}%
\definecolor{currentstroke}{rgb}{0.678431,1.000000,0.184314}%
\pgfsetstrokecolor{currentstroke}%
\pgfsetstrokeopacity{0.500000}%
\pgfsetdash{}{0pt}%
\pgfpathmoveto{\pgfqpoint{3.603097in}{2.498845in}}%
\pgfusepath{stroke}%
\end{pgfscope}%
\begin{pgfscope}%
\pgfpathrectangle{\pgfqpoint{0.100000in}{0.100000in}}{\pgfqpoint{5.307240in}{3.397500in}}%
\pgfusepath{clip}%
\pgfsetbuttcap%
\pgfsetroundjoin%
\definecolor{currentfill}{rgb}{0.678431,1.000000,0.184314}%
\pgfsetfillcolor{currentfill}%
\pgfsetfillopacity{0.500000}%
\pgfsetlinewidth{0.250937pt}%
\definecolor{currentstroke}{rgb}{0.000000,0.000000,0.000000}%
\pgfsetstrokecolor{currentstroke}%
\pgfsetstrokeopacity{0.500000}%
\pgfsetdash{}{0pt}%
\pgfsys@defobject{currentmarker}{\pgfqpoint{-0.095833in}{-0.095833in}}{\pgfqpoint{0.095833in}{0.095833in}}{%
\pgfpathmoveto{\pgfqpoint{0.000000in}{-0.095833in}}%
\pgfpathcurveto{\pgfqpoint{0.025415in}{-0.095833in}}{\pgfqpoint{0.049793in}{-0.085736in}}{\pgfqpoint{0.067764in}{-0.067764in}}%
\pgfpathcurveto{\pgfqpoint{0.085736in}{-0.049793in}}{\pgfqpoint{0.095833in}{-0.025415in}}{\pgfqpoint{0.095833in}{0.000000in}}%
\pgfpathcurveto{\pgfqpoint{0.095833in}{0.025415in}}{\pgfqpoint{0.085736in}{0.049793in}}{\pgfqpoint{0.067764in}{0.067764in}}%
\pgfpathcurveto{\pgfqpoint{0.049793in}{0.085736in}}{\pgfqpoint{0.025415in}{0.095833in}}{\pgfqpoint{0.000000in}{0.095833in}}%
\pgfpathcurveto{\pgfqpoint{-0.025415in}{0.095833in}}{\pgfqpoint{-0.049793in}{0.085736in}}{\pgfqpoint{-0.067764in}{0.067764in}}%
\pgfpathcurveto{\pgfqpoint{-0.085736in}{0.049793in}}{\pgfqpoint{-0.095833in}{0.025415in}}{\pgfqpoint{-0.095833in}{0.000000in}}%
\pgfpathcurveto{\pgfqpoint{-0.095833in}{-0.025415in}}{\pgfqpoint{-0.085736in}{-0.049793in}}{\pgfqpoint{-0.067764in}{-0.067764in}}%
\pgfpathcurveto{\pgfqpoint{-0.049793in}{-0.085736in}}{\pgfqpoint{-0.025415in}{-0.095833in}}{\pgfqpoint{0.000000in}{-0.095833in}}%
\pgfpathclose%
\pgfusepath{stroke,fill}%
}%
\begin{pgfscope}%
\pgfsys@transformshift{3.603097in}{2.498845in}%
\pgfsys@useobject{currentmarker}{}%
\end{pgfscope}%
\end{pgfscope}%
\begin{pgfscope}%
\pgfpathrectangle{\pgfqpoint{0.100000in}{0.100000in}}{\pgfqpoint{5.307240in}{3.397500in}}%
\pgfusepath{clip}%
\pgfsetrectcap%
\pgfsetroundjoin%
\pgfsetlinewidth{1.505625pt}%
\definecolor{currentstroke}{rgb}{0.678431,1.000000,0.184314}%
\pgfsetstrokecolor{currentstroke}%
\pgfsetstrokeopacity{0.500000}%
\pgfsetdash{}{0pt}%
\pgfpathmoveto{\pgfqpoint{3.631698in}{2.589353in}}%
\pgfusepath{stroke}%
\end{pgfscope}%
\begin{pgfscope}%
\pgfpathrectangle{\pgfqpoint{0.100000in}{0.100000in}}{\pgfqpoint{5.307240in}{3.397500in}}%
\pgfusepath{clip}%
\pgfsetbuttcap%
\pgfsetroundjoin%
\definecolor{currentfill}{rgb}{0.678431,1.000000,0.184314}%
\pgfsetfillcolor{currentfill}%
\pgfsetfillopacity{0.500000}%
\pgfsetlinewidth{0.250937pt}%
\definecolor{currentstroke}{rgb}{0.000000,0.000000,0.000000}%
\pgfsetstrokecolor{currentstroke}%
\pgfsetstrokeopacity{0.500000}%
\pgfsetdash{}{0pt}%
\pgfsys@defobject{currentmarker}{\pgfqpoint{-0.075000in}{-0.075000in}}{\pgfqpoint{0.075000in}{0.075000in}}{%
\pgfpathmoveto{\pgfqpoint{0.000000in}{-0.075000in}}%
\pgfpathcurveto{\pgfqpoint{0.019890in}{-0.075000in}}{\pgfqpoint{0.038968in}{-0.067098in}}{\pgfqpoint{0.053033in}{-0.053033in}}%
\pgfpathcurveto{\pgfqpoint{0.067098in}{-0.038968in}}{\pgfqpoint{0.075000in}{-0.019890in}}{\pgfqpoint{0.075000in}{0.000000in}}%
\pgfpathcurveto{\pgfqpoint{0.075000in}{0.019890in}}{\pgfqpoint{0.067098in}{0.038968in}}{\pgfqpoint{0.053033in}{0.053033in}}%
\pgfpathcurveto{\pgfqpoint{0.038968in}{0.067098in}}{\pgfqpoint{0.019890in}{0.075000in}}{\pgfqpoint{0.000000in}{0.075000in}}%
\pgfpathcurveto{\pgfqpoint{-0.019890in}{0.075000in}}{\pgfqpoint{-0.038968in}{0.067098in}}{\pgfqpoint{-0.053033in}{0.053033in}}%
\pgfpathcurveto{\pgfqpoint{-0.067098in}{0.038968in}}{\pgfqpoint{-0.075000in}{0.019890in}}{\pgfqpoint{-0.075000in}{0.000000in}}%
\pgfpathcurveto{\pgfqpoint{-0.075000in}{-0.019890in}}{\pgfqpoint{-0.067098in}{-0.038968in}}{\pgfqpoint{-0.053033in}{-0.053033in}}%
\pgfpathcurveto{\pgfqpoint{-0.038968in}{-0.067098in}}{\pgfqpoint{-0.019890in}{-0.075000in}}{\pgfqpoint{0.000000in}{-0.075000in}}%
\pgfpathclose%
\pgfusepath{stroke,fill}%
}%
\begin{pgfscope}%
\pgfsys@transformshift{3.631698in}{2.589353in}%
\pgfsys@useobject{currentmarker}{}%
\end{pgfscope}%
\end{pgfscope}%
\begin{pgfscope}%
\pgfpathrectangle{\pgfqpoint{0.100000in}{0.100000in}}{\pgfqpoint{5.307240in}{3.397500in}}%
\pgfusepath{clip}%
\pgfsetrectcap%
\pgfsetroundjoin%
\pgfsetlinewidth{1.505625pt}%
\definecolor{currentstroke}{rgb}{0.678431,1.000000,0.184314}%
\pgfsetstrokecolor{currentstroke}%
\pgfsetstrokeopacity{0.500000}%
\pgfsetdash{}{0pt}%
\pgfpathmoveto{\pgfqpoint{3.565466in}{2.372422in}}%
\pgfusepath{stroke}%
\end{pgfscope}%
\begin{pgfscope}%
\pgfpathrectangle{\pgfqpoint{0.100000in}{0.100000in}}{\pgfqpoint{5.307240in}{3.397500in}}%
\pgfusepath{clip}%
\pgfsetbuttcap%
\pgfsetroundjoin%
\definecolor{currentfill}{rgb}{0.678431,1.000000,0.184314}%
\pgfsetfillcolor{currentfill}%
\pgfsetfillopacity{0.500000}%
\pgfsetlinewidth{0.250937pt}%
\definecolor{currentstroke}{rgb}{0.000000,0.000000,0.000000}%
\pgfsetstrokecolor{currentstroke}%
\pgfsetstrokeopacity{0.500000}%
\pgfsetdash{}{0pt}%
\pgfsys@defobject{currentmarker}{\pgfqpoint{-0.095139in}{-0.095139in}}{\pgfqpoint{0.095139in}{0.095139in}}{%
\pgfpathmoveto{\pgfqpoint{0.000000in}{-0.095139in}}%
\pgfpathcurveto{\pgfqpoint{0.025231in}{-0.095139in}}{\pgfqpoint{0.049432in}{-0.085114in}}{\pgfqpoint{0.067273in}{-0.067273in}}%
\pgfpathcurveto{\pgfqpoint{0.085114in}{-0.049432in}}{\pgfqpoint{0.095139in}{-0.025231in}}{\pgfqpoint{0.095139in}{0.000000in}}%
\pgfpathcurveto{\pgfqpoint{0.095139in}{0.025231in}}{\pgfqpoint{0.085114in}{0.049432in}}{\pgfqpoint{0.067273in}{0.067273in}}%
\pgfpathcurveto{\pgfqpoint{0.049432in}{0.085114in}}{\pgfqpoint{0.025231in}{0.095139in}}{\pgfqpoint{0.000000in}{0.095139in}}%
\pgfpathcurveto{\pgfqpoint{-0.025231in}{0.095139in}}{\pgfqpoint{-0.049432in}{0.085114in}}{\pgfqpoint{-0.067273in}{0.067273in}}%
\pgfpathcurveto{\pgfqpoint{-0.085114in}{0.049432in}}{\pgfqpoint{-0.095139in}{0.025231in}}{\pgfqpoint{-0.095139in}{0.000000in}}%
\pgfpathcurveto{\pgfqpoint{-0.095139in}{-0.025231in}}{\pgfqpoint{-0.085114in}{-0.049432in}}{\pgfqpoint{-0.067273in}{-0.067273in}}%
\pgfpathcurveto{\pgfqpoint{-0.049432in}{-0.085114in}}{\pgfqpoint{-0.025231in}{-0.095139in}}{\pgfqpoint{0.000000in}{-0.095139in}}%
\pgfpathclose%
\pgfusepath{stroke,fill}%
}%
\begin{pgfscope}%
\pgfsys@transformshift{3.565466in}{2.372422in}%
\pgfsys@useobject{currentmarker}{}%
\end{pgfscope}%
\end{pgfscope}%
\begin{pgfscope}%
\pgfpathrectangle{\pgfqpoint{0.100000in}{0.100000in}}{\pgfqpoint{5.307240in}{3.397500in}}%
\pgfusepath{clip}%
\pgfsetrectcap%
\pgfsetroundjoin%
\pgfsetlinewidth{1.505625pt}%
\definecolor{currentstroke}{rgb}{0.678431,1.000000,0.184314}%
\pgfsetstrokecolor{currentstroke}%
\pgfsetstrokeopacity{0.500000}%
\pgfsetdash{}{0pt}%
\pgfpathmoveto{\pgfqpoint{3.368042in}{2.488585in}}%
\pgfusepath{stroke}%
\end{pgfscope}%
\begin{pgfscope}%
\pgfpathrectangle{\pgfqpoint{0.100000in}{0.100000in}}{\pgfqpoint{5.307240in}{3.397500in}}%
\pgfusepath{clip}%
\pgfsetbuttcap%
\pgfsetroundjoin%
\definecolor{currentfill}{rgb}{0.678431,1.000000,0.184314}%
\pgfsetfillcolor{currentfill}%
\pgfsetfillopacity{0.500000}%
\pgfsetlinewidth{0.250937pt}%
\definecolor{currentstroke}{rgb}{0.000000,0.000000,0.000000}%
\pgfsetstrokecolor{currentstroke}%
\pgfsetstrokeopacity{0.500000}%
\pgfsetdash{}{0pt}%
\pgfsys@defobject{currentmarker}{\pgfqpoint{-0.065278in}{-0.065278in}}{\pgfqpoint{0.065278in}{0.065278in}}{%
\pgfpathmoveto{\pgfqpoint{0.000000in}{-0.065278in}}%
\pgfpathcurveto{\pgfqpoint{0.017312in}{-0.065278in}}{\pgfqpoint{0.033917in}{-0.058400in}}{\pgfqpoint{0.046158in}{-0.046158in}}%
\pgfpathcurveto{\pgfqpoint{0.058400in}{-0.033917in}}{\pgfqpoint{0.065278in}{-0.017312in}}{\pgfqpoint{0.065278in}{0.000000in}}%
\pgfpathcurveto{\pgfqpoint{0.065278in}{0.017312in}}{\pgfqpoint{0.058400in}{0.033917in}}{\pgfqpoint{0.046158in}{0.046158in}}%
\pgfpathcurveto{\pgfqpoint{0.033917in}{0.058400in}}{\pgfqpoint{0.017312in}{0.065278in}}{\pgfqpoint{0.000000in}{0.065278in}}%
\pgfpathcurveto{\pgfqpoint{-0.017312in}{0.065278in}}{\pgfqpoint{-0.033917in}{0.058400in}}{\pgfqpoint{-0.046158in}{0.046158in}}%
\pgfpathcurveto{\pgfqpoint{-0.058400in}{0.033917in}}{\pgfqpoint{-0.065278in}{0.017312in}}{\pgfqpoint{-0.065278in}{0.000000in}}%
\pgfpathcurveto{\pgfqpoint{-0.065278in}{-0.017312in}}{\pgfqpoint{-0.058400in}{-0.033917in}}{\pgfqpoint{-0.046158in}{-0.046158in}}%
\pgfpathcurveto{\pgfqpoint{-0.033917in}{-0.058400in}}{\pgfqpoint{-0.017312in}{-0.065278in}}{\pgfqpoint{0.000000in}{-0.065278in}}%
\pgfpathclose%
\pgfusepath{stroke,fill}%
}%
\begin{pgfscope}%
\pgfsys@transformshift{3.368042in}{2.488585in}%
\pgfsys@useobject{currentmarker}{}%
\end{pgfscope}%
\end{pgfscope}%
\begin{pgfscope}%
\pgfpathrectangle{\pgfqpoint{0.100000in}{0.100000in}}{\pgfqpoint{5.307240in}{3.397500in}}%
\pgfusepath{clip}%
\pgfsetrectcap%
\pgfsetroundjoin%
\pgfsetlinewidth{1.505625pt}%
\definecolor{currentstroke}{rgb}{0.678431,1.000000,0.184314}%
\pgfsetstrokecolor{currentstroke}%
\pgfsetstrokeopacity{0.500000}%
\pgfsetdash{}{0pt}%
\pgfpathmoveto{\pgfqpoint{3.529313in}{2.412087in}}%
\pgfusepath{stroke}%
\end{pgfscope}%
\begin{pgfscope}%
\pgfpathrectangle{\pgfqpoint{0.100000in}{0.100000in}}{\pgfqpoint{5.307240in}{3.397500in}}%
\pgfusepath{clip}%
\pgfsetbuttcap%
\pgfsetroundjoin%
\definecolor{currentfill}{rgb}{0.678431,1.000000,0.184314}%
\pgfsetfillcolor{currentfill}%
\pgfsetfillopacity{0.500000}%
\pgfsetlinewidth{0.250937pt}%
\definecolor{currentstroke}{rgb}{0.000000,0.000000,0.000000}%
\pgfsetstrokecolor{currentstroke}%
\pgfsetstrokeopacity{0.500000}%
\pgfsetdash{}{0pt}%
\pgfsys@defobject{currentmarker}{\pgfqpoint{-0.065278in}{-0.065278in}}{\pgfqpoint{0.065278in}{0.065278in}}{%
\pgfpathmoveto{\pgfqpoint{0.000000in}{-0.065278in}}%
\pgfpathcurveto{\pgfqpoint{0.017312in}{-0.065278in}}{\pgfqpoint{0.033917in}{-0.058400in}}{\pgfqpoint{0.046158in}{-0.046158in}}%
\pgfpathcurveto{\pgfqpoint{0.058400in}{-0.033917in}}{\pgfqpoint{0.065278in}{-0.017312in}}{\pgfqpoint{0.065278in}{0.000000in}}%
\pgfpathcurveto{\pgfqpoint{0.065278in}{0.017312in}}{\pgfqpoint{0.058400in}{0.033917in}}{\pgfqpoint{0.046158in}{0.046158in}}%
\pgfpathcurveto{\pgfqpoint{0.033917in}{0.058400in}}{\pgfqpoint{0.017312in}{0.065278in}}{\pgfqpoint{0.000000in}{0.065278in}}%
\pgfpathcurveto{\pgfqpoint{-0.017312in}{0.065278in}}{\pgfqpoint{-0.033917in}{0.058400in}}{\pgfqpoint{-0.046158in}{0.046158in}}%
\pgfpathcurveto{\pgfqpoint{-0.058400in}{0.033917in}}{\pgfqpoint{-0.065278in}{0.017312in}}{\pgfqpoint{-0.065278in}{0.000000in}}%
\pgfpathcurveto{\pgfqpoint{-0.065278in}{-0.017312in}}{\pgfqpoint{-0.058400in}{-0.033917in}}{\pgfqpoint{-0.046158in}{-0.046158in}}%
\pgfpathcurveto{\pgfqpoint{-0.033917in}{-0.058400in}}{\pgfqpoint{-0.017312in}{-0.065278in}}{\pgfqpoint{0.000000in}{-0.065278in}}%
\pgfpathclose%
\pgfusepath{stroke,fill}%
}%
\begin{pgfscope}%
\pgfsys@transformshift{3.529313in}{2.412087in}%
\pgfsys@useobject{currentmarker}{}%
\end{pgfscope}%
\end{pgfscope}%
\begin{pgfscope}%
\pgfpathrectangle{\pgfqpoint{0.100000in}{0.100000in}}{\pgfqpoint{5.307240in}{3.397500in}}%
\pgfusepath{clip}%
\pgfsetrectcap%
\pgfsetroundjoin%
\pgfsetlinewidth{1.505625pt}%
\definecolor{currentstroke}{rgb}{0.678431,1.000000,0.184314}%
\pgfsetstrokecolor{currentstroke}%
\pgfsetstrokeopacity{0.500000}%
\pgfsetdash{}{0pt}%
\pgfpathmoveto{\pgfqpoint{3.653761in}{2.416147in}}%
\pgfusepath{stroke}%
\end{pgfscope}%
\begin{pgfscope}%
\pgfpathrectangle{\pgfqpoint{0.100000in}{0.100000in}}{\pgfqpoint{5.307240in}{3.397500in}}%
\pgfusepath{clip}%
\pgfsetbuttcap%
\pgfsetroundjoin%
\definecolor{currentfill}{rgb}{0.678431,1.000000,0.184314}%
\pgfsetfillcolor{currentfill}%
\pgfsetfillopacity{0.500000}%
\pgfsetlinewidth{0.250937pt}%
\definecolor{currentstroke}{rgb}{0.000000,0.000000,0.000000}%
\pgfsetstrokecolor{currentstroke}%
\pgfsetstrokeopacity{0.500000}%
\pgfsetdash{}{0pt}%
\pgfsys@defobject{currentmarker}{\pgfqpoint{-0.078472in}{-0.078472in}}{\pgfqpoint{0.078472in}{0.078472in}}{%
\pgfpathmoveto{\pgfqpoint{0.000000in}{-0.078472in}}%
\pgfpathcurveto{\pgfqpoint{0.020811in}{-0.078472in}}{\pgfqpoint{0.040773in}{-0.070204in}}{\pgfqpoint{0.055488in}{-0.055488in}}%
\pgfpathcurveto{\pgfqpoint{0.070204in}{-0.040773in}}{\pgfqpoint{0.078472in}{-0.020811in}}{\pgfqpoint{0.078472in}{0.000000in}}%
\pgfpathcurveto{\pgfqpoint{0.078472in}{0.020811in}}{\pgfqpoint{0.070204in}{0.040773in}}{\pgfqpoint{0.055488in}{0.055488in}}%
\pgfpathcurveto{\pgfqpoint{0.040773in}{0.070204in}}{\pgfqpoint{0.020811in}{0.078472in}}{\pgfqpoint{0.000000in}{0.078472in}}%
\pgfpathcurveto{\pgfqpoint{-0.020811in}{0.078472in}}{\pgfqpoint{-0.040773in}{0.070204in}}{\pgfqpoint{-0.055488in}{0.055488in}}%
\pgfpathcurveto{\pgfqpoint{-0.070204in}{0.040773in}}{\pgfqpoint{-0.078472in}{0.020811in}}{\pgfqpoint{-0.078472in}{0.000000in}}%
\pgfpathcurveto{\pgfqpoint{-0.078472in}{-0.020811in}}{\pgfqpoint{-0.070204in}{-0.040773in}}{\pgfqpoint{-0.055488in}{-0.055488in}}%
\pgfpathcurveto{\pgfqpoint{-0.040773in}{-0.070204in}}{\pgfqpoint{-0.020811in}{-0.078472in}}{\pgfqpoint{0.000000in}{-0.078472in}}%
\pgfpathclose%
\pgfusepath{stroke,fill}%
}%
\begin{pgfscope}%
\pgfsys@transformshift{3.653761in}{2.416147in}%
\pgfsys@useobject{currentmarker}{}%
\end{pgfscope}%
\end{pgfscope}%
\begin{pgfscope}%
\pgfpathrectangle{\pgfqpoint{0.100000in}{0.100000in}}{\pgfqpoint{5.307240in}{3.397500in}}%
\pgfusepath{clip}%
\pgfsetrectcap%
\pgfsetroundjoin%
\pgfsetlinewidth{1.505625pt}%
\definecolor{currentstroke}{rgb}{0.678431,1.000000,0.184314}%
\pgfsetstrokecolor{currentstroke}%
\pgfsetstrokeopacity{0.500000}%
\pgfsetdash{}{0pt}%
\pgfpathmoveto{\pgfqpoint{3.593074in}{2.526880in}}%
\pgfusepath{stroke}%
\end{pgfscope}%
\begin{pgfscope}%
\pgfpathrectangle{\pgfqpoint{0.100000in}{0.100000in}}{\pgfqpoint{5.307240in}{3.397500in}}%
\pgfusepath{clip}%
\pgfsetbuttcap%
\pgfsetroundjoin%
\definecolor{currentfill}{rgb}{0.678431,1.000000,0.184314}%
\pgfsetfillcolor{currentfill}%
\pgfsetfillopacity{0.500000}%
\pgfsetlinewidth{0.250937pt}%
\definecolor{currentstroke}{rgb}{0.000000,0.000000,0.000000}%
\pgfsetstrokecolor{currentstroke}%
\pgfsetstrokeopacity{0.500000}%
\pgfsetdash{}{0pt}%
\pgfsys@defobject{currentmarker}{\pgfqpoint{-0.079167in}{-0.079167in}}{\pgfqpoint{0.079167in}{0.079167in}}{%
\pgfpathmoveto{\pgfqpoint{0.000000in}{-0.079167in}}%
\pgfpathcurveto{\pgfqpoint{0.020995in}{-0.079167in}}{\pgfqpoint{0.041133in}{-0.070825in}}{\pgfqpoint{0.055979in}{-0.055979in}}%
\pgfpathcurveto{\pgfqpoint{0.070825in}{-0.041133in}}{\pgfqpoint{0.079167in}{-0.020995in}}{\pgfqpoint{0.079167in}{0.000000in}}%
\pgfpathcurveto{\pgfqpoint{0.079167in}{0.020995in}}{\pgfqpoint{0.070825in}{0.041133in}}{\pgfqpoint{0.055979in}{0.055979in}}%
\pgfpathcurveto{\pgfqpoint{0.041133in}{0.070825in}}{\pgfqpoint{0.020995in}{0.079167in}}{\pgfqpoint{0.000000in}{0.079167in}}%
\pgfpathcurveto{\pgfqpoint{-0.020995in}{0.079167in}}{\pgfqpoint{-0.041133in}{0.070825in}}{\pgfqpoint{-0.055979in}{0.055979in}}%
\pgfpathcurveto{\pgfqpoint{-0.070825in}{0.041133in}}{\pgfqpoint{-0.079167in}{0.020995in}}{\pgfqpoint{-0.079167in}{0.000000in}}%
\pgfpathcurveto{\pgfqpoint{-0.079167in}{-0.020995in}}{\pgfqpoint{-0.070825in}{-0.041133in}}{\pgfqpoint{-0.055979in}{-0.055979in}}%
\pgfpathcurveto{\pgfqpoint{-0.041133in}{-0.070825in}}{\pgfqpoint{-0.020995in}{-0.079167in}}{\pgfqpoint{0.000000in}{-0.079167in}}%
\pgfpathclose%
\pgfusepath{stroke,fill}%
}%
\begin{pgfscope}%
\pgfsys@transformshift{3.593074in}{2.526880in}%
\pgfsys@useobject{currentmarker}{}%
\end{pgfscope}%
\end{pgfscope}%
\begin{pgfscope}%
\pgfpathrectangle{\pgfqpoint{0.100000in}{0.100000in}}{\pgfqpoint{5.307240in}{3.397500in}}%
\pgfusepath{clip}%
\pgfsetrectcap%
\pgfsetroundjoin%
\pgfsetlinewidth{1.505625pt}%
\definecolor{currentstroke}{rgb}{0.678431,1.000000,0.184314}%
\pgfsetstrokecolor{currentstroke}%
\pgfsetstrokeopacity{0.500000}%
\pgfsetdash{}{0pt}%
\pgfpathmoveto{\pgfqpoint{3.668502in}{2.381203in}}%
\pgfusepath{stroke}%
\end{pgfscope}%
\begin{pgfscope}%
\pgfpathrectangle{\pgfqpoint{0.100000in}{0.100000in}}{\pgfqpoint{5.307240in}{3.397500in}}%
\pgfusepath{clip}%
\pgfsetbuttcap%
\pgfsetroundjoin%
\definecolor{currentfill}{rgb}{0.678431,1.000000,0.184314}%
\pgfsetfillcolor{currentfill}%
\pgfsetfillopacity{0.500000}%
\pgfsetlinewidth{0.250937pt}%
\definecolor{currentstroke}{rgb}{0.000000,0.000000,0.000000}%
\pgfsetstrokecolor{currentstroke}%
\pgfsetstrokeopacity{0.500000}%
\pgfsetdash{}{0pt}%
\pgfsys@defobject{currentmarker}{\pgfqpoint{-0.081250in}{-0.081250in}}{\pgfqpoint{0.081250in}{0.081250in}}{%
\pgfpathmoveto{\pgfqpoint{0.000000in}{-0.081250in}}%
\pgfpathcurveto{\pgfqpoint{0.021548in}{-0.081250in}}{\pgfqpoint{0.042216in}{-0.072689in}}{\pgfqpoint{0.057452in}{-0.057452in}}%
\pgfpathcurveto{\pgfqpoint{0.072689in}{-0.042216in}}{\pgfqpoint{0.081250in}{-0.021548in}}{\pgfqpoint{0.081250in}{0.000000in}}%
\pgfpathcurveto{\pgfqpoint{0.081250in}{0.021548in}}{\pgfqpoint{0.072689in}{0.042216in}}{\pgfqpoint{0.057452in}{0.057452in}}%
\pgfpathcurveto{\pgfqpoint{0.042216in}{0.072689in}}{\pgfqpoint{0.021548in}{0.081250in}}{\pgfqpoint{0.000000in}{0.081250in}}%
\pgfpathcurveto{\pgfqpoint{-0.021548in}{0.081250in}}{\pgfqpoint{-0.042216in}{0.072689in}}{\pgfqpoint{-0.057452in}{0.057452in}}%
\pgfpathcurveto{\pgfqpoint{-0.072689in}{0.042216in}}{\pgfqpoint{-0.081250in}{0.021548in}}{\pgfqpoint{-0.081250in}{0.000000in}}%
\pgfpathcurveto{\pgfqpoint{-0.081250in}{-0.021548in}}{\pgfqpoint{-0.072689in}{-0.042216in}}{\pgfqpoint{-0.057452in}{-0.057452in}}%
\pgfpathcurveto{\pgfqpoint{-0.042216in}{-0.072689in}}{\pgfqpoint{-0.021548in}{-0.081250in}}{\pgfqpoint{0.000000in}{-0.081250in}}%
\pgfpathclose%
\pgfusepath{stroke,fill}%
}%
\begin{pgfscope}%
\pgfsys@transformshift{3.668502in}{2.381203in}%
\pgfsys@useobject{currentmarker}{}%
\end{pgfscope}%
\end{pgfscope}%
\begin{pgfscope}%
\pgfpathrectangle{\pgfqpoint{0.100000in}{0.100000in}}{\pgfqpoint{5.307240in}{3.397500in}}%
\pgfusepath{clip}%
\pgfsetrectcap%
\pgfsetroundjoin%
\pgfsetlinewidth{1.505625pt}%
\definecolor{currentstroke}{rgb}{0.678431,1.000000,0.184314}%
\pgfsetstrokecolor{currentstroke}%
\pgfsetstrokeopacity{0.500000}%
\pgfsetdash{}{0pt}%
\pgfpathmoveto{\pgfqpoint{3.664736in}{2.500735in}}%
\pgfusepath{stroke}%
\end{pgfscope}%
\begin{pgfscope}%
\pgfpathrectangle{\pgfqpoint{0.100000in}{0.100000in}}{\pgfqpoint{5.307240in}{3.397500in}}%
\pgfusepath{clip}%
\pgfsetbuttcap%
\pgfsetroundjoin%
\definecolor{currentfill}{rgb}{0.678431,1.000000,0.184314}%
\pgfsetfillcolor{currentfill}%
\pgfsetfillopacity{0.500000}%
\pgfsetlinewidth{0.250937pt}%
\definecolor{currentstroke}{rgb}{0.000000,0.000000,0.000000}%
\pgfsetstrokecolor{currentstroke}%
\pgfsetstrokeopacity{0.500000}%
\pgfsetdash{}{0pt}%
\pgfsys@defobject{currentmarker}{\pgfqpoint{-0.086806in}{-0.086806in}}{\pgfqpoint{0.086806in}{0.086806in}}{%
\pgfpathmoveto{\pgfqpoint{0.000000in}{-0.086806in}}%
\pgfpathcurveto{\pgfqpoint{0.023021in}{-0.086806in}}{\pgfqpoint{0.045102in}{-0.077659in}}{\pgfqpoint{0.061381in}{-0.061381in}}%
\pgfpathcurveto{\pgfqpoint{0.077659in}{-0.045102in}}{\pgfqpoint{0.086806in}{-0.023021in}}{\pgfqpoint{0.086806in}{0.000000in}}%
\pgfpathcurveto{\pgfqpoint{0.086806in}{0.023021in}}{\pgfqpoint{0.077659in}{0.045102in}}{\pgfqpoint{0.061381in}{0.061381in}}%
\pgfpathcurveto{\pgfqpoint{0.045102in}{0.077659in}}{\pgfqpoint{0.023021in}{0.086806in}}{\pgfqpoint{0.000000in}{0.086806in}}%
\pgfpathcurveto{\pgfqpoint{-0.023021in}{0.086806in}}{\pgfqpoint{-0.045102in}{0.077659in}}{\pgfqpoint{-0.061381in}{0.061381in}}%
\pgfpathcurveto{\pgfqpoint{-0.077659in}{0.045102in}}{\pgfqpoint{-0.086806in}{0.023021in}}{\pgfqpoint{-0.086806in}{0.000000in}}%
\pgfpathcurveto{\pgfqpoint{-0.086806in}{-0.023021in}}{\pgfqpoint{-0.077659in}{-0.045102in}}{\pgfqpoint{-0.061381in}{-0.061381in}}%
\pgfpathcurveto{\pgfqpoint{-0.045102in}{-0.077659in}}{\pgfqpoint{-0.023021in}{-0.086806in}}{\pgfqpoint{0.000000in}{-0.086806in}}%
\pgfpathclose%
\pgfusepath{stroke,fill}%
}%
\begin{pgfscope}%
\pgfsys@transformshift{3.664736in}{2.500735in}%
\pgfsys@useobject{currentmarker}{}%
\end{pgfscope}%
\end{pgfscope}%
\begin{pgfscope}%
\pgfpathrectangle{\pgfqpoint{0.100000in}{0.100000in}}{\pgfqpoint{5.307240in}{3.397500in}}%
\pgfusepath{clip}%
\pgfsetrectcap%
\pgfsetroundjoin%
\pgfsetlinewidth{1.505625pt}%
\definecolor{currentstroke}{rgb}{0.678431,1.000000,0.184314}%
\pgfsetstrokecolor{currentstroke}%
\pgfsetstrokeopacity{0.500000}%
\pgfsetdash{}{0pt}%
\pgfpathmoveto{\pgfqpoint{3.495414in}{2.630395in}}%
\pgfusepath{stroke}%
\end{pgfscope}%
\begin{pgfscope}%
\pgfpathrectangle{\pgfqpoint{0.100000in}{0.100000in}}{\pgfqpoint{5.307240in}{3.397500in}}%
\pgfusepath{clip}%
\pgfsetbuttcap%
\pgfsetroundjoin%
\definecolor{currentfill}{rgb}{0.678431,1.000000,0.184314}%
\pgfsetfillcolor{currentfill}%
\pgfsetfillopacity{0.500000}%
\pgfsetlinewidth{0.250937pt}%
\definecolor{currentstroke}{rgb}{0.000000,0.000000,0.000000}%
\pgfsetstrokecolor{currentstroke}%
\pgfsetstrokeopacity{0.500000}%
\pgfsetdash{}{0pt}%
\pgfsys@defobject{currentmarker}{\pgfqpoint{-0.065972in}{-0.065972in}}{\pgfqpoint{0.065972in}{0.065972in}}{%
\pgfpathmoveto{\pgfqpoint{0.000000in}{-0.065972in}}%
\pgfpathcurveto{\pgfqpoint{0.017496in}{-0.065972in}}{\pgfqpoint{0.034278in}{-0.059021in}}{\pgfqpoint{0.046649in}{-0.046649in}}%
\pgfpathcurveto{\pgfqpoint{0.059021in}{-0.034278in}}{\pgfqpoint{0.065972in}{-0.017496in}}{\pgfqpoint{0.065972in}{0.000000in}}%
\pgfpathcurveto{\pgfqpoint{0.065972in}{0.017496in}}{\pgfqpoint{0.059021in}{0.034278in}}{\pgfqpoint{0.046649in}{0.046649in}}%
\pgfpathcurveto{\pgfqpoint{0.034278in}{0.059021in}}{\pgfqpoint{0.017496in}{0.065972in}}{\pgfqpoint{0.000000in}{0.065972in}}%
\pgfpathcurveto{\pgfqpoint{-0.017496in}{0.065972in}}{\pgfqpoint{-0.034278in}{0.059021in}}{\pgfqpoint{-0.046649in}{0.046649in}}%
\pgfpathcurveto{\pgfqpoint{-0.059021in}{0.034278in}}{\pgfqpoint{-0.065972in}{0.017496in}}{\pgfqpoint{-0.065972in}{0.000000in}}%
\pgfpathcurveto{\pgfqpoint{-0.065972in}{-0.017496in}}{\pgfqpoint{-0.059021in}{-0.034278in}}{\pgfqpoint{-0.046649in}{-0.046649in}}%
\pgfpathcurveto{\pgfqpoint{-0.034278in}{-0.059021in}}{\pgfqpoint{-0.017496in}{-0.065972in}}{\pgfqpoint{0.000000in}{-0.065972in}}%
\pgfpathclose%
\pgfusepath{stroke,fill}%
}%
\begin{pgfscope}%
\pgfsys@transformshift{3.495414in}{2.630395in}%
\pgfsys@useobject{currentmarker}{}%
\end{pgfscope}%
\end{pgfscope}%
\begin{pgfscope}%
\pgfpathrectangle{\pgfqpoint{0.100000in}{0.100000in}}{\pgfqpoint{5.307240in}{3.397500in}}%
\pgfusepath{clip}%
\pgfsetrectcap%
\pgfsetroundjoin%
\pgfsetlinewidth{1.505625pt}%
\definecolor{currentstroke}{rgb}{0.678431,1.000000,0.184314}%
\pgfsetstrokecolor{currentstroke}%
\pgfsetstrokeopacity{0.500000}%
\pgfsetdash{}{0pt}%
\pgfpathmoveto{\pgfqpoint{2.087960in}{2.433121in}}%
\pgfusepath{stroke}%
\end{pgfscope}%
\begin{pgfscope}%
\pgfpathrectangle{\pgfqpoint{0.100000in}{0.100000in}}{\pgfqpoint{5.307240in}{3.397500in}}%
\pgfusepath{clip}%
\pgfsetbuttcap%
\pgfsetroundjoin%
\definecolor{currentfill}{rgb}{0.678431,1.000000,0.184314}%
\pgfsetfillcolor{currentfill}%
\pgfsetfillopacity{0.500000}%
\pgfsetlinewidth{0.250937pt}%
\definecolor{currentstroke}{rgb}{0.000000,0.000000,0.000000}%
\pgfsetstrokecolor{currentstroke}%
\pgfsetstrokeopacity{0.500000}%
\pgfsetdash{}{0pt}%
\pgfsys@defobject{currentmarker}{\pgfqpoint{-0.063194in}{-0.063194in}}{\pgfqpoint{0.063194in}{0.063194in}}{%
\pgfpathmoveto{\pgfqpoint{0.000000in}{-0.063194in}}%
\pgfpathcurveto{\pgfqpoint{0.016759in}{-0.063194in}}{\pgfqpoint{0.032835in}{-0.056536in}}{\pgfqpoint{0.044685in}{-0.044685in}}%
\pgfpathcurveto{\pgfqpoint{0.056536in}{-0.032835in}}{\pgfqpoint{0.063194in}{-0.016759in}}{\pgfqpoint{0.063194in}{0.000000in}}%
\pgfpathcurveto{\pgfqpoint{0.063194in}{0.016759in}}{\pgfqpoint{0.056536in}{0.032835in}}{\pgfqpoint{0.044685in}{0.044685in}}%
\pgfpathcurveto{\pgfqpoint{0.032835in}{0.056536in}}{\pgfqpoint{0.016759in}{0.063194in}}{\pgfqpoint{0.000000in}{0.063194in}}%
\pgfpathcurveto{\pgfqpoint{-0.016759in}{0.063194in}}{\pgfqpoint{-0.032835in}{0.056536in}}{\pgfqpoint{-0.044685in}{0.044685in}}%
\pgfpathcurveto{\pgfqpoint{-0.056536in}{0.032835in}}{\pgfqpoint{-0.063194in}{0.016759in}}{\pgfqpoint{-0.063194in}{0.000000in}}%
\pgfpathcurveto{\pgfqpoint{-0.063194in}{-0.016759in}}{\pgfqpoint{-0.056536in}{-0.032835in}}{\pgfqpoint{-0.044685in}{-0.044685in}}%
\pgfpathcurveto{\pgfqpoint{-0.032835in}{-0.056536in}}{\pgfqpoint{-0.016759in}{-0.063194in}}{\pgfqpoint{0.000000in}{-0.063194in}}%
\pgfpathclose%
\pgfusepath{stroke,fill}%
}%
\begin{pgfscope}%
\pgfsys@transformshift{2.087960in}{2.433121in}%
\pgfsys@useobject{currentmarker}{}%
\end{pgfscope}%
\end{pgfscope}%
\begin{pgfscope}%
\pgfpathrectangle{\pgfqpoint{0.100000in}{0.100000in}}{\pgfqpoint{5.307240in}{3.397500in}}%
\pgfusepath{clip}%
\pgfsetrectcap%
\pgfsetroundjoin%
\pgfsetlinewidth{1.505625pt}%
\definecolor{currentstroke}{rgb}{0.678431,1.000000,0.184314}%
\pgfsetstrokecolor{currentstroke}%
\pgfsetstrokeopacity{0.500000}%
\pgfsetdash{}{0pt}%
\pgfpathmoveto{\pgfqpoint{2.192898in}{2.218510in}}%
\pgfusepath{stroke}%
\end{pgfscope}%
\begin{pgfscope}%
\pgfpathrectangle{\pgfqpoint{0.100000in}{0.100000in}}{\pgfqpoint{5.307240in}{3.397500in}}%
\pgfusepath{clip}%
\pgfsetbuttcap%
\pgfsetroundjoin%
\definecolor{currentfill}{rgb}{0.678431,1.000000,0.184314}%
\pgfsetfillcolor{currentfill}%
\pgfsetfillopacity{0.500000}%
\pgfsetlinewidth{0.250937pt}%
\definecolor{currentstroke}{rgb}{0.000000,0.000000,0.000000}%
\pgfsetstrokecolor{currentstroke}%
\pgfsetstrokeopacity{0.500000}%
\pgfsetdash{}{0pt}%
\pgfsys@defobject{currentmarker}{\pgfqpoint{-0.039583in}{-0.039583in}}{\pgfqpoint{0.039583in}{0.039583in}}{%
\pgfpathmoveto{\pgfqpoint{0.000000in}{-0.039583in}}%
\pgfpathcurveto{\pgfqpoint{0.010498in}{-0.039583in}}{\pgfqpoint{0.020567in}{-0.035413in}}{\pgfqpoint{0.027990in}{-0.027990in}}%
\pgfpathcurveto{\pgfqpoint{0.035413in}{-0.020567in}}{\pgfqpoint{0.039583in}{-0.010498in}}{\pgfqpoint{0.039583in}{0.000000in}}%
\pgfpathcurveto{\pgfqpoint{0.039583in}{0.010498in}}{\pgfqpoint{0.035413in}{0.020567in}}{\pgfqpoint{0.027990in}{0.027990in}}%
\pgfpathcurveto{\pgfqpoint{0.020567in}{0.035413in}}{\pgfqpoint{0.010498in}{0.039583in}}{\pgfqpoint{0.000000in}{0.039583in}}%
\pgfpathcurveto{\pgfqpoint{-0.010498in}{0.039583in}}{\pgfqpoint{-0.020567in}{0.035413in}}{\pgfqpoint{-0.027990in}{0.027990in}}%
\pgfpathcurveto{\pgfqpoint{-0.035413in}{0.020567in}}{\pgfqpoint{-0.039583in}{0.010498in}}{\pgfqpoint{-0.039583in}{0.000000in}}%
\pgfpathcurveto{\pgfqpoint{-0.039583in}{-0.010498in}}{\pgfqpoint{-0.035413in}{-0.020567in}}{\pgfqpoint{-0.027990in}{-0.027990in}}%
\pgfpathcurveto{\pgfqpoint{-0.020567in}{-0.035413in}}{\pgfqpoint{-0.010498in}{-0.039583in}}{\pgfqpoint{0.000000in}{-0.039583in}}%
\pgfpathclose%
\pgfusepath{stroke,fill}%
}%
\begin{pgfscope}%
\pgfsys@transformshift{2.192898in}{2.218510in}%
\pgfsys@useobject{currentmarker}{}%
\end{pgfscope}%
\end{pgfscope}%
\begin{pgfscope}%
\pgfpathrectangle{\pgfqpoint{0.100000in}{0.100000in}}{\pgfqpoint{5.307240in}{3.397500in}}%
\pgfusepath{clip}%
\pgfsetrectcap%
\pgfsetroundjoin%
\pgfsetlinewidth{1.505625pt}%
\definecolor{currentstroke}{rgb}{0.678431,1.000000,0.184314}%
\pgfsetstrokecolor{currentstroke}%
\pgfsetstrokeopacity{0.500000}%
\pgfsetdash{}{0pt}%
\pgfpathmoveto{\pgfqpoint{3.723591in}{3.052795in}}%
\pgfusepath{stroke}%
\end{pgfscope}%
\begin{pgfscope}%
\pgfpathrectangle{\pgfqpoint{0.100000in}{0.100000in}}{\pgfqpoint{5.307240in}{3.397500in}}%
\pgfusepath{clip}%
\pgfsetbuttcap%
\pgfsetroundjoin%
\definecolor{currentfill}{rgb}{0.678431,1.000000,0.184314}%
\pgfsetfillcolor{currentfill}%
\pgfsetfillopacity{0.500000}%
\pgfsetlinewidth{0.250937pt}%
\definecolor{currentstroke}{rgb}{0.000000,0.000000,0.000000}%
\pgfsetstrokecolor{currentstroke}%
\pgfsetstrokeopacity{0.500000}%
\pgfsetdash{}{0pt}%
\pgfsys@defobject{currentmarker}{\pgfqpoint{-0.034722in}{-0.034722in}}{\pgfqpoint{0.034722in}{0.034722in}}{%
\pgfpathmoveto{\pgfqpoint{0.000000in}{-0.034722in}}%
\pgfpathcurveto{\pgfqpoint{0.009208in}{-0.034722in}}{\pgfqpoint{0.018041in}{-0.031064in}}{\pgfqpoint{0.024552in}{-0.024552in}}%
\pgfpathcurveto{\pgfqpoint{0.031064in}{-0.018041in}}{\pgfqpoint{0.034722in}{-0.009208in}}{\pgfqpoint{0.034722in}{0.000000in}}%
\pgfpathcurveto{\pgfqpoint{0.034722in}{0.009208in}}{\pgfqpoint{0.031064in}{0.018041in}}{\pgfqpoint{0.024552in}{0.024552in}}%
\pgfpathcurveto{\pgfqpoint{0.018041in}{0.031064in}}{\pgfqpoint{0.009208in}{0.034722in}}{\pgfqpoint{0.000000in}{0.034722in}}%
\pgfpathcurveto{\pgfqpoint{-0.009208in}{0.034722in}}{\pgfqpoint{-0.018041in}{0.031064in}}{\pgfqpoint{-0.024552in}{0.024552in}}%
\pgfpathcurveto{\pgfqpoint{-0.031064in}{0.018041in}}{\pgfqpoint{-0.034722in}{0.009208in}}{\pgfqpoint{-0.034722in}{0.000000in}}%
\pgfpathcurveto{\pgfqpoint{-0.034722in}{-0.009208in}}{\pgfqpoint{-0.031064in}{-0.018041in}}{\pgfqpoint{-0.024552in}{-0.024552in}}%
\pgfpathcurveto{\pgfqpoint{-0.018041in}{-0.031064in}}{\pgfqpoint{-0.009208in}{-0.034722in}}{\pgfqpoint{0.000000in}{-0.034722in}}%
\pgfpathclose%
\pgfusepath{stroke,fill}%
}%
\begin{pgfscope}%
\pgfsys@transformshift{3.723591in}{3.052795in}%
\pgfsys@useobject{currentmarker}{}%
\end{pgfscope}%
\end{pgfscope}%
\begin{pgfscope}%
\pgfpathrectangle{\pgfqpoint{0.100000in}{0.100000in}}{\pgfqpoint{5.307240in}{3.397500in}}%
\pgfusepath{clip}%
\pgfsetrectcap%
\pgfsetroundjoin%
\pgfsetlinewidth{1.505625pt}%
\definecolor{currentstroke}{rgb}{0.678431,1.000000,0.184314}%
\pgfsetstrokecolor{currentstroke}%
\pgfsetstrokeopacity{0.500000}%
\pgfsetdash{}{0pt}%
\pgfpathmoveto{\pgfqpoint{3.736053in}{3.243999in}}%
\pgfusepath{stroke}%
\end{pgfscope}%
\begin{pgfscope}%
\pgfpathrectangle{\pgfqpoint{0.100000in}{0.100000in}}{\pgfqpoint{5.307240in}{3.397500in}}%
\pgfusepath{clip}%
\pgfsetbuttcap%
\pgfsetroundjoin%
\definecolor{currentfill}{rgb}{0.678431,1.000000,0.184314}%
\pgfsetfillcolor{currentfill}%
\pgfsetfillopacity{0.500000}%
\pgfsetlinewidth{0.250937pt}%
\definecolor{currentstroke}{rgb}{0.000000,0.000000,0.000000}%
\pgfsetstrokecolor{currentstroke}%
\pgfsetstrokeopacity{0.500000}%
\pgfsetdash{}{0pt}%
\pgfsys@defobject{currentmarker}{\pgfqpoint{-0.069444in}{-0.069444in}}{\pgfqpoint{0.069444in}{0.069444in}}{%
\pgfpathmoveto{\pgfqpoint{0.000000in}{-0.069444in}}%
\pgfpathcurveto{\pgfqpoint{0.018417in}{-0.069444in}}{\pgfqpoint{0.036082in}{-0.062127in}}{\pgfqpoint{0.049105in}{-0.049105in}}%
\pgfpathcurveto{\pgfqpoint{0.062127in}{-0.036082in}}{\pgfqpoint{0.069444in}{-0.018417in}}{\pgfqpoint{0.069444in}{0.000000in}}%
\pgfpathcurveto{\pgfqpoint{0.069444in}{0.018417in}}{\pgfqpoint{0.062127in}{0.036082in}}{\pgfqpoint{0.049105in}{0.049105in}}%
\pgfpathcurveto{\pgfqpoint{0.036082in}{0.062127in}}{\pgfqpoint{0.018417in}{0.069444in}}{\pgfqpoint{0.000000in}{0.069444in}}%
\pgfpathcurveto{\pgfqpoint{-0.018417in}{0.069444in}}{\pgfqpoint{-0.036082in}{0.062127in}}{\pgfqpoint{-0.049105in}{0.049105in}}%
\pgfpathcurveto{\pgfqpoint{-0.062127in}{0.036082in}}{\pgfqpoint{-0.069444in}{0.018417in}}{\pgfqpoint{-0.069444in}{0.000000in}}%
\pgfpathcurveto{\pgfqpoint{-0.069444in}{-0.018417in}}{\pgfqpoint{-0.062127in}{-0.036082in}}{\pgfqpoint{-0.049105in}{-0.049105in}}%
\pgfpathcurveto{\pgfqpoint{-0.036082in}{-0.062127in}}{\pgfqpoint{-0.018417in}{-0.069444in}}{\pgfqpoint{0.000000in}{-0.069444in}}%
\pgfpathclose%
\pgfusepath{stroke,fill}%
}%
\begin{pgfscope}%
\pgfsys@transformshift{3.736053in}{3.243999in}%
\pgfsys@useobject{currentmarker}{}%
\end{pgfscope}%
\end{pgfscope}%
\begin{pgfscope}%
\pgfpathrectangle{\pgfqpoint{0.100000in}{0.100000in}}{\pgfqpoint{5.307240in}{3.397500in}}%
\pgfusepath{clip}%
\pgfsetrectcap%
\pgfsetroundjoin%
\pgfsetlinewidth{1.505625pt}%
\definecolor{currentstroke}{rgb}{0.678431,0.847059,0.901961}%
\pgfsetstrokecolor{currentstroke}%
\pgfsetstrokeopacity{0.500000}%
\pgfsetdash{}{0pt}%
\pgfpathmoveto{\pgfqpoint{4.513035in}{3.245107in}}%
\pgfusepath{stroke}%
\end{pgfscope}%
\begin{pgfscope}%
\pgfpathrectangle{\pgfqpoint{0.100000in}{0.100000in}}{\pgfqpoint{5.307240in}{3.397500in}}%
\pgfusepath{clip}%
\pgfsetbuttcap%
\pgfsetroundjoin%
\definecolor{currentfill}{rgb}{0.678431,0.847059,0.901961}%
\pgfsetfillcolor{currentfill}%
\pgfsetfillopacity{0.500000}%
\pgfsetlinewidth{0.250937pt}%
\definecolor{currentstroke}{rgb}{0.000000,0.000000,0.000000}%
\pgfsetstrokecolor{currentstroke}%
\pgfsetstrokeopacity{0.500000}%
\pgfsetdash{}{0pt}%
\pgfsys@defobject{currentmarker}{\pgfqpoint{-0.034722in}{-0.034722in}}{\pgfqpoint{0.034722in}{0.034722in}}{%
\pgfpathmoveto{\pgfqpoint{0.000000in}{-0.034722in}}%
\pgfpathcurveto{\pgfqpoint{0.009208in}{-0.034722in}}{\pgfqpoint{0.018041in}{-0.031064in}}{\pgfqpoint{0.024552in}{-0.024552in}}%
\pgfpathcurveto{\pgfqpoint{0.031064in}{-0.018041in}}{\pgfqpoint{0.034722in}{-0.009208in}}{\pgfqpoint{0.034722in}{0.000000in}}%
\pgfpathcurveto{\pgfqpoint{0.034722in}{0.009208in}}{\pgfqpoint{0.031064in}{0.018041in}}{\pgfqpoint{0.024552in}{0.024552in}}%
\pgfpathcurveto{\pgfqpoint{0.018041in}{0.031064in}}{\pgfqpoint{0.009208in}{0.034722in}}{\pgfqpoint{0.000000in}{0.034722in}}%
\pgfpathcurveto{\pgfqpoint{-0.009208in}{0.034722in}}{\pgfqpoint{-0.018041in}{0.031064in}}{\pgfqpoint{-0.024552in}{0.024552in}}%
\pgfpathcurveto{\pgfqpoint{-0.031064in}{0.018041in}}{\pgfqpoint{-0.034722in}{0.009208in}}{\pgfqpoint{-0.034722in}{0.000000in}}%
\pgfpathcurveto{\pgfqpoint{-0.034722in}{-0.009208in}}{\pgfqpoint{-0.031064in}{-0.018041in}}{\pgfqpoint{-0.024552in}{-0.024552in}}%
\pgfpathcurveto{\pgfqpoint{-0.018041in}{-0.031064in}}{\pgfqpoint{-0.009208in}{-0.034722in}}{\pgfqpoint{0.000000in}{-0.034722in}}%
\pgfpathclose%
\pgfusepath{stroke,fill}%
}%
\begin{pgfscope}%
\pgfsys@transformshift{4.513035in}{3.245107in}%
\pgfsys@useobject{currentmarker}{}%
\end{pgfscope}%
\end{pgfscope}%
\begin{pgfscope}%
\definecolor{textcolor}{rgb}{0.000000,0.000000,0.000000}%
\pgfsetstrokecolor{textcolor}%
\pgfsetfillcolor{textcolor}%
\pgftext[x=3.886097in,y=3.010803in,left,base]{\color{textcolor}\setmainfont{Lato}\rmfamily\fontsize{8.000000}{9.600000}\selectfont +5.0pp}%
\end{pgfscope}%
\begin{pgfscope}%
\definecolor{textcolor}{rgb}{0.000000,0.000000,0.000000}%
\pgfsetstrokecolor{textcolor}%
\pgfsetfillcolor{textcolor}%
\pgftext[x=3.894643in,y=3.201958in,left,base]{\color{textcolor}\setmainfont{Lato}\rmfamily\fontsize{8.000000}{9.600000}\selectfont +10.0pp}%
\end{pgfscope}%
\begin{pgfscope}%
\definecolor{textcolor}{rgb}{0.000000,0.000000,0.000000}%
\pgfsetstrokecolor{textcolor}%
\pgfsetfillcolor{textcolor}%
\pgftext[x=4.616870in,y=3.207452in,left,base]{\color{textcolor}\setmainfont{Lato}\rmfamily\fontsize{8.000000}{9.600000}\selectfont -5.0pp}%
\end{pgfscope}%
\begin{pgfscope}%
\definecolor{textcolor}{rgb}{0.000000,0.000000,0.000000}%
\pgfsetstrokecolor{textcolor}%
\pgfsetfillcolor{textcolor}%
\pgftext[x=3.203359in,y=3.235459in,left,base]{\color{textcolor}\setmainfont{Lato}\rmfamily\fontsize{8.000000}{9.600000}\selectfont Legend:}%
\end{pgfscope}%
\end{pgfpicture}%
\makeatother%
\endgroup%

\vspace{-3mm}

\footnotesize{Source: Bureau of Labor Statistics; \ Full Table: \tbllink{msa_unemp_rate.csv} \\ \**Pct Ch is percent change in labor force from \input{text/unemp_map_date.txt}}
\end{minipage}
\end{document}
