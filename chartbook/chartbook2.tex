% % % % % % % % % % % % % % 
%
%	U.S. Chartbook
%	Brian W. Dew (brianwdew@gmail.com)
%	Updated: September 23, 2019
%	GitHub repo contains to do list (issues)
%   https://github.com/bdecon/US-chartbook
%
% % % % % % % % % % % % % %
\PassOptionsToPackage{table}{xcolor}
\documentclass{report}

%
% % % % % % Packages % % % % % % % % % 
%
	
	\usepackage[letterpaper, margin=1.18in]{geometry}
	\usepackage{microtype}
	\usepackage[default]{lato}
	\usepackage{pgfplots, pgfplotstable}
	\usepackage{xcolor}
	\usepackage{array}
	\usepackage{fontawesome}
	\usetikzlibrary{pgfplots.dateplot}

%
% % % % % Document Settings % % % % % % % 
%

	% Paragraph spacing
		\setlength{\parskip}{8pt}
		\setlength{\parindent}{0pt}
		
%
% % % % % Graph Settings % % % % % % % 
%
	
	% Color square
	\newcommand{\cbox}[1]{
		\begin{tikzpicture} \draw [#1, line width=6](0,0) -- (.2,0);  
		\end{tikzpicture}}
	\newcommand{\colorline}[2]{
		\begin{tikzpicture} \draw [#1, line width=1.8](0,0.2) -- +(0.6,0) node[right, black!80] {#2}; 
		\end{tikzpicture}}
	
	% Last two digits of year
	\makeatletter
	\newcommand*\short[1]{\expandafter\@gobbletwo\number\numexpr#1\relax}
	\makeatother	
	
	% Column width and alignment
	\newcolumntype{R}[1]{>{\raggedleft\let\newline\\\arraybackslash\hspace{0pt}}m{#1}}	
	
	% Style for date plots
	\pgfplotsset{compat=newest, 
		scaled y ticks=false,
		axis line style={black!20}, 
		xtick style={black!20}, ytick style={draw=none},
		every tick label/.style={black!50, font=\scriptsize,
			/pgf/number format/assume math mode=true},
		width=13.0cm, height=4.8cm, 
		xticklabel style={align=left}, 
		yticklabel style={text width=0.85em, align=right},       
		axis x line*=bottom, x axis line style={black!50},
	    axis y line=left, y axis line style={opacity=0},
	    ymajorgrids, grid style={very thin, black!10},	        
	    every node near coord/.style={/pgf/number format/fixed,
	    	font=\scriptsize, style={black!70}},
	    legend style={legend columns=-1, draw=none, fill=none,
	    	/tikz/every even column/.append style={column sep=0.3cm}}}
	
	% stacked diverging bar
	\newcommand{\sbar}[4]{
		\addplot[ybar stacked, bar width=2.7pt, draw opacity=0, fill=#1] 
			table [x=#2, y=#3, col sep=comma]{#4};}
					
	% text node
	\newcommand{\stdnode}[3]{\node[below, align=left, shift=({#1,#2})]{#3};}	        
		        
	% Date (X) Axis Tick Marks, one tick per year, every even year labeled
	\newcommand{\dateaxisticks}{
		date coordinates in=x, axis line style={draw=none},
		xmax={2019-10-01},
		max space between ticks=40,	    
		xtick={{1990-01-01}, {1992-01-01}, {1994-01-01}, 
			{1996-01-01}, {1998-01-01}, {2000-01-01}, 
			{2002-01-01}, {2004-01-01}, {2006-01-01},
			{2008-01-01}, {2010-01-01}, {2012-01-01}, {2014-01-01},
		    {2016-01-01}, {2018-01-01}},
		minor xtick={{1989-01-01}, {1991-01-01}, {1993-01-01},
			{1995-01-01}, {1997-01-01}, {1999-01-01}, 
			{2001-01-01}, {2003-01-01}, {2005-01-01}, {2007-01-01},
		    {2009-01-01}, {2011-01-01}, {2013-01-01}, {2015-01-01},
		    {2017-01-01}, {2019-01-01}},
		enlarge y limits={0.06}, enlarge x limits={0.01},
		}
		
	% Settings for y label text in horizontal bar charts
	\newcommand{\barylab}[2]{yticklabel style={text width=#1, align=right, 
		style={black!70}, text height=#2},}
	
	% Solid bars at significant  x or y values
	\newcommand{\bbar}[2]{extra #1 ticks = {{#2}}, extra #1 tick labels = ,
		extra #1 tick style = {grid=major, grid style={thick, black!25}},}
		
	% Standard line
	\newcommand{\stdline}[4]{\addplot[very thick, no markers, color=#1] 
		table [x=#2, y=#3, col sep=comma] {#4};	}
		
	% Thicker line
	\newcommand{\thickline}[4]{\addplot[ultra thick, no markers, color=#1] 
		table [x=#2, y=#3, col sep=comma] {#4};	}
		
	% Style for bar plots legend symbol		
		\pgfplotsset{/pgfplots/area legend/.style={/pgfplots/legend image code/.code={
            \fill[##1] (0cm,-0.1cm) rectangle (0.6cm,0.1cm);}},}			
            
	% Additional bar plot settings
	\newcommand{\barplotnogrid}{xbar=0pt,
	    y axis line style={opacity=0},   
	    x axis line style={opacity=0}, 
	    yticklabel style={align=left, anchor=east},
      		xmajorticks=false, ymajorgrids=false,   
	    ytick=data, tickwidth=0pt, area legend, reverse legend,
	    nodes near coords, nodes near coords align={horizontal},}  
		
	% Recession bars		
	\newcommand{\rbars}{
		\fill[color=black!10] (axis cs:{1990-07-01},\pgfkeysvalueof{/pgfplots/ymin}) rectangle 
			(axis cs:{1991-03-01}, \pgfkeysvalueof{/pgfplots/ymax});
		\fill[color=black!10] (axis cs:{2007-12-01},\pgfkeysvalueof{/pgfplots/ymin}) rectangle 
			(axis cs:{2009-07-01}, \pgfkeysvalueof{/pgfplots/ymax});
		\fill[color=black!10] (axis cs:{2001-03-01},\pgfkeysvalueof{/pgfplots/ymin}) rectangle 
			(axis cs:{2001-11-01}, \pgfkeysvalueof{/pgfplots/ymax});}
			
	\newcommand{\rebars}{
		\fill[color=black!10] (axis cs:{2007-12-01},\pgfkeysvalueof{/pgfplots/ymin}) rectangle 
			(axis cs:{2009-07-01}, \pgfkeysvalueof{/pgfplots/ymax});
		\fill[color=black!10] (axis cs:{2001-03-01},\pgfkeysvalueof{/pgfplots/ymin}) rectangle 
			(axis cs:{2001-11-01}, \pgfkeysvalueof{/pgfplots/ymax});}
	
	\newfontfamily\seriffont{SourceSerifPro}	
	
	
	\pgfplotstableread[header=true, col sep=comma]{data/cpi_comp.csv}\cpi
			    		    
% % % % % % % %
%
%  Begin Document
%
% % % % % % % %		
\begin{document}

\begin{minipage}{0.76\textwidth}

\section*{\color{darkgray}\LARGE \seriffont Prices}

\small \input{text/cpi_main.txt} \\

\vspace{2mm}

\noindent \normalsize \textbf{Consumer Price Index}\\
\footnotesize{\textit{annual growth, percent}}\\ 
\noindent \hspace*{-2mm} \begin{tikzpicture}
	\begin{axis}[\bbar{y}{0}, \dateaxisticks ytick={-2, 0, 2, 4, 6}, 
		xticklabel={`\short{\year}}, clip=false, 
		legend style={at={(0.98, 1.13)}},
		height=4.4cm, enlarge y limits={0.11}]
	\rbars
	\stdline{blue!60!cyan}{date}{value}{data/cpi.csv}
	\stdline{gray}{date}{value2}{data/cpi.csv}
	%\stdnode{2.4cm}{0.35cm}{\scriptsize \input{text/cpi.txt}}
	\legend{All-items, Core};
	\end{axis}
\end{tikzpicture}\\
\footnotesize{Source: Bureau of Labor Statistics} \\

\end{minipage}

\vspace{2mm}

\begin{minipage}{0.3\textwidth}
\small Text here about components. This section will need to be quite long to take up the right amount of space. Perhaps mention all items or give a breakdown of relative importance and recent growth data. Also try to capture the importance on some items, such as housing and medical care. Most of the inflation is coming from housing. More text here.\\

More text here. Possibly mention process for identifying contributions to the all items total. Another option is to discuss why housing inflation isn't eased through higher interest rates. Probably could mention volatility of food and energy or overall story related to inflation. 
\end{minipage} \hspace{6mm}
\begin{minipage}{0.36\textwidth}
\noindent \normalsize \textbf{Consumer Price Index}\\
\footnotesize{\textit{contribution to annual growth, percentage points}}\\ 
\noindent \hspace*{-7mm} \begin{tikzpicture}
  	\begin{axis}[\barplotnogrid axis y line=left, \barylab{3.3cm}{1.5ex}
    	width=5.0cm, bar width=1.6ex, height=7.8cm,
    	enlarge y limits={abs=0.35cm}, enlarge x limits=0.3, \bbar{x}{0},
		yticklabels from table={\cpi}{name}, 
		yticklabel style={font=\footnotesize, xshift=-4pt},
		nodes near coords style={/pgf/number format/.cd, fixed zerofill,
			precision=2, assume math mode},
		legend style={text=black!70, at={(-0.18,1.08)}, anchor=north, legend columns=-1, 
				fill=none, draw=none,
		        /tikz/every even column/.append style={column sep=0.4cm}}]
  	\addplot[fill=cyan!40, draw=none] 
  		table [y expr=-\coordindex, x index=2] {\cpi};
  	\addplot[fill=blue!70, draw=none] 
  		table [y expr=-\coordindex, x index=1] {\cpi};
 	\legend{\input{text/cpi_mo2.txt}, \input{text/cpi_mo1.txt}}
  	\end{axis}
\end{tikzpicture}\\
\footnotesize{Source: Bureau of Labor Statistics} \\


\end{minipage}

\end{document}